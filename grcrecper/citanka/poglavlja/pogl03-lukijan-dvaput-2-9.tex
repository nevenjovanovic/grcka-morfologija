% redaktura NZ
\section*{O autoru}

Lukijan \textgreek[variant=ancient]{(Λουκιανός,} Lucianus) bio je retoričar i prozni pisac (Samozata na rijeci Eufratu, danas Samsat u Turskoj, oko 115. – Atena, nakon 180.\ po Kr.). Životopis mu se može tek uvjetno rekonstruirati iz rijetkih autobiografskih očitovanja. Materinski mu je jezik najvjerojatnije bio aramejski, no izvrsno je ovladao grčkim i stekao zavidno retoričko i književno obrazovanje; nakon kraće odvjetničke karijere i mnogobrojnih putovanja, na kojima je, u tradiciji druge sofistike, držao javna predavanja, deklamirao i poučavao retoriku, nastanio se u Ateni i posvetio pisanju.

Očuvano je oko osamdesetak njegovih spisa (desetak ih je dvojbene autentičnosti). Žanrovski se obično se dijele na retoričke spise, dijaloge, menipske satire, pamflete i pripovjedna djela. Danas se možda najpopularnije \textit{Istinite pripovijesti} \textgreek[variant=ancient]{(Ἀληϑῆ δıηγήματα),} žanrovski hibrid parodije, menipske satire i znanstvenofantastične pripovijesti.

Skeptičan prema svakoj dogmi, nemilosrdan u izrugivanju ljudske gluposti, Lukijan je bio osobito rado čitan u humanizmu (Poggio Bracciolini, Giovanni Pontano, Ulrich von Hutten, Philipp Melanchthon); Erazmova \textit{Pohvala ludosti} izravan je plod oduševljenja Lukijanom.

\section*{O tekstu}

Filozofski dijalog \textgreek[variant=ancient]{Δὶς κατηγορούμενος} \textit{(Bis accusatus}, \textit{Dvaput optužen}) počinje Zeusovom invektivom upućenom filozofima koji smatraju da je bogovima lako. Zeus objašnjava da nipošto nije tako.

\newpage

\section*{Pročitajte naglas grčki tekst.}

Luc.\ Bis accusatus 2.9

\medskip

{\large
\begin{greek}
\noindent Ἐγὼ δὲ αὐτὸς ὁ πάντων βασιλεὺς καὶ πατὴρ ὅσας μὲν ἀηδίας ἀνέχομαι, ὅσα δὲ πράγματα ἔχω πρὸς τοσαύτας φροντίδας διῃρημένος· ᾧ πρῶτα μὲν τὰ τῶν ἄλλων θεῶν ἔργα ἐπισκοπεῖν ἀναγκαῖον ὁπόσοι τι ἡμῖν συνδιαπράττουσι τῆς ἀρχῆς, ὡς μὴ βλακεύωσιν ἐν αὐτοῖς, ἔπειτα δὲ καὶ αὐτῷ μυρία ἄττα πράττειν καὶ σχεδὸν ἀνέφικτα ὑπὸ λεπτότητος· οὐ γὰρ μόνον τὰ κεφάλαια ταῦτα τῆς  διοικήσεως, ὑετοὺς καὶ χαλάζας καὶ πνεύματα καὶ ἀστραπὰς αὐτὸς οἰκονομησάμενος καὶ διατάξας πέπαυμαι τῶν ἐπὶ μέρους φροντίδων ἀπηλλαγμένος, ἀλλά με δεῖ καὶ ταῦτα μὲν ποιεῖν ἀποβλέπειν δὲ κατὰ τὸν αὐτὸν χρόνον ἁπανταχόσε καὶ πάντα ἐπισκοπεῖν ὥσπερ τὸν ἐν τῇ Νεμέᾳ βουκόλον, τοὺς κλέπτοντας, τοὺς ἐπιορκοῦντας, τοὺς θύοντας, εἴ τις ἔσπεισε, πόθεν ἡ κνῖσα καὶ ὁ καπνὸς ἀνέρχεται, τίς νοσῶν ἢ πλέων ἐκάλεσεν, καὶ τὸ πάντων ἐπιπονώτατον, ὑφ' ἕνα καιρὸν ἔν τε ᾿Ολυμπίᾳ τῇ ἑκατόμβῃ παρεῖναι καὶ ἐν Βαβυλῶνι τοὺς πολεμοῦντας ἐπισκοπεῖν καὶ ἐν Γέταις χαλαζᾶν καὶ ἐν Αἰθίοψιν εὐωχεῖσθαι.

\end{greek}

}

