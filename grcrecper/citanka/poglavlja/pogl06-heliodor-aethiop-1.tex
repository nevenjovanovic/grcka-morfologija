% unesi korekture NČ 2019-09-06
\section*{O autoru}

Heliodor \textgreek[variant=ancient]{(Ἡλιόδωρος),} grčki romanopisac (Emesa, danas Homs u Siriji, III. ili IV. st.), autor je najopsežnijega grčkoga ljubavnoga romana \textit{Etiopske priče o Teagenu i Harikleji} \textgreek[variant=ancient]{(Αἰϑιοπικὰ τὰ περὶ Θεαγένην καὶ Χαρίκλειαν),} u deset knjiga. 

Djelo otkriva temeljito poznavanje žanrovske tradicije (motivima otmice, gusarskih prepada, lažne smrti, napasnih snubitelja), ali i značajnu inovativnost u pripovjednom oblikovanju (izokretanje fabularnoga slijeda događaja pripovijedanjem u \textit{flash-backu}, pripovjedači s ograničenim znanjem, česte promjene tempa). 

Priča o djevojci koju je majka, tamnoputa etiopska kraljica, izložila, zato što se rodila bijela, imala je snažan odjek i u bizantskoj književnosti (Teodor Prodrom, Niketa Eugenijan) i na Zapadu (Cervantes, Calderón, Tasso, Racine, Shakespeare), gdje je značajno utjecala na barokni roman. Vjerojatno posljednji odjek bogatog naslijeđa Verdijeva je opera \textit{Aida}.

\section*{O tekstu}

U ranu zoru, blizu ušća Nila u more, egipatski razbojnici nailaze na neobičan prizor: teško natovaren brod vezan je uz obalu, ali bez mornara; svuda uokolo po obali leže mrtvaci i umirući. Netom je završila neobična bitka, u kojoj je oružje bio pribor za gozbu, no nije jasno gdje su pobjednici, kao što nije jasno ni zašto poraženi i njihov brod nisu opljačkani. Na stijeni pored broda, međutim, razbojnici opažaju nešto još neobičnije. Radi se o prekrasnoj djevojci u punoj ratnoj opremi; u njezinu je krilu ranjeni mladić.

%\newpage

\section*{Pročitajte naglas grčki tekst.}

Heliod.\ Aeth.\ 1.2

%Naslov prema izdanju

\medskip

\begin{greek}
{\large
{ \noindent Κόρη καθῆστο ἐπὶ πέτρας, ἀμήχανόν τι κάλλος καὶ θεὸς εἶναι ἀναπείθουσα, τοῖς μὲν παροῦσι περιαλγοῦσα φρονήματος δὲ εὐγενοῦς ἔτι πνέουσα.  Δάφνῃ τὴν κεφαλὴν ἔστεπτο καὶ φαρέτραν τῶν ὤμων ἐξῆπτο καὶ τῷ λαιῷ βραχίονι τὸ τόξον ὑπεστήρικτο· ἡ λοιπὴ δὲ χεὶρ ἀφροντίστως ἀπῃώρητο. Μηρῷ δὲ τῷ δεξιῷ τὸν ἀγκῶνα θατέρας χειρὸς ἐφεδράζουσα καὶ τοῖς δακτύλοις τὴν παρειὰν ἐπιτρέψασα, κάτω νεύουσα καί τινα προκείμενον ἔφηβον περισκοποῦσα τὴν κεφαλὴν ἀνεῖχεν.  

Ὁ δὲ τραύμασι μὲν κατῄκιστο καὶ μικρὸν ἀναφέρειν ὥσπερ ἐκ βαθέος ὕπνου τοῦ παρ' ὀλίγον θανάτου κατεφαίνετο, ἤνθει δὲ καὶ ἐν τούτοις ἀνδρείῳ τῷ κάλλει καὶ ἡ παρειὰ καταρρέοντι τῷ αἵματι φοινιττομένη λευκότητι πλέον ἀντέλαμπεν. Ὀφθαλμοὺς δὲ ἐκείνου οἱ μὲν πόνοι κατέσπων, ἡ δὲ ὄψις τῆς κόρης ἐφ' ἑαυτὴν ἀνεῖλκε καὶ τοῦτο ὁρᾶν αὐτοὺς ἠνάγκαζεν, ὅτι ἐκείνην ἑώρων.  

Ὡς δὲ πνεῦμα συλλεξάμενος καὶ βύθιόν τι ἀσθμήνας λεπτὸν ὑπεφθέγξατο καὶ ``ὦ γλυκεῖα,'' ἔφη ``σῴζῃ μοι ὡς ἀληθῶς, ἢ γέγονας καὶ αὐτὴ τοῦ πολέμου πάρεργον, οὐκ ἀνέχῃ δὲ ἄλλως οὐδὲ μετὰ θάνατον ἀποστατεῖν ἡμῶν, ἀλλὰ φάσμα τὸ σὸν καὶ ψυχὴ τὰς ἐμὰς περιέπει τύχας;'', ``ἐν σοὶ'' ἔφη ``τὰ ἐμὰ'' ἡ κόρη ``σῴζεσθαί τε καὶ μή· τοῦτο γοῦν ὁρᾷς;'' δείξασα ἐπὶ τῶν γονάτων ξίφος, ``εἰς δεῦρο ἤργησεν ὑπὸ τῆς σῆς ἀναπνοῆς ἐπεχόμενον.''

}
}
\end{greek}

