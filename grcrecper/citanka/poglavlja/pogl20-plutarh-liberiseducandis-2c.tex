% Unesi korekture NČ 2019-09-20
%\section*{O autoru}



\section*{O tekstu}

Ovo je prvi esej u Plutarhovoj zbirci \textit{Moralia}. Na osnovi složene interne i eksterne argumentacije, filolozi osporavaju Plutarhovo autorstvo toga teksta. Osvrćući se na odgojne prilike svojeg doba, autor priznaje različitost talenata, ali inzistira na vrijednosti vježbanja: tjelesnoga, vojničkog, ali i vježbanja filozofije, koje je glavni cilj obrazovanja. Kritiziraju se roditelji koji ne mare za obrazovanje djece i oni koji ne žele platiti primjerene honorare za valjane učitelje. Upozorava se i na neobuzdanost mladića i na opasnost laskanja.

Ovdje doneseni odlomak dio je početne pohvale vježbanja. Tvrdnja da ono može kompenzirati i nedostatke (dok, s druge strane, nedostatak vježbe može upropastiti i najveći talent) potkrepljuje se usporedbama s pojavama iz prirodnog svijeta i procesima iz svakodnevnog života.

%\newpage

\section*{Pročitajte naglas grčki tekst.}
Plut. De liberis educandis 2c
%Naslov prema izdanju

\medskip

{\large
\begin{greek}
\noindent Εἰ δέ τις οἴεται τοὺς οὐκ εὖ πεφυκότας μαθήσεως καὶ μελέτης τυχόντας ὀρθῆς πρὸς ἀρετὴν οὐκ ἂν τὴν τῆς φύσεως ἐλάττωσιν εἰς τοὐνδεχόμενον ἀναδραμεῖν, ἴστω πολλοῦ, μᾶλλον δὲ τοῦ παντὸς διαμαρτάνων. φύσεως μὲν γὰρ ἀρετὴν διαφθείρει ῥᾳθυμία, φαυλότητα δ' ἐπανορθοῖ διδαχή· καὶ τὰ μὲν ῥᾴδια τοὺς ἀμελοῦντας φεύγει, τὰ δὲ χαλεπὰ ταῖς ἐπιμελείαις ἁλίσκεται. καταμάθοις δ' ἂν ὡς ἀνύσιμον πρᾶγμα καὶ τελεσιουργὸν ἐπιμέλεια καὶ πόνος ἐστίν, ἐπὶ πολλὰ τῶν γιγνομένων ἐπιβλέψας. σταγόνες μὲν γὰρ ὕδατος πέτρας κοιλαίνουσι, σίδηρος δὲ καὶ χαλκὸς ταῖς ἐπαφαῖς τῶν χειρῶν ἐκτρίβονται, οἱ δ' ἁρμάτειοι τροχοὶ πόνῳ καμφθέντες οὐδ' ἂν εἴ τι γένοιτο τὴν ἐξ ἀρχῆς δύναιντ' ἀναλαβεῖν εὐθυωρίαν· τάς γε μὴν καμπύλας τῶν ὑποκριτῶν βακτηρίας ἀπευθύνειν ἀμήχανον, ἀλλὰ τὸ παρὰ φύσιν τῷ πόνῳ τοῦ κατὰ φύσιν ἐγένετο κρεῖττον. καὶ μόνα ἆρα ταῦτα τὴν τῆς ἐπιμελείας ἰσχὺν διαδείκνυσιν; οὔκ, ἀλλὰ καὶ μυρί' ἐπὶ μυρίοις. ἀγαθὴ γῆ πέφυκεν· ἀλλ' ἀμεληθεῖσα χερσεύεται, καὶ ὅσῳ τῇ φύσει βελτίων ἐστί, τοσούτῳ μᾶλλον ἐξαργηθεῖσα δι' ἀμέλειαν ἐξαπόλλυται.

\end{greek}
}

