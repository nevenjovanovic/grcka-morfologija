%Unesi korekture NČ 2019-09-09
%\section*{O autoru}


\section*{O tekstu}

U ovom odlomku Tukididove \textit{Povijesti} problematizira se početak naseljavanja Grčke. Trebalo je dugo vremena dok naseobine Grka nisu postale trajne.

%\newpage

\section*{Pročitajte naglas grčki tekst.}

Thuc.\ Historiae 1.2

%Naslov prema izdanju

\medskip

\begin{greek}
{\large
{ \noindent Φαίνεται γὰρ ἡ νῦν Ἑλλὰς καλουμένη οὐ πάλαι βεβαίως οἰκουμένη, ἀλλὰ μεταναστάσεις τε οὖσαι τὰ πρότερα καὶ ῥᾳδίως ἕκαστοι τὴν ἑαυτῶν ἀπολείποντες βιαζόμενοι ὑπό τινων αἰεὶ πλειόνων. τῆς γὰρ ἐμπορίας οὐκ οὔσης, οὐδ’ ἐπιμειγνύντες ἀδεῶς ἀλλήλοις οὔτε κατὰ γῆν οὔτε διὰ θαλάσσης, νεμόμενοί τε τὰ αὑτῶν ἕκαστοι ὅσον ἀποζῆν καὶ περιουσίαν χρημάτων οὐκ ἔχοντες οὐδὲ γῆν φυτεύοντες, ἄδηλον ὂν ὁπότε τις ἐπελθὼν καὶ ἀτειχίστων ἅμα ὄντων ἄλλος ἀφαιρήσεται, τῆς τε καθ’ ἡμέραν ἀναγκαίου τροφῆς πανταχοῦ ἂν ἡγούμενοι ἐπικρατεῖν, οὐ χαλεπῶς ἀπανίσταντο, καὶ δι’ αὐτὸ οὔτε μεγέθει πόλεων ἴσχυον οὔτε τῇ ἄλλῃ παρασκευῇ. μάλιστα δὲ τῆς γῆς ἡ ἀρίστη αἰεὶ τὰς μεταβολὰς τῶν οἰκητόρων εἶχεν, ἥ τε νῦν Θεσσαλία καλουμένη καὶ Βοιωτία Πελοποννήσου τε τὰ πολλὰ πλὴν Ἀρκαδίας, τῆς τε ἄλλης ὅσα ἦν κράτιστα.

}
}
\end{greek}
