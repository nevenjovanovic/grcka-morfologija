%Unesi korekcije NZ, NJ



\section*{O tekstu}

Aristotelova \textit{Retorika}, najvažnije i najutjecajnije djelo zapadne civilizacije kojim se usustavljuje znanje o umijeću uvjeravanja, nastajala je tijekom dvaju razdoblja u kojima je autor živio u Ateni: između 367.\ i 347.\ te između 335.\ i 322.\ pr.~Kr. 

U prvoj od tri knjige djela Aristotel razmatra odnos retorike i dijalektike, daje definiciju retorike i uvodi trojaku podjelu sredstava uvjeravanja, koja se mogu temeljiti na karakteru govornika \textgreek[variant=ancient]{(ἦθος),} na osjećajima slušatelja \textgreek[variant=ancient]{(πάθος)} te na logičkom argumentu \textgreek[variant=ancient]{(λόγος).} Zatim, upravo u ovdje odabranom odlomku, donosi – ponovno trojaku – podjelu govorništva, s obzirom na to da se njime što savjetuje ili na što upozorava u skupštini (savjetodavno govorništvo), da se koga optužuje ili brani na sudu (sudsko govorništvo), odnosno hvali ili kudi o različitim prigodama (epideiktičko govorništvo). 

U nastavku prve knjige slijedi detaljnija rasprava o trima navedenim vrstama, osobito s obzirom na upotrebu logičkih argumenata, dok se druga knjiga fokusira na govorničko izazivanje osjećaja kod slušatelja te na različite vrste karaktera. Treća se knjiga bavi pitanjima stila i dijelovima govora.


%\newpage

\section*{Pročitajte naglas grčki tekst.}
Arist.\ Rhetorica 1358a
%Naslov prema izdanju

\medskip

{\large
\begin{greek}
\noindent ῎Εστιν δὲ τῆς ῥητορικῆς εἴδη τρία τὸν ἀριθμόν· τοσοῦτοι γὰρ καὶ οἱ ἀκροαταὶ τῶν λόγων ὑπάρχουσιν ὄντες. σύγκειται μὲν γὰρ ἐκ τριῶν ὁ λόγος, ἔκ τε τοῦ λέγοντος καὶ περὶ οὗ λέγει καὶ πρὸς ὅν, καὶ τὸ τέλος πρὸς τοῦτόν ἐστιν, λέγω δὲ τὸν ἀκροατήν. ἀνάγκη δὲ τὸν ἀκροατὴν ἢ θεωρὸν εἶναι ἢ κριτήν, κριτὴν δὲ ἢ τῶν γεγενημένων ἢ τῶν μελλόντων. ἔστιν δ' ὁ μὲν περὶ τῶν μελλόντων κρίνων ὁ ἐκκλησιαστής, ὁ δὲ περὶ τῶν γεγενημένων [οἷον] ὁ δικαστής, ὁ δὲ περὶ τῆς δυνάμεως ὁ θεωρός, ὥστ' ἐξ ἀνάγκης ἂν εἴη τρία γένη τῶν λόγων τῶν ῥητορικῶν, συμβουλευτικόν, δικανικόν, ἐπιδεικτικόν. συμβουλῆς δὲ τὸ μὲν προτροπή, τὸ δὲ ἀποτροπή· ἀεὶ γὰρ καὶ οἱ ἰδίᾳ συμβουλεύοντες καὶ οἱ κοινῇ δημηγοροῦντες τούτων θάτερον ποιοῦσιν. δίκης δὲ τὸ μὲν κατηγορία, τὸ δ' ἀπολογία· τούτων γὰρ ὁποτερονοῦν ποιεῖν ἀνάγκη τοὺς ἀμφισβητοῦντας. ἐπιδεικτικοῦ δὲ τὸ μὲν ἔπαινος τὸ δὲ ψόγος.

\end{greek}
}


