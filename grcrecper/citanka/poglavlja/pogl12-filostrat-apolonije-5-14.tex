%unesi korekture NČ 2019-08-24
\section*{O autoru}

Filostrat \textgreek[variant=ancient]{(Φλαούιος Φιλόστρατος,} lat.\ Flavius Philostratus): na temelju nepovezanih obavijesti u natuknicama bizantskog leksikona \textit{Suda} danas poznajemo četiri člana iste obitelji koje skupno nazivamo Filostrati. Autor \textit{Života Apolonija iz Tijane} \textgreek[variant=ancient]{(τὰ εἰς τὸν Τυανέα Ἀπολλώνιον)} sin je prvog Filostrata, Verova sina, sofista čije se nijedno djelo nije sačuvalo. 

Lucije Flavije Filostrat (``Atenjanin'' – obnašao je dvije visoke dužnosti u Ateni) dospio je, zahvaljujući sofističkom umijeću i vezama, na dvor Septimija Severa (145.–211). Careva ga je žena Julija Domna, kako sam tvrdi, potakla da napiše romansiranu biografiju Apolonija iz Tijane u osam knjiga, objavljenu nakon 217. Kasnije je, možda 242, dovršio \textit{Živote sofista} posvećene caru Gordijanu. Mogući je autor i djela \textit{O kultu heroja, O tjelovježbi, Pisma, Slike} i \textit{Neron}. 

Zbog opsega i žanrovske raznolikosti djela, stilističke virtuoznosti u kratkim rečenicama, sklonosti metafori i paradoksu, Filostrata smatraju vodećom kreativnom osobnošću u grčkoj književnosti carskoga razdoblja. Bio je cijenjen i u Bizantu i Renesansi.

\section*{O tekstu}

\textit{Život Apolonija iz Tijane} svojevrstan je biografski eksperiment u kojem se prožimaju aspekti tada omiljenih ``idealnih'' romana s njihovim pretečom, Ksenofonotovom \textit{Kirupedijom}. Filostrat je Apolonija, oko čijeg su se lika rano počele raspredati čudesne pripovijesti, želio od vrača uzdići do ranga novopitagorskog asketa i čudotvorca \textgreek[variant=ancient]{(θεῖος ἀνήρ).} Povezujući s aretalogijom motive bajkovitih romana o putovanjima, dijelovima svoje pripovijesti, poput onog o mudračevu boravku u Indiji, dao je orijentalan ugođaj, kakav je posebno privlačio Filostratovu moćnu zaštitnicu Juliju Domnu. 

U ovom se odlomku tematizira upotreba i učinak mita, odnosno fikcije, kod Ezopa, nasuprot obradi kod uglednih pjesnika.

%\newpage

\section*{Pročitajte naglas grčki tekst.}

Philostr.\ Vita Apollonii 5.14

%Naslov prema izdanju

\medskip

\begin{greek}
{\large
{ \noindent  Ἄρξαι δ' αὐτῶν τὸν Ἀπολλώνιον ὧδε ἐρόμενον τοὺς ἑταίρους ``ἔστι τι μυθολογία;'' 

\noindent ``νὴ Δί'”, εἶπεν ὁ Μένιππος ``ἥν γε οἱ ποιηταὶ ἐπαινοῦσι''. 

\noindent ``τὸν δὲ δὴ Αἴσωπον τί ἡγῇ;'' 

\noindent ``μυθολόγον'' εἶπε ``καὶ λογοποιὸν πάντα''. 

\noindent ``πότεροι δὲ σοφοὶ τῶν μύθων;'' 

\noindent ``οἱ τῶν ποιητῶν'', εἶπεν ``ἐπειδὴ ὡς γεγονότες ᾄδονται''. 

\noindent ``οἱ δὲ δὴ Αἰσώπου τί;'' 

\noindent ``βάτραχοι'' ἔφη ``καὶ ὄνοι καὶ λῆροι γραυσὶν οἷοι μασᾶσθαι καὶ παιδίοις''. 

\noindent ``καὶ μὴν'' ἔφη ``ἐμοὶ'' ὁ Ἀπολλώνιος, ``ἐπιτηδειότεροι πρὸς σοφίαν οἱ τοῦ Αἰσώπου φαίνονται· οἱ μὲν γὰρ περὶ τοὺς ἥρωας, ὧν ποιητικὴ πᾶσα ἔχεται, καὶ διαφθείρουσι τοὺς ἀκροωμένους, ἐπειδὴ ἔρωτάς τε ἀτόπους οἱ ποιηταὶ ἑρμηνεύουσι καὶ ἀδελφῶν γάμους καὶ διαβολὰς ἐς θεοὺς καὶ βρώσεις παίδων καὶ πανουργίας ἀνελευθέρους καὶ δίκας, καὶ τὸ ὡς γεγονὸς αὐτῶν ἄγει καὶ τὸν ἐρῶντα καὶ τὸν ζηλοτυποῦντα καὶ τὸν ἐπιθυμοῦντα πλουτεῖν ἢ τυραννεύειν ἐφ' ἅπερ οἱ μῦθοι, Αἴσωπος δὲ ὑπὸ σοφίας πρῶτον μὲν οὐκ ἐς τὸ κοινὸν τῶν ταῦτα ᾀδόντων ἑαυτὸν κατέστησεν, ἀλλ' ἑαυτοῦ τινα ὁδὸν ἐτράπετο, εἶτα, ὥσπερ οἱ τοῖς εὐτελεστέροις βρώμασι καλῶς ἑστιῶντες, ἀπὸ σμικρῶν πραγμάτων διδάσκει μεγάλα, καὶ προθέμενος τὸν λόγον ἐπάγει αὐτῷ τὸ πρᾶττε ἢ μὴ πρᾶττε, εἶτα τοῦ φιλαλήθους μᾶλλον ἢ οἱ ποιηταὶ ἥψατο.''

}
}
\end{greek}

