% Unio ispravke NČ 2019-09
%\section*{O autoru}


\section*{O tekstu}

Izabrani odlomak govori o događaju iz 415.\ pr.~Kr., kad su u Ateni, tijekom noći uoči odlaska na Sicilsku ekspediciju, oskvrnute gotovo sve gradske herme, a to je u narodu izazvalo veliku uznemirenost jer je sve protumačeno kao loš znamen na početku ratnog pohoda. Za oskvrnuće je optužen Alkibijad; optužili su ga političari kojima je smetao da preuzmu vlast u gradu; isticali su i Alkibijadove navodne namjere da u Ateni ukine demokraciju.

%\newpage

\section*{Pročitajte naglas grčki tekst.}

Thuc.\ Historiae 6.27

%Naslov prema izdanju

\medskip

\begin{greek}
{\large
{ \noindent Ἐν δὲ τούτῳ, ὅσοι Ἑρμαῖ ἦσαν λίθινοι ἐν τῇ πόλει τῇ Ἀθηναίων (εἰσὶ δὲ κατὰ τὸ ἐπιχώριον, ἡ τετράγωνος ἐργασία, πολλοὶ καὶ ἐν ἰδίοις προθύροις καὶ ἐν ἱεροῖς), μιᾷ νυκτὶ οἱ πλεῖστοι περιεκόπησαν τὰ πρόσωπα. καὶ τοὺς δράσαντας ᾔδει οὐδείς, ἀλλὰ μεγάλοις μηνύτροις δημοσίᾳ οὗτοί τε ἐζητοῦντο καὶ προσέτι ἐψηφίσαντο, καὶ εἴ τις ἄλλο τι οἶδεν ἀσέβημα γεγενημένον, μηνύειν ἀδεῶς τὸν βουλόμενον καὶ ἀστῶν καὶ ξένων καὶ δούλων. καὶ τὸ πρᾶγμα μειζόνως ἐλάμβανον· τοῦ τε γὰρ ἔκπλου οἰωνὸς ἐδόκει εἶναι καὶ ἐπὶ ξυνωμοσίᾳ ἅμα νεωτέρων πραγμάτων καὶ δήμου καταλύσεως  γεγενῆσθαι. μηνύεται οὖν ἀπὸ μετοίκων τέ τινων καὶ ἀκολούθων περὶ μὲν τῶν Ἑρμῶν οὐδέν, ἄλλων δὲ ἀγαλμάτων περικοπαί τινες πρότερον ὑπὸ νεωτέρων μετὰ παιδιᾶς καὶ οἴνου γεγενημέναι, καὶ τὰ μυστήρια ἅμα ὡς ποιεῖται ἐν οἰκίαις ἐφ' ὕβρει· ὧν καὶ τὸν Ἀλκιβιάδην ἐπῃτιῶντο. καὶ αὐτὰ ὑπολαμβάνοντες οἱ μάλιστα τῷ Ἀλκιβιάδῃ ἀχθόμενοι ἐμποδὼν ὄντι σφίσι μὴ αὐτοῖς τοῦ δήμου βεβαίως προεστάναι, καὶ νομίσαντες, εἰ αὐτὸν ἐξελάσειαν, πρῶτοι ἂν εἶναι, ἐμεγάλυνον καὶ ἐβόων ὡς ἐπὶ δήμου καταλύσει τά τε μυστικὰ καὶ ἡ τῶν Ἑρμῶν περικοπὴ γένοιτο καὶ οὐδὲν εἴη αὐτῶν ὅ τι οὐ μετ' ἐκείνου ἐπράχθη, ἐπιλέγοντες τεκμήρια τὴν ἄλλην αὐτοῦ ἐς τὰ ἐπιτηδεύματα οὐ δημοτικὴν παρανομίαν. 

}
}
\end{greek}

