% redaktura NZ (ne slažem se s ὅσον πλοῦν i δυνάμει ῾Ελληνικῇ), manja redaktura NJ
%\section*{O autoru}



\section*{O tekstu}

Sedamnaeste godine Peloponeskog rata (415.\ pr.\ Kr.) izglasan je pohod Atenjana i saveznika protiv gradova na Siciliji (tzv.\ sicilska ekspedicija), te se golema flota od preko stotinu trirema i nekoliko tisuća ljudi sprema isploviti iz luke Pirej.

%\newpage

\section*{Pročitajte naglas grčki tekst.}

%Naslov prema izdanju

Thuc.\ Historiae 6.30

\medskip

{\large
\begin{greek}
\noindent Αὐτοὶ δ' ᾿Αθηναῖοι καὶ εἴ τινες τῶν ξυμμάχων παρῆσαν, ἐς τὸν Πειραιᾶ καταβάντες ἐν ἡμέρᾳ ῥητῇ ἅμα ἕῳ ἐπλήρουν τὰς ναῦς ὡς ἀναξόμενοι. ξυγκατέβη δὲ καὶ ὁ ἄλλος ὅμιλος ἅπας ὡς εἰπεῖν ὁ ἐν τῇ πόλει καὶ ἀστῶν καὶ ξένων, οἱ μὲν ἐπιχώριοι τοὺς σφετέρους αὐτῶν ἕκαστοι προπέμποντες, οἱ μὲν ἑταίρους, οἱ δὲ ξυγγενεῖς, οἱ δὲ υἱεῖς, καὶ μετ' ἐλπίδος τε ἅμα ἰόντες καὶ ὀλοφυρμῶν, τὰ μὲν ὡς κτήσοιντο, τοὺς δ' εἴ ποτε ὄψοιντο, ἐνθυμούμενοι ὅσον πλοῦν ἐκ τῆς σφετέρας ἀπεστέλλοντο. καὶ ἐν τῷ παρόντι καιρῷ, ὡς ἤδη ἔμελλον μετὰ κινδύνων ἀλλήλους ἀπολιπεῖν, μᾶλλον αὐτοὺς ἐσῄει τὰ δεινὰ ἢ ὅτε ἐψηφίζοντο πλεῖν· ὅμως δὲ τῇ παρούσῃ ῥώμῃ, διὰ τὸ πλῆθος ἑκάστων ὧν ἑώρων, τῇ ὄψει ἀνεθάρσουν. οἱ δὲ ξένοι καὶ ὁ ἄλλος ὄχλος κατὰ θέαν ἧκεν ὡς ἐπ' ἀξιόχρεων καὶ ἄπιστον διάνοιαν. παρασκευὴ γὰρ αὕτη πρώτη ἐκπλεύσασα μιᾶς πόλεως δυνάμει ῾Ελληνικῇ πολυτελεστάτη δὴ καὶ εὐπρεπεστάτη τῶν ἐς ἐκεῖνον τὸν χρόνον ἐγένετο.

\end{greek}

}

