% Unesi korekture NČ i NZ, 2019-09-24



\section*{O tekstu}

Životopis Demostena (384.–322.\ pr.~Kr.), najslavnijeg grčkog govornika, Plutarh je pridružio opisu najslavnijeg rimskog govornika, Cicerona (106.–44.\ pr.~Kr.). U izabranom odlomku pripovijeda se kako je Demosten vježbao govorničke vještine i pokušavao ispraviti vlastite nedostatke te koliko je cijenio govorničku izvedbu.

%\newpage

\section*{Pročitajte naglas grčki tekst.}
Plut.\ Demosthenes 11
%Naslov prema izdanju

\medskip

{\large
\begin{greek}

\noindent Τοῖς δὲ σωματικοῖς ἐλαττώμασι τοιαύτην ἐπῆγεν ἄσκησιν, ὡς ὁ Φαληρεὺς Δημήτριος ἱστορεῖ, λέγων αὐτοῦ Δημοσθένους ἀκοῦσαι πρεσβύτου γεγονότος· τὴν μὲν γὰρ ἀσάφειαν καὶ τραυλότητα τῆς γλώττης ἐκβιάζεσθαι καὶ διαρθροῦν εἰς τὸ στόμα ψήφους λαμβάνοντα καὶ ῥήσεις ἅμα λέγοντα, τὴν δὲ φωνὴν γυμνάζειν ἐν τοῖς δρόμοις καὶ ταῖς πρὸς τὰ σιμ' ἀναβάσεσι διαλεγόμενον καὶ λόγους τινὰς ἢ στίχους ἅμα τῷ πνεύματι πυκνουμένῳ προφερόμενον· εἶναι δ' αὐτῷ μέγα κάτοπτρον οἴκοι, καὶ πρὸς τοῦτο τὰς μελέτας ἱστάμενον ἐξ ἐναντίας περαίνειν. 

λέγεται δ' ἀνθρώπου προσελθόντος αὐτῷ δεομένου συνηγορίας καὶ διεξιόντος ὡς ὑπό του λάβοι πληγάς, ``ἀλλὰ σύ γε'', φάναι τὸν Δημοσθένην, ``τούτων ὧν λέγεις οὐδὲν πέπονθας.'' ἐπιτείναντος δὲ τὴν φωνὴν τοῦ ἀνθρώπου καὶ βοῶντος ``ἐγὼ Δημόσθενες οὐδὲν πέπονθα;'' ``νὴ Δία'' φάναι, ``νῦν ἀκούω φωνὴν ἀδικουμένου καὶ πεπονθότος.'' οὕτως ᾤετο μέγα πρὸς πίστιν εἶναι τὸν τόνον καὶ τὴν ὑπόκρισιν τῶν λεγόντων.

\end{greek}

}

