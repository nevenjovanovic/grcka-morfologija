% redaktura NZ, unio NJ
%\section*{O autoru}



\section*{O tekstu}

Djelo \textit{Razgovori bogova} \textgreek[variant=ancient]{(Θεῶν διάλογοι)} zbirka je sastavljena od 26 samostalnih dijaloga grčkih bogova i heroja, koji su kod Lukijana prikazani kao kompleksni likovi, svojim manama, prepirkama i skandalima kudikamo bliži običnim smrtnicima nego uzvišenim bogovima homerske i heziodovske pjesničke tradicije. 

Ovdje se donosi odlomak iz epizode o Parisovu sudu, koji je tradicionalno prenošen kao dio zbirke \textit{Razgovori bogova}, ali ga se katkad, s obzirom na obilježja različita od onih u ostalim dijalozima, izdvaja i kao zasebno djelo pod nazivom \textit{Sud o božicama (Iudicium dearum).} U razgovoru sudjeluju Zeus, Hermes, Hera, Atena, Afrodita te Paris, koji, prema Zeusovoj odluci, mora presuditi kojoj će od tri olimpske božice pripasti Eridina jabuka s natpisom \textgreek[variant=ancient]{καλλίστῃ} (``najljepšoj'').

%\newpage

\section*{Pročitajte naglas grčki tekst.}

%Naslov prema izdanju

Luc.\ Dialogi deorum 20.7

\medskip

{\large
\begin{greek}
\noindent Πῶς ἂν οὖν, ὦ δέσποτα  Ἑρμῆ, δυνηθείην ἐγὼ θνητὸς αὐτὸς καὶ ἀγροῖκος ὢν δικαστὴς γενέσθαι παραδόξου θέας καὶ μείζονος ἢ κατὰ βουκόλον; τὰ γὰρ τοιαῦτα κρίνειν τῶν ἁβρῶν μᾶλλον καὶ ἀστικῶν· τὸ δὲ ἐμόν, αἶγα μὲν αἰγὸς ὁποτέρα ἡ καλλίων καὶ δάμαλιν ἄλλης δαμάλεως, τάχ' ἂν δικάσαιμι κατὰ τὴν τέχνην· αὗται δὲ πᾶσαί τε ὁμοίως καλαὶ καὶ οὐκ οἶδ' ὅπως ἄν τις ἀπὸ τῆς ἑτέρας ἐπὶ τὴν ἑτέραν μεταγάγοι τὴν ὄψιν ἀποσπάσας· οὐ γὰρ ἐθέλει ἀφίστασθαι ῥᾳδίως, ἀλλ' ἔνθα ἂν ἀπερείσῃ τὸ πρῶτον, τούτου ἔχεται καὶ τὸ παρὸν ἐπαινεῖ· κἂν ἐπ' ἄλλο μεταβῇ, κἀκεῖνο καλὸν ὁρᾷ καὶ παραμένει, καὶ ὑπὸ τῶν πλησίον παραλαμβάνεται. καὶ ὅλως περικέχυταί μοι τὸ κάλλος αὐτῶν καὶ ὅλον περιείληφέ με καὶ ἄχθομαι, ὅτι μὴ καὶ αὐτὸς ὥσπερ ὁ Ἄργος ὅλῳ βλέπειν δύναμαι τῷ σώματι. δοκῶ δ' ἄν μοι καλῶς δικάσαι πάσαις ἀποδοὺς τὸ μῆλον.

\end{greek}

}

