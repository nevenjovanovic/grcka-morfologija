% Unesi korekture NČ, NZ 2019-09-23
\section*{O tekstu}

Kada je 349.\ pr.~Kr.\ Filip Makedonski napao Olint, grad na Halkidici koji je bio sklopio savez s Atenom, Demosten je održao tri govora potičući Atenjane da interveniraju. Usprkos Demostenovu zauzimanju, pravovremena i adekvatna akcija Atene je izostala. Drugi olintski govor reakcija je na oklijevanje Atenjana, koji ne žele uputiti Olinćanima obećanu pomoć. Demosten ih ohrabruje i dokazuje da su Filipa napustili saveznici, te da Filip sam nema više dovoljno snage. U ovom odlomku Demosten opisuje kako se Filip ponaša prema stranim vojnicima i suradnicima, ističući kao negativan primjer državnog roba Kaliju.

%\newpage

\section*{Pročitajte naglas grčki tekst.}
Dem. Olynthiaca II 18
%Naslov prema izdanju

\medskip

{\large
\begin{greek}
\noindent Εἰ μὲν γάρ τις ἀνήρ ἐστιν ἐν αὐτοῖς οἷος ἔμπειρος πολέμου καὶ ἀγώνων, τούτους μὲν φιλοτιμίᾳ πάντας ἀπωθεῖν αὐτὸν ἔφη, βουλόμενον πάνθ' αὑτοῦ δοκεῖν εἶναι τἄργα (πρὸς γὰρ αὖ τοῖς ἄλλοις καὶ  τὴν φιλοτιμίαν ἀνυπέρβλητον εἶναι)· εἰ δέ τις σώφρων ἢ δίκαιος ἄλλως, τὴν καθ' ἡμέραν ἀκρασίαν τοῦ βίου καὶ μέθην καὶ κορδακισμοὺς οὐ δυνάμενος φέρειν, παρεῶσθαι καὶ ἐν οὐδενὸς εἶναι μέρει τὸν τοιοῦτον. λοιποὺς δὴ περὶ αὐτὸν εἶναι λῃστὰς καὶ κόλακας καὶ τοιούτους ἀνθρώπους οἵους μεθυσθέντας ὀρχεῖσθαι τοιαῦθ' οἷ' ἐγὼ νῦν ὀκνῶ πρὸς ὑμᾶς ὀνομάσαι. δῆλον δ' ὅτι ταῦτ' ἐστὶν ἀληθῆ· καὶ γὰρ οὓς ἐνθένδε πάντες ἀπήλαυνον ὡς πολὺ τῶν θαυματοποιῶν ἀσελγεστέρους ὄντας, Καλλίαν ἐκεῖνον τὸν δημόσιον καὶ τοιούτους ἀνθρώπους, μίμους γελοίων καὶ ποιητὰς αἰσχρῶν ᾀσμάτων, ὧν εἰς τοὺς συνόντας ποιοῦσιν εἵνεκα τοῦ γελασθῆναι, τούτους ἀγαπᾷ καὶ περὶ αὑτὸν ἔχει.

\end{greek}

}

