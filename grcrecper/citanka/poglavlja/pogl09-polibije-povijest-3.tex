% korektura NZ, NJ
\section*{O autoru}

Polibije \textgreek[variant=ancient]{(Πολύβιος)} rođen je u Megalopolu u Arkadiji 200.\ pr.~Kr. Pripadao je utjecajnoj obitelji i kao mladić se borio pod zapovjedništvom vojskovođe Filopemena. U dobi od 30 godina bio je jedan od vođa Ahejskog saveza, udruženja grčkih polisa koji su za vrijeme sukoba između Rima i Makedonije pokušavali ostati neutralni, no i ta neutralnost je imala svoju cijenu. Nakon bitke kod Pidne (168.\ pr.~Kr), u kojoj su Rimljani porazili makedonskog kralja Perzeja, Polibije je stigao u Rim kao jedan od tisuću ahejskih talaca. Zbližio se s krugom oko Emilija Paula i Publija Kornelija Scipiona Mlađeg. Potonjeg je pratio u Trećem punskom ratu (149.\ – 146.\ pr.~Kr) gdje je njegovo prethodno ratno iskustvo bilo od velike koristi (osmislio je sustav šifriranih signala). Kad su Rimljani konačno pokorili i Ahejski savez, Polibije se trudio isposlovati prihvatljive uvjete predaje za svoje sunarodnjake. U znak zahvalnosti mu je šest polisa podiglo spomenik. Umro je 118.\ pr.~Kr.

\section*{O tekstu}

U djelu \textit{Povijest} \textgreek[variant=ancient]{(Ἱστορίαι)} Polibije prikazuje vojni i politički uspon Rima. Zahvaljujući prijateljstvu s brojnim rimskim uglednicima, autor je imao pristup državnim dokumentima i arhivima. Za potrebe istraživanja razgovarao je i s velikim brojem sudionika u suvremenim povijesnim zbivanjima. Naposljetku, svojim očima je vidio kako su Rimljani Kartagu sravnili sa zemljom, te je \textit{Povijest} dijelom rezultat autopsije. Djelo se sastojalo od 40 knjiga od kojih je samo prvih pet sačuvano u cijelosti. Slijedi odlomak iz treće knjige u kojem se Hanibal obraća vojnicima prije bitke kod Kane u Apuliji, u jugoistočnoj Italiji (216.\ pr.~Kr); u ovoj će velikoj bici Hanibalove snage pobijediti nadmoćnu vojsku Rimske Republike.

%\newpage

\section*{Pročitajte naglas grčki tekst.}
Polyb.\ Historiae 3.111.3
%Naslov prema izdanju

\medskip

{\large
\begin{greek}
\noindent Πρῶτον μὲν τοῖς θεοῖς ἔχετε χάριν· ἐκεῖνοι γὰρ ἡμῖν συγκατασκευάζοντες τὴν νίκην εἰς τοιούτους τόπους ἤχασι τοὺς ἐχθρούς· δεύτερον δ' ἡμῖν, ὅτι καὶ μάχεσθαι τοὺς πολεμίους συνηναγκάσαμεν· οὐ γὰρ ἔτι δύνανται τοῦτο διαφυγεῖν· καὶ μάχεσθαι προφανῶς ἐν τοῖς ἡμετέροις προτερήμασι. τὸ δὲ παρακαλεῖν ὑμᾶς νῦν διὰ πλειόνων εὐθαρσεῖς καὶ προθύμους εἶναι πρὸς τὸν κίνδυνον οὐδαμῶς μοι δοκεῖ καθήκειν. ὅτε μὲν γὰρ ἀπείρως διέκεισθε τῆς πρὸς ῾Ρωμαίους μάχης, ἔδει τοῦτο ποιεῖν, καὶ μεθ' ὑποδειγμάτων ἐγὼ πρὸς ὑμᾶς πολλοὺς διεθέμην λόγους· ὅτε δὲ κατὰ τὸ συνεχὲς τρισὶ μάχαις τηλικαύταις ἐξ ὁμολογουμένου νενικήκατε ῾Ρωμαίους, ποῖος ἂν ἔτι λόγος ὑμῖν ἰσχυρότερον παραστήσαι θάρσος αὐτῶν τῶν ἔργων; διὰ μὲν οὖν τῶν πρὸ τοῦ κινδύνων κεκρατήκατε τῆς χώρας καὶ τῶν ἐκ ταύτης ἀγαθῶν κατὰ τὰς ἡμετέρας ἐπαγγελίας, ἀψευστούντων ἡμῶν ἐν πᾶσι τοῖς πρὸς ὑμᾶς εἰρημένοις· ὁ δὲ νῦν ἀγὼν ἐνέστηκεν περὶ τῶν πόλεων καὶ τῶν ἐν αὐταῖς ἀγαθῶν. οὗ κρατήσαντες κύριοι μὲν ἔσεσθε παραχρῆμα πάσης ᾿Ιταλίας, ἀπαλλαγέντες δὲ τῶν νῦν πόνων, γενόμενοι συμπάσης ἐγκρατεῖς τῆς ῾Ρωμαίων εὐδαιμονίας, ἡγεμόνες ἅμα καὶ δεσπόται πάντων γενήσεσθε διὰ ταύτης τῆς μάχης. διόπερ οὐκέτι λόγων ἀλλ' ἔργων ἐστὶν ἡ χρεία· θεῶν γὰρ βουλομένων ὅσον οὔπω βεβαιώσειν ὑμῖν πέπεισμαι τὰς ἐπαγγελίας.

\end{greek}
}

