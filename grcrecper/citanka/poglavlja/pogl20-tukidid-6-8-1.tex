% Unesi korekture NZ, NJ 2019-08-09
%\section*{O autoru}



\section*{O tekstu}

Izabrani odlomak iz šeste knjige Tukididova \textit{Spisa o ratu Peloponežana i Atenjana} \textgreek[variant=ancient]{(Ξυγγραφὴ περὶ τoῦ πολέμoυ τῶν Пελoπoννησίων καὶ Ἀϑηναίων)} govori o događanjima s početka 415.\ pr.~Kr, kad je sicilski grad Segesta \textgreek[variant=ancient]{(Ἔγεστα)} Atenjanima poslao zahtjev i mjesečnu novčanu naknadu za slanje šezdeset ratnih brodova, da ih podupru u ratnim sukobima sa Selinuntom \textgreek[variant=ancient]{(Σελινοῦς).} Segešćani lažno predstavljaju svoju financijsku moć, no Atenjani odlučuju zahtjevu udovoljiti, a usput, ako im ratne prilike dozvole, pomoći i žiteljima sicilskoga grada Leontina \textgreek[variant=ancient]{(Λεοντῖνοι).} To je bio početak znamenite i za Atenu pogubne Sicilske ekspedicije (415–413).

%\newpage

\section*{Pročitajte naglas grčki tekst.}

%Naslov prema izdanju

Thuc.\ Historiae 6.8.1

\medskip

{\large
\begin{greek}
\noindent Τοῦ δ' ἐπιγιγνομένου θέρους ἅμα ἦρι οἱ τῶν Ἀθηναίων πρέσβεις ἧκον ἐκ τῆς Σικελίας καὶ οἱ Ἐγεσταῖοι μετ' αὐτῶν ἄγοντες ἑξήκοντα τάλαντα ἀσήμου ἀργυρίου ὡς ἐς ἑξήκοντα ναῦς μηνὸς μισθόν, ἃς ἔμελλον δεήσεσθαι πέμπειν. καὶ οἱ Ἀθηναῖοι ἐκκλησίαν ποιήσαντες καὶ ἀκούσαντες τῶν τε Ἐγεσταίων καὶ τῶν σφετέρων πρέσβεων τά τε ἄλλα ἐπαγωγὰ καὶ οὐκ ἀληθῆ καὶ περὶ τῶν χρημάτων ὡς εἴη ἑτοῖμα ἔν τε τοῖς ἱεροῖς πολλὰ καὶ ἐν τῷ κοινῷ, ἐψηφίσαντο ναῦς ἑξήκοντα πέμπειν ἐς Σικελίαν καὶ στρατηγοὺς αὐτοκράτορας Ἀλκιβιάδην τε τὸν Κλεινίου καὶ Νικίαν τὸν Νικηράτου καὶ Λάμαχον τὸν Ξενοφάνους, βοηθοὺς μὲν Ἐγεσταίοις πρὸς Σελινουντίους, ξυγκατοικίσαι δὲ καὶ Λεοντίνους, ἤν τι περιγίγνηται αὐτοῖς τοῦ πολέμου, καὶ τἆλλα τὰ ἐν τῇ Σικελίᾳ πρᾶξαι ὅπῃ ἂν γιγνώσκωσιν ἄριστα Ἀθηναίοις.


\end{greek}

}

