%korektura NZ
%\section*{O autoru}



\section*{O tekstu}

Odlomak opisuje glasovitog atenskog političara i vojskovođu Temistokla \textgreek[variant=ancient]{(Θεμιστοκλῆς}, oko 524.\ – 459.\ pr.~Kr). Uočavamo osobine Plutarhova pristupa biografiji: prikazuju se događaji koji najbolje ocrtavaju karakterne osobine Temistokla još od doba dok je bio dječak. 

Temistoklo je bio zaslužan za izgradnju zida od Atene do Pireja i, u kontekstu perzijskog pritiska na Grčku, zagovarao je blisko vojno povezivanje sa Spartom. Sudjelovao je u bitci na Maratonskom polju (490.\ pr.~Kr), a nakon poraza kod Termopila vodio je grčke snage do pobjede u bitci kod Salamine (480.\ pr.~Kr). Oko 470.\ pr.~Kr.\ ostracizmom biva protjeran iz Atene. Umire kao namjesnik perzijskoga kralja u Magneziji. 

U \textit{Paralelnim životopisima} Temistoklova je biografija pridružena onoj rimskog političara Marka Furija Kamila (oko 446. – 365.\ pr.~Kr). 

%\newpage

\section*{Pročitajte naglas grčki tekst.}
Plut. Themistocles 2.1
%Naslov prema izdanju

\medskip

{\large
\begin{greek}
\noindent ῎Ετι δὲ παῖς ὢν ὁμολογεῖται φορᾶς μεστὸς εἶναι, καὶ τῇ μὲν φύσει συνετός, τῇ δὲ προαιρέσει μεγαλοπράγμων καὶ πολιτικός. ἐν γὰρ ταῖς ἀνέσεσι καὶ σχολαῖς ἀπὸ τῶν μαθημάτων γιγνόμενος, οὐκ ἔπαιζεν οὐδ' ἐρρᾳθύμει καθάπερ οἱ πολλοὶ παῖδες, ἀλλ' εὑρίσκετο λόγους τινὰς μελετῶν καὶ συνταττόμενος πρὸς ἑαυτόν. ἦσαν δ' οἱ λόγοι κατηγορία τινὸς ἢ συνηγορία τῶν παίδων. ὅθεν εἰώθει λέγειν πρὸς αὐτὸν ὁ διδάσκαλος ὡς ‘οὐδὲν ἔσει, παῖ, σὺ μικρόν, ἀλλὰ μέγα πάντως ἀγαθὸν ἢ κακόν’. ἐπεὶ καὶ τῶν παιδεύσεων τὰς μὲν ἠθοποιοὺς ἢ πρὸς ἡδονήν τινα καὶ χάριν ἐλευθέριον σπουδαζομένας ὀκνηρῶς καὶ ἀπροθύμως ἐξεμάνθανε, τῶν δ' εἰς σύνεσιν ἢ πρᾶξιν † λεγομένων δῆλος ἦν ὑπερερῶν παρ' ἡλικίαν ὡς τῇ φύσει πιστεύων.

\end{greek}
}

