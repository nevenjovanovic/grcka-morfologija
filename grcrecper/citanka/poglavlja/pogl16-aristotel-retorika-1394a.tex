% Unesi korekture NČ, NZ 2019-09-23

\section*{O tekstu}

U drugoj knjizi \textit{Retorike} Aristotel izlaže načine na koje govornik može uvjeriti slušaoce, i preduvjete za takvo uvjeravanje. Poseban su način uvjeravanja takozvana opća mjesta, dokazi da se nešto može ili ne može dogoditi. Osim logičkih dokaza, opća su mjesta i primjeri; njihova su pak posebna podvrsta pripovijesti, poput prispodoba i basni. Drugi po redu primjer za upotrebu basne u procesu političkog odlučivanja Aristotel pripisuje Ezopu, legendarnom basnopiscu, po predaji tračkom robu iz VI.~st.\ pr.~Kr.\ prodanom na otok Sam.

%\newpage

\section*{Pročitajte naglas grčki tekst.}

%Naslov prema izdanju
Arist.\ Rhetorica 1394a
\medskip

{\large
\begin{greek}
\noindent Αἴσωπος δὲ ἐν Σάμῳ δημηγορῶν κρινομένου δημαγωγοῦ περὶ θανάτου ἔφη ἀλώπεκα διαβαίνουσαν ποταμὸν ἀπωσθῆναι εἰς φάραγγα, οὐ δυναμένην δὲ ἐκβῆναι πολὺν χρόνον κακοπαθεῖν καὶ κυνοραιστὰς πολλοὺς ἔχεσθαι αὐτῆς, ἐχῖνον δὲ πλανώμενον, ὡς εἶδεν αὐτήν, κατοικτείραντα ἐρωτᾶν εἰ ἀφέλοι αὐτῆς τοὺς κυνοραιστάς, τὴν δὲ οὐκ ἐᾶν· ἐρομένου δὲ διὰ τί, ``ὅτι οὗτοι μὲν'' φάναι ``ἤδη μου πλήρεις εἰσὶ καὶ ὀλίγον ἕλκουσιν αἷμα, ἐὰν δὲ τούτους ἀφέλητε, ἕτεροι ἐλθόντες πεινῶντες ἐκπιοῦνταί μου τὸ λοιπὸν αἷμα''. ``ἀτὰρ καὶ ὑμᾶς, ἄνδρες Σάμιοι, οὗτος μὲν οὐδὲν ἔτι βλάψει (πλούσιος γάρ ἐστιν), ἐὰν δὲ τοῦτον ἀποκτείνητε, ἕτεροι ἥξουσι πένητες, οἳ ὑμᾶς ἀναλώσουσι τὰ λοιπὰ κλέπτοντες.''

\end{greek}
}

