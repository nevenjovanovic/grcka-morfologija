% ispravci NZ
\section*{O autoru}

Platon \textgreek[variant=ancient]{(Πλάτων,} 427.–347.\ pr.~Kr.)  potječe iz ugledne atenske plemićke loze. Oko 407.\ postaje Sokratov učenik. Nakon učiteljeva smaknuća 399.\ napušta Atenu te poduzima brojna putovanja. S četrdesetak godina dolazi u Sirakuzu na Siciliji. Njegov nazor o nužnosti vladavine filozofa postaje preopasan tamošnjem vladaru, tiraninu Dioniziju I., pa oko 387.\ mora napustiti grad. Dva sljedeća putovanja u Sirakuzu (367.–65.\ i 361.–60.) k Dioniziju II.\ završila su također bez uspjeha. Oko 360.\ Platon se posvećuje isključivo vođenju Akademije, koju je  u Ateni osnovao oko 387.\ (Akademija će, mada ne bez prekida, djelovati do 529.\ po Kr, kad ju je zatvorio car Justinijan).

Spisateljsku djelatnost Platon je počeo nakon Sokratove smrti 399. Vjerojatno su sačuvani svi Platonovi spisi, tj.\ tridesetak dijaloga i niz pisama (manji se njihov dio smatra neautentičnim). Platon je djela sastavljao gotovo isključivo u dijaloškoj formi, u kojoj njegovo mišljenje kruži oko svoga predmeta u otvorenom, ispitivačkom razgovoru. Tako misli sudionika u razgovoru postižu iznimno veliku zornost i životnost. Djela se obično dijele u tri skupine: 1.\ rani dijalozi (do prvog sicilskog putovanja), među njima su \textit{Apologija, Protagora, Eutifron, Lahet, Država} I; 2.\ srednji dijalozi (do drugog sicilskog putovanja), najvažniji su \textit{Gorgija, Kratil, Menon, Fedon, Simpozij, Država} II-X, \textit{Fedar}; 3.\ kasni dijalozi (nakon 365.): \textit{Teetet, Parmenid, Sofist, Timej, Kritija, Zakoni, Državnik, Fileb}. 

Interpretacija dijaloga nameće brojne probleme. Na primjer, teško je od Platonova udjela razlučiti misaoni udio historijskog Sokrata, koji u gotovo svim dijalozima ima glavnu ulogu. Nadalje, zbog izrazito dugog vremenskog raspona nastanka dijalozi odražavaju dinamiku razvoja Platonova nauka. U ranim se djelima pomoću etičkih pojmovnih određenja provodi sokratovska metoda \textgreek[variant=ancient]{(μαιευτική,} majeutika, tj.\ primaljska vještina). Dominantna je tema vrlina \textgreek[variant=ancient]{(ἀρετή),} a dijalozi većinom završavaju bez rezultata, u aporijama. U srednjim dijalozima Platon razvija nauk o idejama koje postaju temelj teorijama o čovjeku i idealnoj državi. U kasnim dijalozima ta se diskusija produbljuje, a nauk o idejama samokritički se podvrgava temeljitoj reviziji. 

Utjecaj Platona u duhovnoj povijesti teško je prenaglasiti. Dovoljno je navesti misao britanskog filozofa A.~N.\ Whiteheada: “The safest general characterization of the European philosophical tradition is that it consists of a series of footnotes to Plato. I do not mean the systematic scheme of thought which scholars have doubtfully extracted from his writings. I allude to the wealth of general ideas scattered through them”.

\section*{O tekstu}

\textit{Meneksen} \textgreek[variant=ancient]{(Μενέξενος)} pripada ranim Platonovim dijalozima (pretpostavlja se da je nastao između 386.\ i 380.\ pr.~Kr.). Sokrat i Meneksen razgovaraju u trenutku kada se u Ateni očekuje izbor govornika koji će pogrebnim govorom popratiti predstojeće javne pogrebne počasti. Potaknut Sokratovim ironičnim primjedbama o onodobnim govornicima, Meneksen ga izaziva da sam iznese uzoran pogrebni govor, a Sokrat nato izlaže onaj koji je dan prije čuo od Aspazije, čuvene atenske intelektualke za koju kaže da je govorništvu učila i njega jednako kao i Perikla. U Sokratovu govoru u čast poginulim ratnicima, koji zauzima najveći dio \textit{Meneksena}, mnogi vide parodiju glasovitog pogrebnoga govora koji je palim Atenjanima održao Periklo, a koji prenosi Tukidid u \textit{Povijesti Peloponeskog rata}. 

Odabrani odlomak dio je Sokratova elogija Atike kao bogate i uspješne zemlje koja daje dobro podrijetlo i vrsnu naobrazbu.

%\newpage

\section*{Pročitajte naglas grčki tekst.}

Plat.\ Menex.\ 237c

\medskip

{\large
\begin{greek}
῎Εστι δὲ ἀξία ἡ χώρα καὶ ὑπὸ πάντων ἀνθρώπων ἐπαινεῖσθαι, οὐ μόνον ὑφ' ἡμῶν, πολλαχῇ μὲν καὶ ἄλλῃ, πρῶτον δὲ καὶ μέγιστον ὅτι τυγχάνει οὖσα θεοφιλής. μαρτυρεῖ δὲ ἡμῶν τῷ λόγῳ ἡ τῶν ἀμφισβητησάντων περὶ αὐτῆς θεῶν ἔρις τε καὶ κρίσις· ἣν δὴ θεοὶ ἐπῄνεσαν, πῶς οὐχ ὑπ' ἀνθρώπων γε συμπάντων δικαία ἐπαινεῖσθαι; δεύτερος δὲ ἔπαινος δικαίως ἂν αὐτῆς εἴη, ὅτι ἐν ἐκείνῳ τῷ χρόνῳ, ἐν ᾧ ἡ πᾶσα γῆ ἀνεδίδου καὶ ἔφυε ζῷα παντοδαπά, θηρία τε καὶ βοτά, ἐν τούτῳ ἡ ἡμετέρα θηρίων μὲν ἀγρίων ἄγονος καὶ καθαρὰ ἐφάνη, ἐξελέξατο δὲ τῶν ζῴων καὶ ἐγέννησεν ἄνθρωπον, ὃ συνέσει τε ὑπερέχει τῶν ἄλλων καὶ δίκην καὶ θεοὺς μόνον νομίζει.
\end{greek}

}

