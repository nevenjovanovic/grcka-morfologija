%Unesi korekture NČ 2019-09-19
%\section*{O autoru}


\section*{O tekstu}

Aristotelova je \textit{Retorika} teoretski i praktični priručnik govorništva u tri knjige koji se bavi metodama uvjeravanja riječima. Uvelike se oslanja na Platonovo poimanje istinskog govorničkog umijeća izloženo u \textit{Fedru}. Tri su sredstva uvjeravanja: 1) osobnost govornika \textgreek[variant=ancient]{(ἦθος),} 2) sposobnost da se emocionalno djeluje na slušatelja \textgreek[variant=ancient]{(πάθος)} i 3) umijeće da se iskaz tako oblikuje da djeluje istinito ili vjerojatno \textgreek[variant=ancient]{(λόγος).}

Nakon što je razmatrao posebne dokaze, u ovom poglavlju druge knjige Aristotel se osvrće na dokaze koji su zajednički svim granama retorike: primjer \textgreek[variant=ancient]{(παράδειγμα)} i entimem \textgreek[variant=ancient]{(ἐνθύμημα).}
%\newpage

\section*{Pročitajte naglas grčki tekst.}

Arist.\ Rhetorica 1393a 28

%Naslov prema izdanju

\medskip

\begin{greek}
{\large
{ \noindent Παραδειγμάτων δὲ εἴδη δύο· ἓν μὲν γάρ ἐστιν παραδείγματος εἶδος τὸ λέγειν πράγματα προγεγενημένα, ἓν δὲ τὸ αὐτὸν ποιεῖν. τούτου δὲ ἓν μὲν παραβολὴ ἓν δὲ λόγοι, οἷον οἱ Αἰσώπειοι καὶ Λιβυκοί. ἔστιν δὲ τὸ μὲν πράγματα λέγειν τοιόνδε τι, ὥσπερ εἴ τις λέγοι ὅτι δεῖ πρὸς βασιλέα παρασκευάζεσθαι καὶ μὴ ἐᾶν Αἴγυπτον χειρώσασθαι· καὶ γὰρ πρότερον Δαρεῖος οὐ πρότερον διέβη πρὶν Αἴγυπτον ἔλαβεν, λαβὼν δὲ διέβη, καὶ πάλιν Ξέρξης οὐ πρότερον ἐπεχείρησεν πρὶν ἔλαβεν, λαβὼν δὲ διέβη, ὥστε καὶ οὗτος ἐὰν λάβῃ, διαβήσεται, διὸ οὐκ ἐπιτρεπτέον. παραβολὴ δὲ τὰ Σωκρατικά, οἷον εἴ τις λέγοι ὅτι οὐ δεῖ κληρωτοὺς ἄρχειν· ὅμοιον γὰρ ὥσπερ ἂν εἴ τις τοὺς ἀθλητὰς κληροίη μὴ οἳ δύνανται ἀγωνίζεσθαι ἀλλ' οἳ ἂν λάχωσιν, ἢ τῶν πλωτήρων ὅντινα δεῖ κυβερνᾶν κληρώσειεν, ὡς δέον τὸν λαχόντα ἀλλὰ μὴ τὸν ἐπιστάμενον. λόγος δέ, οἷος ὁ Στησιχόρου περὶ Φαλάριδος καὶ $\langle$ὁ$\rangle$ Αἰσώπου ὑπὲρ τοῦ δημαγωγοῦ.

}
}
\end{greek}
