% Redaktura NZ
\section*{O autoru}

Povjesničar Tukidid \textgreek[variant=ancient]{(Θoυκυδίδης,} oko 460. – oko 396.\ pr.~Kr.) kao atenski vojskovođa sudjelovao je u prvom razdoblju Peloponeskoga rata (od 431.) sve do 424., kada je, zbog neuspjeha u obrani važnog savezničkog grada Amfipola, bio optužen za izdaju, te prognan.

Već po izbijanju rata počeo je raditi na \textit{Spisu o ratu Peloponežana i Atenjana} \textgreek[variant=ancient]{(Ξυγγραφὴ περὶ τoῦ πολέμoυ τῶν Пελoπoννησίων καὶ Ἀϑηναίων,} poznat i kao \textit{Povijest Peloponeskog rata}). Konačno ga je uobličio nakon atenskoga konačnog poraza. Djelo ima osam knjiga, ostalo je nedovršeno, a prikazuje događaje do 411.\ pr.~Kr.

\section*{O tekstu}

Na početku prve knjige \textit{Povijesti} Tukidid sažeto prikazuje stanje u Grčkoj od najranijeg doba. Odlomak tog pregleda donesen ovdje govori o Perzijskim ratovima (490.\ – 479.\ pr.~Kr.), tijekom kojih su Sparta i Atena postale najutjecajnije i vojnički najjače grčke države.

\newpage

\section*{Pročitajte naglas grčki tekst.}

Thuc. Historiae 1.18.1

\medskip

{\large
\begin{greek}
\noindent Μετὰ δὲ τὴν τῶν τυράννων κατάλυσιν ἐκ τῆς ῾Ελλάδος οὐ πολλοῖς ἔτεσιν ὕστερον καὶ ἡ ἐν Μαραθῶνι μάχη Μήδων πρὸς ᾿Αθηναίους ἐγένετο. δεκάτῳ δὲ ἔτει μετ' αὐτὴν αὖθις ὁ βάρβαρος τῷ μεγάλῳ στόλῳ ἐπὶ τὴν ῾Ελλάδα δουλωσόμενος ἦλθεν. καὶ μεγάλου κινδύνου ἐπικρεμασθέντος οἵ τε Λακεδαιμόνιοι τῶν ξυμπολεμησάντων ῾Ελλήνων ἡγήσαντο δυνάμει προύχοντες, καὶ οἱ ᾿Αθηναῖοι ἐπιόντων τῶν Μήδων διανοηθέντες ἐκλιπεῖν τὴν πόλιν καὶ ἀνασκευασάμενοι ἐς τὰς ναῦς ἐσβάντες ναυτικοὶ ἐγένοντο. κοινῇ τε ἀπωσάμενοι τὸν βάρβαρον, ὕστερον οὐ πολλῷ διεκρίθησαν πρός τε ᾿Αθηναίους καὶ Λακεδαιμονίους οἵ τε ἀποστάντες βασιλέως ῞Ελληνες καὶ οἱ ξυμπολεμήσαντες. δυνάμει γὰρ ταῦτα μέγιστα διεφάνη· ἴσχυον γὰρ οἱ μὲν κατὰ γῆν, οἱ δὲ ναυσίν. καὶ ὀλίγον μὲν χρόνον ξυνέμεινεν ἡ ὁμαιχμία, ἔπειτα διενεχθέντες οἱ Λακεδαιμόνιοι καὶ ᾿Αθηναῖοι ἐπολέμησαν μετὰ τῶν ξυμμάχων πρὸς ἀλλήλους.
\end{greek}

}

