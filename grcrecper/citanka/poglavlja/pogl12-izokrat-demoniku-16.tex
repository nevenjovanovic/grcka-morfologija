% redaktura NZ, unio NJ
%\section*{O autoru}



\section*{O tekstu}

Rasprava o praktičnoj etici \textit{Demoniku} \textgreek[variant=ancient]{(Πρὸς Δημόνικον)} ima oblik otvorenog pisma. Pripisuje se Izokratu (436.–338.\ p.~n.~e), no nije isključeno da joj je autor neki njegov učenik. Nudeći upute za svakodnevni život, u raspravi se razlikuju tri životna područja u kojima ta pravila treba primjenjivati: 1) čovjek u odnosu prema bogovima, 2) čovjek u odnosu prema drugim ljudima, ponajprije prema roditeljima i prijateljima, 3) čovjek u odnosu prema sebi samom i skladan razvoj njegova karaktera. Pravila se nižu bez strogog reda, što je tipično za ``gnomsku'' književnost V.–IV.~st. U tom smislu rasprava se može usporediti s pjesničkim opusom Megaranina Teognida (VI.~st.\ p.~n.~e).

%\newpage

\section*{Pročitajte naglas grčki tekst.}

%Naslov prema izdanju

Isoc.\ Ad Demonicum 16

\medskip

{\large
\begin{greek}
\noindent Μηδέποτε μηδὲν αἰσχρὸν ποιήσας ἔλπιζε λήσειν· καὶ γὰρ ἂν τοὺς ἄλλους λάθῃς, σεαυτῷ συνειδήσεις. Τοὺς μὲν θεοὺς φοβοῦ, τοὺς δὲ γονεῖς τίμα, τοὺς δὲ φίλους αἰσχύνου, τοῖς δὲ νόμοις πείθου. Τὰς ἡδονὰς θήρευε τὰς μετὰ δόξης· τέρψις γὰρ σὺν τῷ καλῷ μὲν ἄριστον, ἄνευ δὲ τούτου κάκιστον.  Εὐλαβοῦ τὰς διαβολὰς, κἂν ψευδεῖς ὦσιν· οἱ γὰρ πολλοὶ τὴν μὲν ἀλήθειαν ἀγνοοῦσιν, πρὸς δὲ τὴν δόξαν ἀποβλέπουσιν. Ἅπαντα δόκει ποιεῖν ὡς μηδένα λήσων· καὶ γὰρ ἂν παραυτίκα κρύψῃς, ὕστερον ὀφθήσει. Μάλιστα δ' ἂν εὐδοκιμοίης, εἰ φαίνοιο ταῦτα μὴ πράττων ἃ τοῖς ἄλλοις ἂν πράττουσιν ἐπιτιμῴης. Ἐὰν ᾖς φιλομαθὴς, ἔσει πολυμαθής. Ἃ μὲν ἐπίστασαι, ταῦτα διαφύλαττε ταῖς μελέταις, ἃ δὲ μὴ μεμάθηκας, προσλάμβανε ταῖς ἐπιστήμαις· ὁμοίως γὰρ αἰσχρὸν ἀκούσαντα χρήσιμον λόγον μὴ μαθεῖν καὶ διδόμενόν τι ἀγαθὸν παρὰ τῶν φίλων μὴ λαβεῖν.
 
\end{greek}

}

