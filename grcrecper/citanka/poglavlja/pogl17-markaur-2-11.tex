% Unesi korekture NČ 2019-08-16
\section*{O autoru}

Rimski car Marko Aurelije (Marcus Aurelius, \textgreek[variant=ancient]{Μάρκος Αὐρήλιος}) rođen je 121.\ po Kr.\ u Rimu. Vladao je od 161. do smrti 180. Učitelji su mu bili, između ostalih, Fronton i Herod Atički te stoici Junije Arulen Rustik i Apolonije iz Halkedona. Vladavinu Marka Aurelija obilježili su sukobi u Britaniji (162.), Germaniji i na dunavskom limesu (167. – 168.), Egiptu (172.) te višegodišnji rat s Partima (162. – 166.). Uspješno se suprotstavio prvom valu seobe naroda. Nasljednikom je proglasio sina Komoda te time prekinuo niz tzv.\ adoptivnih careva (Trajan – Hadrijan – Antonin Pio). 

Filozofski obrazovan, uz Seneku i Epikteta Marko Aurelije najpoznatiji je predstavnik kasne stoe.

Sačuvana su njegova pisma te filozofski spis \textit{Razgovori sa samim sobom} \textgreek[variant=ancient]{(Tὰ εἰς ἑαυτόν),} sastavljen na grčkom, kojim se car služio jednako kao i latinskim (dok je za njega latinski bio jezik ``poslovanja'', upravljanja i donošenja odluka, grčki je bio jezik intimnih razmišljanja).


\section*{O tekstu}

Djelo \textgreek[variant=ancient]{Tὰ εἰς ἑαυτόν} pisano je jednostavnim jezikom bliskim svakodnevnom govoru tog doba \textgreek[variant=ancient]{(κοıνή).} Autor mu vjerojatno nije sam dao naslov. U djelu nije sustavno prikazao cjelokupnost stoičke filozofije, već stoicizam kao način života, kao što su učinili i Seneka i Epiktet. Neke od tema kojima se bavi jesu božja providnost, kratkoća ljudskog života, tolerancija, zajedništvo svih ljudi, obaveza svih da rade za opće dobro, dužnost i moralna odgovornost prema drugima.

U odabranom odlomku Marko Aurelije razmišlja o smrti i argumentira zašto ona nije strašna, kao što ni sve ostalo što se događa ljudima – i ono što se smatra dobrim i ono što se smatra lošim – zapravo nije ni dobro ni loše.

\section*{Pročitajte naglas grčki tekst.}

M. Aur. Ad se ipsum 2.11

%Naslov prema izdanju

\medskip


{\large
{ 
\begin{greek}

\noindent  Ὡς ἤδη δυνατοῦ ὄντος ἐξιέναι τοῦ βίου, οὕτως ἕκαστα ποιεῖν καὶ λέγειν καὶ διανοεῖσθαι. τὸ δὲ ἐξ ἀνθρώπων ἀπελθεῖν, εἰ μὲν θεοὶ εἰσίν, οὐδὲν δεινόν· κακῷ γάρ σε οὐκ ἂν περιβάλοιεν· εἰ δὲ ἤτοι οὐκ εἰσὶν ἢ οὐ μέλει αὐτοῖς τῶν ἀνθρωπείων, τί μοι ζῆν ἐν κόσμῳ κενῷ θεῶν ἢ προνοίας κενῷ; ἀλλὰ καὶ εἰσὶ καὶ μέλει αὐτοῖς τῶν ἀνθρωπείων καὶ τοῖς μὲν κατ' ἀλήθειαν κακοῖς ἵνα μὴ περιπίπτῃ ὁ ἄνθρωπος, ἐπ' αὐτῷ τὸ πᾶν ἔθεντο· τῶν δὲ λοιπῶν εἴ τι κακὸν ἦν, καὶ τοῦτο ἂν προείδοντο, ἵνα ἐπὶ παντὶ ᾖ τὸ μὴ περιπίπτειν αὐτῷ. (ὃ δὲ χείρω μὴ ποιεῖ ἄνθρωπον, πῶς ἂν τοῦτο βίον ἀνθρώπου χείρω ποιήσειεν;) οὔτε δὲ κατ' ἄγνοιαν οὔτε εἰδυῖα μέν, μὴ δυναμένη δὲ προφυλάξασθαι ἢ διορθώσασθαι ταῦτα ἡ τῶν ὅλων φύσις παρεῖδεν ἄν, οὔτ' ἂν τηλικοῦτον ἥμαρτεν ἤτοι παρ' ἀδυναμίαν ἢ παρ' ἀτεχνίαν, ἵνα τὰ ἀγαθὰ καὶ τὰ κακὰ ἐπίσης τοῖς τε ἀγαθοῖς ἀνθρώποις καὶ τοῖς κακοῖς πεφυρμένως συμβαίνῃ. θάνατος δέ γε καὶ ζωή, δόξα καὶ ἀδοξία, πόνος καὶ ἡδονή, πλοῦτος καὶ πενία, πάντα ταῦτα ἐπίσης συμβαίνει ἀνθρώπων τοῖς τε ἀγαθοῖς καὶ τοῖς κακοῖς, οὔτε καλὰ ὄντα οὔτε αἰσχρά. οὔτ' ἄρ' ἀγαθὰ οὔτε κακά ἐστι.


\end{greek}


}
}


