% Unio ispravke NČ 2019-09-04
%\section*{O autoru}


\section*{O tekstu}

Druga knjiga Herodotove povijesti \textgreek[variant=ancient]{(Εὐτέρπη)} posvećena je Egiptu. U izabranom odlomku Herodot opisuje prehrambene navike Egipćana, napose kako pripremaju ribu i ptice, kao i zanimljivost o običaju da na gozbama bogatih oko stolova kruži drveni kipić mrtvaca u koji, dok piju i zabavljaju se, gledaju i pomišljaju na vlastitu smrtnost.

%\newpage

\section*{Pročitajte naglas grčki tekst.}

Hdt.\ Historiae 2.77.12

%Naslov prema izdanju

\medskip

\begin{greek}
{\large
{ \noindent Ἀρτοφαγέουσι δὲ ἐκ τῶν ὀλυρέων ποιεῦντες ἄρτους, τοὺς ἐκεῖνοι κυλλήστις ὀνομάζουσι. Οἴνῳ δὲ ἐκ κριθέων πεποιημένῳ διαχρέωνται· οὐ γάρ σφί εἰσι ἐν τῇ χώρῃ ἄμπελοι. Ἰχθύων δὲ τοὺς μὲν πρὸς ἥλιον αὐήναντες ὠμοὺς σιτέονται, τοὺς δὲ ἐξ ἅλμης τεταριχευμένους· ὀρνίθων δὲ τούς τε ὄρτυγας καὶ τὰς νήσσας καὶ τὰ σμικρὰ τῶν ὀρνιθίων ὠμὰ σιτέονται προταριχεύσαντες· τὰ δὲ ἄλλα ὅσα ἢ ὀρνίθων ἢ ἰχθύων σφί ἐστι ἐχόμενα, χωρὶς ἢ ὁκόσοι σφι ἱροὶ ἀποδεδέχαται, τοὺς λοιποὺς ὀπτοὺς καὶ ἑφθοὺς σιτέονται. Ἐν δὲ τῇσι συνουσίῃσι τοῖσι εὐδαίμοσι αὐτῶν, ἐπεὰν ἀπὸ δείπνου γένωνται, περιφέρει ἀνὴρ νεκρὸν ἐν σορῷ ξύλινον πεποιημένον, μεμιμημένον ἐς τὰ μάλιστα καὶ γραφῇ καὶ ἔργῳ, μέγαθος ὅσον τε πάντῃ πηχυαῖον ἢ δίπηχυν, δεικνὺς δὲ ἑκάστῳ τῶν συμποτέων λέγει· ``Ἐς τοῦτον ὁρέων πῖνέ τε καὶ τέρπεο· ἔσεαι γὰρ ἀποθανὼν τοιοῦτος.'' Ταῦτα μὲν παρὰ τὰ συμπόσια ποιεῦσι.

}
}
\end{greek}

