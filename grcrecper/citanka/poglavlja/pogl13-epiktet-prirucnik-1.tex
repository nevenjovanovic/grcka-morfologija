\section*{O autoru}

Filozof Epiktet \textgreek[variant=ancient]{(Ἐπίκτητος)} rođen je oko 55.\ po Kr.\ u frigijskom Hijerapolu \textgreek[variant=ancient]{(Ἱεράπολις,} danas Pamukkale). Nije nam poznato njegovo pravo ime – \textgreek[variant=ancient]{Ἐπίκτητος} je nadimak izveden od glagola \textgreek[variant=ancient]{ἐπικτάομαι} ``steći''. Mladost je proveo u Rimu kao rob Neronova oslobođenika i tajnika Epafrodita, koji mu je dopustio da pohađa filozofsku školu utjecajnog stoika Muzonija Rufa. Epiktet je s vremenom i sam počeo predavati filozofiju, a Epafrodit ga je oslobodio. Kad je car Domicijan protjerao filozofe iz Italije (93.-94), Epiktet je napustio Rim i otišao u Nikopol \textgreek[variant=ancient]{(Νικόπολις),} bogati grad u Epiru. Epiktetova škola privlačila je mnoge učenike, među kojima je navodno bio i car Hadrijan. Epiktet je umro oko 135. 

Epiktet je, uz Seneku i Marka Aurelija, glavni predstavnik stoičke škole kasnog, rimskog perioda. Poput Sokrata i učitelja Muzonija, Epiktet nije zapisivao svoje učenje; zapisao ga je njegov učenik Flavije Arijan. Sačuvane su četiri od osam knjiga \textit{Razgovora} \textgreek[variant=ancient]{(Διατριβαί)} te \textit{Priručnik} \textgreek[variant=ancient]{(Ἐγχειρίδιον),} ``praktični vodič za početnike u stoičkoj filozofiji''. 

\textgreek[variant=ancient]{Ἐγχειρίδιον} je pisan jednostavnim jezikom bliskim govornoj \textgreek[variant=ancient]{κοινή,} grčkome standardu helenističkoga i carskoga doba. Bio je vrlo popularno djelo. U VI.~st.\ nastao je Simplicijev komentar, a 1497.\ je humanist Angelo Poliziano preveo djelo na latinski.

\section*{O tekstu}

Na početku \textit{Priručnika} Epiktet poučava da sreća ovisi isključivo o čovjeku samom i njegovoj percepciji. Moramo shvatiti što možemo kontrolirati \textgreek[variant=ancient]{(τὰ ἐφ' ἡμῖν:} npr.\ um, stavovi), a što ne \textgreek[variant=ancient]{(τὰ οὐκ ἐφ' ἡμῖν:} npr.\ tijelo, imetak, tuđa mišljenja, društveni status). U onome što ne možemo kontrolirati ne treba tražiti sreću.

%\newpage

\section*{Pročitajte naglas grčki tekst.}

Epict.\ Enchiridion 1.1

%Naslov prema izdanju

\medskip

\begin{greek}
{\large
{ \noindent Τῶν ὄντων τὰ μέν ἐστιν ἐφ' ἡμῖν, τὰ δὲ οὐκ ἐφ' ἡμῖν. ἐφ' ἡμῖν μὲν ὑπόληψις, ὁρμή, ὄρεξις, ἔκκλισις καὶ ἑνὶ λόγῳ ὅσα ἡμέτερα ἔργα· οὐκ ἐφ' ἡμῖν δὲ τὸ σῶμα, ἡ κτῆσις, δόξαι, ἀρχαὶ καὶ ἑνὶ λόγῳ ὅσα οὐχ ἡμέτερα ἔργα. καὶ τὰ μὲν ἐφ' ἡμῖν ἐστι φύσει ἐλεύθερα, ἀκώλυτα, ἀπαραπόδιστα, τὰ δὲ οὐκ ἐφ' ἡμῖν ἀσθενῆ, δοῦλα, κωλυτά, ἀλλότρια. μέμνησο οὖν, ὅτι, ἐὰν τὰ φύσει δοῦλα ἐλεύθερα οἰηθῇς καὶ τὰ ἀλλότρια ἴδια, ἐμποδισθήσῃ, πενθήσεις, ταραχθήσῃ, μέμψῃ καὶ θεοὺς καὶ ἀνθρώπους, ἐὰν δὲ τὸ σὸν μόνον οἰηθῇς σὸν εἶναι, τὸ δὲ ἀλλότριον, ὥσπερ ἐστίν, ἀλλότριον, οὐδείς σε ἀναγκάσει οὐδέποτε, οὐδείς σε κωλύσει, οὐ μέμψῃ οὐδένα, οὐκ ἐγκαλέσεις τινί, ἄκων πράξεις οὐδὲ ἕν, οὐδείς σε βλάψει, ἐχθρὸν οὐχ ἕξεις, οὐδὲ γὰρ βλαβερόν τι πείσῃ. τηλικούτων οὖν ἐφιέμενος μέμνησο, ὅτι οὐ δεῖ μετρίως  κεκινημένον ἅπτεσθαι αὐτῶν, ἀλλὰ τὰ μὲν ἀφιέναι παντελῶς, τὰ δ' ὑπερτίθεσθαι πρὸς τὸ παρόν. ἐὰν δὲ καὶ ταῦτ' ἐθέλῃς καὶ ἄρχειν καὶ πλουτεῖν, τυχὸν μὲν οὐδ' αὐτῶν τούτων τεύξῃ διὰ τὸ καὶ τῶν προτέρων ἐφίεσθαι, πάντως γε μὴν ἐκείνων ἀποτεύξῃ, δι' ὧν μόνων ἐλευθερία καὶ εὐδαιμονία περιγίνεται. εὐθὺς οὖν πάσῃ φαντασίᾳ τραχείᾳ μελέτα ἐπιλέγειν ὅτι ‘φαντασία εἶ καὶ οὐ πάντως τὸ φαινόμενον’. ἔπειτα ἐξέταζε αὐτὴν καὶ δοκίμαζε τοῖς κανόσι τούτοις οἷς ἔχεις, πρώτῳ δὲ τούτῳ καὶ μάλιστα, πότερον περὶ τὰ ἐφ' ἡμῖν ἐστιν ἢ περὶ τὰ οὐκ ἐφ' ἡμῖν· κἂν περί τι τῶν οὐκ ἐφ' ἡμῖν ᾖ, πρόχειρον ἔστω τὸ διότι ‘οὐδὲν πρὸς ἐμέ’.

}
}
\end{greek}

