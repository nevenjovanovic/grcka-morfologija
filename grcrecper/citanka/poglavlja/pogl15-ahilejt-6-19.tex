% Unesi korekture NČ, 2019-08-18
\section*{O autoru}

Ahilej Tatije ili Tacije \textgreek[variant=ancient]{(Ἀχιλλεὺς Τάτıος)} autor je ljubavnog romana \textit{Zgode Leukipe i Klitofonta} \textgreek[variant=ancient]{(Tὰ κατὰ Λευκίππην καὶ Κλειτοφῶντα).} Živio je u 2.~st.\ po~Kr., vjerojatno u Aleksandriji. Navodi da se pod stare dane obratio na kršćanstvo i postao biskup najvjerojatnije nisu točni. Sudeći po broju sačuvanih papirusa, \textit{Zgode Leukipe i Klitofonta} bile su najpopularniji grčki ljubavni roman. Njegovi su glavni junaci dvoje mladih koji se ludo zaljube, ali ih okrutna sudbina razdvaja nizom nesretnih događaja poput brodoloma, otmice od strane gusara koji prodaju junake u roblje itd.

\section*{O tekstu}

Jedna od karakteristika grčkih ljubavnih romana jest izražen interes za osjećaje likova. Kad oteta Leukipa odbije udvaranje nasilnog Tersandra jer voli Klitofonta, Tersandar osjeća ljubav i bijes istovremeno. To je autoru povod za digresiju, kratko filozofsko razmatranje tih emocija i njihovog međusobnog odnosa. Ovakve digresije česte su u Tacijevu romanu. 

Odabrani se odlomak nalazi između Tersandrovog agresivnog pokušaja zavođenja i svađe koja uslijedi, a u kojoj ni Leukipa ni Tersandar ne biraju riječi.

\newpage

\section*{Pročitajte naglas grčki tekst.}

Ach. Tat. Leucippe et Clitophon 6.19.1

%Naslov prema izdanju

\medskip


{\large
{ 
\begin{greek}

\noindent Θυμὸς δὲ καὶ ἔρως δύο λαμπάδες· ἔχει γὰρ καὶ ὁ θυμὸς ἄλλο πῦρ, καὶ ἔστι τὴν μὲν φύσιν ἐναντιώτατον, τὴν δὲ βίαν ὅμοιον. ὁ μὲν γὰρ παροξύνει μισεῖν, ὁ δὲ ἀναγκάζει φιλεῖν· καὶ ἀλλήλων πάροικος ἡ τοῦ πυρός ἐστι πηγή. ὁ μὲν γὰρ εἰς τὸ ἧπαρ κάθηται, ὁ δὲ τῇ καρδίᾳ περιβέβληται.  ὅταν οὖν ἄμφω τὸν ἄνθρωπον καταλάβωσι, γίνεται μὲν αὐτοῖς ἡ ψυχὴ τρυτάνη, τὸ δὲ πῦρ ἑκατέρου ταλαντεύεται. μάχονται δὲ ἄμφω περὶ τῆς ῥοπῆς· καὶ τὰ πολλὰ μὲν ὁ ἔρως εἴωθε νικᾶν, ὅταν εἰς τὴν ἐπιθυμίαν εὐτυχῇ. ἢν δὲ αὐτὸν ἀτιμάσῃ τὸ ἐρώμενον, αὐτὸς τὸν θυμὸν εἰς συμμαχίαν καλεῖ.  κἀκεῖνος ὡς γείτων πείθεται, καὶ ἀνάπτουσιν ἄμφω τὸ πῦρ. ἂν δὲ ἅπαξ ὁ θυμὸς τὸν ἔρωτα παρ' αὑτῷ λάβῃ καὶ τῆς οἰκείας ἕδρας ἐκπεσόντα κατάσχῃ, φύσει γε ὢν ἄσπονδος, οὐχ ὡς φίλῳ πρὸς τὴν ἐπιθυμίαν συμμαχεῖ, ἀλλ' ὡς δοῦλον τῆς ἐπιθυμίας πεδήσας κρατεῖ· οὐκ ἐπιτρέπει δὲ αὐτῷ σπείσασθαι πρὸς τὸ ἐρώμενον, κἂν θέλῃ.  ὁ δὲ τῷ θυμῷ βεβαπτισμένος καταδύεται, καὶ εἰς τὴν ἰδίαν ἀρχὴν ἐκπηδῆσαι θέλων οὐκέτι ἐστὶν ἐλεύθερος, ἀλλὰ μισεῖν ἀναγκάζεται τὸ φιλούμενον. ὅταν δὲ ὁ θυμὸς καχλάζων γεμισθῇ καὶ τῆς ἐξουσίας ἐμφορηθεὶς ἀποβλύσῃ, κάμνει μὲν ἐκ τοῦ κόρου, καμὼν δὲ παρίεται, καὶ ὁ ἔρως ἀμύνεται καὶ ὁπλίζει τὴν ἐπιθυμίαν καὶ τὸν θυμὸν ἤδη καθεύδοντα νικᾷ. ὁρῶν δὲ τὰς ὕβρεις, ἃς κατὰ τῶν φιλτάτων ἐπαρῴνησεν, ἀλγεῖ καὶ πρὸς τὸ ἐρώμενον ἀπολογεῖται καὶ εἰς ὁμιλίαν παρακαλεῖ καὶ τὸν θυμὸν ἐπαγγέλλεται καταμαλάττειν ἡδονῇ.  τυχὼν μὲν οὖν ὧν ἠθέλησεν, ἵλεως γίνεται, ἀτιμούμενος δὲ πάλιν εἰς τὸν θυμὸν καταδύεται· ὁ δὲ καθεύδων ἐξεγείρεται καὶ τὰ ἀρχαῖα ποιεῖ· ἀτιμίᾳ γὰρ ἔρωτος σύμμαχός ἐστι θυμός.

\end{greek}
}
}

