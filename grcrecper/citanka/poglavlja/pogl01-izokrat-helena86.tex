%Unesi korekture NZ, NJ 2019-08-09
%\section*{O autoru}



\section*{O tekstu}

Izokrat u \textit{Pohvali} nastoji umanjiti Heleninu krivicu za Trojanski rat. Uspoređujući je s Tezejem, on ističe njegovo junaštvo i prvenstvo pred Heraklom, čime želi ukazati da je i žena koja je osvojila srce takvoga junaka također vrijedna pohvale. U ovom se odlomku opisuje kako je Tezej Atenu učinio najvećim helenskim gradom i kako je građane potaknuo da se takmiče u vrlinama, nadajući se ipak da će on sam još uvijek svojim kvalitetama biti iznad njih, dok će mu, s druge strane, biti ljepše biti čašćenim od produhovljenih građana nego od građana ropskoga duha. Pritom je toliko pazio da ne učini ništa protiv volje građana da im je na koncu  omogućio i da preuzmu vlast u državi, a oni su ipak, umjesto demokracije, izabrali da nad njima vlada Tezej.

%\newpage

\section*{Pročitajte naglas grčki tekst.}

Isoc.\ Helenae encomium 35

%Naslov prema izdanju

\medskip

\begin{greek}
{\large

\noindent καὶ πρῶτον μὲν τὴν πόλιν σποράδην καὶ κατὰ κώμας οἰκοῦσαν εἰς ταὐτὸν συναγαγὼν τηλικαύτην ἐποίησεν ὥστ' ἔτι καὶ νῦν ἀπ' ἐκείνου τοῦ χρόνου μεγίστην τῶν ῾Ελληνίδων εἶναι· μετὰ δὲ ταῦτα, κοινὴν τὴν πατρίδα καταστήσας καὶ τὰς ψυχὰς τῶν συμπολιτευομένων ἐλευθερώσας, ἐξ ἴσου τὴν ἅμιλλαν αὐτοῖς περὶ τῆς ἀρετῆς ἐποίησεν, πιστεύων μὲν ὁμοίως αὐτῶν προέξειν ἀσκούντων ὥσπερ ἀμελούντων, εἰδὼς δὲ τὰς τιμὰς ἡδίους οὔσας τὰς παρὰ τῶν μέγα φρονούντων ἢ τὰς παρὰ τῶν δουλευόντων. Τοσούτου δ' ἐδέησεν ἀκόντων τι ποιεῖν τῶν πολιτῶν ὥσθ' ὁ μὲν τὸν δῆμον καθίστη κύριον τῆς πολιτείας, οἱ δὲ μόνον αὐτὸν ἄρχειν ἠξίουν, ἡγούμενοι πιστοτέραν καὶ κοινοτέραν εἶναι τὴν ἐκείνου μοναρχίαν τῆς αὑτῶν δημοκρατίας.


}
\end{greek}


%kraj
