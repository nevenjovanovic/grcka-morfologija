% 1. redaktura NJ, 2019-04-20; unesi NZ 2019-08-07
%\section*{O autoru}



\section*{O tekstu}

U drugoj knjizi \textit{Retorike} (djelo je nastajalo tijekom Aristotelovih dvaju boravaka u Ateni, 367. – 347.\ pr.~Kr, kad ga je Platon primio u Akademiju, i 335. – 322. pr.~Kr, kad je Aristotel vodio vlastitu školu, Licej) Aristotel izlaže načine kojima govornik može uvjeriti slušaoce, i preduvjete za takvo uvjeravanje. Pošto je prikazao emocije, i različitost emocija ovisno o dobi i društvenom statusu slušalaca, u 20.\ poglavlju govori o takozvanim τόποι, \textit{općim mjestima} kojima se dokazuje da se nešto može ili ne može dogoditi. U takve općenite dokazima, uz logičke, ulaze i primjeri \textgreek[variant=ancient]{(παραδείγματα);} njihova su posebna podvrsta pripovijesti \textgreek[variant=ancient]{(λόγοι),} poput prispodoba i basni. 

Ovaj tekst donosi Aristotelov primjer za primjenu basne u uvjeravanju; primjer je i sam priča, o Stezihoru \textgreek[variant=ancient]{(Στησίχορος),} grčkom korskom pjesniku arhajskog razdoblja (oko 640.\ – oko 555.\ pr.~Kr), koji je živio u Himeri na Siciliji, i o Falaridu \textgreek[variant=ancient]{(Φάλαρις),} tiraninu sicilskoga Akraganta između 570.\ i 555.\ pr.~Kr, vladaru čuvenom po okrutnosti.

\newpage

\section*{Pročitajte naglas grčki tekst.}

Arist.\ Rhetorica 1393b 10
%Naslov prema izdanju

\medskip

{\large
\begin{greek}
\noindent Στησίχορος μὲν γὰρ ἑλομένων στρατηγὸν αὐτοκράτορα τῶν ῾Ιμεραίων Φάλαριν καὶ μελλόντων φυλακὴν διδόναι τοῦ σώματος, τἆλλα διαλεχθεὶς εἶπεν αὐτοῖς λόγον ὡς ἵππος κατεῖχε λειμῶνα μόνος, ἐλθόντος δ' ἐλάφου καὶ διαφθείροντος τὴν νομὴν βουλόμενος τιμωρήσασθαι τὸν ἔλαφον ἠρώτα τινὰ ἄνθρωπον εἰ δύναιτ' ἂν μετ' αὐτοῦ τιμωρήσασθαι τὸν ἔλαφον, ὁ δ' ἔφησεν, ἐὰν λάβῃ χαλινὸν καὶ αὐτὸς ἀναβῇ ἐπ' αὐτὸν ἔχων ἀκόντια· συνομολογήσας δὲ καὶ ἀναβάντος ἀντὶ τοῦ τιμωρήσασθαι αὐτὸς ἐδούλευσε τῷ ἀνθρώπῳ. ``οὕτω δὲ καὶ ὑμεῖς'', ἔφη, ``ὁρᾶτε μὴ βουλόμενοι τοὺς πολεμίους τιμωρήσασθαι τὸ αὐτὸ πάθητε τῷ ἵππῳ· τὸν μὲν γὰρ χαλινὸν ἔχετε ἤδη, ἑλόμενοι στρατηγὸν αὐτοκράτορα· ἐὰν δὲ φυλακὴν δῶτε καὶ ἀναβῆναι ἐάσητε, δουλεύσετε ἤδη Φαλάριδι''.

\end{greek}
}

