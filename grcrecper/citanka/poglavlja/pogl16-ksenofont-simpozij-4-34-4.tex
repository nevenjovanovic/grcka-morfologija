%Unio korekture NZ, 2019-08-10
\section*{O autoru}

Ksenofont \textgreek[variant=ancient]{(Ξενοφῶν}; Atena, oko 428. – Korint, oko 354. pr.~Kr.) bio je učenik Sokratov, podrijetlom iz aristokratske obitelji naklonjene Sparti. Zbog političke orijentacije veći dio života proveo je izvan domovine. 

U antici su ga držali poglavito filozofom, zbog spisa u kojima središnje mjesto pripada njegovu učitelju: \textgreek[variant=ancient]{Ἀπολογία Σωκράτους} \textit{(Obrana Sokratova),} \textgreek[variant=ancient]{Ἀπομνημονεύματα Σωκράτους} \textit{(Uspomene na Sokrata),} \textgreek[variant=ancient]{Συμπόσιον} \textit{(Gozba).}  Gospodarenjem u obitelji i državi bave se \textgreek[variant=ancient]{Οἰκονομικός} \textit{(Rasprava o gospodarstvu)}  i spis \textgreek[variant=ancient]{Περὶ πόρων} \textit{(O prihodima).}  Didaktičnošću se odlikuju \textgreek[variant=ancient]{Περὶ ἱππικῆς} \textit{(O jahačkoj vještini)}  te \textgreek[variant=ancient]{Κυνηγετικός} \textit{(Rasprava o lovu)} i \textgreek[variant=ancient]{Ἱππαρχικός} \textit{(Ras\-pra\-va o konjičkom zapovjedniku)}.

Prema modernom razumijevanju značajnija je Ksenofontova historiografska ostavština, u kojoj se ističu \textgreek[variant=ancient]{Ἑλληνικά} \textit{(Grčka povijest)}, s prikazom razdoblja od 411. do 362. pr.~Kr., zamišljena kao nastavak Tukididova djela, i \textgreek[variant=ancient]{Κύρου ἀνάβασις} \textit{(Kirov pohod} ili \textit{A\-na\-baza),} opis neuspjeloga pohoda Kira Mlađega protiv brata Artakserksa II.; u pohodu je značajnu ulogu imao i sam pisac. U biografiji \textgreek[variant=ancient]{Κύρου παιδεία} \textit{(Kirov odgoj)} na primjeru Kira Starijega obrazlaže vlastite političke poglede; politički su intonirani i dijalog \textgreek[variant=ancient]{Ἱέρων} \textit{(Hijeron),}  pohvalni posmrtni govor \textgreek[variant=ancient]{Ἀγησίλαος} \textit{(Agezilaj)}, te \textgreek[variant=ancient]{Λακεδαιμονίων πολιτεία} \textit{(Lakedemonski ustav)}.


Ksenofont je u antici smatran piscem uzorna jezika i stila (Diogen Laertije naziva ga Atičkom Muzom); upravo zahvaljujući takvoj procjeni jedan je od rijetkih klasičnih autora čiji se opus u cijelosti sačuvao.

\section*{O tekstu}

\textit{Simpozij}, napisan oko 365., opisuje fiktivnu gozbu na kojoj sudjeluje Sokrat; po tome sliči poznatijem istoimenom Platonovu djelu. Kod Ksenofonta, gozba se održava 422.\ pr.~Kr.\ u kući atenskoga bogataša Kalije; gosti vode razbibrižnu raspravu o tome tko se čime ponosi. No, nakon što se pokrene rasprava o ljubavi, Sokrat o tome iznosi ozbiljan i promišljen govor, posebice o homoseksualnoj ljubavi, razdvajajući pritom etički nevrijednu tjelesnu od duhovne, uzvišene i plemenite. 

U izabranom odlomku govori se o tomu da mnogi ne mogu utažiti svoju glad za bogatstvom, jer ono se ne nalazi u novcu nego u duši.


\section*{Pročitajte naglas grčki tekst.}

Xen.\ Symposium 4.34.4

%Naslov prema izdanju

\medskip

{\large
\begin{greek}
\noindent Νομίζω, ὦ ἄνδρες, τοὺς ἀνθρώπους οὐκ ἐν τῷ οἴκῳ τὸν πλοῦτον καὶ τὴν πενίαν ἔχειν ἀλλ' ἐν ταῖς ψυχαῖς. ὁρῶ γὰρ πολλοὺς μὲν ἰδιώτας, οἳ πάνυ πολλὰ ἔχοντες χρήματα οὕτω πένεσθαι ἡγοῦνται ὥστε πάντα μὲν πόνον, πάντα δὲ κίνδυνον ὑποδύονται, ἐφ' ᾧ πλείω κτήσονται, οἶδα δὲ καὶ ἀδελφούς, οἳ τὰ ἴσα λαχόντες ὁ μὲν αὐτῶν τἀρκοῦντα ἔχει καὶ περιττεύοντα τῆς δαπάνης, ὁ δὲ τοῦ παντὸς ἐνδεῖται· αἰσθάνομαι δὲ καὶ τυράννους τινάς, οἳ οὕτω πεινῶσι χρημάτων ὥστε ποιοῦσι πολὺ δεινότερα τῶν ἀπορωτάτων· δι' ἔνδειαν μὲν γὰρ δήπου οἱ μὲν κλέπτουσιν, οἱ δὲ τοιχωρυχοῦσιν, οἱ δὲ ἀνδραποδίζονται· τύραννοι δ' εἰσί τινες οἳ ὅλους μὲν οἴκους ἀναιροῦσιν, ἁθρόους δ' ἀποκτείνουσι, πολλάκις δὲ καὶ ὅλας πόλεις χρημάτων ἕνεκα ἐξανδραποδίζονται. τούτους μὲν οὖν ἔγωγε καὶ πάνυ οἰκτίρω τῆς ἄγαν χαλεπῆς νόσου. ὅμοια γάρ μοι δοκοῦσι πάσχειν ὥσπερ εἴ τις πολλὰ ἔχοι καὶ πολλὰ ἐσθίων μηδέποτε ἐμπίμπλαιτο.

\end{greek}

}

