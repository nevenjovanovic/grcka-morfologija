% Unesene korekture NČ 2019-08-12 (osim općenitih)
\section*{O tekstu}

U prvom odlomku treće knjige Razgovora sa samim sobom \textgreek[variant=ancient]{(Tὰ εἰς ἑαυτόν)} Marko Aurelije razmatra slabljenje intelektualnih sposobnosti pod stare dane.

\section*{Pročitajte naglas grčki tekst.}

M.~Aur.\ Ad se ipsum 3.1

%Naslov prema izdanju

\medskip


{\large

\begin{greek}

\noindent Οὐχὶ τοῦτο μόνον δεῖ λογίζεσθαι, ὅτι καθ' ἑκάστην ἡμέραν ἀπαναλίσκεται ὁ βίος καὶ μέρος ἔλαττον αὐτοῦ καταλείπεται, ἀλλὰ κἀκεῖνο λογιστέον, ὅτι, εἰ ἐπὶ πλέον βιῴη τις, ἐκεῖνό γε ἄδηλον, εἰ ἐξαρκέσει ὁμοία αὖθις ἡ διάνοια πρὸς τὴν σύνεσιν τῶν πραγμάτων καὶ τῆς θεωρίας τῆς συντεινούσης εἰς τὴν ἐμπειρίαν τῶν τε θείων καὶ τῶν ἀνθρωπείων. ἐὰν γὰρ παραληρεῖν ἄρξηται, τὸ μὲν διαπνεῖσθαι καὶ τρέφεσθαι καὶ φαντάζεσθαι καὶ ὁρμᾶν καὶ ὅσα ἄλλα τοιαῦτα, οὐκ ἐνδεήσει· τὸ δὲ ἑαυτῷ χρῆσθαι καὶ τοὺς τοῦ καθήκοντος ἀριθμοὺς ἀκριβοῦν καὶ τὰ προφαινόμενα διαρθροῦν καὶ περὶ αὐτοῦ τοῦ εἰ ἤδη ἐξακτέον αὑτὸν ἐφιστάνειν καὶ ὅσα τοιαῦτα λογισμοῦ συγγεγυμνασμένου πάνυ χρῄζει, προαποσβέννυται. χρὴ οὖν ἐπείγεσθαι οὐ μόνον τῷ ἐγγυτέρω τοῦ θανάτου ἑκάστοτε γίνεσθαι, ἀλλὰ καὶ διὰ τὸ τὴν ἐννόησιν τῶν πραγμάτων καὶ τὴν παρακολούθησιν προαπολήγειν.


\end{greek}


}

