% Redaktura NZ, unio NJ
\section*{O autoru}

Flavije Arijan \textgreek[variant=ancient]{(Φλάουιος Ἀρριανός,} Flavius Arrianus; Nikomedija u Bitiniji, oko 95. – Atena, oko 175.) bio je grčki povjesničar i filozof carskoga doba, učenik filozofa Epikteta (oko 55.–135). Za Hadrijana, oko 128.–129, Arijan je bio rimski konzul, a 131.–138. namjesnik (legatus Augusti pro praetore) u Kapadociji. Kao pedesetogodišnjak povukao se iz javnog života, boravio u Ateni (ἄρχων ἐπώνυμος 145./146). Književni mu je model bio Ksenofont (oko 430.–354. pr.~Kr); Ksenofontovo je ime Arijan nosio i kao nadimak.

Glavno je Arijanovo historiografsko djelo \textit{Aleksandrov pohod} \textgreek[variant=ancient]{(Ἀνάβασις Ἀλεξάνδρου)} ili \textit{Anabaza}; u sedam knjiga faktografski precizno i kritično iznio je povijest vladavine Aleksandra III. Velikog. Autor je i povijesno-geografskog \textit{Opisa Indije} \textgreek[variant=ancient]{(Ἰνδικὴ συγγραφή),} sastavljenog prema Aleksandrovim povjesničarima Megastenu i Nearhu; u \textit{Opisu} piše jonskim grčkim poput Hekateja i Herodota. Priredio je Epiktetove \textit{Rasprave} \textgreek[variant=ancient]{(Διατριβαί)} u osam knjiga (sačuvane samo prve četiri) – taj je filozofski angažman usporediv s Ksenofontovim \textgreek[variant=ancient]{Ἀπομνημονεύματα Σωκράτους} – te \textit{Priručnik} \textgreek[variant=ancient]{(Ἐγχειρίδιον)} za Epiktetovu etiku.

\section*{O tekstu}

U sljedećem odlomku Arijan iznosi dvije verzije legende o Aleksandru i gordijskom čvoru; drugu, manje uobičajenu, navodi prema povjesničaru Aristobulu (umro nakon 301. pr.~Kr), koji pripovijeda da čvor nije bio presječen, već razriješen, pošto je izvučen klin \textgreek[variant=ancient]{(ἕστωρ)} koji ga je držao. Riječ \textgreek[variant=ancient]{ἕστωρ} je homerska i u kasnijem grčkom neuobičajena; njome se koristi \textit{Ilijada} 24, 272, opisujući kako se klinom učvršćuje čvor. Po svemu sudeći, Gordijeva su kola bila vezana na homerski način, što je Aleksandar shvatio i prepoznao (obzirom da je Homera, kako je posvjedočeno, čitao pod Aristotelovim vodstvom). 

Stil teksta obilježen je za Arijanovo vrijeme ``starinskom'' atičkom jednostavnošću, i po tome je oprečan ``modernom'' grčkom Arijanova starijeg suvremenika Plutarha (oko 46. – oko 120).

%\newpage

\section*{Pročitajte naglas grčki tekst.}

%Naslov prema izdanju

Arr.\ Anabasis 2.3.6

\medskip

{\large
\begin{greek}
\noindent Πρὸς δὲ δὴ τούτοις καὶ τόδε περὶ τῆς ἁμάξης ἐμυθεύετο, ὅστις λύσειε τοῦ ζυγοῦ τῆς ἁμάξης τὸν δεσμόν, τοῦτον χρῆναι ἄρξαι τῆς Ἀσίας. ἦν δὲ ὁ δεσμὸς ἐκ φλοιοῦ κρανίας καὶ τούτου οὔτε τέλος οὔτε ἀρχὴ ἐφαίνετο. Ἀλέξανδρος δὲ ὡς ἀπόρως μὲν εἶχεν ἐξευρεῖν λύσιν τοῦ δεσμοῦ, ἄλυτον δὲ περιιδεῖν οὐκ ἤθελε, μή τινα καὶ τοῦτο ἐς τοὺς πολλοὺς κίνησιν ἐργάσηται, οἱ μὲν λέγουσιν, ὅτι παίσας τῷ ξίφει διέκοψε τὸν δεσμὸν καὶ λελύσθαι ἔφη· Ἀριστόβουλος δὲ λέγει ἐξελόντα τὸν ἕστορα τοῦ ῥυμοῦ, ὃς ἦν τύλος διαβεβλημένος διὰ τοῦ ῥυμοῦ διαμπάξ, ξυνέχων τὸν δεσμόν, ἐξελκύσαι ἔξω τοῦ ῥυμοῦ τὸ$\langle$ν$\rangle$ ζυγόν. ὅπως μὲν δὴ ἐπράχθη τὰ ἀμφὶ τῷ δεσμῷ τούτῳ Ἀλεξάνδρῳ οὐκ ἔχω ἰσχυρίσασθαι. ἀπηλλάγη  δ' οὖν ἀπὸ τῆς ἁμάξης αὐτός τε καὶ οἱ ἀμφ' αὐτὸν ὡς τοῦ λογίου τοῦ ἐπὶ τῇ λύσει τοῦ δεσμοῦ ξυμβεβηκότος. καὶ γὰρ καὶ τῆς νυκτὸς ἐκείνης βρονταί τε καὶ σέλας ἐξ οὐρανοῦ ἐπεσήμηναν· καὶ ἐπὶ τούτοις ἔθυε τῇ ὑστεραίᾳ Ἀλέξανδρος τοῖς φήνασι θεοῖς τά τε σημεῖα καὶ τοῦ δεσμοῦ τὴν λύσιν.
\end{greek}

}


