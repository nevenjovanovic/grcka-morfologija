% Unesi korekture NČ i NZ, 2019-09-24

\section*{O tekstu}

Spis pod naslovom Κυνικός \textit{(Cinik)} jedan je od tekstova kojem neki filolozi osporavaju Lukijanovo autorstvo, smatraju ga pseudolukijanskim. Djelo ima oblik dijaloga; glavna su lica Likin i neimenovani kinički filozof. Autorove stavove izlaže Likin. U ovom ekscerptu kinički filozof slikovito opisuje odnos ljudi i njihovih strasti.

%\newpage

\section*{Pročitajte naglas grčki tekst.}
Luc.\ Cynicus 18.8

%Naslov prema izdanju

\medskip

{\large
\begin{greek}
\noindent Πάσχετε δὲ παραπλήσιόν τι ὅ φασι παθεῖν τινα ἐφ' ἵππον ἀναβάντα μαινόμενον· ἁρπάσας γὰρ αὐτὸν ἔφερεν ἄρα ὁ ἵππος· ὁ δὲ οὐκέτι καταβῆναι τοῦ ἵππου θέοντος ἐδύνατο. καί τις ἀπαντήσας ἠρώτησεν αὐτὸν ποίαν ἄπεισιν; ὁ δὲ εἶπεν, Ὅπου ἂν τούτῳ δοκῇ, δεικνὺς τὸν ἵππον. καὶ ὑμᾶς ἄν τις ἐρωτᾷ, ποῖ φέρεσθε; τἀληθὲς ἐθέλοντες λέγειν ἐρεῖτε ἁπλῶς μέν, ὅπουπερ ἂν ταῖς ἐπιθυμίαις δοκῇ, κατὰ μέρος δέ, ὅπουπερ ἂν τῇ ἡδονῇ δοκῇ, ποτὲ δέ, ὅπου τῇ δόξῃ, ποτὲ δὲ αὖ, τῇ φιλοκερδίᾳ· ποτὲ δὲ ὁ θυμός, ποτὲ δὲ ὁ φόβος, ποτὲ δὲ ἄλλο τι τοιοῦτον ὑμᾶς ἐκφέρειν φαίνεται· οὐ γὰρ  ἐφ' ἑνός, ἀλλ' ἐπὶ πολλῶν ὑμεῖς γε ἵππων βεβηκότες ἄλλοτε ἄλλων, καὶ μαινομένων πάντων, φέρεσθε. τοιγαροῦν ἐκφέρουσιν ὑμᾶς εἰς βάραθρα καὶ κρημνούς. ἴστε δ' οὐδαμῶς πρὶν πεσεῖν ὅτι πεσεῖσθαι μέλλετε.

\end{greek}

}

