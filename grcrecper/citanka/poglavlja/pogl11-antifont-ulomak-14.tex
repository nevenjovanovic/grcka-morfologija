% Unio ispravke NČ 2019-09
\section*{O autoru}

Sofist Antifont \textgreek[variant=ancient]{(Ἀντιφῶν,} lat. Antiphon) iz Atene živio je u V.~st.\ pr.~Kr. Učenjaci se razilaze oko toga je li identičan s poznatim govornikom Antifontom iz atičke općine Ramnunta. Pripisuju mu se djela \textgreek[variant=ancient]{Περὶ ὁμονοίας} (\textit{O slozi}) i \textgreek[variant=ancient]{Περὶ ἀληθείας} (\textit{O istini}). Od drugospomenutog su djela sačuvani papirusni fragmenti (DK 44) koji kritiziraju konvencionalni moral s gledišta vlastitog interesa.

\section*{O tekstu}

Poruka je ovog ulomka, sačuvanog u Stobejevoj zbirci \textit{Florilegij} \textgreek[variant=ancient]{(Ἰωάννης ὁ Στοβαῖος,  Ἀνθολόγιον,} V.~st.\ po.~Kr; Stobaei Flor., XVI, 29) da se uspjeh ne sastoji u posjedovanju materijalnog bogatstva. Naime, oni koji gomilaju svoj novac, a da ga pritom ne koriste, nemaju veću korist od njega nego što bi imali od nekog kamena. Meta Antifontove kritike mogli bi biti ljudi ograničenog poimanja bogatstva, koji nisu sposobni upotrijebiti svoj um da bi shvatili što je pravo bogatstvo.

%\newpage

\section*{Pročitajte naglas grčki tekst.}

Antiph. Fr.14 (B54 DK)

%Naslov prema izdanju

\medskip

\begin{greek}
{\large
{ \noindent Ἰδὼν ἀνὴρ ἄνδρα ἕτερον ἀργύριον ἀναιρούμενον πολὺ ἐδεῖτό οἱ δανεῖσαι ἐπὶ τόκῳ, ὁ δ' οὐκ ἠθέλησεν, ἀλλ' ἦν οἷος ἀπιστεῖν τε καὶ μὴ ὠφελεῖν μηδένα, φέρων δ' ἀπέθετο ὅποι δή· καί τις καταμαθὼν τοῦτο ποιοῦντα ὑφείλετο, ὑστέρῳ δὲ χρόνῳ ἐλθὼν οὐχ ηὕρισκε τὰ χρήματα ὁ καταθέμενος. Περιαλγῶν οὖν τῇ συμφορᾷ τά τε ἄλλα καὶ ὅτι οὐκ ἔχρησε τῷ δεομένῳ, ὃ ἂν αὐτῷ καὶ σῶον ἦν καὶ ἕτερον προσέφερεν, ἀπαντήσας δὴ τῷ ἀνδρὶ τῷ τότε δανειζομένῳ ἀπωλοφύρετο τὴν συμφοράν, ὅτι ἐξήμαρτε καὶ ὅτι οἱ μεταμέλει οὐ χαρισαμένῳ, ἀλλ' ἀχαριστήσαντι, ὡς πάντως οἱ ἀπολόμενον τὸ ἀργύριον. ῾Ο δ' αὐτὸν ἐκέλευε μὴ φροντίζειν, ἀλλὰ νομίζειν αὑτῷ εἶναι καὶ μὴ ἀπολωλέναι, καταθέμενον λίθον εἰς τὸ αὐτὸ χωρίον· πάντως γὰρ οὐδ' ὅτε ἦν σοι ἐχρῶ αὐτῷ, ὅθεν μηδὲ νῦν νόμιζε στέρεσθαι μηδενός. ῞Οτῳ γάρ τις μὴ ἐχρήσατο μηδὲ χρήσεται, ὄντος ἢ μὴ ὄντος αὐτῷ οὐδὲν οὔτε πλέον οὔτε ἔλασσον βλάπτεται. ῞Οταν γὰρ ὁ θεὸς μὴ παντελῶς βούληται ἀγαθὰ διδόναι ἀνδρὶ, χρημάτων μὲν πλοῦτον παρασχών, τοῦ δὲ φρονεῖν καλῶς πένητα ποιήσας, τὸ ἕτερον ἀφελόμενος ἑκατέρων ἀπεστέρησεν.

}
}
\end{greek}

