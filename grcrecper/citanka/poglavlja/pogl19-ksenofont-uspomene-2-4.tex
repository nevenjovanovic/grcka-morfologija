%\section*{O autoru}



\section*{O tekstu}

\textit{Uspomene na Sokrata} \textgreek[variant=ancient]{(Ἀπομνημονεύματα Σωκράτους)} sastavljene su u četiri knjige i zapravo predstavljaju skup razgovora i epizoda sa Sokratom u glavnoj ulozi. U drugoj knjizi Sokrat, između ostalog, govori i o prijateljstvu. U ovdje odabranom odjeljku Sokrat, podsjetivši na opće mišljenje da je dobar prijatelj najbolja stečevina, ipak primjećuje da se većina više brine o svemu drugome, nego o prijateljima.

%\newpage

\section*{Pročitajte naglas grčki tekst.}

%Naslov prema izdanju

Xen.\ Memorabilia 2.4.1

\medskip

{\large
\begin{greek}
\noindent  Ἤκουσα δέ ποτε αὐτοῦ καὶ περὶ φίλων διαλεγομένου ἐξ ὧν ἔμοιγε ἐδόκει μάλιστ' ἄν τις ὠφελεῖσθαι πρὸς φίλων κτῆσίν τε καὶ χρείαν. τοῦτο μὲν γὰρ δὴ πολλῶν ἔφη ἀκούειν, ὡς πάντων κτημάτων κράτιστον ἂν εἴη φίλος σαφὴς καὶ ἀγαθός· ἐπιμελομένους δὲ παντὸς μᾶλλον ὁρᾶν ἔφη τοὺς πολλοὺς ἢ φίλων κτήσεως. καὶ γὰρ οἰκίας καὶ ἀγροὺς καὶ ἀνδράποδα καὶ βοσκήματα καὶ σκεύη κτωμένους τε ἐπιμελῶς ὁρᾶν ἔφη καὶ τὰ ὄντα σῴζειν πειρωμένους, φίλον δέ, ὃ μέγιστον ἀγαθὸν εἶναί φασιν, ὁρᾶν ἔφη τοὺς πολλοὺς οὔτε ὅπως κτήσωνται φροντίζοντας οὔτε ὅπως οἱ ὄντες αὐτοῖς σῴζωνται. ἀλλὰ καὶ καμνόντων φίλων τε καὶ οἰκετῶν ὁρᾶν τινας ἔφη τοῖς μὲν οἰκέταις καὶ ἰατροὺς εἰσάγοντας καὶ τἆλλα τὰ πρὸς ὑγίειαν ἐπιμελῶς παρασκευάζοντας, τῶν δὲ φίλων ὀλιγωροῦντας, ἀποθανόντων τε ἀμφοτέρων ἐπὶ μὲν τοῖς οἰκέταις ἀχθομένους τε καὶ ζημίαν ἡγουμένους, ἐπὶ δὲ τοῖς φίλοις οὐδὲν οἰομένους ἐλαττοῦσθαι, καὶ τῶν μὲν ἄλλων κτημάτων οὐδὲν ἐῶντας ἀθεράπευτον οὐδ' ἀνεπίσκεπτον, τῶν δὲ φίλων ἐπιμελείας δεομένων ἀμελοῦντας.
\end{greek}

}

