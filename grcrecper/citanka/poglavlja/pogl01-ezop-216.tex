% pregledali NČ, NZ
\section*{O autoru}

Ezop \textgreek[variant=ancient]{(Αἴσωπος,} ? 620-564. pr. Kr.; \textit{floruit} oko 570.\ pr.~Kr.) grčki je autor basni, čija izvorna djela nisu sačuvana, a o životu se malo toga zna.

Točno mjesto Ezopova rođenja nije poznato. Izvori se slažu da je Ezop prvotno bio rob, koji je zatim bio oslobođen. Prema Plutarhu (Plut.\ de sera, 556 E 1-557 A 10), ubijen je u Delfima. O Ezopovu životu je sačuvano mnoštvo anegdota. Jedan takav anegdotalan podatak odnosi se na njegovu navodnu iznimnu tjelesnu deformiranost.

U tradiciji se Ezopovo ime prvenstveno se povezuje sa životinjskom basnom \textgreek[variant=ancient]{(αἶνος),} književnom vrstom u kojoj se miješaju elementi satire, fantazije i moralne poduke. U basnama su često glavni likovi životinje s ljudskim karakteristikama (govor, razmišljanje\dots). Filolozi pretpostavljaju da su Ezopove basne bile zapisane već u 5. st. pr. Kr, a bile su vrlo popularne u Ateni klasičnog perioda.

Ezopov korpus basni sačuvan je predajom niza grčkih i latinskih autora. Demetrije iz Falera sastavio je (za nas izgubljenu) zbirku u deset knjiga kao priručnik govornicima. Najpoznatiji sakupljač Ezopovih basni bio je Fedar, koji ih je u 1. st. prepjevao na latinski.


\section*{O tekstu}

U ovoj basni jasne moralne poduke suprotstavljene su strane majka i sin: kako nije na vrijeme ispravljala djetetove krive postupke i kažnjavala krađu, dijete nije naučilo što je loše ponašanje. Nastavljajući krasti, sada već kao mladić, sin bude uhvaćen na djelu i osuđen na smrt. Dok ga je majka tada korila, ljutiti joj sin odgrize uho. Na majčino pitanje zašto je to učinio, sin joj je objasnio da stradava jer ga je ostavila bez roditeljskog odgoja.

\section*{Pročitajte naglas grčki tekst.}

Aesop. Fabulae 216

\medskip

{\large
\begin{greek}
\noindent ΠΑΙΣ ΚΛΕΠΤΗΣ ΚΑΙ ΜΗΤΗΡ 

\medskip

\noindent Παῖς ἐκ διδασκαλείου τὴν τοῦ συμφοιτητοῦ δέλτον ἀφελόμενος τῇ μητρὶ ἐκόμισε. τῆς δὲ οὐ μόνον αὐτὸν μὴ ἐπιπληξάσης, ἀλλὰ καὶ ἐπαινεσάσης αὐτὸν ἐκ δευτέρου ἱμάτιον κλέψας ἤνεγκεν αὐτῇ· ἔτι δὲ μᾶλλον ἀποδεξαμένης αὐτῆς προϊὼν τοῖς χρόνοις ὡς νεανίας ἐγένετο, ἤδη καὶ τὰ μείζονα κλέπτειν  ἐπεχείρει. ληφθεὶς δέ ποτε ἐπ' αὐτοφώρῳ καὶ περιαγκωνισθεὶς ἐπὶ τὸν δήμιον ἀπήγετο. τῆς δὲ μητρὸς ἐπακολουθούσης αὐτῷ καὶ στερνοκοπούσης εἶπε βούλεσθαί τι αὐτῇ πρὸς τὸ οὖς εἰπεῖν καὶ προσελθούσης αὐτῆς ταχέως τοῦ ὠτίου ἐπιλαβόμενος καταδήξας ἀφείλετο. τῆς δὲ κατηγορούσης αὐτοῦ δυσσέβειαν, εἴπερ μὴ ἀρκεσθεὶς οἷς ἤδη πεπλημμέληκε καὶ τὴν μητέρα ἐλωβήσατο, ἐκεῖνος ὑπολαβὼν ἔφη· ``ἀλλ' ὅτε σοι πρῶτον τὴν δέλτον κλέψας ἤνεγκα, εἰ ἐπέπληξάς μοι, οὐκ ἂν μέχρι τούτου ἐχώρησα καὶ ἐπὶ θάνατον ἠγόμην.''

ὁ λόγος δηλοῖ, ὅτι τὸ κατ' ἀρχὰς μὴ κολαζόμενον ἐπὶ μεῖζον αὔξεται. 
\end{greek}

}

