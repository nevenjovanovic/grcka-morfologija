%\section*{O autoru}



\section*{O tekstu}
Ἀθηναίων πολιτεία (\textit{Ustav atenski}) u rukopisima je preneseno kao jedno od Ksenofontovih djela. Već od Diogena Laertija, a osobito od XIX.~stoljeća, Ksenofontovo je autorstvo prijeporno, te danas nepoznatog autora nazivamo ``Pseudo-Ksenofont'' ili, u angloameričkoj tradiciji, ``stari oligarh'' – ``stari'' zbog toga što se, po svemu sudeći, radi o najstarijem poznatom nam atičkom proznom djelu, a ne zbog autorove dobi.

Naslov sugerira da se radi o sustavnom prikazu organizacije atenske države, kakav će kasnije napisati Aristotel (a sam je Ksenofont autor jednako sustavnog \textit{Ustava lakedemonskog}), ali zapravo nije tako. Pseudo-Ksenofontovo je djelo kritička refleksija o političkom ustroju Atene iz doba Peloponeskog rata  (431.–404.\ pr.~Kr) . Razmatranje prednosti i nedostataka demokracije služi kao nekonvencionalan izazov, provokacija gospodarski i vojno manje sposobnim aristokratima, možda pripadnicima nekog ekskluzivnog ``kluba'' \textgreek[variant=ancient]{(ἑταιρεία).}

U drugom od tri poglavlja djela autor je nabrajao prednosti ``gospodarenja morem'' \textgreek[variant=ancient]{(θαλασσοκρατία);} sad ističe jedini nedostatak Atene kao gospodarice mora.


\newpage

\section*{Pročitajte naglas grčki tekst.}
Ps.-Xen.\ Atheniensium respublica 2.14.2
%Naslov prema izdanju

\medskip

{\large
\begin{greek}
\noindent Eἰ γὰρ νῆσον οἰκοῦντες θαλασσοκράτορες ἦσαν ᾿Αθηναῖοι, ὑπῆρχεν ἂν αὐτοῖς ποιεῖν μὲν κακῶς, εἰ ἐβούλοντο, πάσχειν δὲ μηδέν, ἕως τῆς θαλάττης ἦρχον, μηδὲ τμηθῆναι τὴν ἑαυτῶν γῆν μηδὲ προσδέχεσθαι τοὺς πολεμίους· νῦν δὲ οἱ γεωργοῦντες καὶ οἱ πλούσιοι ᾿Αθηναίων ὑπέρχονται τοὺς πολεμίους μᾶλλον, ὁ δὲ δῆμος, ἅτε εὖ εἰδὼς ὅτι οὐδὲν τῶν σφῶν ἐμπρήσουσιν οὐδὲ τεμοῦσιν, ἀδεῶς ζῇ καὶ οὐχ ὑπερχόμενος αὐτούς. πρὸς δὲ τούτοις καὶ ἑτέρου δέους ἀπηλλαγμένοι ἂν ἦσαν, εἰ νῆσον ᾤκουν, μηδέποτε προδοθῆναι τὴν πόλιν ὑπ' ὀλίγων μηδὲ πύλας ἀνοιχθῆναι μηδὲ πολεμίους ἐπεισπεσεῖν· πῶς γὰρ νῆσον οἰκούντων ταῦτ' ἂν ἐγίγνετο; μηδ' αὖ στασιάσαι τῷ δήμῳ μηδέν, εἰ νῆσον ᾤκουν· νῦν μὲν γὰρ εἰ στασιάσαιεν, ἐλπίδα ἂν ἔχοντες ἐν τοῖς πολεμίοις στασιάσειαν, ὡς κατὰ γῆν ἐπαξόμενοι· εἰ δὲ νῆσον ᾤκουν, καὶ ταῦτ' ἂν ἀδεῶς εἶχεν αὐτοῖς. ἐπειδὴ οὖν ἐξ ἀρχῆς οὐκ ἔτυχον οἰκήσαντες νῆσον, νῦν τάδε ποιοῦσι· τὴν μὲν οὐσίαν ταῖς νήσοις παρατίθενται, πιστεύοντες τῇ ἀρχῇ τῇ κατὰ θάλατταν, τὴν δὲ ᾿Αττικὴν γῆν περιορῶσι τεμνομένην, γιγνώσκοντες ὅτι εἰ αὐτὴν ἐλεήσουσιν, ἑτέρων ἀγαθῶν μειζόνων στερήσονται.

\end{greek}
}
