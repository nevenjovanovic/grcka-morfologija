% Unio korekture NZ, NJ 2019-08-08
%\section*{O autoru}



\section*{O tekstu}

Demosten je tijekom 349.~pr.~Kr.\ sastavio tri politička govora motivirana napadom makedonskoga kralja Filipa II. (Aleksandrova oca) na grad Olint, na poluotoku Halkidici. Kako je Olint u to vrijeme bio saveznik Atene, Demosten je u olintskim govorima poticao atenske političare da pomognu tom grčkom polisu.

U nekoliko navrata Olinćani su slali poklisare u Atenu, moleći za vojnu pomoć. No, Atenjani nisu bili voljni poduzeti vojni pohod jer je Olint bio predaleko. U prvom \textit{Olintskom govoru}, napisanom povodom dolaska prvog olintskog poslanstva, Demosten poziva Atenjane da po hitnom postupku izglasaju slanje vojske u Olint. Demosten u ovom izvatku upozorava da ne suprotstaviti se Filipu znači smrtnu opasnost.

%\newpage

\section*{Pročitajte naglas grčki tekst.}

Dem.\ Olynthiaca I 15
%Naslov prema izdanju

\medskip

{\large
\begin{greek}
\noindent Πρὸς θεῶν, τίς οὕτως εὐήθης ἐστὶν ὑμῶν ὅστις ἀγνοεῖ τὸν ἐκεῖθεν πόλεμον δεῦρ'  ἥξοντα, ἂν ἀμελήσωμεν; ἀλλὰ μήν, εἰ τοῦτο γενήσεται, δέδοικ', ὦ ἄνδρες ᾿Αθηναῖοι, μὴ τὸν αὐτὸν τρόπον ὥσπερ οἱ δανειζόμενοι ῥᾳδίως ἐπὶ τοῖς μεγάλοις [τόκοις] μικρὸν εὐπορήσαντες χρόνον ὕστερον καὶ τῶν ἀρχαίων ἀπέστησαν, οὕτω καὶ ἡμεῖς [ἂν] ἐπὶ πολλῷ φανῶμεν ἐρρᾳθυμηκότες, καὶ ἅπαντα πρὸς ἡδονὴν ζητοῦντες πολλὰ καὶ χαλεπὰ ὧν οὐκ ἐβουλόμεθ' ὕστερον εἰς ἀνάγκην ἔλθωμεν ποιεῖν, καὶ κινδυνεύσωμεν περὶ τῶν ἐν αὐτῇ τῇ χώρᾳ.

\end{greek}
}

