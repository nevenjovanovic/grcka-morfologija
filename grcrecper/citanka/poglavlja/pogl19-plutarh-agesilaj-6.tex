% Unesi korekture NČ 2019-09-21
%\section*{O autoru}



\section*{O tekstu}

U \textit{Paralelnim biografijama} Agesilajev je životopis u paru s Pompejevim. Agesilaj II. (oko 443.\ – 359/358.\ pr.~Kr.) bio je spartanski kralj (vladao 398.–359/358.) i vrlo utjecajan grčki političar i vojskovođa. Niska rasta, kljast od rođenja, kraljem je postao ponešto neočekivano. Uspješno je ratovao u Maloj Aziji i tijekom Korintskog rata (395.–387.). Agesilajev je prijatelj bio i povjesničar Ksenofont, koji je također sastavio njegov životopis. 

Ovdje odabrani odlomak prikazuje prvi Agesilajev pohod pošto je postao kralj, uz podršku spartanskoga vojskovođe i admirala (i Agesilajeva ljubavnika) Lisandra (koji je 405.\ pobijedio Atenjane kod Egospotama, okončao Peloponeski rat, osigurao dominaciju Sparte u Grčkoj).

Na Lisandrov nagovor, Agesilaj se 396.\ sprema prijeći u Aziju s vojskom sastavljenom od neodamoda (oslobođenih helota), saveznika i vrlo malog broja Spartanaca, pod izlikom da želi grčke gradove u Aziji osloboditi perzijske vlasti.

%\newpage

\section*{Pročitajte naglas grčki tekst.}
Plut. Agesilaus 6.1
%Naslov prema izdanju

\medskip

{\large
\begin{greek}
\noindent Τοῦ δὲ Ἀγησιλάου τὴν βασιλείαν νεωστὶ παρειληφότος, ἀπήγγελλόν τινες ἐξ Ἀσίας ἥκοντες ὡς ὁ Περσῶν βασιλεὺς παρασκευάζοιτο μεγάλῳ στόλῳ Λακεδαιμονίους ἐκβαλεῖν τῆς θαλάσσης. ὁ δὲ Λύσανδρος ἐπιθυμῶν αὖθις εἰς Ἀσίαν ἀποσταλῆναι καὶ βοηθῆσαι τοῖς φίλοις, οὓς αὐτὸς μὲν ἄρχοντας καὶ κυρίους τῶν πόλεων ἀπέλιπε, κακῶς δὲ χρώμενοι καὶ βιαίως τοῖς πράγμασιν ἐξέπιπτον ὑπὸ τῶν πολιτῶν καὶ ἀπέθνησκον, ἀνέπεισε τὸν Ἀγησίλαον ἐπιθέσθαι τῇ στρατείᾳ καὶ προπολεμῆσαι τῆς Ἑλλάδος, ἀπωτάτω διαβάντα καὶ φθάσαντα τὴν τοῦ βαρβάρου παρασκευήν. ἅμα δὲ τοῖς ἐν Ἀσίᾳ φίλοις ἐπέστελλε πέμπειν εἰς Λακεδαίμονα καὶ στρατηγὸν Ἀγησίλαον αἰτεῖσθαι. παρελθὼν οὖν εἰς τὸ πλῆθος Ἀγησίλαος ἀνεδέξατο τὸν πόλεμον, εἰ δοῖεν αὐτῷ τριάκοντα μὲν ἡγεμόνας καὶ συμβούλους Σπαρτιάτας, νεοδαμώδεις δὲ λογάδας δισχιλίους, τὴν δὲ συμμαχικὴν εἰς ἑξακισχιλίους δύναμιν. συμπράττοντος δὲ τοῦ Λυσάνδρου πάντα προθύμως ἐψηφίσαντο, καὶ τὸν Ἀγησίλαον ἐξέπεμπον εὐθὺς ἔχοντα τοὺς τριάκοντα Σπαρτιάτας, ὧν ὁ Λύσανδρος ἦν πρῶτος, οὐ διὰ τὴν ἑαυτοῦ δόξαν καὶ δύναμιν μόνον, ἀλλὰ καὶ διὰ τὴν Ἀγησιλάου φιλίαν, ᾧ μεῖζον ἐδόκει τῆς βασιλείας ἀγαθὸν διαπεπρᾶχθαι τὴν στρατηγίαν ἐκείνην.

\end{greek}
}

