% Unio ispravke NČ 2019-09-05
%\section*{O autoru}


\section*{O tekstu}

Na početku Peloponeskog rata, prilikom prve provale Spartanaca u Atiku, pod vodstvom Arhidama, u ljeto 431.\ pr.~Kr, Periklo se odlučio za nekonvencionalnu taktiku: umjesto da se sukobe sa Spartancima na otvorenom (tradicionalno ratovanje poljoprivrednih državica redovno se svodilo na provociranje uništavanjem usjeva te kratku i ubojitu odlučnu bitku nakon toga), Atenjani su se povukli unutar gradskih zidina i ostali ondje, oslanjajući se na svoju financijsku, političku i morsku prevlast. Znali su da sve što su Spartanci uništili mogu lako nadoknaditi uz potporu saveznika i dopremajući namirnice brodovima, a da će atenska mornarica, zauzvrat, opustošiti Peloponez. Pa ipak, nije bilo lako gledati uništavanje vrijedne imovine. Zato je u zimu, kad su Atenjani svečano sahranili prve žrtve rata, Periklo održao slavni nadgrobni govor (Thuc. Hist. 2.35-46). Potom je, međutim, u Ateni izbila kuga, i nakon druge ljetne provale (430.) Atenjani su počeli očajavati, kritizirati Periklovu strategiju, slati poslanike Spartancima. Novim ih je govorom Periklo uspio barem načelno umiriti – no gorčina zbog privatnih gubitaka nije odmah nestala.

%\newpage

\section*{Pročitajte naglas grčki tekst.}

Thuc.\ Historiae 2.65.2

%Naslov prema izdanju

\medskip

\begin{greek}
{\large
{ \noindent Oἱ δὲ δημοσίᾳ μὲν τοῖς λόγοις ἀνεπείθοντο καὶ οὔτε πρὸς τοὺς Λακεδαιμονίους ἔτι ἔπεμπον ἔς τε τὸν πόλεμον μᾶλλον ὥρμηντο, ἰδίᾳ δὲ τοῖς παθήμασιν ἐλυποῦντο, ὁ μὲν δῆμος ὅτι ἀπ' ἐλασσόνων ὁρμώμενος ἐστέρητο καὶ τούτων, οἱ δὲ δυνατοὶ καλὰ κτήματα κατὰ τὴν χώραν οἰκοδομίαις τε καὶ πολυτελέσι κατασκευαῖς ἀπολωλεκότες, τὸ δὲ μέγιστον, πόλεμον ἀντ' εἰρήνης ἔχοντες. οὐ μέντοι πρότερόν γε οἱ ξύμπαντες ἐπαύσαντο ἐν ὀργῇ ἔχοντες αὐτὸν πρὶν ἐζημίωσαν χρήμασιν. ὕστερον δ' αὖθις οὐ πολλῷ, ὅπερ φιλεῖ ὅμιλος ποιεῖν, στρατηγὸν εἵλοντο καὶ πάντα τὰ πράγματα ἐπέτρεψαν, ὧν μὲν περὶ τὰ οἰκεῖα ἕκαστος ἤλγει ἀμβλύτεροι ἤδη ὄντες, ὧν δὲ ἡ ξύμπασα πόλις προσεδεῖτο πλείστου ἄξιον νομίζοντες εἶναι. ὅσον τε γὰρ χρόνον προύστη τῆς πόλεως ἐν τῇ εἰρήνῃ, μετρίως ἐξηγεῖτο καὶ ἀσφαλῶς διεφύλαξεν αὐτήν, καὶ ἐγένετο ἐπ' ἐκείνου μεγίστη, ἐπειδή τε ὁ πόλεμος κατέστη, ὁ δὲ φαίνεται καὶ ἐν τούτῳ προγνοὺς τὴν δύναμιν.

}
}
\end{greek}

