% Unesi ispravke NČ 2019-09-22
%\section*{O autoru}



\section*{O tekstu}

U ovom dijelu apologije mitske ljepotice Helene retor i sofist sa Sicilije dokazuje da je Helena samo \textit{pretrpjela} nepravdu i sramotu; Paris, njezin otmičar, zgriješio je nakanom, kršenjem zakonitosti i djelom; Helena je zapravo žrtva. Na prigovor da je Helenin postupak skrivila riječ (λόγος), Gorgija odgovara prikazujući moć ovoga načina komunikacije i najvišeg oblika upotrebe riječi – poezije.

%\newpage

\section*{Pročitajte naglas grčki tekst.}
Gorg. Helenae encomium Fr.\ 11.41
%Naslov prema izdanju

\medskip

{\large
\begin{greek}
\noindent Εἰ δὲ βίαι ἡρπάσθη καὶ ἀνόμως ἐβιάσθη καὶ ἀδίκως ὑβρίσθη, δῆλον ὅτι ὁ $\langle$μὲν$\rangle$ ἁρπάσας ὡς ὑβρίσας ἠδίκησεν, ἡ δὲ ἁρπασθεῖσα ὡς ὑβρισθεῖσα ἐδυστύχησεν. ἄξιος οὖν ὁ μὲν ἐπιχειρήσας βάρβαρος βάρβαρον ἐπιχείρημα καὶ λόγωι καὶ νόμωι καὶ ἔργωι λόγωι μὲν αἰτίας, νόμωι δὲ ἀτιμίας, ἔργωι δὲ ζημίας τυχεῖν· ἡ δὲ βιασθεῖσα καὶ τῆς πατρίδος στερηθεῖσα καὶ τῶν φίλων ὀρφανισθεῖσα πῶς οὐκ ἂν εἰκότως ἐλεηθείη μᾶλλον ἢ κακολογηθείη; ὁ μὲν γὰρ ἔδρασε δεινά, ἡ δὲ ἔπαθε· δίκαιον οὖν τὴν μὲν οἰκτῖραι, τὸν δὲ μισῆσαι. 

εἰ δὲ λόγος ὁ πείσας καὶ τὴν ψυχὴν ἀπατήσας, οὐδὲ πρὸς τοῦτο χαλεπὸν ἀπολογήσασθαι καὶ τὴν αἰτίαν ἀπολύσασθαι ὧδε. λόγος δυνάστης μέγας ἐστίν, ὃς σμικροτάτωι σώματι καὶ ἀφανεστάτωι θειότατα ἔργα ἀποτελεῖ· δύναται γὰρ καὶ φόβον παῦσαι καὶ λύπην ἀφελεῖν καὶ χαρὰν ἐνεργάσασθαι καὶ ἔλεον ἐπαυξῆσαι. ταῦτα δὲ ὡς οὕτως ἔχει δείξω· δεῖ δὲ καὶ δόξηι δεῖξαι τοῖς ἀκούουσι· τὴν ποίησιν ἅπασαν καὶ νομίζω καὶ ὀνομάζω λόγον ἔχοντα μέτρον· ἧς τοὺς ἀκούοντας εἰσῆλθε καὶ φρίκη περίφοβος καὶ ἔλεος πολύδακρυς καὶ πόθος φιλοπενθής, ἐπ' ἀλλοτρίων τε πραγμάτων καὶ σωμάτων εὐτυχίαις καὶ δυσπραγίαις ἴδιόν τι πάμα διὰ τῶν λόγων ἔπαθεν ἡ ψυχή.

\end{greek}

}

