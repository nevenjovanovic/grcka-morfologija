%\section*{O autoru}



\section*{O tekstu}

U Plutarhovim \textit{Paralelnim životopisima} \textgreek[variant=ancient]{(Βίοι παράλληλοι)} biografija Solona Atenjanina (639.–559.\ pr.~Kr.) stoji uz životopis Rimljanina Publija Valerija Poplikole (ili Publikole, umro 503.\ pr.~Kr.).

Pošto je, na početku Solonova životopisa, kratko prikazao porijeklo atenskoga mudraca, njegovo prijateljstvo s kasnijim tiraninom Pizistratom i bavljenje trgovinom (koju Solon nije smatrao nedostojnom svoga plemenitog porijekla), Plutarh čini digresiju i pripovijeda o Talesu iz maloazijskog Mileta, prvome koji je filozofiju učinio spekulativnom, a ne isključivo praktičnom i pragmatičnom djelatnošću. 

I Tales i Solon, kao i Bijant iz Prijene, pripadali su sedmorici drevnih grčkih mudraca \textgreek[variant=ancient]{(οἱ ἑπτὰ σοφοί).} Oni su, prema Plutarhu, status mudraca potvrdili ponajprije u događaju s tronošcem \textgreek[variant=ancient]{(τρίπους)} o kojem pripovijeda odabrani odlomak.

%\newpage

\section*{Pročitajte naglas grčki tekst.}

%Naslov prema izdanju
Plut.\ Solon 4.3

\medskip

{\large
\begin{greek}
\noindent Κῴων γὰρ ὥς φασι καταγόντων σαγήνην, καὶ ξένων ἐκ Μιλήτου πριαμένων τὸν βόλον οὔπω φανερὸν ὄντα, χρυσοῦς ἐφάνη τρίπους ἑλκόμενος, ὃν λέγουσιν Ἐλένην πλέουσαν ἐκ Τροίας αὐτόθι καθεῖναι, χρησμοῦ τινος ἀναμνησθεῖσαν παλαιοῦ. γενομένης δὲ τοῖς ξένοις πρῶτον ἀντιλογίας πρὸς τοὺς ἁλιέας περὶ τοῦ τρίποδος, εἶτα τῶν πόλεων ἀναδεξαμένων τὴν διαφοράν, ἄχρι πολέμου προελθοῦσαν, ἀνεῖλεν ἀμφοτέροις ἡ Πυθία τῷ σοφωτάτῳ τὸν τρίποδα ἀποδοῦναι. καὶ πρῶτον μὲν ἀπεστάλη πρὸς Θαλῆν εἰς Μίλητον, ἑκουσίως τῶν Κῴων ἑνὶ δωρουμένων ἐκείνῳ περὶ οὗ πρὸς ἅπαντας ὁμοῦ Μιλησίους ἐπολέμησαν. Θάλεω δὲ Βίαντα σοφώτερον ἀποφαίνοντος αὑτοῦ, πρὸς ἐκεῖνον ἧκεν, ὑπ' ἐκείνου δ' αὖθις ἀπεστάλη πρὸς ἄλλον ὡς σοφώτερον. εἶτα περιιὼν καὶ ἀναπεμπόμενος, οὕτως ἐπὶ Θαλῆν τὸ δεύτερον ἀφίκετο, καὶ τέλος εἰς Θήβας ἐκ Μιλήτου κομισθεὶς τῷ Ἰσμηνίῳ Ἀπόλλωνι καθιερώθη.
\end{greek}

}

