%Unesene korekcije NZ i NJ
%\section*{O autoru}



\section*{O tekstu}

Poučna priča o savezništvu orla i lisice, o izdaji i božjoj kazni, dio je zbirke Ezopovih basni \textgreek[variant=ancient]{(Μῦθοι);} mada su one prvi put u proznu zbirku okupljene oko 300.\ pr.~Kr, u obliku u kojem su do nas došle zapravo su prozne parafraze kasnijih Babrijevih jampskih pjesama (I./II.\ st.\ po Kr) i radovi škola retorike različitih vremena. 

Jednu je varijantu basne o orlu i lisici obradio već pjesnik Arhiloh \textgreek[variant=ancient]{(Ἀρχίλοχος,} oko 650.\ pr.~Kr), drugu donose latinski stihovi oslobođenog rimskog roba Fedra (Phaedrus, oko 15.\ pr.~Kr. – 50.\ po Kr).

%\newpage

\section*{Pročitajte naglas grčki tekst.}
Aesop. Fabulae 1
%Naslov prema izdanju

\medskip

{\large
\begin{greek}
\noindent ΑΕΤΟΣ ΚΑΙ ΑΛΩΠΗΞ

\noindent Ἀετὸς καὶ ἀλώπηξ φιλίαν πρὸς ἀλλήλους σπεισάμενοι πλησίον ἑαυτῶν οἰκεῖν διέγνωσαν βεβαίωσιν φιλίας τὴν συνήθειαν ποιούμενοι. καὶ δὴ ὁ μὲν ἀναβὰς ἐπί τι περίμηκες δένδρον ἐνεοττοποιήσατο, ἡ δὲ εἰς τὸν ὑποκείμενον θάμνον ἔτεκεν. ἐξελθούσης δέ ποτε αὐτῆς ἐπὶ νομὴν ὁ ἀετὸς ἀπορῶν τροφῆς καταπτὰς εἰς τὸν θάμνον καὶ τὰ γεννήματα ἀναρπάσας μετὰ τῶν αὑτοῦ νεοττῶν κατεθοινήσατο. ἡ δὲ ἀλώπηξ ἐπανελθοῦσα ὡς ἔγνω τὸ πραχθέν, οὐ μᾶλλον ἐπὶ τῷ τῶν νεοττῶν θανάτῳ ἐλυπήθη, ὅσον ἐπὶ τῆς ἀμύνης· χερσαία γὰρ οὖσα πετεινὸν διώκειν ἠδυνάτει. διόπερ πόρρωθεν στᾶσα, ὃ μόνον τοῖς ἀσθενέσιν καὶ ἀδυνάτοις ὑπολείπεται, τῷ ἐχθρῷ κατηρᾶτο. συνέβη δὲ αὐτῷ τῆς εἰς τὴν φιλίαν ἀσεβείας οὐκ εἰς μακρὰν δίκην ὑποσχεῖν. θυόντων γάρ τινων αἶγα ἐπ' ἀγροῦ καταπτὰς ἀπὸ τοῦ βωμοῦ σπλάγχνον ἔμπυρον ἀνήνεγκεν· οὗ κομισθέντος ἐπὶ τὴν καλιὰν σφοδρὸς ἐμπεσὼν ἄνεμος ἐκ λεπτοῦ καὶ παλαιοῦ κάρφους λαμπρὰν φλόγα ἀνῆψε. καὶ διὰ τοῦτο καταφλεχθέντες οἱ νεοττοὶ — καὶ γὰρ ἦσαν ἔτι ἀτελεῖς οἱ πτηνοὶ — ἐπὶ τὴν γῆν κατέπεσον. καὶ ἡ ἀλώπηξ προσδραμοῦσα ἐν ὄψει τοῦ ἀετοῦ πάντας αὐτοὺς κατέφαγεν. 

ὁ λόγος δηλοῖ, ὅτι οἱ φιλίαν παρασπονδοῦντες, κἂν τὴν τῶν ἠδικημένων ἐκφύγωσι κόλασιν, ἀλλ' οὖν γε τὴν ἐκ θεοῦ τιμωρίαν οὐ διακρούσονται.

\end{greek}
}

