% Unesi korekture NČ 2019-09-19
\section*{O autoru}

Diodor Sicilski \textgreek[variant=ancient]{(Διόδωρος Σικελιώτης,} Diodorus Siculus; 90.-30.\ pr.~Kr.) rođen je u Agiriju \textgreek[variant=ancient]{(Ἀγύριον)} na Siciliji, dok mu mjesto smrti nije poznato. Svjedočanstava o njegovu životu, osim onih koji se daju iščitati iz samog djela, gotovo da nema. 

Bio je veliki putnik. Za boravka u Rimu, zahvaljujući dobrom znanju latinskog jezika, mogao se služiti rimskim arhivima i knjižnicama. Životno mu je djelo \textit{Knjižnica} \textgreek[variant=ancient]{(Βιβλιοθήκη)}; na djelu je radio trideset godina. Sam naslov je svojevrsno priznanje da se autor pri pisanju služio djelima brojnih povjesničara iz prošlosti.  Povijesni je prikaz obuhvaćao 40 knjiga i sezao je sve od mitskih početaka pa do Cezarova galskog rata. U cijelosti su očuvane knjige 1–5 i 11–20, a ostale samo u ulomcima. Diodorova koncepcija univerzalne povijesti svoj je uzor imala u djelu grčkog povjesničara Efora (400.–330.\ pr.~Kr.); uključivala je i Grke i barbare, a ponajprije Rim, koji je postao najvažnija sila poznatoga svijeta. Diodorova je posebnost u tome što je u djelo uvrstio i prikaz mitskoga doba. Velik je piščev doprinos što je povijest svijeta, s osvrtom na sve epohe i regije, od nastanka univerzuma do Cezarova doba, prikazana kao kontinuum.

Djelo je pisano jedinstvenim stilom, lako je čitljivo i razumljivo. Iz izvora preuzetoj građi autor je dao novo i moderno jezično ruho. Jezik je atički, prožet elementima iz koine; obiluje nominalnim konstrukcijama. Poput Polibija i Diodor izbjegava hijat. 

Popularna u antici i srednjem vijeku, \textit{Knjižnica} je u 16.~st. prevedena na latinski, a potom i na nacionalne jezike.

\section*{O tekstu}

U ovom ulomku Diodor opisuje mučan način života prvih ljudi. Primitivno stanje u kojem se nalazio ljudski rod prije pronalaska elementarnih znanja i vještina bilo je predmet spekulacije pisaca različitih književnih vrsta kroz stoljeća predaje. Predodžbe starih o prvotnom životu u oskudici – bez ognja, odjeće, stana i kruha – prizivaju ideju napretka od krajnje oskudice ka sve savršenijim oblicima egzistencije pojedinca i zajednice.

%\newpage

\section*{Pročitajte naglas grčki tekst.}

Diod.\ Sic.\ Bibliotheca historica 1.8.5

%Naslov prema izdanju

\medskip

\begin{greek}
{\large
{ \noindent Τοὺς οὖν πρώτους τῶν ἀνθρώπων μηδενὸς τῶν πρὸς βίον χρησίμων εὑρημένου ἐπιπόνως διάγειν, γυμνοὺς μὲν ἐσθῆτος ὄντας, οἰκήσεως δὲ καὶ πυρὸς ἀήθεις, τροφῆς δ' ἡμέρου παντελῶς ἀνεννοήτους. καὶ γὰρ τὴν συγκομιδὴν τῆς ἀγρίας τροφῆς ἀγνοοῦντας μηδεμίαν τῶν καρπῶν εἰς τὰς ἐνδείας ποιεῖσθαι παράθεσιν· διὸ καὶ πολλοὺς αὐτῶν ἀπόλλυσθαι κατὰ τοὺς χειμῶνας διά τε τὸ ψῦχος καὶ τὴν σπάνιν τῆς τροφῆς. ἐκ δὲ τοῦ κατ' ὀλίγον ὑπὸ τῆς πείρας διδασκομένους εἴς τε τὰ σπήλαια καταφεύγειν ἐν τῷ χειμῶνι καὶ τῶν καρπῶν τοὺς φυλάττεσθαι δυναμένους ἀποτίθεσθαι. γνωσθέντος δὲ τοῦ πυρὸς καὶ τῶν ἄλλων τῶν χρησίμων κατὰ μικρὸν καὶ τὰς τέχνας εὑρεθῆναι καὶ τἄλλα τὰ δυνάμενα τὸν κοινὸν βίον ὠφελῆσαι.

}
}
\end{greek}


%kraj
