% Redaktura NZ, NJ
\section*{O autoru}

Izokrat (Ἰσοκράτης, 436.–338.) potječe iz bogate atenske obitelji, no većinu je imovine izgubio u Peloponeskom ratu tako da se 403.\ prihvaća pisanja govora (logografije). Oko 390.\ napušta to zanimanje i počinje se baviti pisanjem i podučavanjem izlažući svoje pedagoške, filozofske i političke stavove u ogledima koji su preuzeli oblik govora, no nisu bili namijenjeni javnoj i usmenoj izvedbi. Zalagao se za prilagodbu rastućoj moći Filipa Makedonskog i za panhelensko jedinstvo. Izokrat se smatrao filozofom i pedagogom, a ne govornikom i retoričarom. Njegova škola usvojila je široko poimanje retorike i primijenjene filozofije, a privlačila je učenike iz cijeloga grčkog svijeta (uključujući Izeja, Likurga i Hiperida), postavši tako glavni takmac Platonovoj Akademiji. Izokrat je silno utjecao na obrazovanje i retoriku u helenističko, rimsko i moderno doba, sve do XVIII. stoljeća.

\section*{O tekstu}

\textit{Buzirid} (Βούσιρις) je napisan poslije 390.\ pr.~Kr., kada je Izokrat započeo javno i obrazovno djelovanje. U to vrijeme, obilježeno mišlju i djelom filozofa Platona, na društvenoj i misaonoj razini dominiraju pitanja politike i uvjerljivog govorenja, filozofije i obrazovanja. \textit{Buzirid} je jedan od rijetkih sačuvanih spisa toga vremena koji izlažu ideje drugačije ili konkurentne Platonovima. 

Izokratovo je djelo oblikovano kao pismo retoričaru i sofistu Polikratu čije djelovanje predstavlja sve ono čemu se Izokrat u retorici i obrazovanju protivi. Svoju estetiku Izokrat zaodijeva u pohvalu egipatskog kralja Buzirida. Pritom razmatra egipatsku civilizaciju i njezin značaj. Sadržajnu raznovrsnost prati i raznolikost stila pa se u tekstu izmjenjuju polemičnost, sarkazam, šaljivost, pripovijedanje i epideiktičan ton. \textit{Buzirid} je važan za razumijevanje javne uloge Izokrata kao pisc.\ i odgojitelja.

Grčka mitološka literatura različito prikazuje Buziridovo porijeklo i život. Prema mitografu Apolodoru (II.~st.\ pr.~Kr.), Buzirid je bio sin Posejdona i Epafove kćeri Lizijanase: nakon što je Egipat devet godina morila glad, prorok Frazije s Cipra obznanjuje da će ona prestati ako Egipćani svake godine žrtvuju Zeusu jednoga stranca. Buzirid je prihvatio savjet i prvo ubio samog Frazija, a zatim i sve ostale strance. Kad je Heraklo došao u Egipat, i njega su svezali da ga žrtvuju. No, Heraklo je rastrgnuo lance te ubio i Buzirida i njegovu svitu. 

Početak Izokratova djela, ovdje donesen, izlaže kako je Pitagora od Egipćana naučio filozofiju i donio je u Grčku.

%\newpage

\section*{Pročitajte naglas grčki tekst.}

%Naslov prema izdanju

Isoc.\ Busiris 28.1

\medskip

{\large
\begin{greek}
\noindent Ἀφικόμενος εἰς Αἴγυπτον καὶ μαθητὴς ἐκείνων γενόμενος τήν τ' ἄλλην φιλοσοφίαν πρῶτος εἰς τοὺς ῞Ελληνας ἐκόμισεν, καὶ τὰ περὶ τὰς θυσίας καὶ τὰς ἁγιστείας τὰς ἐν τοῖς ἱεροῖς ἐπιφανέστερον τῶν ἄλλων ἐσπούδασεν, ἡγούμενος, εἰ καὶ μηδὲν αὐτῷ διὰ ταῦτα πλέον γίγνοιτο παρὰ τῶν θεῶν, ἀλλ' οὖν παρά γε τοῖς ἀνθρώποις ἐκ τούτων μάλιστ' εὐδοκιμήσειν. ῞Οπερ αὐτῷ καὶ συνέβη· τοσοῦτον γὰρ εὐδοξίᾳ τοὺς ἄλλους ὑπερέβαλεν ὥστε καὶ τοὺς νεωτέρους ἅπαντας ἐπιθυμεῖν αὐτοῦ μαθητὰς εἶναι, καὶ τοὺς πρεσβυτέρους ἥδιον ὁρᾶν τοὺς παῖδας τοὺς αὑτῶν ἐκείνῳ συγγιγνομένους ἢ τῶν οἰκείων ἐπιμελουμένους. Καὶ τούτοις οὐχ οἷόν τ' ἀπιστεῖν· ἔτι γὰρ καὶ νῦν τοὺς προσποιουμένους ἐκείνου μαθητὰς εἶναι μᾶλλον σιγῶντας θαυμάζουσιν ἢ τοὺς ἐπὶ τῷ λέγειν μεγίστην δόξαν ἔχοντας.

\end{greek}

}

