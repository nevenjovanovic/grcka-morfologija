%Unesi korekture NČ, 2019-08-16
%\section*{O autoru}


\section*{O tekstu}

Grčki govornik Lizija napisao je i, prema predaji, sâm predstavio \textit{Olimpijski govor} 338.\ pr.~Kr.\ na svečanostima prigodom Olimpijskih igara. Ovaj pledoaje za jedinstvo i slobodu Grka svrstava se u prigodno (epideiktično) govorništvo.

\textit{Olimpijski govor} sačuvan je kao opsežan citat kod grčkog retora i povjesničara Dionizija Halikarnašanina (\textit{O starim govornicima}, 29-30). Dodatne podatke vezane za ovo Lizijino djelo nalazimo kod povjesničara Diodora Sicilskoga (\textit{Knjižnica}, XIV. 105). 

Kao sin doseljenika iz Sirakuze u Atenu, Lizija u Ateni nije smio javno govoriti. To je smio, međutim, na svečanom okupljanju svih Grka povodom Olimpijskih igara, gdje se obraća okupljenima s jasnim i izravnim pozivom na svrgavanje sirakuškoga tiranina Dionizija Starijeg. Naime, od dolaska na vlast 405.\ pr.~Kr.\ Dionizije Stariji širio je svoju vlast, prvo iz Sirakuze na cijelu Siciliju, zatim i na južnu Italiju, a od 395.\ pr.~Kr., kada počinje Korintski rat, imao je ulogu i u tom sukobu grčkih polisa, Sparte, Atene i Perzije. Zato je, drži govornik, tiranin Sirakuze za Grke opasan koliko i perzijski kralj Artakserkso II.

Godine 388.\ pr.~Kr.\ Dionizije Stariji pripremio je i bogato opremio poslanstvo u Olimpiju da tamo u njegovo ime prinese raskošnu žrtvu. Šatori poslanika bili su ukrašeni zlatom, na utrkama četveroprega natjecali su se Dionizijevi konji, a recitatori su izvodili Dionizijeve pjesme. Raskoš, bogatstvo i moć trebali su impresionirati Grke. No, demokracijom prožetoj Ateni samovlada i moć Dionizija Starijega bila je odbojna. Apel za svrgnuće tiranina i oslobođenje Sicilije Lizija gradi na tom sentimentu.

%\newpage

\section*{Pročitajte naglas grčki tekst.}

Lys.\ Olympiacus 1

%Naslov prema izdanju

\medskip


{\large
{ \noindent I

\begin{greek}

\noindent Ἄλλων τε πολλῶν καὶ καλῶν ἔργων ἕνεκα, ὦ ἄνδρες, ἄξιον Ἡρακλέους μεμνῆσθαι, καὶ ὅτι τόνδε τὸν ἀγῶνα πρῶτος συνήγειρε δι' εὔνοιαν τῆς Ἑλάδος. ἐν μὲν γὰρ τῷ τέως χρόνῳ ἀλλοτρίως αἱ πόλεις πρὸς ἀλλήλας διέκειντο· ἐπειδὴ δὲ ἐκεῖνος τοὺς τυράννους ἔπαυσε καὶ τοὺς ὑβρίζοντας ἐκώλυσεν, ἀγῶνα μὲν σωμάτων ἐποίησε, φιλοτιμίαν $\langle$δὲ$\rangle$ πλούτου, γνώμης δ' ἐπίδειξιν ἐν τῷ καλλίστῳ τῆς Ἑλλάδος, ἵνα τούτων ἁπάντων ἕνεκα εἰς τὸ αὐτὸ συνέλθωμεν, τὰ μὲν ὀψόμενοι, τὰ δ' ἀκουσόμενοι· ἡγήσατο γὰρ τὸν ἐνθάδε σύλλογον ἀρχὴν γενήσεσθαι τοῖς  Ἕλλησι τῆς πρὸς ἀλλήλους φιλίας. 

\end{greek}

\noindent II

\begin{greek}

\noindent  Ἐγὼ δὲ ἥκω οὐ μικρολογησόμενος οὐδὲ περὶ τῶν ὀνομάτων μαχούμενος. ἡγοῦμαι γὰρ ταῦτα ἔργα μὲν εἶναι σοφιστῶν λίαν ἀχρήστων καὶ σφόδρα βίου δεομένων, ἀνδρὸς δὲ ἀγαθοῦ καὶ πολίτου πολλοῦ ἀξίου περὶ τῶν μεγίστων συμβουλεύειν, ὁρῶν οὕτως αἰσχρῶς διακειμένην τὴν Ἑλλάδα, καὶ πολλὰ μὲν αὐτῆς ὄντα ὑπὸ τῷ βαρβάρῳ, πολλὰς δὲ πόλεις ὑπὸ τυράννων ἀναστάτους γεγενημένας. καὶ ταῦτα εἰ μὲν δι' ἀσθένειαν ἐπάσχομεν, στέργειν ἂν ἦν ἀνάγκη τὴν τύχην· ἐπειδὴ δὲ διὰ στάσιν καὶ τὴν πρὸς ἀλλήλους φιλονικίαν, πῶς οὐκ ἄξιον τῶν μὲν παύσασθαι τὰ δὲ κωλῦσαι, εἰδότας ὅτι φιλονικεῖν μέν ἐστιν εὖ πραττόντων, γνῶναι δὲ τὰ βέλτιστα τῶν οἵων ἡμῶν; ὁρῶμεν γὰρ τοὺς κινδύνους καὶ μεγάλους καὶ πανταχόθεν περιεστηκότας· ἐπίστασθε δὲ ὅτι ἡ μὲν ἀρχὴ τῶν κρατούντων τῆς θαλάττης, τῶν δὲ χρημάτων βασιλεὺς ταμίας, τὰ δὲ τῶν Ἑλλήνων σώματα τῶν δαπανᾶσθαι δυναμένων, ναῦς δὲ πολλὰς $\langle$μὲν$\rangle$ αὐτὸς κέκτηται, πολλὰς δ' ὁ τύραννος τῆς Σικελίας. ὥστε ἄξιον τὸν μὲν πρὸς ἀλλήλους πόλεμον καταθέσθαι, τῇ δ' αὐτῇ γνώμῃ χρωμένους τῆς σωτηρίας ἀντέχεσθαι, καὶ περὶ μὲν τῶν παρεληλυθότων αἰσχύνεσθαι, περὶ δὲ τῶν μελλόντων ἔσεσθαι δεδιέναι, καὶ πρὸς τοὺς προγόνους ἁμιλλᾶσθαι, οἳ τοὺς μὲν βαρβάρους ἐποίησαν τῆς ἀλλοτρίας ἐπιθυμοῦντας τῆς σφετέρας αὐτῶν στερεῖσθαι, τοὺς δὲ τυράννους ἐξελάσαντες κοινὴν ἅπασι τὴν ἐλευθερίαν κατέστησαν.
\end{greek}

}
}

