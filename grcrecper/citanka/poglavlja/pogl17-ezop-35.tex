% Unesi korekture NČ, NZ 2019-09-23
%\section*{O autoru}



\section*{O tekstu}

Ova Ezopova basna priča je o satiru koji je ostao zatečen vidjevši svojega prijatelja čovjeka kako zbog velike hladnoće puše u prste da ih ugrije, ali odmah potom puše i u toplo jelo da ga rashladi. Kako mu je bilo nepojmljivo da iz istih usta izlaze i toplina i hladnoća, satir je naprasno prekinuo prijateljstvo s takvim čovjekom. Moralna pouka, koju je Erazmo Roterdamski u svojoj zbirci \textit{Adagia} sažeo u izreci \textit{Ex eodem ore calidum et frigidum efflare}, sugerirala je da se treba kloniti dvoličnih ljudi, onih koji jedno govore, a drugo misle te im se zbog toga ne može vjerovati. Kasnije, u doba prosvjetiteljstva, basnu su tumačili mislioci poput Voltairea i Lessinga, upozoravajući na u njoj prisutnu nelogičnost: čovjek je doista, pušući u ruke i u jelo, činio ono što je u danom trenutku trebao činiti, koliko god to djelovalo proturječno. Zbog pomalo nategnute poruke pomišljalo se i na mogućnost da je basna naknadno smišljena kako bi dala okvir već postojećoj poslovici.

%\newpage

\section*{Pročitajte naglas grčki tekst.}
Aesop.\ Fabulae 35
%Naslov prema izdanju

\medskip

{\large
\begin{greek}
\noindent ΑΝΘΡΩΠΟΣ ΚΑΙ ΣΑΤΥΡΟΣ 

\noindent Ἄνθρωπόν ποτε λέγεται πρὸς σάτυρον φιλίαν σπείσασθαι. καὶ δὴ χειμῶνος καταλαβόντος καὶ ψύχους γενομένου ὁ ἄνθρωπος προσφέρων τὰς χεῖρας τῷ στόματι ἐπέπνει. τοῦ δὲ σατύρου τὴν αἰτίαν ἐρομένου δι' ἣν τοῦτο πράττει, ἔλεγεν, ὅτι θερμαίνει τὰς χεῖρας διὰ τὸ κρύος. ὕστερον δὲ παρατεθείσης αὐτοῖς τραπέζης καὶ προσφαγήματος θερμοῦ σφόδρα ὄντος ὁ ἄνθρωπος ἀναιρούμενος κατὰ μικρὸν τῷ στόματι προσέφερε καὶ ἐφύσα. πυνθανομένου δὲ πάλιν τοῦ σατύρου, τί τοῦτο ποιεῖ, ἔφασκε καταψύχειν τὸ ἔδεσμα, ἐπεὶ λίαν θερμόν ἐστι. κἀκεῖνος ἔφη πρὸς αὐτόν· ``ἀλλ' ἀποτάσσομαί σου τῇ φιλίᾳ, ὦ οὗτος, ὅτι ἐκ τοῦ αὐτοῦ στόματος τὸ θερμὸν καὶ τὸ ψυχρὸν ἐξιεῖς.''

ἀτὰρ οὖν καὶ ἡμᾶς περιφεύγειν δεῖ τὴν φιλίαν, ὧν ἀμφίβολός ἐστιν ἡ διάθεσις.

\end{greek}

}

