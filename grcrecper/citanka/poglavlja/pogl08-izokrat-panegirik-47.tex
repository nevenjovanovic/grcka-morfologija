% Redaktura NČ, NJ; unio daljnje korekture NČ 2019-09-05
%\section*{O autoru}


\section*{O tekstu}

U korpusu od 21 sačuvanog Izokratova govora, \textit{Panegirik} zauzima posebno mjesto, kao iznimno uspješan početak autorove političko-književne publicistike (Aristotel ga na više mjesta u \textit{Retorici} uzima za primjer vrsnog govora). Djelo, na kojem je godinama radio, Izokrat je objavio 380.\ pr.~Kr. Formalno, to je govor pred svečanom panhelenskom skupštinom \textgreek[variant=ancient]{(πανήγυρις);} takve su govore u Olimpiji ranije održali i Gorgija i Lizija. Izokratov vjerojatno nije bio zaista izveden, već se širio samo u pisanom obliku. 

\textit{Panegirik} iznosi politički program, zahtjev za mir među Grcima i poziv na rat protiv Perzijanaca; oboje je moguće samo ako se dominantna Sparta sporazumije s tada slabijom Atenom. Zahtjev da Atena sudjeluje u vodstvu Grčke Izokrat argumentira dugom pohvalom svojeg rodnog grada, dokazujući kako porijeklom i zaslugama nadmašuje Spartu; jedna od zasluga Atene je i njegovanje lijepih umijeća i filozofije, o čemu se govori u odlomku koji slijedi.

\newpage

\section*{Pročitajte naglas grčki tekst.}

Isocr.\ Panegyricus 47

%Naslov prema izdanju

\medskip

\begin{greek}
{\large
{ \noindent Φιλοσοφίαν τοίνυν, ἣ πάντα ταῦτα συνεξεῦρε καὶ συγκατεσκεύασεν καὶ πρός τε τὰς πράξεις ἡμᾶς ἐπαίδευσεν καὶ πρὸς ἀλλήλους ἐπράϋνε καὶ τῶν συμφορῶν τάς τε δι' ἀμαθίαν καὶ τὰς ἐξ ἀνάγκης γιγνομένας διεῖλεν καὶ τὰς μὲν φυλάξασθαι, τὰς δὲ καλῶς ἐνεγκεῖν ἐδίδαξεν, ἡ πόλις ἡμῶν κατέδειξεν, καὶ λόγους ἐτίμησεν, ὧν πάντες μὲν ἐπιθυμοῦσιν, τοῖς δ' ἐπισταμένοις φθονοῦσιν, συνειδυῖα μὲν ὅτι τοῦτο μόνον ἐξ ἁπάντων τῶν ζῴων ἴδιον ἔφυμεν ἔχοντες καὶ διότι τούτῳ πλεονεκτήσαντες καὶ τοῖς ἄλλοις ἅπασιν αὐτῶν διηνέγκαμεν, ὁρῶσα δὲ περὶ μὲν τὰς ἄλλας πράξεις οὕτω ταραχώδεις οὔσας τὰς τύχας ὥστε πολλάκις ἐν αὐταῖς καὶ τοὺς φρονίμους ἀτυχεῖν καὶ τοὺς ἀνοήτους κατορθοῦν, τῶν δὲ λόγων τῶν καλῶς καὶ τεχνικῶς ἐχόντων οὐ μετὸν τοῖς φαύλοις, ἀλλὰ ψυχῆς εὖ φρονούσης ἔργον ὄντας, καὶ τούς τε σοφοὺς καὶ τοὺς ἀμαθεῖς δοκοῦντας εἶναι ταύτῃ πλεῖστον ἀλλήλων διαφέροντας, ἔτι δὲ τοὺς εὐθὺς ἐξ ἀρχῆς ἐλευθέρως τεθραμμένους ἐκ μὲν ἀνδρίας καὶ πλούτου καὶ τῶν τοιούτων ἀγαθῶν οὐ γιγνωσκομένους, ἐκ δὲ τῶν λεγομένων μάλιστα καταφανεῖς γιγνομένους, καὶ τοῦτο σύμβολον τῆς παιδεύσεως ἡμῶν ἑκάστου πιστότατον ἀποδεδειγμένον, καὶ τοὺς λόγῳ καλῶς χρωμένους οὐ μόνον ἐν ταῖς αὑτῶν δυναμένους, ἀλλὰ καὶ παρὰ τοῖς ἄλλοις ἐντίμους ὄντας.

}
}
\end{greek}

