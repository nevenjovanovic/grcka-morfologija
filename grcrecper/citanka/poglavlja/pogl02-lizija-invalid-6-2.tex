% Unio korekture NZ, NJ 2019-08-08
%\section*{O autoru}



\section*{O tekstu}

Atenski građanin koji bi bio radno nesposoban, ili se ne bi mogao uzdržavati od svog rada, dobivao je od države malu pripomoć; prema ovom govoru, jedan obol na dan (drugi navode iznos od dva ili pet obola). Takva bi penzija bila isplaćivana sve dok je netko ne bi osporio, prigovarajući da primatelj ima dovoljno snage ili sredstava, da nije u dostatnoj mjeri nesposoban za rad, da poznaje zanat kojim bi se mogao uzdržavati. Optuženik se branio pred nadležnim Vijećem pet stotina.

U ovome Lizijinu govoru optuženik kratko predstavlja sebe u povoljnom, a tužitelja u nepovoljnom svjetlu, i potom dokazuje da mu obitelj nema imovine, da je njegova invalidnost teška, da mu zanat nije dovoljno unosan za preživljavanje.

Kritičari su govor smatrali uzornim jer u njemu, bez obzira na svakodnevnost teme, ima ``i plemenitosti i dostojanstva, i oštroumlja, i emocija, i delikatne ironije; ton je ozbiljan, ali ne pedantan; prisan, ali ne banalan; zabavan, ali ne lakrdijaški.''

%\newpage

\section*{Pročitajte naglas grčki tekst.}

Lys.\ Ὑπὲρ τοῦ ἀδυνάτου 6.2
%Naslov prema izdanju

\medskip

{\large
\begin{greek}
\noindent Ἐμοὶ γὰρ ὁ μὲν πατὴρ κατέλιπεν οὐδέν, τὴν δὲ μητέρα τελευτήσασαν πέπαυμαι τρέφων τρίτον ἔτος τουτί, παῖδες δέ μοι οὔπω εἰσὶν οἵ με θεραπεύσουσι. τέχνην δὲ κέκτημαι βραχέα δυναμένην ὠφελεῖν, ἣν αὐτὸς μὲν ἤδη χαλεπῶς ἐργάζομαι, τὸν διαδεξόμενον δ' αὐτὴν οὔπω δύναμαι κτήσασθαι. πρόσοδος δέ μοι οὐκ ἔστιν ἄλλη πλὴν ταύτης, ἣν ἐὰν ἀφέλησθέ με, κινδυνεύσαιμ' ἂν ὑπὸ τῇ δυσχερεστάτῃ γενέσθαι τύχῃ. μὴ τοίνυν, ἐπειδή γε ἔστιν, ὦ βουλή, σῶσαί με δικαίως, ἀπολέσητε ἀδίκως· μηδὲ ἃ νεωτέρῳ καὶ μᾶλλον ἐρρωμένῳ ὄντι ἔδοτε, πρεσβύτερον καὶ ἀσθενέστερον γιγνόμενον ἀφέλησθε· μηδὲ πρότερον καὶ περὶ τοὺς οὐδὲν ἔχοντας κακὸν ἐλεημονέστατοι δοκοῦντες εἶναι νυνὶ διὰ τοῦτον τοὺς καὶ τοῖς ἐχθροῖς ἐλεινοὺς ὄντας ἀγρίως ἀποδέξησθε· μηδ' ἐμὲ τολμήσαντες ἀδικῆσαι καὶ τοὺς ἄλλους τοὺς ὁμοίως ἐμοὶ διακειμένους ἀθυμῆσαι ποιήσητε. καὶ γὰρ ἂν ἄτοπον εἴη, ὦ βουλή, εἰ ὅτε μὲν ἁπλῆ μοι ἦν ἡ συμφορά, τότε μὲν φαινοίμην λαμβάνων τὸ ἀργύριον τοῦτο, νῦν δ' ἐπειδὴ καὶ γῆρας καὶ νόσοι καὶ τὰ τούτοις ἑπόμενα κακὰ προσγίγνεταί μοι, τότε ἀφαιρεθείην.

\end{greek}
}
