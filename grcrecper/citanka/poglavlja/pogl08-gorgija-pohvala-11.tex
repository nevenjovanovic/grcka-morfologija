% Unesene korekture NČ, NZ, NJ
\section*{O autoru}

Gorgija \textgreek[variant=ancient]{(Γοργίας),} grčki retor iz Leontina na Siciliji (oko 485.\ – oko 380.\ pr.~Kr), uz Protagoru najugledniji sofist. Kao poslanik Sirakuze 427.\ pr.~Kr.\ došao je u Atenu, gdje je osnovao i vodio školu govorništva; među njegovim učenicima bio je i Izokrat. Umro je u dubokoj starosti, u Tesaliji, na dvoru tiranina Jazona iz Fere.

Gorgija, veliki inovator retoričke prakse, svojim je artificijelnim i intelektualističkim stilom začetnik antičke umjetničke proze. Naglašena ritma, rima, asonancija, simetričnih članaka \textgreek[variant=ancient]{(κῶλα),} bogati antitezama, paralelizmima, metaforama i igrama riječi, njegovi tekstovi stoje između proze i poezije. U skladu sa sofističkim učenjima, Gorgija demonstrira neograničenu moć riječi, čak i onkraj racionalnog uvjeravanja.

\section*{O tekstu}

Uz nekoliko ulomaka, sačuvana su samo dva Gorgijina djela: \textit{Pohvala Helene} i \textit{Obrana Palameda}. U \textit{Pohvali Helene} Gorgija preuzima temu kojom se već bavio Stezihor (VII./VI.~st.\ pr.~Kr) u \textit{Palinodiji} (\textit{Opozivnoj pjesmi}), te dokazuje nevinost mitske ljepotice: njezin su odlazak u Troju prouzročili sudbina, bogovi, Parisove zavodljive riječi.

%\newpage

\section*{Pročitajte naglas grčki tekst.}
Gorg.\ Helenae encomium Fr.~11.~13
%Naslov prema izdanju

\medskip

{\large
\begin{greek}
\noindent Ὅτι μὲν οὖν φύσει καὶ γένει τὰ πρῶτα τῶν πρώτων ἀνδρῶν καὶ γυναικῶν ἡ γυνὴ περὶ ἧς ὅδε ὁ λόγος, οὐκ ἄδηλον οὐδὲ ὀλίγοις. δῆλον γὰρ ὡς μητρὸς μὲν Λήδας, πατρὸς δὲ τοῦ μὲν γενομένου θεοῦ, λεγομένου δὲ θνητοῦ, Τυνδάρεω καὶ Διός, ὧν ὁ μὲν διὰ τὸ εἶναι ἔδοξεν, ὁ δὲ διὰ τὸ φάναι ἠλέγχθη, καὶ ἦν ὁ μὲν ἀνδρῶν κράτιστος ὁ δὲ πάντων τύραννος. 

ἐκ τοιούτων δὲ γενομένη ἔσχε τὸ ἰσόθεον κάλλος, ὃ λαβοῦσα καὶ οὐ λαθοῦσα ἔσχε· πλείστας δὲ πλείστοις ἐπιθυμίας ἔρωτος ἐνειργάσατο, ἑνὶ δὲ σώματι πολλὰ σώματα συνήγαγεν ἀνδρῶν ἐπὶ μεγάλοις μέγα φρονούντων, ὧν οἱ μὲν πλούτου μεγέθη, οἱ δὲ εὐγενείας παλαιᾶς εὐδοξίαν, οἱ δὲ ἀλκῆς ἰδίας εὐεξίαν, οἱ δὲ σοφίας ἐπικτήτου δύναμιν ἔσχον· καὶ ἧκον ἅπαντες ὑπ' ἔρωτός τε φιλονίκου φιλοτιμίας τε ἀνικήτου. ὅστις μὲν οὖν καὶ δι' ὅτι καὶ ὅπως ἀπέπλησε τὸν ἔρωτα τὴν ῾Ελένην λαβών, οὐ λέξω· τὸ γὰρ τοῖς εἰδόσιν ἃ ἴσασι λέγειν πίστιν μὲν ἔχει, τέρψιν δὲ οὐ φέρει. τὸν χρόνον δὲ τῶι λόγωι τὸν τότε νῦν ὑπερβὰς ἐπὶ τὴν ἀρχὴν τοῦ μέλλοντος λόγου προβήσομαι, καὶ προθήσομαι τὰς αἰτίας, δι' ἃς εἰκὸς ἦν γενέσθαι τὸν τῆς ῾Ελένης εἰς τὴν Τροίαν στόλον.

\end{greek}

}

