% Unesi korekture NČ 2019-09-07
%\section*{O autoru}

\section*{O tekstu}

U ovoj Ezopovoj basni susrećemo lisicu u nevolji i jarca. Valjano promislivši o poteškoći koja ju je zadesila, lisica se spašava, dok jarac zbog nepromišljenosti u nevolju upada. Više je pouka basne: prije nego što odlučimo započeti nešto, razmislimo dobro o tome kako ćemo to završiti i hoćemo li u tome biti uspješni. Nadalje, ako nismo razboriti i ne pazimo te poslušamo li savjet loše osobe, lako se može dogoditi da završimo u velikoj nevolji. Mudar čovjek u postupcima ne sluša osjetila, već se oslanja na svoj razum, promišljajući kako da najbolje postupi.

%\newpage

\section*{Pročitajte naglas grčki tekst.}

Aesop.\ Fabulae 9

%Naslov prema izdanju

\medskip

\begin{greek}
{\large
{ \noindent ΑΛΩΠΗΞ ΚΑΙ ΤΡΑΓΟΣ 

\medskip

\noindent Ἀλώπηξ πεσοῦσα εἰς φρέαρ ἐπάναγκες ἔμενε πρὸς τὴν ἀνάβασιν ἀμηχανοῦσα. τράγος δὲ δίψῃ συνεχόμενος ὡς  ἐγένετο κατὰ τὸ αὐτὸ φρέαρ, θεασάμενος αὐτὴν ἐπυνθάνετο, εἰ καλὸν εἴη τὸ ὕδωρ. ἡ δὲ τὴν συντυχίαν ἀσμενισαμένη πολὺν ἔπαινον τοῦ ὕδατος κατέτεινε λέγουσα ὡς χρηστὸν εἴη καὶ δὴ καὶ αὐτὸν καταβῆναι παρῄνει. τοῦ δὲ ἀμελετήτως καθαλλομένου διὰ τὸ μόνην ὁρᾶν τότε τὴν ἐπιθυμίαν καὶ ἅμα τῷ τὴν δίψαν σβέσαι ἀναδῦναι μετὰ τῆς ἀλώπεκος σκοποῦντος χρήσιμόν τι ἡ ἀλώπηξ ἔφη ἐπινενοηκέναι εἰς τὴν ἀμφοτέρων σωτηρίαν. 

\noindent ``ἐὰν γὰρ θελήσῃς τοὺς ἐμπροσθίους πόδας τῷ τοίχῳ προσερείσας ἐγκλῖναι καὶ τὰ κέρατα, ἀναδραμοῦσα αὐτὴ διὰ τοῦ σοῦ νώτου καὶ σὲ ἀνασπάσω.'' τοῦ δὲ καὶ πρὸς τὴν δευτέραν παραίνεσιν ἑτοίμως ὑπηρετήσαντος ἡ ἀλώπηξ ἀναλλομένη διὰ τῶν σκελῶν αὐτοῦ ἐπὶ τὸν νῶτον ἀνέβη καὶ ἀπ' ἐκείνου ἐπὶ τὰ κέρατα διερεισαμένη ἐπὶ τὸ στόμα τοῦ φρέατος ηὑρέθη καὶ ἀνελθοῦσα ἀπηλλάττετο. 

\noindent τοῦ δὲ τράγου μεμφομένου αὐτὴν ὡς τὰς ὁμολογίας παραβαίνουσαν ἡ ἀλώπηξ ἐπιστραφεῖσα εἶπεν· ``ὦ οὗτος, ἀλλ' εἰ τοσαύτας φρένας εἶχες, ὅσας ἐν τῷ πώγωνι τρίχας, οὐ πρότερον ἂν καταβεβήκεις πρὶν ἢ τὴν ἄνοδον ἐσκέψω.''

}
}
\end{greek}

