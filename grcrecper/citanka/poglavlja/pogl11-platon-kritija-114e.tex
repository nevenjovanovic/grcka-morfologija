% redigirao NZ, unio NJ 10. 5. 2019.
%\section*{O autoru}



\section*{O tekstu}

\textit{Kritija} se ubraja među Platonove kasne dijaloge, iz razdoblja nakon filozofova drugog puta na Siciliju (367). Dijalog je trebao biti središnji dio trilogije kojem bi prethodio \textit{Timej}, a za njim bi slijedio \textit{Hermokrat}, koji vjerojatno nikad nije napisan; i \textit{Kritija} je ostao nedovršen. 

Uz Sokrata, u razgovoru sudjeluju Atenjanin Kritija, pripadnik zloglasne tridesetorice tirana, zatim Hermokrat, vođa oligarhijske stranke u Sirakuzi, te filozof Timej iz Lokra u Italiji, naslovni lik dijaloga \textit{Timej}.  \textit{Kritija} je posvećen slavi drevne Atene, njezinih građana, državnog ustroja i postignuća i po tome je sličan dijalogu \textit{Meneksen}. Dok u \textit{Meneksenu} Sokrat u pohvalnom govoru slavi Atenjane jer su odbili pokušaj Perzijanaca da pokore Grčku, ovdje drevna Atena, zamišljena kao idealna država, u borbi za opstanak pobjeđuje moćno svjetsko carstvo Atlantidu. Ta je sila prijetila da će pokoriti cijelu Europu i Aziju, no, udaljivši se od bogova, iznenada je potonula u ocean, i to prije devet tisuća godina. 

Platon, tvorac mita o Atlantidi, nije mogao ni slutiti da će ljudi tisućljećima kasnije, s istom revnošću s kojom na mapama upisuju putanju Odisejevih putovanja, i dalje tragati za potonulim divovskim otokom s one strane Heraklovih stupova. 

Nakon što je rečeno da je moćna Atlantida mogla računati na obilje dobara iz inozemstva, u odabranom se odlomku opisuju prirodni resursi toga otoka.

\newpage

\section*{Pročitajte naglas grčki tekst.}

%Naslov prema izdanju

Plat.\ Critias 114e

\medskip

{\large
\begin{greek}
\noindent Πλεῖστα δὲ ἡ νῆσος αὐτὴ παρείχετο εἰς τὰς τοῦ βίου κατασκευάς, πρῶτον μὲν ὅσα ὑπὸ μεταλλείας ὀρυττόμενα στερεὰ καὶ ὅσα τηκτὰ γέγονε, καὶ τὸ νῦν ὀνομαζόμενον μόνον — τότε δὲ πλέον ὀνόματος ἦν τὸ γένος ἐκ γῆς ὀρυττόμενον ὀρειχάλκου κατὰ τόπους πολλοὺς τῆς νήσου, πλὴν χρυσοῦ τιμιώτατον ἐν τοῖς τότε ὄν — καὶ ὅσα ὕλη πρὸς τὰ τεκτόνων διαπονήματα παρέχεται, πάντα φέρουσα ἄφθονα, τά τε αὖ περὶ τὰ ζῷα ἱκανῶς ἥμερα καὶ ἄγρια τρέφουσα. καὶ δὴ καὶ ἐλεφάντων ἦν ἐν αὐτῇ γένος πλεῖστον· νομὴ γὰρ τοῖς τε ἄλλοις ζῴοις, ὅσα καθ' ἕλη καὶ λίμνας καὶ ποταμούς, ὅσα τ' αὖ κατ' ὄρη καὶ ὅσα ἐν τοῖς πεδίοις νέμεται, σύμπασιν παρῆν ἅδην, καὶ τούτῳ κατὰ ταὐτὰ τῷ ζῴῳ, μεγίστῳ πεφυκότι καὶ πολυβορωτάτῳ. πρὸς δὲ τούτοις, ὅσα εὐώδη τρέφει που γῆ τὰ νῦν, ῥιζῶν ἢ χλόης ἢ ξύλων ἢ χυλῶν στακτῶν εἴτε ἀνθῶν ἢ καρπῶν, ἔφερέν τε ταῦτα καὶ ἔτρεφεν εὖ.
\end{greek}

}

