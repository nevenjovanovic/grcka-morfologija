% Redaktura NZ
%\section*{O autoru}



\section*{O tekstu}

\textit{Fedon}, Φαίδων, pripada dijalozima zrelog ili srednjeg razdoblja (oko 387.–365. pr. Kr.) atenskog filozofa Platona (427.–347. pr. Kr.). U tom razdoblju autor razvija učenje o idejama, ἰδέαι ili εἴδη. \textit{Fedon} je četvrti, i posljednji, dijalog s temom posljednjeg dana u životu Platonova učitelja Sokrata (469.–399.), kojeg je atenska država osudila na smrt zbog bezbožnosti i kvarenja mladeži. Iz perspektive Fedona iz Elide, još jednog od Sokratovih učenika, djelo izvještava o Sokratovoj raspravi s prijateljima o besmrtnosti, a završava eshatološkim mitom o zagrobnom životu te opisom Sokratove smrti. 

U ovome odlomku, pošto je izložio suprotnost osjetilnog svijeta i svijeta ideja, te tijela i duše, Sokrat govori što će se nakon smrti dogoditi s čistim dušama; za razliku od nečistih, koje će se vratiti u različite vrste životinja, prema stanju svojih vrlina, filozofi će se pridružiti bogovima. Nakon duže šutnje, Sokrat sugerira sugovornicima, Simiji i Kebetu, da ne brinu zbog prigovora na njegove teze; jer on je, poput labuda, u službi Apolona, i može proreći što će se dogoditi nakon smrti.

\newpage

\section*{Pročitajte naglas grčki tekst.}

%Naslov prema izdanju
Plat.\ Phaedo 85a

\medskip

{\large
\begin{greek}
\noindent Ἐπειδὰν οἱ κύκνοι αἴσθωνται ὅτι δεῖ αὐτοὺς ἀποθανεῖν, ᾄδοντες καὶ ἐν τῷ πρόσθεν χρόνῳ, τότε δὴ πλεῖστα καὶ κάλλιστα ᾄδουσι, γεγηθότες ὅτι μέλλουσι παρὰ τὸν θεὸν ἀπιέναι οὗπέρ εἰσι θεράποντες. οἱ δ' ἄνθρωποι διὰ τὸ αὑτῶν δέος τοῦ θανάτου καὶ τῶν κύκνων καταψεύδονται, καί φασιν αὐτοὺς θρηνοῦντας τὸν θάνατον ὑπὸ λύπης ἐξᾴδειν, καὶ οὐ λογίζονται ὅτι οὐδὲν ὄρνεον ᾄδει ὅταν πεινῇ ἢ ῥιγῷ ἤ τινα ἄλλην λύπην λυπῆται, οὐδὲ αὐτὴ ἥ τε ἀηδὼν καὶ χελιδὼν καὶ ὁ ἔποψ, ἃ δή φασι διὰ λύπην θρηνοῦντα ᾄδειν. ἀλλ' οὔτε ταῦτά μοι φαίνεται λυπούμενα ᾄδειν οὔτε οἱ κύκνοι, ἀλλ' ἅτε οἶμαι τοῦ ᾿Απόλλωνος ὄντες, μαντικοί τέ εἰσι καὶ προειδότες τὰ ἐν ῞Αιδου ἀγαθὰ ᾄδουσι καὶ τέρπονται ἐκείνην τὴν ἡμέραν διαφερόντως ἢ ἐν τῷ ἔμπροσθεν χρόνῳ. ἐγὼ δὲ καὶ αὐτὸς ἡγοῦμαι ὁμόδουλός τε εἶναι τῶν κύκνων καὶ ἱερὸς τοῦ αὐτοῦ θεοῦ, καὶ οὐ χεῖρον ἐκείνων τὴν μαντικὴν ἔχειν παρὰ τοῦ δεσπότου,  οὐδὲ δυσθυμότερον αὐτῶν τοῦ βίου ἀπαλλάττεσθαι.

\end{greek}

}

