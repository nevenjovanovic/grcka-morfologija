% Redaktura NJ; unio korekture NZ 2019-08-10
%\section*{O autoru}



\section*{O tekstu}

Oponašajući Ikara, filozof Menip izrađuje krila i leti u nebo do bogova, gdje saznaje da je Zeus odlučio uništiti sve filozofe jer su beskorisni. U izabranom odlomku Zeus u Menipovu društvu sjeda uz svojevrsne prozorske otvore kroz koje sa zemlje pristižu ljudske molbe upućene bogovima, da ih sasluša. Dio molbi, one pravedne, Zeus pripušta na nebo, dok one bezbožne i neispunjive otpuhuje ne dozvoljavajući da se nebu uopće približe.

%\newpage

\section*{Pročitajte naglas grčki tekst.}

%Naslov prema izdanju
Luc.\ Icaromenippus 25.3

\medskip

{\large
\begin{greek}
\noindent Θυρίδες δὲ ἦσαν ἑξῆς τοῖς στομίοις τῶν φρεάτων ἐοικυῖαι πώματα ἔχουσαι, καὶ παρ' ἑκάστῃ θρόνος ἔκειτο χρυσοῦς. καθίσας οὖν ἑαυτὸν ἐπὶ τῆς πρώτης ὁ Ζεὺς καὶ ἀφελὼν τὸ πῶμα παρεῖχε τοῖς εὐχομένοις ἑαυτόν· εὔχοντο δὲ πανταχόθεν τῆς γῆς διάφορα καὶ ποικίλα. συμπαρακύψας γὰρ καὶ αὐτὸς ἐπήκουον ἅμα τῶν εὐχῶν. ἦσαν δὲ τοιαίδε, ``῏Ω Ζεῦ, βασιλεῦσαί μοι γένοιτο·'' ``῏Ω Ζεῦ, τὰ κρόμμυά μοι φῦναι καὶ τὰ σκόροδα·'' ``῏Ω θεοί, τὸν πατέρα μοι ταχέως ἀποθανεῖν·'' ὁ δέ τις ἂν ἔφη, ``Εἴθε κληρονομήσαιμι τῆς γυναικός,'' ``Εἴθε λάθοιμι ἐπιβουλεύσας τῷ ἀδελφῷ,'' ``Γένοιτό μοι νικῆσαι τὴν δίκην,'' ``Δὸς στεφθῆναι τὰ ᾿Ολύμπια.'' τῶν πλεόντων δὲ ὁ μὲν βορέαν εὔχετο ἐπιπνεῦσαι, ὁ δὲ νότον, ὁ δὲ γεωργὸς ᾔτει ὑετόν, ὁ δὲ γναφεὺς ἥλιον. Ἐπακούων δὲ ὁ Ζεὺς καὶ τὴν εὐχὴν ἑκάστην ἀκριβῶς ἐξετάζων οὐ πάντα ὑπισχνεῖτο, ἀλλ' ἕτερον μὲν ἔδωκε πατήρ, ἕτερον δ' ἀνένευσε· τὰς μὲν γὰρ δικαίας τῶν εὐχῶν προσίετο ἄνω διὰ τοῦ στομίου καὶ ἐπὶ τὰ δεξιὰ κατετίθει φέρων, τὰς δὲ ἀνοσίους ἀπράκτους αὖθις ἀπέπεμπεν ἀποφυσῶν κάτω, ἵνα μηδὲ πλησίον γένοιντο τοῦ οὐρανοῦ. 
\end{greek}

}

