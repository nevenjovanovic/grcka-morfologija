%\section*{O autoru}
% pregledao i redigirao NJ, 14. 4. 2019. NZ (unio NJ 10. 5. 2019.)


\section*{O tekstu}

Ovaj je govor, deveti od 21 koji su stigli do nas, Izokratova nadgrobna pohvala ciparskog kralja Euagore (411–374), oca kralja Nikokla (njemu je pak Izokrat posvetio svoj drugi i treći govor). U odabranom odlomku Izokrat uspoređuje pjesnike i govornike, ističući da je pjesničkim izrazom mnogo lakše pridobiti  naklonost slušatelja nego uobičajenim sredstvima koje govorniku na raspolaganje stavlja žanr javnog govora.

%\newpage

\section*{Pročitajte naglas grčki tekst.}

%Naslov prema izdanju

Isocr.\ Evagoras 9

\medskip

{\large
\begin{greek}
\noindent Τοῖς μὲν γὰρ ποιηταῖς πολλοὶ δέδονται κόσμοι· καὶ γὰρ πλησιάζοντας τοὺς θεοὺς τοῖς ἀνθρώποις οἷόν τ' αὐτοῖς ποιῆσαι καὶ διαλεγομένους καὶ συναγωνιζομένους οἷς ἂν βουληθῶσιν, καὶ περὶ τούτων δηλῶσαι μὴ μόνον τοῖς τεταγμένοις ὀνόμασιν, ἀλλὰ τὰ μὲν ξένοις, τὰ δὲ καινοῖς, τὰ δὲ μεταφοραῖς, καὶ μηδὲν παραλιπεῖν, ἀλλὰ πᾶσιν τοῖς εἴδεσιν διαποικῖλαι τὴν ποίησιν· τοῖς δὲ περὶ τοὺς λόγους οὐδὲν ἔξεστιν τῶν τοιούτων, ἀλλ' ἀποτόμως καὶ τῶν ὀνομάτων τοῖς πολιτικοῖς μόνον καὶ τῶν ἐνθυμημάτων τοῖς περὶ αὐτὰς τὰς πράξεις ἀναγκαῖόν ἐστιν χρῆσθαι. Πρὸς δὲ τούτοις οἱ μὲν μετὰ μέτρων καὶ ῥυθμῶν ἅπαντα ποιοῦσιν, οἱ δ' οὐδενὸς τούτων κοινωνοῦσιν· ἃ τοσαύτην ἔχει χάριν ὥστ', ἂν καὶ τῇ λέξει καὶ τοῖς ἐνθυμήμασιν ἔχῃ κακῶς, ὅμως αὐταῖς ταῖς εὐρυθμίαις καὶ ταῖς συμμετρίαις ψυχαγωγοῦσιν τοὺς ἀκούοντας.

\end{greek}

}

