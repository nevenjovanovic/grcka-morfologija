\section*{O autoru}

Plutarh (Πλούταρχος, Heroneja 46.–125.) bio je povjesničar, biograf i filozof. Smatra se posljednjim predstavnikom opće grčke obrazovanosti. Studirao je filozofiju, matematiku i govorništvo na atenskoj Akademiji. U Rimu je, pod zaštitom careva Trajana i Hadrijana, poučavao filozofiju, a potkraj života bio je Apolonov svećenik u rodnoj Heroneji. 

U filozofiji je nastojao pomiriti platonovske i aristotelovske teze. Pod imenom \textit{Moralia} (Ἠϑικά, zbirka od 78 eseja i govora) sačuvale su se njegove popularno-odgojne rasprave filozofskoga, religijskoga, političkoga, prirodoslovnoga, književnog i drugog sadržaja. Najpoznatiji je po \textit{Usporednim životopisima} \textgreek[variant=ancient]{(Βίοι παράλληλοι),} u kojima u parovima donosi biografije jednoga Grka i jednog Rimljanina (npr.\ Tezej i Romul, Likurg i Numa Pompilije) kako bi se istaknule njihove moralne vrijednosti i mane. Sačuvana su dvadeset tri para biografija, kao i četiri samostalne biografije. Tim izvanredno živo, anegdotski pisanim životopisima (Koriolan, Timon Atenjanin, Julije Cezar, Antonije, Kleopatra) služio se i Shakespeare u svojim dramama.

\section*{O tekstu}

Rasprava \textit{O brbljavosti} \textgreek[variant=ancient]{(Περὶ ἀδολεσχίας,} \textit{De garrulitate)} pripada zbirci \textit{Moralia}. Plutarh brbljavost prikazuje kao bolest za koju lijek pruža filozofija. U izabranom odlomku autor predstavlja tri načina odgovaranja na pitanja, na primjeru upita je li Sokrat kod kuće.

\newpage

\section*{Pročitajte naglas grčki tekst.}

%Naslov prema izdanju

Plut.\ De garrulitate 513A

\medskip

{\large
\begin{greek}
\noindent  Ἔστι τοίνυν τρία γένη τῶν πρὸς τὰς ἐρωτήσεις ἀποκρίσεων, τὸ μὲν ἀναγκαῖον τὸ δὲ φιλάνθρωπον τὸ δὲ περισσόν. οἷον πυθομένου τινὸς εἰ Σωκράτης ἔνδον, ὁ μὲν ὥσπερ ἄκων καὶ ἀπροθύμως ἀποκρίνεται τὸ ‘οὐκ ἔνδον’, ἐὰν δὲ βούληται λακωνίζειν, καὶ τὸ ‘ἔνδον’ ἀφελὼν αὐτὴν μόνην φθέγξεται τὴν ἀπόφασιν· ὡς ἐκεῖνοι, Φιλίππου γράψαντος εἰ δέχονται τῇ πόλει αὐτόν, εἰς χάρτην ‘οὐ’ μέγα γράψαντες ἀπέστειλαν. ὁ δὲ φιλανθρωπότερον ἀποκρίνεται ‘οὐκ ἔνδον ἀλλ' ἐπὶ ταῖς τραπέζαις’, κἂν βούληται προσεπιμετρῆσαι, ‘ξένους τινὰς ἐκεῖ περιμένων.’ ὁ δὲ περιττὸς καὶ ἀδολέσχης, ἄν γε δὴ τύχῃ καὶ τὸν Κολοφώνιον ἀνεγνωκὼς Ἀντίμαχον, ‘οὐκ ἔνδον’ φησίν ‘ἀλλ' ἐπὶ ταῖς τραπέζαις, ξένους ἀναμένων  Ἴωνας, ὑπὲρ ὧν αὐτῷ γέγραφεν Ἀλκιβιάδης περὶ Μίλητον ὢν καὶ παρὰ Τισσαφέρνῃ διατρίβων, τῷ τοῦ μεγάλου σατράπῃ βασιλέως, ὃς πάλαι μὲν ἐβοήθει Λακεδαιμονίοις, νῦν δὲ προστίθεται δι' Ἀλκιβιάδην Ἀθηναίοις· ὁ γὰρ Ἀλκιβιάδης ἐπιθυμῶν κατελθεῖν εἰς τὴν πατρίδα τὸν Τισσαφέρνην μετατίθησι’.

\end{greek}
}

