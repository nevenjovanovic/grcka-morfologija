%Unio korekture NČ, 2019-08-13
\section*{O autoru}

Neoplatonizam je posljednji veliki sustav grčke filozofije, pokušaj sinteze čitave grčke filozofske tradicije i novijih misaonih tendencija. Glavni je predstavnik ovog sustava Plotin \textgreek[variant=ancient]{(Πλωτῖνος,} oko 205. – oko 270.). Rođen u egipatskom Likopolu, studirao je filozofiju u Aleksandriji, u školi Amonija Sake. Zanimao se za perzijsku i indijsku filozofiju, i zato se pridružio neuspješnom pohodu cara Gordijana III.\ na Perziju, 243. Osnovao je školu u Rimu pod pokroviteljstvom cara Galijena (vladao 253.–268.). Planirao je čak i osnivanje Platonopolisa, zajednice po Platonovim načelima, u Kampaniji; to se nije ostvarilo.

\section*{O tekstu}

Plotinove je spise učenik Porfirije \textgreek[variant=ancient]{(Πορφύριος)} prikupio u Eneade \textgreek[variant=ancient]{(ἐννέα} ``devet''), šest svezaka po devet spisa, organizirajući ih po temama: etičkim, prirodno-filozofskim, o duši \textgreek[variant=ancient]{(ψυχή,} četvrta eneada), o duhu \textgreek[variant=ancient]{(νοῦς,} peta) i o Jednome \textgreek[variant=ancient]{(ἕν,} šesta). Duša, duh i Jedno pripadaju gornjem svijetu, koji se može spoznati samo mišljenjem – pri čemu Jedno, posve savršena svjetska duša, nije spoznatljiva čak ni na taj način, već samo kroz svojevrsno kontemplativno izlaženje iz sebe sama; to je ἔκστασις, koja vodi do ἕνωσις (\textit{unio mystica}, ujedinjenje) s najvišom razinom postojanja.

U ovdje odabranim odlomcima Plotin opisuje kako ljubav prema Jednom \textgreek[variant=ancient]{(μόνον),} onome koje nema oblik, utječe na dušu; pod tim utjecajem i sama se duša oslobađa oblika. Saznajemo i kako se duša osjeća \textgreek[variant=ancient]{ἐκεῖ} (``ondje''), u svijetu iznad samog neba.

\section*{Pročitajte naglas grčki tekst.}

Plot.\ Enneades 6.7.34

%Naslov prema izdanju

\medskip


{\large
{ \noindent I

\begin{greek}

\noindent  Ἡ ψυχή, ὅταν αὐτοῦ ἔρωτα σύντονον λάβῃ, ἀποτίθεται πᾶσαν ἣν ἔχει μορφήν, καὶ ἥτις ἂν καὶ νοητοῦ ᾖ ἐν αὐτῇ. Οὐ γάρ ἐστιν ἔχοντά τι ἄλλο καὶ ἐνεργοῦντα περὶ αὐτὸ οὔτε ἰδεῖν οὔτε ἐναρμοσθῆναι. Ἀλλὰ δεῖ μήτε κακὸν μήτ' αὖ ἀγαθὸν μηδὲν ἄλλο πρόχειρον ἔχειν, ἵνα δέξηται μόνη μόνον. Ὅταν δὲ τούτου εὐτυχήσῃ ἡ ψυχὴ καὶ ἥκῃ πρὸς αὐτήν, μᾶλλον δὲ παρὸν φανῇ, ὅταν ἐκείνη ἐκνεύσῃ τῶν παρόντων καὶ παρασκευάσασα αὑτὴν ὡς ὅτι μάλιστα καλὴν καὶ εἰς ὁμοιότητα ἐλθοῦσα — ἡ δὲ παρασκευὴ καὶ ἡ κόσμησις δήλη που τοῖς παρασκευαζομένοις — ἰδοῦσα δὲ ἐν αὐτῇ ἐξαίφνης φανέντα — μεταξὺ γὰρ οὐδὲν οὐδ' ἔτι δύο, ἀλλ' ἓν ἄμφω· οὐ γὰρ ἂν διακρίναις ἔτι, ἕως πάρεστι· μίμησις δὲ τούτου καὶ οἱ ἐνταῦθα ἐρασταὶ καὶ ἐρώμενοι συγκρῖναι θέλοντες — καὶ οὔτε σώματος ἔτι αἰσθάνεται, ὅτι ἐστὶν ἐν αὐτῷ, οὔτε ἑαυτὴν ἄλλο τι λέγει, οὐκ ἄνθρωπον, οὐ ζῷον, οὐκ ὄν, οὐδὲ πᾶν — ἀνώμαλος γὰρ ἡ τούτων πως θέα — καὶ οὐδὲ σχολὴν ἄγει πρὸς αὐτὰ οὔτε θέλει, ἀλλὰ καὶ αὐτὸ ζητήσασα ἐκείνῳ παρόντι ἀπαντᾷ κἀκεῖνο ἀντ' αὐτῆς βλέπει· τίς δὲ οὖσα βλέπει, οὐδὲ τοῦτο σχολάζει ὁρᾶν.

\end{greek}

\noindent II

\begin{greek}

\noindent  Ἔνθα δὴ οὐδὲν πάντων ἀντὶ τούτου ἀλλάξαιτο, οὐδ' εἴ τις αὐτῇ πάντα τὸν οὐρανὸν ἐπιτρέποι, ὡς οὐκ ὄντος ἄλλου ἔτι ἀμείνονος οὐδὲ μᾶλλον ἀγαθοῦ· οὔτε γὰρ ἀνωτέρω τρέχει τά τε ἄλλα πάντα κατιούσης, κἂν ᾖ ἄνω.  Ὥστε τότε ἔχει καὶ τὸ κρίνειν καλῶς καὶ γιγνώσκειν, ὅτι τοῦτό ἐστιν οὗ ἐφίετο, καὶ τίθεσθαι, ὅτι μηδέν ἐστι κρεῖττον αὐτοῦ. Οὐ γάρ ἐστιν ἀπάτη ἐκεῖ· ἢ ποῦ ἂν τοῦ ἀληθοῦς ἀληθέστερον τύχοι;  Ὃ οὖν λέγει, ἐκεῖνό ἐστι, καὶ ὕστερον λέγει, καὶ σιωπῶσα δὲ λέγει καὶ εὐπαθοῦσα οὐ ψεύδεται, ὅτι εὐπαθεῖ· οὐδὲ γαργαλιζομένου λέγει τοῦ σώματος, ἀλλὰ τοῦτο γενομένη, ὃ πάλαι, ὅτε εὐτύχει. Ἀλλὰ καὶ τὰ ἄλλα πάντα, οἷς πρὶν ἥδετο, ἀρχαῖς ἢ δυνάμεσιν ἢ πλούτοις ἢ κάλλεσιν ἢ ἐπιστήμαις, ταῦτα ὑπεριδοῦσα λέγει οὐκ ἂν εἰποῦσα μὴ κρείττοσι συντυχοῦσα τούτων· οὐδὲ φοβεῖται, μή τι πάθῃ, μετ' ἐκείνου οὖσα οὐδ' ὅλως ἰδοῦσα· εἰ δὲ καὶ τὰ ἄλλα τὰ περὶ αὐτὴν φθείροιτο, εὖ μάλα καὶ βούλεται, ἵνα πρὸς τούτῳ ᾖ μόνον· εἰς τόσον ἥκει εὐπαθείας.

\end{greek}

}
}


