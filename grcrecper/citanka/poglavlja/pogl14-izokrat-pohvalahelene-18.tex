%\section*{O autoru}



\section*{O tekstu}

Na početku Izokratova govora \textit{Pohvala Helene} stoji duži uvod u kojem se autor osvrće na pomodnu, sofističku retoriku svojega vremena, zalažući se za govorništvo koje će odabirom vjerodostojnih tema i njihovim ozbiljnim tretmanom biti istinski korisno u izobrazbi mladih za građanske dužnosti. Sam panegirik Heleni kronološki slijedi njezin životopis: od priče o božanskom podrijetlu koje Helena vuče od Zeusa, preko Tezejeve otmice te povratka uz pomoć Kastora i Polideuka, do Parisova suda i otmice; nastavlja se pohvalom Helenine ljepote i stjecanja besmrtnosti, a završava njezinom simboličnom ulogom u ujedinjenju svih Grka i u njihovoj pobjedi nad barbarima. U ovdje odabranom odlomku prikazan je učinak Helenine ljepote na Tezeja, koji ju je, budući da je bila premlada za udaju, oteo i doveo iz Lakedemona u Atiku. Tim odlomkom započinje digresija u pohvalu Tezeja; on metaforički predstavlja veličinu i sjaj grada Atene.

%\newpage

\section*{Pročitajte naglas grčki tekst.}
Isoc. Helenae encomium 18
%Naslov prema izdanju

\medskip

{\large
\begin{greek}
\noindent Θησεὺς, ὁ λεγόμενος μὲν Αἰγέως γενόμενος δ' ἐκ Ποσειδῶνος, ἰδὼν αὐτὴν οὔπω μὲν ἀκμάζουσαν, ἤδη δὲ τῶν ἄλλων διαφέρουσαν, τοσοῦτον ἡττήθη τοῦ κάλλους, ὁ κρατεῖν τῶν ἄλλων εἰθισμένος, ὥσθ' ὑπαρχούσης αὐτῷ καὶ πατρίδος μεγίστης καὶ βασιλείας ἀσφαλεστάτης ἡγησάμενος οὐκ ἄξιον εἶναι ζῆν ἐπὶ τοῖς παροῦσιν ἀγαθοῖς ἄνευ τῆς πρὸς ἐκείνην οἰκειότητος, ἐπειδὴ παρὰ τῶν κυρίων οὐχ οἷός τ' ἦν αὐτὴν λαβεῖν, ἀλλ' ἐπέμενον τήν τε τῆς παιδὸς ἡλικίαν καὶ τὸν χρησμὸν τὸν παρὰ τῆς Πυθίας, ὑπεριδὼν τὴν ἀρχὴν τὴν Τυνδάρεω καὶ καταφρονήσας τῆς ῥώμης τῆς Κάστορος καὶ Πολυδεύκους καὶ πάντων τῶν ἐν Λακεδαίμονι δεινῶν ὀλιγωρήσας, βίᾳ λαβὼν αὐτὴν εἰς Ἄφιδναν τῆς Ἀττικῆς κατέθετο.

\end{greek}

}

