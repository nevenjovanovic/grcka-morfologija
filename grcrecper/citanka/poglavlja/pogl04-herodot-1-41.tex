%Unesi korekture NČ 2019-09-09
\section*{O autoru}

Grčki povjesničar Herodot (Ἡρόδοτος) iz Halikarnasa u Kariji (nakon 490.\ – Atena, nakon 430.\ pr.~Kr.), proputovao je velik dio Grcima poznatog svijeta, bio je prijatelj Perikla i Sofokla, sudjelovao u osnivanju atenske kolonije Turija, u Tarantskom zaljevu u Velikoj Grčkoj (444.\ pr.~Kr); doživio je početak Peloponeskog rata.

U prvome sačuvanom proznom djelu grčke književnosti, opsežnoj \textit{Povijesti} (Ἱστορίαı) o sukobu Grka i Perzijanaca do 479, Herodot je opisao uspon Perzijskog Carstva, pobunu u Joniji i dvije perzijske provale u Grčku, uz mnogo geografskih, etnografskih i mitoloških ekskursa. \textit{Povijest} je, u biti, niz zanimljivih i poučnih priča, često o slavnim i legendarnim likovima. Djelo, sastavljeno jonskim dijalektom, bilo je namijenjeno javnom izvođenju, kao svojevrsna prozna rapsodija; nije jasno je li sam autor oblikovao cjelovito izdanje. 

\textit{Povijest} je do nas stigla podijeljena u devet knjiga, od kojih svaka nosi ime po jednoj Muzi. Podjelu su proveli aleksandrijski filolozi u III.~st.\ pr.~Kr.

\section*{O tekstu}

U prvoj knjizi \textit{Povijesti, Klio,} Herodot pripovijeda o propasti kraljevstva lidijskog kralja Kreza (Κροῖσος; vladao 563.–546.\ pr.~Kr), fantastično bogatog Gigova nasljednika. Perzijski vladar Kir II.\ Veliki (staroperzijski Kūruš, grčki Κῦρος, između 590.\ i 580.\ – 529.\ pr.~Kr) osvojio je Krezovu prijestolnicu Sard i zarobio samog kralja (koji je postao Kirov savjetnik). Prva knjiga \textit{Povijesti} završit će prikazom Kirove smrti. U ovdje donesenom odlomku Kir se priprema osvojiti grčke gradove u Maloj Aziji, u kojima žive Jonjani i Eoljani; ti su gradovi dotad priznavali Krezovu vlast.

%\newpage

\section*{Pročitajte naglas grčki tekst.}

Hdt.\ Historiae 1.141

%Naslov prema izdanju

\medskip

\begin{greek}
{\large
{ \noindent Ἴωνες δὲ καὶ Αἰολέες, ὡς οἱ Λυδοὶ τάχιστα κατεστράφατο ὑπὸ Περσέων, ἔπεμπον ἀγγέλους ἐς Σάρδις παρὰ Κῦρον, ἐθέλοντες ἐπὶ τοῖσι αὐτοῖσι εἶναι τοῖσι καὶ Κροίσῳ ἦσαν κατήκοοι. Ὁ δὲ ἀκούσας αὐτῶν τὰ προΐσχοντο ἔλεξέ σφι λόγον, ἄνδρα φὰς αὐλητὴν ἰδόντα ἰχθῦς ἐν τῇ θαλάσσῃ αὐλέειν, δοκέοντά σφεας ἐξελεύσεσθαι ἐς γῆν. Ὡς δὲ ψευσθῆναι τῆς ἐλπίδος, λαβεῖν ἀμφίβληστρον καὶ περιβαλεῖν τε πλῆθος πολλὸν τῶν ἰχθύων καὶ ἐξειρύσαι, ἰδόντα δὲ παλλομένους εἰπεῖν ἄρα αὐτὸν πρὸς τοὺς ἰχθῦς· ``Παύεσθέ μοι ὀρχεόμενοι, ἐπεὶ οὐδ' ἐμέο αὐλέοντος ἠθέλετε ἐκβαίνειν [ὀρχεόμενοι].'' 

Κῦρος μὲν τοῦτον τὸν λόγον τοῖσι Ἴωσι καὶ τοῖσι Αἰολεῦσι τῶνδε εἵνεκα ἔλεξε, ὅτι δὴ οἱ Ἴωνες πρότερον αὐτοῦ Κύρου δεηθέντος δι' ἀγγέλων ἀπίστασθαί σφεας ἀπὸ Κροίσου οὐκ ἐπείθοντο, τότε δὲ κατεργασμένων τῶν πρηγμάτων ἦσαν ἕτοιμοι πείθεσθαι Κύρῳ. 

Ὁ μὲν δὴ ὀργῇ ἐχόμενος ἔλεγέ σφι τάδε.  Ἴωνες δὲ ὡς ἤκουσαν τούτων ἀνενειχθέντων ἐς τὰς πόλις, τείχεά τε περιεβάλοντο ἕκαστοι καὶ συνελέγοντο ἐς Πανιώνιον οἱ ἄλλοι πλὴν Μιλησίων· πρὸς μούνους γὰρ τούτους ὅρκιον Κῦρος ἐποιήσατο ἐπ' οἷσί περ ὁ Λυδός· τοῖσι δὲ λοιποῖσι [Ἴωσι] ἔδοξε κοινῷ λόγῳ πέμπειν ἀγγέλους ἐς Σπάρτην δεησομένους Ἴωσι τιμωρέειν.
}
}
\end{greek}
