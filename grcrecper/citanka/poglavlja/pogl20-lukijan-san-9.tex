%Unesi korekture NČ, 2019-08-12
%\section*{O autoru}


\section*{O tekstu}

Djelo \textgreek[variant=ancient]{Περὶ τοῦ Ἐνυπνίου ἤτοι Βίος Λουκιανοῦ} \textit{(Somnium, San ili Lukijanov život)} predstavlja Lukijanovu autobiografiju i glavni je izvor za poznavanje piščeva života. Lukijan opisuje pokušaj svojih roditelja da ga obrazuju za kipara i klesara. Prvi dan u kiparskoj radionici neslavno je završio: mladi Lukijan oštetio je vrijedan komad mramora. Te je noći usnuo san u kojem mu pristupaju personifikacije kiparstva \textgreek[variant=ancient]{(Ἑρμογλυφικὴ τέχνη)} i obrazovanja \textgreek[variant=ancient]{(Παιδεία)} koje se otimaju za njega. Slijedi govor Obrazovanja u liku lijepe i elegantne žene, za razliku od muškobanjaste i prljave personifikacije kiparstva.

\section*{Pročitajte naglas grčki tekst.}

Luc.\ Somnium sive vita Luciani 9

%Naslov prema izdanju

\medskip


{\large
{ 
\begin{greek}

\noindent ``Ἐγὼ δέ, ὦ τέκνον, Παιδεία εἰμὶ ἤδη συνήθης σοι καὶ γνωρίμη, εἰ καὶ μηδέπω εἰς τέλος μου πεπείρασαι. ἡλίκα μὲν οὖν τὰ ἀγαθὰ ποριῇ λιθοξόος γενόμενος, αὕτη προείρηκεν· οὐδὲν γὰρ ὅτι μὴ ἐργάτης ἔσῃ τῷ σώματι πονῶν κἀν τούτῳ τὴν ἅπασαν ἐλπίδα τοῦ βίου τεθειμένος, ἀφανὴς μὲν αὐτὸς ὤν, ὀλίγα καὶ ἀγεννῆ λαμβάνων, ταπεινὸς τὴν γνώμην, εὐτελὴς δὲ τὴν πρόοδον, οὔτε φίλοις ἐπιδικάσιμος οὔτε ἐχθροῖς φοβερὸς οὔτε τοῖς πολίταις ζηλωτός, ἀλλ' αὐτὸ μόνον ἐργάτης καὶ τῶν ἐκ τοῦ πολλοῦ δήμου εἷς, ἀεὶ τὸν προὔχοντα ὑποπτήσσων καὶ τὸν λέγειν δυνάμενον θεραπεύων, λαγὼ βίον ζῶν καὶ τοῦ κρείττονος ἕρμαιον ὤν· εἰ δὲ καὶ Φειδίας ἢ Πολύκλειτος γένοιο καὶ πολλὰ θαυμαστὰ ἐξεργάσαιο, τὴν μὲν τέχνην ἅπαντες ἐπαινέσονται, οὐκ ἔστι δὲ ὅστις τῶν ἰδόντων, εἰ νοῦν ἔχοι, εὔξαιτ' ἂν σοὶ ὅμοιος γενέσθαι· οἷος γὰρ ἂν ᾖς, βάναυσος καὶ χειρῶναξ καὶ ἀποχειροβίωτος νομισθήσῃ.

Ἢν δ' ἐμοὶ πείθῃ, πρῶτον μέν σοι πολλὰ ἐπιδείξω παλαιῶν ἀνδρῶν ἔργα καὶ πράξεις θαυμαστὰς καὶ λόγους αὐτῶν ἀπαγγελῶ, καὶ πάντων ὡς εἰπεῖν ἔμπειρον ἀποφανῶ, καὶ τὴν ψυχήν, ὅπερ σοι κυριώτατόν ἐστι, κατακοσμήσω πολλοῖς καὶ ἀγαθοῖς κοσμήμασι — σωφροσύνῃ, δικαιοσύνῃ, εὐσεβείᾳ, πρᾳότητι, ἐπιεικείᾳ, συνέσει, καρτερίᾳ, τῷ τῶν καλῶν ἔρωτι, τῇ πρὸς τὰ σεμνότατα ὁρμῇ· ταῦτα γάρ ἐστιν ὁ τῆς ψυχῆς ἀκήρατος ὡς ἀληθῶς κόσμος. λήσει δέ σε οὔτε παλαιὸν οὐδὲν οὔτε νῦν γενέσθαι δέον, ἀλλὰ καὶ τὰ μέλλοντα προόψει μετ' ἐμοῦ, καὶ ὅλως ἅπαντα ὁπόσα ἐστί, τά τε θεῖα τά τ' ἀνθρώπινα, οὐκ εἰς μακράν σε διδάξομαι.''


\end{greek}


}
}
