% redaktura NZ (ne slažem se s autou gen. part.), mala red. NJ

\section*{O autoru}

Lizija (Λυσίας, Atena, oko 440. pr. Kr. – oko 378. pr. Kr.), govornik i pisac sudbenih govora \textgreek[variant=ancient]{(λογογράφος),} autor preko dvjesto takvih djela – u antici mu ih je pripisivano 425 – od kojih je sačuvano 31. Sin bogatog trgovca i poduzetnika iz Sirakuze, govorničku karijeru počeo je tek 403. pr. Kr., nakon pada Tridesetorice tirana (govorom \textit{Protiv Eratostena}, \textgreek[variant=ancient]{Κατὰ Ἐρατοσϑένους,} jedinim koji je sam održao, optužuje jednoga od oligarha za smrt svojeg brata). Sastavljao govore za najraznolikije parnice, od ubojstva i izdaje do preljuba i pronevjere, tako da su njegova djela bogat izvor podataka o atenskoj povijesti, pravu i običajima. Spominje ga Platon u dijalozima. Jedan je od desetorice kanonskih atičkih govornika.

\section*{O tekstu}

Agorat je bio profesionalni potkazivač čije su prijave skrivile smrt niza uglednih građana, među njima i izvjesnog Dionizodora. Nakon pada Tridesetorice, Dionizodorov svak i rođak, Agorata za to optužuje pred sudom, 403.\ ili 402.\ pr. Kr. Opsežno pripovijeda kako su Atenjani poraženi u Peloponeskom ratu, i kako je Agorat radio za neprijatelje atenske slobode, prijavljujući, između ostalih, one koji su za Peloponeskog rata bili zapovjednici – stratezi i taksijarsi (Dionizodor je bio taksijarh). Naš odlomak govori o rastanku osuđenika na smrt s rođakinjama i suprugama.

\newpage

\section*{Pročitajte naglas grčki tekst.}

%Naslov prema izdanju
Lys.\ In Agoratum 39.1

\medskip

{\large
\begin{greek}
\noindent ᾿Επειδὴ τοίνυν, ὦ ἄνδρες δικασταί, θάνατος αὐτῶν κατεγνώσθη καὶ ἔδει αὐτοὺς ἀποθνῄσκειν, μεταπέμπονται εἰς τὸ δεσμωτήριον ὁ μὲν ἀδελφήν, ὁ δὲ μητέρα, ὁ δὲ γυναῖκα, ὁ δ' ἥ τις ἦν ἑκάστῳ αὐτῶν προσήκουσα, ἵνα τὰ ὕστατα ἀσπασάμενοι τοὺς αὑτῶν οὕτω τὸν βίον τελευτήσειαν. καὶ δὴ καὶ Διονυσόδωρος μεταπέμπεται τὴν ἀδελφὴν τὴν ἐμὴν εἰς τὸ δεσμωτήριον, γυναῖκα ἑαυτοῦ οὖσαν· πυθομένη δ' ἐκείνη ἀφικνεῖται, μέλαν τε ἱμάτιον ἠμφιεσμένη \dots, ὡς εἰκὸς ἦν ἐπὶ τῷ ἀνδρὶ αὐτῆς τοιαύτῃ συμφορᾷ κεχρημένῳ. ἐναντίον δὲ τῆς ἀδελφῆς τῆς ἐμῆς Διονυσόδωρος τά τε οἰκεῖα τὰ αὑτοῦ διέθετο ὅπως αὐτῷ ἐδόκει, καὶ περὶ ᾿Αγοράτου τουτουὶ ἔλεγεν ὅτι $\langle$οἱ$\rangle$ αἴτιος ἦν τοῦ θανάτου, καὶ ἐπέσκηπτεν ἐμοὶ καὶ Διονυσίῳ τουτῳί, τῷ ἀδελφῷ τῷ αὑτοῦ, καὶ τοῖς φίλοις πᾶσι τιμωρεῖν ὑπὲρ αὑτοῦ ᾿Αγόρατον· καὶ τῇ γυναικὶ τῇ αὑτοῦ ἐπέσκηπτε, νομίζων αὐτὴν κυεῖν ἐξ αὑτοῦ, ἐὰν γένηται αὐτῇ παιδίον, φράζειν τῷ γενομένῳ ὅτι τὸν πατέρα αὐτοῦ ᾿Αγόρατος ἀπέκτεινε, καὶ κελεύειν τιμωρεῖν ὑπὲρ αὑτοῦ ὡς φονέα ὄντα.

\end{greek}

}

