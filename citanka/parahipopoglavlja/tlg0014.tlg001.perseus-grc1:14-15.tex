%\section*{O autoru}

%TKTK


\section*{O tekstu}


%\newpage

\section*{Pročitajte naglas grčki tekst.}

Dem. Olynthiaca I 14-15

%Naslov prema izdanju

\medskip


{\large

\begin{greek}

\noindent  τί οὖν, ἄν τις εἴποι, ταῦτα λέγεις ἡμῖν νῦν; ἵνα γνῶτ', ὦ ἄνδρες Ἀθηναῖοι, καὶ αἴσθησθ' ἀμφότερα, καὶ τὸ προΐεσθαι καθ' ἕκαστον ἀεί τι τῶν πραγμάτων ὡς ἀλυσιτελές, καὶ τὴν φιλοπραγμοσύνην ᾗ χρῆται καὶ συζῇ Φίλιππος, ὑφ' ἧς οὐκ ἔστιν ὅπως ἀγαπήσας τοῖς πεπραγμένοις ἡσυχίαν σχήσει. εἰ δ' ὁ μὲν ὡς ἀεί τι μεῖζον τῶν ὑπαρχόντων δεῖ πράττειν ἐγνωκὼς ἔσται, ἡμεῖς δ' ὡς οὐδενὸς ἀντιληπτέον ἐρρωμένως τῶν πραγμάτων, σκοπεῖσθ' εἰς τί ποτ' ἐλπὶς ταῦτα τελευτῆσαι.

πρὸς θεῶν, τίς οὕτως εὐήθης ἐστὶν ὑμῶν ὅστις ἀγνοεῖ τὸν ἐκεῖθεν πόλεμον δεῦρ' ἥξοντα, ἂν ἀμελήσωμεν; ἀλλὰ μήν, εἰ τοῦτο γενήσεται, δέδοικ', ὦ ἄνδρες Ἀθηναῖοι, μὴ τὸν αὐτὸν τρόπον ὥσπερ οἱ δανειζόμενοι ῥᾳδίως ἐπὶ τοῖς μεγάλοις [τόκοις] μικρὸν εὐπορήσαντες χρόνον ὕστερον καὶ τῶν ἀρχαίων ἀπέστησαν, οὕτω καὶ ἡμεῖς [ἂν] ἐπὶ πολλῷ φανῶμεν ἐρρᾳθυμηκότες, καὶ ἅπαντα πρὸς ἡδονὴν ζητοῦντες πολλὰ καὶ χαλεπὰ ὧν οὐκ ἐβουλόμεθ' ὕστερον εἰς ἀνάγκην ἔλθωμεν ποιεῖν, καὶ κινδυνεύσωμεν περὶ τῶν ἐν αὐτῇ τῇ χώρᾳ.

\end{greek}

}


\section*{Analiza i komentar}

%1

{\large
\begin{greek}
\noindent 
\end{greek}
}

\begin{description}[noitemsep]


\end{description}

%2


%kraj
