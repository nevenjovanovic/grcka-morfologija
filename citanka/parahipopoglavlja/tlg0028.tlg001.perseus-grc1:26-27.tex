\section*{O autoru}

\textgreek[variant=ancient]{Ἀντιφῶν ὁ Ῥαμνούσιος,} Antifont iz Ramnunta (oko 480.–411.\ pr.~Kr.) atenski je političar i govornik, u povijesti zabilježen kao prvi \textgreek[variant=ancient]{λογογράφος,} prvi autor govora za druge koje sastavlja za novac.
 
Živio je bez isticanja i bez sudjelovanja u javnom i političkom životu Atene sve do 411.\ pr.~Kr.\ kada postaje jedan od glavnih aktera uspostave oligarhijske vlade Vijeća četiri stotine. Nakon pada Vijeća, za samo tri mjeseca, Antifont je doveden pred sud gdje je zbog svoje uloge u zbivanjima osuđen na smrt. Antifontov govor u vlastitu obranu bio je jedan od najglasovitijih obrambenih govora starine, a Tukidid ga je smatrao najvećim govorom čovjeka osuđenim na smrt.

Do nas je stiglo petnaest od šezdeset govora poznatih u Augustovo doba (već su se tada neki smatrali neautentičnima). Dvanaest govora vezano je u takozvane tetralogije, govore za obje strane u izmišljenim sporovima: dva su govora optužbe i dva obrane, a redaju se naizmjence. Svi su govori sastavljeni za slučajeve ubojstva \textgreek[variant=ancient]{(φονικοὶ λόγοι).}

Uzmah demokracije od polovine 5.~st.\ pr.~Kr, s naglaskom na aktivnom sudjelovanju građana u javnom prostoru (politika, sudstvo…) potaknuo je razvoj atičke proze. Snažnu potvrdu važnosti nove vještine govorenja u javnom životu Atene filolozi vide i u Antifontovim govorima, koji se nalaze na razmeđi između starog i novog govorništva.

\section*{O tekstu}

Tekst koji čitamo pripada Antifontovu govoru \textgreek[variant=ancient]{Φαρμακείας κατὰ τῆς μητρυιᾶς}, \textit{Protiv maćehe radi ubojstva}. Govor se najčešće datira između 419.\ i 414.\ pr.~Kr.

Riječ je o govoru optužbe na suđenju za ubojstvo; sin optužuje maćehu da je prije nekoliko godina ubila oca. Optužba se temelji na dvije teze: da je maćeha dogovorila nabavku otrova, te da je oca ubila namjerno. Maćeha je nagovorila neku robinju da nabavi otrov – ljubavni napitak – i da ga ocu. Dok su otac i njegov  prijatelj Filonej, robinjin ljubavnik, bili zajedno na večeri, robinja ih je otrovala. Obojica su umrla, a robinja je odmah kažnjena smrću. No „prava Klitemnestra“, tvrdi optužba, tužiteljeva je maćeha.

U ovom izvatku tužitelj upozorava da se smilovati treba pri slučajnim nezgodama ili nesrećama, a ovako beskrupulozna žena koja je namjerno ubila čovjeka to ne zaslužuje.


%\newpage

\section*{Pročitajte naglas grčki tekst.}

Antipho Orator, In novercam 26–27

%Naslov prema izdanju

\medskip


{\large

\begin{greek}

\noindent ἡ μὲν γὰρ ἑκουσίως καὶ βουλεύσασα τὸν θάνατον ἀπέκτεινεν, ὁ δ᾽ ἀκουσίως καὶ βιαίως ἀπέθανε. πῶς γὰρ οὐ βιαίως ἀπέθανεν, ὦ ἄνδρες; ὅς γ᾽ ἐκπλεῖν ἔμελλεν ἐκ τῆς γῆς τῆσδε, παρά τε ἀνδρὶ φίλῳ αὑτοῦ εἱστιᾶτο· ἡ δὲ πέμψασα τὸ φάρμακον καὶ κελεύσασα ἐκείνῳ δοῦναι πιεῖν ἀπέκτεινεν ἡμῶν τὸν πατέρα. πῶς οὖν ταύτην ἐλεεῖν ἄξιόν ἐστιν ἢ αἰδοῦς τυγχάνειν παρ᾽ ὑμῶν ἢ ἄλλου του; ἥτις αὐτὴ οὐκ ἠξίωσεν ἐλεῆσαι τὸν ἑαυτῆς ἄνδρα, ἀλλ᾽ ἀνοσίως καὶ αἰσχρῶς ἀπώλεσεν.

οὕτω δέ τοι καὶ ἐλεεῖν ἐπὶ τοῖς ἀκουσίοις παθήμασι μᾶλλον προσήκει ἢ τοῖς ἑκουσίοις καὶ ἐκ προνοίας ἀδικήμασι καὶ ἁμαρτήμασι. καὶ ὥσπερ ἐκεῖνον αὕτη οὔτε θεοὺς οὔθ᾽ ἥρωας οὔτ᾽ ἀνθρώπους αἰσχυνθεῖσα οὐδὲ δείσασ᾽ ἀπώλεσεν, οὕτω καὶ αὐτὴ ὑφ᾽ ὑμῶν καὶ τοῦ δικαίου ἀπολομένη, καὶ μὴ τυχοῦσα μήτ᾽ αἰδοῦς μήτ᾽ ἐλέου μήτ᾽ αἰσχύνης μηδεμιᾶς παρ᾽ ὑμῶν, τῆς δικαιοτάτης ἂν τύχοι τιμωρίας.

\end{greek}

}


\section*{Komentar}

%1

\begin{description}[noitemsep]
\item[ἡ μὲν… ὁ δ'] koordinacija pomoću čestica μὲν i δέ ovdje izriče suprotnost koju podupire i suprotnost subjekata: ona\dots\ a on\dots
\end{description}

%2

\begin{description}[noitemsep]
\item[Πῶς… ἀπέθανεν] upitni prilog πῶς uvodi direktno pitanje, “kako…”
\end{description}

%3

\begin{description}[noitemsep]
\item[κελεύσασα ] §~267, glagol otvara mjesto dopuni u infinitivu δοῦναι: da\dots
\item[δοῦναι] §~307, glagol ima dopunu u infinitivu πιεῖν
\item[πῶς… ἄξιόν ἐστιν] upitni prilog πῶς uvodi direktno pitanje: kako\dots
\item[ἄξιόν ἐστιν] imenski predikat, fraza koja otvara mjesto dopuni u infinitivu
\item[ἐλεεῖν] rekcija: τινα; §~243, infinitiv u službi subjekta §~492
\item[τυγχάνειν] rekcija: τινος; §~231 (osnove §~321.19), infinitiv u službi subjekta §~492
\item[ἥτις… οὐκ ἠξίωσεν] odnosa zamjenica ἥτις uvodi zavisnu odnosnu rečenicu, „ona koja…“
\item[ἠξίωσεν] §~267, augment §~235, glagol otvara mjesto dopuni u infinitivu
\item[ἐλεῆσαι] rekcija: τινα; §~267
\item[ἀλλ'… ἀπώλεσεν] suprotni veznik ἀλλά uvodi nezavisnu suprotnu rečenicu, „nego…“
\end{description}

%4
\begin{description}[noitemsep]
\item[ἐλεεῖν] rekcija: τινα; §~243, infinitiv u službi subjekta §~492
\item[προσήκει] §~231, glagol otvara mjesto dopuni u infinitivu s funkcijom subjekta §~491
\end{description}

%5
\begin{description}[noitemsep]
\item[Καὶ ὥσπερ… οὕτω καὶ ] poredbeni veznik ὥσπερ uvodi zavisnu poredbenu rečenicu u korelaciji s rečenicom koju uvodi korelativ οὕτω: i kao što… tako i…
\end{description}


%kraj
