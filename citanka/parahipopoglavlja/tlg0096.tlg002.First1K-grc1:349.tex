%\section*{O autoru}

%TKTK


\section*{O tekstu}

U ovoj basni, sačuvanoj pod imenom glasovitog basnopisca Ezopa, razmatra se moć i neizbježnost sudbine.

Neki otac usne loš san o sinu jedincu. U strahu za njegov život, poduzme sve kako se san ne bi nikada ostvario. No, brinući se prvenstveno za sinovljevu sigurnost, nije mislio na to je li sin i sretan. Osjećaj frustracije pokrenut će nesretan slijed događaja i tako će sin susresti svoju sudbinu, mada pred nacrtanim lavom.

%\newpage

\section*{Pročitajte naglas grčki tekst.}

Aesop. Fabulae 279 (349)

%Naslov prema izdanju
%NČ

\medskip


{\large

\begin{greek}

\noindent ΠΑΙΣ, ΠΑΤΗΡ ΚΑΙ ΛΕΩΝ ΓΕΓΡΑΜΜΕΝΟΣ

\medskip

\noindent Υἱόν τις γέρων δειλὸς μονογενῆ ἔχων γενναῖον, κυνηγεῖν ἐφιέμενον, εἶδε τοῦτον καθʼ ὕπνους ὑπὸ λέοντος ἀναλωθέντα. Φοβηθεὶς δὲ, μή πως ὁ ὄνειρος ἀληθεύσῃ, οἴκημα κάλλιστον καὶ μετέωρον κατεσκεύασε· κἀκεῖσε τὸν υἱὸν εἰσαγαγὼν ἐφύλαττεν. Ἐζωγράφησε δὲ ἐν τῷ οἰκήματι πρὸς τέρψιν τοῦ υἱοῦ παντοῖα ζῶα, ἐν οἷς ἦν καὶ λέων. Ὁ δὲ ταῦτα ὁρῶν πλείονα λύπην εἶχε. Καὶ δή ποτε πλησίον τοῦ λέοντος στὰς εἶπεν· ``ὡ κάκιστον θηρίον, διὰ σὲ καὶ τὸν ψευδῆ ὄνειρον τοῦ ἐμοῦ πατρὸς τῇδε τῇ οἰκίᾳ κατεκλείσθην, ὡς ἐν φρουρᾷ· τί σοι ποιήσω;'' Καὶ εἰπὼν ἐπέβαλε τῷ τοίχῳ τὴν χεῖρα, ἐκτυφλῶσαι τὸν λέοντα. Σκόλοψ δὲ τῷ δακτύλῳ αὐτοῦ ἐμπαρεὶς, ὄγκωμα καὶ φλεγμονὴν μέχρι βουβῶνος εἰργάσατο· πυρετὸς δὲ ἐπιγενόμενος αὐτῷ, θᾶττον τοῦ βίου μετέστησεν. Ὁ δὲ λέων καὶ οὕτως ἀνῄρηκε τὸν παῖδα, μηδὲν τῷ τοῦ πατρὸς ὠφεληθέντα σοφίσματι.

Ὁ μῦθος δηλοῖ, ὅτι οὐδεὶς δύναται τὸ μέλλον ἐκφυγεῖν.

\end{greek}

}


\section*{Komentar}

%1

%2


%3

\begin{description}[noitemsep]
\item[φοβηθεὶς] sc.\ \textgreek[variant=ancient]{τις γέρων δειλὸς}
\item[μήπως ὁ ὄνειρος ἀληθεύσῃ] namjerna rečenica koje uvodi glagol bojazni, izriče bojazan da bi se nešto moglo desiti (umjesto μήπως uobičajeniji su veznici μή ili ὅπως μή)
\item[εἰσαγαγὼν τὸν υἱὸν] i \textbf{ἐφύλαττεν τὸν υἱὸν} oba glagola imaju istu dopunu u direktnom objektu pa se ne objekt ne ponavlja; u hrvatskom se objekt najčešće izriče: „uvevši sina, čuvao ga je“
\end{description}

%4


%5

\begin{description}[noitemsep]
\item[ὁ δὲ] adverzativna čestica δὲ uvodi novi subjekt, ``a…'', sc.\ ὁ δὲ υἱός
\item[λύπην εἶχε] = ἐλυπεῖτο, ``bio je tužan''
\end{description}


%6


%7


%8

%9

%10

\begin{description}[noitemsep]
\item[ὠφεληθέντα] rekcija: τινος τινι, „na korist za koga u čemu“
\end{description}

%11


\begin{description}[noitemsep]
\item[ὅτι... δύναται] glagol δηλοῖ otvara mjesto izričnom vezniku ὅτι koji uvodi izričnu rečenicu: „da…“
\end{description}



%kraj
