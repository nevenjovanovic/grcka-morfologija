%\section*{O autoru}


\section*{O tekstu}

U četvrtoj knjizi \textgreek[variant=ancient]{Ἱστορίαι} (\textit{Povijesti}) Polibije pripovijeda o grčkom Savezničkom ili Etolskom ratu. Rat je Helenski savez, predvođen Filipom V.\ Makedonskim, vodio protiv Etolskog saveza, Sparte i Elide, od 220.\ do 217.\ p.~n.~e.

Tijekom priprema za rat, 220. je došlo do skandala – Mesenija, koja je bila povod ratu, izglasala je, na poticaj svojih efora i nekih pristaša oligarhije, da neće ratovati protiv Etolaca prije nego što im bude oduzeta Figaleja, grad koji ugrožava mesenski teritorij. Ovaj je događaj potaknuo Polibija da razmišlja o miru koji može biti sramotan ako se osigurava pod svaku cijenu.




%\newpage

\section*{Pročitajte naglas grčki tekst.}

Plb.\ Historiae 4.31.3
%Naslov prema izdanju

\medskip


{\large

\begin{greek}

\noindent ἐγὼ γὰρ φοβερὸν μὲν εἶναί φημι τὸν πόλεμον, οὐ μὴν οὕτω γε φοβερὸν ὥστε πᾶν ὑπομένειν χάριν τοῦ μὴ προσδέξασθαι πόλεμον.

\noindent ἐπεὶ τί καὶ θρασύνομεν τὴν ἰσηγορίαν καὶ παρρησίαν καὶ τὸ τῆς ἐλευθερίας ὄνομα πάντες, εἰ μηδὲν ἔσται προυργιαίτερον τῆς εἰρήνης; οὐδὲ γὰρ Θηβαίους ἐπαινοῦμεν κατὰ τὰ Μηδικά, διότι τῶν ὑπὲρ τῆς Ἑλλάδος ἀποστάντες κινδύνων τὰ Περσῶν εἵλοντο διὰ τὸν φόβον, οὐδὲ Πίνδαρον τὸν συναποφηνάμενον αὐτοῖς ἄγειν τὴν ἡσυχίαν διὰ τῶνδε τῶν ποιημάτων,
\begin{verse}
τὸ κοινόν τις ἀστῶν ἐν εὐδίᾳ τιθεὶς\\
ἐρευνασάτω μεγαλάνορος ἡσυχίας\\
τὸ φαιδρὸν φάος.\\

\end{verse}
δόξας γὰρ παραυτίκα πιθανῶς εἰρηκέναι, μετʼ οὐ πολὺ πάντων αἰσχίστην εὑρέθη καὶ βλαβερωτάτην πεποιημένος ἀπόφασιν· εἰρήνη γὰρ μετὰ μὲν τοῦ δικαίου καὶ πρέποντος κάλλιστόν ἐστι κτῆμα καὶ λυσιτελέστατον, μετὰ δὲ κακίας ἢ δειλίας ἐπονειδίστου πάντων αἴσχιστον καὶ βλαβερώτατον.

\end{greek}

}

%\newpage

\section*{Komentar}

%1


\begin{description}[noitemsep]
\item[φοβερὸν μὲν\dots] \textbf{οὐ μὴν οὕτω γε φοβερὸν\dots}\ adverzativna upotreba kombinacija čestica; μὴν odgovara na tvrdnju uvedenu s μὲν (γε baš): \dots\ ali ipak ne baš\dots
\item[ὑπομένειν] §~231; infinitiv kao predikat zavisno posljedične rečenice; LSJ ὑπομένω II.2
\item[χάριν] prilog, rekcija τινός, LSJ χάρις A.VI.1.
\end{description}

%2


\begin{description}[noitemsep]
\item[τὸ τῆς ἐλευθερίας ὄνομα] adnominalni genitiv, Smyth 1290–1296
\end{description}

%3

\begin{description}[noitemsep]
\item[τὰ Μηδικά] supstantivirani pridjev, §~373; ovdje u specifičnom značenju ``Perzijski ratovi''
\item[ἀποστάντες] složenica glagola ἵστημι, §~306; LSJ ἀφίστημι B
\item[τὰ Περσῶν] supstantivirani genitiv posvojni, §~373; ovdje u specifičnom značenju ``strana Perzijanaca''
\item[συναποφηνάμενον] §~254; složenica φαίνω, s. 118; rekcija τινί + infinitiv: nekome da (nešto)\dots
\item[ἄγειν] §~231; ovom je infinitivu mjesto otvorio oblik συναποφηνάμενον
\item[τὸ κοινόν τις ἀστῶν\dots] ovo je Pindarov ulomak 109, iz ode Tebancima, sačuvan također i u \textit{Izvacima} Ivana Stobeja \textgreek[variant=ancient]{(Ἰωάννης ὁ Στοβαῖος)} iz V. st. n.~e.\ (Stob. ecl. 4.16.6); moguće je da oda i ne govori o događajima iz Grčko-perzijskog rata 480.
\item[τὸ κοινόν] supstantivirani pridjev, §~373
\item[μεγαλάνορος ἡσυχίας] atributna dopuna imenice φάος; adnominalni genitiv, Smyth 1290–1296

\end{description}


%4

\begin{description}[noitemsep]
\item[τοῦ δικαίου καὶ πρέποντος] supstantivirani pridjevi, §~373
\end{description}



%kraj
