%\section*{O autoru}

%TKTK


\section*{O tekstu}

Platonov dijalog \textit{Protagora} \textgreek[variant=ancient]{(Πρωταγόρας)} nosi ime po jednome od glavnih likova, slavnom sofistu iz Abdere (\textgreek[variant=ancient]{Ἄβδηρα}, grad na sjevernoj obali Egejskog mora, u Platonovo je doba pod vlašću Atene). Protagora, prvi koji se prozvao \textgreek[variant=ancient]{σοφιστής} i prvi koji je djelovao kao učitelj, nakon četiri desetljeća takva rada umro je oko 420.\ p.~n.~e, u dobi od oko sedamdeset godina. 

Djelo pripada skupini dijaloga u kojima Platon kritizira sofističke
načine uvjeravanja i njihovu amoralnost (uz \textit{Protagoru}, u skupinu ulaze još \textit{Gorgija} i \textit{Sofist}); važna su, naime, gesla sofista, koja je formulirao upravo Protagora, bila \textgreek[variant=ancient]{τὸν ἥττω λόγον κρείττω ποιεῖν} (Aristotel, \textit{Retorika} 1402a), i \textgreek[variant=ancient]{δύο λόγους εἶναι περὶ παντὸς πράγματος ἀντικειμένους ἀλλήλοις} (Diogen Laertije 9, 8 51). \textit{Protagora} je nastao krajem prve faze Platonova pisanja, koja traje u devedesetim i osamdesetim godinama IV.~st.\ p.~n.~e. Vrijeme fiktivne radnje stoga prethodi vremenu samog pisanja dijaloga nekih pedesetak godina.

Sokrat pripovijeda prijateljima što se dogodilo pri njegovu susretu s Protagorom u Kalijinoj kući. Tog ga je jutra probudio mladi prijatelj Hipokrat, uzbuđeno ga pozivajući da zajedno s njim posjeti najslavnijeg od svih sofista, koji je stigao u Atenu. Kako je bilo još rano za posjetu, Sokrat i Hipokrat su prošetali. Pritom je Hipokrat pokušao objasniti Sokratu zašto toliko želi postati Protagorin učenik; isprva je tvrdio da želi i sam biti sofist, profesionalac poput Protagore, ali, pred Sokratovim pitanjima, uzmakne i ustvrdi da želi učiti kod Protagore da bi stekao obrazovanje dostojno slobodna čovjeka. Zna da je sofist mudrac koji omogućuje drugima da dobro govore, ali \textit{o čemu} govore, to Hipokrat ne može reći.

U Kalijinoj kući zatječu Protagoru, druge sofiste, i učenike oko njih. Sokrat i Protagora zameću raspravu o tome može li se vrlina (\textgreek[variant=ancient]{ἀρετή}) naučiti; Protagorin nastup posebno je obilježen mitom o Prometeju i Epimeteju, odnosno o ljudskom razvoju od divljaštva do civiliziranosti pomoću tehničkih i političkih umijeća – u poučavanju potonjih upravo je Protagora predvodnik. Na početku rasprave, dakle, Protagora tvrdi da se vrlina \textit{može} naučiti, a Sokrat to osporava; no, pošto Sokrat pitanjima natjera Protagoru u škripac, a sofist se pokuša izvuči retoričkim akrobacijama (koje izazivaju odobravanje publike, ali ne i Sokratovo), u drugom krugu rasprave -- nakon intermezza koji čine govori ostalih sofističkih zvijezda, Alkibijada, Prodika i Hipije – sad \textit{Sokrat} zastupa tezu da se vrlina može naučiti, i opet dovede Protagoru u bezizlaznu poziciju; tako porazi sofista njegovim vlastitim oružjem, pokazujući da može uspješnije od njega argumentirati suprotne strane spora (\textgreek[variant=ancient]{λόγους ἀντικειμένους ἀλλήλοις}). Rasprava o \textgreek[variant=ancient]{ἀρετή} ujedno određuje osnovne crte Platonova etičkog intelektualizma, pokazujući da je znanje temelj i bit vrline, da je poznavanje dobra nužno i dostatno za ispravno djelovanje, da vrlinu ima onaj koji poznaje dobro, a da nitko ne bira zlo dragovoljno, već zbog pogrešne procjene, nepoznavanja onoga što je dobro.

U ovdje odabranom odlomku Platonova dijaloga govori Protagora. Pošto je ispričao mit o Prometeju i Epimeteju, na Sokratov argument da se vrlina ne može naučiti jer bi inače i država osigurala njezino poučavanje, Protagora odgovara pokazujući da je upoznavanje vrline važna sastavnica tradicionalnog grčkog obrazovanja, već od najranije dobi djeteta, a osobito kod učitelja, kada se počnu čitati djela dobrih pjesnika (\textgreek[variant=ancient]{ποιητῶν ἀγαθῶν ποιήματα}).

\newpage

\section*{Pročitajte naglas grčki tekst.}

Plat. Protagora 325c-326a

%Naslov prema izdanju

\medskip


{\large

\begin{greek}

\noindent ἐπειδὰν θᾶττον συνιῇ τις τὰ λεγόμενα, καὶ τροφὸς καὶ μήτηρ καὶ παιδαγωγὸς καὶ αὐτὸς ὁ πατὴρ περὶ τούτου διαμάχονται, ὅπως ὡς βέλτιστος ἔσται ὁ παῖς, παρ' ἕκαστον καὶ ἔργον καὶ λόγον διδάσκοντες καὶ ἐνδεικνύμενοι ὅτι τὸ μὲν δίκαιον, τὸ δὲ ἄδικον, καὶ τόδε μὲν καλόν, τόδε δὲ αἰσχρόν, καὶ τόδε μὲν ὅσιον, τόδε δὲ ἀνόσιον, καὶ τὰ μὲν ποίει, τὰ δὲ μὴ ποίει. καὶ ἐὰν μὲν ἑκὼν πείθηται· εἰ δὲ μή, ὥσπερ ξύλον διαστρεφόμενον καὶ καμπτόμενον εὐθύνουσιν ἀπειλαῖς καὶ πληγαῖς. μετὰ δὲ ταῦτα εἰς διδασκάλων πέμποντες πολὺ μᾶλλον ἐντέλλονται ἐπιμελεῖσθαι εὐκοσμίας τῶν παίδων ἢ γραμμάτων τε καὶ κιθαρίσεως· οἱ δὲ διδάσκαλοι τούτων τε ἐπιμελοῦνται, καὶ ἐπειδὰν αὖ γράμματα μάθωσιν καὶ μέλλωσιν συνήσειν τὰ γεγραμμένα ὥσπερ τότε τὴν φωνήν, παρατιθέασιν αὐτοῖς ἐπὶ τῶν βάθρων ἀναγιγνώσκειν ποιητῶν ἀγαθῶν ποιήματα καὶ ἐκμανθάνειν ἀναγκάζουσιν, ἐν οἷς πολλαὶ μὲν νουθετήσεις ἔνεισιν πολλαὶ δὲ διέξοδοι καὶ ἔπαινοι καὶ ἐγκώμια παλαιῶν ἀνδρῶν ἀγαθῶν, ἵνα ὁ παῖς ζηλῶν μιμῆται καὶ ὀρέγηται τοιοῦτος γενέσθαι.

\end{greek}

}


\section*{Komentar}

%1

\begin{description}[noitemsep]
\item[ἐπειδὰν θᾶττον] komparativ pridjeva ταχύς upotrijebljen kao prilog, u frazi: ``odmah čim\dots''
\end{description}

%2

\begin{description}[noitemsep]
\item[ἐὰν\dots\ ἑκὼν πείθηται] § 232; sc.\ εὖ ἔχει, ἐάν otvara mjesto konjunktivu prezenta; protaza eventualne pogodbene zavisne rečenice izriče iterativnu radnju, u apodozi se očekuje indikativ prezenta § 475; ἑκών upotrijebljen kao ekvivalent priloga (adverbno), Smyth 1095
\end{description}
%3


\begin{description}[noitemsep]
\item[ὀρέγηται τοιοῦτος γενέσθαι] medijalni oblik ὀρέγω otvara mjesto infinitivu, LSJ ὀρέγω II.2.b; imenski predikat s kopulativnim glagolom, Smyth 909

\end{description}


%kraj
