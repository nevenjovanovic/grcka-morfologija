%\section*{O autoru}

%TKTK


\section*{O tekstu}

Atenski pisac Ksenofont – plaćeni vojnik, povjesničar i filozof – bio je u mladosti i Sokratov učenik. I na Ksenofonta je, kao i na mnoge druge (Platona među njima), Sokrat ostavio dubok dojam. Poslije Sokratove smrti (399.\ pr.~Kr.), u nizu djela Ksenofont s raznih aspekata razmatra Sokratovo suđenje. To su takozvani sokratski spisi: \textit{Obrana Sokratova} \textgreek[variant=ancient]{(Ἀπολογία Σωκράτους)}, \textit{Uspomene na Sokrata} \textgreek[variant=ancient]{(Ἀπομνημονεύματα Σωκράτους)}, \textit{Gozba} \textgreek[variant=ancient]{(Συμπόσιον)} i \textit{Rasprava o gospodarstvu} \textgreek[variant=ancient]{(Οἰκονομικός)}. Ksenofont nije osobno svjedočio sudskom procesu Sokratu; u to je vrijeme bio u Perziji, kao grčki plaćenik u vojnom pohodu Kira Mlađeg.

U četvrtoj knjizi \textit{Uspomena na Sokrata}, odakle potječe ovaj odlomak, Ksenofont Sokratove poglede na obrazovanje predstavlja kroz dijalog Sokrata i njegova učenika, lijepog Eutidema. Razumnost ili umjerenost, \textgreek[variant=ancient]{σωφροσύνη,} prva je stvar koju Sokrat zahtijeva, a potrebna je i u odnosu prema bogovima. Opis načina na koji su bogovi ljudima uredili svijet smatra se jednim od najranijih primjera teleološkog argumenta.

\newpage

\section*{Pročitajte naglas grčki tekst.}

Xen. Memorabilia 4.3.3

%Naslov prema izdanju

\medskip


{\large

\begin{greek}

\noindent εἰπέ μοι, ἔφη, ὦ Εὐθύδημε, ἤδη ποτέ σοι ἐπῆλθεν ἐνθυμηθῆναι ὡς ἐπιμελῶς οἱ θεοὶ ὧν οἱ ἄνθρωποι δέονται κατεσκευάκασι; καὶ ὅς, μὰ τὸν Δίʼ, ἔφη, οὐκ ἔμοιγε. 

ἀλλʼ οἶσθά γʼ, ἔφη, ὅτι πρῶτον μὲν φωτὸς δεόμεθα, ὃ ἡμῖν οἱ θεοὶ παρέχουσι; 

νὴ Δίʼ, ἔφη, ὅ γʼ εἰ μὴ εἴχομεν, ὅμοιοι τοῖς τυφλοῖς ἂν ἦμεν ἕνεκά γε τῶν ἡμετέρων ὀφθαλμῶν. 

ἀλλὰ μὴν καὶ ἀναπαύσεώς γε δεομένοις ἡμῖν νύκτα παρέχουσι κάλλιστον ἀναπαυτήριον.

πάνυ γʼ, ἔφη, καὶ τοῦτο χάριτος ἄξιον. 

οὐκοῦν καὶ ἐπειδὴ ὁ μὲν ἥλιος φωτεινὸς ὢν τάς τε ὥρας τῆς ἡμέρας ἡμῖν καὶ τἆλλα πάντα σαφηνίζει, ἡ δὲ νὺξ διὰ τὸ σκοτεινὴ εἶναι ἀσαφεστέρα ἐστίν, ἄστρα ἐν τῇ νυκτὶ ἀνέφηναν, ἃ ἡμῖν τῆς νυκτὸς τὰς ὥρας ἐμφανίζει, καὶ διὰ τοῦτο πολλὰ ὧν δεόμεθα πράττομεν; 

ἔστι ταῦτα, ἔφη. 

ἀλλὰ μὴν ἥ γε σελήνη οὐ μόνον τῆς νυκτός, ἀλλὰ καὶ τοῦ μηνὸς τὰ μέρη φανερὰ ἡμῖν ποιεῖ.

πάνυ μὲν οὖν, ἔφη. 

τὸ δʼ, ἐπεὶ τροφῆς δεόμεθα, ταύτην ἡμῖν ἐκ τῆς γῆς ἀναδιδόναι καὶ ὥρας ἁρμοττούσας πρὸς τοῦτο παρέχειν, αἳ ἡμῖν οὐ μόνον ὧν δεόμεθα πολλὰ καὶ παντοῖα παρασκευάζουσιν, ἀλλὰ καὶ οἷς εὐφραινόμεθα;

πάνυ, ἔφη, καὶ ταῦτα φιλάνθρωπα.

\end{greek}

}


\section*{Komentar}

%1

\begin{description}[noitemsep]
\item[ἤδη ποτέ] vremenski: “već kada”
\item[ὡς ἐπιμελῶς] ὡς kao adverb u pokaznom značenju, “tako brižno” (Smyth § 2988)
\item[ὧν οἱ ἄνθρωποι δέονται] odnosna zamjenica ὧν uvodi odnosnu rečenicu, “od onoga što\dots”, dopunjava misao οἱ θεοὶ κατεσκευάκασι
\end{description}

%2

\begin{description}[noitemsep]
\item[καὶ ὅς\dots\ ἔφη] § 214.5; kombinacija καὶ ὅς na početku označava promjenu subjekta/govornika, “a on\dots”
\end{description}

%3


%4


%5

\begin{description}[noitemsep]
\item[᾿Αλλὰ μὴν καὶ] § 515, kombinacija čestica izražava potvrđivanje prethodne misli novom mišlju, “i stvarno, doista”
\end{description}


%6



\begin{description}[noitemsep]
\item[Πάνυ γ'] < Πάνυ γε, čestica naglašava prilog: “svakako, bez sumnje”
\item[τοῦτο\dots\ ἄξιον] u imenskom predikatu izostavljena kopula (ἐστιν)
\end{description}


%7


\begin{description}[noitemsep]
\item[Οὐκοῦν] § 520.b, § 516.2; pitanja koja uvodi čestica Οὐκοῦν karakteristična su za diskurs filozofskog razgovora, osim kod Ksenofonta česta su kod Platona
\item[ἐπειδὴ... σαφηνίζει] uzročna rečenica, “budući da...”
\end{description}


%8

%9


\begin{description}[noitemsep]
\item[᾿Αλλὰ μὴν ] § 515.4, “ali zasigurno”, ova kombinacija čestica označava povratak na prethodnu misao i njezin nastavak
\end{description}


%10


%11


%12

%kraj
