%\section*{O autoru}


\section*{O tekstu}

Od Tukidida saznajemo za atenski običaj godišnjih javnih sahrana onih koji su pali u borbi za državu; tom bi prigodom neki ugledan građanin održao javni nadgrobni govor. Znamenit je primjer Periklova govora (Thuc.~2, 35–46). I drugi tekst sačuvan u Lizijinu korpusu (Lys.~2) takav je govor u čast palih u Korintskom ratu; rat je izbio 395, kad se Atena pridružila Korintu, Argu i Tebi u pobuni protiv Sparte. Rat je trajao je do 387, kad su, uz perzijsku podršku, Spartanci situaciju uspjeli okrenuti u svoju korist.

Lys.~2 jedan je od dva Lizijina teksta koji nisu namijenjeni sudskim procesima. Stilom odudara od sudskih, i zato su filolozi dovodili u pitanje Lizijino autorstvo (označava se atributom \textit{spurius}, ``dvojbena porijekla''). Osim toga, Lizija, koji nije bio atenski građanin, teško da je govor držao osobno, a malo je vjerojatno i da bi ga pisao za drugoga; građanin dovoljno ugledan (to znači, dovoljno politički istaknut) da bude govornik u ovoj prigodi vjerojatno bi i sam mogao sastaviti govor. Ovaj je tekst možda retorička vježba s početka IV.~st.\ p.~n.~e, ili u to vrijeme nastao pamflet namijenjen cirkulaciji u pisanom obliku. No, nemamo dovoljno podataka da bilo koju od ovih teorija opovrgnemo ili potvrdimo.

Važan je dio nadgrobnih govora veličanje postignuća predaka. U tom kontekstu, odmah na početku govora, Lizija prikazuje napad Amazonki na Atenu; drevni Atenjani dokazali su da su ove zastrašujuće ratnice bile ipak samo žene, i nijedna se od njih nije vratila u domovinu da bi izvukla pouku iz ovog nepromišljenog čina.


\newpage

\section*{Pročitajte naglas grčki tekst.}

Lys.\ Epitaphius [Sp.] 4–6

%Naslov prema izdanju

\medskip


{\large

\begin{greek}

\noindent Ἀμαζόνες γὰρ Ἄρεως μὲν τὸ παλαιὸν ἦσαν θυγατέρες, οἰκοῦσαι δὲ παρὰ τὸν Θερμώδοντα ποταμόν, μόναι μὲν ὡπλισμέναι σιδήρῳ τῶν περὶ αὐτάς, πρῶται δὲ τῶν πάντων ἐφʼ ἵππους ἀναβᾶσαι, οἷς ἀνελπίστως διʼ ἀπειρίαν τῶν ἐναντίων ᾕρουν μὲν τοὺς φεύγοντας, ἀπέλειπον δὲ διώκοντας· ἐνομίζοντο δὲ διὰ τὴν εὐψυχίαν μᾶλλον ἄνδρες ἢ διὰ τὴν φύσιν γυναῖκες· πλέον γὰρ ἐδόκουν τῶν ἀνδρῶν ταῖς ψυχαῖς διαφέρειν ἢ ταῖς ἰδέαις ἐλλείπειν.

ἄρχουσαι δὲ πολλῶν ἐθνῶν, καὶ ἔργῳ μὲν τοὺς περὶ αὐτὰς καταδεδουλωμέναι, λόγῳ δὲ περὶ τῆσδε τῆς χώρας ἀκούουσαι κλέος μέγα, πολλῆς δόξης καὶ μεγάλης ἐλπίδος χάριν παραλαβοῦσαι τὰ μαχιμώτατα τῶν ἐθνῶν ἐστράτευσαν ἐπὶ τήνδε τὴν πόλιν. τυχοῦσαι δʼ ἀγαθῶν ἀνδρῶν ὁμοίας ἐκτήσαντο τὰς ψυχὰς τῇ φύσει, καὶ ἐναντίαν τὴν δόξαν τῆς προτέρας λαβοῦσαι μᾶλλον ἐκ τῶν κινδύνων ἢ ἐκ τῶν σωμάτων ἔδοξαν εἶναι γυναῖκες.

μόναις δʼ αὐταῖς οὐκ ἐξεγένετο ἐκ τῶν ἡμαρτημένων μαθούσαις περὶ τῶν λοιπῶν ἄμεινον βουλεύσασθαι, οὐδʼ οἴκαδε ἀπελθούσαις ἀπαγγεῖλαι τήν τε σφετέραν αὐτῶν δυστυχίαν καὶ τὴν τῶν ἡμετέρων προγόνων ἀρετήν· αὐτοῦ γὰρ ἀποθανοῦσαι, καὶ δοῦσαι δίκην τῆς ἀνοίας, τῆσδε μὲν τῆς πόλεως διὰ τὴν ἀρετὴν ἀθάνατον τὴν μνήμην ἐποίησαν, τὴν δὲ ἑαυτῶν πατρίδα διὰ τὴν ἐνθάδε συμφορὰν ἀνώνυμον κατέστησαν. ἐκεῖναι μὲν οὖν τῆς ἀλλοτρίας ἀδίκως ἐπιθυμήσασαι τὴν ἑαυτῶν δικαίως ἀπώλεσαν.

\end{greek}

}

\newpage

\section*{Komentar}

%1

\begin{description}[noitemsep]
\item[γὰρ] čestica najavljuje iznošenje objašnjenja (tvrdnje iz prethodne rečenice): naime\dots
\item[πρῶται] u rečenici otvara mjesto (dijelnom) genitivu τῶν πάντων
\item[τῶν πάντων] supstantivirani pridjev § 373
\item[οἷς] odnosna zamjenica uvodi zavisnu odnosnu rečenicu (antecedent ἵππους); dativ instrumentalni § 414: pomoću njih\dots
\item[ᾕρουν] LSJ αἱρέω II
\item[γὰρ] čestica najavljuje iznošenje objašnjenja: naime\dots
\item[ἐλλείπειν] složenica λείπω; LSJ ἐλλείπω A.5, rekcija τινός τινι u odnosu na nekog po nečemu
\end{description}

%2

\begin{description}[noitemsep]
\item[ἔργῳ μὲν\dots\ λόγῳ δὲ\dots] koordinacija rečeničnih članaka pomoću para čestica
\item[περὶ τῆσδε τῆς χώρας] rečeno iz pozicije govornika (Lizije), koji govori u Ateni (na groblju u Keramiku)
\item[παραλαβοῦσαι] složenica glagola λαμβάνω; LSJ παραλαμβάνω A.4
\item[ἐστράτευσαν] rekcija ἐπί τι na nešto, protiv nečega

\end{description}

%3

\begin{description}[noitemsep]
\item[τυχοῦσαι] rekcija (objekta) τινός
\item[ὁμοίας] rekcija τινί
\item[ἐναντίαν] rekcija τινός
\item[ἔδοξαν] § 267; § 325.2; kao \textit{verbum sentiendi}, glagol otvara u rečenici mjesto dopuni, ovdje je to infinitiv
\item[εἶναι γυναῖκες] §~315; kopulativni glagol otvara mjesto nužnoj predikatnoj dopuni (imenski predikat, Smyth 910)
\end{description}

%4

\begin{description}[noitemsep]
\item[ἐξεγένετο] bezlično, složenica γίγνομαι, LSJ ἐκγίγνομαι III; glagol u rečenici otvara mjesto dativu osobe (τινί) i infinitivu (ovdje ih ima više)
\item[βουλεύσασθαι] LSJ βουλεύω B.1
\item[τήν τε\dots\ καὶ τὴν] koordinacija sastavnih veznika, pri čemu je drugi član para naglašeniji
\item[αὐτοῦ] prilog mjesta, LSJ s.\ v.
\item[δοῦσαι δίκην] \textgreek[variant=ancient]{δίκας διδόναι τινός LSJ δίκη} IV.3
\item[τῆσδε μὲν\dots\ τὴν δὲ\dots] koordinacija rečeničnih članova
\item[ἀθάνατον\dots\ ἐποίησαν] \textgreek[variant=ancient]{ποιῆσαί τινα} s pridjevom LSJ \textgreek[variant=ancient]{ποιέω} III
\item[ἀνώνυμον κατέστησαν] složenica glagola ἵστημι; rekcija τινά s pridjevom LSJ \textgreek[variant=ancient]{καθίστημι} II.4

\end{description}


\begin{description}[noitemsep]
\item[τῆς ἀλλοτρίας] sc.\ χώρας
\item[ἐπιθυμήσασαι] rekcija τινος
\item[ἀδίκως\dots\ δικαίως] antiteza
\item[ἀπώλεσαν] LSJ ἀπόλλυμι II
\end{description}

%kraj
