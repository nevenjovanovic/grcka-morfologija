\section*{O autoru}

Heliodor iz Emese \textgreek[variant=ancient]{(Ἡλιόδορος ὁ Ἐμεσηνός),} grčki je pisac carskoga razdoblja, čiji se \textit{floruit} datira dvojako, ili oko 250.\ ili oko 363. Najpouzdaniji podatak o autoru potječe od njega samog; doznajemo da je bio Feničanin iz Emese, današnjega sirijskoga grada Homsa.

Heliodor je \textit{Etiopske priče o Teagenu i Harikleji} \textgreek[variant=ancient]{(Αἰθιοπικὰ τὰ περὶ Θεαγένην καὶ Χαρίκλειαν)} napisao u tradiciji grčkog ljubavnog (i pustolovnog) romana. No, po nekim se obilježjima Heliodorove priče mogu povezati i s Herodotom i s Homerom. Heliodora, naime, zanimaju stvari slične onima koje je Herodot zabilježio pišući o Egiptu: navike i običaji stranih naroda, neobični podaci iz stranih krajeva, vojna taktika i prirodne znanosti. Iz Heliodorova teksta očito je dobro poznavanje Egipta.


\section*{O tekstu}

Etiopska kraljica Persina zagledala se trudna u sliku gole Andromede. Prenijevši dojam slike na dijete u utrobi, rodila je bjeloputu djevojčicu. U strahu od optužbi da je prevarila supruga, kralja Hidaspa, odlučila je odreći se djeteta. Gimnosofist Sisimitra odveo je djevojčicu u Egipat i predao je pitijskom svećeniku Hariklu. Iz Egipta je djevojčica, imenom Harikleja, prešla u Delfe i postala Artemidina svećenica. U nju se zaljubio Teagen, mladić plemenita porijekla. No, par je prije sretnoga kraja morao proći kroz razne avanture i opasnosti, iskušavajući svoju ljubav.

Utjecaji žanra grčkog romana vidljivi su iz teme (ljubav mladoga para na kušnjama) i pojedinih motiva (gusari, otmice, lažna smrt\dots). No, Heliodor nasljeduje i Homera, napose u vještini s kojom planira ukupnu strukturu zapleta. Poput \textit{Odiseje, Etiopske priče} počinju \textit{in medias res}; radnja se zatim vraća u prošlost kako bi se pripovijest dovela do središnje točke od koje se postupno i polako kreće do kulminacije. Struktura djela, koja narušava kronološki slijed fabule, smatra se važnom Heliodorovom inovacijom. 

Roman je snažno odjeknuo u bizantskoj književnosti te su ga imitirali bizantski grčki autori (Teodor Prodrom napisao je pod Heliodorovim utjecajem roman \textit{Priča o Rodanti i Dosiklu,} Niketa Eugenijan roman o Drosili i Hariklu). Heliodor je ostavio traga i u potonjoj europskoj književnosti. Navodno su Racineu \textit{Etiopske priče} bila najdraža knjiga, a Cervantes je djelo \textit{Persiles i Sigismunda ili priča sa sjevernih strana} oblikovao prema Heliodorovu romanu.

U ekscerptu koji čitamo uočljiva je fascinacija grčkog ljubavnog romana čudesnim i fantastičnim. Etiopski kralj Hidasp prima čudesne darove sa svih strana svijeta; posljednji mu svoje poklone prinose saveznici Aksiomiti, koji mu, među ostalim, daruju neobičnu životinju kameloparda (\textgreek[variant=ancient]{καμηλοπάρδαλις} je grčka riječ za žirafu).



%\newpage

\section*{Pročitajte naglas grčki tekst.}

Heliod.\ Aethiopica 10.27
%Naslov prema izdanju

\medskip


{\large

\begin{greek}

\noindent δῶρα καὶ οὗτοι προσῆγον, ἄλλα τε καὶ δὴ καὶ ζώου τινὸς εἶδος ἀλλοκότου τε ἅμα καὶ θαυμασίου τὴν φύσιν, μέγεθος μὲν εἰς καμήλου μέτρον ὑψούμενον, χροιὰν δὲ καὶ δορὰν παρδάλεως φολίσιν ἀνθηραῖς ἐστιγμένον. 

\noindent ἦν δὲ αὐτῷ τὰ μὲν ὀπίσθια καὶ μετὰ κενεῶνας χαμαίζηλά τε καὶ λεοντώδη, τὰ δὲ ὠμιαῖα καὶ πόδες πρόσθιοι καὶ στέρνα πέρα τοῦ ἀναλόγου τῶν ἄλλων μελῶν ἐξανιστάμενα. λεπτὸς ὁ αὐχήν, καὶ ἐκ μεγάλου τοῦ λοιποῦ σώματος εἰς κύκνειον φάρυγγα μηκυνόμενος. ἡ κεφαλὴ τὸ μὲν εἶδος καμηλίζουσα, τὸ δὲ μέγεθος στρουθοῦ Λιβύσσης εἰς διπλάσιον ὀλίγον ὑπερφέρουσα, καὶ ὀφθαλμοὺς ὑπογεγραμμένους βλοσυρῶς σοβοῦσα. παρήλλακτο καὶ τὸ βάδισμα χερσαίου τε ζώου καὶ ἐνύδρου παντὸς ὑπεναντίως σαλευόμενον, τῶν σκελῶν οὐκ ἐναλλὰξ ἑκατέρου καὶ παρὰ μέρος ἐπιβαίνοντος, ἀλλ̓ ἰδίᾳ μὲν τοῖν δυοῖν καὶ ἅμα τῶν ἐν δεξιᾷ, χωρὶς δὲ καὶ ζυγηδὸν τῶν εὐωνύμων σὺν ἑκατέρᾳ τῇ ἐπαιωρουμένῃ πλευρᾷ μετατιθεμένων. ὁλκὸν δὲ οὕτω τὴν κίνησιν καὶ τιθασὸν τὴν ἕξιν ὥστε ὑπὸ λεπτῆς μηρίνθου, τῇ κορυφῇ περιελιχθείσης, ἄγεσθαι πρὸς τοῦ θηροκόμου, καθάπερ ἀφύκτῳ δεσμῷ τῷ ἐκείνου βουλήματι ὁδηγούμενον. 

\noindent τοῦτο φανὲν τὸ ζῶον τὸ μὲν πλῆθος ἅπαν ἐξέπληξε, καὶ ὄνομα τὸ εἶδος ἐλάμβανεν, ἐκ τῶν ἐπικρατεστέρων τοῦ σώματος αὐτοσχεδίως πρὸς τοῦ δήμου καμηλοπάρδαλις κατηγορηθέν\dots


\end{greek}

}

%\newpage

\section*{Komentar}

%1

\begin{description}[noitemsep]
\item[καὶ δὴ καὶ] kombinacija čestica označava prijelaz od općenitog iskaza prema posebnom: a napose\dots
\item[ἀλλοκότου τε\dots\ καὶ θαυμασίου] koordinacija rečeničnih članova pomoću postpozitivnog τε i καὶ
\item[τὴν φύσιν\dots] \textbf{μέγεθος\dots\ χροιὰν δὲ καὶ δορὰν} niz akuzativa obzira, §~389: obzirom na\dots
\item[ὑψούμενον] §~243; atributivni particip  kao dopuna imenskoj riječi §~499
\item[μέγεθος μὲν\dots\ χροιὰν δὲ] koordinacija rečeničnih članova pomoću čestica μέν i δέ u opisu obilježja
\item[ἐστιγμένον] §~291.b, atributivni particip kao dopuna imenskoj riječi §~499
\end{description}

%2

\begin{description}[noitemsep]
\item[῏Ην] kopulativni glagol otvara mjesto nužnoj predikatnoj dopuni (imenski predikat, Smyth §~910); ἐστί τινι imati, LSJ εἰμί III
\item[ἐξανιστάμενα] predikatni particip kao dopuna kopulativnom glagolu imenskoga predikata §~500; Senc ἐξανίστημι A.1.a
\item[τὸ μὲν εἶδος\dots\ τὸ μέγεθος δὲ] koordinirani akuzativi obzira, §~389
\item[καμηλίζουσα\dots\ ὑπερφέρουσα\dots\ σοβοῦσα] sva tri predikatna participa odnose se na subjekt rečenice §~500
\item[καμηλίζουσα] §~231, §~500; glagol izveden od κάμηλος
\item[ὀλίγου] priložno, modificira priložnu oznaku εἰς διπλάσιον; LSJ ὀλίγος IV.1
\end{description}

%3
\begin{description}[noitemsep]
\item[χερσαίου τε\dots\ καὶ ἐνύδρου] povezivanje rečeničnih članova pomoću sastavnih veznika (τε, καί)
\item[οὐκ ἐναλλὰξ\dots] \textbf{ἀλλ' ἰδίᾳ\dots}\ koordinacija suprotstavljenih članova
\item[παρὰ μέρος] LSJ παρά C.I.9 i μέρος II.2
\item[καὶ ἅμα] \textbf{\dots\ χωρὶς δὲ καὶ ζυγηδὸν} koordinacija suprotstavljenih priloga
\item[τῶν ἐν δεξιᾷ] sc.\ τῶν σκελῶν; supstantivirana priložna oznaka, ujedno dio genitiva apsolutnog ovisan o \textgreek[variant=ancient]{μετατιθεμένων}
\item[τῶν εὐωνύμων] sc.\ τῶν σκελῶν; dio genitiva apsolutnog ovisan o \textgreek[variant=ancient]{μετατιθεμένων}
\end{description}

%4

\begin{description}[noitemsep]
\item[ὁλκὸν\dots\ καὶ τίθασον] sc.\ τὸ ζῷον
\item[περιελιχθείσης] složenica glagola ἑλίσσω; rekcija: τινι; §~296, atributivni particip kao dopuna imenici u istom padežu
\item[ὥστε\dots\ ἄγεσθαι] veznik ὥστε otvara mjesto zavisnoj posljedičnoj rečenici (s pomišljenom, mogućom posljedicom) §~473
\item[καθάπερ\dots\ ὁδηγούμενον] veznik καθάπερ uvodi zavisnu poredbenu rečenicu
\item[ὁδηγούμενον] sc.\ τὸ ζῷον; §~243, predikatni particip čije se značenje odnosi na subjekt §~500

\end{description}

%5
\begin{description}[noitemsep]
\item[φανὲν] §~292; predikatni particip koji se odnosi na subjekt, §~500
\item[τὸ μὲν πλῆθος ἅπαν] tj.\ ljudi koji su promatrali povorku s darovima
\item[τὸ εἶδος] akuzativ obzira, §~389: prema\dots
\item[τῶν ἐπικρατεστέρων] supstantiviran pridjev, u prijevodu dodaj imenicu poput ``obilježje'' ili ``oznaka''
\item[πρὸς τοῦ δήμου] narod je promatrao povorku
\end{description}


%kraj
