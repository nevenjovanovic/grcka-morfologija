%\section*{O autoru}

%TKTK


\section*{O tekstu}

Lizija \textgreek[variant=ancient]{(Λυσίας,} oko 445.\ – nakon 380.\ p.~n.~e.) djelovao je u Ateni kao \textgreek[variant=ancient]{λογογράφος,} profesionalni autor sudskih govora za druge; od nekoliko stotina njegovih govora sačuvano je tridesetak.

Oko 387.\ Liziju je angažirao nepoznati Atenjanin, sin tuženoga (ovaj je u međuvremenu preminuo) i šogor Aristofanov \textgreek[variant=ancient]{(Ἀριστοφάνης;} on nema veze s komičkim pjesnikom). Aristofan je, zajedno s ocem Nikofemom, iz nepoznatih razloga smaknut oko 390, a njihova je imovina \textgreek[variant=ancient]{(χρήματα)} u Ateni konfiscirana. Međutim, vrijednost imetka bila je ispod onoga što je javnost očekivala (Aristofan i Nikofem bili su bliski dvojici vrlo uspješnih vojskovođa, Kononu i Euagori). Zato je Aristofanov tast, otac Lizijina klijenta (za kojeg je govor napisan i koji je govor na sudu održao), optužen da je otuđio dio imovine u korist svoje kćeri i unuka. Optužba je tražila da Lizijin klijent državnoj riznici nadoknadi dio Aristofanove imovine koji je njegov otac navodno otuđio.

Lizijin govor ima dva cilja. Prvo, pokušava pokazati da veza Aristofana i klijentova oca nije bila tako tijesna kao što se tvrdi; bilo je drugih kojima je Aristofan više vjerovao, kojima bi prije povjerio vođenje svojih poslova, a govornikov otac, iako rodoljub, nije bio zainteresiran za politiku. Drugo, dokazuje da Aristofan i nije bio onoliko bogat koliko ljudi misle; imao je velike izdatke da bi postigao ugled u društvu, a i mnogi drugi ugledni Atenjani, pokazalo se nakon njihove smrti, bili su zapravo puno siromašniji nego što bi se očekivalo od njih, od njihovih očeva ili djedova.

Odabrani je odlomak dio argumentacije \textgreek[variant=ancient]{(πίστεις),} koja u govoru slijedi odmah nakon proslova \textgreek[variant=ancient]{(προοίμιον).} Govornik dokazuje da nije vjerojatno \textgreek[variant=ancient]{(εἰκός)} da je njegov otac posjedovao Aristofanovu imovinu, zato što ženidbena veza s Nikofemovom obitelji nije proizašla iz novčanih interesa, jednako kao što ni ostali brakovi u govornikovoj obitelji (oca s govornikovom majkom, govornikove druge sestre, govornika samog) nisu sklapani radi financijske koristi.


%\newpage

\section*{Pročitajte naglas grčki tekst.}

Lys.\ 19.\ Ὑπὲρ τῶν Ἀριστοφάνους χρημάτων 14–17

%Naslov prema izdanju

\medskip


{\large

\begin{greek}

\noindent  ἐκεῖνος γὰρ ὅτʼ ἦν ἐν τῇ ἡλικίᾳ, παρὸν μετὰ πολλῶν χρημάτων γῆμαι ἄλλην, τὴν ἐμὴν μητέρα ἔλαβεν οὐδὲν ἐπιφερομένην, ὅτι δὲ Ξενοφῶντος ἦν θυγάτηρ τοῦ Εὐριπίδου ὑέος, ὃς οὐ μόνον ἰδίᾳ χρηστὸς ἐδόκει εἶναι, ἀλλὰ καὶ στρατηγεῖν αὐτὸν ἠξιώσατε, ὡς ἐγὼ ἀκούω.

τὰς τοίνυν ἐμὰς ἀδελφὰς ἐθελόντων τινῶν λαβεῖν ἀπροίκους πάνυ πλουσίων οὐκ ἔδωκεν, ὅτι ἐδόκουν κάκιον γεγονέναι, ἀλλὰ τὴν μὲν Φιλομήλῳ τῷ Παιανιεῖ, ὃν οἱ πολλοὶ βελτίω ἡγοῦνται εἶναι ἢ πλουσιώτερον, τὴν δὲ πένητι γεγενημένῳ οὐ διὰ κακίαν, ἀδελφιδῷ δὲ ὄντι Φαίδρῳ τῷ Μυρρινουσίῳ, ἐπιδοὺς τετταράκοντα μνᾶς, κᾆτʼ Ἀριστοφάνει τὸ ἴσον. πρὸς δὲ τούτοις ἐμοὶ πολλὴν ἐξὸν πάνυ προῖκα λαβεῖν ἐλάττω συνεβούλευσεν, ὥστε εὖ εἰδέναι ὅτι κηδεσταῖς χρησοίμην κοσμίοις καὶ σώφροσι. καὶ νῦν ἔχω γυναῖκα τὴν Κριτοδήμου θυγατέρα τοῦ Ἀλωπεκῆθεν, ὃς ὑπὸ Λακεδαιμονίων ἀπέθανεν, ὅτε ἡ ναυμαχία ἐγένετο ἐν Ἑλλησπόντῳ.

καίτοι, ὦ ἄνδρες δικασταί, ὅστις αὐτός τε ἄνευ χρημάτων ἔγημε τοῖν τε θυγατέροιν πολὺ ἀργύριον ἐπέδωκε τῷ τε ὑεῖ ὀλίγην προῖκα ἔλαβε, πῶς οὐκ εἰκὸς περὶ τούτου πιστεύειν ὡς οὐχ ἕνεκα χρημάτων τούτοις κηδεστὴς ἐγένετο;


\end{greek}

}


\section*{Komentar}

%1


%2

\begin{description}[noitemsep]
\item[τοίνυν] zaključni veznik, § 516.3
\item[κάκιον γεγονέναι] § 272; § 325.11; predikatni pridjev u srednjem rodu singulara slaže se sa subjektom muškog roda, § 365
\end{description}

%3
\begin{description}[noitemsep]
\item[πολλὴν\dots\ πάνυ] prilog modificira pridjev, LSJ πάνυ A.1
\item[ὥστε εὖ εἰδέναι] § 317.4; ὥστε u zavisno posljedičnoj rečenici s infinitivom, § 473; \textit{verbum sentiendi} otvara mjesto vezniku ὅτι i zavisno izričnoj rečenici, § 467
\item[ὅτι\dots\ χρησοίμην] s.\ 116; § 258; optativ u zavisno izričnoj rečenici iza sporednog (historijskog) vremena, § 467
\end{description}

%4
\begin{description}[noitemsep]
\item[ὑπὸ Λακεδαιμονίων ἀπέθανεν] § 254; § 324.8; \textgreek[variant=ancient]{ἀποθνῄσκω ὑπό τινος} kao pasiv glagola \textgreek[variant=ancient]{ἀποκτείνω, LSJ ἀποθνῄσκω} II.
\item[ὅτε\dots\ ἐγένετο] § 254; § 325.11; LSJ γίγνομαι A.I.3; veznik ὅτε uvodi zavisnu vremensku rečenicu, § 487
\end{description}

%5

%kraj
