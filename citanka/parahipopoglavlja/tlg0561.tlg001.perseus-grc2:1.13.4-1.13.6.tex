\section*{O autoru}

Poput Ahileja Tacija, Haritona, Ksenofonta Efeškog i Heliodora, \textgreek[variant=ancient]{Λόγγος} (Longo, II–III st.\ n.~e.) je autor helenističkog ljubavnog romana. O Longu nema pouzdanih biografskih podataka, čak ni što se imena tiče; ni njegova povezanost s otokom Lezbom nije zajamčena.

\section*{O tekstu}

Longov roman \textgreek[variant=ancient]{Τῶν κατὰ Δάφνιν καὶ Χλόην λόγοι Δ} (\textit{Pripovijesti o Dafnisu i Hloji}), ili \textgreek[variant=ancient]{Ποιμενικὰ τὰ κατὰ Δάφνιν καὶ Χλόην} (\textit{Pastirske zgode Dafnisa i Hloje}), kraće \textgreek[variant=ancient]{Δάφνις καὶ Χλόη,} pripovijeda tipičnu priču grčkog ljubavnog romana, ali na mnogo jednostavniji način. Dvoje prekrasnih mladih ljudi zaljubi se jedno u drugo, ali ljubav nailazi na niz prepreka koje im ne daju da budu zajedno; napokon, nakon niza peripetija koje uključuju i dva neočekivana prepoznavanja, sve se rješava i otkriva se da Dafnis i Hloja nisu ubogi pastiri, nego bogati zemljoposjednici; nakon svadbe, oni žive sretni i zadovoljni na rodnome Lezbu.

Specifičnost je Longova romana bukolski, pastirski ambijent, i nadomještanje ``geografskih'' peripetija (putovanja razdvojenih ljubavnika) psihološkima: dvoje mladih polako i postupno shvaća da su zaljubljeni, prolazeći svojevrstan ``sentimentalni odgoj'' (grčki roman inače konvencionalno prikazuje ``ljubav na prvi pogled''); također, ritam pastirskog života, koji izaziva razdvojenost zaljubljenih, u ovom romanu nadomješta igre sudbine i mahinacije zlotvora.

Odlomak iz prve knjige \textit{Pastirskih zgoda Dafnisa i Hloje} prikazuje petnaestogodišnjeg Dafnisa i trinaestogodišnju Hloju, čuvare ovaca i koza, koji se u proljeće, provodeći na paši čitav dan zajedno, svake večeri razilaze, svatko svojoj kući. Dafnis je jednog dana upao u zamku za vukove, izvukao se neozlijeđen, ali se morao oprati; Hloja je ostala opčinjena njegovom ljepotom, ali je smatrala da je uzrok te ljepote kupanje. Sljedećeg dana, dok stada pasu, Dafnis svira siringu (Panovu frulu).

%\newpage

\section*{Pročitajte naglas grčki tekst.}

Longus scr.~erot.\ Daphnis et Chloe 1.13.4–1.13.6

%Naslov prema izdanju

\medskip


{\large

\begin{greek}

\noindent Τῆς δὲ ἐπιούσης ὡς ἧκον εἰς τὴν νομήν, ὁ μὲν Δάφνις ὑπὸ τῇ δρυῒ τῇ συνήθει καθεζόμενος ἐσύριττε καὶ ἅμα τὰς αἶγας ἐπεσκόπει κατακειμένας καὶ ὥσπερ τῶν μελῶν ἀκροωμένας, ἡ δὲ Χλόη πλησίον καθημένη τὴν ἀγέλην μὲν τῶν προβάτων ἐπέβλεπε, τὸ δὲ πλέον εἰς Δάφνιν ἑώρα· καὶ ἐδόκει καλὸς αὐτῇ συρίττων πάλιν, καὶ αὖθις αἰτίαν ἐνόμιζε τὴν μουσικὴν τοῦ κάλλους, ὥστε μετʼ ἐκεῖνον καὶ αὐτὴ τὴν σύριγγα ἔλαβεν, εἴ πως γένοιτο καὶ αὐτὴ καλή.

Ἔπεισε δὲ αὐτὸν καὶ λούσασθαι πάλιν καὶ λουόμενον εἶδε καὶ ἰδοῦσα ἥψατο καὶ ἀπῆλθε πάλιν ἐπαινέσασα, καὶ ὁ ἔπαινος ἦν ἔρωτος ἀρχή. Ὅ τι μὲν οὖν ἔπασχεν οὐκ ᾔδει νέα κόρη καὶ ἐν ἀγροικίᾳ τεθραμμένη καὶ οὐδὲ ἄλλου λέγοντος ἀκούσασα τὸ τοῦ ἔρωτος ὄνομα· ἄση δὲ αὐτῆς εἶχε τὴν ψυχήν, καὶ τῶν ὀφθαλμῶν οὐκ ἐκράτει καὶ πολλὰ ἐλάλει Δάφνιν· τροφῆς ἠμέλει, νύκτωρ ἠγρύπνει, τῆς ἀγέλης κατεφρόνει· νῦν ἐγέλα, νῦν ἔκλαεν· εἶτα ἐκάθευδεν, εἶτα ἀνεπήδα\dots

\end{greek}

}


\section*{Komentar}

%1
\begin{description}[noitemsep]
\item[ἐπιούσης] podrazumijeva se ἡμέρας; LSJ ἔπειμι B.II; § 314.1
\item[ἐσύριττε] § 231; συρίττω je atički oblik glagola συρίζω
\item[ἐνόμιζε] § 231; otvara mjesto dvama akuzativima LSJ νομίζω II
\item[εἴ πως] uvodi upitnu rečenicu: bi li kako\dots
\item[γένοιτο] § 254, osnove § 325.11; kao kopulativni glagol (nepotpuna značenja) traži imensku dopunu (ovdje pridjev)

\end{description}

%2

\begin{description}[noitemsep]
\item[ἥψατο] LSJ ἅπτω A.II (metaforično)
\end{description}


%3

\begin{description}[noitemsep]
\item[Ὅ τι μὲν\dots\ ἄση δὲ\dots] koordinacija rečeničnih članova; ὅ τι je oblik upitne zamjenice ὅστις, uvodi zavisno upitnu rečenicu
\item[πολλά] priložna upotreba, LSJ πολύς III.a
\item[ἐλάλει] rekcija: τινα \textit{o nekome}
\item[ἔκλαεν] κλάω je varijanta glagola κλαίω, LSJ s.~v.
\end{description}


%kraj
