\section*{O autoru}

Hariton \textgreek[variant=ancient]{(Χαρίτων),} autor ljubavnog romana Zgode Hereje i Kaliroje \textgreek[variant=ancient]{(Τὰ περὶ Χαιρέαν καὶ Καλλιρόην),} živio je najvjerojatnije između 50.\ pr.~Kr. i 50.\ po~Kr. Malo se zna o njemu osim onoga što je sam naveo na početku svog romana: da je živio u maloazijskom gradu Afrodizijadi i da je bio tajnik retora Atenagore.

\section*{O tekstu}

Roman \textit{Zgode Hereje i Kaliroje} dugo se smatrao kronološki posljednjim grčkim ljubavnim romanom antike, ali su papirusni nalazi s jedne i jezična analiza s druge strane pokazali da je zapravo jedan od najranijih. 

U romanu susrećemo tipične žanrovske motive: ljubav dvoje iznimno lijepih, plemenitih i osjećajnih mladih ljudi, putovanja egzotičnim zemljama, gusare. Glavni su junaci Hereja i Kaliroja, dvoje sicilskih Grka koji se zaljube na prvi pogled te se vrlo brzo i vjenčaju, no zavist drugih mladića koji su se borili za Kalirojinu ruku pokrene niz događaja zbog kojih Hereja mora krenuti u potragu za svojom dragom. Ta će ga potraga odvesti sve do Babilona, gdje će se za nju boriti na sudu, a potom će se istaknuti i kao sposoban vojskovođa u Egiptu. Za to vrijeme Kaliroja, oteta i prodana u roblje, za dobrobit svojeg nerođenog djeteta (čiji je otac Hereja) sklapa brak s maloazijskim Grkom Dionizijem. 

U odlomku koji slijedi Kaliroju dočekuju Dionizijeve sluškinje i dive se njezinoj ljepoti.


%\newpage

\section*{Pročitajte naglas grčki tekst.}

Charito, Scr.\ Erot.\ De Chaerea et Callirhoe 2.2.1–4
%Naslov prema izdanju

\medskip


{\large

\begin{greek}

\noindent Πρὸς δὲ τὴν Καλλιρρόην εἰσῆλθον αἱ ἄγροικοι γυναῖκες καὶ εὐθὺς ὡς δέσποιναν ἤρξαντο κολακεύειν. Πλαγγὼν δέ, ἡ τοῦ οἰκονόμου γυνή, ζῶον οὐκ ἄπρακτον, ἔφη πρὸς αὐτὴν ``ζητεῖς μέν, ὦ τέκνον, πάντως τοὺς σεαυτῆς· ἀλλὰ καλῶς καὶ τοὺς ἐνθάδε νόμιζε σούς· Διονύσιος γάρ, ὁ δεσπότης ἡμῶν, χρηστός ἐστι καὶ φιλάνθρωπος. Εὐτυχῶς σε ἤγαγεν εἰς ἀγαθὴν ὁ θεὸς οἰκίαν. Ὥσπερ ἐν πατρίδι διάξεις. Ἐκ μακρᾶς οὖν θαλάσσης ἀπόλουσαι τὴν ἄσιν· ἔχεις θεραπαινίδας.'' Μόλις μὲν καὶ μὴ βουλομένην, προήγαγε δ' ὅμως εἰς τὸ βαλανεῖον. Εἰσελθοῦσαν δὲ ἤλειψάν τε καὶ ἀπέσμηξαν ἐπιμελῶς καὶ μᾶλλον ἀποδυσαμένης κατεπλάγησαν· ὥστε ἐνδεδυμένης αὐτῆς θαυμάζουσαι τὸ πρόσωπον θεῖον πρόσωπον ἔδοξαν ἰδοῦσαι· ὁ χρὼς γὰρ λευκὸς ἔστιλψεν εὐθὺς μαρμαρυγῇ τινι ὅμοιον ἀπολάμπων· τρυφερὰ δὲ σάρξ, ὥστε δεδοικέναι μὴ καὶ ἡ τῶν δακτύλων ἐπαφὴ μέγα τραῦμα ποιήσῃ.

\noindent Ἡσυχῆ δὲ διελάλουν πρὸς ἀλλήλας ``καλὴ μὲν ἡ δέσποινα ἡμῶν καὶ περιβόητος· ταύτης δὲ ἂν θεραπαινὶς ἔδοξεν.'' Ἐλύπει τὴν Καλλιρρόην ὁ ἔπαινος καὶ τοῦ μέλλοντος οὐκ ἀμάντευτος ἦν. Ἐπεὶ δὲ ἐλέλουτο καὶ τὴν κόμην συνεδέσμουν, καθαρὰς αὐτῇ προσήνεγκαν ἐσθῆτας· ἡ δὲ οὐ πρέπειν ἔλεγε ταῦτα τῇ νεωνήτῳ.

\noindent ``Χιτῶνά μοι δότε δουλικόν· καὶ γὰρ ὑμεῖς ἐστέ μου κρείττονες.'' Ἐνεδύσατο μὲν οὖν τι τῶν ἐπιτυχόντων· κἀκεῖνο δὲ ἔπρεπεν αὐτῇ καὶ πολυτελὲς ἔδοξε καταλαμπόμενον ὑπὸ κάλλους.


\end{greek}

}

%\newpage

\section*{Komentar}

%1

\begin{description}[noitemsep]
\item[δὲ] čestica označava nadovezivanje na prethodnu rečenicu
\item[εὐθὺς] prilog od εὐθύς (LSJ εὐθύς B)
\item[κολακεύειν] rekcija τινά (hrv.\ laskati nekome)

\end{description}

%2
\begin{description}[noitemsep]
\item[πρὸς αὐτὴν] priložna oznaka koja ovdje zamjenjuje dativ 
\item[καὶ ἐνθάδε] sc.\ stanovnike ovog mjesta
\end{description}


%3

%4

\begin{description}[noitemsep]
\item[ἐκ μακρᾶς\dots\ θαλάσσης] od duge plovidbe morem
\item[ἀπόλουσαι] složenica λούω
\end{description}

%5

\begin{description}[noitemsep]
\item[μὲν\dots\ δ'] koordinacija parom čestica (prvi dio ima dopusno značenje, kako se naslućuje iz ὅμως)
\item[προήγαγε] složenica glagola ἄγω, s. 116, §~257

\end{description}

%6

\begin{description}[noitemsep]
\item[κατεπλάγησαν] složenica πλήσσω, §~327.10; §~292
\item[ὥστε] veznik otvara mjesto posljedičnoj rečenici §~473
\item[γὰρ] uvodi objašnjenje: naime\dots
\item[εὐθὺς] prilog od εὐθύς (LSJ εὐθύς B)
\item[ὅμοιον] prilog; rekcija ὅμοιος τινι (LSJ C)
\item[δὲ] nadovezivanje na prethodnu tvrdnju: a\dots
\item[τρυφερὰ δὲ σάρξ] u imenskom predikatu izostavljena je kopula
\item[μὴ] uvodi rečenicu uz \textit{verbum timendi}, §~471
\end{description}

%7
\begin{description}[noitemsep]
\item[πρὸς ἀλλήλας] međusobno
\end{description}

%8
\begin{description}[noitemsep]
\item[τοῦ μέλλοντος] §~394; supstantivirani particip u značenju ``budućnost''

\end{description}

%9

\begin{description}[noitemsep]
\item[συνεδέσμουν] složenica glagola δεσμεύω (δεσμέω, LSJ συνδεσμεύω) §~231, §~243
\item[πρέπειν] rekcija τινι
\end{description}

%10


%11
\begin{description}[noitemsep]
\item[ἐνεδύσατο] §~267; složenica glagola δύω
\item[ἐνεδύσατο μὲν\dots\ κἀκεῖνο δὲ] koordinacija rečeničnih dijelova pomoću para čestica
\item[τῶν ἐπιτυχόντων] τι otvara mjesto genitivu partitivnom §~395
\item[ἔδοξε] §~267, §~325.2; kopulativni glagol čija je dopuna πολυτελὲς
\end{description}



%kraj
