%\section*{O autoru}

%TKTK


\section*{O tekstu}

Sačuvana Plutarhova djela okupljena su u dvije zbirke: \textgreek[variant=ancient]{Βίοι παράλληλοι} (\textit{Usporedni životopisi}) i \textgreek[variant=ancient]{Ἠϑικά} (\textit{Moralia, Spisi o moralu}). \textit{Usporedni životopisi} donose biografije slavnih Grka i Rimljana, većinom historijskih ličnosti, u analognim parovima (Demosten i Ciceron, Aleksandar Veliki i Cezar, itd). Do nas je stiglo pedeset takvih tekstova u dvadeset jednom paru, jednoj četverostrukoj kombinaciji (Agid i Kleomen – Tiberije i Gaj Grakho) i četiri pojedinačna životopisa (među njima je i Artakserkso, jedini koji nije ni Grk ni Rimljanin). Životopisa je bilo više (izgubljen je npr. par Epaminonda – Scipion).

Parovi životopisa često su popraćeni rezimeom sličnosti i razlika \textgreek[variant=ancient]{(σύγκρισις),} gdje redovno prednost ostvaruje Grk; \textit{Usporedni životopisi} tako su i glorifikacija grčke prošlosti. No, Plutarha ne zanima toliko historiografija koliko biografija. On sam kaže \textgreek[variant=ancient]{οὔτε γὰρ ἱστορίας γράφομεν, ἀλλὰ βίους} (Plut.\ Alex.~1.2). Životopisi se od povjesnice za nj razlikuju po tome što se karakter i osobnost, koji su u centru interesa biografa, mogu razumjeti i iz detalja, malih, svakodnevnih postupaka: \textgreek[variant=ancient]{οὔτε ταῖς ἐπιφανεστάταις πράξεσι πάντως ἔνεστι δήλωσις ἀρετῆς ἢ κακίας, ἀλλὰ πρᾶγμα βραχὺ πολλάκις καὶ ῥῆμα καὶ παιδιά τις ἔμφασιν ἤθους ἐποίησε μᾶλλον ἢ μάχαι μυριόνεκροι καὶ παρατάξεις αἱ μέγισται καὶ πολιορκίαι πόλεων} (``Vrlina ili mana često se ne otkrivaju u sjajnim pothvatima onoliko koliko u nevažnoj stvari, riječi, šali, koja karakter ocrtava bolje od bitaka s tisućama poginulih, od silnih vojničkih pohoda i opsada gradova'', ibid). Životopisi su, tako, plod filozofsko-etičkog pristupa, i imaju jasnu etičko-pedagošku (a ne zabavnu) intenciju. 

Alkibijad \textgreek[variant=ancient]{(Ἀλκιβιάδης,} oko 450. – 404.\ p.~n.~e) bio je slobodouman, iznimno bogat i talentiran atenski aristokrat, dominantna figura političke scene posljednjih desetljeća V.\ st. Posebno je važnu ulogu odigrao pri odlučivanju o Sicilskoj ekspediciji (Thuc.~6). Izuzetnom je osobnom karizmom i govorničkim darom bio u stanju uspješno uvjeravati, čak i opčiniti narodnu skupštinu. Bio je Periklov štićenik i omiljeni Sokratov učenik. No, Alkibijad je i krajnje kontroverzan lik; i suvremenici i historiografi, od Tukidida i Ksenofonta do Plutarha, njegovo djelovanje i život ocjenjuju na različite načine. Dojmljive portrete Alkibijada ostavili su Platon (u dijalogu \textgreek[variant=ancient]{Συμπόσιον,} \textit{Gozba}) i pseudo-Andokid (govor \textgreek[variant=ancient]{Κατὰ Ἀλκιβιάδου,} \textit{Protiv Alkibijada}).

U odabranom se odlomku prikazuje Alkibijadova \textgreek[variant=ancient]{φύσις,} temeljna komponenta karaktera, koja se očituje već u djetinjstvu; kasnije će karakter oblikovati, osim slučaja \textgreek[variant=ancient]{(Τύχη),} i vanjski utjecaji, a prvenstveno \textit{odabir} \textgreek[variant=ancient]{(προαίρεσις)} između dobra i zla, koji svaki veliki čovjek mora prije ili kasnije učiniti. Alkibijadova se narav na dojmljiv način očitovala u dvama događajima, jednom prilikom rvanja \textgreek[variant=ancient]{(παλαίειν)} i drugi put za igranja piljcima \textgreek[variant=ancient]{(ἀστράγαλοι).}

%\newpage

\section*{Pročitajte naglas grčki tekst.}

Plut.\ Alcibiades 2.1–2.3

%Naslov prema izdanju

\medskip


{\large

\begin{greek}

\noindent φύσει δὲ πολλῶν ὄντων καὶ μεγάλων παθῶν ἐν αὐτῷ, τὸ φιλόνεικον ἰσχυρότατον ἦν καὶ τὸ φιλόπρωτον, ὡς δῆλόν ἐστι τοῖς παιδικοῖς ἀπομνημονεύμασιν.

ἐν μὲν γὰρ τῷ παλαίειν πιεζούμενος, ὑπὲρ τοῦ μὴ πεσεῖν ἀναγαγὼν πρὸς τὸ στόμα τὰ ἅμματα τοῦ πιεζοῦντος, οἷος ἦν διαφαγεῖν τὰς χεῖρας. ἀφέντος δὲ τὴν λαβὴν ἐκείνου καὶ εἰπόντος· δάκνεις, ὦ Ἀλκιβιάδη, καθάπερ αἱ γυναῖκες, οὐκ ἔγωγε, εἶπεν, ἀλλ' ὡς οἱ λέοντες.

ἔτι δὲ μικρὸς ὢν ἔπαιζεν ἀστραγάλοις ἐν τῷ στενωπῷ, τῆς δὲ βολῆς καθηκούσης εἰς αὐτὸν ἅμαξα φορτίων ἐπῄει. πρῶτον μὲν οὖν ἐκέλευε περιμεῖναι τὸν ἄγοντα τὸ ζεῦγος· ὑπέπιπτε γὰρ ἡ βολὴ τῇ παρόδῳ τῆς ἁμάξης· μὴ πειθομένου δὲ δι' ἀγροικίαν, ἀλλ' ἐπάγοντος, οἱ μὲν ἄλλοι παῖδες διέσχον, ὁ δ' Ἀλκιβιάδης καταβαλὼν ἐπὶ στόμα πρὸ τοῦ ζεύγους καὶ παρατείνας ἑαυτόν, ἐκέλευεν οὕτως, εἰ βούλεται, διεξελθεῖν, ὥστε τὸν μὲν ἄνθρωπον ἀνακροῦσαι τὸ ζεῦγος ὀπίσω δείσαντα, τοὺς δ' ἰδόντας ἐκπλαγῆναι καὶ μετὰ βοῆς συνδραμεῖν πρὸς αὐτόν.

\end{greek}

}


\section*{Komentar}

%1

%2

\begin{description}[noitemsep]
\item[οἷος ἦν] § 315; kopulativni glagol otvara mjesto nužnoj predikatnoj dopuni (imenski predikat, Smyth 910); fraza οἷός εἰμι § 473 B. 4, LSJ οἷος III.1.b
\end{description}

%3

%4

\begin{description}[noitemsep]
\item[τῆς δὲ βολῆς] βολή je ``bacanje'' u igri
\end{description}

%5

\begin{description}[noitemsep]
\item[ἐκέλευε] § 231; kao \textit{verbum dicendi} otvara mjesto akuzativu s infinitivom
\item[ὑπέπιπτε γὰρ] § 231; § 238; čestica γάρ najavljuje iznošenje razloga ili objašnjenja, ``naime''
\item[δι' ἀγροικίαν] ἀγροικία je ``nekultura'', ``priprostost''; četvrti u djelu \textgreek[variant=ancient]{Χαρακτῆρες} ili \textgreek[variant=ancient]{Ἠθικοὶ χαρακτῆρες,} \textit{Karakteri}, peripatetičara Teofrasta iz Ereza (\textgreek[variant=ancient]{Θεόφραστος,} oko 371.\ – oko 287.\ p.~n.~e), naslovljen je \textgreek[variant=ancient]{(χαρακτήρ) ἀγροικίας}
\item[οἱ μὲν ἄλλοι\dots\ ὁ δ' Ἀλκιβιάδης\dots] koordinacija rečeničnih članova pomoću čestica μέν\dots\ δέ\dots
\item[εἰ βούλεται] § 232; veznik εἰ uvodi protazu realne pogodbene rečenice (apodoza je οὕτως\dots\ διεξελθεῖν), § 475
\end{description}


%kraj
