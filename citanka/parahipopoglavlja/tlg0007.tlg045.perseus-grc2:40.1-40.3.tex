%\section*{O autoru}

%TKTK


\section*{O tekstu}

Plutarhov životopis rimskog vojskovođe i političara Pompeja (Gnaeus Pompeius Magnus, 106–48 pr.~Kr) dio je zbirke 22 usporedna životopisa u kojoj su u parove \textgreek[variant=ancient]{(συζυγία)} združeni po jedan veliki Grk i Rimljanin. Pompejev parnjak je Agesilaj (oko 444 – oko 360 pr.~Kr), spartanski kralj i prijatelj povjesničara Ksenofonta. 

Plutarhu je slavu donijelo upravo biografsko spisateljstvo, vođeno mišlju da druženje s velikim ljudima prošlosti treba polučiti djelovanje njihovih visokih kvaliteta na našu vlastitu narav. Takvo shvaćanje vrijednosti životopisa odraz je peripatetičkog nauka po kojem postupci imaju presudno značenje za etičke kvalitete koje kroz njih postaju vidljivima. S druge strane, za peripatetičare ``etičke vrline'' nisu prirodno urođene, nego nastaju tek s djelovanjem, kao navikom stečeni načini ponašanja. 

Plutarhova sklonost anegdotama izvire pak iz uvjerenja da karakterne crte povijesnih velikana ne dolaze do izražaja samo u velikim djelima, nego još i više u sitnoj gesti ili iskazu. 

Sljedeći odlomak donosi anegdotu o Katonu Mlađem (Marcus Porcius Cato Uticensis / Cato Minor, 95. – 46. pr. Kr.) zvanom „filozof“ zbog nepokolebljivog pristajanja uz stoičke principe. Prigodom posjete Antiohiji Katon je dramatično precijenio vlastitu popularnost, što je odmah izazvalo podsmijeh među njegovom pratnjom. Naime, građani Antiohije su očekivali dolazak utjecajnog Pompejeva oslobođenika Demetrija, u nadi da će ga nagovoriti da se kod Pompeja zauzme za njih. Zbog toga su upriličili ceremoniju za koju je Katon, negodujući, pretpostavio da je u njegovu čast. Isti je događaj Plutarh opisao i u 13. poglavlju Katonova životopisa.

%\newpage

\section*{Pročitajte naglas grčki tekst.}

Plut.\ Pompeius 40.1–3

%Naslov prema izdanju

\medskip


{\large

\begin{greek}

\noindent  Κάτων ὁ φιλόσοφος ἔτι μὲν ὢν νέος, ἤδη δὲ μεγάλην ἔχων δόξαν καὶ μέγα φρονῶν, ἀνέβαινεν εἰς Ἀντιόχειαν, οὐκ ὄντος αὐτόθι Πομπηΐου, βουλόμενος ἱστορῆσαι τὴν πόλιν.

\noindent αὐτὸς μὲν οὖν, ὥσπερ ἀεί, πεζὸς ἐβάδιζεν, οἱ δὲ φίλοι συνώδευον ἵπποις χρώμενοι. κατιδὼν δὲ πρὸ τῆς πύλης ὄχλον ἀνδρῶν ἐν ἐσθῆσι λευκαῖς καὶ παρὰ τὴν ὁδὸν ἔνθεν μὲν τοὺς ἐφήβους, ἔνθεν δὲ τοὺς παῖδας διακεκριμένους, ἐδυσχέραινεν οἰόμενος εἰς τιμήν τινα καὶ θεραπείαν ἑαυτοῦ μηδὲν δεομένου ταῦτα γίνεσθαι.

\noindent τοὺς μέντοι φίλους ἐκέλευσε καταβῆναι καὶ πορεύεσθαι μετʼ αὐτοῦ· γενομένοις δὲ πλησίον ὁ πάντα διακοσμῶν ἐκεῖνα καὶ καθιστὰς ἔχων στέφανον καὶ ῥάβδον ἀπήντησε, πυνθανόμενος παρʼ αὐτῶν ποῦ Δημήτριον ἀπολελοίπασι καὶ πότε ἀφίξεται. τοὺς μὲν οὖν φίλους τοῦ Κάτωνος γέλως ἔλαβεν, ὁ δὲ Κάτων εἰπών, ``ὢ τῆς ἀθλίας πόλεως'', παρῆλθεν, οὐδὲν ἕτερον ἀποκρινάμενος.

\end{greek}

}

\newpage

\section*{Komentar}

%1


\begin{description}[noitemsep]
\item[μὲν οὖν] dakako, §~519.7
\item[πεζὸς ἐβάδιζεν] predikatni pridjev gdje bi u hrvatskom stajao prilog, §~369, Smyth 1042; §~231
\item[συνώδευον] složenica glagola ὁδεύω; §~231
\item[χρώμενοι] §~232; s.~116; adverbni particip može označavati bilo kakvu popratnu okolnost u odnosu na glavnu radnju (najčešće ἔχων, ἄγων, χρώμενος i λαβών), Smyth §~2068
\end{description}



%2
\begin{description}[noitemsep]
\item[κατιδὼν] složenica ὁράω; §~327.3; jaki aorist §~254; adverbni particip; s predikativnim participom (koji se proteže na objekt) §~502.a
\item[ἔνθεν μὲν\dots\ ἔνθεν δὲ\dots] koordinacija rečeničnih članova pomoću čestica: s jedne strane\dots\  s druge strane\dots
\item[διακεκριμένους] perfekt §~272; složenica κρίνω; διακρίνομαι odvojen biti; predikatni particip uz \textit{verbum sentiendi} καθοράω
\item[ταῦτα γίνεσθαι] konstrukciji A+I mjesto otvara \textit{verbum sentiendi} οἰόμενος; §~232; §~325.11; LSJ γίγνομαι, jonski i nakon Aristotela γίνομαι

\end{description}

%3


\begin{description}[noitemsep]
\item[ἐκέλευσε ] §~267; κελεύω + A+I naložiti komu što
\item[καταβῆναι] složenica βαίνω; §~321.6; bestematski aorist §~316
\item[γενομένοις] jaki aorist §~254; dativ zbog rekcije ἀπαντάω; πλησίον γίγνεσθαι blizu doći
\item[ἔχων] §~231; adverbni particip može označavati bilo kakvu popratnu okolnost u odnosu na glavnu radnju (najčešće ἔχων, ἄγων, χρώμενος i λαβών), Smyth §~2068
\item[ἀπήντησε] složenica glagola ἀντάω, rekcija τινί
\item[πότε ἀφίξεται] upitna zavisna rečenica §~469, §~446; futur §~258; §~321.8

\end{description}

%4

\begin{description}[noitemsep]
\item[μὲν οὖν] §~519.7; Smyth §~2901
\item[τῆς ἀθλίας πόλεως] \textit{genitivus causae} uz uzvike §~406
\end{description}


%kraj
