%\section*{O autoru}

%TKTK


\section*{O tekstu}

Pišući po uzoru na pet stoljeća starijeg Ksenofonta (na kojeg aludira već naslov djela), Flavije Arijan (oko 95. – Atena, oko 175) slavi podvige Aleksandra Velikog \textgreek[variant=ancient]{(Ἀλέξανδρος ὁ Μέγας,} 356.–323. p.~n.~e). Tog makedonsko-grčkog junaka iz daleke prošlosti – Arijan je od Aleksandra udaljen kao mi od Marka Marulića – historiograf prikazuje trudeći se da od mnoštva postojećih povjesnica odabere one najpouzdanije. U potrazi za istinom, nastaje portret Aleksandra koji je nevjerojatno uspješan, ali ne bez mana, te čije je spektakularne pohode okončala prerana smrt (13.\ lipnja 323. p.~n.~e).

Posljednja od sedam knjiga \textit{Aleksandrova pohoda}  pripovijeda o kraju slavnog vojskovođe. Stigavši do Pasargade i Perzepolisa, prijestolnice Ahemenidskog Carstva, Aleksandar je razmišljao što dalje. Želio je ići do Perzijskog mora i ušća Inda, ali, neki tvrde, i u Sjevernu Afriku, pa preko Heraklovih stupova do Gadesa (Cádiza); drugi tvrde da je namjeravao osvojiti Skitiju i Meotsko jezero (Azovsko more); treći, da je planirao i pohod na Siciliju i Japigiju (Kalabriju). Arijan iznosi svoje mišljenje o Aleksandrovim namjerama, točnije o nesaznatljivosti tih namjera, naznačava veličinu njegove ambicije i njezin psihički izvor, ali iznosi i anegdotu o reakciji indijskih mudraca na tu ambiciju.

\newpage

\section*{Pročitajte naglas grčki tekst.}

Arr. Alexandri anabasis 7.1.4-7.1.6

%Naslov prema izdanju

\medskip


{\large

\begin{greek}

\noindent  ἐγὼ δὲ ὁποῖα μὲν ἦν Ἀλεξάνδρου τὰ ἐνθυμήματα οὔτε ἔχω ἀτρεκῶς ξυμβαλεῖν οὔτε μέλει ἔμοιγε εἰκάζειν, ἐκεῖνο δὲ καὶ αὐτὸς ἄν μοι δοκῶ ἰσχυρίσασθαι, οὔτε μικρόν τι καὶ φαῦλον ἐπινοεῖν Ἀλέξανδρον οὔτε μεῖναι ἂν ἀτρεμοῦντα ἐπ' οὐδενὶ τῶν ἤδη κεκτημένων, οὐδὲ εἰ τὴν Εὐρώπην τῇ Ἀσίᾳ προσέθηκεν, οὐδ' εἰ τὰς Βρεττανῶν νήσους τῇ Εὐρώπῃ, ἀλλὰ ἔτι ἂν ἐπέκεινα ζητεῖν τι τῶν ἠγνοημένων, εἰ καὶ μὴ ἄλλῳ τῳ, ἀλλὰ αὐτόν γε αὑτῷ ἐρίζοντα.

καὶ ἐπὶ τῷδε ἐπαινῶ τοὺς σοφιστὰς τῶν Ἰνδῶν, ὧν λέγουσιν ἔστιν οὓς καταληφθέντας ὑπ' Ἀλεξάνδρου ὑπαιθρίους ἐν λειμῶνι, ἵναπερ αὐτοῖς διατριβαὶ ἦσαν, ἄλλο μὲν οὐδὲν ποιῆσαι πρὸς τὴν ὄψιν αὐτοῦ τε καὶ τῆς στρατιᾶς, κρούειν δὲ τοῖς ποσὶ τὴν γῆν ἐφ' ἧς βεβηκότες ἦσαν. ὡς δὲ ἤρετο Ἀλέξανδρος δι' ἑρμηνέων ὅ τι νοοῖ αὐτοῖς τὸ ἔργον, τοὺς δὲ ὑποκρίνασθαι ὧδε·

βασιλεῦ Ἀλέξανδρε, ἄνθρωπος μὲν ἕκαστος τοσόνδε τῆς γῆς κατέχει ὅσονπερ τοῦτό ἐστιν ἐφ' ὅτῳ βεβήκαμεν· σὺ δὲ ἄνθρωπος ὢν παραπλήσιος τοῖς ἄλλοις, πλήν γε δὴ ὅτι πολυπράγμων καὶ ἀτάσθαλος, ἀπὸ τῆς οἰκείας τοσαύτην γῆν ἐπεξέρχῃ πράγματα ἔχων τε καὶ παρέχων ἄλλοις. καὶ οὖν καὶ ὀλίγον ὕστερον ἀποθανὼν τοσοῦτον καθέξεις τῆς γῆς ὅσον ἐξαρκεῖ ἐντεθάφθαι τῷ σώματι.

\end{greek}

}


\section*{Komentar}

%1

\begin{description}[noitemsep]
\item[ὁποῖα μὲν ἦν\dots] \textbf{ἐκεῖνο δὲ\dots}\ koordinacija rečeničnih članova pomoću čestica μέν\dots\ δέ\dots; korelacija surečenica ostvarena odnosnom zamjenicom i pokaznim antecedentom (koji ovdje dolazi nakon odnosne rečenice)
\item[ξυμβαλεῖν] § 231; ξυμβάλλω je alternativni oblik glagola συμβάλλω
\item[μέλει ἔμοιγε] § 325.16; naglašen oblik lične zamjenice § 206.2
\item[ἄν\dots\ ἰσχυρίσασθαι] § 267; § 261, § 269; infinitiv s ἄν ima potencijalno značenje § 506; ἰσχυρίζομαι kao \textit{verbum sentiendi} otvara mjesto akuzativu s infinitivom
\item[μεῖναι ἂν ἀτρεμοῦντα] infinitiv s ἄν ima potencijalno značenje § 506 (apodoza irealne pogodbene rečenice)
\item[εἰ\dots\ προσέθηκεν] složenica glagola τίθημι, § 306; protaza irealne pogodbene rečenice, § 478 (apodoza je akuzativ s infinitivom μεῖναι ἂν ἀτρεμοῦντα)
\item[οὐδ' εἰ τὰς Βρεττανῶν νήσους] sc. προσέθηκεν
\item[ἂν\dots\ ζητεῖν] § 231; infinitiv s ἄν ima potencijalno značenje § 506 (apodoza irealne pogodbene rečenice)
\item[αὐτόν γε] čestica ističe zamjenicu, koja ovdje ima funkciju lične, limitativno: \textit{on} (Aleksandar, kakav je bio)
\end{description}

%2

\begin{description}[noitemsep]
\item[ἔστιν οὓς] akuzativ (zbog akuzativa s infinitivom) fraze ἔστιν οἵ ``neki'', § 443 bilj. 2, LSJ εἰμί A.IV
\item[ἵναπερ] zavisni veznik ἵνα otvara mjesto namjernoj rečenici § 470; veznik je pojačan enklitičnom česticom περ, § 519.2
\item[αὐτοῖς\dots\ ἦσαν] preterit u namjernoj rečenici zbog asimilacije načina, Smyth 2205
\end{description}

%3

\begin{description}[noitemsep]
\item[νοοῖ]	§ 243; u zavisno upitnoj rečenici iza sporednoga (historijskog) vremena predikat je u optativu § 469
\end{description}

%4

\begin{description}[noitemsep]
\item[τοσόνδε\dots\ ὅσονπερ\dots] korelacija surečenica ostvarena pokaznim antecedentom i odnosnim konektorom, Smyth 2503; kod pokazne zamjenice naglašen je demonstrativni aspekt, LSJ τοσόσδε
\item[κατέχει] § 231, ὅσοσπερ ``baš koliko'' LSJ ὅσος III.4
\item[πλήν γε δὴ] kombinacija čestica naglašava ograničavanje izrečeno nepravim prijedlogom πλήν (Denniston 245)
\item[πλήν\dots\ ὅτι] kombinacija nepravog prijedloga i veznika uvodi zavisno izričnu rečenicu, ``osim što''
\item[πολυπράγμων καὶ ἀτάσθαλος] sc. εἶ ili ὢν
\item[ἔχων τε καὶ παρέχων] § 231; kombinacija veznika τε καὶ povezuje komplementarne elemente, ponekad je drugi član značenjski jači od prvog, Smyth 4.60.208 2974
\end{description}

%5
\begin{description}[noitemsep]
\item[καὶ οὖν] kombinacija čestica znači ``i zaista''; Denniston, \textit{Greek Particles} (Oxford 1934, 1954), smatra je vrlo rijetkom
\item[τοσοῦτον\dots\ ὅσον] korelacija surečenica ostvarena pokaznim antecedentom i odnosnom zamjenicom kao konektorom, Smyth 2503
\item[ἐξαρκεῖ] § 243; ἐξαρκέω τινί s infinitivom koji pokazuje svrhu, § 495
\item[ἐντεθάφθαι] oblik glagola ἐνταφιάζω
\end{description}


%kraj
