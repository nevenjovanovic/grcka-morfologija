\section*{O tekstu}

Koncem Peloponeskoga rata Andokid se okoristio općom amnestijom i vratio se iz progonstva u Atenu, gdje je potom obavljao niz javnih funkcija. No, njegovi protivnici – možda motivirani zavišću zbog Andokidova dobra materijalna statusa? – nisu mirovali. Već 399. pr. Kr. Andokid je opet pred sucima, ponovo optužen za sudjelovanje u svetogrdnim postupcima (oskvrnuću Eleuzinskih misterija) 415. Andokid se branio govorom \textgreek[variant=ancient]{Περὶ τῶν μυστηρίων} (\textit{O misterijama}) i bio je oslobođen.

Govor \textit{O misterijama} smatra se Andokidovim najuspjelijim radom; danas je posebno zanimljiv jer osvjetljava okolnosti iznimnih događaja u Ateni 415.\ pr.~Kr. Ostaje, dakako, pitanje u kojoj mjeri Andokidova verzija odgovara stvarnim zbivanjima.

Govor ima tradicionalnu strukturu kakvu nalazimo kod govornika poput Antifonta ili Lizije: nakon proemija (προοίμιον) slijedi kratka πρόθεσις (opći nacrt slučaja), pa pripovijedanje, argumentacija i epilog. Ekscerpt koji čitamo citira tekst službenog dokumenta, zakona koji se naziva Demofantov dekret, a na koji se Andokid referira kao na Solonov zakon: ako tko sudjeluje u protudemokratskim postupanjima u Ateni bit će kažnjen, baš kao što će onaj koji im se suprotstavi biti nagrađen.

Filološka istraživanja potvrdila su da zakonski tekst nije autentičan.



\newpage

\section*{Pročitajte naglas grčki tekst.}

And.\ De mysteriis 97

%Naslov prema izdanju

\medskip


{\large

\begin{greek}

\noindent ὁ δὲ ὅρκος ἔστω ὅδε: κτενῶ καὶ λόγῳ καὶ ἔργῳ καὶ ψήφῳ καὶ τῇ ἐμαυτοῦ χειρί, ἂν δυνατὸς ὦ, ὃς ἂν καταλύσῃ τὴν δημοκρατίαν τὴν Ἀθήνησι, καὶ ἐάν τις ἄρξῃ τιν᾽ ἀρχὴν καταλελυμένης τῆς δημοκρατίας τὸ λοιπόν, καὶ ἐάν τις τυραννεῖν ἐπαναστῇ ἢ τὸν τύραννον συγκαταστήσῃ· καὶ ἐάν τις ἄλλος ἀποκτείνῃ, ὅσιον αὐτὸν νομιῶ εἶναι καὶ πρὸς θεῶν καὶ δαιμόνων, ὡς πολέμιον κτείναντα τὸν Ἀθηναίων, καὶ τὰ κτήματα τοῦ ἀποθανόντος πάντα ἀποδόμενος ἀποδώσω τὰ ἡμίσεα τῷ ἀποκτείναντι, καὶ οὐκ ἀποστερήσω οὐδέν.

ἐὰν δέ τις κτείνων τινὰ τούτων ἀποθάνῃ ἢ ἐπιχειρῶν, εὖ ποιήσω αὐτόν τε καὶ τοὺς παῖδας τοὺς ἐκείνου καθάπερ Ἁρμόδιόν τε καὶ Ἀριστογείτονα καὶ τοὺς ἀπογόνους αὐτῶν. ὁπόσοι δὲ ὅρκοι ὀμώμονται Ἀθήνησιν ἢ ἐν τῷ στρατοπέδῳ ἢ ἄλλοθί που ἐναντίοι τῷ δήμῳ τῷ Ἀθηναίων, λύω καὶ ἀφίημι.

\end{greek}

}


\section*{Komentar}

%1


\begin{description}[noitemsep]
\item[ἂν δυνατὸς ὦ] § 315; umetnuta rečenica bez veznika povezana s prethodnom, pomišlja se eventualna pogodbena ili zavisna vremenska rečenica s pogodbenim značenjem, „ako budem…“, „dok god…“
\item[ὃς ἂν καταλύσῃ] odnosna zamjenica ὃς uvodi zavisnu odnosnu rečenicu; N.~B. zamjenica je u nominativu umjesto u akuzativu!

\end{description}

%2


\begin{description}[noitemsep]
\item[ἐάν… ἄρξῃ] pogodbeni veznik ἐάν uvodi zavisnu pogodbenu rečenicu eventualnog značenja, „ako…“
\item[τὸ λοιπόν] priložno: ``nadalje'', ``na drugi način'', ``u ostalome''
\item[ἐάν… ἐπαναστῇ] pogodbeni veznik ἐάν uvodi zavisnu pogodbenu rečenicu eventualnog značenja, „ako…“
\item[ἢ… συγκαταστήσῃ] pogodba se nastavlja povezivanjem veznikom ἢ, „ili“, bez izricanja pogodbenog veznika, „ili (ako)…“

\end{description}

%3


\begin{description}[noitemsep]
\item[ἐάν… ἀποκτείνῃ] pogodbeni veznik ἐάν uvodi zavisnu pogodbenu rečenicu eventualnog futurskog značenja, „ako…“
\item[κτείναντα] particip s uzročnim ili poredbenim veznikom ὡς: „kao onaj koji…“, ``jer je onaj koji\dots''
\item[ἀποδόμενος] rekcija mediopasiva: τι τινος, „prodavati što od koga“; § 306
\end{description}

%4

\begin{description}[noitemsep]
\item[ἐπιχειρῶν] § 243, zamišlja se dopuna u infinitivu ἐπιχειρῶν κτείνειν
\item[Εὰν δέ… ἀποθάνῃ] pogodbeni veznik Εὰν uvodi zavisnu pogodbenu rečenicu eventualnog futurskog značenja, „ako bude…“
\item[Ἁρμόδιόν τε καὶ Ἀριστογείτονα] Harmodije i Aristogiton, legendarni prijatelji koji su ubili jednog od dvojice atenskih tirana, Pizistratova sina Hiparha (514.\ pr.~Kr). Harmodije je ubijen na mjestu atentata, a Aristogiton je izdahnuo na mučilištu. Preostali tiranin, Hiparhov brat Hipija, srušen je 508, te je Klistenovim reformama u Ateni uspostavljena demokracija. Atenjani su Harmodija i Aristogitona štovali kao osloboditelje i ``ubojice tirana'' \textgreek[variant=ancient]{(τυραννόκτονοι)} i podigli im spomenik (oko 510.\ pr.~Kr.).
\end{description}


%5

\begin{description}[noitemsep]
\item[Οπόσοι… ὀμώμονται] odnosna zamjenica Οπόσοι uvodi zavisnu odnosnu rečenicu, „koje god…“
\end{description}


%kraj
