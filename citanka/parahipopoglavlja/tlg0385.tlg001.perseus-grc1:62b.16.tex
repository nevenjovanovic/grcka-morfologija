\section*{O autoru}

Povjesničar Dion Kasije (navodni njegov \textit{cognomen} Cocceianus, Kokcejan, spominje se tek u bizantsko doba i vjerojatno je fiktivan), grčki \textgreek[variant=ancient]{Δίων ὁ Κάσσιος,} latinski Lucius Cassius Dio, podrijetlom Grk iz Nikeje u Bitiniji, rođen je oko 155.\ n.~e., a umro oko 235. Vršio je službu konzula i prokonzula.

Najpoznatiji je po djelu \textgreek[variant=ancient]{Ῥωμαϊκὴ ἱστορία}, \textit{Rimska povijest}, u 80 knjiga; od njih su u cijelosti sačuvane samo one od 36. do 60. Djelo počinje Enejinim dolaskom u Italiju, a završava godinom 229.\ n.~e. 

Dion je pisao po uzoru na analiste, najranije rimske povjesničare koji su događaje navodili po godinama.

\section*{O tekstu}

U ovom se odlomku opisuje kako je car Neron 64.\ n.~e.\ ostvario svoju dugogodišnju želju da zapali Rim. Riječ je o možda i najpoznatijoj epizodi iz Neronova života koja se često spominjala u književnosti i prikazivala na filmu. Međutim, valja imati na umu da su izvještavali pisci izrazito neskloni Neronu, primjerice Tacit; Dion Kasije ne smatra se sasvim pouzdanim izvorom za ranija povijesna razdoblja. Današnji povjesničari smatraju malo vjerojatnim da je veliki požar podmetnuo upravo car.

% Δίωνος Κασσίου Κοκκηιανοῦ Ῥωμαϊκὴ ἱστορία
\newpage

\section*{Pročitajte naglas grčki tekst.}

Dio Cassius Historiae Romanae 62.16.1-62.16.4
%Naslov prema izdanju

\medskip


{\large

\begin{greek}

\noindent μετὰ δὲ ταῦτα ἐπεθύμησεν ὅπερ που ἀεὶ ηὔχετο, τήν τε πόλιν ὅλην καὶ τὴν βασιλείαν ζῶν ἀναλῶσαι· τὸν γοῦν Πρίαμον καὶ αὐτὸς θαυμαστῶς ἐμακάριζεν ὅτι καὶ τὴν πατρίδα ἅμα καὶ τὴν ἀρχὴν ἀπολομένας εἶδεν. λάθρᾳ γάρ τινας ὡς καὶ μεθύοντας ἢ καὶ κακουργοῦντάς τι ἄλλως διαπέμπων, τὸ μὲν πρῶτον ἕν που καὶ δύο καὶ πλείονα ἄλλα ἄλλοθι ὑπεπίμπρα, ὥστε τοὺς ἀνθρώπους ἐν παντὶ ἀπορίας γενέσθαι, μήτ᾽ ἀρχὴν τοῦ κακοῦ ἐξευρεῖν μήτε τέλος ἐπαγαγεῖν δυναμένους ἀλλὰ πολλὰ μὲν ὁρῶντας πολλὰ δὲ ἀκούοντας ἄτοπα. 

\noindent οὔτε γὰρ θεάσασθαι ἄλλο τι ἦν ἢ πυρὰ πολλὰ ὥσπερ ἐν στρατοπέδῳ, οὔτε ἀκοῦσαι λεγόντων τινῶν ἢ ὅτι τὸ καὶ τὸ καίεται. ποῦ; πῶς; ὑπὸ τίνος; βοηθεῖτε.

\noindent θόρυβός τε οὖν ἐξαίσιος πανταχοῦ πάντας κατελάμβανε, καὶ διέτρεχον οἱ μὲν τῇ οἱ δὲ τῇ ὥσπερ ἔμπληκτοι. καὶ ἄλλοις τινὲς ἐπαμύνοντες ἐπυνθάνοντο τὰ οἴκοι καιόμενα· καὶ ἕτεροι πρὶν καὶ ἀκοῦσαι ὅτι τῶν σφετέρων τι ἐμπέπρησται, ἐμάνθανον ὅτι ἀπόλωλεν. οἵ τε ἐκ τῶν οἰκιῶν ἐς τοὺς στενωποὺς ἐξέτρεχον ὡς καὶ ἔξωθεν αὐταῖς βοηθήσοντες, καὶ οἱ ἐκ τῶν ὁδῶν εἴσω ἐσέθεον ὡς καὶ ἔνδον τι ἀνύσοντες.

\end{greek}

}

%\newpage

\section*{Komentar}

%1

\begin{description}[noitemsep]
\item[ἐπεθύμησεν] §~267 (subjekt je Neron); glagol otvara mjesto dopuni u infinitivu
\item[που] valjda 
\item[ἀναλῶσαι] §~267; glagol je ἀναλίσκω (neke osnove tvori od ἀναλόω), §~324.6
\item[γοῦν] čestica koja služi isticanju ili pojačavanju  
\end{description}

%2
\begin{description}[noitemsep]
\item[διαπέμπων] složenica glagola πέμπω, §~231
\item[μὲν] postpozitivna čestica §~519.7
\item[που] negdje
\item[ὑπεπίμπρα] složenica glagola πίμπρημι, §~312.3; ovaj imperfekt tvori se kao da je osnova ὑποπιμπράω, usp. LSJ ἐμπίμπρημι, morfološki opis
\item[ἐν παντὶ] sasvim
\item[γενέσθαι] dopuna u genitivu (LSJ γίγνομαι II.3)
\item[μήτ'\dots\ μήτε] sastavni veznici §~513
\item[ἐξευρεῖν] složenica glagola εὑρίσκω, §~324.7, §~254
\item[ἐπαγαγεῖν] složenica glagola ἄγω, s. 116, §~257 
\item[δυναμένους] §~312.5, §~232, glagol nepotpuna značenja otvara mjesto dopunama u infinitivu
\item[πολλὰ μὲν\dots\ πολλὰ δὲ] koordinacija rečeničnih članova pomoću para čestica
\end{description}

%3

\begin{description}[noitemsep]
\item[οὔτε\dots\ οὔτε] koordinacija pomoću sastavnih veznika §~513
\item[γὰρ] čestica uvodi objašnjenje prethodnog navoda: naime\dots
\item[ἦν] §~315; u značenju ``biti moguće'' glagol otvara mjesto dopuni u infinitivu (LSJ εἰμί A.VI)
\item[ἢ] osim
\item[ἀκοῦσαι] rekcija τινός
\item[τὸ καὶ τὸ] član kao pokazna zamjenica: to i to, §~370
\end{description}

%4

\begin{description}[noitemsep]
\item[τε] postponirani (sastavni) veznik
\item[οὖν] zaključna čestica: dakle
\item[κατελάμβανε] složenica glagola λαμβάνω §~321.14, §~231
\item[διέτρεχον] složenica glagola τρέχω §~327.4, §~231
\item[οἱ μὲν τῇ, οἱ δὲ τῇ] koordinacija pomoću čestica: jedni ovamo, a drugi onamo

\end{description}


%5
\begin{description}[noitemsep]
\item[ἐπαμύνοντες] složenica glagola ἀμύνω; rekcija τινί; §~231
\item[ἐπυνθάνοντο] §~232 (s participom kao dopunom, LSJ πυνθάνομαι A.5)
\item[πρὶν] uvodi zavisnu vremensku rečenicu s predikatom u infinitivu, §~488.1; πρὶν καὶ prije nego što\dots
\item[τῶν σφετέρων] genitiv partitivni §~395; supstantivirana zamjenica, §~373
\item[ἐμπέπρησται] složenica glagola πίμπρημι, §~272
\end{description}

%6

\begin{description}[noitemsep]
\item[οἵ τε\dots\ καὶ οἱ] koordinacija rečeničnih članaka sastavnim veznicima
\item[ἐξέτρεχον] složenica glagola τρέχω, §~327.4, §~231
\item[βοηθήσοντες] rekcija τινί, §~259
\item[ἐσέθεον] složenica glagola θέω, §~231
\end{description}


%kraj
