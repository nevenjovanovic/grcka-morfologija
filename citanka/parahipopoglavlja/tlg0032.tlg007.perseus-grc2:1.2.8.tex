%\section*{O autoru}

%TKTK


\section*{O tekstu}

Ksenofontovo djelo \textgreek[variant=ancient]{Κύρου παιδεία,} \textit{Kirov odgoj}, nastalo je vjerojatno poslije 371.\ pr.~Kr. U osnovi, ono je fiktivna, romansirana biografija perzijskog vladara Kira Velikog (oko 600. – 530.\ pr.~Kr.); prikaz Kirova života, međutim, Ksenofontu služi i da predstavi i razjasni svoje političke ideje. Utoliko djelo pripada i političkom žanru \textgreek[variant=ancient]{πολιτεία} (poput Platonove \textit{Države} ili djela Starog oligarha). U opisima perzijskih običaja i priča Ksenofont se pak naslanja na tradiciju etnografije drugog, ne-grčkog (predstavljenu Herodotom). 

Iz aspekta dugotrajnih grčkih ratova s Perzijancima, mogli bi začuditi odabir teme i pohvalan ton djela. No, Grke Perzija zanima i fascinira već od IV.~st.\ pr.~Kr., a samog Kira Velikog, koji je vladao četrdesetak je godina prije početka grčko-perzijskih sukoba, grčka književnost u pozitivnom svjetlu prikazuje još od Eshila. 

\textit{Kirov odgoj} uzoran je primjer klasične atičke proze IV.~st.\ pr.~Kr., te se već u antici smatrao remek-djelom. Ponovno je otkriven u srednjem vijeku, kao jedan od uzora za književnu vrstu \textit{ogledalo vladara} (na primjer, za Machiavellijevo djelo \textit{Vladar}). 

Ovdje odabrani ulomak potječe iz prve knjige \textit{Kirova odgoja}, i na vrlo općenit način govori o perzijskom obrazovnom sustavu, opisujući odgoj dječaka do 16 odnosno 17 godina. Posebno se naglašava uloga starijih kao uzora, te disciplina pri prehrani. Naglasak na usvajanju umjerenosti ili razumnog pristupanja životu \textgreek[variant=ancient]{(σωφροσύνη)} doziva u sjećanje etiku Ksenofontova učitelja Sokrata.

\newpage

\section*{Pročitajte naglas grčki tekst.}

Xen.\ Cyropaedia 1.2.8

%Naslov prema izdanju

\medskip


{\large

\begin{greek}

\noindent  διδάσκουσι δὲ τοὺς παῖδας καὶ σωφροσύνην· μέγα δὲ συμβάλλεται εἰς τὸ μανθάνειν σωφρονεῖν αὐτοὺς ὅτι καὶ τοὺς πρεσβυτέρους ὁρῶσιν ἀνὰ πᾶσαν ἡμέραν σωφρόνως διάγοντας. διδάσκουσι δὲ αὐτοὺς καὶ πείθεσθαι τοῖς ἄρχουσι· μέγα δὲ καὶ εἰς τοῦτο συμβάλλεται ὅτι ὁρῶσι τοὺς πρεσβυτέρους πειθομένους τοῖς ἄρχουσιν ἰσχυρῶς. διδάσκουσι δὲ καὶ ἐγκράτειαν γαστρὸς καὶ ποτοῦ· μέγα δὲ καὶ εἰς τοῦτο συμβάλλεται ὅτι ὁρῶσι τοὺς πρεσβυτέρους οὐ πρόσθεν ἀπιόντας γαστρὸς ἕνεκα πρὶν ἂν ἀφῶσιν οἱ ἄρχοντες, καὶ ὅτι οὐ παρὰ μητρὶ σιτοῦνται οἱ παῖδες, ἀλλὰ παρὰ τῷ διδασκάλῳ, ὅταν οἱ ἄρχοντες σημήνωσι. φέρονται δὲ οἴκοθεν σῖτον μὲν ἄρτον, ὄψον δὲ κάρδαμον, πιεῖν δέ, ἤν τις διψῇ, κώθωνα, ὡς ἀπὸ τοῦ ποταμοῦ ἀρύσασθαι. πρὸς δὲ τούτοις μανθάνουσι καὶ τοξεύειν καὶ ἀκοντίζειν. μέχρι μὲν δὴ ἓξ ἢ ἑπτακαίδεκα ἐτῶν ἀπὸ γενεᾶς οἱ παῖδες ταῦτα πράττουσιν, ἐκ τούτου δὲ εἰς τοὺς ἐφήβους ἐξέρχονται.

\end{greek}

}


\section*{Komentar}

%1


\begin{description}[noitemsep]
\item[διάγοντας] §~498, §~301.B (s.~116), §~238; glagol ἄγω i njegove složenice, poput διάγω, mogu imati i prijelazno i neprelazno značenje; u ovom slučaju narav dopune, priloga σωφρόνως, pomaže nam to odrediti.
\end{description}

%2

%3


%4

%5

%6


\begin{description}[noitemsep]
\item[μὲν δὴ] ovom se kombinacijom čestica povjesničari često koriste kao formulom prijelaza, surečenica s μὲν δὴ služi kao rezime prethodno rečenog: ``tako dakle\dots'' (Denniston, \textit{Greek Particles}, 258)%Denniston
\end{description}


%kraj
