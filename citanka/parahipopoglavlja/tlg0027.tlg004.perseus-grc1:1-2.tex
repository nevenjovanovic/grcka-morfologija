\section*{O autoru}

Andokid \textgreek[variant=ancient]{(Ἀνδοκίδης,} oko 440.\ – nakon 391.\ p.~n.~e), rodom iz ugledne i bogate atenske obitelji, pripadao je ``zlatnoj mladeži'' Atene; mnogi od njegovih drugova javljaju se kao protagonisti u Platonovim dijalozima. Kao pripadnik aristokratskih, prema demokraciji neprijateljskih društava bio je tijekom Peloponeskog rata 415, zajedno s Alkibijadom, upleten u skandal oskvrnuća hermi uoči Sicilske ekspedicije; bogohulni je čin teško uvrijedio običan atenski narod, osobito zato što se sumnjalo da se Alkibijad, zajedno s prijateljima, u svojoj kući izrugivao i Eleuzinskim misterijama.

Andokid je završio u zatvoru (Alkibijad, koji je otplovio za Siciliju, bio je osuđen u odsutnosti), te je na savjet rođaka Harmida – onog po kojem se zove jedan Platonov dijalog – prokazao četiri člana svojeg aristokratskog udruženja. Nakon priznanja, osuđen je na progonstvo na Cipar. Vratio se u Atenu 407.\ i pokušao govorom ishoditi oprost, ali neuspješno; pomilovan je tek u okviru opće amnestije 403. Ponovo se počeo baviti politikom i opet je optužen za bezbožnost; branio se govorom \textit{O misterijama} (399). Bio je član poslanstva koje je 391.\ pregovaralo sa Spartom o miru, no, zbog držanja tijekom poslanstva optužen je za veleizdaju (branio se govorom \textit{O miru sa Spartancima}) i ponovno osuđen na progonstvo. Umro je izvan Atene, ne zna se kada.

Andokidovi govori dragocjen su povijesni izvor, ali pokazuju i dijalektičku sposobnost i dobro poznavanje atenskoga sudskog stila (iako Andokid nije bio profesionalni govornik). Andokid riječi nalazi u životu, ne u školi, a dramatične učinke ostvaruje ne naučenom retoričkom tehnikom, već iznoseći ono što je sam doživio.

\section*{O tekstu}

Govor \textgreek[variant=ancient]{Κατὰ Ἀλκιβιάδου} (\textit{Protiv Alkibijada}) već se u antičko doba nije smatrao Andokidovim djelom. Prema navodima u tekstu i onome što o povijesti znamo, govor bi bio održan 417, kada je \textgreek[variant=ancient]{ὀστρακοφορία} (glasovanje ostrakama) trebala odlučiti hoće li prognan biti Alkibijad ili Nikija; demagog Hiperbol pokušao se riješiti barem jednog od dvojice utjecajnih oponenata, ali Alkibijad se nagodio s Nikijom, tako da je ostrakizmom prognan sam Hiperbol. Andokid je tada imao oko dvadeset godina, te se ne može s njim povezati spomen uspješne političke karijere, koja je uključivala šest diplomatskih poslanstava u zapadnoj Grčkoj i na Siciliji, što ni sam Andokid ne spominje u svojim autentičnim govorima. Postoje i pravno-proceduralni problemi: ostrakizam nije bio sudski postupak, i nije uključivao govore optužbe i obrane.

Situacija govora, međutim, nije posve izmišljena. Plutarh, koji o ostrakizmu 417.\ pripovijeda na više mjesta, daje naslutiti da je upleten bio i vođa treće stranke, Feak \textgreek[variant=ancient]{(Φαίαξ),} tako da je bilo pokušaja da se govor pripiše njemu; ovo otežava spominjanje osvajanja Mela (iz 416). 

Najvjerojatnije je, stoga, da se radi o književnoj vježbi, možda ``iz uloge'' Feaka; vježba je nastala kad detalji procedure ostrakizma više nisu bili posve jasni, vjerojatno u ranom IV.~ st.

U prvom dijelu govora govornik dokazuje svoje zasluge i nedužnost, ističe da, mada je četiri puta bio tužen zbog političkih prijestupa, nikad nije osuđen. Drugi je dio napad na javno i privatno djelovanje Alkibijada. Potom se uspoređuju obitelji govornika i Alkibijada (oba su Alkibijadova djeda bila dvaput ostrakizmom prognana), ističe se spremnost govornika da i na sudu odgovara za svoje postupke (što Alkibijad nikad nije htio učiniti), odbija se prigovor da su govornika oslobodili zbog nesposobnosti tužitelja. Naposljetku govornik upozorava da će Alkibijad pokušati pridobiti suosjećanje publike, ali u javnom je interesu da se progna njega, a ne govornika, koji je u prošlosti mnogim djelima zadužio Atenu.

Ovdje odabrani odlomak stoji na samom početku govora, i najprije razmatra načelne opasnosti bavljenja politikom, kojih je govornik itekako svjestan, ali im se izložio u interesu općeg dobra, uz podršku sebi sklonih, čak i po cijenu sukoba s moćnim neprijateljima. Zatim se iznosi glavni problem govora: koga od trojice treba prognati na deset godina.

\newpage

\section*{Pročitajte naglas grčki tekst.}

And.\ In Alcibiadem [Sp.] 1-2

%Naslov prema izdanju

\medskip


{\large

\begin{greek}

\noindent οὐκ ἐν τῷ παρόντι μόνον γιγνώσκω τῶν πολιτικῶν πραγμάτων ὡς σφαλερόν ἐστιν ἅπτεσθαι, ἀλλὰ καὶ πρότερον χαλεπὸν ἡγούμην, πρὶν τῶν κοινῶν ἐπιμελεῖσθαί τινος. πολίτου δὲ ἀγαθοῦ νομίζω προκινδυνεύειν ἐθέλειν τοῦ πλήθους, καὶ μὴ καταδείσαντα τὰς ἔχθρας τὰς ἰδίας ὑπὲρ τῶν δημοσίων ἔχειν ἡσυχίαν· διὰ μὲν γὰρ τοὺς τῶν ἰδίων ἐπιμελουμένους οὐδὲν αἱ πόλεις μείζους καθίστανται, διὰ δὲ τοὺς τῶν κοινῶν μεγάλαι καὶ ἐλεύθεραι γίγνονται.

ὧν [τῶν ἀγαθῶν] εἷς ἐγὼ βουληθεὶς ἐξετάζεσθαι μεγίστοις περιπέπτωκα κινδύνοις, προθύμων μὲν καὶ ἀγαθῶν ἀνδρῶν ὑμῶν τυγχάνων, δι' ὅπερ σῴζομαι, πλείστοις δὲ καὶ δεινοτάτοις ἐχθροῖς χρώμενος, ὑφ' ὧν διαβάλλομαι. ὁ μὲν οὖν ἀγὼν ὁ παρὼν οὐ στεφανηφόρος, ἀλλ' εἰ χρὴ μηδὲν ἀδικήσαντα τὴν πόλιν δέκα ἔτη φεύγειν· οἱ δ' ἀνταγωνιζόμενοι περὶ τῶν ἄθλων τούτων ἐσμὲν ἐγὼ καὶ Ἀλκιβιάδης καὶ Νικίας, ὧν ἀναγκαῖον ἕνα τῇ συμφορᾷ περιπεσεῖν.

\end{greek}

}


\section*{Komentar}

%1


\begin{description}[noitemsep]
\item[ἐν τῷ παρόντι] LSJ πάρειμι II.
\end{description}

%2

\begin{description}[noitemsep]
\item[τῶν δημοσίων] supstantiviranje, § 373; DGE / Logeion δημόσιος IV
\end{description}


%3

\begin{description}[noitemsep]
\item[χρώμενος] § 243; χρῆσθαί τινι; LSJ χράομαι IV.b % otvara mjesto dativu
\end{description}

%4
\begin{description}[noitemsep]
\item[ὁ\dots\ ἀγὼν ὁ παρὼν] jače istaknut atributni položaj, § 375; LSJ πάρειμι II.% položaj participa, značenje "sadašnji"
\item[περιπεσεῖν] § 254; složenica glagola πίπτω; περιπίπτω τινί, LSJ II.3 % s dativom

\end{description}

%kraj
