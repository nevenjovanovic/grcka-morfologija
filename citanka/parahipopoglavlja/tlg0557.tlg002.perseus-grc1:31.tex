%\section*{O autoru}

%TKTK


\section*{O tekstu}

Arijan \textgreek[variant=ancient]{(Φλάουιος Ἀρριανός,} Nikomedija u Bitiniji, oko 95. – Atena, oko 175.) priredio je u \textit{Priručniku} \textgreek[variant=ancient]{(Ἐγχειρίδιον)} sažetak učenja stoičkog filozofa Epikteta (55.–135.) kao uvijek spreman vodič za susrete sa svakodnevicom.

U 31.\ poglavlju \textit{Priručnika} Arijan i Epiktet poučavaju o pobožnosti \textgreek[variant=ancient]{(εὐσέβεια).} Stoička je filozofija, naime, duboko religiozna, te je stoicizam katkad smatran vrstom osobne religije. Stoici smatraju da bogovi postoje i da pozitivno utječu na svijet, te se treba pokoravati njihovoj volji. Život u skladu s naravi \textgreek[variant=ancient]{(κατὰ φύσιν)} znači spoznati što možemo kontrolirati, a što ne \textgreek[variant=ancient]{(τὰ ἐφ' ἡμῖν, τὰ οὐκ ἐφ' ἡμῖν).} Budući da sva živa bića izbjegavaju i mrze ono što nanosi štetu \textgreek[variant=ancient]{(βλάπτειν),} važno je pravilno pojmiti pojam štete (i koristi).

%\newpage

\section*{Pročitajte naglas grčki tekst.}

Epict.\ Enchiridion 31

%Naslov prema izdanju

\medskip


{\large

\begin{greek}

\noindent τῆς περὶ τοὺς θεοὺς εὐσεβείας ἴσθι ὅτι τὸ κυριώτατον ἐκεῖνό ἐστιν, ὀρθὰς ὑπολήψεις περὶ αὐτῶν ἔχειν ὡς ὄντων καὶ διοικούντων τὰ ὅλα καλῶς καὶ δικαίως καὶ σαυτὸν εἰς τοῦτο κατατεταχέναι, τὸ πείθεσθαι αὐτοῖς καὶ εἴκειν πᾶσι τοῖς γινομένοις καὶ ἀκολουθεῖν ἑκόντα ὡς ὑπὸ τῆς ἀρίστης γνώμης ἐπιτελουμένοις. οὕτω γὰρ οὐ μέμψῃ ποτὲ τοὺς θεοὺς οὔτε ἐγκαλέσεις ὡς ἀμελούμενος.

\noindent ἄλλως δὲ οὐχ οἷόν τε τοῦτο γίνεσθαι, ἐὰν μὴ ἄρῃς ἀπὸ τῶν οὐκ ἐφ' ἡμῖν καὶ ἐν τοῖς ἐφ' ἡμῖν μόνοις θῇς τὸ ἀγαθὸν καὶ τὸ κακόν. ὡς, ἄν γέ τι ἐκείνων ὑπολάβῃς ἀγαθὸν ἢ κακόν, πᾶσα ἀνάγκη, ὅταν ἀποτυγχάνῃς ὧν θέλεις καὶ περιπίπτῃς οἷς μὴ θέλεις, μέμψασθαί σε καὶ μισεῖν τοὺς αἰτίους.

\noindent πέφυκε γὰρ πρὸς τοῦτο πᾶν ζῷον τὰ μὲν βλαβερὰ φαινόμενα καὶ τὰ αἴτια αὐτῶν φεύγειν καὶ ἐκτρέπεσθαι, τὰ δὲ ὠφέλιμα καὶ τὰ αἴτια αὐτῶν μετιέναι τε καὶ τεθηπέναι. ἀμήχανον οὖν βλάπτεσθαί τινα οἰόμενον χαίρειν τῷ δοκοῦντι βλάπτειν, ὥσπερ καὶ τὸ αὐτῇ τῇ βλάβῃ χαίρειν ἀδύνατον.

\end{greek}

}


\section*{Analiza i komentar}

%1

{\large
\begin{greek}
\noindent τῆς περὶ τοὺς θεοὺς εὐσεβείας \\
ἴσθι ὅτι \\
\tabto{2em} τὸ κυριώτατον \\
\tabto{2em} ἐκεῖνό ἐστιν, \\
\tabto{4em} ὀρθὰς ὑπολήψεις \\
\tabto{6em} περὶ αὐτῶν \\
\tabto{4em} ἔχειν \\
\tabto{6em} ὡς ὄντων καὶ διοικούντων \\
\tabto{8em} τὰ ὅλα \\
\tabto{6em} καλῶς καὶ δικαίως \\
\tabto{4em} καὶ σαυτὸν \\
\tabto{6em} εἰς τοῦτο \\
\tabto{4em} κατατεταχέναι, \\
\tabto{4em} τὸ πείθεσθαι αὐτοῖς \\
\tabto{4em} καὶ εἴκειν πᾶσι τοῖς γινομένοις \\
\tabto{4em} καὶ ἀκολουθεῖν ἑκόντα \\
\tabto{6em} ὡς ὑπὸ τῆς ἀρίστης γνώμης \\
\tabto{4em} ἐπιτελουμένοις. \\

\end{greek}
}

\begin{description}[noitemsep]
\item[τῆς\dots\ εὐσεβείας] genitiv ovisan o τὸ κυριώτατον
\item[ἴσθι] §~317.4; \textit{verbum sentiendi} otvara mjesto zavisnoj izričnoj rečenici
\item[ὅτι] veznik uvodi zavisnu izričnu rečenicu
\item[ἐκεῖνό ἐστιν] §~315; imenski predikat, Smyth 909; zamjenica najavljuje infinitive (s dopunama)
\item[ὑπολήψεις] LSJ s.~v.\ II
\item[ἔχειν\dots\ καὶ\dots\ κατατεταχέναι] §~231; §~272, složenica τάσσω, osnove s.~116; infinitivi ovisni o ἐκεῖνό
\item[ὡς ὄντων καὶ διοικούντων] poredbeni veznik: kao da\dots; §~315 (LSJ εἰμί A.I); §~243
\item[εἰς τοῦτο] zamjenica najavljuje (supstantivirane) infinitive
\item[τὸ πείθεσθαι] §~232; supstantivirani infinitiv, §~497
\item[εἴκειν] §~231
\item[τοῖς γινομένοις] §~232; jonski i helenistički oblik glagola γίγνομαι; supstantiviranje participa članom §~499
\item[ἀκολουθεῖν ἑκόντα] §~243; A+I
\item[ὡς ὑπὸ τῆς\dots] \textbf{γνώμης ἐπιτελουμένοις} poredbeni veznik: kao da\dots; vršitelj pasivne radnje izražen priložnom oznakom ὑπό τινος; §~243; supstantiviranje participa članom §~499

\end{description}

%2


{\large
\begin{greek}
\noindent οὕτω γὰρ οὐ μέμψῃ ποτὲ \\
\tabto{2em} τοὺς θεοὺς \\
οὔτε ἐγκαλέσεις \\
\tabto{2em} ὡς ἀμελούμενος.\\

\end{greek}
}

\begin{description}[noitemsep]
\item[γὰρ] čestica upozorava da se navodi razlog: naime\dots
\item[οὐ μέμψῃ\dots] \textbf{οὔτε ἐγκαλέσεις\dots}\ koordinacija rečeničnih članova (niječnim) sastavnim veznikom; §~258; §~259
\item[ὡς ἀμελούμενος] poredbeni veznik: kao da\dots; §~243

\end{description}
%3

{\large
\begin{greek}
\noindent ἄλλως δὲ οὐχ οἷόν τε \\
\tabto{2em} τοῦτο \\
γίνεσθαι, \\
\tabto{2em} ἐὰν μὴ ἄρῃς \\
\tabto{4em} ἀπὸ τῶν οὐκ ἐφ' ἡμῖν \\
\tabto{2em} καὶ ἐν τοῖς ἐφ' ἡμῖν μόνοις \\
\tabto{2em} θῇς \\
\tabto{2em} τὸ ἀγαθὸν καὶ τὸ κακόν. \\

\end{greek}
}%οἷον

\begin{description}[noitemsep]
\item[οἷόν τε\dots] fraza otvara mjesto infinitivu; LSJ οἷος III.2
\item[γίνεσθαι] §~232; jonski i helenistički oblik glagola γίγνομαι
\item[ἐὰν μὴ ἄρῃς] pogodbeni veznik uvodi pogodbenu protazu, ovdje iterativnog eventualnog oblika, §~476.2; §~254, osnove s.~118
\item[τῶν οὐκ ἐφ' ἡμῖν\dots\ τοῖς ἐφ' ἡμῖν] supstantiviranje članom svih vrsta riječi i izraza §~373
\item[θῇς] također predikat pogodbene protaze; §~306; LSJ τίθημι B.II.3
\item[τὸ ἀγαθὸν καὶ τὸ κακόν] supstantiviranje članom svih vrsta riječi §~373
\end{description}
%4

{\large
\begin{greek}
\noindent ὡς, \\
\tabto{2em} ἄν γέ τι \\
\tabto{4em} ἐκείνων \\
\tabto{2em} ὑπολάβῃς \\
\tabto{2em} ἀγαθὸν ἢ κακόν, \\
\tabto{2em} πᾶσα ἀνάγκη, \\
\tabto{4em} ὅταν ἀποτυγχάνῃς \\
\tabto{6em} ὧν θέλεις \\
\tabto{4em} καὶ περιπίπτῃς \\
\tabto{6em} οἷς μὴ θέλεις, \\
\tabto{2em} μέμψασθαί σε \\
\tabto{2em} καὶ μισεῖν \\
\tabto{4em} τοὺς αἰτίους.\\

\end{greek}
}

\begin{description}[noitemsep]
\item[ὡς] veznik uvodi uzročnu rečenicu §~473
\item[ἄν γέ\dots\ ὑπολάβῃς] pogodbeni veznik (ἄν < ἐάν) uvodi pogodbenu protazu, ovdje iterativnog eventualnog oblika, §~476.2; čestica γε naglašava protazu (možemo zamisliti da je veznik složen kurzivom: \textit{ako}\dots); §~254, složenica λαμβάνω, osnove §~321.14; LSJ ὑπολαμβάνω III
\item[πᾶσα ἀνάγκη] sc.\ ἐστί; imenski predikat, Smyth 909; fraza otvara mjesto infinitivu
\item[ὅταν ἀποτυγχάνῃς\dots\ καὶ περιπίπτῃς\dots] vremenska zavisna rečenica s predikatom oblika ἄν (ὅταν) + konjunktiv izriče iterativnu radnja u sadašnjosti §~488.2: kad god\dots; §~231; ἀποτυγχάνω τινός; περιπίπτω τινί
\item[ὧν θέλεις] zamjenica uvodi zavisnu odnosnu rečenicu, ujedno služi kao objekt ἀποτυγχάνῃς: ono što\dots; §~231
\item[οἷς μὴ θέλεις] zamjenica uvodi zavisnu odnosnu rečenicu, ujedno služi kao objekt περιπίπτῃς: onime što\dots; §~231
\item[μέμψασθαί\dots\ καὶ μισεῖν] oblici ovisni o πᾶσα ἀνάγκη; §~267; §~243
\item[τοὺς αἰτίους] supstantiviranje članom svih vrsta riječi §~373

\end{description}
%5

{\large
\begin{greek}
\noindent πέφυκε γὰρ \\
\tabto{2em} πρὸς τοῦτο \\
πᾶν ζῷον \\
\tabto{2em} τὰ μὲν βλαβερὰ φαινόμενα \\
\tabto{4em} καὶ τὰ αἴτια \\
\tabto{6em} αὐτῶν \\
\tabto{2em} φεύγειν καὶ ἐκτρέπεσθαι, \\
\tabto{2em} τὰ δὲ ὠφέλιμα \\
\tabto{4em} καὶ τὰ αἴτια \\
\tabto{6em} αὐτῶν \\
\tabto{2em} μετιέναι τε καὶ τεθηπέναι. \\

\end{greek}
}

\begin{description}[noitemsep]
\item[πέφυκε] §~272
\item[πρὸς τοῦτο] zamjenica otvara mjesto infinitivima
\item[τὰ μὲν βλαβερὰ φαινόμενα\dots\ τὰ δὲ ὠφέλιμα\dots] §~232; LSJ φαίνω B.II (glagol traži nužnu imensku dopunu); supstantiviranje participa članom §~499; koordinacija parom čestica
\item[φεύγειν καὶ ἐκτρέπεσθαι] φεύγω τινά §~231, ἐκτρέπω med. τινά §~232 (LSJ s.~v.\ A.2) 
\item[μετιέναι τε καὶ τεθηπέναι] koordinacija parom sastavnih veznika; složenica εἶμι, §~314.1; §~272, LSJ τέθηπα (glagol nema drugih osnova)

\end{description}
%6

{\large
\begin{greek}
\noindent ἀμήχανον οὖν \\
\tabto{4em} βλάπτεσθαί  \\
\tabto{2em} τινα οἰόμενον \\
\tabto{2em} χαίρειν \\
\tabto{4em} τῷ δοκοῦντι \\
\tabto{6em} βλάπτειν, \\
ὥσπερ καὶ \\
\tabto{2em} τὸ αὐτῇ τῇ βλάβῃ χαίρειν \\
ἀδύνατον.\\

\end{greek}
}

\begin{description}[noitemsep]
\item[ἀμήχανον οὖν] sc.\ ἐστί; imenski predikat, Smyth 909; fraza otvara mjesto A+I τινα\dots\ χαίρειν
\item[βλάπτεσθαί] §~232, oblik ovisan o οἰόμενον%rekcija tina
\item[οἰόμενον] §~232; kao \textit{verbum sentiendi} glagol ima nužnu dopunu u infinitivu
\item[χαίρειν] §~231; rekcija τινί
\item[τῷ δοκοῦντι] supstantiviranje participa članom §~499; glagol ima nužnu dopunu u infinitivu; u raspravi o pobožnosti χαίρειν τῷ δοκοῦντι βλάπτειν opisuje odnos pojedinca prema božanstvu (bogovima) u slučaju da taj pojedinac smatra da mu je nanesena šteta
\item[βλάπτειν] §~231
\item[ὥσπερ] veznik uvodi poredbenu rečenicu
\item[τὸ\dots\ χαίρειν] §~231; supstantiviranje infinitiva, §~497
\item[ἀδύνατον] sc.\ ἐστί; imenski predikat, Smyth 909

\end{description}



%kraj
