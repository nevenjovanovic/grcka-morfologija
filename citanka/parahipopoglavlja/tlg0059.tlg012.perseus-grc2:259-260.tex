%\section*{O autoru}

%TKTK


\section*{O tekstu}

\textit{Fedar} je Platonov dijalog sastavljen oko 370.\ pr.~Kr. Posvećen pitanju ljubavi, dotiče se i drugih tema (razmatraju se pitanja umijeća retorike u praksi, duše, metempsihoze, božanskog nadahnuća). Za razliku od ostalih Platonovih djela, u kojima se pripovijeda u retrospektivi, \textit{Fedar} pripovijeda zbivanja i radnju dok se odvijaju, riječima samih sudionika dijaloga, Sokrata i Fedra.

Susrevši se na periferiji Atene, Sokrat i Fedar odlaze u prirodu (ovo je jedini Platonov dijalog čija se radnja odvija u prirodi). Fedar je upravo slušao Liziju, glasovitog atenskog govornika, i njegov govor o ljubavi, a Sokrata živo interesira što je Lizija rekao. Fedar ima govor zapisan te ga naposljetku Sokrat nagovori da ga pročita. U sjeni drveća, uz zvuke prirode kraj potoka, njih dvojica nastavljaju razgovor. 

Dio koji čitamo Platonova je priča o cvrčcima, Sokratova digresija na koju ga je nadahnulo prirodno okruženje. Digresija je ujedno i period predaha u tijeku dijaloga, Platonov prijelaz na novu temu. No, živopisnost i ideja digresije pomogli su da i ona postane slavna. 

Dok cvrčci cvrče, prate hoće li njihova pjesma uspavati ljude ili će se oni moći oduprijeti. U rana vremena, dok još nije bilo Muza, cvrčci su bili ljudi. Kada je s Muzama nastala i pjesma, cvrčci su se prepustili pjevanju i plesu. Izgubivši interes za hranu i spavanje, umirali bi i nesvjesni smrti. Muze su ih nagradile darom da pjevaju od rođenja do smrti, ali su im dale i zadatak: oni svakoj od Muza javljaju koji ih čovjek štuje, a koji ne.


\newpage

\section*{Pročitajte naglas grčki tekst.}

Plat.\ Phaedrus 259b

%Naslov prema izdanju

\medskip


{\large

\begin{greek}

\noindent ΣΩ. (\dots) λέγεται δʼ ὥς ποτʼ ἦσαν οὗτοι ἄνθρωποι τῶν πρὶν μούσας γεγονέναι, γενομένων δὲ Μουσῶν καὶ φανείσης ᾠδῆς οὕτως ἄρα τινὲς τῶν τότε ἐξεπλάγησαν ὑφʼ ἡδονῆς, ὥστε ᾁδοντες ἠμέλησαν σίτων τε καὶ ποτῶν, καὶ ἔλαθον τελευτήσαντες αὑτούς· ἐξ ὧν τὸ τεττίγων γένος μετʼ ἐκεῖνο φύεται, γέρας τοῦτο παρὰ Μουσῶν λαβόν, μηδὲν τροφῆς δεῖσθαι γενόμενον, ἀλλʼ ἄσιτόν τε καὶ ἄποτον εὐθὺς ᾁδειν, ἕως ἂν τελευτήσῃ, καὶ μετὰ ταῦτα ἐλθὸν παρὰ μούσας ἀπαγγέλλειν τίς τίνα αὐτῶν τιμᾷ τῶν ἐνθάδε. Τερψιχόρᾳ μὲν οὖν τοὺς ἐν τοῖς χοροῖς τετιμηκότας αὐτὴν ἀπαγγέλλοντες ποιοῦσι προσφιλεστέρους, τῇ δὲ Ἐρατοῖ τοὺς ἐν τοῖς ἐρωτικοῖς, καὶ ταῖς ἄλλαις οὕτως, κατὰ τὸ εἶδος ἑκάστης τιμῆς· τῇ δὲ πρεσβυτάτῃ Καλλιόπῃ καὶ τῇ μετʼ αὐτὴν Οὐρανίᾳ τοὺς ἐν φιλοσοφίᾳ διάγοντάς τε καὶ τιμῶντας τὴν ἐκείνων μουσικὴν ἀγγέλλουσιν, αἳ δὴ μάλιστα τῶν Μουσῶν περί τε οὐρανὸν καὶ λόγους οὖσαι θείους τε καὶ ἀνθρωπίνους ἱᾶσιν καλλίστην φωνήν. πολλῶν δὴ οὖν ἕνεκα λεκτέον τι καὶ οὐ καθευδητέον ἐν τῇ μεσημβρίᾳ.
\end{greek}

}

\newpage

\section*{Analiza i komentar}

%1

{\large
\begin{greek}
\noindent λέγεται δ' \\
\tabto{2em} ὥς ποτ' ἦσαν οὗτοι ἄνθρωποι \\
\tabto{4em} τῶν πρὶν Μούσας γεγονέναι, \\
\tabto{2em} γενομένων δὲ Μουσῶν \\
\tabto{2em} καὶ φανείσης ᾠδῆς \\
\tabto{4em} οὕτως ἄρα τινὲς \\
\tabto{6em} τῶν τότε \\
\tabto{4em} ἐξεπλάγησαν \\
\tabto{6em} ὑφ' ἡδονῆς, \\
\tabto{4em} ὥστε \\
\tabto{4em} ᾄδοντες \\
\tabto{4em} ἠμέλησαν \\
\tabto{6em} σίτων τε καὶ ποτῶν, \\
\tabto{4em} καὶ ἔλαθον \\
\tabto{6em} τελευτήσαντες αὑτούς·

\tabto{6em} ἐξ ὧν \\
\tabto{6em} τὸ τεττίγων γένος \\
\tabto{8em} μετ' ἐκεῖνο \\
\tabto{6em} φύεται, \\
\tabto{6em} γέρας τοῦτο \\
\tabto{8em} παρὰ Μουσῶν \\
\tabto{6em} λαβόν, \\
\tabto{8em} μηδὲν τροφῆς \\
\tabto{6em} δεῖσθαι \\
\tabto{6em} γενόμενον, \\
\tabto{6em} ἀλλ' ἄσιτόν τε \\
\tabto{6em} καὶ ἄποτον \\
\tabto{6em} εὐθὺς ᾄδειν, \\
\tabto{8em} ἕως ἂν τελευτήσῃ, \\
\tabto{6em} καὶ \\
\tabto{8em} μετὰ ταῦτα \\
\tabto{6em} ἐλθὸν \\
\tabto{8em} παρὰ Μούσας \\
\tabto{8em} ἀπαγγέλλειν \\
\tabto{10em} τίς \\
\tabto{10em} τίνα αὐτῶν \\
\tabto{10em} τιμᾷ \\
\tabto{12em} τῶν ἐνθάδε. \\

\end{greek}
}

\begin{description}[noitemsep]
\item[δ'] čestica (bez korelacije s μέν), ovdje u kopulativnoj funkciji povezivanja rečenica; dopunjava se prethodni iskaz s kojim ovaj ne stoji u suprotnosti
\item[λέγεται] §~231, osnove §327.7
\item[ὥς] veznik ὥς uvodi zavisnu izričnu rečenicu kao dopunu (objekt) glagolu govorenja λέγεται
\item[οὗτοι] οὗτοι se odnosi na οἱ τέττιγες iz (ovdje necitiranog) poglavlja 258e
\item[ἦσαν] §~315.2; kopulativni glagol (glagolski dio imenskoga predikata) otvara mjesto nužnoj dopuni, ovdje imenskoj riječi
\item[ἄνθρωποι ] imenski dio imenskoga predikata 
\item[τῶν πρὶν Μούσας γεγονέναι] sc.\ τινὲς τῶν ὄντων πρίν\dots; §~291.b; vremenski veznik πρίν uvodi zavisnu vremensku rečenicu u kojoj stoji infinitiv (ovdje A+I) ako je glavna rečenica potvrdna §~488. bilj. 1
\item[ὥς ποτ' ἦσαν\dots\ γενομένων δὲ\dots] koordinacija pomoću čestice
\item[γενομένων δὲ Μουσῶν] GA
\item[γενομένων] §~254
\item[φανείσης ᾠδῆς ] GA
\item[φανείσης] §~292, osnove s.~118
\item[ἐξεπλάγησαν] §~292, složenica πλήσσω, LSJ ἐκπλήσσω, A.II.2
\item[ὥστε\dots\ ἠμέλησαν] posljedični veznik ὥστε uvodi zavisnu posljedičnu rečenicu, u glavnoj rečenici mjesto im otvara adverb οὕτως, §~473
\item[ᾄδοντες] §~231
\item[ἠμέλησαν] rekcija τινος; §~267
\item[ἔλαθον] §~254, osnove §~321.15
\item[τελευτήσαντες] §~267
\item[ἔλαθον τελευτήσαντες] τελευτήσαντες je predikatni particip kao dopuna glagolu ἔλαθον; na hrvatski se ἔλαθον prevodi prilogom, (``potajno''), a predikatni particip τελευτήσαντες postaje glavni glagol
\item[ἐξ ὧν\dots\ φύεται] odnosna zamjenica u prijedložnom izrazu uvodi zavisnu odnosnu rečenicu, antecedent je n.~pl.\ m.~r.\ οἵ (sc.\ οἱ ἄνθρωποι)
\item[φύεται] §~231
\item[λαβόν] §~254
\item[δεῖσθαι] §~243, osnove §~325.14 i §~325.15; γέρας τοῦτο sastoji se u μηδὲν τροφῆς δεῖσθαι i εὐθὺς ᾄδειν (kao adnominalni genitiv, Smyth 1290 i 1291, 1295 i 1296)
\item[γενόμενον] §~254
\item[ᾄδειν] §~231
\item[τελευτήσῃ] §~267
\item[ἕως ἂν τελευτήσῃ] vremenski veznik ἕως uvodi zavisnu vremensku rečenicu sa značenjem eventualne sadašnje iterativne rečenice: dok god ne\dots, §~488
\item[ἐλθὸν] §~254
\item[ἀπαγγέλλειν] §~231, složenica ἀγγέλλω, osnove s.~118
\item[τιμᾷ] §~243
\item[τίς τίνα αὐτῶν τιμᾷ τῶν ἐνθάδε] τίς τῶν ἐνθάδε, τίνα αὐτῶν (genitivi partitivni)

\end{description}

%2


{\large
\begin{greek}
\noindent 
Τερψιχόρᾳ μὲν οὖν \\
\tabto{2em} τοὺς \\
\tabto{4em} ἐν τοῖς χοροῖς \\
\tabto{2em} τετιμηκότας αὐτὴν \\
ἀπαγγέλλοντες \\
\tabto{2em} ποιοῦσι προσφιλεστέρους, \\
τῇ δὲ ᾿Ερατοῖ \\
\tabto{2em} τοὺς \\
\tabto{4em} ἐν τοῖς ἐρωτικοῖς, \\
καὶ ταῖς ἄλλαις \\
\tabto{2em} οὕτως, \\
\tabto{4em} κατὰ τὸ εἶδος \\
\tabto{6em} ἑκάστης \\
\tabto{8em} τιμῆς· \\
τῇ δὲ πρεσβυτάτῃ Καλλιόπῃ \\
καὶ τῇ \\
\tabto{2em} μετ' αὐτὴν \\
Οὐρανίᾳ \\
\tabto{2em} τοὺς \\
\tabto{4em} ἐν φιλοσοφίᾳ \\
\tabto{2em} διάγοντάς τε \\
\tabto{2em} καὶ τιμῶντας \\
\tabto{4em} τὴν ἐκείνων μουσικὴν \\
ἀγγέλλουσιν, \\
\tabto{2em} αἳ δὴ μάλιστα \\
\tabto{4em} τῶν Μουσῶν \\
\tabto{4em} περί τε οὐρανὸν \\
\tabto{6em} καὶ λόγους \\
\tabto{2em} οὖσαι \\
\tabto{6em} θείους τε καὶ ἀνθρωπίνους \\
\tabto{2em} ἱᾶσιν \\
\tabto{4em} καλλίστην φωνήν.\\

\end{greek}
}

\begin{description}[noitemsep]
\item[μὲν οὖν ] u ovoj kombinaciji μὲν ističe iskaz koji slijedi, a οὖν ističe logičku povezanost prethodnog i novog iskaza: onda dakle\dots
\item[τοὺς τετιμηκότας] §~272
\item[ἀπαγγέλλοντες] rekcija τινί τι; §~231, složenica ἀγγέλλω
\item[ποιοῦσι] §~243, s dopunom u dva akuzativa (προσφιλεστέρους i αὐτὴν)
\item[τοὺς ἐν τοῖς ἐρωτικοῖς] prethodno iskazani particip τετιμηκότας ovdje je neizrečen \textgreek[variant=ancient]{(τοὺς ἐν τοῖς ἐρωτικοῖς} sc.\ \textgreek[variant=ancient]{τετιμηκότας)}
\item[διάγοντάς] §~231, složenica ἄγω
\item[περί τε οὐρανὸν καὶ λόγους] kombinacija čestica (sastavnih veznika) τε καὶ povezuje komplemente u slabiju vezu nego καί\dots\ καί\dots; u paru τε καὶ drugi je član istaknutiji
\item[τιμῶντας] §~243
\item[ἀγγέλλουσιν] §~231
\item[αἳ δὴ] čestica ovdje ističe i intenzivira značenje riječi: baš one\dots
\item[οὖσαι] §~315, LSJ εἰμί IV.6 baviti se čime (εἶναι πρός τινι ili εἶναι περί τι)
\item[ἱᾶσιν] §~305
\item[αἳ\dots\ ἱᾶσιν] odnosna zamjenica αἳ uvodi zavisnu odnosnu rečenicu, antecedent su \textgreek[variant=ancient]{τῇ δὲ\dots\ Καλλιόπῃ καὶ τῇ\dots\ Οὐρανίᾳ}

\end{description}

%3

{\large
\begin{greek}
\noindent 
πολλῶν \\
\tabto{2em} δὴ οὖν \\
ἕνεκα \\
\tabto{2em} λεκτέον τι \\
\tabto{2em} καὶ οὐ καθευδητέον \\
\tabto{4em} ἐν τῇ μεσημβρίᾳ.\\

\end{greek}
}

\begin{description}[noitemsep]
\item[πολλῶν δὴ οὖν ἕνεκα] Sokrat izvlači konkretan zaključak iz svoje priče: dakle, iz mnogih razloga nas dvojica moramo\dots
\item[δὴ οὖν] ove dvije čestice u kombinaciji (vrlo česte baš kod Platona) iskazuju potvrđivanje i isticanje: dakle, očito\dots
\item[λεκτέον] glagolski pridjev §~300
\item[καθευδητέον] glagolski pridjev §~300
\end{description}

%kraj
