\section*{O autoru}

Epikur \textgreek[variant=ancient]{(Ἐπίκουρος,} oko 341.\ – oko 271.\ p.~n.~e), rođen na Samu, sin atenskih građana, djelovao isprva u Maloj Aziji, da bi 306.\ u Ateni, konkurirajući Akademiji, Peripatu i Stoi, otvorio školu \textgreek[variant=ancient]{(Κῆπος,} ``vrt'', prema vili s vrtom u kojoj je škola djelovala). Epikur je bio iznimno karizmatičan učitelj. Pošto je, nakon duge i teške bolesti, koju je hrabro trpio, umro, godišnjicu njegova rođenja (10.\ gamelij, približno u siječnju) učenici su obilježavali gozbom.

Epikur je sastavio stotinjak spisa od kojih se vrlo malo sačuvalo. Svoj je sustav najpotpunije prikazao u djelu \textgreek[variant=ancient]{Περì φύσεως} (u 37 knjiga, danas izgubljeno); djelo je rimski pjesnik Lukrecije (96.–55.\ p.~n.~e) prepjevao u latinski didaktički ep \textit{De rerum natura}. Najvažnija načela svojeg sustava Epikur je jednostavno i sažeto protumačio u nizu pisama, od kojih su, zahvaljujući Diogenu Laertiju \textgreek[variant=ancient]{(Διογένης Λαέρτıος,} III.~st.\ n.~e, Epikuru je posvetio čitavu desetu, i posljednju, knjigu svojih \textgreek[variant=ancient]{Βίοι καὶ γνῶμαι τῶν ἐν φιλοσοφίᾳ εὐδοκιμησάντων,} \textit{Životopisi i misli znamenitih filozofa}), sačuvana tri pisma: izvjesnom Herodotu, o fizici (Epikur slijedi Demokritovo učenje o atomima), Pitoklu o astronomiji i meteorologiji, te Menekeju o etici. Sačuvane su i dvije kasnije nastale zbirke Epikurovih filozofskih sentencija, \textgreek[variant=ancient]{Κύρıαı δόξαı} (\textit{Glavni nauci}, 40 izreka) i \textit{Gnomologium Vaticanum} (\textit{Vatikanske izreke}, po rukopisnom kodeksu Vatikanske biblioteke u kojem su sačuvane; 81 izreka). Važan su izvor epikurejskih tekstova i pougljenjene papirusne knjige iz Vile Pizona u Herkulaneju, gdje je djelovala epikurejska škola pod vodstvom Filodema iz Gadare \textgreek[variant=ancient]{(Φιλόδημος ὁ Γαδαρεύς,} oko 110.\ – oko 35.\ p.~n.~e). Jedan od papirusa iz Herkulaneja (pap.\ Herc.\ 1005, 4.9–14) sačuvao je i najvažnija Epikurova učenja sažeta u τετραφάρμακος (``lijek od četiri sastojka''): \textgreek[variant=ancient]{Ἄφοβον ὁ θεός, ἀνύποπτον ὁ θάνατος, καὶ τἀγαθὸν μὲν εὔκτητον, τὸ δὲ δεινὸν εὐκαρτέρητον.}

Od tri područja filozofije – logike, fizike i etike – Epikur se najviše bavio potonjim dvama, koja izravno utječu na ``dobar život'' \textgreek[variant=ancient]{(τὸ καλῶς ζῆν),} cilj njegova nauka; etika poučava o dužnostima i stavovima dok fizika otkriva tajne prirode i uči da ih se ne treba bojati. 

Dobro se za Epikura izjednačava s ugodom \textgreek[variant=ancient]{(ἡδονή);} najviša je ugoda definirana negativno, kao odsutnost tjelesne i duševne neugode: ἀταραξία. Svakoj ugodi prethodi ili slijedi neka neugoda, i treba realno procijeniti njihov odnos da bi se vidjelo je li ugoda vrijedna truda, i je li neugoda vrijedna podnošenja. Ovakvo shvaćanje Epikura vodi do vizije humanoga društva zasnovanog na pragmatičnim vrijednostima (pravda, koja sama po sebi nije ugoda, omogućava sigurnost i korist, te je zato sekundarno dobro; slično je i s prijateljstvom itd). Ugoda se, smatrao je Epikur, najpouzdanije postiže u intimnom prijateljskom krugu, a političko djelovanje siguran je izvor neugode; otud važno epikurejsko načelo \textgreek[variant=ancient]{λάθε βιώσας} (zapisao Plutarh, mor.~1128), dijametralno suprotno načelima npr.\ rimske elite.

Epikurovo je učenje bilo iznimno popularno u helenističkom i carskom razdoblju grčke i rimske antike (uz Lukrecija, epikurejac je i Horacije); nije bilo ograničeno na dobrostojeće muškarce, u κῆπος su pristup imali pripadnici nižih klasa, žene, robovi, barbari kao i Grci; još u III.\ i IV.~st.\ n.~e.\ kršćanski se propovjednici ogorčeno bore protiv epikurejskih nauka.

\section*{O tekstu}

\textit{Pismo Menekeju} rješava tri najvažnija problema epikurejske filozofije: pitanje bogova, smrti, sreće. U ovdje odabranom odlomku jednostavnim, neukrašenim stilom Epikur pokazuje da se ljudi boje smrti zbog boli – ali ne boli ih sama smrt, već pate zbog tjeskobe njezina očekivanja. Ta je tjeskoba neosnovana. Odlomak uključuje i jednu od najslavnijih Epikurovih formula: dok ima nas, smrti nema, a kad ima smrti, nema nas.

%\newpage

\section*{Pročitajte naglas grčki tekst.}

Epicur.\ Epistula ad Menoeceum 124–126

%Naslov prema izdanju

\medskip


{\large

\begin{greek}

\noindent Συνέθιζε δὲ ἐν τῷ νομίζειν μηδὲν πρὸς ἡμᾶς εἶναι τὸν θάνατον· ἐπεὶ πᾶν ἀγαθὸν καὶ κακὸν ἐν αἰσθήσει· στέρησις δέ ἐστιν αἰσθήσεως ὁ θάνατος. ὅθεν γνῶσις ὀρθὴ τοῦ μηθὲν εἶναι πρὸς ἡμᾶς τὸν θάνατον ἀπολαυστὸν ποιεῖ τὸ τῆς ζωῆς θνητόν, οὐκ ἄπειρον προστιθεῖσα χρόνον, ἀλλὰ τὸν τῆς ἀθανασίας ἀφελομένη πόθον. οὐθὲν γάρ ἐστιν ἐν τῷ ζῆν δεινὸν τῷ κατειληφότι γνησίως τὸ μηδὲν ὑπάρχειν ἐν τῷ μὴ ζῆν δεινόν. ὥστε μάταιος ὁ λέγων δεδιέναι τὸν θάνατον οὐχ ὅτι λυπήσει παρών, ἀλλ' ὅτι λυπεῖ μέλλων. ὃ γὰρ παρὸν οὐκ ἐνοχλεῖ, προσδοκώμενον κενῶς λυπεῖ. τὸ φρικωδέστατον οὖν τῶν κακῶν ὁ θάνατος οὐθὲν πρὸς ἡμᾶς, ἐπειδήπερ ὅταν μὲν ἡμεῖς ὦμεν, ὁ θάνατος οὐ πάρεστιν, ὅταν δὲ ὁ θάνατος παρῇ, τόθ' ἡμεῖς οὐκ ἐσμέν.

\end{greek}

}


\section*{Komentar}

%1

\begin{description}[noitemsep]
\item[πρὸς ἡμᾶς] πρός LSJ C.III.1 ``u odnosu na\dots'', ``što se tiče\dots''
\item[ἐν αἰσθήσει] sc. ἐστιν

\end{description}

\begin{description}[noitemsep]
\item[μηθὲν] alternativni oblik za μηδέν
\end{description}

%3

\begin{description}[noitemsep]
\item[οὐθὲν] alternativni (kasniji) oblik umjesto οὐδέν
\item[γνησίως] u Epikurovu filozofskom diskurzu ovaj prilog označava razliku između površnog i dubinskog uvjerenja; svi mogu reći ``ne bojim se smrti'', ali ne mogu svi to uvjerenje prevesti u praksu
\end{description}

%4
\begin{description}[noitemsep]
\item[μάταιος ὁ λέγων] pridjev upotrijebljen uz predikat odgovara hrvatskom prilogu (potpumbeni predikat), § 369
\item[μέλλων] § 231; μέλλω izriče da će se nešto dogoditi, pa se μέλλων upotrebljava u značenju našeg pridjeva ``budući''
\end{description}

%5


%6

%kraj
