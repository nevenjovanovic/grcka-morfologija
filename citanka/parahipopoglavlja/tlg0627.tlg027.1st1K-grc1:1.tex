\section*{O autoru}

\textgreek[variant=ancient]{Ἱπποκράτης ὁ Κῷος,} Hipokrat s Kosa (oko 460.\ – oko 370.\ pr.~Kr.) najznamenitiji je grčki liječnik, jedna od najvažnijih ličnosti povijesti medicine; često se naziva i ocem medicine. Smatra se utemeljiteljem hipokratske škole medicine koja je liječništvo utemeljila kao zasebnu znanstvenu disciplinu, odvojivši je s jedne strane od filozofije i s druge strane od magijskih i nadriliječničkih praksi. 

Hipokratovo medicinsko učenje sačuvano je u medicinskim spisima zbirke \textit{Corpus Hippocraticum}, šezdesetak djela uglavnom nepoznatih autora; autorstvo samoga Hipokrata nije dokazano ni za jedan spis. Zbirka okuplja djela takozvane koške škole, središta stjecanja i prakticiranja medicinskog znanja s otoka Kosa, a nastala je u drugoj polovici 5.~st.\ pr.~Kr.

Tekstovi hipokratskog korpusa pripadaju jonskoj prozi; u počecima formiranja proznog izraza nekoliko značajnih medicinskih autora bilo je porijeklom iz jonske Grčke, te je njihov dijalekt postao kanonskim za medicinske tekstove. 

\section*{O tekstu}

Odlomak koji donosimo dio je spisa \textgreek[variant=ancient]{Περὶ ἱερῆς νούσου,} \textit{O svetoj bolesti}, posvećenog epilepsiji (ili nizu neuroloških poremećaja s epilepsiji sličnim simptomima), za koju se u starini smatralo da je uzrokuju i ljudima daju božanstva. Osporavajući takva poimanja hipokratski autor već na početku iznosi stav da je epilepsija božanska utoliko koliko su i sve ostale bolesti, uzrokovane prirodnim elementima, božanske. Sveta se bolest, prema hipokratskom autoru, može liječiti, ali to nikako ne smije biti uz pomoć raznih magijskih ili drugih nemedicinskih, neznanstvenih postupaka.

Od obilježja jonskog dijalekta, uz uobičajeno jonsko eta, ovdje ćemo naići i na nestegnute oblike \textgreek[variant=ancient]{(καλεομένης, θεοσεβέες, δοκέουσιν\dots),} na kapa refleks umjesto pi oblika \textgreek[variant=ancient]{(ὅκως} umjesto \textgreek[variant=ancient]{ὅπως),} na jonske oblike deklinacije \textgreek[variant=ancient]{(προφάσιος)} i drugo \textgreek[variant=ancient]{(νούσημα} umjesto \textgreek[variant=ancient]{νόσημα).}

%\newpage

\section*{Pročitajte naglas grčki tekst.}

Hippoc.\ De morbo sacro 1

%Naslov prema izdanju

\medskip


{\large

\begin{greek}

\noindent Περὶ μὲν τῆς ἱερῆς νούσου καλεομένης ὧδ’ ἔχει· οὐδέν τί μοι δοκέει τῶν ἄλλων θειοτέρη εἶναι νούσων οὐδὲ ἱερωτέρη, ἀλλὰ φύσιν μὲν ἔχει ἣν καὶ τὰ λοιπὰ νουσήματα, ὅθεν γίνεται. Φύσιν δὲ αὐτῇ καὶ πρόφασιν οἱ ἄνθρωποι ἐνόμισαν θεῖόν τι πρῆγμα εἶναι ὑπὸ ἀπειρίης καὶ θαυμασιότητος, ὅτι οὐδὲν ἔοικεν ἑτέρῃσι νούσοισιν· καὶ κατὰ μὲν τὴν ἀπορίην αὐτοῖσι τοῦ μὴ γινώσκειν τὸ θεῖον αὐτῇ διασώζεται, κατὰ δὲ τὴν εὐπορίην τοῦ τρόπου τῆς ἰήσιος ᾧ ἰῶνται, ἀπόλλυται, ὅτι καθαρμοῖσί τε ἰῶνται καὶ ἐπαοιδῇσιν. Εἰ δὲ διὰ τὸ θαυμάσιον θεῖον νομιεῖται, πολλὰ τὰ ἱερὰ νουσήματα ἔσται καὶ οὐχὶ ἓν, ὡς ἐγὼ ἀποδείξω ἕτερα οὐδὲν ἧσσον ἐόντα θαυμάσια οὐδὲ τερατώδεα, ἃ οὐδεὶς νομίζει ἱερὰ εἶναι. Τοῦτο μὲν γὰρ οἱ πυρετοὶ οἷ ἀμφημερινοὶ καὶ οἱ τριταῖοι καὶ οἱ τεταρταῖοι οὐδὲν ἧσσόν μοι δοκέουσιν ἱεροὶ εἶναι καὶ ὑπὸ θεοῦ γίνεσθαι ταύτης τῆς νούσου, ὧν οὐ θαυμασίως γ’ ἔχουσιν· τοῦτο δὲ ὁρέω μαινομένους ἀνθρώπους καὶ παραφρονέοντας ἀπὸ μηδεμιῆς προφάσιος ἐμφανέος, καὶ πολλά τε καὶ ἄκαιρα ποιέοντας, ἔν τε τῷ ὕπνῳ οἶδα πολλοὺς οἰμώζοντας καὶ βοῶντας, τοὺς δὲ πνιγομένους, τοὺς δὲ καὶ ἀναΐσσοντάς τε καὶ φεύγοντας ἔξω καὶ παραφρονέοντας μέχρις ἂν ἐπέγρωνται, ἔπειτα δὲ ὑγιέας ἐόντας καὶ φρονέοντας ὥσπερ καὶ πρότερον, ἐόντας τ’ αὐτέους ὠχρούς τε καὶ ἀσθενέας, καὶ ταῦτα οὐχ ἅπαξ, ἀλλὰ πολλάκις, ἄλλα τε πολλά ἐστι καὶ παντοδαπὰ ὧν περὶ ἑκάστου λέγειν πουλὺς ἂν εἴη λόγος. Ἐμοὶ δὲ δοκέουσιν οἱ πρῶτοι τοῦτο τὸ νόσημα ἀφιερώσαντες τοιοῦτοι εἶναι ἄνθρωποι οἷοι καὶ νῦν εἰσι μάγοι τε καὶ καθάρται καὶ ἀγύρται καὶ ἀλαζόνες, ὁκόσοι δὴ προσποιέονται σφόδρα θεοσεβέες εἶναι καὶ πλέον τι εἰδέναι.

\end{greek}

}


\section*{Komentar}

%1

\begin{description}[noitemsep]
\item[καλεομένης] §~243, jonski (nestegnuti) oblik umjesto καλουμένης
\item[ὧδ' ἔχει] ἔχω + adverb: biti\dots + adverb ili pridjev
\item[μοι δοκέει ] §~243, jonski umjesto δοκεῖ, glagol otvara mjesto dopuni u vidu nominativa s inifinitivom §~491.2
\item[ἣν ] odnosna zamjenica ἣν uvodi zavisnu odnosnu rečenicu, odnosi se na riječ φύσιν; glagol je neizrečen jer isti kao i u glavnoj rečenici (ἔχει)
\item[γίνεται] §~231 (osnove §~325.11), jonski oblik umjesto γίγνομαι
\item[ὅθεν γίνεται] odnosni prilog ὅθεν uvodi zavisnu odnosnu rečenicu: odakle, iz čega

\end{description}

%2

\begin{description}[noitemsep]
\item[τοῦ μὴ γινώσκειν] §~231, jonski oblik umjesto γιγνώσκειν (osnove §~324.12), supstantivirani infinitiv §~497

\end{description}

%3

\begin{description}[noitemsep]
\item[Εἰ… νομιεῖται] pogodbeni veznik εἰ uvodi zavisnu realnu pogodbenu rečenicu (stvarni uvjet ispunjenja radnje)
\item[ἐόντα] §~315, jonski oblik umjesto ὄντα, predikatni particip: da…
\item[ἃ… νομίζει] odnosna zamjenica ἃ uvodi zavisnu odnosnu rečenicu, odnosi se na \textgreek[variant=ancient]{ἕτερα (νουσήματα)}
\end{description}

%4

\begin{description}[noitemsep]
\item[μοι δοκέουσιν] §~243, jonski oblik umjesto δοκοῦσιν, otvara mjesto dopuni u formi nominativa s infinitivom §~491.2: da…
\item[ὧν… ἔχουσιν] odnosna zamjenica ὧν uvodi zavisnu odnosnu rečenicu, odnosi se na nominative iz glavne rečenice
\item[ὁρέω] §~243, jonski (nestegnuti) oblik umjesto ὁράω, akuzativi koji slijede dopune su ovom glagolu
\item[παραφρονέοντας] §~243, jonski (nestegnuti) oblik umjesto παραφρονοῦντας
\item[ποιέοντας] §~243, jonski (nestegnuti) oblik umjesto ποιοῦντας
\item[παραφρονέοντας] §~243, jonski (nestegnuti) oblik umjesto παραφρονοῦντας
\item[ἐπέγρωνται] složenica ἐγείρω, §~292; osnovu vidi LSJ s. v.%jaki aorist mediopasivni: ἠγρόμην!!!
\item[μέχρις ἂν ἐπέγρωνται] vremenski veznik μέχρις uvodi zavisnu vremensku rečenicu pogodbenog značenja (eventualna iterativna rečenica): „dok god…“
\item[ἐόντας] §~315, jonski oblik (nestegnuti) umjesto ὄντας
\item[φρονέοντας] §~243, jonski (nestegnuti) oblik umjesto φρονοῦντας
\item[ἐόντας] §~315, jonski oblik (nestegnuti) umjesto ὄντας
\item[παντοδαπὰ] (ἐστι) imenski predikat; kopula nije izrečena jer je jednaka prethodnoj
\item[ὧν] \textbf{περὶ ἑκάστου λέγειν πουλὺς ἂν εἴη λόγος} odnosna zamjenica ὧν uvodi zavisnu odnosnu rečenicu: od toga…

\end{description}

%5

\begin{description}[noitemsep]
\item[δοκέουσιν] §~243, jonski nestegnuti oblik umjesto δοκοῦσιν, otvara mjesto dopuni u formi nominativa s infinitivom §~491.2
\item[τοιοῦτοι… οἷοι… ὁκόσοι] niz korelativnih rečenica koje uvode korelativne zamjenice: takvi… koji… koliki…
\item[καθάρται] čistitelji, oni koji izvode ritual pročišćavanja
\item[προσποιέονται] §~243, jonski nestegnuti oblik umjesto προσποιοῦνται, otvara mjesto dopuni u vidu nominativa s infinitivom
\item[εἰδέναι] §~317.4, nominativ s infinitivom §~491.2 (nominativ se ne ponavlja)

\end{description}



%kraj
