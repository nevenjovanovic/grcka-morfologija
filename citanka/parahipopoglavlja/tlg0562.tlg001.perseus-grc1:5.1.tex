%\section*{O autoru}

%TKTK


\section*{O tekstu}

Odlomak koji ovdje čitamo početak je pete od dvanaest knjiga djela \textit{Razgovori sa samim sobom} ili \textit{Samom sebi} \textgreek[variant=ancient]{(Τὰ εἰς ἑαυτὸν),} (poznato i kao \textit{Meditacije} ili \textit{Razmišljanja}); djelo je u posljednjem desetljeću života sastavio rimski car i filozof Marko Aurelije Antonin (121.–180.).

U dijaloškoj formi čovjeka se potiče da ne bude lijen i neumjeren. Priroda nam je sve odredila s mjerom – odmor, jelo, piće – i dala nam je narav da radimo ljudske stvari. Jedan od smisla ljudskog postojanja jest raditi.



%\newpage

\section*{Pročitajte naglas grčki tekst.}

M.\ Aur.\ Ad se ipsum 5.1

%Naslov prema izdanju

\medskip


{\large

\begin{greek}

\noindent  Ὄρθρου, ὅταν δυσόκνως ἐξεγείρῃ, πρόχειρον ἔστω ὅτι ἐπὶ ἀνθρώπου ἔργον ἐγείρομαι· τί οὖν δυσκολαίνω, εἰ πορεύομαι ἐπὶ τὸ ποιεῖν ὧν ἕνεκεν γέγονα καὶ ὧν χάριν προῆγμαι εἰς τὸν κόσμον; ἢ ἐπὶ τοῦτο κατεσκεύασμαι, ἵνα κατακείμενος ἐν στρωματίοις ἐμαυτὸν θάλπω; ``ἀλλὰ τοῦτο ἥδιον.'' πρὸς τὸ ἥδεσθαι οὖν γέγονας, ὅλως δὲ σὺ πρὸς πεῖσιν ἢ πρὸς ἐνέργειαν; οὐ βλέπεις τὰ φυτάρια, τὰ στρουθάρια, τοὺς μύρμηκας, τοὺς ἀράχνας, τὰς μελίσσας τὸ ἴδιον ποιούσας, τὸ καθ' αὑτὰς συγκροτούσας κόσμον; ἔπειτα σὺ οὐ θέλεις τὰ ἀνθρωπικὰ ποιεῖν; οὐ τρέχεις ἐπὶ τὸ κατὰ τὴν σὴν φύσιν; 

\noindent ``ἀλλὰ δεῖ καὶ ἀναπαύεσθαι.'' φημὶ κἀγώ· ἔδωκε μέντοι καὶ τούτου μέτρα ἡ φύσις, ἔδωκε μέντοι καὶ τοῦ ἐσθίειν καὶ πίνειν καὶ ὅμως σὺ ὑπὲρ τὰ μέτρα, ὑπὲρ τὰ ἀρκοῦντα προχωρεῖς, ἐν δὲ ταῖς πράξεσιν οὐκέτι, ἀλλ̓ ἐντὸς τοῦ δυνατοῦ. οὐ γὰρ φιλεῖς σεαυτόν, ἐπεί τοι καὶ τὴν φύσιν ἄν σου καὶ τὸ βούλημα ταύτης ἐφίλεις.

\end{greek}

}


\section*{Analiza i komentar}

%1

{\large
\begin{greek}
\noindent ῎Ορθρου, \\
\tabto{2em} ὅταν δυσόκνως ἐξεγείρῃ,\\
πρόχειρον ἔστω \\
\tabto{2em} ὅτι \\
\tabto{4em} ἐπὶ ἀνθρώπου ἔργον \\
\tabto{4em} ἐγείρομαι· \\
\tabto{2em} ἔτι οὖν δυσκολαίνω, \\
\tabto{4em} εἰ πορεύομαι \\
\tabto{6em} ἐπὶ τὸ ποιεῖν \\
\tabto{8em} ὧν ἕνεκεν γέγονα \\
\tabto{8em} καὶ \\
\tabto{8em} ὧν χάριν προῆγμαι \\
\tabto{10em} εἰς τὸν κόσμον; \\

\end{greek}
}

\begin{description}[noitemsep]
\item[ὅταν\dots\ ἐξεγείρῃ] vremenski veznik ὅταν uvodi zavisnu vremensku rečenicu s pogodbenim značenjem: kad god\dots
\item[ἐξεγείρῃ] §~231, složenica ἐγείρω
\item[ἔστω] §~315, kopula kao dio imenskog predikata, imenski je dio πρόχειρον
\item[ὅτι\dots\ ἐγείρομαι] veznik ὅτι uvodi zavisnu izričnu rečenicu: da\dots
\item[ἐγείρομαι] §~231; rekcija ἐπί τι LSJ ἐγείρω I.2
\item[ἔτι οὖν\dots\ τὸν κόσμον;] upitna rečenica
\item[δυσκολαίνω] §~231
\item[εἰ πορεύομαι] veznik εἰ uvodi zavisnu pogodbenu rečenicu, ovdje realnog oblika
\item[πορεύομαι] §~231, med. s rekcijom ἐπί τι LSJ πορεύω II.2
\item[τὸ ποιεῖν] §~243, supstantivirani infinitiv §~497
\item[ὧν\dots\ γέγονα] odnosna zamjenica ὧν uvodi zavisnu odnosnu rečenicu, antecedent je neizrečen: ono zbog čega\dots
\item[γέγονα] §~272
\item[ὧν\dots προῆγμαι] odnosna zamjenica ὧν uvodi zavisnu odnosnu rečenicu, antecedent je neizrečen: ono radi čega\dots
\item[προῆγμαι] §~291.b, reduplikacija kao augment §~235, složenica προάγω
\end{description}

%2

{\large
\begin{greek}
\noindent ἢ \\
\tabto{2em} ἐπὶ τοῦτο κατεσκεύασμαι, \\
\tabto{4em} ἵνα κατακείμενος \\
\tabto{6em} ἐν στρωματίοις \\
\tabto{4em} ἐμαυτὸν θάλπω;\\

\end{greek}
}

\begin{description}[noitemsep]
\item[ἢ\dots\ κατεσκεύασμαι] veznik uvodi nezavisnu upitnu rečenicu
\item[κατεσκεύασμαι] §~291.b, složenica σκευάζω
\item[ἵνα] veznik uvodi zavisnu namjernu rečenicu
\item[κατακείμενος] §~315.a.4, složenica κεῖμαι
\item[θάλπω] §~231

\end{description}

%3

{\large
\begin{greek}
\noindent  ``ἀλλὰ τοῦτο ἥδιον.''\\
πρὸς τὸ ἥδεσθαι οὖν \\
\tabto{2em} γέγονας, \\
ὅλως δὲ πρὸς πεῖσιν, \\
οὐ πρὸς ἐνέργειαν;\\

\end{greek}
}

\begin{description}[noitemsep]
\item[ἥδιον] imenski dio imenskog predikata, kopula je izostavljena, sc.\ ἥδιόν ἐστιν
\item[πρὸς\dots\ ἐνέργειαν; ] nezavisna upitna rečenica; pitanje se formira intonacijom, a prepoznaje interpunkcijom
\item[τὸ ἥδεσθαι] §~231, supstantivirani infinitiv §~497
\item[γέγονας] §~272
\item[δὲ] koordinacija rečeničnih članova česticom
\item[πρὸς πεῖσιν, οὐ πρὸς ἐνέργειαν] sc.\ γέγονας
\end{description}

%5

{\large
\begin{greek}
\noindent οὐ βλέπεις τὰ φυτάρια, \\
\tabto{2em} τὰ στρουθάρια, \\
\tabto{2em} τοὺς μύρμηκας,\\
\tabto{2em} τοὺς ἀράχνας, \\
\tabto{2em} τὰς μελίσσας\\
\tabto{4em} τὸ ἴδιον \\
\tabto{6em} ποιούσας, \\
\tabto{4em} τὸ καθ' αὑτὰς \\
\tabto{6em} συγκοσμούσας κόσμον;\\

\end{greek}
}

\begin{description}[noitemsep]
\item[οὐ\dots\ κόσμον;] nezavisna upitna rečenica; pitanje se formira intonacijom, a prepoznaje interpunkcijom
\item[βλέπεις] §~231
\item[ποιούσας] §~243
\item[τὸ καθ' αὑτὰς] u skladu sa sobom, tj.\ svojom naravi; važno je načelo antičke filozofije κατὰ φύσιν
\item[συγκοσμούσας] §~243, složenica κοσμέω

\end{description}

%6

{\large
\begin{greek}
\noindent ἔπειτα σὺ οὐ θέλεις \\
\tabto{2em} τὰ ἀνθρωπικὰ \\
\tabto{4em} ποιεῖν· \\
οὐ τρέχεις \\
\tabto{2em} ἐπὶ τὸ κατὰ τὴν σὴν φύσιν· \\
``ἀλλὰ δεῖ \\
\tabto{2em} καὶ ἀναπαύεσθαι.''\\
δεῖ· \\
φημὶ κἀγώ· \\
ἔδωκε μέντοι \\
\tabto{2em} καὶ τούτου \\
μέτρα \\
ἡ φύσις,\\
ἔδωκε μέντοι \\
\tabto{2em} καὶ τοῦ ἐσθίειν \\
\tabto{2em} καὶ πίνειν, \\
καὶ ὅμως σὺ\\
\tabto{2em} ὑπὲρ τὰ μέτρα, \\
\tabto{2em} ὑπὲρ τὰ ἀρκοῦντα \\
προχωρεῖς, \\
ἐν δὲ ταῖς πράξεσιν οὐκ ἔτι, \\
ἀλλ' ἐντὸς τοῦ δυνατοῦ.\\

\end{greek}
}

\begin{description}[noitemsep]
\item[ἔπειτα] izražava iznenađenje, prezir: onda\dots; LSJ s.~v.\ II.
\item[θέλεις] §~231, otvara mjesto dopuni u infinitivu
\item[ποιεῖν] §~243
\item[τρέχεις] §~231
\item[ἐπὶ τὸ κατὰ τὴν σὴν φύσιν] u ovoj priložnoj oznaci izostavljen je infinitiv koji se supstantivira, sc.\ ἐπὶ τὸ\dots\ ποιεῖν
\item[δεῖ] §~243, bezlično, otvara mjesto dopuni u infinitivu
\item[ἀναπαύεσθαι] §~231
\item[δεῖ·] §~243, sc.\ ἀναπαύεσθαι
\item[φημὶ] §~312.8; LSJ φημί III.
\item[ἔδωκε] §~306
\item[μέτρα] μέτρα τινός
\item[τοῦ ἐσθίειν καὶ πίνειν] supstantivirani infinitivi §~497
\item[ἐσθίειν] §~231
\item[πίνειν] §~231
\item[τὰ ἀρκοῦντα ] §~243, supstantivirani particip §~499
\item[προχωρεῖς] §~243, složenica χωρέω
\item[ἐν δὲ ταῖς πράξεσιν] čestica δέ označava suprotnost: ali\dots
\item[οὐκ ἔτι] usp.\ LSJ οὐκέτι; sc.\ προχωρεῖς
\item[ἐντὸς τοῦ δυνατοῦ] sc.\ πράττεις ili μένεις
\end{description}

%7


{\large
\begin{greek}
\noindent οὐ γὰρ φιλεῖς ἑαυτόν, \\
\tabto{2em} ἐπεί τοι \\
\tabto{4em} καὶ τὴν φύσιν ἄν σου \\
\tabto{4em} καὶ τὸ βούλημα ταύτης \\
\tabto{2em} ἐφίλεις.\\

\end{greek}
}

\begin{description}[noitemsep]
\item[φιλεῖς] §~243
\item[τοι] čestica τοι često se u upotrebi izjednačava s dativom (etičkim) osobne zamjenice σύ; u tekstu implicira prisutnost sugovornika, zbog čega se često koristi u dijalozima (kažem ti, vidiš, znaš)
\item[ἐπεί\dots\ ἄν\dots\ ἐφίλεις] veznik ἐπεί uvodi zavisnu uzročnu rečenicu; ἄν s indikativom prošlog vremena izriče irealnost, Smyth 2243: jer bi\dots, inače bi\dots
\item[ἐφίλεις] §~243

\end{description}



%kraj
