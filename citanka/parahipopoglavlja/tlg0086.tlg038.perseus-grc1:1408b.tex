%\section*{O autoru}

%TKTK


\section*{O tekstu}

U odlomku iz treće knjige Aristotelove \textit{Retorike} čitamo zapažanja o formalnim karakteristikama umjetničke proze. Zahtjev za posebnim oblikovanjem nevezanog govora prvi su, po antičkom uvjerenju, iznijeli sofisti iz V.~st.\ p.~n.~e.\ Trazimah iz Halkedona i Gorgija iz Leontina (obojica se pojavljuju i kao likovi u Platonovim djelima); Trazimah je usto definirao konvenciju koje će se držati čitava antika, da umjetnička proza mora biti periodizirana, odnosno ritmička.

Osnovnu ideju – da prozni zapis ne smije biti sastavljen u pravilnom metričkom obrascu, ali da ne smije ni biti bez dojma metra – Aristotel dopunjava dodatnim objašnjenjima i određenjima. Proza s formom metra nije uvjerljiva i čini se odviše umjetnom; metar, osim toga, privlači pažnju slušatelja, a zbog metričkih obrazaca takva proza postaje predvidljivom. S druge strane, proza bez metra previše je neodređena. Kako je broj onaj koji daje oblik stvarima, potreban je i prozi. Taj broj po Aristotelu u prozi je prepoznatljiv kao \textbf{ritam}. 

Aristotel zatim preispituje razne vrste ritmova (herojski, jamb, trohej i pean) i pokazuje zašto je pean najprimjereniji za prozu.

Osim prvog nama poznatog teorijskog razmatranja poetike proze, odlomak je zanimljiv i kao ilustracija Aristotelove metode izlaganja; on polazi od općenite tvrdnje koju potom objašnjava te po potrebi dopunjava dodatnim podacima i ilustracijama. Pritom je Aristotelova dikcija ovdje odmjerena, a sintaksa relativno jednostavna. Prevladavaju imenski predikati, karakteristični za diskurs definiranja i objašnjavanja. Kombiniraju se čestice μέν, δέ i γάρ koje uvode dopune misli i izricanja suprotnosti.


\newpage

\section*{Pročitajte naglas grčki tekst.}

Arist.\ Rhetorica 1408b 21

%Naslov prema izdanju

\medskip

{\large
\begin{greek}
\noindent τὸ δὲ σχῆμα τῆς λέξεως δεῖ μήτε ἔμμετρον εἶναι μήτε ἄρρυθμον· τὸ μὲν γὰρ ἀπίθανον (πεπλάσθαι γὰρ δοκεῖ), καὶ ἅμα καὶ ἐξίστησι· προσέχειν γὰρ ποιεῖ τῷ ὁμοίῳ, πότε πάλιν ἥξει· ὥσπερ οὖν τῶν κηρύκων προλαμβάνουσι τὰ παιδία τὸ ``τίνα αἱρεῖται ἐπίτροπον ὁ ἀπελευθερούμενος;'' ``Κλέωνα''· τὸ δὲ ἄρρυθμον ἀπέραντον, δεῖ δὲ πεπεράνθαι μέν, μὴ μέτρῳ δέ· ἀηδὲς γὰρ καὶ ἄγνωστον τὸ ἄπειρον. περαίνεται δὲ ἀριθμῷ πάντα· ὁ δὲ τοῦ σχήματος τῆς λέξεως ἀριθμὸς ῥυθμός ἐστιν, οὗ καὶ τὰ μέτρα τμήματα· διὸ ῥυθμὸν δεῖ ἔχειν τὸν λόγον, μέτρον δὲ μή· ποίημα γὰρ ἔσται. ῥυθμὸν δὲ μὴ ἀκριβῶς· τοῦτο δὲ ἔσται ἐὰν μέχρι του ᾖ. τῶν δὲ ῥυθμῶν ὁ μὲν ἡρῷος σεμνῆς ἀλλ᾽ οὐ λεκτικῆς ἁρμονίας δεόμενος, ὁ δ᾽ ἴαμβος αὐτή ἐστιν ἡ λέξις ἡ τῶν πολλῶν (διὸ μάλιστα πάντων τῶν μέτρων ἰαμβεῖα φθέγγονται λέγοντες), δεῖ δὲ σεμνότητα γενέσθαι καὶ ἐκστῆσαι. ὁ δὲ τροχαῖος κορδακικώτερος· δηλοῖ δὲ τὰ τετράμετρα· ἔστι γὰρ τροχερὸς ῥυθμὸς τὰ τετράμετρα. λείπεται δὲ παιάν, ᾧ ἐχρῶντο μὲν ἀπὸ Θρασυμάχου ἀρξάμενοι, οὐκ εἶχον δὲ λέγειν τίς ἦν. ἔστι δὲ τρίτος ὁ παιάν, καὶ ἐχόμενος τῶν εἰρημένων· τρία γὰρ πρὸς δύ᾽ ἐστίν, ἐκείνων δὲ ὁ μὲν ἓν πρὸς ἕν, ὁ δὲ δύο πρὸς ἕν, ἔχεται δὲ τῶν λόγων τούτων ὁ ἡμιόλιος· οὗτος δ᾽ ἐστὶν ὁ παιάν. οἱ μὲν οὖν ἄλλοι διά τε τὰ εἰρημένα ἀφετέοι, καὶ διότι μετρικοί· ὁ δὲ παιὰν ληπτέος· ἀπὸ μόνου γὰρ οὐκ ἔστι μέτρον τῶν ῥηθέντων ῥυθμῶν, ὥστε μάλιστα λανθάνειν.

\end{greek}
}

\newpage

\section*{Komentar}

%1

\begin{description}[noitemsep]
\item[μὲν γὰρ ] odnosi se na prvu od prethodno izrečenih tvrdnji o formama dikcije, ἔμμετρον: prvo…; \textgreek[variant=ancient]{τὸ μὲν γὰρ ἀπίθανον} prvi je dio koordinacije koji slijedi \textgreek[variant=ancient]{τὸ δὲ ἄρρυθμον ἀπέραντον}
\item[γὰρ] čestica ovdje u uzročnom eksplanatornom značenju: jer naime…
\item[ἐξίστησι] § 305, § 311; ovdje u značenju: odvraćati pažnju, LSJ s.v. A.2
\item[προλαμβάνουσι ] rekcija: τινος; augment § 238 (glag.\ osnove § 321.14); LSJ s.v. II.3.b.: očekivati od koga
\item[ὁ ἀπελευθερούμενος] supstantivirani particip § 498-499; \textgreek[variant=ancient]{ἀπελευθερούμενος} je oslobođenik (oslobođeni rob), koji ne uživa puno građansko pravo, te mu u sudskim postupcima i sličnim situacijama treba zastupnik ili zaštitnik; Kleon je primjer političara-populista, na strani malih ljudi i potlačenih, te su i djeca znala odgovor u takvom slučaju
\item[τὸ] \textbf{\textgreek[variant=ancient]{``τίνα αἱρεῖται ἐπίτροπον ὁ ἀπελευθερούμενος;''}} član τὸ supstantivira direktno pitanje koje se citira pod navodnicima
\item[τίνα… αἱρεῖται] upitna zamjenica τίνα uvodi direktno, nezavisno pitanje: koga…; rekonstrukcija: \textgreek[variant=ancient]{τὰ παιδία προλαμβάνουσι τὸ (ἐρώτημα) τῶν κηρύκων “τίνα αἱρεῖται ἐπίτροπον ὁ ἀπελευθερούμενος;” (τὰ παιδία ἀμείβονται) “Κλέωνα”}
\item[τὸ δὲ ἄρρυθμον] „a drugo…“
\item[ἀπέραντον] imenski dio imenskog predikata (glagolski dio, kopula ἐστί, je neizrečen)
\item[ἀηδὲς γὰρ καὶ ἄγνωστον] imenski dijelovi imenskog predikata (glagolski dio, kopula ἐστί, je neizrečen)
\end{description}

%2

\begin{description}[noitemsep]
\item[διὸ]  = δι᾽ ὅ; zaključno: zato…
\item[δεῖ] glagol δεῖ otvara mjesto dopuni u infinitivu; ovdje u vidu akuzativa s infinitivom: zato treba da govor ima ritam…
\item[γὰρ ] čestica eksplanatorno-uzročnog značenja ovdje podupire istinitost prethodne tvrdnje: jer inače…
\end{description}

%3

\begin{description}[noitemsep]
\item[ἐὰν… ᾖ] pogodbeni veznik ἐάν uvodi zavisnu pogodbenu rečenicu eventualnog futurskog značenja: ako…
\item[μέχρι του ᾖ] imenski predikat u kojem prijedložni izraz ima funkciju imenskog dijela
\end{description}

%4

\begin{description}[noitemsep]
\item[ὁ ἡρῷος] herojski metar, tj.\ daktil (u kvantitativnoj metrici: dugi, kratki, kratki slog, ¯ ˘ ˘)
\item[ὁ ἴαμβος] jamb, u kvantitativnoj metrici: kratki, dugi slog, ˘ ¯
\item[διὸ… φθέγγονται] umetnuta zavisna odnosna rečenica koju uvodi veznik διὸ (nastao od prijedložnog izraza odnosne zamjenice = δι᾽ ὅ: zbog čega…)
\end{description}

%5

\begin{description}[noitemsep]
\item[ὁ τροχαῖος] trohej, u kvantitativnoj metrici: dug, kratak ¯ ˘
\item[κορδακικώτερος] imenski dio imenskog predikata (glagolski dio, kopula ἐστί ostaje neizrečen)
\item[γὰρ] čestica γὰρ se ovdje veže na prethodno izrečenu misao i dopunjuje ju: naime… ili jer…
\end{description}

%6

\begin{description}[noitemsep]
\item[παιάν] pean, u kvantitativnoj metrici: različite kombinacije jednog dugog i tri kratka sloga (dugi može biti na prvom, drugom, trećem ili četvrtom mjestu)
\item[ᾧ ἐχρῶντο] odnosna zamjenica ᾧ uvodi zavisnu odnosnu rečenicu: kojim…
\item[ἐχρῶντο μὲν… εἶχον δὲ] koordinacija pomoću čestica μέν… δέ\dots: μέν daje prvo objašnjenje početne misli (λείπεται παιάν), a δέ ističe novu, dodatnu informaciju: naime… ali…
\item[εἶχον ] glagol otvara mjesto dopuni u infinitivu uz značenjski pomak uz glagol govorenja: moći
\end{description}

%7
\begin{description}[noitemsep]
\item[τρία γὰρ πρὸς δύ' ἐστίν] pean se može opisati razmjerom 3:2, dok ostale metre opisuju razmjeri 1:1 i 2:1
\item[ἐκείνων δὲ…] \textbf{ὁ μὲν… ὁ δὲ\dots}\ prvi δὲ ovdje ima značenje adverzativnoga veznika: ali…; dok je ὁ μὲν… ὁ δὲ\dots\ u koordinaciji: jedan… drugi…
\end{description}

%8

\begin{description}[noitemsep]
\item[ἀπὸ μόνου] tj.\ pean nije prilagođen metričkom sistemu, jer se njegov razmjer ne može iskazati cijelim brojem 
\item[τῶν ῥηθέντων] § 296 (glag.\ osnove § 327.7), izraz ima tekstualnu funkciju: od gorespomenutih\dots
\item[ὥστε… λανθάνειν] posljedični veznik ὥστε uvodi zavisnu posljedičnu rečenicu: tako da… § 473
\end{description}



%kraj
