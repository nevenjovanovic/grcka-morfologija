%\section*{O autoru}

%TKTK


\section*{O tekstu}

Arijan iz Nikomedije \textgreek[variant=ancient]{(Φλάουιος Ἀρριανός,} Flavius Arrianus; Nikomedija u Bitiniji, oko 95. – Atena, oko 175.) priredio je u \textit{Priručniku} \textgreek[variant=ancient]{(Ἐγχειρίδιον)} sažetak Epiktetovih učenja kao uvijek spreman vodič za filozofske susrete sa svakodnevicom. \textit{Priručnik} nije sustavna razrada etike, već zbirka misli i uputa korisnih u konkretnim situacijama. Za razliku od dijaloškog i argumentativnog oblika \textit{Dijatriba} \textgreek[variant=ancient]{(Δıατριβαί),} \textit{Priručnik} je apodiktičan, slobodan od metafizičkih i teoloških razmatranja. Pisan je fragmentarnim stilom, u kratkim odlomcima, uobličen u niz formula (vrlo su česti imperativi) i pamtljivih slika koje stoiku služe kao poticaji za dnevne meditacije.

Ovdje odabrani odlomak ističe važnost prethodnog promišljanja; prije svakog pothvata, treba dobro procijeniti djelatnost u koju se upuštamo, ali i točno spoznati \textgreek[variant=ancient]{(καταμανθάνειν)} kakve su naše snage. Na životnom putu posebno je važna dosljednost \textgreek[variant=ancient]{(ἕνα ἄνθρωπον εἶναι);} tek posvećivanje onome izvanjskome, odnosno unutarnjem, određuje hoće li netko biti \textgreek[variant=ancient]{φιλόσοφος ili ἰδιώτης.}

\newpage

\section*{Pročitajte naglas grčki tekst.}

Epict. Enchiridion 29

%Naslov prema izdanju

\medskip


{\large

\begin{greek}

\noindent ἄνθρωπε, πρῶτον ἐπίσκεψαι, ὁποῖόν ἐστι τὸ πρᾶγμα· εἶτα καὶ τὴν σεαυτοῦ φύσιν κατάμαθε, εἰ δύνασαι βαστάσαι. πένταθλος εἶναι βούλει ἢ παλαιστής; ἴδε σεαυτοῦ τοὺς βραχίονας, τοὺς μηρούς, τὴν ὀσφὺν κατάμαθε. ἄλλος γὰρ πρὸς ἄλλο πέφυκε. 

δοκεῖς, ὅτι ταῦτα ποιῶν ὡσαύτως δύνασαι ἐσθίειν, ὡσαύτως πίνειν, ὁμοίως ὀρέγεσθαι, ὁμοίως δυσαρεστεῖν; ἀγρυπνῆσαι δεῖ, πονῆσαι, ἀπὸ τῶν οἰκείων ἀπελθεῖν, ὑπὸ παιδαρίου καταφρονηθῆναι, ὑπὸ τῶν ἀπαντώντων καταγελασθῆναι, ἐν παντὶ ἧττον ἔχειν, ἐν τιμῇ, ἐν ἀρχῇ, ἐν δίκῃ, ἐν πραγματίῳ παντί. ταῦτα ἐπίσκεψαι, εἰ θέλεις ἀντικαταλλάξασθαι τούτων ἀπάθειαν, ἐλευθερίαν, ἀταραξίαν· εἰ δὲ μή, μὴ προσάγαγε. μὴ ὡς τὰ παιδία νῦν φιλόσοφος, ὕστερον δὲ τελώνης, εἶτα ῥήτωρ, εἶτα ἐπίτροπος Καίσαρος. ταῦτα οὐ συμφωνεῖ. ἕνα σε δεῖ ἄνθρωπον ἢ ἀγαθὸν ἢ κακὸν εἶναι· ἢ τὸ ἡγεμονικόν σε δεῖ ἐξεργάζεσθαι τὸ σαυτοῦ ἢ τὸ ἐκτὸς ἢ περὶ τὰ ἔσω φιλοτεχνεῖν ἢ περὶ τὰ ἔξω· τοῦτ' ἔστιν ἢ φιλοσόφου τάξιν ἐπέχειν ἢ ἰδιώτου.

\end{greek}

}


\section*{Komentar}

%1


%2

%3


\begin{description}[noitemsep]
\item[ἄλλος\dots\ πρὸς ἄλλο] hrv.\ ``svatko za nešto drugo''
\end{description}

%4

\begin{description}[noitemsep]
\item[δοκεῖς, ὅτι\dots] izrična zavisna rečenica iza ὅτι, §~467
\end{description}

%5

\begin{description}[noitemsep]
\item[ἧττον ἔχειν] LSJ ἔχω B.II.2
\end{description}

%6

\begin{description}[noitemsep]
\item[ἀντικαταλλάξασθαι] §~267, §~269, ἀντικαταλλάσσω, atički ἀντικαταλλάττω τι τινός, složenica glagola αλλάσσω (αλλάττω)
\item[εἰ δὲ μή] sc. \textgreek[variant=ancient]{εἰ δὲ οὐκ θέλεις ἀντικαταλλάξασθαι}
\end{description}

%7

\begin{description}[noitemsep]
\item[μὴ] sc.\ ἴσθι
\item[ὡς] veznik otvara mjesto poredbenoj zavisnoj rečenici, Smyth 2475
\item[νῦν\dots\ ὕστερον δὲ] koordinacija rečeničnih članova pomoću čestice δέ
\end{description}

%8

%9


\begin{description}[noitemsep]
\item[τὸ ἡγεμονικόν\dots\ τὸ σαυτοῦ] supstantivirane zamjenice, §~373
\item[τὸ ἡγεμονικόν] stoički \textit{terminus technicus} za upravljački dio duše, njezin ``zapovjedni centar''
\item[τοῦτ' ἔστιν] §~315; LSJ εἰμί B; otvara mjesto infinitivu

\end{description}


%kraj
