%\section*{O autoru}

%TKTK


\section*{O tekstu}

\textgreek[variant=ancient]{Ἐνάλιοι διάλογοι,} \textit{Dijalozi morskih bogova}, naziv je zbirke petnaest kratkih razgovora u kojima se sirijski retoričar i satiričar Lukijan šali na račun kanonske grčke mitologije.

U odlomku koji čitamo kiklop Polifem pripovijeda ocu Posejdonu kako ga je oslijepio Odisej. Kiklop je na početku bio prilično samouvjeren: za njega je donijeti čitavo stablo iz planine tek sitnica, isto kao i obračunati se s nekolicinom Odisejevih drugova. No, ti bezveznjaci nadmudrili su ga pomoću opojnog sredstva i kolca.


%\newpage

\section*{Pročitajte naglas grčki tekst.}

Luc. Dialogi marini 2.2

%Naslov prema izdanju
%NČ

\medskip


{\large

\begin{greek}

\noindent (Κύκλωψ.) κατέλαβον ἐν τῷ ἄντρῳ ἀπὸ τῆς νομῆς ἀναστρέψας πολλούς τινας, ἐπιβουλεύοντας δῆλον ὅτι τοῖς ποιμνίοις· ἐπεὶ γὰρ ἐπέθηκα τῇ θύρᾳ τὸ πῶμα — πέτρα δέ ἐστι παμμεγέθης — καὶ τὸ πῦρ ἀνέκαυσα ἐναυσάμενος ὃ ἔφερον δένδρον ἀπὸ τοῦ ὄρους, ἐφάνησαν ἀποκρύπτειν αὑτοὺς πειρώμενοι· ἐγὼ δὲ συλλαβών τινας αὐτῶν, ὥσπερ εἰκὸς ἦν, κατέφαγον λῃστάς γε ὄντας. ἐνταῦθα ὁ πανουργότατος ἐκεῖνος, εἴτε Οὖτις εἴτε Ὀδυσσεὺς ἦν, δίδωσί μοι πιεῖν φάρμακόν τι ἐγχέας, ἡδὺ μὲν καὶ εὔοσμον, ἐπιβουλότατον δὲ καὶ ταραχωδέστατον· ἅπαντα γὰρ εὐθὺς ἐδόκει μοι περιφέρεσθαι πιόντι καὶ τὸ σπήλαιον αὐτὸ ἀνεστρέφετο καὶ οὐκέτι ὅλως ἐν ἐμαυτοῦ ἤμην, τέλος δὲ ἐς ὕπνον κατεσπάσθην. ὁ δὲ ἀποξύνας τὸν μοχλὸν καὶ πυρώσας γε προσέτι ἐτύφλωσέ με καθεύδοντα, καὶ ἀπʼ ἐκείνου τυφλός εἰμί σοι, ὦ Πόσειδον.


\end{greek}

}


\section*{Komentar}

%1

\begin{description}[noitemsep]
\item[δῆλον ὅτι] od izraza δῆλόν ἐστιν ὅτι, „jasno je da“, koji je otvarao mjesto izričnoj rečenici; izostavljanjem kopule i čestim korištenjem s vremenom se izgubila potreba za glagolom i dopunom – ὅτι gubi snagu veznika, a značenje postaje adverbno: očito\dots
\item[ὃ ἔφερον] odnosna zamjenica ὃ uvodi umetnutu odnosnu zavisnu rečenicu
\item[δὲ] čestica daje surečenici adverzativno značenje: a…
\item[ὥσπερ εἰκὸς ἦν] umetnuta poredbena rečenica: kao što… (Smyth 2462)
\item[γε] čestica naglašava, gotovo pretvara u uzvik razlog Kiklopova postupka: ``pa bili su...!''
\end{description}

%2

\begin{description}[noitemsep]
\item[ἐν ἐμαυτοῦ] prijedlog ἐν ovdje naoko stoji uza genitiv umjesto dativa; no, riječ je o vrsti elipse u kojoj se izostavlja riječ na koju se odnosi ἐμαυτοῦ, u dativu prema rekciji prijedloga ἐν: ἐν ἐμαυτοῦ οἰκίᾳ (εἶναι) pri sebi (biti): LSJ ἐν A. I. 2. 
\item[ἤμην] §~315; kasniji oblik, čest kod Lukijana = ἦν
\item[καὶ… καὶ… δὲ\dots] rečenice prvo ustrojene usporedno (καὶ\dots\ καὶ\dots), nabrajanjem, završnu misao uvodi adverzativno δὲ: i\dots\ i\dots\ a\dots

\end{description}


%3

%kraj
