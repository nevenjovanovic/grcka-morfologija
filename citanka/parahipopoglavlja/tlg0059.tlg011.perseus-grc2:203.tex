%\section*{O autoru}

%TKTK


\section*{O tekstu}

Odlomak je dio Sokratova govora o Erotu iz Platonova dijaloga \textgreek[variant=ancient]{Συμπόσιον} \textit{(Gozba),} djela nastalog između 385.\ i 370.\ p.~n.~e. Dijalog, koji se odigrava na gozbi kod atenskoga tragičkog pjesnika Agatona, nakon njegove pobjede na Dionizijama 416, razmatra pojam ljubavi \textgreek[variant=ancient]{(ἔρως).} Odavde potječe Sokratova pohvala Erotu koja je bila jedan od izvora renesansnog pojma \textit{platonske ljubavi.}

U Platonovu prikazu, Sokrat izlaže posljednji, poslije petorice sudionika gozbe. No, počinje tuđim riječima; prepričava što je čuo i naučio od proročice Diotime iz Mantineje. Dotad je Sokrat o erosu mislio poput svojih drugova s gozbe. Tako je Diotima Sokratu ono što je on svojim učenicima.

Pohvala Erotu u ovom odlomku prikazuje njegovo mitsko porijeklo koje je odredilo i njegov karakter.


%\newpage

\section*{Pročitajte naglas grčki tekst.}

Plat.\ Symposium 203b-203e

%Naslov prema izdanju

\medskip


{\large

\begin{greek}

\noindent ὅτε γὰρ ἐγένετο ἡ Ἀφροδίτη, ἡστιῶντο οἱ θεοὶ οἵ τε ἄλλοι καὶ ὁ τῆς Μήτιδος ὑὸς Πόρος. ἐπειδὴ δὲ ἐδείπνησαν, προσαιτήσουσα οἷον δὴ εὐωχίας οὔσης ἀφίκετο ἡ Πενία, καὶ ἦν περὶ τὰς θύρας. ὁ οὖν Πόρος μεθυσθεὶς τοῦ νέκταρος — οἶνος γὰρ οὔπω ἦν — εἰς τὸν τοῦ Διὸς κῆπον εἰσελθὼν βεβαρημένος ηὗδεν. ἡ οὖν Πενία ἐπιβουλεύουσα διὰ τὴν αὑτῆς ἀπορίαν παιδίον ποιήσασθαι ἐκ τοῦ Πόρου, κατακλίνεταί τε παρʼ αὐτῷ καὶ ἐκύησε τὸν ἔρωτα. διὸ δὴ καὶ τῆς Ἀφροδίτης ἀκόλουθος καὶ θεράπων γέγονεν ὁ Ἔρως, γεννηθεὶς ἐν τοῖς ἐκείνης γενεθλίοις, καὶ ἅμα φύσει ἐραστὴς ὢν περὶ τὸ καλὸν καὶ τῆς Ἀφροδίτης καλῆς οὔσης. ἅτε οὖν Πόρου καὶ Πενίας ὑὸς ὢν ὁ Ἔρως ἐν τοιαύτῃ τύχῃ καθέστηκεν. πρῶτον μὲν πένης ἀεί ἐστι, καὶ πολλοῦ δεῖ ἁπαλός τε καὶ καλός, οἷον οἱ πολλοὶ οἴονται, ἀλλὰ σκληρὸς καὶ αὐχμηρὸς καὶ ἀνυπόδητος καὶ ἄοικος, χαμαιπετὴς ἀεὶ ὢν καὶ ἄστρωτος, ἐπὶ θύραις καὶ ἐν ὁδοῖς ὑπαίθριος κοιμώμενος, τὴν τῆς μητρὸς φύσιν ἔχων, ἀεὶ ἐνδείᾳ σύνοικος. κατὰ δὲ αὖ τὸν πατέρα ἐπίβουλός ἐστι τοῖς καλοῖς καὶ τοῖς ἀγαθοῖς, ἀνδρεῖος ὢν καὶ ἴτης καὶ σύντονος, θηρευτὴς δεινός, ἀεί τινας πλέκων μηχανάς, καὶ φρονήσεως ἐπιθυμητὴς καὶ πόριμος, φιλοσοφῶν διὰ παντὸς τοῦ βίου, δεινὸς γόης καὶ φαρμακεὺς καὶ σοφιστής· καὶ οὔτε ὡς ἀθάνατος πέφυκεν οὔτε ὡς θνητός, ἀλλὰ τοτὲ μὲν τῆς αὐτῆς ἡμέρας θάλλει τε καὶ ζῇ, ὅταν εὐπορήσῃ, τοτὲ δὲ ἀποθνῄσκει, πάλιν δὲ ἀναβιώσκεται διὰ τὴν τοῦ πατρὸς φύσιν, τὸ δὲ ποριζόμενον ἀεὶ ὑπεκρεῖ, ὥστε οὔτε ἀπορεῖ Ἔρως ποτὲ οὔτε πλουτεῖ, σοφίας τε αὖ καὶ ἀμαθίας ἐν μέσῳ ἐστίν.

\end{greek}

}


\section*{Komentar}

%1

\begin{description}[noitemsep]
\item[γὰρ] čestica ovdje u eksplanatornoj ili emfatičkoj funkciji: naime, baš
\item[ὅτε γὰρ ἐγένετο] vremenski veznik ὅτε uvodi vremensku rečenicu u inverznom položaju: kada…
\item[Πόρος] sin Metide, u \textit{Gozbi} personifikacija snalažljivosti

\end{description}

%2

\begin{description}[noitemsep]
\item[δὲ] čestica ovdje blago adverzativnog značenja, u naraciji nastavlja razvijanje misli: a…
\item[ἐπειδὴ δὲ ἐδείπνησαν] vremenski veznik ἐπειδή uvodi vremensku rečenicu u inverznom položaju: kada
\end{description}
%3

\begin{description}[noitemsep]
\item[γὰρ] čestica ovdje u eksplanatornoj ili emfatičkoj funkciji: naime, baš
\end{description}
%4

%5

\begin{description}[noitemsep]
\item[διὸ] (= δι᾽ ὅ) zato, zbog toga
\item[δὴ] čestica naglašava i ističe neki dio iskaza vrlo precizno: upravo, stvarno, u stvari, točno
\end{description}
%6

\begin{description}[noitemsep]
\item[ἅτε… ὢν] § 503.2.b, ἅτε s participom ovdje izriče uzrok kako ga doživljava govornik
\end{description}
%7

\begin{description}[noitemsep]
\item[πολλοῦ δεῖ ] fraza u rečenici ima adverbnu funkciju: daleko od…
\item[οἷον… οἴονται] poredbena rečenica koju uvodi odnosni korelativ οἷον u značenju „kao što…”
\end{description}

%8

\begin{description}[noitemsep]
\item[τοτὲ μὲν… τοτὲ δὲ\dots] koordinirana suprotnost: jednom… zatim\dots
\end{description}

%kraj
