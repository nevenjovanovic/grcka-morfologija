%\section*{O autoru}

%TKTK


\section*{O tekstu}

Osamnaesta godina Peloponeskog rata (414./413.\ p.~n.~e) ujedno je bila i druga godina takozvane Sicilske ekspedicije, u kojoj su atenske snage opsjedale daleku Sirakuzu. Sirakužanima su vojno pomagali Spartanci; početkom godine, u trenutku kad su Sirakužani već razmišljali o predaji, Spartanci su im poslali novog zapovjednika, Gilipa, s korintskim i spartanskim pojačanjima. Počela je grozničava utrka u podizanju zidova na strateški važnoj visoravni Epipoli \textgreek[variant=ancient]{(Ἐπιπολαί),} neposredno iznad grada. Atenjani su pokušavali izgraditi zid i tako blokirati Sirakuzu, a Sirakužani su gradili protuzid sa svoje strane, kako bi presjekli liniju Atenjana i tako spriječili potpuno opkoljavanje. Između dva zida sukobile su se vojske. Prvi su put Sirakužani bili poraženi. No, u sljedećem će sukobu na istom mjestu izgubiti Atenjani, a Sirakužani će uspjeti blokirati daljnje napredovanje opsadnog zida. To će biti prekretnica u Sicilskoj ekspediciji, fatalni gubitak atenske prednosti i inicijative.

\newpage

\section*{Pročitajte naglas grčki tekst.}

Thuc.\ Historiae 7.5.1–7.5.4

%Naslov prema izdanju

\medskip


{\large

\begin{greek}

\noindent ὁ δὲ Γύλιππος ἅμα μὲν ἐτείχιζε τὸ διὰ τῶν Ἐπιπολῶν τεῖχος, τοῖς λίθοις χρώμενος οὓς οἱ Ἀθηναῖοι προπαρεβάλοντο σφίσιν, ἅμα δὲ παρέτασσεν ἐξάγων αἰεὶ πρὸ τοῦ τειχίσματος τοὺς Συρακοσίους καὶ τοὺς ξυμμάχους· καὶ οἱ Ἀθηναῖοι ἀντιπαρετάσσοντο.

ἐπειδὴ δὲ ἔδοξε τῷ Γυλίππῳ καιρὸς εἶναι, ἦρχε τῆς ἐφόδου· καὶ ἐν χερσὶ γενόμενοι ἐμάχοντο μεταξὺ τῶν τειχισμάτων, ᾗ τῆς ἵππου τῶν Συρακοσίων οὐδεμία χρῆσις ἦν.

καὶ νικηθέντων τῶν Συρακοσίων καὶ τῶν ξυμμάχων καὶ νεκροὺς ὑποσπόνδους ἀνελομένων καὶ τῶν Ἀθηναίων τροπαῖον στησάντων, ὁ Γύλιππος ξυγκαλέσας τὸ στράτευμα οὐκ ἔφη τὸ ἁμάρτημα ἐκείνων, ἀλλ ἑαυτοῦ γενέσθαι· τῆς γὰρ ἵππου καὶ τῶν ἀκοντιστῶν τὴν ὠφελίαν τῇ τάξει ἐντὸς λίαν τῶν τειχῶν ποιήσας ἀφελέσθαι· νῦν οὖν αὖθις ἐπάξειν.

καὶ διανοεῖσθαι οὕτως ἐκέλευεν αὐτοὺς ὡς τῇ μὲν παρασκευῇ οὐκ ἔλασσον ἕξοντας, τῇ δὲ γνώμῃ οὐκ ἀνεκτὸν ἐσόμενον εἰ μὴ ἀξιώσουσι Πελοποννήσιοί τε ὄντες καὶ Δωριῆς Ἰώνων καὶ νησιωτῶν καὶ ξυγκλύδων ἀνθρώπων κρατήσαντες ἐξελάσασθαι ἐκ τῆς χώρας.

\end{greek}

}


\section*{Komentar}

%1

%2

\begin{description}[noitemsep]
\item[ἐν χερσὶ γενόμενοι] fraza koja znači: stupiti u blisku borbu, LSJ χείρ II.6.f
\item[ᾗ] LSJ ὅς, ἥ, ὅ 0-2 A II (izražava mjesto)
\item[τῆς ἵππου] LSJ ἵππος II (kao kolektivna imenica)
\end{description}

%3

\begin{description}[noitemsep]
\item[ξυγκαλέσας] § 267; ξυγ- je atička varijanta za συγκαλέω
\item[οὐκ ἔφη] § 312.8; fraza οὔ φημι: poreći, zanijekati, odbiti (LSJ φημί III); otvara mjesto akuzativu s infinitivom i infinitivu
\item[ἐντὸς\dots\ ποιήσας] § 269; fraza, LSJ ἐντός II: uvući, uvesti%imenski pred.
\end{description}

%4

\begin{description}[noitemsep]
\item[οὕτως\dots\ ὡς\dots] \textbf{οὐκ ἔλασσον ἕξοντας} § 327.13; § 258; LSJ ἔχω B.II.2; poredbeno: tako\dots\ kao da\dots%izrično
\item[εἰ μὴ ἀξιώσουσι] § 259; otvara mjesto dopuni u infinitivu; pogodbena realna protaza: ako ne\dots
\end{description}


%kraj
