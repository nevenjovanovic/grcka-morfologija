%Unio korekture NČ, 2019-08-13
\section*{O autoru}

Neoplatonizam je posljednji veliki sustav grčke filozofije, pokušaj sinteze čitave grčke filozofske tradicije i novijih misaonih tendencija. Glavni je predstavnik ovog sustava Plotin \textgreek[variant=ancient]{(Πλωτῖνος,} oko 205. – oko 270.). Rođen u egipatskom Likopolu, studirao je filozofiju u Aleksandriji, u školi Amonija Sake. Zanimao se za perzijsku i indijsku filozofiju, i zato se pridružio neuspješnom pohodu cara Gordijana III.\ na Perziju, 243. Osnovao je školu u Rimu pod pokroviteljstvom cara Galijena (vladao 253.–268.). Planirao je čak i osnivanje Platonopolisa, zajednice po Platonovim načelima, u Kampaniji; to se nije ostvarilo.

\section*{O tekstu}

Plotinove je spise učenik Porfirije \textgreek[variant=ancient]{(Πορφύριος)} prikupio u Eneade \textgreek[variant=ancient]{(ἐννέα} ``devet''), šest svezaka po devet spisa, organizirajući ih po temama: etičkim, prirodno-filozofskim, o duši \textgreek[variant=ancient]{(ψυχή,} četvrta eneada), o duhu \textgreek[variant=ancient]{(νοῦς,} peta) i o Jednome \textgreek[variant=ancient]{(ἕν,} šesta). Duša, duh i Jedno pripadaju gornjem svijetu, koji se može spoznati samo mišljenjem – pri čemu Jedno, posve savršena svjetska duša, nije spoznatljiva čak ni na taj način, već samo kroz svojevrsno kontemplativno izlaženje iz sebe sama; to je ἔκστασις, koja vodi do ἕνωσις (\textit{unio mystica}, ujedinjenje) s najvišom razinom postojanja.

U ovdje odabranim odlomcima Plotin opisuje kako ljubav prema Jednom \textgreek[variant=ancient]{(μόνον),} onome koje nema oblik, utječe na dušu; pod tim utjecajem i sama se duša oslobađa oblika. Saznajemo i kako se duša osjeća \textgreek[variant=ancient]{ἐκεῖ} (``ondje''), u svijetu iznad samog neba.

\section*{Pročitajte naglas grčki tekst.}

Plot.\ Enneades 6.7.34

%Naslov prema izdanju

\medskip


{\large
{ \noindent I

\begin{greek}

\noindent  Ἡ ψυχή, ὅταν αὐτοῦ ἔρωτα σύντονον λάβῃ, ἀποτίθεται πᾶσαν ἣν ἔχει μορφήν, καὶ ἥτις ἂν καὶ νοητοῦ ᾖ ἐν αὐτῇ. Οὐ γάρ ἐστιν ἔχοντά τι ἄλλο καὶ ἐνεργοῦντα περὶ αὐτὸ οὔτε ἰδεῖν οὔτε ἐναρμοσθῆναι. Ἀλλὰ δεῖ μήτε κακὸν μήτ' αὖ ἀγαθὸν μηδὲν ἄλλο πρόχειρον ἔχειν, ἵνα δέξηται μόνη μόνον. Ὅταν δὲ τούτου εὐτυχήσῃ ἡ ψυχὴ καὶ ἥκῃ πρὸς αὐτήν, μᾶλλον δὲ παρὸν φανῇ, ὅταν ἐκείνη ἐκνεύσῃ τῶν παρόντων καὶ παρασκευάσασα αὑτὴν ὡς ὅτι μάλιστα καλὴν καὶ εἰς ὁμοιότητα ἐλθοῦσα — ἡ δὲ παρασκευὴ καὶ ἡ κόσμησις δήλη που τοῖς παρασκευαζομένοις — ἰδοῦσα δὲ ἐν αὐτῇ ἐξαίφνης φανέντα — μεταξὺ γὰρ οὐδὲν οὐδ' ἔτι δύο, ἀλλ' ἓν ἄμφω· οὐ γὰρ ἂν διακρίναις ἔτι, ἕως πάρεστι· μίμησις δὲ τούτου καὶ οἱ ἐνταῦθα ἐρασταὶ καὶ ἐρώμενοι συγκρῖναι θέλοντες — καὶ οὔτε σώματος ἔτι αἰσθάνεται, ὅτι ἐστὶν ἐν αὐτῷ, οὔτε ἑαυτὴν ἄλλο τι λέγει, οὐκ ἄνθρωπον, οὐ ζῷον, οὐκ ὄν, οὐδὲ πᾶν — ἀνώμαλος γὰρ ἡ τούτων πως θέα — καὶ οὐδὲ σχολὴν ἄγει πρὸς αὐτὰ οὔτε θέλει, ἀλλὰ καὶ αὐτὸ ζητήσασα ἐκείνῳ παρόντι ἀπαντᾷ κἀκεῖνο ἀντ' αὐτῆς βλέπει· τίς δὲ οὖσα βλέπει, οὐδὲ τοῦτο σχολάζει ὁρᾶν.

\end{greek}

\noindent II

\begin{greek}

\noindent  Ἔνθα δὴ οὐδὲν πάντων ἀντὶ τούτου ἀλλάξαιτο, οὐδ' εἴ τις αὐτῇ πάντα τὸν οὐρανὸν ἐπιτρέποι, ὡς οὐκ ὄντος ἄλλου ἔτι ἀμείνονος οὐδὲ μᾶλλον ἀγαθοῦ· οὔτε γὰρ ἀνωτέρω τρέχει τά τε ἄλλα πάντα κατιούσης, κἂν ᾖ ἄνω.  Ὥστε τότε ἔχει καὶ τὸ κρίνειν καλῶς καὶ γιγνώσκειν, ὅτι τοῦτό ἐστιν οὗ ἐφίετο, καὶ τίθεσθαι, ὅτι μηδέν ἐστι κρεῖττον αὐτοῦ. Οὐ γάρ ἐστιν ἀπάτη ἐκεῖ· ἢ ποῦ ἂν τοῦ ἀληθοῦς ἀληθέστερον τύχοι;  Ὃ οὖν λέγει, ἐκεῖνό ἐστι, καὶ ὕστερον λέγει, καὶ σιωπῶσα δὲ λέγει καὶ εὐπαθοῦσα οὐ ψεύδεται, ὅτι εὐπαθεῖ· οὐδὲ γαργαλιζομένου λέγει τοῦ σώματος, ἀλλὰ τοῦτο γενομένη, ὃ πάλαι, ὅτε εὐτύχει. Ἀλλὰ καὶ τὰ ἄλλα πάντα, οἷς πρὶν ἥδετο, ἀρχαῖς ἢ δυνάμεσιν ἢ πλούτοις ἢ κάλλεσιν ἢ ἐπιστήμαις, ταῦτα ὑπεριδοῦσα λέγει οὐκ ἂν εἰποῦσα μὴ κρείττοσι συντυχοῦσα τούτων· οὐδὲ φοβεῖται, μή τι πάθῃ, μετ' ἐκείνου οὖσα οὐδ' ὅλως ἰδοῦσα· εἰ δὲ καὶ τὰ ἄλλα τὰ περὶ αὐτὴν φθείροιτο, εὖ μάλα καὶ βούλεται, ἵνα πρὸς τούτῳ ᾖ μόνον· εἰς τόσον ἥκει εὐπαθείας.

\end{greek}

}
}


\section*{Analiza i komentar}

%1

{\large
\begin{greek}
\noindent Ἡ ψυχή, \\
\tabto{2em} ὅταν \\
\tabto{4em} αὐτοῦ ἔρωτα σύντονον \\
\tabto{4em} λάβῃ, \\
ἀποτίθεται \\
\tabto{2em} πᾶσαν \\
\tabto{4em} ἣν ἔχει \\
\tabto{2em} μορφήν, \\
\tabto{2em} καὶ ἥτις ἂν καὶ νοητοῦ ᾖ \\
\tabto{4em} ἐν αὐτῇ.\\

\end{greek}
}

\begin{description}[noitemsep]
\item[Ἡ ψυχή] §~90
\item[ὅταν\dots\ λάβῃ\dots\ ἀποτίθεται] vremenska rečenica ima značenje pogodbene eventualne protaze, ὅταν kad god, §~488.2
\item[αὐτοῦ] §~207; zamjenica se odnosi na ranije, izvan našeg odlomka spomenuto \textgreek[variant=ancient]{τὸ τοὺς δεινοὺς πόθους παρέχον} ono što izaziva takvu silnu čežnju
\item[ἔρωτα] §~123
\item[σύντονον] §~106
\item[λάβῃ] λαμβάνω primiti, osjetiti; 3. l. sg. konj. aor. akt.
\item[ἀποτίθεται] ἀποτίθημι med. odložiti; 3. l. sg. ind. prez. medpas.
\item[πᾶσαν] §~193
\item[ἣν] §~215
\item[ἔχει] ἔχω imati; 3. l. sg. ind. prez. akt.
\item[μορφήν] §~90
\item[ἥτις] §~217
\item[ἂν\dots\ ᾖ] hipotetička relativna rečenica eventualnog oblika, §~485.2; εἰμί biti; 3. l. sg. konj. prez. akt.
\item[καὶ νοητοῦ] καί \textit{ovdje} čak; §~103
\item[ἐν αὐτῇ] §~426; §~207

\end{description}

%2


{\large
\begin{greek}
\noindent Οὐ γάρ ἐστιν \\
\tabto{2em} \underline{ἔχοντά} τι ἄλλο \\
\tabto{2em} καὶ \underline{ἐνεργοῦντα} \\
\tabto{2em} \tabto{2em} περὶ αὐτὸ \\
\tabto{2em} οὔτε \underline{ἰδεῖν}\\
\tabto{2em} οὔτε \underline{ἐναρμοσθῆναι}.\\

\end{greek}
}

\begin{description}[noitemsep]
\item[γάρ ἐστιν] §~40; čestica γάρ najavljuje iznošenje dokaza ili uzroka prethodne tvrdnje: naime\dots §~517; ἔστι (bezlično, s infinitivom ili akuzativom s infinitivom kao objektom) biti moguće; 3. l. sg. ind. prez. akt.
\item[ἔχοντά τι] §~40; ἔχω imati; a. sg. m. r. ptc. prez. akt.; §~217; subjekt infinitiva razlikuje se od subjekta glavne rečenice, pa stoji u akuzativu §~491.1; ``netko tko ima''
\item[ἄλλο] §~212
\item[ἐνεργοῦντα] ἐνεργέω djelovati, baviti se, rekcija περί τι nečim; a. sg. m. r. ptc. prez. akt.
\item[περὶ αὐτὸ] §~433; §~207; zamjenica se referira na prethodno τι ἄλλο
\item[ἰδεῖν] ὁράω gledati, vidjeti; inf. aor. akt.
\item[ἐναρμοσθῆναι] ἐναρμόζω uklopiti, prilagoditi; inf. aor. pas.

\end{description}

%3


{\large
\begin{greek}
\noindent ᾿Αλλὰ δεῖ \\
\tabto{2em} μήτε κακὸν \\
\tabto{2em} μήτ' αὖ ἀγαθὸν μηδὲν ἄλλο \\
\tabto{2em} πρόχειρον ἔχειν, \\
\tabto{4em} ἵνα δέξηται \\
\tabto{6em} μόνη \\
\tabto{6em} μόνον.\\

\end{greek}
}

\begin{description}[noitemsep]
\item[δεῖ] δεῖ \textit{bezlično} treba; 3. l. sg. ind. prez. akt.
\item[μήτε\dots\ μήτ'\dots] koordinacija pomoću (niječnih) čestica: niti\dots\ niti\dots
\item[κακὸν] §~103
\item[μήτ' αὖ] §~68
\item[ἀγαθὸν] §~103
\item[μηδὲν] §~224.2
\item[ἄλλο] §~212
\item[πρόχειρον] §~106
\item[ἔχειν] ἔχω imati; inf. prez. akt.
\item[ἵνα δέξηται] zavisna namjerna rečenica, §~470; δέχομαι primiti; 3. l. sg. konj. aor. med.
\item[μόνη] §~103
\item[μόνον] §~103; supstantivirano, Plotinov filozofski termin: Jedno

\end{description}

%4


{\large
\begin{greek}
\noindent ῞Οταν δὲ \\
\tabto{2em} τούτου \\
εὐτυχήσῃ \\
ἡ ψυχὴ \\
καὶ ἥκῃ \\
\tabto{2em} πρὸς αὐτήν, \\
μᾶλλον δὲ \\
παρὸν \\
φανῇ, \\
ὅταν \\
\tabto{2em} ἐκείνη \\
\tabto{2em} ἐκνεύσῃ \\
\tabto{4em} τῶν παρόντων \\
\tabto{2em} καὶ παρασκευάσασα αὑτὴν \\
\tabto{4em} ὡς ὅτι μάλιστα καλὴν \\
\tabto{2em} καὶ \\
\tabto{4em} εἰς ὁμοιότητα ἐλθοῦσα —\\
\tabto{4em} ἡ δὲ παρασκευὴ \\
\tabto{4em} καὶ ἡ κόσμησις \\
\tabto{4em} δήλη που \\
\tabto{6em} τοῖς παρασκευαζομένοις —\\
ἰδοῦσα δὲ \\
\tabto{2em} ἐν αὐτῇ \\
\tabto{2em} ἐξαίφνης φανέντα —\\
μεταξὺ γὰρ \\
\tabto{2em} οὐδὲν οὐδ' ἔτι δύο, \\
\tabto{2em} ἀλλ' ἓν ἄμφω· \\
οὐ γὰρ ἂν \\
\tabto{2em} διακρίναις ἔτι, \\
\tabto{2em} \tabto{2em} ἕως πάρεστι· \\
μίμησις δὲ \\
\tabto{2em} τούτου \\
καὶ οἱ ἐνταῦθα ἐρασταὶ \\
καὶ ἐρώμενοι \\
\tabto{2em} συγκρῖναι θέλοντες —\\
καὶ οὔτε \\
\tabto{2em} σώματος ἔτι \\
\tabto{2em} αἰσθάνεται, \\
\tabto{4em} \tabto{2em} ὅτι ἐστὶν ἐν αὐτῷ,  \\
οὔτε \\
\tabto{2em} ἑαυτὴν \\
\tabto{2em} ἄλλο τι λέγει, \\
\tabto{4em} οὐκ ἄνθρωπον, \\
\tabto{4em} οὐ ζῷον, \\
\tabto{4em} οὐκ ὄν, \\
\tabto{4em} οὐδὲ πᾶν —\\
ἀνώμαλος γὰρ \\
ἡ τούτων πως θέα —\\
καὶ οὐδὲ \\
\tabto{2em} σχολὴν ἄγει \\
\tabto{4em} \tabto{2em} πρὸς αὐτὰ \\
\tabto{2em} οὔτε θέλει, \\
\tabto{4em} ἀλλὰ καὶ \\
\tabto{4em} αὐτὸ \\
\tabto{4em} ζητήσασα \\
\tabto{6em} ἐκείνῳ παρόντι \\
\tabto{4em} ἀπαντᾷ \\
\tabto{4em} κἀκεῖνο \\
\tabto{6em} ἀντ' αὐτῆς \\
\tabto{4em} βλέπει· \\
τίς δὲ οὖσα \\
βλέπει, \\
οὐδὲ τοῦτο \\
\tabto{2em} σχολάζει \\
\tabto{4em} ὁρᾶν.\\

\end{greek}
}

\begin{description}[noitemsep]
\item[῞Οταν\dots] \textbf{εὐτυχήσῃ\dots\ καὶ ἥκῃ} vremenska rečenica ima značenje pogodbene eventualne protaze, ὅταν kad god, §~488.2
\item[δὲ] čestica δέ povezuje rečenicu s prethodnom kao suprotni veznik: a\dots §~515
\item[τούτου] §~213.2
\item[εὐτυχήσῃ] εὐτυχέω τινός posjedovati nešto, domoći se nečega; 3. l. sg. konj. aor. akt.
\item[ἡ ψυχὴ] §~90
\item[ἥκῃ] ἥκω doći, biti prisutan; 3. l. sg. konj. prez. akt.
\item[πρὸς αὐτήν] §~435; §~207
\item[μᾶλλον] §~204.3
\item[δὲ] čestica δέ povezuje surečenicu s prethodnom kao suprotni veznik: a\dots §~515
\item[παρὸν] πάρειμι biti prisutan; n. sg. s. r. ptc. prez. akt.
\item[φανῇ] φαίνω \textit{u aoristu pasivnom} pojaviti se; 3. l. sg. konj. aor. pas.
\item[ὅταν\dots\ ἐκνεύσῃ] vremenska rečenica ima značenje pogodbene eventualne protaze, ὅταν kad god, §~488.2; ἐκνεύω τινός okrenuti se od nečega; 3. l. sg. konj. aor. akt.
\item[ἐκείνη] §~213.3
\item[τῶν παρόντων] πάρειμι biti prisutan; g. pl. s. r. ptc. prez. akt.
\item[παρασκευάσασα] παρασκευάζω pripremiti; n. sg. ž. r. ptc. aor. akt.
\item[αὑτὴν] §~209.1
\item[ὅτι μάλιστα] najviše što je moguće; §~204.3
\item[καλὴν] §~103
\item[εἰς ὁμοιότητα] §~419; §~123
\item[ἐλθοῦσα] ἔρχομαι ići; n. sg. ž. r. ptc. aor. akt.
\item[ἡ\dots\ παρασκευὴ] §~90
\item[δὲ] čestica δέ povezuje surečenicu s prethodnom kao suprotni veznik: a\dots §~515
\item[ἡ κόσμησις] §~165
\item[δήλη που] §~40; §~103;  που ograničava ili pojačava doseg prethodne riječi: ``nekako\dots'' ili ``itekako\dots''
\item[τοῖς παρασκευαζομένοις] παρασκευάζω pripremiti; d. pl. m. r. ptc. prez. medpas.
\item[ἰδοῦσα] ὁράω gledati, vidjeti; n. sg. ž. r. ptc. aor. akt.
\item[δὲ] čestica δέ povezuje surečenicu s prethodnom kao suprotni veznik: a\dots §~515
\item[ἐν αὐτῇ] §~426; §~207
\item[φανέντα] φαίνω aor. pas. pojaviti se; a. sg. m. r. ptc. aor. pas.
\item[γὰρ] čestica najavljuje iznošenje dokaza ili uzroka prethodne tvrdnje: naime\dots §~517
\item[οὐδὲν] §~224.2
\item[οὐδ' ἔτι] §~68
\item[δύο] §~224
\item[ἀλλ' ἓν] §~68; §~224
\item[ἄμφω] §~224.3
\item[γὰρ] čestica najavljuje iznošenje dokaza ili uzroka prethodne tvrdnje: naime\dots §~517
\item[ἂν διακρίναις] optativ s ἂν kao potencijal sadašnji, §~464.2; διακρίνω razlučiti; 2. l. sg. opt. aor. akt.
\item[οὐ\dots\ ἔτι] ne više
\item[ἕως πάρεστι] zavisna vremenska rečenica, §~487; πάρειμι biti prisutan; 3. l. sg. ind. prez. akt.
\item[μίμησις] §~165
\item[δὲ] čestica δέ povezuje rečenicu s prethodnom kao suprotni veznik: a\dots §~515
\item[τούτου] §~213.2
\item[οἱ ἐνταῦθα ἐρασταὶ] ἐνταῦθα upotrijebljeno kao atribut; §~100
\item[ἐρώμενοι] ἐράω voljeti, ljubiti; n. pl. m. r. ptc. prez. medpas.
\item[συγκρῖναι] συγκρίνω spojiti; inf. aor. akt.
\item[θέλοντες] θέλω željeti; n. pl. m. r. ptc. prez. akt.
\item[σώματος] §~123
\item[οὔτε\dots\ ἔτι\dots, οὔτε\dots] οὔτε ἔτι i ne više; koordinacija niječnim veznicima οὔτε\dots\ οὔτε\dots\ i ne\dots\ i ne\dots
\item[αἰσθάνεται] αἰσθάνομαί τινος osjećati nešto
\item[ἐστὶν] εἰμί biti; 3. l. sg. ind. prez. akt.
\item[ἐν αὐτῷ] §~426; §~207
\item[ἑαυτὴν] §~208
\item[ἄλλο τι] §~40;  §~212; §~217
\item[λέγει] λέγω τινά τι govoriti nešto o nekome; 3. l. sg. ind. prez. akt.
\item[ἄνθρωπον] §~82
\item[ζῷον] §~82
\item[ὄν] εἰμί biti; a. sg. s. r. ptc. prez. akt.
\item[πᾶν] §~193
\item[ἀνώμαλος] §~106
\item[γὰρ] čestica najavljuje iznošenje dokaza ili uzroka prethodne tvrdnje: naime\dots §~517
\item[ἡ\dots\ θέα] §~90 (pazi na naglasak, ovo nije imenica θεά!)
\item[τούτων πως] §~40; §~213.2; πως izražava nesigurnost: nekako\dots
\item[οὐδὲ\dots\ οὔτε\dots, ἀλλὰ\dots] koordinacija (niječnim) sastavnim i suprotnim veznikom
\item[σχολὴν] §~90
\item[ἄγει] σχολὴν ἄγω πρός τι imati vremena za nešto; 3. l. sg. ind. prez. akt.
\item[πρὸς αὐτὰ] §~435; §~207
\item[θέλει] θέλω željeti; 3. l. sg. ind. prez. akt.
\item[αὐτὸ] §~207
\item[ζητήσασα] ζητέω tražiti, željeti; n. sg. ž. r. ptc. aor. akt.
\item[ἐκείνῳ] §~213.3
\item[παρόντι] πάρειμι biti prisutan; d. sg. s. r. ptc. prez. akt.
\item[ἀπαντᾷ] ἀπαντάω τινί susresti nekoga
\item[κἀκεῖνο] §~66;  §~213.3
\item[ἀντ' αὐτῆς] §~68; §~422; §~207
\item[βλέπει] βλέπω gledati; 3. l. sg. ind. prez. akt.
\item[τίς δὲ] §~217; čestica δέ povezuje surečenicu s prethodnom kao suprotni veznik: a\dots §~515
\item[οὖσα] εἰμί biti; n. sg. ž. r. ptc. prez. akt.
\item[βλέπει] βλέπω gledati; 3. l. sg. ind. prez. akt.
\item[τοῦτο] §~213.2
\item[σχολάζει] σχολάζω imati vremena; 3. l. sg. ind. prez. akt.
\item[ὁρᾶν] ὁράω gledati, vidjeti; inf. prez. akt.

\end{description}

%5


{\large
\begin{greek}
\noindent ῎Ενθα δὴ \\
οὐδὲν \\
\tabto{2em} πάντων \\
ἀντὶ τούτου \\
ἀλλάξαιτο, \\
οὐδ' \\
\tabto{2em} εἴ τις \\
\tabto{2em} \tabto{2em} αὐτῇ \\
\tabto{2em} πάντα τὸν οὐρανὸν \\
\tabto{2em} ἐπιτρέποι, \\
ὡς \\
\tabto{2em} \uuline{οὐκ ὄντος ἄλλου} \\
\tabto{4em} \uuline{ἔτι ἀμείνονος} \\
\tabto{4em} \uuline{οὐδὲ μᾶλλον ἀγαθοῦ}· \\
οὔτε γὰρ ἀνωτέρω \\
τρέχει \\
τά τε ἄλλα πάντα \\
\uuline{κατιούσης}, \\
κἂν \\
\tabto{2em} ᾖ ἄνω.\\

\end{greek}
}

\begin{description}[noitemsep]
\item[῎Ενθα δὴ] na \textit{tom} mjestu\dots; čestica naglašava prilog mjesta
\item[οὐδὲν] §~224.2
\item[πάντων] §~193
\item[ἀντὶ τούτου] §~422; §~213.2
\item[ἀλλάξαιτο] ἀλλάσσω med. τι ἀντί τινος zamijeniti nešto za nešto; 3. l. sg. opt. aor. med.
\item[οὐδ' εἴ τις] §~68; §~40
\item[ἀλλάξαιτο\dots\ εἴ\dots\ ἐπιτρέποι] zavisna pogodbena rečenica, potencijalni oblik, ali optativ u apodozi ovdje je bez  ἄν, §~477; ἐπιτρέπω ponuditi; 3. l. sg. opt. prez. akt.
\item[τις] §~217
\item[αὐτῇ] §~207
\item[πάντα] §~193
\item[τὸν οὐρανὸν] §~82
\item[ὡς\dots\ ὄντος] GA s adverbnim, kauzalnim značenjem, §~503.2; εἰμί biti; g. sg. m. r. ptc. prez. akt.
\item[ἄλλου] §~212
\item[ἀμείνονος] §~202
\item[μᾶλλον] §~204.3
\item[ἀγαθοῦ] §~103
\item[γὰρ] čestica najavljuje iznošenje dokaza ili uzroka prethodne tvrdnje: naime\dots §~517
\item[οὔτε\dots\  τε] koordinacija česticama (niječnom i potvrdnom) sastavnog značenja: ne\dots\ i\dots
\item[ἀνωτέρω] §~204.b2
\item[τρέχει] τρέχω ići, stremiti; 3. l. sg. ind. prez. akt.
\item[τά τε ἄλλα] §~40; §~373; §~212
\item[πάντα] §~193
\item[κατιούσης] sc.\ ψυχῆς; κάτειμι sići; g. sg. ž. r. ptc. fut. akt.; akuzativ označava kamo se silazi; subjekt GA često se izostavlja ako se lako razumije §~504.a
\item[κἂν ᾖ] §~66; κἂν uvodi zavisnu dopusnu rečenicu, ovdje potencijalnu (to pokazuje konjunktiv): ma kako\dots §~480; εἰμί biti; 3. l. sg. konj. prez. akt.

\end{description}

%6


{\large
\begin{greek}
\noindent ῞Ωστε \\
\tabto{2em} τότε \\
\tabto{2em} ἔχει \\
\tabto{4em} καὶ τὸ κρίνειν καλῶς \\
\tabto{4em} καὶ γιγνώσκειν, \\
\tabto{6em} ὅτι \\
\tabto{8em} τοῦτό ἐστιν \\
\tabto{10em} οὗ ἐφίετο, \\
\tabto{4em} καὶ τίθεσθαι, \\
\tabto{6em} ὅτι \\
\tabto{8em} μηδέν ἐστι \\
\tabto{10em} κρεῖττον αὐτοῦ.\\

\end{greek}
}

\begin{description}[noitemsep]
\item[ἔχει] ἔχω uz infinitiv: moći\dots; 3. l. sg. ind. prez. akt.
\item[καὶ\dots\ καὶ\dots\ καὶ\dots] koordinacija pomoću sastavnih veznika
\item[τὸ κρίνειν καλῶς] §~373; κρίνω procijeniti; inf. prez. akt.
\item[τὸ\dots\ γιγνώσκειν] §~373; γιγνώσκω spoznati, prepoznati; inf. prez. akt.
\item[τοῦτό ἐστιν] §~40; §~213.2; εἰμί biti; 3. l. sg. ind. prez. akt.
\item[οὗ] §~215
\item[ἐφίετο] ἐφίημι med. τινός željeti nešto; 3. l. sg. impf. med.
\item[τίθεσθαι] τίθημι med. smatrati, zaključiti; inf. prez. medpas.
\item[μηδέν ἐστι] §~40; §~224.2
\item[ἐστι] εἰμί biti; 3. l. sg. ind. prez. akt.
\item[κρεῖττον] §~202
\item[αὐτοῦ] §~207

\end{description}

%7


{\large
\begin{greek}
\noindent Οὐ γάρ ἐστιν ἀπάτη \\
\tabto{2em} ἐκεῖ· \\
ἢ ποῦ ἂν \\
\tabto{2em} τοῦ ἀληθοῦς \\
\tabto{2em} ἀληθέστερον τύχοι;\\

\end{greek}
}

\begin{description}[noitemsep]
\item[γάρ ἐστιν] §~40; čestica γάρ najavljuje iznošenje dokaza ili uzroka prethodne tvrdnje: naime\dots §~517; ἔστι (bezlično) postojati (zanijekano: nema\dots); 3. l. sg. ind. prez. akt.
\item[ἀπάτη] §~90
\item[ποῦ] upitni prilog
\item[ἂν\dots\ τύχοι] optativ s ἂν pokazuje mogućnost u sadašnjosti (potencijal sadašnji) §~464.2; τυγχάνω naći se; 3. l. sg. opt. aor. akt.
\item[τοῦ ἀληθοῦς] §~373; §~153
\item[ἀληθέστερον] §~198

\end{description}

%8


{\large
\begin{greek}
\noindent ῝Ο οὖν λέγει, \\
ἐκεῖνό ἐστι, \\
καὶ ὕστερον λέγει, \\
καὶ σιωπῶσα δὲ λέγει \\
καὶ εὐπαθοῦσα οὐ ψεύδεται, \\
\tabto{2em} ὅτι εὐπαθεῖ· \\
οὐδὲ \uuline{γαργαλιζομένου} λέγει \uuline{τοῦ σώματος}, \\
ἀλλὰ τοῦτο γενομένη, \\
\tabto{2em} ὃ πάλαι, \\
\tabto{2em} ὅτε εὐτύχει.\\

\end{greek}
}

\begin{description}[noitemsep]
\item[῝Ο] §~215
\item[λέγει] λέγω govoriti; 3. l. sg. ind. prez. akt.
\item[ἐκεῖνό ἐστι] §~40
\item[ἐκεῖνό] §~213.3
\item[ἐστι] εἰμί biti; 3. l. sg. ind. prez. akt.
\item[ὕστερον] §~204.3
\item[λέγει] λέγω govoriti; 3. l. sg. ind. prez. akt.
\item[καὶ\dots\ δὲ\dots] kombinacija čestica najavljuje dodavanje nečega suprotnog prethodnome: a i\dots
\item[σιωπῶσα] σιωπάω šutjeti; n. sg. ž. r. ptc. prez. akt.
\item[λέγει] λέγω govoriti; 3. l. sg. ind. prez. akt.
\item[εὐπαθοῦσα] εὐπαθέω uživati; n. sg. ž. r. ptc. prez. akt.
\item[ψεύδεται] ψεύδω med. lagati; 3. l. sg. ind. prez. medpas.
\item[εὐπαθεῖ] εὐπαθέω uživati; 3. l. sg. ind. prez. akt.
\item[γαργαλιζομένου] γαργαλίζω škakljati, \textit{preneseno} uzbuđivati; g. sg. s. r. ptc. prez. medpas.; particip kao dio GA (s uzročnim značenjem), u hiperbatu
\item[λέγει] λέγω govoriti; 3. l. sg. ind. prez. akt.
\item[τοῦ σώματος] §~123
\item[τοῦτο\dots, ὃ\dots] §~213.2; §~215; koordinacija pomoću antecedenta (ovdje: pokazne zamjenice) i konektora (ovdje: odnosne zamjenice)
\item[γενομένη] γίγνομαι postati; n. sg. ž. r. ptc. aor. (med.)
\item[ὃ] §~215
\item[ὅτε εὐτύχει] ὅτε kad\dots; uvodi zavisnu vremensku rečenicu §~487; εὐτυχέω biti dobro, imati sreće; 3. l. sg. impf. akt.

\end{description}

%9


{\large
\begin{greek}
\noindent ᾿Αλλὰ καὶ τὰ ἄλλα πάντα, \\
\tabto{2em} οἷς πρὶν ἥδετο, \\
\tabto{2em} ἀρχαῖς ἢ δυνάμεσιν ἢ πλούτοις ἢ κάλλεσιν ἢ ἐπιστήμαις, \\
ταῦτα \\
ὑπεριδοῦσα λέγει \\
\tabto{2em} οὐκ ἂν εἰποῦσα \\
\tabto{2em} μὴ κρείττοσι συντυχοῦσα \\
\tabto{4em} τούτων· \\
οὐδὲ φοβεῖται, \\
\tabto{2em} μή τι πάθῃ, \\
μετ' ἐκείνου οὖσα \\
οὐδ' ὅλως ἰδοῦσα· \\
εἰ δὲ καὶ τὰ ἄλλα \\
\tabto{2em} τὰ περὶ αὐτὴν \\
φθείροιτο, \\
εὖ μάλα καὶ βούλεται, \\
\tabto{2em} ἵνα πρὸς τούτῳ ᾖ μόνον· \\
εἰς τόσον \\
\tabto{2em} ἥκει \\
εὐπαθείας.\\

\end{greek}
}

\begin{description}[noitemsep]
\item[τὰ ἄλλα πάντα] §~373; §~375; §~212; §~193
\item[οἷς] §~215
\item[ἥδετο] ἥδομαί τινι uživati u nečemu; 3. l. sg. impf. prez. med.
\item[ἀρχαῖς] §~90
\item[δυνάμεσιν] §~165
\item[πλούτοις] §~82
\item[κάλλεσιν] §~153
\item[ἐπιστήμαις] §~90
\item[ταῦτα ] §~213.2
\item[ὑπεριδοῦσα ] ὑπεροράω prezirati; n. sg. ž. r. ptc. aor. akt.
\item[λέγει ] λέγω govoriti; 3. l. sg. ind. prez. akt.
\item[ἂν εἰποῦσα μὴ\dots\ συντυχοῦσα] particip s ἂν zamjenjuje rečenicu s irealnim značenjem, §~506; λέγω govoriti; n. sg. ž. r. ptc. aor. akt; μή s participom izriče uvjet (pogodbu): kad ne bi\dots
\item[κρείττοσι ] §~202
\item[συντυχοῦσα ] συντυγχάνω τινί susresti se s nekim; n. sg. ž. r. ptc. aor. akt.
\item[τούτων] §~213.2
\item[οὐδὲ] ima funkciju sastavnog (niječnog) veznika: i ne\dots
\item[φοβεῖται, μή\dots] φοβέω med. bojati se; 3. l. sg. ind. prez. medpas.; μή iza izraza bojazni uvodi zavisnu zahtjevnu (namjernu) rečenicu: da će\dots §~471
\item[μή τι] §~40
\item[τι ] §~217
\item[πάθῃ] πάσχω trpjeti; 3. l. sg. konj. aor. akt.
\item[μετ' ἐκείνου ] §~68;  §~213.3
\item[οὖσα ] εἰμί biti; n. sg. ž. r. ptc. prez. akt.
\item[οὐδ' ὅλως] §~68;  §~204
\item[ἰδοῦσα] ὁράω gledati, vidjeti; n. sg. ž. r. ptc. aor. akt.
\item[εἰ\dots\ φθείροιτο\dots\  βούλεται] pogodbena rečenica: potencijalna protaza, realna apodoza, §~479; φθείρω uništiti; 3. l. sg. opt. prez. medpas.
\item[δὲ] čestica δέ povezuje rečenicu s prethodnom kao suprotni veznik: a\dots §~515
\item[τὰ ἄλλα τὰ περὶ αὐτὴν] §~373; §~375; §~212; §~433; §~207
\item[εὖ μάλα ] vrlo jako; εὖ pojačava sljedeći prilog
\item[βούλεται] βούλομαι htjeti, željeti; 3. l. sg. ind. prez. med.
\item[ἵνα\dots\ ᾖ] namjerna rečenica s konjunktivom (jer je u glavnoj glavno vrijeme), §~470; εἰμί biti; 3. l. sg. konj. prez. akt.
\item[πρὸς τούτῳ ] §~435; §~213.2
\item[εἰς τόσον\dots\ εὐπαθείας] §~419; genitiv partitivni, §~395; §~219; §~90
\item[ἥκει] ἥκω doći, biti prisutan; 3. l. sg. ind. prez. akt.

\end{description}



%kraj
