% pregledali NČ, NZ
\section*{O autoru}

Ezop \textgreek[variant=ancient]{(Αἴσωπος,} ? 620.\ – 564.\ pr.~Kr.; \textit{floruit} oko 570.\ pr.~Kr.) grčki je autor basna, čija izvorna djela nisu sačuvana, a o životu se malo toga zna.

Točno mjesto Ezopova rođenja nije poznato. Izvori se slažu da je Ezop prvotno bio rob i da je kasnije oslobođen; navodno je bio iznimno tjelesno unakažen; prema Plutarhu, ubijen je u Delfima.

U tradiciji se Ezopovo ime prvenstveno povezuje sa životinjskom basnom \textgreek[variant=ancient]{(αἶνος),} književnom vrstom u kojoj se miješaju elementi satire, fantazije i moralne poduke. U basnama su često glavni likovi životinje s ljudskim značajkama (govor, razmišljanje\dots). Filolozi pretpostavljaju da su Ezopove basne bile zapisane već u 5. st. pr.~Kr., a bile su vrlo popularne u Ateni klasičnog doba.

Ezopov korpus basna sačuvan je predajom niza grčkih i latinskih autora. Demetrije iz Falera sastavio je (za nas izgubljenu) zbirku u deset knjiga kao priručnik govornicima. Najpoznatiji sakupljač Ezopovih basni bio je Fedar, koji ih je u 1. st. prepjevao na latinski.


\section*{O tekstu}

U ovoj basni poučno su suprotstavljeni majka i sin; majka nije na vrijeme ispravljala djetetove pogrešne postupke i kažnjavala krađu te dijete nije naučilo što je loše ponašanje. Sin je nastavio krasti i kao mladić; uhvatili su ga na djelu i osudili na smrt. Pri posljednjem je susretu majci odgrizao uho, osvećujući se jer ga je ostavila bez roditeljskog odgoja i time odvela u propast.

\section*{Pročitajte naglas grčki tekst.}

Aesop. Fabulae 216

\medskip

{\large
\begin{greek}
\noindent ΠΑΙΣ ΚΛΕΠΤΗΣ ΚΑΙ ΜΗΤΗΡ 

\medskip

\noindent Παῖς ἐκ διδασκαλείου τὴν τοῦ συμφοιτητοῦ δέλτον ἀφελόμενος τῇ μητρὶ ἐκόμισε. τῆς δὲ οὐ μόνον αὐτὸν μὴ ἐπιπληξάσης, ἀλλὰ καὶ ἐπαινεσάσης αὐτὸν ἐκ δευτέρου ἱμάτιον κλέψας ἤνεγκεν αὐτῇ· ἔτι δὲ μᾶλλον ἀποδεξαμένης αὐτῆς προϊὼν τοῖς χρόνοις ὡς νεανίας ἐγένετο, ἤδη καὶ τὰ μείζονα κλέπτειν  ἐπεχείρει. ληφθεὶς δέ ποτε ἐπ' αὐτοφώρῳ καὶ περιαγκωνισθεὶς ἐπὶ τὸν δήμιον ἀπήγετο. τῆς δὲ μητρὸς ἐπακολουθούσης αὐτῷ καὶ στερνοκοπούσης εἶπε βούλεσθαί τι αὐτῇ πρὸς τὸ οὖς εἰπεῖν καὶ προσελθούσης αὐτῆς ταχέως τοῦ ὠτίου ἐπιλαβόμενος καταδήξας ἀφείλετο. τῆς δὲ κατηγορούσης αὐτοῦ δυσσέβειαν, εἴπερ μὴ ἀρκεσθεὶς οἷς ἤδη πεπλημμέληκε καὶ τὴν μητέρα ἐλωβήσατο, ἐκεῖνος ὑπολαβὼν ἔφη· ``ἀλλ' ὅτε σοι πρῶτον τὴν δέλτον κλέψας ἤνεγκα, εἰ ἐπέπληξάς μοι, οὐκ ἂν μέχρι τούτου ἐχώρησα καὶ ἐπὶ θάνατον ἠγόμην.''

ὁ λόγος δηλοῖ, ὅτι τὸ κατ' ἀρχὰς μὴ κολαζόμενον ἐπὶ μεῖζον αὔξεται. 
\end{greek}

}

%\newpage

\section*{Analiza i komentar}

% komentar naslova dodao NJ

{\large
\noindent ΠΑΙΣ ΚΛΕΠΤΗΣ \\
ΚΑΙ \\
ΜΗΤΗΡ\\

}

\begin{description}[noitemsep]

\item[ΠΑΙΣ] §~127
\item[ΚΛΕΠΤΗΣ] §~100
\item[ΜΗΤΗΡ] §~148
\end{description}

{\large
\noindent Παῖς \\ 
\tabto{2em} ἐκ διδασκαλείου\\ 
τὴν \\
\tabto{2em} τοῦ συμφοιτητοῦ \\
δέλτον \\
ἀφελόμενος\\ 
\tabto{2em} τῇ μητρὶ \\
ἐκόμισε.\\

}


\begin{description}[noitemsep]
\item[Παῖς] §~127
\item[ἐκ διδασκαλείου] §~82; ἐκ + g.: iz, §~418
\item[τοῦ συμφοιτητοῦ] §~80, §~82
\item[ἀφελόμενος] ἀφαιρέομαι uzeti si; n. sg. m. r. ptc. aor. medpas.: uzevši si, otevši
\item[τῇ μητρὶ] §~146, 148
\item[ἐκόμισε] κομίζω τινί odnijeti komu; 3. l. sg. ind. aor. akt. 
\end{description}

{\large
\noindent \uuline{τῆς} δὲ\\
\tabto{2em} οὐ μόνον \\
\tabto{4em} αὐτὸν \uuline{μὴ ἐπιπληξάσης}, \\
\tabto{2em} ἀλλὰ καὶ \\
\tabto{4em} \uuline{ἐπαινεσάσης} αὐτὸν \\
ἐκ δευτέρου \\
\tabto{2em} ἱμάτιον κλέψας \\
ἤνεγκεν αὐτῇ·\\

}


\begin{description}[noitemsep]
\item[οὐ μόνον… ἀλλὰ καὶ] koordinacija: ne samo\dots\ nego i\dots, §~513.1
\item[τῆς μὴ ἐπιπληξάσης] ἐπιπλήσσω kazniti, prekoriti; g. sg. ž. r. ptc. aor. akt.; GA ima vrijednost vremenske ili uzročne rečenice: pošto\dots, budući da\dots
\item[(τῆς) ἐπαινεσάσης] ἐπαινέω pohvaliti; g.~sg. ž. r. ptc.\ aor.\ akt.; GA
\item[αὐτὸν] §~207
\item[ἐκ δευτέρου] priložna oznaka vremena: drugi put, §~223
\item[ἱμάτιον] §~82
\item[κλέψας] κλέπτω krasti; n. sg. m. r. ptc. aor. akt.
\item[ἤνεγκεν] φέρω nositi, donositi; 3. l. sg. ind. aor. akt.
\item[αὐτῇ] §~207
\end{description}

{\large
\noindent ἔτι δὲ μᾶλλον \\
\tabto{2em}  \uuline{ἀποδεξαμένης αὐτῆς} \\
προϊὼν\\
\tabto{2em} τοῖς χρόνοις \\
\tabto{2em} ὡς νεανίας ἐγένετο, \\
ἤδη καὶ \\
τὰ μείζονα \\
κλέπτειν\\
\tabto{2em} ἐπεχείρει.\\

}

%komentar
\begin{description}[noitemsep]

\item[δὲ] postpozitivna čestica, ovdje suprotnog značenja: a\dots\  (hrvatska se istovrijednica stavlja na prvo mjesto u rečenici)
\item[μᾶλλον] §~204. 3; komparativ priloga: više
\item[ἀποδεξαμένης αὐτῆς] ἀποδέχομαι dopuštati, odobravati; g. sg. ž. r. ptc. aor. med.; GA kao priložna oznaka vremena: kad\dots; §~207
\item[τοῖς χρόνοις] s vremenom; §~80, §~82; dativ kao priložna oznaka (vremena ili uzroka) 
\item[προϊὼν] προερχόμαι / πρόειμι nastaviti, napredovati; n. sg. m. r. ptc. prez. akt.
\item[ὡς\dots\ ἐγένετο] zavisna vremenska rečenica: kad\dots
\item[νεανίας] §~100
\item[ἐγένετο] γίγνομαι postajati, biti (kopulativan glagol, traži nužnu imensku dopunu); 3. l. sg. ind. aor. med. 
\item[τὰ μείζονα] §~196, §~200
\item[κλέπτειν] κλέπτω krasti; inf. prez. akt. 
\item[ἐπεχείρει] ἐπιχειρέω usuditi se; 3. l. sg. impf. akt.; otvara mjesto dopuni u infinitivu κλέπτειν
\end{description}


{\large
\noindent ληφθεὶς δέ\\
\tabto{2em} ποτε \\
\tabto{2em} ἐπ' αὐτοφώρῳ \\
καὶ περιαγκωνισθεὶς \\
\tabto{2em} ἐπὶ τὸν δήμιον \\
ἀπήγετο.\\

}

%komentar

\begin{description}[noitemsep]

\item[ληφθεὶς] λαμβάνω uhvatiti; n. sg. m. r. ptc. aor. pas.
\item[ληφθεὶς\dots\ ἐπ' αὐτοφώρῳ]  ἐπ᾽ αὐτοφώρῳ λαμβάνειν uhvatiti na djelu
\item[περιαγκωνισθεὶς] περιαγκωνίζω svezati ruke na leđima; n. sg. m. r. ptc. aor. pas.: ruku svezanih na leđima
\item[ἐπὶ τὸν δήμιον] §~ 80, §~82; prijedložni izraz, ἐπὶ + a.: k  (oznaka mjesta), §~418, §~436
\item[ἀπήγετο] ἀπάγω uhititi i odvesti; 3. l. sg. impf. medpas.
\end{description}


%\newpage

{\large
\noindent \uuline{τῆς δὲ μητρὸς ἐπακολουθούσης} αὐτῷ \\
\tabto{2em} \uuline{καὶ στερνοκοπούσης} \\
εἶπε \\
\tabto{2em} βούλεσθαί \\
\tabto{4em} τι \\
\tabto{4em} αὐτῇ \\
\tabto{6em} πρὸς τὸ οὖς \\
\tabto{4em} εἰπεῖν \\
καὶ \\
\tabto{2em} \uuline{προσελθούσης αὐτῆς} \\
ταχέως \\
\tabto{2em} τοῦ ὠτίου \\
\tabto{2em} ἐπιλαβόμενος \\
καταδήξας \\
\tabto{2em} ἀφείλετο.\\

}

%komentar

\begin{description}[noitemsep]

\item[τῆς δὲ μητρὸς ἐπακολουθούσης] ἐπακολουθέω τινί pratiti koga; g. sg. ž. r. ptc. prez. akt., GA ima vrijednost uzročne rečenice: kako\dots; §~146, §~148
\item[(τῆς μητρὸς) στερνοκοπούσης] στερνοκοπέομαι udarati se u prsa od žalosti; g. sg. ž. r. ptc. prez. akt.; GA kao priložna oznaka načina: udarajući se u prsa\dots; §~146, §~148, §~504
\item[αὐτῷ] §~207
\item[εἶπε] ἀγορεύω govoriti; 3. l. sg. ind. aor. akt.
\item[βούλεσθαί] βούλομαι željeti; inf. prez. medpas. 
\item[τι] §~217
\item[αὐτῇ] §~207
\item[πρὸς τὸ οὖς] §~128; πρὸς + a.: na, §~418, §~435
\item[εἰπεῖν] ἀγορεύω govoriti, inf. aor. akt.
\item[αὐτῆς] §~207
\item[προσελθούσης] προσέρχομαι približiti se, prići bliže; g. sg. ž. r. ptc. aor. akt. 
\item[προσελθούσης αὐτῆς] GA, §~504
\item[ταχέως] §~204, §~353 
\item[τοῦ ὠτίου] §~80, §~82
\item[ἐπιλαβόμενος] ἐπιλαμβάνω med. ἐπιλαμβάνομαι τινός zgrabiti što; n. sg. m. r. ptc. aor. med. 
\item[καταδήξας] καταδάκνω zagristi; n. sg. m. r. ptc. aor. akt. 
\item[ἀφείλετο] ἀφαιρέω odnijeti, \textit{ovdje} otrgnuti; 3. l. sg. ind. aor. med. 
\end{description}

%\newpage

{\large
\noindent \uuline{τῆς δὲ κατηγορούσης} αὐτοῦ \\
\tabto{4em} δυσσέβειαν, \\
\tabto{2em} εἴπερ μὴ ἀρκεσθεὶς \\
\tabto{4em} οἷς ἤδη πεπλημμέληκε \\
\tabto{2em} καὶ\\
\tabto{2em} τὴν μητέρα \\
\tabto{2em} ἐλωβήσατο, \\
\tabto{4em} ἐκεῖνος \\
\tabto{4em} ὑπολαβὼν ἔφη·\\

}

%komentar

\begin{description}[noitemsep]

\item[τῆς δὲ] §~370; član u kombinaciji s česticom δέ naznačuje promjenu subjekta
\item[κατηγορούσης] κατηγορέω τινός τι optužiti koga za što; g. sg. ž. r. ptc. prez. akt.; GA ima vrijednost vremenske rečenice: kad\dots 
\item[αὐτοῦ] §~207
\item[δυσσέβειαν] §~80, §~90
\item[εἴπερ] (baš) kao da
\item[μὴ] negacija, §~508, §~509.2
\item[ἀρκεσθεὶς] ἀρκέω dostajati, pasiv uz τινι biti zadovoljan čime: nije ga zadovoljilo ono\dots; n. sg. m. r. ptc. aor. pas.
\item[οἷς] §~214; odnosna zamjenica uvodi zavisnu odnosnu rečenicu koja u glavnoj ima službu priložne oznake (sredstva): ono što\dots
\item[πεπλημμέληκε] πλημμελέω pogrešno postupati; 3. l. sg. ind. perf. akt.
\item[τὴν μητέρα] §~90, §~148
\item[ἐλωβήσατο] λωβάομαι (figurativno) osakatiti; 3. l. sg. ind. aor. med. 
\item[ἐκεῖνος]  §~213
\item[ὑπολαβὼν] ὑπολαμβάνω odvratiti; n. sg. ptc. aor. akt.
\item[ἔφη] φημί reći; 3. l. sg. impf. akt.; u kontekstu razgovora: ὑπολαβὼν ἔφη u odgovoru reče, tj.\ odgovori
\end{description}

{\large
\noindent ``ἀλλ' \\
\tabto{2em} ὅτε \\
\tabto{2em} σοι \\
\tabto{4em} πρῶτον \\
\tabto{2em} τὴν δέλτον \\
\tabto{4em} κλέψας \\
\tabto{2em} ἤνεγκα, \\
εἰ ἐπέπληξάς μοι, \\
\tabto{2em} οὐκ ἂν \\
\tabto{4em} μέχρι τούτου \\
\tabto{2em} ἐχώρησα\\
καὶ\\
\tabto{2em} ἐπὶ θάνατον ἠγόμην.''\\

}

%komentar

\begin{description}[noitemsep]

\item[ἀλλ' = ἀλλά] §~68.c, suprotni veznik: nego
\item[σοι] §~205
\item[κλέψας] κλέπτω krasti; n. sg. m. r. ptc. aor. akt.
\item[τὴν δέλτον] §~80, §~82, §~83, §~90
\item[ἤνεγκα] φέρω nositi; 1. l. sg. ind. aor. akt. 
\item[εἰ] pogodbeni veznik otvara mjesto zavisnoj irealnoj pogodbenoj rečenici: da\dots
\item[ἐπέπληξάς] ἐπιπλήσσω prekoriti; 2. l. sg. ind. aor. akt.
\item[μοι] §~205
\item[οὐκ ἂν\dots\ ἐχώρησα] χωρέω doći; 1. l. sg. ind. aor. akt.
\item[μέχρι τούτου] §~213.2; priložna oznaka, μέχρι + g.: do, §~417
\item[ἐπὶ θάνατον] §~82; priložna oznaka ἐπὶ + a.: „k'' §~418, §~436
\item[ἠγόμην] ἄγω voditi; 1. l. sg. impf. medpas.
\end{description}

{\large
\noindent ὁ λόγος δηλοῖ, \\
\tabto{2em} ὅτι \\
\tabto{4em} τὸ κατ' ἀρχὰς μὴ κολαζόμενον \\
\tabto{6em} ἐπὶ μεῖζον \\
\tabto{4em} αὔξεται.\\

}

%komentar

\begin{description}[noitemsep]
\item[ὁ λόγος] §~80, §~82
\item[δηλοῖ] δηλόω pokazivati; 3. l. sg. ind. prez. akt.
\item[ὅτι] veznik uvodi zavisnu izričnu rečenicu: da\dots
\item[κατ' ἀρχὰς] priložna oznaka vremena: na početku
\item[τὸ μὴ κολαζόμενον] κολάζω kazniti; n. sg. s. r. ptc. prez. medpas.; poimeničeni particip: ono što se ne sprečava (poimeničenje članom §~373)
\item[ἐπὶ μεῖζον] §~200.1; priložna oznaka; §~418
\item[αὔξεται] αὐξάνω umnožavati se; 3. l. sg. ind. prez. medpas.
\end{description}

