% Redaktura NZ, NJ; NZ 25. 10. 2021.
\section*{O autoru}

Izokrat (Ἰσοκράτης, 436.\ – 338.) potječe iz bogate atenske obitelji, no većinu je imovine izgubio u Peloponeskom ratu tako da se 403.\ prihvaća pisanja govora (logografije). Oko 390.\ napušta to zanimanje i počinje podučavati izlažući pedagoške, filozofske i političke stavove u ogledima koji su imali oblik govora, no nisu bili namijenjeni javnoj i usmenoj izvedbi. Politički, zalagao se za prilagodbu rastućoj moći Filipa Makedonskog i za panhelensko jedinstvo. Sebe je smatrao filozofom i pedagogom, a ne govornikom i retoričarom. Njegova škola usvojila je široko poimanje retorike i primijenjene filozofije, a privlačila je učenike iz cijeloga grčkog svijeta (uključujući Izeja, Likurga i Hiperida) postavši tako glavni takmac Platonovoj Akademiji. Izokrat je silno utjecao na obrazovanje i retoriku u helenističko, rimsko i moderno doba, sve do XVIII. stoljeća.

\section*{O tekstu}

\textit{Buzirid} (Βούσιρις) napisan je poslije 390.\ pr.~Kr., kada je Izokrat započeo javno i obrazovno djelovanje. To je vrijeme kada na društvenoj i misaonoj razini prevladavaju pitanja politike i uvjerljivog govorenja, filozofije i obrazovanja. Sva su ta pitanja obilježena mišlju i djelom filozofa Platona, no \textit{Buzirid} je jedan od rijetkih sačuvanih spisa koji izlažu ideje drukčije od Platonovih ili njima konkurentne. 

Izokratovo je djelo oblikovano kao pismo retoričaru i sofistu Polikratu. Polikratovo djelovanje predstavlja sve ono čemu se Izokrat u retorici i obrazovanju protivi. Svoju estetiku Izokrat zaodijeva u pohvalu mitskog egipatskog kralja Buzirida. Pritom razmatra egipatsku civilizaciju i njezinu važnost. Sadržajnu raznovrsnost prati i raznolikost stila pa se u tekstu izmjenjuju polemičnost, sarkazam, šaljivost, pripovijedanje i epideiktičan ton. \textit{Buzirid} je važan za razumijevanje javne uloge Izokrata kao pisca i odgojitelja.

Grčka mitološka literatura različito prikazuje Buziridovo podrijetlo i život. Prema mitografu Apolodoru (II.~st.\ pr.~Kr.), Buzirid je bio sin boga Posejdona i Epafove kćeri Lizijanase; nakon što je Egipat devet godina morila glad, prorok Frazije s Cipra obznanjuje da će ona prestati ako Egipćani svake godine žrtvuju Zeusu jednog stranca. Buzirid je prihvatio savjet i prvo ubio samog Frazija, a zatim i sve ostale strance. Kad je Heraklo došao u Egipat, i njega su svezali da ga žrtvuju. No, Heraklo je rastrgnuo lance te ubio i Buzirida i njegovu svitu. 

Početak Izokratova djela, ovdje donesen, izlaže kako je Pitagora od Egipćana naučio filozofiju i donio je u Grčku.

%\newpage

\section*{Pročitajte naglas grčki tekst.}

%Naslov prema izdanju

Isoc.\ Busiris 28.1

\medskip

{\large
\begin{greek}
\noindent Ἀφικόμενος εἰς Αἴγυπτον καὶ μαθητὴς ἐκείνων γενόμενος τήν τ' ἄλλην φιλοσοφίαν πρῶτος εἰς τοὺς ῞Ελληνας ἐκόμισεν, καὶ τὰ περὶ τὰς θυσίας καὶ τὰς ἁγιστείας τὰς ἐν τοῖς ἱεροῖς ἐπιφανέστερον τῶν ἄλλων ἐσπούδασεν, ἡγούμενος, εἰ καὶ μηδὲν αὐτῷ διὰ ταῦτα πλέον γίγνοιτο παρὰ τῶν θεῶν, ἀλλ' οὖν παρά γε τοῖς ἀνθρώποις ἐκ τούτων μάλιστ' εὐδοκιμήσειν. ῞Οπερ αὐτῷ καὶ συνέβη· τοσοῦτον γὰρ εὐδοξίᾳ τοὺς ἄλλους ὑπερέβαλεν ὥστε καὶ τοὺς νεωτέρους ἅπαντας ἐπιθυμεῖν αὐτοῦ μαθητὰς εἶναι, καὶ τοὺς πρεσβυτέρους ἥδιον ὁρᾶν τοὺς παῖδας τοὺς αὑτῶν ἐκείνῳ συγγιγνομένους ἢ τῶν οἰκείων ἐπιμελουμένους. Καὶ τούτοις οὐχ οἷόν τ' ἀπιστεῖν· ἔτι γὰρ καὶ νῦν τοὺς προσποιουμένους ἐκείνου μαθητὰς εἶναι μᾶλλον σιγῶντας θαυμάζουσιν ἢ τοὺς ἐπὶ τῷ λέγειν μεγίστην δόξαν ἔχοντας.

\end{greek}

}

%\newpage

\section*{Analiza i komentar}


%1

{\large
\noindent Ἀφικόμενος εἰς Αἴγυπτον\\
καὶ \\
μαθητὴς ἐκείνων γενόμενος \\
τήν τ' ἄλλην φιλοσοφίαν \\
\tabto{2em} πρῶτος \\
\tabto{4em} εἰς τοὺς ῞Ελληνας \\
\tabto{2em} ἐκόμισεν,\\
καὶ τὰ περὶ τὰς θυσίας \\
καὶ τὰς ἁγιστείας τὰς ἐν τοῖς ἱεροῖς \\
ἐπιφανέστερον \\
\tabto{2em} τῶν ἄλλων \\
ἐσπούδασεν,\\
ἡγούμενος, \\
\tabto{2em} εἰ καὶ μηδὲν \\
\tabto{2em} αὐτῷ \\
\tabto{4em} διὰ ταῦτα \\
\tabto{2em} πλέον γίγνοιτο \\
\tabto{4em} παρὰ τῶν θεῶν, \\
\tabto{2em} ἀλλ' οὖν \\
\tabto{4em} παρά γε τοῖς ἀνθρώποις \\
\tabto{4em} ἐκ τούτων \\
\tabto{2em} μάλιστ' εὐδοκιμήσειν.

}

\begin{description}[noitemsep]

\item[Ἀφικόμενος] ἀφικνέομαι doći; n. sg. m. r. ptc. aor. med.; neizrečeni je subjekt \textgreek[variant=ancient]{Πυθαγόρας}
\item[εἰς Αἴγυπτον] §~78, §~82, §~83
\item[μαθητὴς ἐκείνων] §~100, §~213.3, §~442.1; ἐκείνων sc.\ Αἰγυπτίων, genitiv zamjenice izražava pripadnost, odgovara hrvatskoj posvojnoj zamjenici
\item[γενόμενος] γίγνομαι postati; n. sg. m. r. ptc. aor. med.; glagol nepotpuna značenja otvara mjesto imenskoj dopuni (μαθητὴς)
\item[τήν τ' ἄλλην\dots] \textbf{καὶ τὰ περὶ τὰς θυσίας\dots\ καὶ τὰς ἁγιστείας\dots}\ koordinacija rečeničnih članova s pomoću sastavnih veznika; τ' = τε §~68.c
\item[τήν τ' ἄλλην φιλοσοφίαν] §~212, §~82, §~90
\item[πρῶτος] §~223-224
\item[εἰς τοὺς ῞Ελληνας] §~80, §~82, §~131
\item[ἐκόμισεν] κομίζω εἴς τινα prenijeti komu, donijeti komu; 3. l. sg. ind. aor. akt. 
\item[τὰ περὶ τὰς θυσίας] §~82, §~90; prijedložni izraz περὶ + a.: povezano s\dots, §~418, §~433; poimeničenje članom §~373. Za prevođenje ove sintagme treba uz član τὰ zamisliti riječ \textgreek[variant=ancient]{πράγματα: τὰ περὶ τὰς θυσίας πράγματα,} stvari povezane s prinošenjem žrtava
\item[τὰς ἁγιστείας τὰς ἐν τοῖς ἱεροῖς] §~82, §~90, §~82; atributni položaj ostvaren ponavljanjem člana, §~375: svete obrede u hramovima
\item[ἐπιφανέστερον] §~153, §~194.2, §~198; ovdje upotrijebljeno priložno
\item[τῶν ἄλλων] §~80, §~82, §~212; genitiv usporedbe (comparationis), §~404
\item[ἐσπούδασεν] σπουδάζω baviti se; 3. l. sg. ind. aor. akt.
\item[ἡγούμενος] ἡγέομαι smatrati; n. sg. m. r. ptc. prez. medpas.; \textit{verbum sentiendi} otvara mjesto dopuni u infinitivu
\item[εἰ καὶ] kombinacija veznika uvodi dopusnu rečenicu §~480
\item[αὐτῷ] §~207; \textit{dativus commodi} §~412.1
\item[διὰ ταῦτα] §~418, §~428.B, §~213.2; sc.\ διὰ τὰ περὶ τὰς θυσίας itd.; priložna oznaka uzroka: zbog toga\dots
\item[πλέον] §~202, §~196
\item[γίγνοιτο] γίγνομαι postati; 3. l. sg. opt. prez. medpas.; fraza πλέον γίγνεταί τινι netko dobiva više
\item[παρὰ τῶν θεῶν] §~80, §~82; prijedložni izraz παρὰ + g.: od\dots\ §~418, §~434
\item[παρά\dots\ τοῖς ἀνθρώποις] §~80, §~82
\item[γε] naglašava drugi dio izbora: onda\dots
\item[ἐκ τούτων] §~213.2; sc.\ ἐκ τῶν περὶ τὰς θυσίας itd.
\item[εὐδοκιμήσειν] εὐδοκιμέω biti na dobru glasu; inf. fut. akt.; infinitiv ovisan o ἡγούμενος
\end{description}

{\large
\noindent ῞Οπερ αὐτῷ καὶ συνέβη·

}

\begin{description}[noitemsep]

\item[῞Οπερ] §~443.3; a upravo to\dots
\item[αὐτῷ] §~207, sc.\ Πυθαγόρᾳ
\item[συνέβη] συμβαίνω τινί dogoditi se komu; 3. l. sg. ind. aor. akt. 
\end{description}

%\newpage

%3 itd

{\large
\noindent τοσοῦτον γὰρ \\
\tabto{2em} εὐδοξίᾳ \\
τοὺς ἄλλους \\
ὑπερέβαλεν \\
\tabto{2em} ὥστε καὶ \\
\tabto{4em} τοὺς νεωτέρους ἅπαντας ἐπιθυμεῖν \\
\tabto{8em} αὐτοῦ \\
\tabto{6em} μαθητὰς εἶναι, \\
\tabto{4em} καὶ τοὺς πρεσβυτέρους ἥδιον ὁρᾶν \\
\tabto{6em} τοὺς παῖδας τοὺς αὑτῶν \\
\tabto{8em} ἐκείνῳ \\
\tabto{6em} συγγιγνομένους\\
\tabto{4em} ἢ \\
\tabto{8em} τῶν οἰκείων \\
\tabto{6em} ἐπιμελουμένους.\\

}

\begin{description}[noitemsep]

\item[τοσοῦτον] §~213.4; ovdje upotrijebljeno priložno
\item[γὰρ] čestica s uzročnim značenjem: jer\dots, §~517
\item[εὐδοξίᾳ] §~90
\item[τοὺς ἄλλους] §~80, §~82, §~212
\item[ὑπερέβαλεν] ὑπερβάλλω premašivati; 3. l. sg. ind. aor. akt.
\item[ὥστε] veznik uvodi zavisnu posljedičnu rečenicu, kod različitih subjekata otvara mjesto A+I: tako da\dots
\item[τοὺς νεωτέρους ἅπαντας] §~80, §~82, §~197, §~193
\item[ἐπιθυμεῖν] ἐπιθυμέω žudjeti, otvara mjesto infinitivu; inf. prez. akt. ovdje se može prevesti u prošlosti (kao da je inf. imperfekta)
\item[αὐτοῦ] §~207, sc.\ Πυθαγόρου
\item[μαθητὰς] §~100
\item[εἶναι] εἰμί biti; inf. prez. akt.
\item[τοὺς πρεσβυτέρους] §~171, §~198, §~80, §~82; akuzativ ima ulogu subjekta u A+I
\item[ἥδιον] §~200, priložno: radije
\item[ὁρᾶν] ὁράω gledati; inf. prez. akt. 
\item[παῖδας] §~127
\item[τοὺς αὑτῶν] §~80, §~82, §~207; αὑτῶν posvojni genitiv u atributnom položaju §~375
\item[ἐκείνῳ] §~213
\item[συγγιγνομένους] predikatni particip uz gl. ὁράω (kao i ἐπιμελουμένους niže) §~502; \textgreek[variant=ancient]{συγγίγνομαί τινι} družiti se s kime, biti s kime; a. pl. m. r. ptc. prez. medpas.
\item[ἢ] komparativno: nego\dots
\item[τῶν οἰκείων]  §~80, §~82; τὰ οἰκεῖα kućni poslovi
\item[ἐπιμελουμένους] ἐπιμελέομαί τινος brinuti se za što, baviti se čime; a. pl. m. r. ptc. prez. medpas.; predikatni particip uz gl. ὁράω §~502
\end{description}

%\newpage


%4

{\large
\noindent Καὶ τούτοις οὐχ οἷόν τ' \\
\tabto{2em} ἀπιστεῖν· \\
ἔτι γὰρ καὶ νῦν\\
\tabto{2em} τοὺς προσποιουμένους \\
\tabto{6em} ἐκείνου \\
\tabto{4em} μαθητὰς εἶναι \\
\tabto{2em} μᾶλλον σιγῶντας θαυμάζουσιν \\
\tabto{2em} ἢ \\
\tabto{2em} τοὺς \\
\tabto{4em} ἐπὶ τῷ λέγειν \\
\tabto{4em} μεγίστην δόξαν \\
\tabto{2em} ἔχοντας.\\

}

\begin{description}[noitemsep]
\item[τούτοις] §~213.2
\item[οἷον] bezlična fraza οἷόν ἐστί τινι komu je moguće; otvara mjesto dopuni u infinitivu
\item[ἀπιστεῖν] ἀπιστέω ne vjerovati; inf. prez. akt.; infinitiv kao subjekt uz bezlične izraze §~492
\item[γὰρ] čestica, ovdje uzročnog značenja: jer\dots, §~517
\item[τοὺς προσποιουμένους] προσποιέω, med. προσποιέομαι, otvara mjesto infinitivu: izdavati se za što, praviti se, LSJ προσποιέω II.4; a. pl. m. r. ptc. prez. medpas.; poimeničenje participa članom §~499
\item[ἐκείνου] §~213.3, §~442.1
\item[μᾶλλον\dots\  ἢ] više\dots\ nego\dots, koordinacija rečeničnih članova; §~204.3
\item[σιγῶντας] σιγάω šutjeti; a. pl. ptc. m. r. prez. akt.
\item[θαυμάζουσιν] θαυμάζω τινά diviti se komu; 3. l. pl. ind. prez. akt.
\item[ἐπὶ τῷ λέγειν] λέγω govoriti; inf. prez. akt.; poimeničenje infinitiva članom §~497; 418, §~436.B
\item[μεγίστην δόξαν] §~200, §~196, §~80, §~82, §~97
\item[τοὺς ἔχοντας] ἔχω imati, držati; a. pl. m. r. ptc. prez. akt. (objekt predikata θαυμάζουσιν); poimeničenje participa članom §~499: one koji imaju, one koji su stekli
\end{description}

%kraj
