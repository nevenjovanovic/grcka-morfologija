%Unesi korekcije NZ, NJ



\section*{O tekstu}

U prvoj od triju knjiga djela o umijeću uvjeravanja iz IV. st. pr.~Kr. Aristotel razmatra odnos retorike i dijalektike, daje definiciju retorike i uvodi trojaku podjelu sredstava uvjeravanja, koja se mogu temeljiti na karakteru govornika \textgreek[variant=ancient]{(ἦθος),} na osjećajima slušatelja \textgreek[variant=ancient]{(πάθος)} te na logičkom argumentu \textgreek[variant=ancient]{(λόγος).} Zatim, upravo u ovdje odabranom odlomku, donosi – ponovno trojaku – podjelu govorništva, s obzirom na to da se njime što savjetuje ili na što upozorava u skupštini (savjetodavno govorništvo), da se koga optužuje ili brani na sudu (sudsko govorništvo), odnosno hvali ili kudi u različitim prigodama (epideiktičko govorništvo). 

U nastavku prve knjige slijedi detaljnija rasprava o tim trima vrstama, osobito s obzirom na uporabu logičkih argumenata, dok se druga knjiga usredotočuje na govorničko izazivanje osjećaja kod slušatelja te na različite vrste karaktera. Treća se knjiga bavi pitanjima stila i dijelovima govora.


%\newpage

\section*{Pročitajte naglas grčki tekst.}
Arist.\ Rhetorica 1358a
%Naslov prema izdanju

\medskip

{\large
\begin{greek}
\noindent ῎Εστιν δὲ τῆς ῥητορικῆς εἴδη τρία τὸν ἀριθμόν· τοσοῦτοι γὰρ καὶ οἱ ἀκροαταὶ τῶν λόγων ὑπάρχουσιν ὄντες. σύγκειται μὲν γὰρ ἐκ τριῶν ὁ λόγος, ἔκ τε τοῦ λέγοντος καὶ περὶ οὗ λέγει καὶ πρὸς ὅν, καὶ τὸ τέλος πρὸς τοῦτόν ἐστιν, λέγω δὲ τὸν ἀκροατήν. ἀνάγκη δὲ τὸν ἀκροατὴν ἢ θεωρὸν εἶναι ἢ κριτήν, κριτὴν δὲ ἢ τῶν γεγενημένων ἢ τῶν μελλόντων. ἔστιν δ' ὁ μὲν περὶ τῶν μελλόντων κρίνων ὁ ἐκκλησιαστής, ὁ δὲ περὶ τῶν γεγενημένων [οἷον] ὁ δικαστής, ὁ δὲ περὶ τῆς δυνάμεως ὁ θεωρός, ὥστ' ἐξ ἀνάγκης ἂν εἴη τρία γένη τῶν λόγων τῶν ῥητορικῶν, συμβουλευτικόν, δικανικόν, ἐπιδεικτικόν. συμβουλῆς δὲ τὸ μὲν προτροπή, τὸ δὲ ἀποτροπή· ἀεὶ γὰρ καὶ οἱ ἰδίᾳ συμβουλεύοντες καὶ οἱ κοινῇ δημηγοροῦντες τούτων θάτερον ποιοῦσιν. δίκης δὲ τὸ μὲν κατηγορία, τὸ δ' ἀπολογία· τούτων γὰρ ὁποτερονοῦν ποιεῖν ἀνάγκη τοὺς ἀμφισβητοῦντας. ἐπιδεικτικοῦ δὲ τὸ μὲν ἔπαινος τὸ δὲ ψόγος.

\end{greek}
}


%\newpage

\section*{Analiza i komentar}


%1

{\large
\begin{greek}
\noindent ῎Εστιν δὲ τῆς ῥητορικῆς εἴδη \\
\tabto{2em} τρία τὸν ἀριθμόν·\\
τοσοῦτοι γὰρ καὶ οἱ ἀκροαταὶ τῶν λόγων \\
\tabto{2em} ὑπάρχουσιν ὄντες.\\

\end{greek}
}

\begin{description}[noitemsep]
\item[ἔστιν] εἰμί biti; 3. l. sg. ind. prez.; §~361; §~315, bilj. 2, 3
\item[τῆς ῥητορικῆς] §~90
\item[εἴδη] §~153
\item[τρία] §~224
\item[τὸν ἀριθμόν] §~82; akuzativ obzira §~389
\item[τοσοῦτοι] §~224
\item[οἱ ἀκροαταὶ] §~100
\item[τῶν λόγων] §~82
\item[ὑπάρχουσιν] ὑπάρχω poduzeti, početi; 3. l. pl. ind. prez. akt.; LSJ ὑπάρχω B.5 s ptc. ima istu vrijednost kao τυγχάνω s ptc: postoje\dots
\item[ὄντες] εἰμί biti; n. pl. m. r. ptc. prez.

\end{description}

{\large
\begin{greek}
\noindent σύγκειται μὲν γὰρ ἐκ τριῶν ὁ λόγος, \\
\tabto{2em} ἔκ τε τοῦ λέγοντος \\
\tabto{3em} καὶ περὶ οὗ λέγει \\
\tabto{3em} καὶ πρὸς ὅν, \\
\tabto{3em} καὶ τὸ τέλος \\
\tabto{4em} πρὸς τοῦτόν ἐστιν, \\
\tabto{6em} λέγω δὲ τὸν ἀκροατήν.\\

\end{greek}
}

\begin{description}[noitemsep]
\item[σύγκειται] σύγκειμαι ἔκ τινος sastojati se od čega; 3. l. sg. ind. prez. medpas.
\item[ἐκ τριῶν] §~224
\item[ὁ λόγος] §~82
\item[τοῦ λέγοντος] λέγω reći, govoriti; g. sg. m. r. ptc. prez. akt.; poimeničenje članom §~373
\item[περὶ οὗ ] o čemu; §~433.A; §~215
\item[λέγει] λέγω reći, govoriti; 3. l. sg. ind. prez. akt.
\item[πρὸς ὅν] komu; §~435.C.c; §~215
\item[τὸ τέλος] §~153
\item[πρὸς τοῦτόν ] §~435.C.c; §~213
\item[ἐστιν] εἰμί biti; 3. l. sg. ind. prez.; §~39, §~40
\item[λέγω] λέγω reći, govoriti; ovdje „mislim'', „to jest''; 1. l. sg. ind. prez. akt.
\item[τὸν ἀκροατήν] §~100

\end{description}

%3 itd
{\large
\begin{greek}
\noindent ἀνάγκη δὲ \\
\tabto{2em} \underline{τὸν ἀκροατὴν} \\
\tabto{4em} \underline{ἢ θεωρὸν εἶναι} \\
\tabto{4em} \underline{ἢ κριτήν}, \\
\tabto{6em} \underline{κριτὴν} δὲ \\
\tabto{8em} ἢ τῶν γεγενημένων \\
\tabto{8em} ἢ τῶν μελλόντων.

\end{greek}
}

\begin{description}[noitemsep]
\item[ἀνάγκη] §~90; uz imenicu se podrazumijeva kopula ἐστί iako nije izrečena; uz izraz u značenju „potrebno je'', „nužno je'' slijedi A+I
\item[τὸν ἀκροατήν] §~100
\item[ἢ\dots\ ἢ] ili\dots\ ili
\item[θεωρὸν] §~82
\item[εἶναι] εἰμί biti; inf. prez.
\item[κριτήν] §~100
\item[τῶν γεγενημένων] γίγνομαι biti, postati, dogoditi se; g. pl. s. r. ptc. perf. medpas.; poimeničeni particip τὰ γεγενημένα ono što se dogodilo, prošlost
\item[τῶν μελλόντων] μέλλω namjeravati, trebati, htjeti; g. pl. sr. r. ptc. prez. akt.; poimeničeni particip τὰ μέλλοντα ono što će se dogoditi, budućnost

\end{description}

%4
{\large
\begin{greek}
\noindent ἔστιν δ' \\
\tabto{2em} ὁ μὲν περὶ τῶν μελλόντων κρίνων \\
\tabto{4em} ὁ ἐκκλησιαστής, \\
\tabto{2em} ὁ δὲ περὶ τῶν γεγενημένων \\
\tabto{4em} [οἷον] ὁ δικαστής, \\
\tabto{2em} ὁ δὲ περὶ τῆς δυνάμεως \\
\tabto{4em} ὁ θεωρός, \\
ὥστ' ἐξ ἀνάγκης ἂν εἴη \\
\tabto{2em} τρία γένη τῶν λόγων τῶν ῥητορικῶν, \\
\tabto{4em} συμβουλευτικόν, δικανικόν, ἐπιδεικτικόν.\\

\end{greek}
}

\begin{description}[noitemsep]
\item[ἔστιν] εἰμί biti; 3. l. sg. ind. prez.; §~315, bilj. 2, 3
\item[ὁ μὲν\dots\ ὁ δὲ\dots\ ὁ δὲ\dots] koordinacija rečeničnih članova s pomoću čestica μέν\dots\ δέ\dots
\item[ὁ\dots\ κρίνων] κρίνω prosuđivati; n. sg. m. r. ptc. prez. akt.; poimeničenje članom §~373
\item[περὶ τῶν μελλόντων] §~433.A; μέλλω namjeravati, trebati, htjeti; g. pl. s. r. ptc. prez. akt.
\item[ὁ ἐκκλησιαστής] §~100
\item[ὁ δὲ] sc.\ κρίνων
\item[περὶ τῶν γεγενημένων] §~433.A; γίγνομαι biti, postati, dogoditi se; g. pl. s. r. ptc. perf. medpas.
\item[ὁ δικαστής] §~100
\item[ὁ δὲ] sc.\ κρίνων
\item[περὶ τῆς δυνάμεως] §~433.A; §~165; o sposobnosti (sc.\ τοῦ λέγοντος), LSJ δύναμις II.1
\item[ὁ θεωρός] §~82
\item[ὥστ'] ὥστε veznik u posljedičnoj rečenici §~473
\item[ἐξ ἀνάγκης] nužno
\item[ἂν ] čestica koja uz optativ pokazuje mogućnost u sadašnjosti §~464.2
\item[εἴη] εἰμί: biti; 3. l. sg. opt. prez.; prevedite: tako da nužno moraju postojati
\item[τρία] §~224
\item[γένη] §~153
\item[τῶν λόγων] §~82
\item[τῶν ῥητορικῶν] §~103
\item[συμβουλευτικόν] §~103
\item[δικανικόν] §~103
\item[ἐπιδεικτικόν] §~103

\end{description}

%5
{\large
\begin{greek}
\noindent συμβουλῆς δὲ \\
\tabto{2em} τὸ μὲν προτροπή, \\
\tabto{2em} τὸ δὲ ἀποτροπή· \\
\tabto{4em} ἀεὶ γὰρ \\
\tabto{6em} καὶ οἱ ἰδίᾳ συμβουλεύοντες \\
\tabto{6em} καὶ οἱ κοινῇ δημηγοροῦντες \\
\tabto{4em} τούτων θάτερον ποιοῦσιν.\\

\end{greek}
}

\begin{description}[noitemsep]
\item[συμβουλῆς] §~90
\item[τὸ μὲν\dots\ τὸ δὲ\dots] jedno\dots\ a drugo\dots, koordinacija s pomoću čestica μέν\dots\ δέ\dots; podrazumijeva se kopula ἐστίν
\item[προτροπή] §~103
\item[ἀποτροπή] §~103
\item[οἱ\dots\ συμβουλεύοντες] συμβουλεύω savjetovati; n. pl. m. r. ptc. prez. akt.
\item[ἰδίᾳ] §~103; ovdje priložno: privatno
\item[οἱ\dots\ δημηγοροῦντες] δημηγορέω govoriti javno, u narodnoj skupštini; n. pl. m. r. ptc. prez. akt.
\item[κοινῇ] §~103; ovdje priložno: javno
\item[τούτων] §~213; dijelni genitiv §~395
\item[θάτερον] §~103; atički oblik uz ἕτερος, ovdje τούτων θάτερον jedno od toga dvoga
\item[ποιοῦσιν] ποιέω činiti; 3. l. pl. ind. prez. akt.

\end{description}

%6
{\large
\begin{greek}
\noindent δίκης δὲ \\
\tabto{2em} τὸ μὲν κατηγορία, \\
\tabto{2em} τὸ δ' ἀπολογία· \\
\tabto{4em} τούτων γὰρ ὁποτερονοῦν \underline{ποιεῖν} \\
\tabto{4em} ἀνάγκη \\
\tabto{4em} \underline{τοὺς ἀμφισβητοῦντας}.\\

\end{greek}
}

\begin{description}[noitemsep]
\item[δίκης] §~90
\item[τὸ μὲν\dots\ τὸ δ'\dots] jedno\dots\ a drugo\dots; koordinacija s pomoću čestica μέν\dots\ δέ\dots; podrazumijeva se kopula ἐστίν
\item[κατηγορία] §~90
\item[ἀπολογία] §~90
\item[τούτων] §~213; dijelni genitiv §~395
\item[ὁποτερονοῦν] korelativne zamjenice §~219; dodavanje čestice οὖν §~222.3
\item[ποιεῖν] ποιέω činiti; inf. prez. akt.
\item[ἀνάγκη] §~90; izraz ἀνάγκη (ἐστίν) uvodi A+I
\item[τοὺς ἀμφισβητοῦντας] ἀμφισβητέω prepirati se, sporiti se; poimeničeni particip οἱ ἀμφισβητοῦντες parničari; a. pl. m. r. ptc. prez. akt.

\end{description}

%7
{\large
\begin{greek}
\noindent ἐπιδεικτικοῦ δὲ \\
\tabto{2em} τὸ μὲν ἔπαινος \\
\tabto{2em} τὸ δὲ ψόγος.\\

\end{greek}
}

\begin{description}[noitemsep]
\item[ἐπιδεικτικοῦ] §~103
\item[τὸ μὲν\dots\ τὸ δ'\dots] jedno\dots\ a drugo\dots; koordinacija s pomoću čestica μέν\dots\ δέ\dots; podrazumijeva se kopula ἐστίν
\item[ἔπαινος] §~82
\item[ψόγος] §~82

\end{description}


%kraj
