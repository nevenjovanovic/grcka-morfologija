% redaktura NZ, unio NJ
%\section*{O autoru}



\section*{O tekstu}

Rasprava o praktičnoj etici \textit{Demoniku} \textgreek[variant=ancient]{(Πρὸς Δημόνικον)} ima oblik otvorenog pisma. Pripisuje se Izokratu (436.–338.\ p.~n.~e), no nije isključeno da joj je autor neki njegov učenik. Nudeći upute za svakodnevni život, u raspravi se razlikuju tri životna područja u kojima ta pravila treba primjenjivati: 1) čovjek u odnosu prema bogovima, 2) čovjek u odnosu prema drugim ljudima, ponajprije prema roditeljima i prijateljima, 3) čovjek u odnosu prema sebi samom i skladan razvoj njegova karaktera. Pravila se nižu bez strogog reda, što je tipično za ``gnomsku'' književnost V.–IV.~st. U tom smislu rasprava se može usporediti s pjesničkim opusom Megaranina Teognida (VI.~st.\ p.~n.~e).

%\newpage

\section*{Pročitajte naglas grčki tekst.}

%Naslov prema izdanju

Isoc.\ Ad Demonicum 16

\medskip

{\large
\begin{greek}
\noindent Μηδέποτε μηδὲν αἰσχρὸν ποιήσας ἔλπιζε λήσειν· καὶ γὰρ ἂν τοὺς ἄλλους λάθῃς, σεαυτῷ συνειδήσεις. Τοὺς μὲν θεοὺς φοβοῦ, τοὺς δὲ γονεῖς τίμα, τοὺς δὲ φίλους αἰσχύνου, τοῖς δὲ νόμοις πείθου. Τὰς ἡδονὰς θήρευε τὰς μετὰ δόξης· τέρψις γὰρ σὺν τῷ καλῷ μὲν ἄριστον, ἄνευ δὲ τούτου κάκιστον.  Εὐλαβοῦ τὰς διαβολὰς, κἂν ψευδεῖς ὦσιν· οἱ γὰρ πολλοὶ τὴν μὲν ἀλήθειαν ἀγνοοῦσιν, πρὸς δὲ τὴν δόξαν ἀποβλέπουσιν. Ἅπαντα δόκει ποιεῖν ὡς μηδένα λήσων· καὶ γὰρ ἂν παραυτίκα κρύψῃς, ὕστερον ὀφθήσει. Μάλιστα δ' ἂν εὐδοκιμοίης, εἰ φαίνοιο ταῦτα μὴ πράττων ἃ τοῖς ἄλλοις ἂν πράττουσιν ἐπιτιμῴης. Ἐὰν ᾖς φιλομαθὴς, ἔσει πολυμαθής. Ἃ μὲν ἐπίστασαι, ταῦτα διαφύλαττε ταῖς μελέταις, ἃ δὲ μὴ μεμάθηκας, προσλάμβανε ταῖς ἐπιστήμαις· ὁμοίως γὰρ αἰσχρὸν ἀκούσαντα χρήσιμον λόγον μὴ μαθεῖν καὶ διδόμενόν τι ἀγαθὸν παρὰ τῶν φίλων μὴ λαβεῖν.
 
\end{greek}

}

\section*{Analiza i komentar}

%1

{\large
\noindent Μηδέποτε \\
\tabto{2em} μηδὲν αἰσχρὸν \\
\tabto{2em} ποιήσας \\
ἔλπιζε \\
\tabto{2em} λήσειν·\\
καὶ γὰρ ἂν \\
\tabto{2em} τοὺς ἄλλους \\
\tabto{2em} λάθῃς, \\
σεαυτῷ συνειδήσεις.\\

}

\begin{description}[noitemsep]

\item[Μηδέποτε] §~509.2
\item[μηδὲν αἰσχρὸν] §~224, §~103
\item[ποιήσας] ποιέω činiti; n. sg. m. r. ptc. aor. akt; predikatni particip, §~501.b
\item[ἔλπιζε] ἐλπίζω nadati se, očekivati, otvara mjesto dopuni u infinitivu; imper. prez. akt.
\item[λήσειν] λανθάνω; inf. fut. akt
\item[ἂν] = ἐάν = εἰ ἄν; veznik uvodi zavisnu pogodbenu rečenicu, ovdje eventualnog futurskog oblika (konjunktiv u protazi, futur u apodozi)
\item[τοὺς ἄλλους] §~212, supstantiviranje članom §~371 B.4
\item[λάθῃς] λανθάνω τινά ostati neopažen od koga; 2. l. sg. konj. aor. akt. (protaza eventualne pogodbene rečenice)
\item[σεαυτῷ] §~208
\item[συνειδήσεις] σύνοιδα biti svjestan, σεαυτῷ samoga sebe, tj.\ vlastitog čina; 2. l. pl. ind. fut. akt.
\end{description}

{\large
\noindent Τοὺς μὲν θεοὺς φοβοῦ, \\
τοὺς δὲ γονεῖς τίμα, \\
τοὺς δὲ φίλους αἰσχύνου, \\
τοῖς δὲ νόμοις πείθου.\\

}

\begin{description}[noitemsep]
\item[Τοὺς μὲν\dots\ τοὺς δὲ\dots\ τοὺς δὲ\dots\ τοῖς δὲ\dots] koordinacija rečeničnih članova pomoću čestica: a\dots\ a\dots\ a\dots
\item[Τοὺς\dots\ θεοὺς] §~82
\item[φοβοῦ] φοβέομαι bojati se; 2. l. sg. imper. prez. medpas.
\item[τοὺς\dots\ γονεῖς] §~175
\item[τίμα] τιμάω cijeniti; 2. l. sg. imper. prez. akt.
\item[τοὺς\dots\ φίλους] §~82
\item[αἰσχύνου] αἰσχύνομαι poštovati, osjećati sram pred kim, rekcija τινά, LSJ αἰσχύνω B.II.3 c; 2. l. sg. imper. prez. medpas.
\item[τοῖς\dots\ νόμοις] §~82
\item[πείθου] πείθομαι pokoravati se; 2. l. sg. imper. prez. medpas.
\end{description}

%3 

{\large
\noindent Τὰς ἡδονὰς θήρευε \\
\tabto{2em} τὰς μετὰ δόξης· \\
τέρψις γὰρ \\
\tabto{2em} σὺν τῷ καλῷ μὲν \\
\tabto{4em} ἄριστον, \\
\tabto{2em} ἄνευ δὲ τούτου \\
\tabto{4em} κάκιστον.\\

}

\begin{description}[noitemsep]

\item[Τὰς ἡδονὰς] §~90
\item[θήρευε] θηρεύω loviti, za čim ići; 2. l. sg. imper. prez. akt.
\item[τὰς μετὰ δόξης] atributni položaj, §~375
\item[μετὰ] §~430
\item[δόξης] §~90
\item[τέρψις] §~165
\item[γὰρ] čestica najavljuje iznošenje razloga ili objašnjenja: naime\dots, §~517
\item[σὺν τῷ καλῷ μὲν\dots\ ἄνευ δὲ τούτου\dots] koordinacija suprotstavljenih rečeničnih članova pomoću para čestica: a\dots
\item[σὺν] §~427
\item[τῷ καλῷ] §~103, supstantiviranje članom §~373
\item[ἄριστον] §~202; imenski predikat Smyth 909, kopula ἐστι ovdje je izostavljena
\item[ἄνευ] §~417
\item[τούτου] sc.\ τοῦ καλοῦ; §~213.2
\item[κάκιστον] §~202; imenski predikat Smyth 909, kopula ἐστι ovdje je izostavljena
\end{description}

%4

{\large
\noindent Εὐλαβοῦ τὰς διαβολὰς, \\
\tabto{2em} κἂν ψευδεῖς ὦσιν· \\
οἱ γὰρ πολλοὶ \\
\tabto{2em} τὴν μὲν ἀλήθειαν ἀγνοοῦσιν, \\
\tabto{2em} πρὸς δὲ τὴν δόξαν ἀποβλέπουσιν.\\

}

\begin{description}[noitemsep]

\item[Εὐλαβοῦ] εὐλαβέομαι čuvati se; 2. l. sg. imper. prez. medpas.
\item[τὰς διαβολὰς] §~90
\item[κἂν = καὶ εἰ ἄν] §~16; veznik uvodi zavisnu pogodbenu rečenicu, ovdje eventualnog futurskog oblika (imperativ u apodozi, konjunktiv u protazi)
\item[ψευδεῖς ὦσιν] imenski predikat Smyth 909; protaza pogodbene rečenice
\item[ψευδεῖς] §~153
\item[ὦσιν] εἰμί biti; 3. l. pl. konj. prez. akt.
\item[οἱ\dots\ πολλοὶ] §~196, §~371 B.4
\item[γὰρ] čestica najavljuje iznošenje razloga ili objašnjenja: naime\dots, §~517
\item[τὴν μὲν ἀλήθειαν\dots\ πρὸς δὲ τὴν δόξαν\dots] koordinacija suprotstavljenih rečeničnih članova pomoću para čestica: a\dots
\item[τὴν\dots\ ἀλήθειαν] §~90
\item[ἀγνοοῦσιν] ἀγνοέω ne znati; 3. l. pl. ind. prez. akt.
\item[πρὸς] §~435 C
\item[τὴν δόξαν] §~90
\item[ἀποβλέπουσιν] ἀποβλέπω πρός τι paziti na nešto; 3. l. pl. ind. prez. akt.
\end{description}

%5

{\large
\noindent \tabto{2em} Ἅπαντα\\
δόκει\\
\tabto{2em} ποιεῖν \\
\tabto{4em} ὡς μηδένα λήσων· \\
\tabto{2em} καὶ γὰρ\\
\tabto{4em} ἂν παραυτίκα κρύψῃς, \\
\tabto{4em} ὕστερον ὀφθήσει.\\

}

\begin{description}[noitemsep]

\item[Ἅπαντα] §~193
\item[δόκει] δοκέω misliti; 2. l. sg. imper. prez. akt.; otvara mjesto dopuni u infinitivu
\item[ποιεῖν] ποιέω činiti; inf. prez. akt.
\item[ὡς] otvara mjesto adverbnom participu koji izriče (subjektivni) uzrok, §~503b: kao da\dots
\item[μηδένα] §~224
\item[λήσων] λανθάνω τινά ostati neopažen od koga; n. sg. m. r. ptc. fut. akt.
\item[καὶ γὰρ] kombinacija čestica: naime i\dots
\item[ἂν] = ἐάν = εἰ ἄν; veznik uvodi zavisnu pogodbenu rečenicu, ovdje eventualnog futurskog oblika (konjunktiv u protazi, futur u apodozi)
\item[κρύψῃς] κρύπτω kriti; 2. l. sg. konj. aor. akt. (protaza pogodbene rečenice)
\item[ὀφθήσει] ὁράω vidjeti; 2. l. sg. ind. fut. pas. (apodoza pogodbene rečenice)
\end{description}

%6

{\large
\noindent Μάλιστα δ' ἂν εὐδοκιμοίης, \\
εἰ φαίνοιο \\
\tabto{4em} ταῦτα \\
\tabto{2em} μὴ πράττων \\
\tabto{4em} ἃ \\
\tabto{6em} τοῖς ἄλλοις ἂν πράττουσιν \\
\tabto{4em} ἐπιτιμῴης. \\

}

\begin{description}[noitemsep]
\item[δ'] = δέ, čestica označava nadovezivanje na prethodni iskaz, §~515
\item[ἂν εὐδοκιμοίης, εἰ φαίνοιο] pogodbena rečenica potencijalnog oblika: u apodozi ἄν + optativ, u protazi εἰ + optativ, §~477: bi\dots, ako bi\dots
\item[εὐδοκιμοίης] εὐδοκιμέω na dobrom glasu biti; 2. l. sg. opt. prez. akt.; apodoza potencijalne pogodbene rečenice
\item[φαίνοιο] φαίνομαι pojavljivati se; 2. l. sg. opt. prez. medpas.; protaza potencijalne pogodbene rečenice; kopulativni glagol (glagol nepotpuna značenja) otvara mjesto predikatnom participu, §~501.b
\item[ταῦτα] §~213.2
\item[πράττων] πράττω činiti; n. sg. m. r. ptc. prez. akt.; predikatni particip kao dopuna uz φαίνοιο, §~501.b; negacija μή stoji uz particip kojim se izriče mogućnost (pogodbeno) Smyth 2728
\item[ἃ] §~215; uvodi zavisnu odnosnu rečenicu, antecedent je ταῦτα
\item[τοῖς ἄλλοις] §~212, §~371 B.4
\item[ἄν πράττουσιν] πράττω činiti; d. pl. m. r. ptc. prez. akt; ἄν uz particip označava potencijalno značenje §~506: ako\dots
\item[ἐπιτιμῴης] ἐπιτιμάω τινί predbacivati nekome; 2. l. sg. opt. prez. akt.
\end{description}

{\large
\noindent Ἐὰν ᾖς φιλομαθὴς, \\
ἔσει πολυμαθής.\\

}

\begin{description}[noitemsep]

\item[Ἐὰν] = εἰ ἄν; veznik uvodi zavisnu pogodbenu rečenicu, ovdje eventualnog futurskog oblika (konjunktiv u protazi, futur u apodozi)
\item[ᾖς φιλομαθὴς\dots\ ἔσει πολυμαθής] imenski predikati Smyth 909
\item[ᾖς] εἰμί biti; 2. l. sg. konj. prez. 
\item[φιλομαθὴς] §~153
\item[ἔσει] εἰμί biti; 2. l. sg. fut. (med.)
\item[πολυμαθής] §~153
\end{description}

%7

{\large
\noindent Ἃ μὲν ἐπίστασαι, \\
\tabto{2em} ταῦτα διαφύλαττε \\
\tabto{4em} ταῖς μελέταις, \\
ἃ δὲ μὴ μεμάθηκας, \\
\tabto{2em} προσλάμβανε \\
\tabto{4em} ταῖς ἐπιστήμαις· \\
ὁμοίως γὰρ αἰσχρὸν \\
\tabto{2em} \underline{ἀκούσαντα} χρήσιμον λόγον \\
\tabto{2em} \underline{μὴ μαθεῖν} \\
\tabto{2em} καὶ \underline{διδόμενόν} τι ἀγαθὸν \\
\tabto{4em} παρὰ τῶν φίλων \\
\tabto{2em} \underline{μὴ λαβεῖν}.\\

}

\begin{description}[noitemsep]
\item[Ἃ μὲν ἐπίστασαι\dots\ ἃ δὲ μὴ μεμάθηκας\dots] koordinacija značenjski suprotstavljenih rečeničnih članova parom čestica: a\dots
\item[Ἃ] §~215; uvodi zavisnu odnosnu rečenicu, antecedent je ταῦτα
\item[ἐπίστασαι] ἐπίσταμαι znati; 2. l. sg. ind. prez. medpas.
\item[ταῦτα] §~213.2
\item[διαφύλαττε] διαφυλάττω neprestano čuvati; 2. l. sg. imper. prez. akt.
\item[ταῖς μελέταις] §~90, dativ izriče sredstvo (dativus instrumenti §~414)
\item[ἃ] §~215; uvodi zavisnu odnosnu rečenicu koja ima službu objekta προσλάμβανε: ono što\dots
\item[μὴ μεμάθηκας] μανθάνω učiti; 2. l. sg. ind. perf. akt.; negacija μή stoji u odnosnim rečenicama s neodređenim antecedentom, koje izriču tipičan slučaj (generičko μή) Smyth 2705
\item[προσλάμβανε] προσλαμβάνω dodatno steći; 2. l. sg. imper. prez. akt.
\item[ταῖς ἐπιστήμαις] §~90, dativ izriče sredstvo (dativus instrumenti §~414)
\item[γὰρ] čestica najavljuje iznošenje razloga ili objašnjenja: naime\dots, §~517
\item[αἰσχρὸν] §~103; sc.\ ἐστίν (kopula je ovdje neizrečena), imenski predikat Smyth 909; izraz otvara mjesto dvama A+I (ἀκούσαντα\dots\ μὴ μαθεῖν, διδόμενόν\dots\ μὴ λαβεῖν)
\item[ἀκούσαντα] ἀκούω čuti; a. sg. m. r. ptc. aor. akt.
\item[χρήσιμον λόγον] §~103, §~82
\item[μαθεῖν] μανθάνω učiti; inf. aor. akt.; negacija μή uz infinitiv §~509c
\item[διδόμενόν] δίδωμι davati; a. sg. m. r. ptc. prez. medpas.
\item[τι ἀγαθὸν] §~217, §~103
\item[παρὰ] §~434 A
\item[τῶν φίλων] §~82
\item[λαβεῖν] λαμβάνω uzimati; inf. aor. akt.; negacija μή uz infinitiv §~509c
\end{description}



%kraj
