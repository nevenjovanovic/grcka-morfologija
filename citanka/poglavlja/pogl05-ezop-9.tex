% Unesi korekture NČ 2019-09-07
%\section*{O autoru}

\section*{O tekstu}

U ovoj Ezopovoj basni susrećemo lisicu u nevolji i jarca. Valjano promislivši o poteškoći koja ju je zadesila, lisica se spašava, dok jarac zbog nepromišljenosti u nevolju upada. Više je pouka basne: prije nego što odlučimo započeti nešto, razmislimo dobro o tome kako ćemo to završiti i hoćemo li u tome biti uspješni. Nadalje, ako nismo razboriti i ne pazimo te poslušamo li savjet loše osobe, lako se može dogoditi da završimo u velikoj nevolji. Mudar čovjek u postupcima ne sluša osjetila, već se oslanja na svoj razum, promišljajući kako da najbolje postupi.

%\newpage

\section*{Pročitajte naglas grčki tekst.}

Aesop.\ Fabulae 9

%Naslov prema izdanju

\medskip

\begin{greek}
{\large
{ \noindent ΑΛΩΠΗΞ ΚΑΙ ΤΡΑΓΟΣ 

\medskip

\noindent Ἀλώπηξ πεσοῦσα εἰς φρέαρ ἐπάναγκες ἔμενε πρὸς τὴν ἀνάβασιν ἀμηχανοῦσα. τράγος δὲ δίψῃ συνεχόμενος ὡς  ἐγένετο κατὰ τὸ αὐτὸ φρέαρ, θεασάμενος αὐτὴν ἐπυνθάνετο, εἰ καλὸν εἴη τὸ ὕδωρ. ἡ δὲ τὴν συντυχίαν ἀσμενισαμένη πολὺν ἔπαινον τοῦ ὕδατος κατέτεινε λέγουσα ὡς χρηστὸν εἴη καὶ δὴ καὶ αὐτὸν καταβῆναι παρῄνει. τοῦ δὲ ἀμελετήτως καθαλλομένου διὰ τὸ μόνην ὁρᾶν τότε τὴν ἐπιθυμίαν καὶ ἅμα τῷ τὴν δίψαν σβέσαι ἀναδῦναι μετὰ τῆς ἀλώπεκος σκοποῦντος χρήσιμόν τι ἡ ἀλώπηξ ἔφη ἐπινενοηκέναι εἰς τὴν ἀμφοτέρων σωτηρίαν. 

\noindent ``ἐὰν γὰρ θελήσῃς τοὺς ἐμπροσθίους πόδας τῷ τοίχῳ προσερείσας ἐγκλῖναι καὶ τὰ κέρατα, ἀναδραμοῦσα αὐτὴ διὰ τοῦ σοῦ νώτου καὶ σὲ ἀνασπάσω.'' τοῦ δὲ καὶ πρὸς τὴν δευτέραν παραίνεσιν ἑτοίμως ὑπηρετήσαντος ἡ ἀλώπηξ ἀναλλομένη διὰ τῶν σκελῶν αὐτοῦ ἐπὶ τὸν νῶτον ἀνέβη καὶ ἀπ' ἐκείνου ἐπὶ τὰ κέρατα διερεισαμένη ἐπὶ τὸ στόμα τοῦ φρέατος ηὑρέθη καὶ ἀνελθοῦσα ἀπηλλάττετο. 

\noindent τοῦ δὲ τράγου μεμφομένου αὐτὴν ὡς τὰς ὁμολογίας παραβαίνουσαν ἡ ἀλώπηξ ἐπιστραφεῖσα εἶπεν· ``ὦ οὗτος, ἀλλ' εἰ τοσαύτας φρένας εἶχες, ὅσας ἐν τῷ πώγωνι τρίχας, οὐ πρότερον ἂν καταβεβήκεις πρὶν ἢ τὴν ἄνοδον ἐσκέψω.''

}
}
\end{greek}

\section*{Analiza i komentar}

%1

{\large
\begin{greek}
\noindent ΑΛΩΠΗΞ ΚΑΙ ΤΡΑΓΟΣ\\

\end{greek}
}

\begin{description}[noitemsep]
\item[ΑΛΩΠΗΞ] §~115
\item[ΤΡΑΓΟΣ] §~82
\end{description}

%2

{\large
\begin{greek}
\noindent Ἀλώπηξ \\
\tabto{2em} πεσοῦσα εἰς φρέαρ \\
ἐπάναγκες ἔμενε \\
\tabto{2em} πρὸς τὴν ἀνάβασιν \\
\tabto{4em} ἀμηχανοῦσα. \\

\end{greek}
}

\begin{description}[noitemsep]
\item[Ἀλώπηξ] §~115
\item[πεσοῦσα] πίπτω pasti; n. sg. ž. r. ptc. aor. akt.; §~97.β
\item[εἰς φρέαρ] §~123; prijedložni izraz εἰς + a. §~418, §~419
\item[ἔμενε] μένω ostati, čekati; 3. l. sg. impf. akt.
\item[πρὸς τὴν ἀνάβασιν] §~165; prijedložni izraz πρὸς + a.: ``za\dots''  §~418, §~435.4
\item[ἀμηχανοῦσα] ἀμηχανέω ne znati što učiniti; n. sg. ž. r. ptc. prez. akt. §~97.β

\end{description}

%3

{\large
\begin{greek}
\noindent τράγος δὲ\\
\tabto{2em} δίψῃ συνεχόμενος \\
\tabto{4em} ὡς ἐγένετο κατὰ τὸ αὐτὸ φρέαρ, \\
θεασάμενος αὐτὴν \\
ἐπυνθάνετο, \\
\tabto{2em} εἰ καλὸν εἴη τὸ ὕδωρ.\\

\end{greek}
}

\begin{description}[noitemsep]
\item[δὲ] a\dots; čestica δέ povezuje rečenicu s prethodnom kao suprotni veznik §~515
\item[τράγος] §~82
\item[δίψῃ] §~97.β
\item[συνεχόμενος] συνέχω biti mučen; n. sg. m. r. ptc. prez. medpas.; §~103
\item[ἐγένετο] γίγνομαι zadesiti se, biti; 3. l. sg. ind. aor. med.
\item[ὡς\dots\ ἐγένετο] kad\dots; zavisna vremenska rečenica
\item[κατὰ τὸ αὐτὸ φρέαρ] §~80; §~207; §~123; prijedložni izraz κατὰ + a.: „kod\dots“ §~418; §~429.B
\item[θεασάμενος] θεάομαι ugledati; n. sg. m. r. ptc. aor. med.; §~103
\item[αὐτὴν] §~207
\item[ἐπυνθάνετο] πυνθάνομαι pitati; 3. l. sg. impf. medpas.
\item[εἰ] zavisni upitni veznik: ``je li\dots?''
\item[καλὸν] §~103; §~200
\item[εἴη] εἰμί biti; 3. l. sg. opt. prez. akt.
\item[καλὸν εἴη] imenski predikat, Smyth 909
\item[εἰ\dots\ καλὸν εἴη] je li\dots? zavisna upitna rečenica §~469
\item[τὸ ὕδωρ] §~128
\end{description}

%4
{\large
\begin{greek}
\noindent ἡ δὲ \\
\tabto{2em} τὴν συντυχίαν ἀσμενισαμένη \\
πολὺν ἔπαινον τοῦ ὕδατος \\
κατέτεινε \\
λέγουσα \\
\tabto{2em} ὡς χρηστὸν εἴη \\
καὶ δὴ καὶ \\
\tabto{2em} \underline{αὐτὸν καταβῆναι} παρῄνει. \\

\end{greek}
}

\begin{description}[noitemsep]
\item[ἡ δὲ] §~80, §~90; §~370, §~370.2
\item[τὴν συντυχίαν] §~80; §~90
\item[ἀσμενισαμένη ] ἀσμενίζω rado prihvatiti; n. sg. ž. r. ptc. aor. med.; §~103
\item[πολὺν ἔπαινον] §~196; §~82
\item[τοῦ ὕδατος] §~128
\item[κατέτεινε] κατατείνω ἔπαινον držati pohvalni govor; 3. l. sg. impf. akt. 
\item[λέγουσα] λέγω govoriti; n. sg. ž. r. ptc. prez. akt.; §~97.β 
\item[χρηστὸν] §~103
\item[εἴη] εἰμί biti; 3. l. sg. opt. prez. akt.
\item[ὡς\dots\ χρηστὸν εἴη] da je\dots; zavisna izrična rečenica u neupravnom govoru
\item[χρηστὸν εἴη] imenski predikat, Smyth 909
\item[καὶ δὴ καὶ ] i napose\dots; kombinacija čestica prati ideju izrečenu prethodnim rečenicama i izriče njezinu kulminaciju
\item[παρῄνει] παραινέω preporučati; 3. l. sg. impf. akt. 
\item[αὐτὸν ] §~207; imenski dio A+I
\item[καταβῆναι] καταβαίνω sići, spustiti se; inf. aor. akt.; predikatni dio A+I
\end{description}
%5

{\large
\begin{greek}
\noindent \uuline{τοῦ} δὲ ἀμελετήτως \uuline{καθαλλομένου} \\
\tabto{2em} διὰ τὸ μόνην ὁρᾶν τότε τὴν ἐπιθυμίαν \\
καὶ ἅμα τῷ τὴν δίψαν σβέσαι \\
ἀναδῦναι μετὰ τῆς ἀλώπεκος \\
\tabto{2em} \uuline{σκοποῦντος} \\
χρήσιμόν τι \\
ἡ ἀλώπηξ ἔφη \\
ἐπινενοηκέναι \\
\tabto{2em} εἰς τὴν ἀμφοτέρων σωτηρίαν.\\

\end{greek}
}

\begin{description}[noitemsep]
\item[τοῦ καθαλλομένου] καθάλλομαι skočiti dolje; g. sg. m. r. ptc. prez. medpas.; §~103; particip s članom §~499; GA; imenski dio kao subjekt, particip kao predikat adverbne zavisne rečenice
\item[δὲ] a\dots; čestica povezuje rečenicu s prethodnom kao suprotni veznik §~515
\item[τὸ ὁρᾶν] ὁράω gledati; inf. prez. akt.; infinitiv s članom §~497
\item[διὰ τὸ ὁρᾶν] prijedložni izraz, διὰ + a. ``zbog\dots''; §~418, §~428.B
\item[μόνην τὴν ἐπιθυμίαν] §~103; §~80; §~97
\item[ἅμα ] čim, tek što; prilog vremena
\item[τῷ ] §~80, §~82, §~370
\item[σβέσαι] σβέννυμι ugasiti, utoliti; inf. aor. akt.; infinitiv s članom §~497
\item[τὴν δίψαν] §~80, §~97.β
\item[μετὰ τῆς ἀλώπεκος ] §~80; §~90; §~115; prijedložni izraz μετὰ + g.: ``s\dots''; §~418, §~430.A
\item[σκοποῦντος ] σκοπέω razmišljati, razmatrati; g. sg. m. r. ptc. prez. akt.; §~139.α 
\item[τι ] §~217, §~218
\item[χρήσιμόν τι] §~39, §~40
\item[χρήσιμον] §~103
\item[ἀναδῦναι] ἀναδύνω izaći; inf. aor. akt.; infinitiv poslije glagola mišljenja ima vrijednost hrvatske izrične rečenice ``da izađu''
\item[ἡ ἀλώπηξ ] §~80, §~115
\item[ἔφη] φημί reći; 3. l. sg. impf. akt. 
\item[ἐπινενοηκέναι] ἐπινοέω imati na umu, smisliti; inf. perf. akt.;  infinitiv poslije glagola govorenja ima vrijednost hrvatske izrične rečenice
\item[εἰς τὴν ἀμφοτέρων σωτηρίαν] §~80, §~103; §~90; prijedložni izraz εἰς + a.: ``za\dots'', §~418, §~419
\end{description}
%6

{\large
\begin{greek}
\noindent ``ἐὰν γὰρ θελήσῃς \\
\tabto{2em} τοὺς ἐμπροσθίους πόδας \\
\tabto{2em} τῷ τοίχῳ \\
\tabto{2em} προσερείσας ἐγκλῖναι \\
\tabto{2em} καὶ τὰ κέρατα, \\
ἀναδραμοῦσα αὐτὴ \\
\tabto{2em} διὰ τοῦ σοῦ νώτου \\
καὶ σὲ ἀνασπάσω.''\\

\end{greek}
}

\begin{description}[noitemsep]
\item[ἐὰν] ako\dots; veznik uvodi zavisnu pogodbenu eventualnu futursku rečenicu §~474, §~476
\item[γὰρ] naime\dots; čestica iskazuje dokaz ili uzrok prethodne tvrdnje
\item[θελήσῃς] θέλω (i ἐθέλω) htjeti; 2. l. sg. konj. aor. akt.
\item[τοὺς ἐμπροσθίους πόδας] §~80; §~103; §~126
\item[τῷ τοίχῳ] §~80, §~82
\item[προσερείσας] προσερείδω τινά τινι čvrsto se uprijeti čime o što; n. sg. m. r. ptc. aor. akt.; §~139.β
\item[ἐγκλῖναι] ἐγκλίνω nasloniti se; inf. aor. akt.
\item[τὰ κέρατα] §~159
\item[ἀναδραμοῦσα] ἀνατρέχω popeti se; n. sg. ž. r. ptc. aor. akt.; §~97.β
\item[αὐτὴ] §~207
\item[διὰ τοῦ σοῦ νώτου] §~80, §~210, §~82, prijedložni izraz διὰ + g. ``preko\dots''; §~418, §~428
\item[σὲ] §~205, §~206
\item[ἀνασπάσω] ἀνασπάω povući gore; 1. l. sg. ind. fut. akt.

\end{description}



%7

{\large
\begin{greek}
\noindent \uuline{τοῦ} δὲ καὶ πρὸς τὴν δευτέραν παραίνεσιν \\
ἑτοίμως \uuline{ὑπηρετήσαντος} \\
ἡ ἀλώπηξ ἀναλλομένη \\
\tabto{2em} διὰ τῶν σκελῶν αὐτοῦ \\
\tabto{2em} ἐπὶ τὸν νῶτον \\
ἀνέβη \\
καὶ ἀπ' ἐκείνου \\
\tabto{2em} ἐπὶ τὰ κέρατα διερεισαμένη \\
\tabto{2em} ἐπὶ τὸ στόμα τοῦ φρέατος \\
ηὑρέθη \\
καὶ ἀνελθοῦσα ἀπηλλάττετο.\\

\end{greek}
}

\begin{description}[noitemsep]
\item[τοῦ ὑπηρετήσαντος] ὑπηρετέω ugađati, ispuniti želju; g. sg. m. r. ptc. aor. akt.; §~139.β; GA
\item[δὲ] a\dots; čestica povezuje rečenicu s prethodnom kao suprotni veznik §~515
\item[πρὸς τὴν δευτέραν παραίνεσιν] §~80; §~223; §~82; §~165; prijedložni izraz πρὸς + a.: ``poslije\dots''; §~418, §~435.C
\item[ἡ ἀλώπηξ] §~115
\item[ἀναλλομένη] ἀνάλλομαι skočiti; n. sg. ž. r. ptc. prez. medpas.; §~103
\item[διὰ τῶν σκελῶν] §~153; prijedložni izraz διὰ + g.: ``po\dots, preko\dots''; §~418, §~428.A
\item[αὐτοῦ] §~207
\item[ἀνέβη] ἀναβαίνω popeti se; 3. l. sg. ind. aor. akt. 
\item[ἀπ' ἐκείνου] (ἀπ' = ἀπό) §~68; §~213.1; prijedložni izraz ἀπό + g.: ``od\dots'';  §~418, §~423
\item[ἐπὶ τὰ κέρατα] §~159; prijedložni izraz ἐπὶ + a.: ``na\dots''; §~418, §~436.C
\item[διερεισαμένη] διερείδω poduprijeti se; n. sg. ž. r. ptc. aor. med.; §~103
\item[ἐπὶ τὸ στόμα] §~123; prijedložni izraz ἐπὶ + a.: ``na\dots''; §~418; §~436.C
\item[τοῦ φρέατος] §~80, §~123
\item[ηὑρέθη] εὑρίσκω naći; 3. l. sg. ind. aor. pas. (pasivno ``naći se'')
\item[ἀνελθοῦσα] ἀνέρχομαι popeti se, doći gore; n. sg. ž. r. ptc. prez. akt.; §~97.β
\item[ἀπηλλάττετο] ἀπαλλάττω (ἀπαλλάσσω) udaljiti se; 3. l. sg. impf. medpas. 
\end{description}

%8

{\large
\begin{greek}
\noindent \uuline{τοῦ δὲ τράγου μεμφομένου} αὐτὴν\\
\tabto{2em} ὡς τὰς ὁμολογίας παραβαίνουσαν \\
ἡ ἀλώπηξ \\
ἐπιστραφεῖσα εἶπεν·\\

\end{greek}
}

\begin{description}[noitemsep]
\item[δὲ] a\dots; čestica povezuje rečenicu s prethodnom kao suprotni veznik §~515
\item[τοῦ τράγου ] §~80, §~82
\item[μεμφομένου] μέμφομαι grditi koga; g. sg. m. r. ptc. prez. medpas.; §~103
\item[τοῦ τράγου μεμφομένου] GA: imenica u genitivu odgovara hrvatskom subjektu zavisne rečenice, a particip predikatu
\item[αὐτὴν] §~207
\item[τὰς ὁμολογίας] §~80, §~90
\item[παραβαίνουσαν] παραβαίνω τι iznevjeriti nešto; a. sg. ž. r. ptc. prez. akt.
\item[ὡς\dots\ παραβαίνουσαν] jer da je\dots; adverbni particip u zavisnoj uzročnoj rečenici §~503.2
\item[ἡ ἀλώπηξ] §~115
\item[ἐπιστραφεῖσα] ἐπιστρέφω okrenuti se; n. sg. ž. r. ptc. aor. akt.; §~97.β
\item[εἶπεν] ἀγορεύω reći; 3. l. sg. ind. aor. akt.; §~327.7
\end{description}

%9

{\large
\begin{greek}
\noindent ``ὦ οὗτος, \\
ἀλλ' εἰ τοσαύτας φρένας εἶχες, \\
\tabto{2em} ὅσας ἐν τῷ πώγωνι τρίχας, \\
οὐ πρότερον ἂν καταβεβήκεις \\
\tabto{2em} πρὶν ἢ τὴν ἄνοδον ἐσκέψω.''\\

\end{greek}
}

\begin{description}[noitemsep]
\item[ὦ οὗτος] §~213.2
\item[ἀλλ'] §~68
\item[εἰ] da\dots; veznik uvodi zavisnu pogodbenu irealnu rečenicu §~474, §~478
\item[εἶχες] ἔχω imati; 2. l. sg. impf. akt. 
\item[τοσαύτας φρένας ] §~213; §~131
\item[τοσαύτας\dots\  ὅσας\dots] korelacija rečeničnih članova §~219
\item[ὅσας] §~103, §~219, §~443
\item[τρίχας] §~115
\item[ἐν τῷ πώγωνι] §~131; prijedložni izraz ἐν + d.: ``u\dots, na\dots''; §~418, §~426
\item[ἂν καταβεβήκεις] καταβαίνω sići; 2. l. sg. ind. plpf. akt.
\item[οὐ πρότερον\dots\  πρὶν ἢ\dots] ne bi\dots\  prije nego\dots; zavisna vremenska rečenica §~488.1
\item[τὴν ἄνοδον] §~80, §~82, §~83.1
\item[ἐσκέψω] σκέπτομαι promisliti; 2. l. sg. ind. aor. med.

\end{description}

%kraj
