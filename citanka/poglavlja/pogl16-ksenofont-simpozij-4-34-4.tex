%Unio korekture NZ, 2019-08-10
\section*{O autoru}

Ksenofont \textgreek[variant=ancient]{(Ξενοφῶν}; Atena, oko 428. – Korint, oko 354. pr.~Kr.)\ bio je učenik Sokratov, podrijetlom iz aristokratske obitelji naklonjene Sparti. Zbog političke orijentacije veći dio života proveo je izvan domovine. 

U antici su ga smatrali poglavito filozofom zbog spisa u kojima središnje mjesto pripada njegovu učitelju: \textit{Obrana Sokratova} \textgreek[variant=ancient]{(Ἀπολογία Σωκράτους),}  \textit{Uspomene na Sokrata} \textgreek[variant=ancient]{(Ἀπομνημονεύματα Σωκράτους),} \textit{Gozba} \textgreek[variant=ancient]{(Συμπόσιον).} Gospodarenjem u obitelji i državi bave se \textit{Rasprava o gospodarstvu} \textgreek[variant=ancient]{(Οἰκονομικός)} i spis \textit{O prihodima} \textgreek[variant=ancient]{(Περὶ πόρων).}  Didaktičnošću se odlikuju \textit{O jahačkoj vještini} \textgreek[variant=ancient]{(Περὶ ἱππικῆς)} te \textit{Rasprava o lovu} \textgreek[variant=ancient]{(Κυνηγετικός)} i \textit{Ras\-pra\-va o konjičkom zapovjedniku} \textgreek[variant=ancient]{(Ἱππαρχικός).}

Prema modernom razumijevanju važnija je Ksenofontova historiografska ostavština, u kojoj se ističu \textit{Grčka povijest} \textgreek[variant=ancient]{(Ἑλληνικά),} s prikazom razdoblja od 411. do 362. pr.~Kr., zamišljena kao nastavak Tukididova djela, i \textit{Kirov pohod} ili \textit{A\-na\-baza} \textgreek[variant=ancient]{(Κύρου ἀνάβασις),} opis neuspjeloga pohoda Kira Mlađeg protiv brata Artakserksa II.; u pohodu je važnu ulogu imao i sam pisac. U biografiji \textit{Kirov odgoj} \textgreek[variant=ancient]{(Κύρου παιδεία)} na primjeru Kira Starijeg obrazlaže vlastite političke poglede; politički su intonirani i dijalog \textit{Hijeron} \textgreek[variant=ancient]{(Ἱέρων),} pohvalni posmrtni govor \textit{Agezilaj} \textgreek[variant=ancient]{(Ἀγησίλαος)} te \textit{Lakedemonski ustav} \textgreek[variant=ancient]{(Λακεδαιμονίων πολιτεία)}.

Ksenofont je u antici smatran piscem uzorna jezika i stila (Diogen Laertije naziva ga Atičkom Muzom); upravo zahvaljujući takvoj procjeni jedan je od rijetkih klasičnih autora čiji se opus u cijelosti sačuvao.

\section*{O tekstu}

\textit{Simpozij}, napisan oko 365., opisuje fiktivnu gozbu na kojoj sudjeluje Sokrat; po tome sliči poznatijemu istoimenom Platonovu djelu. Kod Ksenofonta gozba se održava 422.\ pr.~Kr.\ u kući atenskoga bogataša Kalije; gosti vode razbibrižnu raspravu o tome tko se čime ponosi. No, nakon što se pokrene rasprava o ljubavi, Sokrat iznosi ozbiljan i promišljen govor, posebice o homoseksualnoj ljubavi, razdvajajući pritom etički nevrijednu tjelesnu od duhovne, uzvišene i plemenite. 

U izabranom odlomku govori se o tomu da mnogi ne mogu utažiti svoju glad za bogatstvom jer se ono ne nalazi u novcu, nego u duši.


\section*{Pročitajte naglas grčki tekst.}

Xen.\ Symposium 4.34.4

%Naslov prema izdanju

\medskip

{\large
\begin{greek}
\noindent Νομίζω, ὦ ἄνδρες, τοὺς ἀνθρώπους οὐκ ἐν τῷ οἴκῳ τὸν πλοῦτον καὶ τὴν πενίαν ἔχειν ἀλλ' ἐν ταῖς ψυχαῖς. ὁρῶ γὰρ πολλοὺς μὲν ἰδιώτας, οἳ πάνυ πολλὰ ἔχοντες χρήματα οὕτω πένεσθαι ἡγοῦνται ὥστε πάντα μὲν πόνον, πάντα δὲ κίνδυνον ὑποδύονται, ἐφ' ᾧ πλείω κτήσονται, οἶδα δὲ καὶ ἀδελφούς, οἳ τὰ ἴσα λαχόντες ὁ μὲν αὐτῶν τἀρκοῦντα ἔχει καὶ περιττεύοντα τῆς δαπάνης, ὁ δὲ τοῦ παντὸς ἐνδεῖται· αἰσθάνομαι δὲ καὶ τυράννους τινάς, οἳ οὕτω πεινῶσι χρημάτων ὥστε ποιοῦσι πολὺ δεινότερα τῶν ἀπορωτάτων· δι' ἔνδειαν μὲν γὰρ δήπου οἱ μὲν κλέπτουσιν, οἱ δὲ τοιχωρυχοῦσιν, οἱ δὲ ἀνδραποδίζονται· τύραννοι δ' εἰσί τινες οἳ ὅλους μὲν οἴκους ἀναιροῦσιν, ἁθρόους δ' ἀποκτείνουσι, πολλάκις δὲ καὶ ὅλας πόλεις χρημάτων ἕνεκα ἐξανδραποδίζονται. τούτους μὲν οὖν ἔγωγε καὶ πάνυ οἰκτίρω τῆς ἄγαν χαλεπῆς νόσου. ὅμοια γάρ μοι δοκοῦσι πάσχειν ὥσπερ εἴ τις πολλὰ ἔχοι καὶ πολλὰ ἐσθίων μηδέποτε ἐμπίμπλαιτο.

\end{greek}

}

\newpage


\section*{Analiza i komentar}

%0

{\large
\noindent Νομίζω, \\
\tabto{2em} ὦ ἄνδρες, \\
\underline{τοὺς ἀνθρώπους} \\
\tabto{2em} οὐκ ἐν τῷ οἴκῳ \\
\tabto{4em} τὸν πλοῦτον καὶ τὴν πενίαν\\
\tabto{4em}  \underline{ἔχειν} \\
\tabto{2em} ἀλλ' ἐν ταῖς ψυχαῖς.\\

}

\begin{description}[noitemsep]

\item[Νομίζω] νομίζω smatrati, misliti; otvara mjesto A+I; 1. l. sg. ind. prez. akt.
\item[ὦ ἄνδρες] §~149
\item[τοὺς ἀνθρώπους] §~82
\item[τῷ οἴκῳ] §~82
\item[τὸν πλοῦτον] §~82
\item[τὴν πενίαν] §~90 
\item[ἔχειν] ἔχω imati; inf. prez. akt.
\item[ἀλλ'] ἀλλά
\item[ταῖς ψυχαῖς] §~90
\item[ἀλλ' ἐν ταῖς ψυχαῖς] sc.\ ἔχειν
\end{description}

%1

{\large
\noindent ὁρῶ γὰρ \\
πολλοὺς μὲν ἰδιώτας,\\
\tabto{2em} οἳ πάνυ πολλὰ \\
\tabto{4em} ἔχοντες \\
\tabto{2em} χρήματα \\
\tabto{4em} οὕτω πένεσθαι \\
\tabto{6em} ἡγοῦνται \\
\tabto{4em} ὥστε πάντα μὲν πόνον, \\
\tabto{4em} πάντα δὲ κίνδυνον \\
\tabto{6em} ὑποδύονται,\\
\tabto{6em} ἐφ' ᾧ \\
\tabto{8em} πλείω κτήσονται,\\
οἶδα δὲ καὶ ἀδελφούς,\\
\tabto{2em} οἳ τὰ ἴσα λαχόντες \\
\tabto{2em} ὁ μὲν \\
\tabto{4em} αὐτῶν \\
\tabto{4em} τἀρκοῦντα \\
\tabto{2em} ἔχει \\
\tabto{4em} καὶ περιττεύοντα \\
\tabto{6em} τῆς δαπάνης, \\
\tabto{2em} ὁ δὲ \\
\tabto{4em} τοῦ παντὸς \\
\tabto{2em} ἐνδεῖται·\\

}

\begin{description}[noitemsep]
\item[ὁρῶ] ὁράω vidjeti; 1. l. sg. ind. prez. akt.
\item[γὰρ] čestica najavljuje iznošenje objašnjenja prethodne tvrdnje: naime\dots
\item[πολλοὺς μὲν\dots\ οἶδα δὲ\dots] koordinacija rečeničnih članova s pomoću para čestica: a\dots
\item[πολλοὺς\dots\ ἰδιώτας] §~196; §~100
\item[οἳ] §~215; odnosna zamjenica čiji je antecedent ἰδιώτας uvodi zavisnu odnosnu rečenicu
\item[πολλὰ\dots\ χρήματα] §~123
\item[ἔχοντες] ἔχω imati; n. pl. m. r. ptc. prez. akt.
\item[οὕτω\dots\ ὥστε\dots] koordinacija demonstrativnog priloga i posljedičnog veznika uvodi zavisnu posljedičnu rečenicu
\item[πένεσθαι] πένομαι biti siromašan; inf. prez. medpas.; dopuna uz ἡγοῦνται
\item[ἡγοῦνται] ἡγέομαι smatrati; 3. l. pl. ind. prez. medpas.; otvara mjesto dopuni u infinitivu (infinitiv kao subjekt §~492)
\item[πάντα μὲν πόνον, πάντα δὲ κίνδυνον] §~193; §~82; koordinacija rečeničnih članova s pomoću para čestica
\item[ὑποδύονται] ὑποδύω med. ὑποδύομαι podnositi, poduzeti; 3. l. pl. ind. prez. medpas.
\item[ὥστε\dots\ ὑποδύονται] veznik ὥστε uvodi zavisnu posljedičnu rečenicu: tako da\dots
\item[ἐφ' ᾧ\dots\ κτήσονται] §~215; odnosna zamjenica (u prijedložnom izrazu) uvodi odnosnu rečenicu s namjernim smislom: da\dots
\item[πλείω] komparativ pridjeva πολύς (§~196; §~202), a. pl. s. r. (§~137)
\item[κτήσονται] κτάομαι stjecati; 3. l. pl. ind. fut. med.
\item[οἶδα] οἶδα znati; 1. l. ind. perf. akt.
\item[ἀδελφούς] §~82
\item[ἀδελφούς, οἳ] §~215; odnosna zamjenica čiji je antecedent ἀδελφούς uvodi zavisnu odnosnu rečenicu
\item[τὰ ἴσα] §~103; sc.\ χρήματα ili slično; poimeničenje članom §~373
\item[λαχόντες] λαγχάνω dobiti; n. pl. m. r. ptc. aor. akt.
\item[ὁ μὲν\dots\ ὁ δὲ] koordinacija s pomoću čestica: jedan\dots\ a drugi\dots
\item[αὐτῶν] §~207, sc.\ τῶν ἀδελφῶν; dijelni genitiv §~395
\item[τἀρκοῦντα καὶ περιττεύοντα] oba oblika a. pl. s. r. ptc. prez. akt.; ἀρκέω dostajati, imati u izobilju; περιττεύω τινος nadmašivati što; τἀρκοῦντα < τὰ ἀρκοῦντα (kraza)
\item[τῆς δαπάνης ] §~90
\item[ἔχει] ἔχω imati; 3. l. sg. ind. prez. akt.
\item[τοῦ παντὸς] §~193; poimeničenje članom §~373
\item[ἐνδεῖται] ἐνδέω τινος oskudijevati čime; 3. l. sg. ind. prez. medpas.
\end{description}

%3 

{\large
\noindent αἰσθάνομαι δὲ \\
\tabto{2em} καὶ τυράννους τινάς,\\
\tabto{4em}  οἳ οὕτω πεινῶσι \\
\tabto{6em} χρημάτων\\
\tabto{4em}  ὥστε ποιοῦσι \\
\tabto{4em} πολὺ δεινότερα \\
\tabto{8em} τῶν ἀπορωτάτων·\\

}

\begin{description}[noitemsep]
\item[αἰσθάνομαι] αἰσθάνομαι opažati; 1. l. sg. ind. prez. medpas.
\item[δὲ] čestica izražava suprotnost u odnosu na prethodnu rečenicu: a\dots
\item[τυράννους τινάς] §~82; §~217
\item[οἳ] odnosna zamjenica uvodi odnosnu rečenicu, njezin je antecedent τυράννους
\item[οὕτω\dots\ ὥστε\dots] koordinacija demonstrativnog priloga i posljedičnog veznika uvodi zavisnu posljedičnu rečenicu, usp.~gore
\item[πεινῶσι] πεινάω τινος gladovati za čime; 3. l. pl. ind. prez. akt.
\item[χρημάτων] §~123
\item[ὥστε] veznik uvodi posljedičnu rečenicu (s indikativom)
\item[ποιοῦσι ] ποιέω činiti; 3. l. pl. ind. prez. akt.
\item[δεινότερα] §~82; §~197
\item[τῶν ἀπορωτάτων] §~82; §~197; ; genitiv usporedbe §~404
\end{description}

%4

{\large
\noindent  δι' ἔνδειαν μὲν γὰρ δήπου\\
\tabto{2em}  οἱ μὲν κλέπτουσιν,\\
\tabto{2em}  οἱ δὲ τοιχωρυχοῦσιν,\\
\tabto{2em}  οἱ δὲ ἀνδραποδίζονται·\\
τύραννοι δ' εἰσί τινες\\
\tabto{2em}  οἳ ὅλους μὲν οἴκους ἀναιροῦσιν,\\ 
\tabto{2em}  ἁθρόους δ' ἀποκτείνουσι,\\
\tabto{2em}  πολλάκις δὲ καὶ ὅλας πόλεις \\
\tabto{4em}  χρημάτων ἕνεκα \\
\tabto{2em}  ἐξανδραποδίζονται.\\

}

\begin{description}[noitemsep]
\item[δι' ἔνδειαν μὲν\dots\ τύραννοι δ' εἰσί\dots] koordinacija rečeničnih članova s pomoću para čestica
\item[δι' ἔνδειαν] §~97; δι' < διά, prijedlog s akuzativom
\item[γὰρ] čestica najavljuje iznošenje objašnjenja prethodne tvrdnje: naime\dots
\item[οἱ μὲν\dots\ οἱ δὲ\dots\ οἱ δὲ] koordinacija: jedni\dots\ drugi\dots\ treći\dots
\item[κλέπτουσιν] κλέπτω krasti; 3. l. pl. ind. prez. akt.
\item[τοιχωρυχοῦσιν] τοιχωρυχέω provaljivati; 3. l. pl. ind. prez. akt.
\item[ἀνδραποδίζονται] ἀνδραποδίζομαι porobljavati; 3. l. pl. ind. prez. (medpas.)
\item[τύραννοι] §~82
\item[εἰσί] εἰμί postojati; 3. l. pl. ind. prez.
\item[τινες] §~217
\item[οἳ] odnosna zamjenica uvodi više odnosnih rečenica, antecedent je τύραννοι\dots\ τινες
\item[ὅλους μὲν\dots\ ἁθρόους δ'\dots\ πολλάκις δὲ καὶ ὅλας\dots] koordinacija rečeničnih članova s pomoću čestica
\item[ὅλους\dots\ οἴκους] LSJ οἶκος III: obitelj; §~103; §~82
\item[ἀναιροῦσιν] ἀναιρέω uništavati; 3. l. pl. ind. prez. akt.
\item[ἁθρόους] sc.\ ἀνθρώπους, čine masovne zločine (usp.\ LSJ ἀθρόος); §~103
\item[ἀποκτείνουσι] ἀποκτείνω ubijati; 3. l. pl. ind. prez. akt.
\item[ὅλας πόλεις] §~103; §~165
\item[χρημάτων ἕνεκα] prijedlog koji otvara mjesto genitivu postponiran je
\item[ἐξανδραποδίζονται] slično što i ἀνδραποδίζονται (v.~gore), prefiks pojačava značenje glagola

\end{description}

%6

{\large
\noindent τούτους μὲν οὖν ἔγωγε καὶ πάνυ οἰκτίρω \\
\tabto{2em}  τῆς ἄγαν χαλεπῆς νόσου.\\

}

\begin{description}[noitemsep]
\item[τούτους] §~213
\item[μὲν οὖν] kombinacija čestica označava prijelaz i rezimiranje: dakle\dots
\item[ἔγωγε] §~206
\item[καὶ πάνυ] pojačano πάνυ: zapravo\dots
\item[οἰκτίρω] οἰκτίρω τινός sažalijevati zbog čega; 1. l. sg. ind. prez. akt.
\item[τῆς\dots\ χαλεπῆς νόσου] genitiv uzroka; §~103, §~82, §~406
\item[ἄγαν] prilog u atributnom položaju §~375
\end{description}

%7

{\large
\noindent ὅμοια γάρ \\
μοι \\
δοκοῦσι\\
\tabto{2em} \underline{πάσχειν}\\
\tabto{4em}  ὥσπερ εἴ τις πολλὰ ἔχοι\\
\tabto{4em}  καὶ πολλὰ ἐσθίων \\
\tabto{6em}  μηδέποτε ἐμπίμπλαιτο.\\

}

\begin{description}[noitemsep]
\item[ὅμοια] §~103
\item[γὰρ] čestica najavljuje iznošenje objašnjenja prethodne tvrdnje: naime\dots
\item[μοι] §~205
\item[δοκοῦσι] δοκέω μοι činiti se komu, otvara mjesto dopuni u infinitivu; 3. l. pl. ind. prez. akt.
\item[πάσχειν] πάσχω trpjeti; inf. prez. akt.
\item[μοι δοκοῦσι πάσχειν] infinitiv kao subjekt u ličnoj konstrukciji §~492; neizrečeni su subjekt infinitiva „oni'', tj.\ spomenuti lakomi tirani
\item[ὥσπερ εἴ] kombinacija čestica uvodi zavisnu rečenicu koja izriče kombinaciju usporedbe i pogodbe; kad je subjekt τις, u protazi i apodozi stoji optativ; Smyth 2478
\item[ἔχοι] ἔχω imati; 3. l. sg. opt. prez. akt.
\item[ἐσθίων] ἐσθίω jesti; n. sg. m. r. ptc. prez. akt.
\item[ἐμπίμπλαιτο] ἐμπίμπλημι zasititi; 3. l. sg. opt. prez. medpas.
\end{description}


%kraj
