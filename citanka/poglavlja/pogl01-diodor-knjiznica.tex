% Unesi korekture NČ 2019-09-19
\section*{O autoru}

Diodor Sicilski \textgreek[variant=ancient]{(Διόδωρος Σικελιώτης,} Diodorus Siculus; 90.\ – 30.\ pr.~Kr.) rođen je u Agiriju \textgreek[variant=ancient]{(Ἀγύριον)} na Siciliji, dok mu mjesto smrti nije poznato. Svjedočanstava o njegovu životu, osim onih koja se daju iščitati iz samog djela, gotovo da nema. 

Bio je veliki putnik. Za boravka u Rimu, zahvaljujući dobrom znanju latinskog jezika, mogao se služiti rimskim arhivima i knjižnicama. Životno mu je djelo \textit{Knjižnica} \textgreek[variant=ancient]{(Βιβλιοθήκη)}; na djelu je radio trideset godina. Sam je naslov svojevrsno priznanje da se autor pri pisanju služio djelima brojnih povjesničara iz prošlosti.  Povijesni je prikaz obuhvaćao 40 knjiga i sezao je sve od mitskih početaka pa do Cezarova galskog rata. U cijelosti su očuvane knjige 1–5 i 11–20, a ostale samo u ulomcima. Diodorova koncepcija opće povijesti uzor je našla u djelu grčkog povjesničara Efora (400.\ – 330.\ pr.~Kr.); uključivala je i Grke i barbare, a ponajprije Rim, koji je postao najvažnija sila poznatog svijeta. Diodorova je posebnost u tome što je u djelo uvrstio i prikaz mitskog doba. Velik je piščev doprinos što je povijest svijeta, s osvrtom na sva razdoblja i sva područja, od nastanka svemira do Cezarova doba, prikazana kao kontinuum.

Djelo je pisano jedinstvenim stilom, lako je čitljivo i razumljivo. Iz izvora preuzetoj građi autor je dao novo i moderno jezično ruho. Jezik je atički, prožet elementima iz \textit{koine;} obiluje nominalnim konstrukcijama. Poput Polibija i Diodor izbjegava hijat.

Popularna u antici i srednjem vijeku, \textit{Knjižnica} je u 16.~st. prevedena na latinski, a potom i na narodne jezike.

\section*{O tekstu}

U ovom ulomku Diodor opisuje mučan način života prvih ljudi. O primitivnom stanju u kojem se nalazio ljudski rod prije pronalaska osnovnih znanja i vještina nagađali su pisci različitih književnih vrsta. Predodžbe starih o prvotnom životu u oskudici – bez ognja, odjeće, stana i kruha – prizivaju ideju napretka od krajnje oskudice k sve savršenijim oblicima postojanja pojedinca i zajednice.

%\newpage

\section*{Pročitajte naglas grčki tekst.}

Diod.\ Sic.\ Bibliotheca historica 1.8.5

%Naslov prema izdanju

\medskip

\begin{greek}
{\large
{ \noindent Τοὺς οὖν πρώτους τῶν ἀνθρώπων μηδενὸς τῶν πρὸς βίον χρησίμων εὑρημένου ἐπιπόνως διάγειν, γυμνοὺς μὲν ἐσθῆτος ὄντας, οἰκήσεως δὲ καὶ πυρὸς ἀήθεις, τροφῆς δ' ἡμέρου παντελῶς ἀνεννοήτους. καὶ γὰρ τὴν συγκομιδὴν τῆς ἀγρίας τροφῆς ἀγνοοῦντας μηδεμίαν τῶν καρπῶν εἰς τὰς ἐνδείας ποιεῖσθαι παράθεσιν· διὸ καὶ πολλοὺς αὐτῶν ἀπόλλυσθαι κατὰ τοὺς χειμῶνας διά τε τὸ ψῦχος καὶ τὴν σπάνιν τῆς τροφῆς. ἐκ δὲ τοῦ κατ' ὀλίγον ὑπὸ τῆς πείρας διδασκομένους εἴς τε τὰ σπήλαια καταφεύγειν ἐν τῷ χειμῶνι καὶ τῶν καρπῶν τοὺς φυλάττεσθαι δυναμένους ἀποτίθεσθαι. γνωσθέντος δὲ τοῦ πυρὸς καὶ τῶν ἄλλων τῶν χρησίμων κατὰ μικρὸν καὶ τὰς τέχνας εὑρεθῆναι καὶ τἄλλα τὰ δυνάμενα τὸν κοινὸν βίον ὠφελῆσαι.

}
}
\end{greek}

\section*{Analiza i komentar}

%1

{\large
\begin{greek}
\noindent \underline{Τοὺς οὖν πρώτους} τῶν ἀνθρώπων \\
\uuline{μηδενὸς} τῶν πρὸς βίον χρησίμων \uuline{εὑρημένου} \\
ἐπιπόνως \underline{διάγειν},  \\
\underline{γυμνοὺς} μὲν ἐσθῆτος\\
\tabto{2em} \underline{ὄντας}, \\
οἰκήσεως δὲ καὶ πυρὸς\\
\tabto{2em}  \underline{ἀήθεις}, \\
τροφῆς δ' ἡμέρου \\
\tabto{2em} παντελῶς \underline{ἀνεννοήτους}.\\

\end{greek}
}

\begin{description}[noitemsep]
\item[Τοὺς πρώτους\dots] čitav je ovaj odlomak niz A+I ovisnih o (neizrečenom) „rekao sam da\dots''
\item[Τοὺς πρώτους] §~223, §~373
\item[οὖν ] kako rekoh, dakle; postpozitivna čestica
\item[τῶν ἀνθρώπων] §~82, §~395
\item[μηδενὸς] §~224 
\item[τῶν πρὸς βίον χρησίμων] §~82, §~106, §~373, §~375, §~435
\item[εὑρημένου] εὑρίσκω pronaći; g. sg. m. r. ptc. perf. medpas. 
\item[μηδενὸς\dots\ εὑρημένου] GA, ima vrijednost uzročne rečenice
\item[ἐπιπόνως ] §~204
\item[διάγειν] διάγω provoditi život; inf. prez. akt.
\item[γυμνοὺς\dots\  ἐσθῆτος] γυμνός τινος §~103, §~123, §~403
\item[γυμνοὺς μὲν\dots\  οἰκήσεως δὲ\dots\  τροφῆς δ'\dots] koordinacija rečeničnih članova s pomoću čestica  
\item[ὄντας] εἰμί biti; a. pl. m. r. ptc. prez. akt. 
\item[οἰκήσεως\dots\  πυρὸς ἀήθεις] §~165, §~146, §~153 
\item[τροφῆς δ' ἡμέρου] §~90, §~106
\item[παντελῶς] §~204
\item[ἀνεννοήτους] ἀνεννόητος τινός bez pojma o čemu
\end{description}

{\large
\begin{greek}
\noindent καὶ γὰρ τὴν συγκομιδὴν \\
\tabto{2em} τῆς ἀγρίας τροφῆς \\
\tabto{4em} \underline{ἀγνοοῦντας} \\
μηδεμίαν \\
\tabto{2em} τῶν καρπῶν \\
\tabto{4em} εἰς τὰς ἐνδείας \\
\underline{ποιεῖσθαι} παράθεσιν· \\
διὸ καὶ \underline{πολλοὺς} αὐτῶν \\
\underline{ἀπόλλυσθαι} \\
\tabto{2em} κατὰ τοὺς χειμῶνας \\
διά τε τὸ ψῦχος \\
καὶ τὴν σπάνιν τῆς τροφῆς.\\

\end{greek}
}

\begin{description}[noitemsep]
\item[καὶ γὰρ] i budući da\dots\ §~517
\item[τὴν συγκομιδὴν] §~90
\item[τῆς ἀγρίας τροφῆς] §~103, §~90 
\item[ἀγνοοῦντας] ἀγνοέω ne poznavati, ne prepoznavati; a. pl. m. r. ptc. prez. akt.; ἀγνοοῦντας sc.\ \textgreek[variant=ancient]{τοὺς πρώτους τῶν ἀνθρώπων}
\item[μηδεμίαν] §~224
\item[τῶν καρπῶν] §~82, §~370 individualni član zastupa posvojnu zamjenicu; ovisno o \textgreek[variant=ancient]{τῆς ἀγρίας τροφῆς}
\item[εἰς τὰς ἐνδείας ] §~90, §~419.c
\item[ποιεῖσθαι] ποιέω, inf. prez. medpas. 
\item[παράθεσιν] §~165; ποιεῖσθαι παράθεσιν urediti si spremište
\item[διὸ] veznik; διὸ nastalo od δι᾿ ὅ zato\dots
\item[πολλοὺς αὐτῶν] §~196, §~207, §~395
\item[ἀπόλλυσθαι ] ἀπόλλυμι uništiti; inf. prez. medpas.
\item[κατὰ τοὺς χειμῶνας ] §~131, §~418
\item[διά τὸ ψῦχος ] §~153, §~418
\item[τε\dots\  καὶ\dots] ne samo\dots\  nego i\dots; koordinacija sastavnim veznicima §~513 
\item[τὴν σπάνιν τῆς τροφῆς ] §~165, §~90
\end{description}

%3 itd

{\large
\begin{greek}
\noindent ἐκ δὲ τοῦ \\
\tabto{2em} κατ' ὀλίγον\\
\tabto{4em} ὑπὸ τῆς πείρας \\
\underline{διδασκομένους} \\
εἴς τε τὰ σπήλαια \\
\underline{καταφεύγειν} \\
\tabto{2em} ἐν τῷ χειμῶνι \\
καὶ \\
τῶν καρπῶν \\
τοὺς φυλάττεσθαι δυναμένους \\
\underline{ἀποτίθεσθαι}.\\

\end{greek}
}

\begin{description}[noitemsep]
\item[ἐκ τοῦ] odonda; §~370 3, §~424.b
\item[κατ' ὀλίγον] pomalo; §~429.B
\item[ὑπὸ τῆς πείρας ] §~90, §~437.A; prijedložni izraz ὑπὸ + g. izriče uzrok
\item[διδασκομένους] διδάσκω učiti; a. pl. m. r. ptc. prez. medpas.; subjektni akuzativ uz καταφεύγειν i ἀποτίθεσθαι
\item[εἴς τὰ σπήλαια] §~82, §~419
\item[καταφεύγειν] καταφεύγω bježati; inf. prez. akt. 
\item[ἐν τῷ χειμῶνι] §~131, §~426.b
\item[τῶν καρπῶν] §~82, §~395
\item[τοὺς] sc.\ καρπούς; otvara mjesto dijelnom genitivu τῶν καρπῶν: one od plodova\dots
\item[τοὺς\dots\ δυναμένους] δύναμαι moći; a. pl. m. r. ptc. prez. medpas.; otvara mjesto dopuni u infinitivu; §~371
\item[φυλάττεσθαι] φυλάττω čuvati; inf. prez. medpas.; objekt participa δυναμένους
\item[ἀποτίθεσθαι] ἀποτίθημι spremati; inf. prez. medpas.
\end{description}

%4

{\large
\begin{greek}
\noindent \uuline{γνωσθέντος δὲ τοῦ πυρὸς}\\
\tabto{2em} καὶ \uuline{τῶν ἄλλων} τῶν χρησίμων \\
κατὰ μικρὸν \\
καὶ \underline{τὰς τέχνας εὑρεθῆναι} \\
καὶ \underline{τἄλλα τὰ δυνάμενα} \\
\tabto{2em} τὸν κοινὸν βίον \underline{ὠφελῆσαι}.\\

\end{greek}
}

\begin{description}[noitemsep]
\item[γνωσθέντος] γιγνώσκω upoznati; g. sg. s. r. ptc. aor. pas.; GA ima vrijednost vremenske rečenice
\item[τοῦ πυρὸς] §~146
\item[τῶν ἄλλων] §~371 τῶν ἄλλων također ovisno o γνωσθέντος, dio konstrukcije GA
\item[τῶν χρησίμων]  §~106, §~395
\item[κατὰ μικρὸν] postupno; §~429.B
\item[τὰς τέχνας] §~90
\item[εὑρεθῆναι ] εὑρίσκω pronalaziti; inf. pas. aor.
\item[τἄλλα] §~16, §~371 τἄλλα = τὰ ἄλλα 
\item[τὰ δυνάμενα ] §~371 δύναμαι moći; a. pl. s. r. ptc. prez. medpas.
\item[τἄλλα τὰ δυνάμενα] (kao i τὰς τέχνας) subjektni su akuzativi uz εὑρεθῆναι
\item[τὸν κοινὸν βίον] §~82, §~103; objekt infinitiva ὠφελῆσαι
\item[ὠφελῆσαι] ὠφελέω τινα koristiti komu; inf. aor. akt.; objekt participa τὰ δυνάμενα
\end{description}


%kraj
