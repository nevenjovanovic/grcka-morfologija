% Unesi korekture NČ, NZ 2019-09-23

\section*{O tekstu}

U drugoj knjizi \textit{Retorike} Aristotel izlaže načine na koje govornik može uvjeriti slušaoce i preduvjete za takvo uvjeravanje. Poseban su način uvjeravanja takozvana opća mjesta, dokazi da se nešto može ili ne može dogoditi. Osim logičkih dokaza, opća su mjesta i primjeri; njihova su pak posebna podvrsta pripovijesti, poput prispodoba i basna. Drugi po redu primjer za uporabu basne u procesu političkog odlučivanja Aristotel pripisuje Ezopu, legendarnom basnopiscu, po predaji tračkom robu iz VI.~st.\ pr.~Kr.\ prodanom na otok Sam.

%\newpage

\section*{Pročitajte naglas grčki tekst.}

%Naslov prema izdanju
Arist.\ Rhetorica 1394a
\medskip

{\large
\begin{greek}
\noindent Αἴσωπος δὲ ἐν Σάμῳ δημηγορῶν κρινομένου δημαγωγοῦ περὶ θανάτου ἔφη ἀλώπεκα διαβαίνουσαν ποταμὸν ἀπωσθῆναι εἰς φάραγγα, οὐ δυναμένην δὲ ἐκβῆναι πολὺν χρόνον κακοπαθεῖν καὶ κυνοραιστὰς πολλοὺς ἔχεσθαι αὐτῆς, ἐχῖνον δὲ πλανώμενον, ὡς εἶδεν αὐτήν, κατοικτείραντα ἐρωτᾶν εἰ ἀφέλοι αὐτῆς τοὺς κυνοραιστάς, τὴν δὲ οὐκ ἐᾶν· ἐρομένου δὲ διὰ τί, ``ὅτι οὗτοι μὲν'' φάναι ``ἤδη μου πλήρεις εἰσὶ καὶ ὀλίγον ἕλκουσιν αἷμα, ἐὰν δὲ τούτους ἀφέλητε, ἕτεροι ἐλθόντες πεινῶντες ἐκπιοῦνταί μου τὸ λοιπὸν αἷμα''. ``ἀτὰρ καὶ ὑμᾶς, ἄνδρες Σάμιοι, οὗτος μὲν οὐδὲν ἔτι βλάψει (πλούσιος γάρ ἐστιν), ἐὰν δὲ τοῦτον ἀποκτείνητε, ἕτεροι ἥξουσι πένητες, οἳ ὑμᾶς ἀναλώσουσι τὰ λοιπὰ κλέπτοντες.''

\end{greek}
}

\newpage

\section*{Analiza i komentar}

%1

{\large
\begin{greek}
\noindent Αἴσωπος δὲ \\
\tabto{2em} ἐν Σάμῳ \\
\tabto{2em} δημηγορῶν \\
\uuline{κρινομένου δημαγωγοῦ} \\
\tabto{2em} περὶ θανάτου \\
ἔφη \\
\tabto{2em} \underline{ἀλώπεκα} \\
\tabto{4em} \underline{διαβαίνουσαν} \\
\tabto{6em} ποταμὸν \\
\tabto{2em} \underline{ἀπωσθῆναι} \\
\tabto{4em} εἰς φάραγγα, \\
\tabto{4em} \underline{οὐ δυναμένην} δὲ \\
\tabto{6em} ἐκβῆναι \\
\tabto{2em} πολὺν χρόνον \\
\tabto{2em} \underline{κακοπαθεῖν} \\
\tabto{2em} καὶ \\
\tabto{4em} \underline{κυνοραιστὰς πολλοὺς} \\
\tabto{2em} \underline{ἔχεσθαι} \\
\tabto{4em} αὐτῆς, \\
\tabto{2em} \underline{ἐχῖνον} δὲ πλανώμενον, \\
\tabto{4em} ὡς εἶδεν \\
\tabto{6em} αὐτήν, \\
\tabto{2em} \underline{κατοικτείραντα} \\
\tabto{2em} \underline{ἐρωτᾶν} \\
\tabto{4em} εἰ ἀφέλοι \\
\tabto{6em} αὐτῆς \\
\tabto{6em} τοὺς κυνοραιστάς, \\
\tabto{2em} \underline{τὴν} δὲ \\
\tabto{2em} \underline{οὐκ ἐᾶν}· \\
\newpage
\tabto{2em} \uuline{ἐρομένου} δὲ \\
\tabto{4em} διὰ τί, \\
\tabto{2em} ``ὅτι \\
\tabto{4em} οὗτοι μὲν'' \\
\tabto{2em} \underline{φάναι} \\
\tabto{4em} ``ἤδη \\
\tabto{4em} μου πλήρεις εἰσὶ \\
\tabto{4em} καὶ ὀλίγον ἕλκουσιν αἷμα, \\
\tabto{4em} ἐὰν δὲ \\
\tabto{6em} τούτους ἀφέλητε, \\
\tabto{4em} ἕτεροι \\
\tabto{6em} ἐλθόντες πεινῶντες \\
\tabto{4em} ἐκπιοῦνταί μου \\
\tabto{6em} τὸ λοιπὸν αἷμα''.\\

\end{greek}
}

\begin{description}[noitemsep]
\item[Αἴσωπος ] §~82
\item[δὲ ] čestica povezuje rečenicu s (izostavljenom) prethodnom: a\dots
\item[ἐν Σάμῳ ] §~426; §~82
\item[δημηγορῶν ] δημηγορέω govoriti u skupštini; n. sg. m. r. ptc. prez. akt.
\item[κρινομένου ] κρίνω suditi; g. sg. m. r. ptc. prez. medpas.
\item[δημαγωγοῦ ] §~82
\item[περὶ θανάτου ] §~433; §~82
\item[ἔφη ] φημί govoriti; 3. l. sg. impf. akt.
\item[ἀλώπεκα ] §~115
\item[διαβαίνουσαν ] διαβαίνω τι prelaziti što; a. sg. ž. r. ptc. prez. akt.
\item[ποταμὸν ] §~82
\item[ἀπωσθῆναι ] ἀπωθέω odgurnuti, odbaciti; inf. aor. pas.
\item[εἰς φάραγγα] §~419; §~115
\item[δυναμένην ] δύναμαι moći; a. sg. ž. r. ptc. prez. med.
\item[ἀλώπεκα\dots] \textbf{δυναμένην δὲ\dots\ ἐχῖνον δὲ\dots\ τὴν δὲ\dots\ ἐρομένου δὲ\dots} koordinacija surečenica česticom δὲ
\item[ἐκβῆναι ] ἐκβαίνω izaći; inf. aor. akt.
\item[πολὺν ] §~196
\item[χρόνον ] §~82
\item[κακοπαθεῖν] κακοπαθέω patiti; inf. prez. akt.
\item[κυνοραιστὰς ] §~100
\item[πολλοὺς ] §~196
\item[ἔχεσθαι ] ἔχω, med. ἔχομαί τινος držati se čega; inf. prez. medpas.
\item[αὐτῆς] §~207
\item[ἐχῖνον ] §~82
\item[πλανώμενον] πλανάω med. lutati; a. sg. m. r. ptc. prez. medpas.
\item[ὡς ] veznik uvodi zavisnu vremensku rečenicu §~487
\item[εἶδεν ] ὁράω gledati; 3. l. sg. ind. aor. akt.
\item[αὐτήν] §~207
\item[κατοικτείραντα ] κατοικτείρω smilovati se; a. sg. m. r. ptc. aor. akt.
\item[ἐρωτᾶν] ἐρωτάω pitati (uvodi zavisnu upitnu rečenicu); inf. prez. akt.
\item[εἰ ἀφέλοι] zavisna upitna rečenica kojoj mjesto otvara infinitiv \textgreek[variant=ancient]{ἐρωτᾶν; ἀφαιρέω τί τινος} maknuti što s koga; 3. l. sg. opt. aor. akt.
\item[αὐτῆς ] §~207
\item[τοὺς κυνοραιστάς] §~100
\item[τὴν δὲ ] §~370.2
\item[ἐᾶν] ἐάω pustiti, ostaviti; inf. prez. akt.
\item[ἐρομένου ] ἔρομαι pitati; g. sg. m. r. ptc. prez. medpas.
\item[διὰ τί] §~428; §~217
\item[ὅτι ] veznik uvodi zavisnu uzročnu rečenicu §~468
\item[οὗτοι μὲν\dots, ἐὰν δὲ\dots] koordinacija surečenica parom čestica
\item[οὗτοι ] §~213.2
\item[φάναι] φημί govoriti; inf. prez. akt.
\item[ἤδη μου ] §~40
\item[πλήρεις εἰσὶ] §~40; imenski predikat, Smyth 909
\item[πλήρεις ] πλήρης τινός pun čega; §~153; imenski dio predikata
\item[εἰσὶ] εἰμί biti; 3. l. pl. ind. prez. akt.; kopula kao dio imenskog predikata
\item[ὀλίγον\dots\ αἷμα] §~103, §~123
\item[ἕλκουσιν ] ἕλκω vući, \textit{ovdje} sisati; 3. l. pl. ind. prez. akt.
\item[ἐὰν\dots\ ἀφέλητε\dots\ ἐκπιοῦνταί] eventualni oblik pogodbene rečenice §~476
\item[ἀφέλητε] ἀφαιρέω maknuti; 2. l. pl. konj. aor. akt.
\item[τούτους] §~213.2
\item[ἕτεροι ] §~103
\item[ἐλθόντες ] ἔρχομαι doći; n. pl. m. r. ptc. aor. akt.
\item[πεινῶντες ] πεινάω biti gladan; n. pl. m. r. ptc. prez. akt.
\item[ἐκπιοῦνταί μου] §~40
\item[ἐκπιοῦνταί ] ἐκπίνω ispiti; 3. l. pl. ind. fut. med.
\item[μου ] §~205
\item[τὸ λοιπὸν αἷμα] §~103; §~123; §~375

\end{description}

%\newpage


{\large
\begin{greek}
\noindent ``ἀτὰρ καὶ ὑμᾶς, \\
\tabto{2em} ἄνδρες Σάμιοι, \\
οὗτος μὲν \\
οὐδὲν ἔτι \\
βλάψει \\
(πλούσιος γάρ ἐστιν), \\
ἐὰν δὲ \\
\tabto{2em} τοῦτον \\
\tabto{2em} ἀποκτείνητε, \\
ἕτεροι ἥξουσι πένητες, \\
\tabto{2em} οἳ \\
\tabto{2em} ὑμᾶς \\
\tabto{2em} ἀναλώσουσι \\
\tabto{4em} τὰ λοιπὰ \\
\tabto{4em} κλέπτοντες.''\\

\end{greek}
}

\begin{description}[noitemsep]
\item[ἀτὰρ] ali\dots; čestica suprotnog značenja, uvodi jako suprotstavljenu misao
\item[ὑμᾶς] §~205
\item[ἄνδρες Σάμιοι] §~149; §~103
\item[οὗτος μὲν\dots, ἐὰν δὲ\dots] koordinacija surečenica parom čestica
\item[οὗτος] §~213.2
\item[οὐδὲν ἔτι ] ništa više; §~224.2
\item[βλάψει] βλάπτω τινά nanositi štetu komu; 3. l. sg. ind. fut. akt.
\item[πλούσιος] §~103; imenski dio predikata
\item[γάρ ἐστιν] §~40
\item[γάρ ] pa\dots, kad\dots; čestica uvodi objašnjenje
\item[ἐστιν] εἰμί biti; 3. l. sg. ind. prez. akt.; kopula imenskog predikata
\item[ἐὰν\dots\ ἀποκτείνητε\dots\ ἥξουσι] eventualni oblik pogodbene rečenice §~476
\item[ἀποκτείνητε] ἀποκτείνω ubiti; 2. l. pl. konj. aor. akt.
\item[τοῦτον] §~213.2
\item[ἕτεροι ] §~103
\item[ἥξουσι ] ἥκω doći; 3. l. pl. ind. fut. akt.
\item[πένητες] §~123
\item[οἳ] §~215
\item[οἳ\dots\ ἀναλώσουσι] odnosna zamjenica uvodi zavisnu odnosnu rečenicu, antecedent je \textgreek[variant=ancient]{ἕτεροι πένητες}
\item[ὑμᾶς] §~205
\item[ἀναλώσουσι ] ἀναλίσκω iskoristiti, uništiti; 3. l. pl. ind. fut. akt.
\item[τὰ λοιπὰ ] §~103; §~373
\item[κλέπτοντες] κλέπτω krasti; n. pl. m. r. ptc. prez. akt.

\end{description}

%3 itd

%kraj
