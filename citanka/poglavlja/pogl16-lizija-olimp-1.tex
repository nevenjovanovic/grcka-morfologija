%Unesi korekture NČ, 2019-08-16
%\section*{O autoru}


\section*{O tekstu}

Grčki govornik Lizija napisao je i, prema predaji, sâm predstavio \textit{Olimpijski govor} 338.\ pr.~Kr.\ na svečanostima prigodom Olimpijskih igara. Ovaj pledoaje za jedinstvo i slobodu Grka svrstava se u prigodno (epideiktično) govorništvo.

\textit{Olimpijski govor} sačuvan je kao opsežan citat kod grčkog retora i povjesničara Dionizija Halikarnašanina (\textit{O starim govornicima}, 29-30). Dodatne podatke vezane za ovo Lizijino djelo nalazimo kod povjesničara Diodora Sicilskoga (\textit{Knjižnica}, XIV. 105). 

Kao sin doseljenika iz Sirakuze u Atenu, Lizija u Ateni nije smio javno govoriti. To je smio, međutim, na svečanom okupljanju svih Grka povodom Olimpijskih igara, gdje se obraća okupljenima s jasnim i izravnim pozivom na svrgavanje sirakuškoga tiranina Dionizija Starijeg. Naime, od dolaska na vlast 405.\ pr.~Kr.\ Dionizije Stariji širio je svoju vlast, prvo iz Sirakuze na cijelu Siciliju, zatim i na južnu Italiju, a od 395.\ pr.~Kr., kada počinje Korintski rat, imao je ulogu i u tom sukobu grčkih polisa, Sparte, Atene i Perzije. Zato je, drži govornik, tiranin Sirakuze za Grke opasan koliko i perzijski kralj Artakserkso II.

Godine 388.\ pr.~Kr.\ Dionizije Stariji pripremio je i bogato opremio poslanstvo u Olimpiju da tamo u njegovo ime prinese raskošnu žrtvu. Šatori poslanika bili su ukrašeni zlatom, na utrkama četveroprega natjecali su se Dionizijevi konji, a recitatori su izvodili Dionizijeve pjesme. Raskoš, bogatstvo i moć trebali su impresionirati Grke. No, demokracijom prožetoj Ateni samovlada i moć Dionizija Starijega bila je odbojna. Apel za svrgnuće tiranina i oslobođenje Sicilije Lizija gradi na tom sentimentu.

%\newpage

\section*{Pročitajte naglas grčki tekst.}

Lys.\ Olympiacus 1

%Naslov prema izdanju

\medskip


{\large
{ \noindent I

\begin{greek}

\noindent Ἄλλων τε πολλῶν καὶ καλῶν ἔργων ἕνεκα, ὦ ἄνδρες, ἄξιον Ἡρακλέους μεμνῆσθαι, καὶ ὅτι τόνδε τὸν ἀγῶνα πρῶτος συνήγειρε δι' εὔνοιαν τῆς Ἑλάδος. ἐν μὲν γὰρ τῷ τέως χρόνῳ ἀλλοτρίως αἱ πόλεις πρὸς ἀλλήλας διέκειντο· ἐπειδὴ δὲ ἐκεῖνος τοὺς τυράννους ἔπαυσε καὶ τοὺς ὑβρίζοντας ἐκώλυσεν, ἀγῶνα μὲν σωμάτων ἐποίησε, φιλοτιμίαν $\langle$δὲ$\rangle$ πλούτου, γνώμης δ' ἐπίδειξιν ἐν τῷ καλλίστῳ τῆς Ἑλλάδος, ἵνα τούτων ἁπάντων ἕνεκα εἰς τὸ αὐτὸ συνέλθωμεν, τὰ μὲν ὀψόμενοι, τὰ δ' ἀκουσόμενοι· ἡγήσατο γὰρ τὸν ἐνθάδε σύλλογον ἀρχὴν γενήσεσθαι τοῖς  Ἕλλησι τῆς πρὸς ἀλλήλους φιλίας. 

\end{greek}

\noindent II

\begin{greek}

\noindent  Ἐγὼ δὲ ἥκω οὐ μικρολογησόμενος οὐδὲ περὶ τῶν ὀνομάτων μαχούμενος. ἡγοῦμαι γὰρ ταῦτα ἔργα μὲν εἶναι σοφιστῶν λίαν ἀχρήστων καὶ σφόδρα βίου δεομένων, ἀνδρὸς δὲ ἀγαθοῦ καὶ πολίτου πολλοῦ ἀξίου περὶ τῶν μεγίστων συμβουλεύειν, ὁρῶν οὕτως αἰσχρῶς διακειμένην τὴν Ἑλλάδα, καὶ πολλὰ μὲν αὐτῆς ὄντα ὑπὸ τῷ βαρβάρῳ, πολλὰς δὲ πόλεις ὑπὸ τυράννων ἀναστάτους γεγενημένας. καὶ ταῦτα εἰ μὲν δι' ἀσθένειαν ἐπάσχομεν, στέργειν ἂν ἦν ἀνάγκη τὴν τύχην· ἐπειδὴ δὲ διὰ στάσιν καὶ τὴν πρὸς ἀλλήλους φιλονικίαν, πῶς οὐκ ἄξιον τῶν μὲν παύσασθαι τὰ δὲ κωλῦσαι, εἰδότας ὅτι φιλονικεῖν μέν ἐστιν εὖ πραττόντων, γνῶναι δὲ τὰ βέλτιστα τῶν οἵων ἡμῶν; ὁρῶμεν γὰρ τοὺς κινδύνους καὶ μεγάλους καὶ πανταχόθεν περιεστηκότας· ἐπίστασθε δὲ ὅτι ἡ μὲν ἀρχὴ τῶν κρατούντων τῆς θαλάττης, τῶν δὲ χρημάτων βασιλεὺς ταμίας, τὰ δὲ τῶν Ἑλλήνων σώματα τῶν δαπανᾶσθαι δυναμένων, ναῦς δὲ πολλὰς $\langle$μὲν$\rangle$ αὐτὸς κέκτηται, πολλὰς δ' ὁ τύραννος τῆς Σικελίας. ὥστε ἄξιον τὸν μὲν πρὸς ἀλλήλους πόλεμον καταθέσθαι, τῇ δ' αὐτῇ γνώμῃ χρωμένους τῆς σωτηρίας ἀντέχεσθαι, καὶ περὶ μὲν τῶν παρεληλυθότων αἰσχύνεσθαι, περὶ δὲ τῶν μελλόντων ἔσεσθαι δεδιέναι, καὶ πρὸς τοὺς προγόνους ἁμιλλᾶσθαι, οἳ τοὺς μὲν βαρβάρους ἐποίησαν τῆς ἀλλοτρίας ἐπιθυμοῦντας τῆς σφετέρας αὐτῶν στερεῖσθαι, τοὺς δὲ τυράννους ἐξελάσαντες κοινὴν ἅπασι τὴν ἐλευθερίαν κατέστησαν.
\end{greek}

}
}


\section*{Analiza i komentar}

%1

{\large
\begin{greek}
\noindent ῎Αλλων τε πολλῶν καὶ καλῶν ἔργων ἕνεκα, \\
ὦ ἄνδρες, \\
ἄξιον \\
\tabto{2em} Ἡρακλέους μεμνῆσθαι,\\
καὶ ὅτι \\
\tabto{2em} τόνδε τὸν ἀγῶνα \\
\tabto{2em} πρῶτος \\
\tabto{2em} συνήγειρε \\
\tabto{2em} δι' εὔνοιαν τῆς Ἑλλάδος.\\

\end{greek}
}

\begin{description}[noitemsep]
\item[τε\dots\ καὶ] koordinacija: ne samo\dots\ nego i\dots, §~513.2
\item[῎Αλλων] §~212.a
\item[πολλῶν] §~196
\item[καλῶν] §~103
\item[ἔργων] §~82
\item[ἕνεκα] radi, zbog; prijedlog s genitivom, ovdje u postpoziciji §~417
\item[ὦ ἄνδρες] §~80, §~149
\item[ἄξιον] sc.\ ἐστίν §~103; imenski predikat Smyth 910
\item[Ηρακλέους] §~153, §~157
\item[μεμνῆσθαι] μιμνήσκω sjetiti se; inf. perf. medpas.
\item[ὅτι] zavisni uzročni veznik: jer\dots
\item[τόνδε ] §~213
\item[τὸν ἀγῶνα] §~131
\item[πρῶτος] §~223, §~103
\item[συνήγειρε] συναγείρω sakupiti, okupiti ljude za natjecanje, organizirati natjecanje; 3. l. sg. ind. aor. akt.
\item[δι' (= διά) εὔνοιαν] §~68, §~90, prijedložni izraz διά + a.: zbog\dots; §~418, §~428.B
\item[τῆς Ἑλλάδος] §~80, §~90, §~123

\end{description}

%2

{\large
\begin{greek}
\noindent ἐν μὲν γὰρ τῷ τέως χρόνῳ \\
ἀλλοτρίως \\
αἱ πόλεις \\
\tabto{2em} πρὸς ἀλλήλας \\
διέκειντο·\\
ἐπειδὴ δὲ ἐκεῖνος \\
τοὺς τυράννους ἔπαυσε\\
καὶ\\
τοὺς ὑβρίζοντας ἐκώλυσεν, \\
ἀγῶνα μὲν σωμάτων \\
ἐποίησε, \\
φιλοτιμίαν $\langle$δὲ$\rangle$ πλούτου, \\
γνώμης δ' ἐπίδειξιν \\
\tabto{2em} ἐν τῷ καλλίστῳ τῆς Ἑλλάδος,\\
ἵνα \\
\tabto{2em} τούτων ἁπάντων ἕνεκα \\
\tabto{2em} εἰς τὸ αὐτὸ \\
συνέλθωμεν, \\
\tabto{2em} τὰ μὲν ὀψόμενοι, \\
\tabto{2em} τὰ δ' ἀκουσόμενοι·\\

\end{greek}
}

\begin{description}[noitemsep]
\item[ἐν μὲν τῷ τέως χρόνῳ\dots] \textbf{ἐπειδὴ δὲ ἐκεῖνος\dots}\ koordinacija rečeničnih članova pomoću čestica μὲν\dots\ δέ\dots
\item[γὰρ] čestica s objasnidbenim ili uzročnim značenjem: naime, jer
\item[τῷ χρόνῳ] §~80, §~82
\item[ἐν τῷ τέως χρόνῳ] prijedložni izraz ἐν + prilog vremena + d.: u\dots; §~418, §~426, §~375.5
\item[αἱ πόλεις] §~80, §~90, §~165
\item[πρὸς ἀλλήλας] §~212, prijedložni izraz πρὸς + a.: prema\dots; §~418, §~435.4.C
\item[διέκειντο] διάκειμαι nahoditi se; 3. l. pl. impf. medpas.
\item[ἐπειδὴ] zavisni vremenski veznik: kad\dots
\item[ἐκεῖνος] §~213.3
\item[τοὺς τυράννους] §~80, §~82
\item[ἔπαυσε] παύω τινά učiniti kraj nekome; 3. l. sg. ind. aor. akt.
\item[τοὺς ὑβρίζοντας] ὑβρίζω ponašati se nasilno; a. pl. m. r. ptc. prez. akt.; §~80, §~82, §~139.α
\item[ἐκώλυσεν] κωλύω spriječiti, okončati; 3. l. sg. ind. aor. akt. 
\item[ἀγῶνα μὲν] \textbf{\textgreek[variant=ancient]{σωμάτων\dots\ φιλοτιμίαν $\langle$δὲ$\rangle$ πλούτου\dots\ γνώμης δ' ἐπίδειξιν}} koordinacija dijelova rečenica (sve objekti predikata ἐποίησε) pomoću čestica μὲν\dots\ δὲ\dots\ §~519.7; prelomljene zagrade označavaju priređivačevu intervenciju kad u sačuvanim prijepisima nema onoga što bi trebalo stajati
\item[ἀγῶνα ] §~131
\item[σωμάτων] §~123
\item[ἐποίησε] ποιέω učiniti, potaknuti, uspostaviti, organizirati; 3. l. sg. ind. aor. akt.
\item[φιλοτιμίαν ] §~90.b
\item[πλούτου] §~82
\item[δ'] §~68
\item[γνώμης] §~90.a
\item[τῆς Ἑλλάδος] §~123
\item[ἐπίδειξιν] §~165
\item[ἐν τῷ καλλίστῳ] §~80, §~82, prijedložni izraz ἐν + d.: u, na\dots
\item[τῆς Ἑλλάδος] §~80, §~90, §~123
\item[ἵνα] zavisni namjerni veznik: da\dots; ἵνα + konj. §~470 
\item[τούτων ἁπάντων ἕνεκα] §~213.2, §~193, prijedložni izraz ἕνεκα + d.: radi, zbog\dots
\item[εἰς τὸ αὐτὸ] §~80, §~82, §~207, prijedložni izraz εἰς + a.: u, na, za\dots; §~418, §~419
\item[συνέλθωμεν] συνέρχομαι sastati se, skupiti se; 1. l. pl. konj. aor. akt. 
\item[τὰ μὲν\dots\ τὰ δ'] §~68; koordinacija: jedno\dots\ drugo\dots
\item[ὀψόμενοι] ὁράω gledati; n. pl. m. r. ptc. fut. med.; §~103
\item[ἀκουσόμενοι] ἀκούω čuti; n. pl. m. r. ptc. fut. med.; §~103

\end{description}


%3

{\large
\begin{greek}
\noindent ἡγήσατο γὰρ \\
\underline{τὸν ἐνθάδε σύλλογον}\\
\underline{ἀρχὴν γενήσεσθαι} \\
\tabto{2em} τοῖς ῞Ελλησι \\
\tabto{2em} τῆς πρὸς ἀλλήλους φιλίας.\\

\end{greek}
}

\begin{description}[noitemsep]
\item[ἡγήσατο] ἡγέομαι misliti, smatrati; 3. l. sg. ind. aor. med. 
\item[γὰρ] čestica objasnidbenog ili uzročnog značenja: naime, jer\dots
\item[τὸν σύλλογον] §~80, §~82
\item[τὸν ἐνθάδε σύλλογον ] atributni položaj priloga §~375.5
\item[ἀρχὴν] §~90.a; dio konstrukcije A+I
\item[γενήσεσθαι] γίγνομαι: postati, desiti se; inf. fut. med.; dio konstrukcije A+I
\item[τοῖς ῞Ελλησι] §~80, §~82, §~131
\item[τῆς φιλίας] §~80, §~90.a
\item[πρὸς ἀλλήλους] §~212, prijedložni izraz πρὸς + a.: prema\dots; §~418, §~435.4.C
\item[τῆς πρὸς ἀλλήλους φιλίας] prijateljstvo jednih prema drugima; atributni položaj prijedložnog izraza §~375.4

\end{description}


%4

{\large
\begin{greek}
\noindent Ἐγὼ δὲ ἥκω \\
\tabto{2em} οὐ μικρολογησόμενος \\
\tabto{2em} οὐδὲ 
\tabto{2em} \tabto{2em} περὶ τῶν ὀνομάτων 
\tabto{2em} μαχούμενος. \\

\end{greek}
}

\begin{description}[noitemsep]
\item[δὲ] čestica povezuje rečenicu s prethodnom: a\dots
\item[Ἐγὼ] §~205, §~206
\item[ἥκω] ἥκω doći, biti tu; 1. l. sg. ind. prez. akt.
\item[οὐ\dots\ οὐδὲ] koordinacija: niti\dots\ niti\dots
\item[μικρολογησόμενος ] μικρολογέομαι razmatrati u detalje; n. sg. m. r. ptc. fut. med.; u funkciji priložne oznake particip futura izražava namjeru §~503.3
\item[περὶ τῶν ὀνομάτων] §~80, §~82, §~123, prijedložni izraz περὶ + g.: o\dots; §~418, §~433
\item[μαχούμενος] μάχομαι prepirati se; n. sg. m. r. ptc. fut. med.

\end{description}

%5
{\large
\begin{greek}
\noindent ἡγοῦμαι γὰρ \\
\tabto{2em} \underline{ταῦτα ἔργα} μὲν \\
\tabto{2em} \underline{εἶναι} σοφιστῶν \\
\tabto{2em} \tabto{2em} λίαν ἀχρήστων \\
\tabto{2em} \tabto{2em} καὶ σφόδρα βίου δεομένων,\\
\tabto{2em} ἀνδρὸς δὲ ἀγαθοῦ \\
\tabto{2em} καὶ πολίτου πολλοῦ ἀξίου \\
\tabto{2em} \tabto{2em} περὶ τῶν μεγίστων \\
\tabto{2em} συμβουλεύειν, \\
ὁρῶν \\
οὕτως αἰσχρῶς διακειμένην τὴν Ἑλλάδα, \\
καὶ πολλὰ μὲν αὐτῆς \\
\tabto{2em} ὄντα ὑπὸ τῷ βαρβάρῳ, \\
πολλὰς δὲ πόλεις \\
\tabto{2em} ὑπὸ τυράννων \\
\tabto{2em} ἀναστάτους γεγενημένας.\\

\end{greek}
}

\begin{description}[noitemsep]
\item[ἡγοῦμαι] ἡγέομαι vjerovati, misliti; 1. l. sg. ind. prez. medpas.
\item[γὰρ] čestica najavljuje iznošenje dokaza ili objašnjenja prethodne tvrdnje: naime, jer\dots
\item[ταῦτα ἔργα μὲν εἶναι σοφιστῶν\dots] \textbf{\textgreek[variant=ancient]{ἀνδρὸς δὲ ἀγαθοῦ\dots}}\ koordinacija pomoću čestica μὲν\dots\ δέ
\item[ταῦτα ἔργα] §~213.2, §~82
\item[εἶναι] εἰμί biti; εἶναί τινος pripadati nekome, biti svojstvo nekoga; inf. prez. akt
\item[σοφιστῶν ἀχρήστων] §~82, §~103
\item[βίου ] §~82
\item[δεομένων] δέομαί τινος željeti, trebati, moliti nešto; g. pl. m. r. ptc. prez. medpas.
\item[ἀνδρὸς ἀγαθοῦ] §~149, §~103
\item[πολίτου πολλοῦ ἀξίου ] §~100.a, §~196, §~103
\item[περὶ τῶν μεγίστων] §~80, §~82, §~200, §~196, prijedložni izraz περὶ + g.: o\dots; §~418, §~433
\item[συμβουλεύειν] συμβουλεύω savjetovati; inf. prez. akt.
\item[ὁρῶν ] ὁράω gledati; n. sg. m. r. ptc. prez. akt.; §~139, §~243; ὁράω, kao glagol spoznavanja \textit{(verbum sentiendi)}, otvara mjesto predikatnom participu koji odgovara našoj izričnoj (objektnoj) rečenici: vidim da\dots, §~502
\item[διακειμένην ] διάκειμαι biti u kakvom položaju, stanju; a. sg. ž. r. ptc. prez. medpas.
\item[τὴν Ἑλλάδα ] §~80, §~90, §~123
\item[καὶ πολλὰ μὲν αὐτῆς\dots] \textbf{\textgreek[variant=ancient]{πολλὰς δὲ πόλεις\dots}}\ koordinacija rečeničnih članova pomoću čestica μὲν\dots\ δέ\dots
\item[πολλὰ αὐτῆς ] §~196, §~207
\item[ὄντα] εἰμί biti; a. sg. m. r. ptc. prez. akt.; predikatni particip §~502
\item[ὑπὸ τῷ βαρβάρῳ] ovdje preneseno: pod vlašću\dots; §~80, §~82, prijedložni izraz ὑπὸ + d.: pod\dots; §~418, §~437.B
\item[πολλὰς πόλεις] §~196, §~165
\item[ὑπὸ τυράννων] §~82, prijedložni izraz ὑπὸ + d.: pod\dots; §~418, §~437.B
\item[ἀναστάτους] §~103
\item[γεγενημένας] γίγνομαι postati; a. pl. ž. r. ptc. perf. medpas.; predikatni particip §~502
\item[ἀναστάτους γεγενημένας] ἀνάστατος γίγνομαι biti uništen

\end{description}

%6
{\large
\begin{greek}
\noindent καὶ ταῦτα εἰ μὲν \\
\tabto{2em} δι' ἀσθένειαν \\
ἐπάσχομεν, \\
\tabto{2em} στέργειν \\
ἂν ἦν ἀνάγκη \\
\tabto{2em} τὴν τύχην·\\
ἐπειδὴ δὲ \\
\tabto{2em} διὰ στάσιν \\
\tabto{2em} καὶ τὴν πρὸς ἀλλήλους φιλονικίαν, \\
πῶς οὐκ ἄξιον \\
\tabto{2em} τῶν μὲν \underline{παύσασθαι} \\
\tabto{2em} τὰ δὲ \underline{κωλῦσαι}, \\
\tabto{2em} \underline{εἰδότας} ὅτι \\
\tabto{4em} φιλονικεῖν μέν ἐστιν \\
\tabto{6em} εὖ πραττόντων, \\
\tabto{4em} γνῶναι δὲ τὰ βέλτιστα \\
\tabto{6em} τῶν οἵων ἡμῶν; \\

\end{greek}
}

\begin{description}[noitemsep]
\item[εἰ μὲν δι' ἀσθένειαν] \textbf{\textgreek[variant=ancient]{ἐπειδὴ δὲ διὰ στάσιν\dots}}\ koordinacija rečeničnih članova pomoću čestica μὲν\dots\ δέ\dots
\item[εἰ\dots\ ἐπάσχομεν\dots] \textbf{\textgreek[variant=ancient]{ἂν ἦν ἀνάγκη}} zavisna pogodbena rečenica u irealnom obliku iskazuje nestvarnu pogodbu u prošlosti: da smo\dots\ onda bi\dots
\item[ἐπάσχομεν] πάσχω trpjeti; 1. l. pl. impf. akt.
\item[δι' ἀσθένειαν] §~68, §~90.b, prijedložni izraz διά + a.: zbog, iz\dots; §~418, §~428.A
\item[ἀνάγκη ] §~90.a
\item[ἦν] εἰμί biti; 3. l. sg. impf. akt.
\item[ἂν ἦν] ἂν + prošli indikativ izražava neostvarenu mogućnost u prošlosti: bilo bi nužno\dots
\item[στέργειν] στέργω voljeti; inf. prez. akt. 
\item[τὴν τύχην] §~80, §~82, §~90.a
\item[ἐπειδὴ] zavisni uzročni veznik: zato što\dots
\item[διὰ στάσιν] §~165, prijedložni izraz διὰ + a.: zbog\dots; §~418, §~428.B
\item[τὴν φιλονικίαν] §~80, §~90.b
\item[πρὸς ἀλλήλους] §~212, prijedložni izraz πρὸς + a.: prema\dots; §~418, §~435.A
\item[τὴν πρὸς ἀλλήλους φιλονικίαν] atributni položaj prijedložnog izraza §~375.4
\item[πῶς οὐκ\dots;] zavisna upitna rečenica: zar nije\dots?
\item[ἄξιον] sc.\ ἐστίν; §~103; imenski predikat (s izostavljenom kopulom) Smyth 909
\item[τῶν μὲν\dots\ τὰ δὲ\dots] §~80, §~82; koordinacija rečeničnih članova pomoću čestica μὲν\dots\ δέ\dots
\item[παύσασθαι ] παύω τινός zaustaviti nešto; inf. aor. med.
\item[κωλῦσαι] κωλύω spriječiti; inf. aor. akt.
\item[εἰδότας] οἶδα znati; a. pl. m. r. ptc. perf. akt.; §~130
\item[φιλονικεῖν μέν\dots\ γνῶναι δὲ\dots] koordinacija pomoću čestica μέν\dots\ δὲ izriče suprotnost (ovdje, dvaju infinitiva)
\item[ἐστιν\dots] \textbf{\textgreek[variant=ancient]{εὖ πραττόντων\dots\ τῶν οἵων ἡμῶν}} \textgreek[variant=ancient]{εἶναί τινος} pripadati nekome, biti svojstvo nekoga
\item[φιλονικεῖν] φιλονικέω nadmetati se; inf. prez. akt. 
\item[γνῶναι ] γιγνώσκω spoznati znati; inf. aor. akt.  
\item[εὖ πραττόντων] εὖ πράττω imati uspjeha, napredovati; g. pl. m. r. ptc. prez. akt.; §~139.α
\item[τὰ βέλτιστα] §~80, §~82, §~202
\item[τῶν οἵων ἡμῶν] §~80, §~82, §~443, §~205

\end{description}

%7
{\large
\begin{greek}
\noindent ὁρῶμεν γὰρ \\
τοὺς κινδύνους \\
\tabto{2em} καὶ μεγάλους \\
\tabto{2em} καὶ πανταχόθεν περιεστηκότας· \\
ἐπίστασθε δὲ 
\tabto{2em} ὅτι ἡ μὲν ἀρχὴ \\
\tabto{4em} τῶν κρατούντων τῆς θαλάττης, \\
\tabto{2em} τῶν δὲ χρημάτων \\
\tabto{4em} βασιλεὺς ταμίας, \\
\tabto{2em} τὰ δὲ τῶν Ἑλλήνων σώματα \\
\tabto{4em} τῶν δαπανᾶσθαι δυναμένων, \\
\tabto{2em} ναῦς δὲ \\
\tabto{4em} πολλὰς $\langle$μὲν$\rangle$ \\
\tabto{6em} αὐτὸς κέκτηται, \\
\tabto{4em} πολλὰς δ' \\
\tabto{6em} ὁ τύραννος τῆς Σικελίας. \\

\end{greek}
}

\begin{description}[noitemsep]
\item[ὁρῶμεν\dots\ περιεστηκότας] ὁράω gledati; 1. l. pl. ind. prez. akt; kao glagol spoznavanja \textit{(verbum sentiendi)} otvara mjesto predikatnom participu §~502
\item[γὰρ] čestica najavljuje objašnjenje ili uzrok prethodne tvrdnje: naime\dots
\item[τοὺς κινδύνους] §~80, §~82
\item[καὶ\dots\ καὶ\dots] koordinacija pomoću para sastavnih veznika
\item[μεγάλους] §~196
\item[περιεστηκότας] περιίστημι nanizati oko, gomilati se; a. pl. m. r. ptc. perf. akt.; atributni položaj §~375.5
\item[ἐπίστασθε] ἐπίσταμαι znati; 2. l. pl. ind. prez. medpas.
\item[ἡ μὲν ἀρχὴ τῶν κρατούντων\dots] \textbf{\textgreek[variant=ancient]{τῶν δὲ χρημάτων βασιλεὺς\dots\ τὰ δὲ τῶν Ἑλλήνων σώματα\dots\ ναῦς δὲ\dots}}\ koordinacija pomoću čestica μὲν\dots\ δέ\dots\ δέ\dots\ δέ\dots
\item[ἡ ἀρχὴ] §~80, §~90.a
\item[τῶν κρατούντων] κρατέω vladati; g. pl. m. r. ptc. prez. akt.; §~80, §~82, §~139.α
\item[τῆς θαλάττης] §~80, §~90.a
\item[τῶν χρημάτων] §~80, §~82, §~123
\item[βασιλεὺς] §~175
\item[ταμίας] §~100.b
\item[τὰ σώματα] §~80, §~82, §~123
\item[τῶν Ἑλλήνων] §~80, §~82, §~131
\item[τῶν δυναμένων] δύναμαι moći; g. pl. m. r. ptc. prez. medpas.; §~80, §~82, §~103; glagol otvara mjesto dopuni u infinitivu
\item[δαπανᾶσθαι] δαπανάω potrošiti, iskoristiti, uništiti; inf. prez. medpas. 
\item[ναῦς πολλὰς] §~180, §~196
\item[πολλὰς μὲν\dots\ πολλὰς δ'\dots] koordinacija pomoću čestica μὲν\dots\ δέ\dots\ (i ponavljanja ključnog pridjeva); §~68
\item[αὐτὸς] §~207
\item[κέκτηται] κτάομαι steći; 3. l. sg. ind. perf. medpas.
\item[πολλὰς] §~196
\item[ὁ τύραννος] §~80, §~82
\item[τῆς Σικελίας] §~80, §~90.b

\end{description}

%8
{\large
\begin{greek}
\noindent ὥστε ἄξιον \\
\tabto{2em} τὸν μὲν \\
\tabto{4em} πρὸς ἀλλήλους \\
\tabto{2em} πόλεμον καταθέσθαι, \\
\tabto{2em} τῇ δ' αὐτῇ γνώμῃ \underline{χρωμένους} \\
\tabto{4em} τῆς σωτηρίας \\
\tabto{2em} \underline{ἀντέχεσθαι},\\
\tabto{2em} καὶ \\
\tabto{4em} περὶ μὲν τῶν παρεληλυθότων \\
\tabto{6em} αἰσχύνεσθαι, \\
\tabto{4em} περὶ δὲ τῶν μελλόντων ἔσεσθαι \\
\tabto{6em} δεδιέναι, \\
\tabto{2em} καὶ \\
\tabto{4em} πρὸς τοὺς προγόνους \\
\tabto{6em} ἁμιλλᾶσθαι, \\
\tabto{8em} οἳ \underline{τοὺς μὲν βαρβάρους} \\
\tabto{8em} ἐποίησαν \\
\tabto{10em} τῆς ἀλλοτρίας \underline{ἐπιθυμοῦντας} \\
\tabto{10em} τῆς σφετέρας αὐτῶν \underline{στερεῖσθαι}, \\
\tabto{8em} τοὺς δὲ τυράννους ἐξελάσαντες \\
\tabto{8em} κοινὴν \\
\tabto{10em} ἅπασι \\
\tabto{8em} τὴν ἐλευθερίαν \\
\tabto{8em} κατέστησαν.\\

\end{greek}
}

\begin{description}[noitemsep]
\item[ὥστε] zavisni posljedični veznik: tako da\dots
\item[ἄξιον]  sc.\ ἐστίν; §~103; imenski predikat Smyth 909; izraz otvara mjesto infinitivu i A+I kao predikatnim dopunama (objektima)
\item[τὸν μὲν πόλεμον\dots] \textbf{τῇ δ' αὐτῇ γνώμῃ\dots}\ koordinacija pomoću čestica μὲν\dots\ δέ\dots; §~80, §~82
\item[πρὸς ἀλλήλους] §~212, prijedložni izraz πρὸς + a.: prema\dots; §~418, §~435.4.C
\item[τὸν πρὸς ἀλλήλους πόλεμον] atributni položaj prijedložnog izraza §~375.4
\item[καταθέσθαι] κατατίθημι dovršiti; inf. aor. med.
\item[τῇ δ' αὐτῇ γνώμῃ] §~68, §~80, §~90, §~207, §~90.a
\item[χρωμένους] χράομαι τινί služiti se nečim; stegnuti a. pl. m. r. ptc. prez. medpas.; dio konstrukcije A+I
\item[τῆς σωτηρίας] §~80, §~90.b
\item[ἀντέχεσθαι] ἀντέχω τῆς σωτηρίας tražiti spas; inf. prez. medpas.; dio konstrukcije A+I
\item[τῶν παρεληλυθότων] παρέρχομαι prolaziti; g. pl. sr. r. ptc. perf. akt.; §~130
\item[περὶ μὲν τῶν παρεληλυθότων\dots] \textbf{περὶ δὲ τῶν μελλόντων ἔσεσθαι} koordinacija pomoću čestica μὲν\dots\ δέ\dots
\item[περὶ τῶν παρεληλυθότων] prijedložni izraz περὶ + g.: o\dots; §~418, §~433
\item[αἰσχύνεσθαι] αἰσχύνω sramiti se; inf. prez. medpas. (ovisan o imenskom predikatu ἄξιον)
\item[τῶν μελλόντων] μέλλω namjeravam; g. pl. sr. r. ptc. prez. akt. §~139.α; μέλλω ἔσεσθαι namjeravam biti (opisno umjesto apstraktne imenice ``budućnost'')
\item[περὶ τῶν μελλόντων] prijedložni izraz περὶ + g.: o\dots; §~418, §~433
\item[ἔσεσθαι] εἰμί biti; inf. fut. (med.)
\item[δεδιέναι] δείδω bojati se; inf. perf. akt.
\item[πρὸς τοὺς προγόνους ] §~80, §~82, prijedložni izraz πρὸς + a.: s\dots, protiv\dots; §~418, §~435.C
\item[ἁμιλλᾶσθαι] ἁμιλλάομαι nadmetati se; inf. prez. medpas. (ovisan o imenskom predikatu ἄξιον)
\item[οἳ ] §~215
\item[οἳ\dots\ ἐποίησαν] koji su učinili da\dots; zavisna odnosna rečenica
\item[τοὺς μὲν βαρβάρους] §~80, §~82; dio konstrukcije A+I
\item[ἐποίησαν ] ποιέω činiti, učiniti; 3. l. sg. ind. aor. akt.; otvara mjesto konstrukciji A+I 
\item[τῆς ἀλλοτρίας ] §~80, §~90.b
\item[ἐπιθυμοῦντας ] ἐπιθυμέω željeti; stegnuti a. pl. m. r. ptc. prez. akt.; §~139.α
\item[τῆς σφετέρας αὐτῶν ] §~80, §~90, §~210.3, §~207
\item[στερεῖσθαι] στερέω lišiti, oduzeti; \textit{u pasivu} izgubiti; inf. prez. medpas.; dio konstrukcije A+I
\item[τοὺς δὲ τυράννους ] §~80, §~82
\item[ἐξελάσαντες ] ἐξελαύνω istjerati; n. pl. m. r. ptc. aor. akt.; §~139.β 
\item[κοινὴν τὴν ἐλευθερίαν] §~103, §~80, §~90.b
\item[ἅπασι ] §~193
\item[κατέστησαν] καθίστημι uspostaviti; 3. l. pl. ind. aor. med.

\end{description}


%kraj
