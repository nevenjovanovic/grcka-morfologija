% Unio korekture NZ, NJ 2019-08-08
%\section*{O autoru}



\section*{O tekstu}

Atenski građanin koji bi bio radno nesposoban ili se ne bi mogao uzdržavati od svojeg rada dobivao je od države malu pripomoć; prema ovom govoru, jedan obol na dan (drugi navode iznos od dvaju ili pet obola). Takva bi naknada bila isplaćivana sve dok je tko ne bi osporio prigovorom da primatelj ima dovoljno snage ili sredstava, da nije dovoljno nesposoban za rad, da poznaje zanat kojim bi se mogao uzdržavati. Optuženik se branio pred nadležnim Vijećem pet stotina sudaca.

U ovom Lizijinu govoru optuženik kratko predstavlja sebe u povoljnom, a tužitelja u nepovoljnom svjetlu, i potom dokazuje da mu obitelj nema imovine, da je njegova invalidnost teška, da mu zanat nije dovoljno unosan za preživljavanje.

Kritičari su govor smatrali uzornim jer u njemu, bez obzira na svakodnevnost teme, ima „i plemenitosti i dostojanstva, i oštroumlja, i emocija, i delikatne ironije; ton je ozbiljan, ali ne pedantan; prisan, ali ne banalan; zabavan, ali ne lakrdijaški.''

%\newpage

\section*{Pročitajte naglas grčki tekst.}

Lys.\ Ὑπὲρ τοῦ ἀδυνάτου 6.2
%Naslov prema izdanju

\medskip

{\large
\begin{greek}
\noindent Ἐμοὶ γὰρ ὁ μὲν πατὴρ κατέλιπεν οὐδέν, τὴν δὲ μητέρα τελευτήσασαν πέπαυμαι τρέφων τρίτον ἔτος τουτί, παῖδες δέ μοι οὔπω εἰσὶν οἵ με θεραπεύσουσι. τέχνην δὲ κέκτημαι βραχέα δυναμένην ὠφελεῖν, ἣν αὐτὸς μὲν ἤδη χαλεπῶς ἐργάζομαι, τὸν διαδεξόμενον δ' αὐτὴν οὔπω δύναμαι κτήσασθαι. πρόσοδος δέ μοι οὐκ ἔστιν ἄλλη πλὴν ταύτης, ἣν ἐὰν ἀφέλησθέ με, κινδυνεύσαιμ' ἂν ὑπὸ τῇ δυσχερεστάτῃ γενέσθαι τύχῃ. μὴ τοίνυν, ἐπειδή γε ἔστιν, ὦ βουλή, σῶσαί με δικαίως, ἀπολέσητε ἀδίκως· μηδὲ ἃ νεωτέρῳ καὶ μᾶλλον ἐρρωμένῳ ὄντι ἔδοτε, πρεσβύτερον καὶ ἀσθενέστερον γιγνόμενον ἀφέλησθε· μηδὲ πρότερον καὶ περὶ τοὺς οὐδὲν ἔχοντας κακὸν ἐλεημονέστατοι δοκοῦντες εἶναι νυνὶ διὰ τοῦτον τοὺς καὶ τοῖς ἐχθροῖς ἐλεινοὺς ὄντας ἀγρίως ἀποδέξησθε· μηδ' ἐμὲ τολμήσαντες ἀδικῆσαι καὶ τοὺς ἄλλους τοὺς ὁμοίως ἐμοὶ διακειμένους ἀθυμῆσαι ποιήσητε. καὶ γὰρ ἂν ἄτοπον εἴη, ὦ βουλή, εἰ ὅτε μὲν ἁπλῆ μοι ἦν ἡ συμφορά, τότε μὲν φαινοίμην λαμβάνων τὸ ἀργύριον τοῦτο, νῦν δ' ἐπειδὴ καὶ γῆρας καὶ νόσοι καὶ τὰ τούτοις ἑπόμενα κακὰ προσγίγνεταί μοι, τότε ἀφαιρεθείην.

\end{greek}
}

\section*{Analiza i komentar}


%\newpage

%1

{\large
\begin{greek}
\noindent Ἐμοὶ γὰρ \\
ὁ μὲν πατὴρ \\
κατέλιπεν \\
οὐδέν, \\
τὴν δὲ μητέρα τελευτήσασαν \\
πέπαυμαι τρέφων \\
\tabto{2em} τρίτον ἔτος τουτί, \\
παῖδες δέ \\
μοι οὔπω εἰσὶν \\
\tabto{2em} οἵ με θεραπεύσουσι.\\

\end{greek}
}

\begin{description}[noitemsep]
\item[Ἐμοὶ] §~205
\item[γὰρ ] čestica najavljuje iznošenje dokaza prethodne tvrdnje: naime\dots
\item[ὁ μὲν πατὴρ\dots,] \textbf{τὴν δὲ μητέρα\dots, παῖδες δέ} koordinacija s pomoću čestica  μὲν\dots\ δὲ\dots\ δὲ\dots; §~148; §~127
\item[κατέλιπεν ] καταλείπω ostaviti; 3. l. sg. ind. aor. akt.
\item[οὐδέν] §~224.2
\item[τελευτήσασαν ] τελευτάω umrijeti; a. sg. ž. r. ptc. aor. akt.
\item[πέπαυμαι ] παύω med. uz ptc. prez. drugog glagola \textgreek[variant=ancient]{(npr. τρέφων):} prestati (uz infinitiv, npr. „prestati uzdržavati''); 1. l. sg. ind. perf. medpas.
\item[τρέφων] τρέφω uzdržavati; n. sg. m. r. ptc. prez. akt.; predikatni particip protegnut na subjekt uz glagole koji znače „prestajati'' §~501.c
\item[τρίτον ] §~223
\item[ἔτος ] §~153; akuzativ protezanja u vremenu §~390
\item[τουτί] §~214.2
\item[δέ μοι] §~40
\item[μοι ] §~205; posvojni dativ §~412.2
\item[οὔπω εἰσὶν] §~40
\item[εἰσὶν ] ἔστιν τινί τι imati što; 3. l. pl. ind. prez. akt.
\item[οἵ με] §~40
\item[οἵ ] §~215; uvodi odnosnu rečenicu posljedičnog značenja s indikativom futura §~484
\item[με ] §~205
\item[θεραπεύσουσι] θεραπεύω τινά brinuti se za koga; 3. l. pl. ind. fut. akt.

\end{description}

%\newpage


{\large
\begin{greek}
\noindent τέχνην δὲ \\
κέκτημαι \\
\tabto{2em} βραχέα δυναμένην ὠφελεῖν, \\
\tabto{4em} ἣν \\
\tabto{4em} αὐτὸς μὲν \\
\tabto{6em} ἤδη \\
\tabto{6em} χαλεπῶς \\
\tabto{4em} ἐργάζομαι, \\
\tabto{4em} τὸν διαδεξόμενον δ' αὐτὴν \\
\tabto{6em} οὔπω \\
\tabto{4em} δύναμαι κτήσασθαι.\\

\end{greek}
}

\begin{description}[noitemsep]
\item[τέχνην ] §~90
\item[δὲ ] čestica δέ povezuje rečenicu s prethodnom: a\dots
\item[κέκτημαι ] κτάομαι steći; 1. l. sg. ind. perf. medpas.
\item[βραχέα ] §~167; srednji rod pridjeva u a. pl. kao prilog §~204.2: malo, neznatno
\item[δυναμένην ] δύναμαι moći; a. sg. ž. r. ptc. prez. medpas.
\item[ὠφελεῖν] ὠφελέω pomoći, koristiti; inf. prez. akt.
\item[ἣν ] §~215
\item[αὐτὸς μὲν\dots, τὸν διαδεξόμενον δ'\dots] §~207; koordinacija česticama μὲν\dots\  δὲ: \dots\  a\dots; διαδέχομαι preuzeti, naslijediti; a. sg. m. r. ptc. fut. med.
\item[χαλεπῶς ] §~204
\item[ἐργάζομαι] ἐργάζομαι raditi; 1. l. sg. ind. prez. med.
\item[δ' αὐτὴν] §~68
\item[αὐτὴν] §~207
\item[δύναμαι ] δύναμαι moći; 1. l. sg. ind. prez. med.
\item[κτήσασθαι] κτάομαι steći; inf. aor. medpas.

\end{description}

%3 itd

{\large
\begin{greek}
\noindent πρόσοδος δέ \\
μοι οὐκ ἔστιν ἄλλη \\
\tabto{2em} πλὴν ταύτης, \\
\tabto{4em} ἣν \\
\tabto{4em} ἐὰν ἀφέλησθέ με, \\
\tabto{2em} κινδυνεύσαιμ' ἂν \\
\tabto{4em} ὑπὸ τῇ δυσχερεστάτῃ γενέσθαι τύχῃ.  \\


\end{greek}
}

\begin{description}[noitemsep]
\item[πρόσοδος ] §~82
\item[δέ μοι] §~40
\item[δέ ] čestica δέ povezuje rečenicu s prethodnom: a\dots
\item[μοι ] §~205; posvojni dativ §~412.2
\item[οὐκ ἔστιν ] ἔστιν τινί τι imati što; 3. l. sg. ind. prez. akt.
\item[ἄλλη ] §~212.a; imenski predikat sa zamjenicom kao predikatnim imenom Smyth 910
\item[πλὴν ταύτης] §~417; §~213.2
\item[ἣν ] §~215
\item[ἐὰν ἀφέλησθέ\dots, κινδυνεύσαιμ' ἂν] inačica eventualne (futurske) pogodbene rečenice
\item[ἀφέλησθέ με] §~40
\item[ἀφέλησθέ] ἀφαιρέω oduzeti; 2. l. pl. konj. aor. med.
\item[με] §~205
\item[κινδυνεύσαιμ' ἂν ] §~68; κινδυνεύω biti u opasnosti; 1. l. sg. opt. aor. akt.; potencijal sadašnji §~464.2
\item[ὑπὸ τῇ\dots\ τύχῃ] §~437; §~90
\item[δυσχερεστάτῃ ] §~197
\item[γενέσθαι] γίγνομαι ὑπό τινι postati izložen čemu, zapasti u što; inf. aor. med.


\end{description}

%4


{\large
\begin{greek}
\noindent μὴ τοίνυν, \\
ἐπειδή γε ἔστιν, \\
ὦ βουλή, \\
\tabto{2em} σῶσαί με δικαίως, \\
ἀπολέσητε ἀδίκως· \\
μηδὲ \\
ἃ νεωτέρῳ καὶ μᾶλλον ἐρρωμένῳ ὄντι \\
\tabto{2em} ἔδοτε, \\
πρεσβύτερον καὶ ἀσθενέστερον γιγνόμενον \\
ἀφέλησθε· \\
μηδὲ \\
\tabto{2em} πρότερον \\
\tabto{2em} καὶ περὶ τοὺς οὐδὲν ἔχοντας κακὸν \\
ἐλεημονέστατοι δοκοῦντες εἶναι \\
\tabto{2em} νυνὶ \\
\tabto{2em} διὰ τοῦτον \\
τοὺς καὶ τοῖς ἐχθροῖς ἐλεινοὺς ὄντας \\
ἀγρίως ἀποδέξησθε· \\
μηδ' \\
ἐμὲ \\
τολμήσαντες ἀδικῆσαι \\
\tabto{2em} καὶ τοὺς ἄλλους \\
\tabto{4em} τοὺς ὁμοίως ἐμοὶ διακειμένους \\
ἀθυμῆσαι ποιήσητε. \\

\end{greek}
}

\begin{description}[noitemsep]
\item[μὴ\dots\ ἀπολέσητε] ἀπόλλυμι upropastiti; 2. l. pl. konj. aor. akt.; prohibitivni konjunktiv §~463.3
\item[ἐπειδή γε ἔστιν] §~40
\item[γε] čestica izražava limitativnost; govornika zanima isključivo ono što je izraženo zavisnom rečenicom, bez obzira na ostale mogućnosti (slično hrvatskom „barem'')
\item[ἔστιν] εἰμί biti; oksitona ἔστιν znači „moguće je'' (= ἔξεστιν); 3. l. sg. ind. prez. akt.; §~204
\item[ὦ βουλή] §~90
\item[σῶσαί με] §~40
\item[σῶσαί ] σῴζω spašavati; inf. aor. akt.
\item[με] §~205
\item[ἀδίκως] §~204
\item[μὴ\dots\ μηδὲ\dots\ μηδὲ\dots\ μηδ'\dots] koordinacija negacijama (s konjunktivom): nemojte\dots\ i nemojte\dots\ i nemojte\dots\ i nemojte\dots
\item[ἃ ] §~215; uvodi odnosnu rečenicu §~481
\item[νεωτέρῳ] §~197
\item[μᾶλλον ἐρρωμένῳ ὄντι ] §~204.3; ῥώννυμι pas. biti u snazi; d. sg. m. r. ptc. perf. medpas.; εἰμί biti; d. sg. m. r. ptc. prez. akt.
\item[ἔδοτε] δίδωμι dati; 2. l. pl. ind. aor. akt.
\item[πρεσβύτερον\dots\ ἀσθενέστερον] §~197
\item[γιγνόμενον ] sc.\ με; γίγνομαι postati; a. sg. m. r. ptc. prez. medpas.
\item[ἀφέλησθε] ἀφαιρέω med. τινά τι oduzeti komu što; 2. l. pl. konj. aor. med.
\item[περὶ τοὺς\dots\ ἔχοντας ]  §~433; ἔχω imati; a. pl. m. r. ptc. prez. akt.; poimeničenje članom §~373
\item[οὐδὲν\dots\ κακὸν] §~224.2; §~103
\item[εἶναι] εἰμί biti; inf. prez. akt.
\item[ἐλεημονέστατοι ] §~197; imenski predikat s pridjevom kao predikatnim imenom Smyth 910; N+I
\item[δοκοῦντες ] δοκέω činiti se, vrijediti za što, biti na glasu; n. pl. m. r. ptc. prez. akt.
\item[διὰ τοῦτον ] zbog njega, na njegov poticaj; §~428; §~213.2
\item[τοὺς\dots\ ἐλεινοὺς ὄντας] §~373; §~103; εἰμί biti; a. pl. m. r. ptc. prez. akt.
\item[τοῖς ἐχθροῖς] §~373; §~103; individualni član često zastupa posvojnu zamjenicu §~370.1.1
\item[ἀγρίως ] §~204
\item[μηδὲ\dots\ ἀποδέξησθε] ἀποδέχομαι prihvatiti, primiti; 2. l. pl. konj. aor. med.; prohibitivni konjunktiv §~463.3
\item[μηδ' ἐμὲ] §~68
\item[ἐμὲ ] §~205
\item[τολμήσαντες ] τολμάω usuđivati se; n. pl. m. r. ptc. aor. akt.
\item[ἀδικῆσαι] ἀδικέω τινά nanositi nepravdu komu; inf. aor. akt.
\item[τοὺς ἄλλους τοὺς\dots\ διακειμένους] poimeničenje članom §~373; §~212.a; διάκειμαι + prilog: biti u (kojem) stanju; a. pl. m. r. ptc. prez. medpas.
\item[ὁμοίως ] §~204
\item[ἐμοὶ ] §~205
\item[ἀθυμῆσαι ] ἀθυμέω biti očajan; inf. aor. akt.
\item[μηδ'\dots\ ποιήσητε] ποιέω + infinitiv: natjerati koga da\dots; 2. l. pl. konj. aor. akt.; prohibitivni konjunktiv §~463.3

\end{description}


%5


{\large
\begin{greek}
\noindent καὶ γὰρ \\
ἂν ἄτοπον εἴη, \\
ὦ βουλή, \\
εἰ \\
\tabto{2em} ὅτε μὲν \\
\tabto{2em} ἁπλῆ μοι ἦν \\
\tabto{2em} ἡ συμφορά, \\
\tabto{2em} τότε μὲν \\
φαινοίμην λαμβάνων τὸ ἀργύριον τοῦτο, \\
νῦν δ' \\
\tabto{2em} ἐπειδὴ καὶ γῆρας καὶ νόσοι καὶ τὰ τούτοις ἑπόμενα κακὰ \\
\tabto{2em} προσγίγνεταί μοι, \\
\tabto{2em} τότε \\
ἀφαιρεθείην.\\

\end{greek}
}

\begin{description}[noitemsep]
\item[καὶ γὰρ ] kombinacija čestica služi za objašnjavanje i naglašavanje: čak\dots
\item[ἂν\dots\ εἴη\dots] \textbf{εἰ\dots\ φαινοίμην\dots\ ἀφαιρεθείην} pogodbena potencijalna rečenica; moguće je da se vrši i posljedica
\item[εἴη] εἰμί biti; 3. l. sg. opt. prez. akt.
\item[ἄτοπον] §~106; imenski predikat s pridjevom kao predikatnim imenom Smyth 910
\item[ὦ βουλή] §~90
\item[φαινοίμην\dots\ ἀφαιρεθείην] φαίνω pas. + ptc. pokazati se da; 1. l. sg. opt. prez. medpas.; ἀφαιρέω oduzeti; 1. l. sg. opt. aor. pas.
\item[ὅτε μὲν\dots\ τότε μὲν\dots\ νῦν δ'\dots] veznik ὅτε uvodi vremensku rečenicu §~487; udvajanje μὲν u koordinaciji μὲν\dots\ δέ\dots: sadržaj prvog dijela antiteze odviše je složen da bi „stao'' u jednu surečenicu
\item[ἁπλῆ] §~107; imenski predikat s pridjevom kao predikatnim imenom Smyth 910
\item[ἁπλῆ μοι] §~40
\item[μοι ] §~205
\item[ἦν ] εἰμί biti; 3. l. sg. impf. akt.
\item[ἡ συμφορά] §~90
\item[λαμβάνων] λαμβάνω uzimati; n. sg. m. r. ptc. prez. akt.; dopuna uz φαινοίμην, predikatni particip protegnut na subjekt §~501.b
\item[τὸ ἀργύριον τοῦτο] §~82; §~213.2
\item[νῦν δ' ἐπειδὴ\dots\  τότε\dots] §~68; koordinacija vremenskog veznika i priloga
\item[γῆρας] §~159
\item[νόσοι] §~82
\item[τὰ\dots\ κακὰ] §~373;  §~103
\item[τούτοις ] §~213.2
\item[ἑπόμενα] ἕπομαί τινι slijediti što ili koga; n. pl. s. r. ptc. prez. med.
\item[προσγίγνεταί μοι] §~40
\item[προσγίγνεταί ] προσγίγνομαί τινι dogoditi se komu; 3. l. sg. ind. prez. medpas.
\item[μοι] §~205

\end{description}

%kraj
