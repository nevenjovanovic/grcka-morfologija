% Unesi ispravke NČ 2019-09-22
%\section*{O autoru}



\section*{O tekstu}

U ovom dijelu apologije mitske ljepotice Helene retor i sofist sa Sicilije dokazuje da je Helena samo \textit{pretrpjela} nepravdu i sramotu; Paris, njezin otmičar, zgriješio je nakanom, kršenjem zakonitosti i djelom; Helena je zapravo žrtva. Na prigovor da je Helenin postupak skrivila riječ (λόγος), Gorgija odgovara prikazujući moć tog načina komunikacije i najvišeg oblika uporabe riječi – poezije.

%\newpage

\section*{Pročitajte naglas grčki tekst.}
Gorg. Helenae encomium Fr.\ 11.41
%Naslov prema izdanju

\medskip

{\large
\begin{greek}
\noindent Εἰ δὲ βίαι ἡρπάσθη καὶ ἀνόμως ἐβιάσθη καὶ ἀδίκως ὑβρίσθη, δῆλον ὅτι ὁ $\langle$μὲν$\rangle$ ἁρπάσας ὡς ὑβρίσας ἠδίκησεν, ἡ δὲ ἁρπασθεῖσα ὡς ὑβρισθεῖσα ἐδυστύχησεν. ἄξιος οὖν ὁ μὲν ἐπιχειρήσας βάρβαρος βάρβαρον ἐπιχείρημα καὶ λόγωι καὶ νόμωι καὶ ἔργωι, λόγωι μὲν αἰτίας, νόμωι δὲ ἀτιμίας, ἔργωι δὲ ζημίας τυχεῖν· ἡ δὲ βιασθεῖσα καὶ τῆς πατρίδος στερηθεῖσα καὶ τῶν φίλων ὀρφανισθεῖσα πῶς οὐκ ἂν εἰκότως ἐλεηθείη μᾶλλον ἢ κακολογηθείη; ὁ μὲν γὰρ ἔδρασε δεινά, ἡ δὲ ἔπαθε· δίκαιον οὖν τὴν μὲν οἰκτῖραι, τὸν δὲ μισῆσαι. 

εἰ δὲ λόγος ὁ πείσας καὶ τὴν ψυχὴν ἀπατήσας, οὐδὲ πρὸς τοῦτο χαλεπὸν ἀπολογήσασθαι καὶ τὴν αἰτίαν ἀπολύσασθαι ὧδε. λόγος δυνάστης μέγας ἐστίν, ὃς σμικροτάτωι σώματι καὶ ἀφανεστάτωι θειότατα ἔργα ἀποτελεῖ· δύναται γὰρ καὶ φόβον παῦσαι καὶ λύπην ἀφελεῖν καὶ χαρὰν ἐνεργάσασθαι καὶ ἔλεον ἐπαυξῆσαι. ταῦτα δὲ ὡς οὕτως ἔχει δείξω· δεῖ δὲ καὶ δόξηι δεῖξαι τοῖς ἀκούουσι· τὴν ποίησιν ἅπασαν καὶ νομίζω καὶ ὀνομάζω λόγον ἔχοντα μέτρον· ἧς τοὺς ἀκούοντας εἰσῆλθε καὶ φρίκη περίφοβος καὶ ἔλεος πολύδακρυς καὶ πόθος φιλοπενθής, ἐπ' ἀλλοτρίων τε πραγμάτων καὶ σωμάτων εὐτυχίαις καὶ δυσπραγίαις ἴδιόν τι πάθημα διὰ τῶν λόγων ἔπαθεν ἡ ψυχή.

\end{greek}

}

\section*{Analiza i komentar}

%1

{\large
\begin{greek}
\noindent Εἰ δὲ \\
\tabto{2em} βίαι ἡρπάσθη \\
\tabto{2em} καὶ ἀνόμως ἐβιάσθη \\
\tabto{2em} καὶ ἀδίκως ὑβρίσθη, \\
δῆλον ὅτι \\
ὁ $\langle$μὲν$\rangle$ ἁρπάσας \\
\tabto{2em} ὡς ὑβρίσας \\
ἠδίκησεν, \\
ἡ δὲ ἁρπασθεῖσα \\
\tabto{2em} ὡς ὑβρισθεῖσα \\
ἐδυστύχησεν.\\

\end{greek}
}

\begin{description}[noitemsep]
\item[δὲ] čestica povezuje rečenicu s prethodnom rečenicom: a\dots
\item[Εἰ\dots\ ἡρπάσθη\dots\ ἐβιάσθη] zavisna pogodbena rečenica: da\dots
\item[βίαι] §~97; dativ načina, §~414.3; \textit{iota adscriptum}, §~8 – takvo je pisanje moguće i kod malih slova (koja su se počela upotrebljavati tek u bizantsko doba)
\item[ἡρπάσθη] ἁρπάζω ugrabiti, oteti; 3. l. sg. ind. aor. pas.
\item[ἐβιάσθη] βιάζω prisiliti, obeščastiti, silovati; 3. l. sg. ind. aor. pas.
\item[ὑβρίσθη] ὑβρίζω obijestan biti, obijesno se ponašati, zlostavljati; 3. l. sg. ind. aor. pas.
\item[δῆλον ὅτι] očito je da, jasno je da\dots
\item[ὁ $\langle$μὲν$\rangle$ ἁρπάσας] atributni particip §~499, 370.1, 370.2; riječ u prelomljenim zagradama označava priređivačev dodatak – ne postoji u rukopisnim izvorima, ali morala je stajati u autorovoj inačici
\item[ὁ $\langle$μὲν$\rangle$\dots\ ἡ δὲ\dots] koordinacija parom čestica
\item[ἁρπάσας] ἁρπάζω ugrabiti, oteti; n. sg. m. r. ptc. aor. akt.; §~139.β
\item[ὑβρίσας ] ὑβρίζω obijestan biti, obijesno se ponašati, zlostavljati; n. sg. m. r. ptc. aor. akt.; §~139.β
\item[ἠδίκησεν] ἀδικέω biti kriv, nepravdu činiti; 3. l. sg. ind. aor. akt. 
\item[ὡς ὑβρίσας ] zavisna uzročna rečenica: jer\dots
\item[ἡ δὲ] §~80, 90, 370.1, 370.2
\item[ἁρπασθεῖσα ] ἁρπάζω ugrabiti, oteti; n. sg. ž. r. ptc. aor. pas. §~97.β
\item[ὑβρισθεῖσα] ὑβρίζω zlostavljati, \textit{pasivno} biti silovan; n. sg. ž. r. ptc. aor. pas. §~97.β
\item[ἐδυστύχησεν] δυστυχέω trpjeti nesreću; 3. l. sg. ind. aor. akt. 
\item[ὡς ὑβρισθεῖσα ] zavisna uzročna rečenica: jer je bila\dots

\end{description}


{\large
\begin{greek}
\noindent ἄξιος οὖν\\
ὁ μὲν ἐπιχειρήσας βάρβαρος \\
\tabto{2em} βάρβαρον ἐπιχείρημα \\
\tabto{2em} καὶ λόγωι καὶ νόμωι καὶ ἔργωι \\
\tabto{4em} λόγωι μὲν αἰτίας, \\
\tabto{4em} νόμωι δὲ ἀτιμίας, \\
\tabto{4em} ἔργωι δὲ ζημίας \\
τυχεῖν· \\
ἡ δὲ βιασθεῖσα \\
\tabto{2em} καὶ τῆς πατρίδος στερηθεῖσα\\
\tabto{2em} καὶ τῶν φίλων ὀρφανισθεῖσα \\
πῶς οὐκ ἂν \\
\tabto{2em} εἰκότως \\
ἐλεηθείη μᾶλλον \\
ἢ κακολογηθείη;\\

\end{greek}
}

\begin{description}[noitemsep]
\item[ἄξιος] §~103
\item[ὁ μὲν ἐπιχειρήσας] ἐπιχειρέω nakaniti, naumiti, primiti se čega; n. sg. m. r. ptc. aor. akt.; particip s članom §~499; §~139.β
\item[ὁ μὲν\dots\ ἡ δὲ\dots] koordinacija parom čestica
\item[βάρβαρος] negrk; §~82 
\item[βάρβαρον ἐπιχείρημα] nasilni poduhvat; §~103, §~123
\item[λόγωι καὶ νόμωι καὶ ἔργωι ] §~82; dativ načina §~414.3
\item[μὲν\dots\ δὲ\dots] jedno\dots\ drugo, sad\dots\ sad; čestice označavaju suprotnost
\item[αἰτίας] §~90.b
\item[ἀτιμίας] §~90.b
\item[ζημίας] §~90.b
\item[τυχεῖν] τυγχάνω zadobiti; inf. aor. akt.
\item[ἡ δὲ βιασθεῖσα] βιάζω prisiliti, obeščastiti, silovati; n. sg. ž. r. ptc. aor. pas.; §~97.β
\item[τῆς πατρίδος] §~80, §~90, §~123
\item[στερηθεῖσα] στερέω τινός lišiti čega; n. sg. ž. r. ptc. aor. pas.; §~97.β
\item[τῶν φίλων] §~80, §~97, §~82
\item[ὀρφανισθεῖσα] ὀρφανίζω τινός odvojiti od čega ili koga; n. sg. ž. r. ptc. aor. pas.; §~97.β
\item[πῶς οὐκ\dots] upitna rečenica; §~221, §~520
\item[ἂν] uz optativ čestica ἂν iskazuje mogućnost u sadašnjosti: mogu + inf., mogao bih + inf.
\item[ἂν ἐλεηθείη] ἐλεέω smilovati se, zadobiti smilovanje; 3. l. sg. opt. aor. pas.
\item[κακολογηθείη] κακολογέω grditi, psovati; 3. l. sg. opt. aor. pas.

\end{description}

%3 itd

{\large
\begin{greek}
\noindent ὁ μὲν γὰρ ἔδρασε δεινά, \\
ἡ δὲ ἔπαθε·\\
δίκαιον οὖν \\
\tabto{2em} τὴν μὲν οἰκτῖραι, \\
\tabto{2em} τὸν δὲ μισῆσαι.\\

\end{greek}
}

\begin{description}[noitemsep]
\item[ὁ μὲν\dots\ ἡ δὲ] on\dots\ ona\dots; koordinacija parom čestica; §~80, §~82, §~90; §~370
\item[γὰρ] jer, naime; čestica, najavljuje iznošenje dokaza ili objašnjenje prethodne misli
\item[ἔδρασε] δράω činiti, raditi; 3. l. sg. ind. aor. akt. 
\item[δεινά] §~103, §~204.2
\item[ἔπαθε] πάσχω trpjeti; 3. l. sg. ind. aor. akt.
\item[δίκαιον ] §~103
\item[δίκαιον] sc.\ ἐστίν (izostavljena kopula); εἰμί biti; 3. l. sg. ind. prez. akt.; imenski predikat Smyth 909
\item[τὴν μὲν\dots\ τὸν δὲ\dots] jednu\dots\ drugog\dots, §~82, §~90, §~370
\item[οἰκτῖραι] οἰκτίρω sažalijevati; inf. aor. akt.
\item[μισῆσαι] μισέω mrziti; inf. aor. akt.

\end{description}

\newpage
%4

{\large
\begin{greek}
\noindent εἰ δὲ \\
\tabto{2em} λόγος ὁ πείσας \\
\tabto{2em} καὶ τὴν ψυχὴν ἀπατήσας, \\
οὐδὲ \\
\tabto{2em} πρὸς τοῦτο \\
\tabto{4em} χαλεπὸν ἀπολογήσασθαι \\
\tabto{2em} καὶ τὴν αἰτίαν \\
\tabto{4em} ἀπολύσασθαι ὧδε.\\

\end{greek}
}

\begin{description}[noitemsep]
\item[δὲ] čestica povezuje rečenicu s prethodnom rečenicom: a\dots
\item[λόγος] sc.\ ἐστίν, izostavljena kopula; εἰμί biti; 3. l. sg. ind. prez. akt.; imenski predikat Smyth 909
\item[εἰ\dots\ λόγος] sc.\ \textbf{ἐστίν} – \textbf{χαλεπὸν} sc.\ \textbf{ἐστίν} zavisna pogodbena rečenica: ako\dots
\item[ὁ πείσας] πείθω nagovoriti, uvjeriti; n. sg. m. r. ptc. aor. akt.; particip s članom, §~499; §~139.β 
\item[τὴν ψυχὴν] §~80, §~90.a
\item[ἀπατήσας] ἀπατάω zavarati; n. sg. m. r. ptc. aor. akt.; particip s članom, §~499; §~139.β
\item[πρὸς τοῦτο] za to (prijedložni izraz) §~213.2; πρὸς + a.: k\dots, prema \dots; §~418, §~435.C
\item[χαλεπὸν] §~113
\item[χαλεπὸν] sc.\ ἐστίν (izostavljena kopula); εἰμί biti; 3. l. sg. ind. prez. akt.; imenski predikat Smyth 909
\item[ἀπολογήσασθαι ] ἀπολογέομαι braniti se; inf. aor. med.
\item[τὴν αἰτίαν ] §~80, §~90.b
\item[ἀπολύσασθαι ] ἀπολύω osloboditi; inf. aor. med.
\item[ὧδε] §~213.1; §~214.3

\end{description}

%5

{\large
\begin{greek}
\noindent λόγος δυνάστης μέγας ἐστίν, \\
\tabto{2em} ὃς \\
\tabto{4em} σμικροτάτωι σώματι καὶ ἀφανεστάτωι \\
\tabto{2em} θειότατα ἔργα ἀποτελεῖ·\\
\tabto{2em} δύναται γὰρ \\
\tabto{4em} καὶ φόβον παῦσαι \\
\tabto{4em} καὶ λύπην ἀφελεῖν \\
\tabto{4em} καὶ χαρὰν ἐνεργάσασθαι \\
\tabto{4em} καὶ ἔλεον ἐπαυξῆσαι.\\

\end{greek}
}

\begin{description}[noitemsep]
\item[λόγος] §~82
\item[δυνάστης ] §~100.a
\item[μέγας] §~196
\item[ἐστίν] εἰμί biti; 3. l. sg. ind. prez. akt.
\item[δυνάστης μέγας ἐστίν] imenski predikat Smyth 909
\item[ὃς] §~215, §~216
\item[ἀποτελεῖ] ἀποτελέω izvršiti; 3. l. sg. ind. prez. akt.
\item[ὃς\dots\ ἀποτελεῖ] zavisna odnosna rečenica §~481, §~482
\item[σμικροτάτωι] §~103, §~197, §~202.3
\item[σώματι] §~123
\item[ἀφανεστάτωι] §~153, §~194.2, §~197
\item[θειότατα] §~103, §~197
\item[ἔργα] §~82
\item[δύναται] δύναμαι moći; 3. l. sg. ind. prez. medpas.; glagol otvara mjesto dopunama u infinitivu
\item[γὰρ ] jer, naime; čestica najavljuje iznošenje dokaza ili objašnjenja prethodne misli
\item[φόβον] §~82
\item[παῦσαι] παύω zaustaviti; inf. aor. akt.
\item[λύπην] §~90.a
\item[ἀφελεῖν] ἀφαιρέω oduzeti; inf. aor. akt.
\item[χαρὰν] §~97.α
\item[ἐνεργάσασθαι] ἐνεργάζομαι prouzročiti; polučiti; inf. aor. med.
\item[ἔλεον] §~82
\item[ἐπαυξῆσαι] ἐπαυξάνω povećati; inf. aor. akt.

\end{description}

%6

{\large
\begin{greek}
\noindent ταῦτα δὲ \\
\tabto{2em} ὡς οὕτως ἔχει \\
δείξω·\\
δεῖ δὲ καὶ 	\\
\tabto{2em} δόξηι δεῖξαι \\
\tabto{2em} τοῖς ἀκούουσι·\\
τὴν ποίησιν ἅπασαν \\
καὶ νομίζω καὶ ὀνομάζω \\
\tabto{2em} λόγον ἔχοντα μέτρον·\\
\tabto{4em} ἧς \\
\tabto{4em} τοὺς ἀκούοντας \\
\tabto{4em} εἰσῆλθε \\
\tabto{6em} καὶ φρίκη περίφοβος \\
\tabto{6em} καὶ ἔλεος πολύδακρυς \\
\tabto{6em} καὶ πόθος φιλοπενθής, \\
\tabto{4em} ἐπ' ἀλλοτρίων τε πραγμάτων καὶ σωμάτων \\
\tabto{6em} εὐτυχίαις καὶ δυσπραγίαις \\
\tabto{4em} ἴδιόν τι πάθημα \\
\tabto{6em} διὰ τῶν λόγων \\
\tabto{4em} ἔπαθεν ἡ ψυχή.\\

\end{greek}
}

\begin{description}[noitemsep]
\item[δὲ] čestica povezuje rečenicu s prethodnom: a\dots
\item[δείξω] δείκνυμι pokazati; 1. l. sg. ind. fut. akt.
\item[οὕτως] §~213.2, §~214.3, 
\item[ἔχει] ἔχω imati; držati; 3. l. sg. ind. prez. akt.
\item[οὕτως ἔχει] tako je
\item[ὡς\dots\ ἔχει] zavisna izrična rečenica: da\dots
\item[δὲ καὶ] a i\dots; česticu δὲ, koja povezuje rečenicu s prethodnom, nadopunjava sastavni veznik
\item[δεῖ] δεῖ treba (bezlično); 3. l. sg. ind. prez. akt.; glagol otvara mjesto dopunama u infinitivu
\item[δόξηι] §~97.b
\item[δεῖξαι] δείκνυμι pokazati; inf. aor. akt.
\item[τοῖς ἀκούουσι] ἀκούω slušati, čuti; d. pl. m. r. ptc. prez. akt.; §~139.α
\item[τὴν ποίησιν] §~80, §~90, §~165
\item[ἅπασαν] §~193
\item[νομίζω] νομίζω smatrati; 1. l. sg. ind. prez. akt.
\item[ὀνομάζω] ὀνομάζω nazivati, zvati; 1. l. sg. ind. prez. akt.
\item[λόγον] §~82
\item[ἔχοντα] ἔχω imati; a. sg. m. r. ptc. prez. akt.; §~139.α
\item[μέτρον] §~82
\item[ἧς] §~215
\item[τοὺς ἀκούοντας] §~80, ἀκούω: čuti, slušati; a. pl. m. r. ptc. prez. akt.;  §~139.α
\item[εἰσῆλθε] εἰσέρχομαί τι ući u što; 3. l. sg. ind. aor. akt.
\item[φρίκη ] §~90
\item[περίφοβος] §~103, §~106
\item[ἔλεος ] §~82
\item[πολύδακρυς] §~173, pridjev s jednim završetkom za muški i ženski rod
\item[πόθος ] §~82
\item[φιλοπενθής] §~153
\item[τε\dots\ καὶ\dots] i\dots\ i\dots, ne samo\dots\ nego i\dots
\item[ἐπ'] \begin{greek}\textbf{(= ἐπί) ἀλλοτρίων τε πραγμάτων καὶ σωμάτων εὐτυχίαις καὶ δυσπραγίαις}\end{greek} §~68, §~103, §~123, §~90.b; prijedložni izraz §~418, §~436.B
\item[ἴδιόν ] §~103
\item[τι ] §~217, §~218; §~39, §~40
\item[πάθημα ] §~123
\item[διὰ τῶν λόγων ] prijedložni izraz; διὰ + g.: kroz\dots; §~418, §~428.A
\item[ἔπαθεν ] πάσχω pretrpjeti; 3. l. sg. ind. aor. akt.
\item[ἡ ψυχή] §~80, §~90, §~190.a

\end{description}

%kraj
