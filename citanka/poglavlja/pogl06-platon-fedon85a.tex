% Redaktura NZ
%\section*{O autoru}



\section*{O tekstu}

\textit{Fedon}, Φαίδων, pripada dijalozima zrelog ili srednjeg razdoblja (oko 387.\ – 365.\ pr.~Kr.) atenskog filozofa Platona (427.\ – 347.\ pr.~Kr.). U tom razdoblju autor razvija učenje o idejama, ἰδέαι ili εἴδη. \textit{Fedon} je četvrti i posljednji dijalog s temom posljednjeg dana u životu Platonova učitelja Sokrata (469.\ – 399.) kojeg je atenska država osudila na smrt zbog bezbožnosti i kvarenja mladeži. Iz perspektive Fedona iz Elide, još jednog od Sokratovih učenika, djelo izvještava o Sokratovoj raspravi s prijateljima o besmrtnosti, a završava eshatološkim mitom o zagrobnom životu te opisom Sokratove smrti. 

U ovom odlomku, pošto je izložio suprotnost osjetilnog svijeta i svijeta ideja te tijela i duše, Sokrat govori što će se nakon smrti dogoditi s čistim dušama; za razliku od nečistih, koje će se vratiti u različite vrste životinja, prema stanju svojih vrlina, filozofi će se pridružiti bogovima. Nakon duže šutnje, Sokrat sugerira sugovornicima Simiji i Kebetu da se ne brinu zbog prigovora na njegove teze jer on je, poput labuda, u službi Apolona i može proreći što će se dogoditi nakon smrti.

\newpage

\section*{Pročitajte naglas grčki tekst.}

%Naslov prema izdanju
Plat.\ Phaedo 85a

\medskip

{\large
\begin{greek}
\noindent Ἐπειδὰν οἱ κύκνοι αἴσθωνται ὅτι δεῖ αὐτοὺς ἀποθανεῖν, ᾄδοντες καὶ ἐν τῷ πρόσθεν χρόνῳ, τότε δὴ πλεῖστα καὶ κάλλιστα ᾄδουσι, γεγηθότες ὅτι μέλλουσι παρὰ τὸν θεὸν ἀπιέναι οὗπέρ εἰσι θεράποντες. οἱ δ' ἄνθρωποι διὰ τὸ αὑτῶν δέος τοῦ θανάτου καὶ τῶν κύκνων καταψεύδονται, καί φασιν αὐτοὺς θρηνοῦντας τὸν θάνατον ὑπὸ λύπης ἐξᾴδειν, καὶ οὐ λογίζονται ὅτι οὐδὲν ὄρνεον ᾄδει ὅταν πεινῇ ἢ ῥιγῷ ἤ τινα ἄλλην λύπην λυπῆται, οὐδὲ αὐτὴ ἥ τε ἀηδὼν καὶ χελιδὼν καὶ ὁ ἔποψ, ἃ δή φασι διὰ λύπην θρηνοῦντα ᾄδειν. ἀλλ' οὔτε ταῦτά μοι φαίνεται λυπούμενα ᾄδειν οὔτε οἱ κύκνοι, ἀλλ' ἅτε οἶμαι τοῦ ᾿Απόλλωνος ὄντες, μαντικοί τέ εἰσι καὶ προειδότες τὰ ἐν ῞Αιδου ἀγαθὰ ᾄδουσι καὶ τέρπονται ἐκείνην τὴν ἡμέραν διαφερόντως ἢ ἐν τῷ ἔμπροσθεν χρόνῳ. ἐγὼ δὲ καὶ αὐτὸς ἡγοῦμαι ὁμόδουλός τε εἶναι τῶν κύκνων καὶ ἱερὸς τοῦ αὐτοῦ θεοῦ, καὶ οὐ χεῖρον ἐκείνων τὴν μαντικὴν ἔχειν παρὰ τοῦ δεσπότου,  οὐδὲ δυσθυμότερον αὐτῶν τοῦ βίου ἀπαλλάττεσθαι.

\end{greek}

}

%\newpage


\section*{Analiza i komentar}


%1

{\large
\noindent Ἐπειδὰν οἱ κύκνοι αἴσθωνται \\
\tabto{2em} ὅτι δεῖ \\
\tabto{4em} αὐτοὺς ἀποθανεῖν, \\
ᾄδοντες καὶ ἐν τῷ πρόσθεν χρόνῳ, \\
τότε δὴ πλεῖστα καὶ κάλλιστα ᾄδουσι,  \\
γεγηθότες \\
\tabto{2em} ὅτι μέλλουσι \\
\tabto{4em} παρὰ τὸν θεὸν \\
\tabto{2em} ἀπιέναι \\
\tabto{4em} οὗπέρ εἰσι θεράποντες.\\

}

\begin{description}[noitemsep]

\item[οἱ κύκνοι ] §~82
\item[αἴσθωνται ] αἰσθάνομαι osjećati; 3. l. pl. konj. aor. med.; ἐπειδάν kad god, s konjunktivom izriče iterativnu radnju za sadašnjost, §~488
\item[δεῖ] δεῖ treba (bezličan glagol); 3. l. sg. ind. prez. akt.; nužna je dopuna, ovdje A+I
\item[αὐτοὺς] §~207; ovdje imenski dio A+I
\item[ἀποθανεῖν ] ἀποθνῄσκω umrijeti; inf. aor. akt.; ovdje glagolski dio A+I
\item[ᾄδοντες] ᾄδω pjevati; n. pl. m. ptc. prez. akt.; odgovara priložnoj, tj. dopusnoj rečenici: premda\dots
\item[ἐν τῷ\dots\ χρόνῳ] §~82, §~426
\item[ἐν τῷ πρόσθεν χρόνῳ] prilog vrši službu atributa i nalazi se u atributnom položaju; §~375
\item[δὴ] naglašava prethodni prilog vremena: tek\dots
\item[πλεῖστα] §~202
\item[κάλλιστα] §~200
\item[ᾄδουσι] ᾄδω pjevati; 3. l. pl. ind. prez. akt.
\item[γεγηθότες ] γηθέω radovati se; n. pl. m. ptc. perf. akt.; vrši službu priložne oznake uzroka: zato što se\dots
\item[μέλλουσι] μέλλω trebati; 3. l. pl. ind. prez. akt.
\item[παρὰ τὸν θεὸν ] §~82, §~434
\item[ἀπιέναι] ἄπειμι otići; inf. prez. akt.
\item[οὗπέρ εἰσι] §~40
\item[οὗπέρ] §~215, §~216
\item[εἰσι] εἰμί biti; 3. l. pl. ind. prez. (akt.)
\item[θεράποντες] §~139
\end{description}

{\large
\noindent οἱ δ' ἄνθρωποι \\
\tabto{2em} διὰ τὸ αὑτῶν δέος \\
\tabto{4em} τοῦ θανάτου \\
καὶ τῶν κύκνων καταψεύδονται, \\
καί φασιν \\
\tabto{2em} \underline{αὐτοὺς θρηνοῦντας} τὸν θάνατον \\
\tabto{4em} ὑπὸ λύπης \\
\tabto{2em} \underline{ἐξᾴδειν}, \\
καὶ οὐ λογίζονται \\
\tabto{2em} ὅτι οὐδὲν ὄρνεον ᾄδει \\
\tabto{4em} ὅταν πεινῇ ἢ ῥιγῷ ἤ τινα ἄλλην λύπην λυπῆται, \\
\tabto{2em} οὐδὲ αὐτὴ ἥ τε ἀηδὼν καὶ χελιδὼν καὶ ὁ ἔποψ, \\
\tabto{4em} \underline{ἃ} δή φασι \\
\tabto{6em} διὰ λύπην \\
\tabto{4em} \underline{θρηνοῦντα ᾄδειν}.\\

}

\begin{description}[noitemsep]

\item[οἱ δ' ἄνθρωποι] §~82; čestica δέ, ovdje elidirana (§~68), povezuje rečenicu s prethodnom: a\dots
\item[διὰ τὸ\dots\ δέος] §~428, §~153
\item[αὑτῶν] §~208, §~209
\item[τοῦ θανάτου] §~82;  δέος τινος strah od čega, objektni genitiv § 394
\item[τῶν κύκνων] §~82
\item[τῶν κύκνων καταψεύδονται] καταψεύδομαί τινος lažno govoriti o čemu; 3. l. pl. ind. prez. medpas.
\item[καί φασιν] §~40
\item[φασιν] φημί govoriti; 3. l. pl. ind. prez. akt.
\item[αὐτοὺς] §~207; imenski dio A+I
\item[θρηνοῦντας] θρηνέω oplakivati; a. pl. m. ptc. prez. akt.; imenski dio A+I
\item[τὸν θάνατον] §~82; imenica je objekt participa
\item[ὑπὸ λύπης] §~437, §~90
\item[ἐξᾴδειν] ἐξᾴδω otpjevati, pjevati posljednju pjesmu; inf. prez. akt.; glagolski dio A+I
\item[λογίζονται] λογίζομαι računati, uzeti u obzir; 3. l. pl. ind. prez. medpas.
\item[οὐδὲν ὄρνεον] §~224, §~82
\item[ᾄδει] ᾄδω pjevati; 3. l. sg. ind. prez. akt.
\item[ὅταν ] ὅταν kad (god), s konjunktivom izriče iterativnu radnju za sadašnjost, §~488
\item[πεινῇ] πεινάω biti gladan; 3. l. sg. konj. prez. akt.
\item[ῥιγῷ] ῥιγόω osjećati hladnoću, zepsti; 3. l. sg. konj. prez. akt.
\item[ἤ τινα] §~40
\item[τινα ἄλλην λύπην] §~217, §~212.1, §~90
\item[λυπῆται] λυπέω tugovati, žalovati; med. biti žalostan; 3. l. sg. konj. prez. medpas.
\item[αὐτὴ ] §~207
\item[ἥ\dots\ ἀηδὼν] §~131
\item[ἥ τε] §~40
\item[χελιδὼν] §~131
\item[ὁ ἔποψ] §~115
\item[ἃ] §~215; antecedent su odnosne zamjenice \textgreek[variant=ancient]{ὄρνεα, sc.\ ἥ τε ἀηδὼν καὶ χελιδὼν καὶ ὁ ἔποψ}
\item[δή] naglašava prethodnu riječ: baš
\item[φασι] φημί govoriti; 3. l. pl. ind. prez. akt.
\item[δή φασι] §~40
\item[διὰ λύπην] §~428, §~90
\item[θρηνοῦντα] θρηνέω oplakivati; a. pl. n. ptc. prez. akt.; imenski dio A+I
\item[ᾄδειν] ᾄδω pjevati; inf. prez. akt.; glagolski dio A+I
\item[φασι\dots\ θρηνοῦντα ᾄδειν] glagol govorenja \textit{(verbum dicendi)} u rečenici otvara mjesto A+I
\end{description}

%3 itd

{\large
\noindent ἀλλ' \\
οὔτε \underline{ταῦτά} \\
\tabto{2em} μοι φαίνεται \\
\underline{λυπούμενα ᾄδειν} \\
οὔτε οἱ κύκνοι, \\
ἀλλ' ἅτε οἶμαι τοῦ ᾿Απόλλωνος ὄντες, \\
μαντικοί τέ εἰσι \\
καὶ προειδότες \\
\tabto{2em} τὰ ἐν ῞Αιδου ἀγαθὰ \\
ᾄδουσι \\
καὶ τέρπονται \\
ἐκείνην τὴν ἡμέραν \\
διαφερόντως \\
ἢ ἐν τῷ ἔμπροσθεν χρόνῳ.\\

}

\begin{description}[noitemsep]

\item[ἀλλ' οὔτε] §~68
\item[ταῦτά] §~213.2, sc.\ ὄρνεα
\item[μοι] §~205
\item[ταῦτά μοι ] §~40
\item[φαίνεται] φαίνω pojaviti se; med. činiti se; 3. l. sg. ind. prez. medpas.; kao obvezatnu dopunu glagol osjećanja \textit{(verbum sentiendi)} ovdje ima N+I% provjeri Musić Majnarić
\item[λυπούμενα] λυπέω tugovati, žalovati; med. biti žalostan; n. pl. s. r. ptc. prez. medpas.; dio N+I
\item[ᾄδειν] ᾄδω pjevati; inf. prez. akt.; dio N+I
\item[οἱ κύκνοι] §~82
\item[ἀλλ' ἅτε] §~68; ἅτε ističe uzročno značenje participa
\item[οἶμαι] οἴομαι smatrati (skraćeni oblik οἶμαι); 1. l. sg. ind. prez. (med.)
\item[τοῦ ᾿Απόλλωνος] §~131
\item[ὄντες] εἰμί biti; n. pl. m. r. ptc. prez. (akt.); εἰμί τινος: biti čiji, pripadati komu; posvojni genitiv §~393
\item[μαντικοί] §~103
\item[εἰσι] εἰμί biti; 3. l. pl. ind. prez. (akt.)
\item[μαντικοί τέ εἰσι] §~40
\item[προειδότες] πρόοιδα znati unaprijed, predviđati; n. pl. m. r. ptc. (perf.) akt.
\item[τὰ ἐν ῞Αιδου ἀγαθὰ] §~426; poimeničenje prijedložnog izraza članom §~373
\item[ᾄδουσι] ᾄδω pjevati; 3. l. pl. ind. prez. akt.
\item[τέρπονται] τέρπομαι uživati; 3. l. pl. ind. prez. (med.)
\item[ἐκείνην τὴν ἡμέραν] §~214.3, §~90
\item[διαφερόντως ἢ ] drukčije nego, za razliku od
\item[ἐν τῷ ἔμπροσθεν χρόνῳ] §~426, §~82; prilog vrši službu atributa i nalazi se u atributnom položaju §~375
\end{description}

% 4

{\large
\noindent ἐγὼ δὲ καὶ \underline{αὐτὸς} ἡγοῦμαι \\
\tabto{2em} \underline{ὁμόδουλός} τε \underline{εἶναι} τῶν κύκνων \\
\tabto{2em} καὶ \underline{ἱερὸς} τοῦ αὐτοῦ θεοῦ, \\
\tabto{2em} καὶ οὐ χεῖρον ἐκείνων τὴν μαντικὴν \underline{ἔχειν} \\
\tabto{4em} παρὰ τοῦ δεσπότου,  \\
\tabto{2em} οὐδὲ δυσθυμότερον αὐτῶν \\
\tabto{4em} τοῦ βίου \underline{ἀπαλλάττεσθαι}. \\

}

\begin{description}[noitemsep]

\item[ἐγὼ] §~205
\item[δὲ] čestica povezuje rečenicu s prethodnom: a\dots
\item[αὐτὸς] §~207
\item[ἡγοῦμαι] ἡγέομαι vjerovati, otvara mjesto N+I; 1. l. sg. ind. prez. (med.), stegnuti oblik §~47
\item[ὁμόδουλός ] §~103
\item[ὁμόδουλός τε] §~40
\item[εἶναι] εἰμί biti; inf. prez. (akt.)
\item[τῶν κύκνων] §~82
\item[ἱερὸς] §~103
\item[τοῦ αὐτοῦ θεοῦ] §~82, §~207
\item[χεῖρον] §~202
\item[χεῖρον ἐκείνων] §~214.3; genitiv usporedbe \textit{(comparationis)} §~404
\item[τὴν μαντικὴν ] §~103
\item[ἔχειν] ἔχω imati; inf. prez. akt.
\item[παρὰ τοῦ δεσπότου] §~434, §~100
\item[δυσθυμότερον αὐτῶν] §~197, §~207; genitiv usporedbe \textit{(comparationis)} §~404
\item[τοῦ βίου] §~82
\item[ἀπαλλάττεσθαι] ἀπαλλάσσω osloboditi (atički dijalekt: ἀπαλλάττω); med., rekcija τινος napustiti što; inf. prez. medpas.


\end{description}


%kraj
