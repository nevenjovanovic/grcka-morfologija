%Unesi korekture NČ, 2019-08-12
%\section*{O autoru}


\section*{O tekstu}

Djelo \textgreek[variant=ancient]{Περὶ τοῦ Ἐνυπνίου ἤτοι Βίος Λουκιανοῦ} \textit{(Somnium, San ili Lukijanov život)} predstavlja Lukijanovu autobiografiju i glavni je izvor za poznavanje piščeva života. Lukijan opisuje pokušaj svojih roditelja da ga obrazuju za kipara i klesara. Prvi dan u kiparskoj radionici neslavno je završio: mladi Lukijan oštetio je vrijedan komad mramora. Te je noći usnuo san u kojem mu pristupaju personifikacije kiparstva \textgreek[variant=ancient]{(Ἑρμογλυφικὴ τέχνη)} i obrazovanja \textgreek[variant=ancient]{(Παιδεία)} koje se otimaju za njega. Slijedi govor Obrazovanja u liku lijepe i elegantne žene, za razliku od muškobanjaste i prljave personifikacije kiparstva.

\section*{Pročitajte naglas grčki tekst.}

Luc.\ Somnium sive vita Luciani 9

%Naslov prema izdanju

\medskip


{\large
{ 
\begin{greek}

\noindent ``Ἐγὼ δέ, ὦ τέκνον, Παιδεία εἰμὶ ἤδη συνήθης σοι καὶ γνωρίμη, εἰ καὶ μηδέπω εἰς τέλος μου πεπείρασαι. ἡλίκα μὲν οὖν τὰ ἀγαθὰ ποριῇ λιθοξόος γενόμενος, αὕτη προείρηκεν· οὐδὲν γὰρ ὅτι μὴ ἐργάτης ἔσῃ τῷ σώματι πονῶν κἀν τούτῳ τὴν ἅπασαν ἐλπίδα τοῦ βίου τεθειμένος, ἀφανὴς μὲν αὐτὸς ὤν, ὀλίγα καὶ ἀγεννῆ λαμβάνων, ταπεινὸς τὴν γνώμην, εὐτελὴς δὲ τὴν πρόοδον, οὔτε φίλοις ἐπιδικάσιμος οὔτε ἐχθροῖς φοβερὸς οὔτε τοῖς πολίταις ζηλωτός, ἀλλ' αὐτὸ μόνον ἐργάτης καὶ τῶν ἐκ τοῦ πολλοῦ δήμου εἷς, ἀεὶ τὸν προὔχοντα ὑποπτήσσων καὶ τὸν λέγειν δυνάμενον θεραπεύων, λαγὼ βίον ζῶν καὶ τοῦ κρείττονος ἕρμαιον ὤν· εἰ δὲ καὶ Φειδίας ἢ Πολύκλειτος γένοιο καὶ πολλὰ θαυμαστὰ ἐξεργάσαιο, τὴν μὲν τέχνην ἅπαντες ἐπαινέσονται, οὐκ ἔστι δὲ ὅστις τῶν ἰδόντων, εἰ νοῦν ἔχοι, εὔξαιτ' ἂν σοὶ ὅμοιος γενέσθαι· οἷος γὰρ ἂν ᾖς, βάναυσος καὶ χειρῶναξ καὶ ἀποχειροβίωτος νομισθήσῃ.

Ἢν δ' ἐμοὶ πείθῃ, πρῶτον μέν σοι πολλὰ ἐπιδείξω παλαιῶν ἀνδρῶν ἔργα καὶ πράξεις θαυμαστὰς καὶ λόγους αὐτῶν ἀπαγγελῶ, καὶ πάντων ὡς εἰπεῖν ἔμπειρον ἀποφανῶ, καὶ τὴν ψυχήν, ὅπερ σοι κυριώτατόν ἐστι, κατακοσμήσω πολλοῖς καὶ ἀγαθοῖς κοσμήμασι — σωφροσύνῃ, δικαιοσύνῃ, εὐσεβείᾳ, πρᾳότητι, ἐπιεικείᾳ, συνέσει, καρτερίᾳ, τῷ τῶν καλῶν ἔρωτι, τῇ πρὸς τὰ σεμνότατα ὁρμῇ· ταῦτα γάρ ἐστιν ὁ τῆς ψυχῆς ἀκήρατος ὡς ἀληθῶς κόσμος. λήσει δέ σε οὔτε παλαιὸν οὐδὲν οὔτε νῦν γενέσθαι δέον, ἀλλὰ καὶ τὰ μέλλοντα προόψει μετ' ἐμοῦ, καὶ ὅλως ἅπαντα ὁπόσα ἐστί, τά τε θεῖα τά τ' ἀνθρώπινα, οὐκ εἰς μακράν σε διδάξομαι.''


\end{greek}


}
}


\section*{Analiza i komentar}

%1

{\large
\begin{greek}
\noindent  Ἐγὼ δέ, ὦ τέκνον, Παιδεία εἰμὶ \\
\tabto{2em} ἤδη συνήθης σοι καὶ γνωρίμη, \\
εἰ καὶ μηδέπω \\
\tabto{2em} εἰς τέλος \\
μου πεπείρασαι.\\

\end{greek}
}

\begin{description}[noitemsep]
\item[Ἐγὼ] §~205
\item[δέ] usporedni suprotni veznik §~515.2, povezuje rečenicu s prethodnom
\item[ὦ τέκνον] §~82
\item[Παιδεία] §~90; imenica kao dio imenskog predikata, Smyth 910
\item[εἰμὶ] εἰμί biti, 1. l . sg. ind. prez. akt. §~315, bilj. 2; glagol kao kopula, dio imenskoga predikata, otvara mjesto imenskoj dopuni Smyth 909
\item[συνήθης] §~153; pridjev kao dio imenskog predikata, Smyth 910
\item[σοι] §~205
\item[γνωρίμη] §~103; pridjev kao dio imenskog predikata, Smyth 910
\item[εἰ καὶ] kombinacija veznika uvodi dopusnu rečenicu, §~480
\item[εἰς τέλος] do kraja, sasvim; §~153; εἰς + a. §~419
\item[μου] §~205
\item[πεπείρασαι] πειράω τινός kušati nešto, iskušavati nešto; 2. l. sg. ind. perf. medpas.
\end{description}

%2

{\large
\begin{greek}
\noindent ἡλίκα μὲν οὖν τὰ ἀγαθὰ \\
ποριῇ \\
\tabto{2em} λιθοξόος γενόμενος, \\
αὕτη προείρηκεν· \\
\tabto{2em} οὐδὲν γὰρ \\
\tabto{4em} ὅτι μὴ ἐργάτης ἔσῃ \\
\tabto{6em} τῷ σώματι πονῶν \\
\tabto{4em} κἀν τούτῳ \\
\tabto{4em} τὴν ἅπασαν ἐλπίδα \\
\tabto{6em} τοῦ βίου \\
\tabto{4em} τεθειμένος, \\
\tabto{4em} ἀφανὴς μὲν αὐτὸς ὤν,\\
\tabto{6em} ὀλίγα καὶ ἀγεννῆ λαμβάνων, \\
\tabto{6em} ταπεινὸς τὴν γνώμην, \\
\tabto{6em} εὐτελὴς δὲ τὴν πρόοδον, \\
\tabto{6em} οὔτε φίλοις ἐπιδικάσιμος\\
\tabto{6em} οὔτε ἐχθροῖς φοβερὸς \\
\tabto{6em} οὔτε τοῖς πολίταις ζηλωτός, \\
\tabto{6em} ἀλλ' αὐτὸ μόνον ἐργάτης \\
\tabto{8em} καὶ τῶν ἐκ τοῦ πολλοῦ δήμου εἷς, \\
\tabto{8em} ἀεὶ τὸν προὔχοντα ὑποπτήσσων \\
\tabto{8em} καὶ τὸν λέγειν δυνάμενον θεραπεύων, \\
\tabto{8em} λαγὼ βίον ζῶν \\
\tabto{8em} καὶ τοῦ κρείττονος ἕρμαιον ὤν· \\
εἰ δὲ καὶ Φειδίας ἢ Πολύκλειτος γένοιο \\
\tabto{2em} καὶ πολλὰ θαυμαστὰ ἐξεργάσαιο, \\
τὴν μὲν τέχνην ἅπαντες ἐπαινέσονται, \\
οὐκ ἔστι δὲ ὅστις \\
\tabto{2em} τῶν ἰδόντων, \\
εἰ νοῦν ἔχοι, \\
εὔξαιτ' ἂν \\
\tabto{2em} σοὶ ὅμοιος γενέσθαι· \\
οἷος γὰρ ἂν ᾖς, \\
\tabto{2em} βάναυσος καὶ χειρῶναξ καὶ ἀποχειροβίωτος \\
νομισθήσῃ.\\

\end{greek}
}

\begin{description}[noitemsep]
\item[ἡλίκα] §~219
\item[οὖν] zaključna čestica §~519.7: dakako
\item[ἡλίκα μὲν\dots\  Ἢν δ' ἐμοὶ\dots] koordinacija rečenica pomoću čestica μέν\dots\ δέ\dots
\item[τὰ ἀγαθὰ] §~103; supstantiviranje članom §~373
\item[ποριῇ] πορίζω pripraviti, pribaviti, 2. l. ind. fut. med., atički futur §~263
\item[λιθοξόος] §~82
\item[γενόμενος] γίγνομαι postati, nastati, n. sg. m. r. ptc. aor. med. 
\item[αὕτη] §~213.2
\item[προείρηκεν] προλέγω ranije reći, 3. l. sg. ind. perf. akt.
\item[οὐδὲν γὰρ] sc.\ προείρηκεν
\item[οὐδὲν] §~224.2
\item[γὰρ] §~517
\item[ὅτι μὴ] osim
\item[ἐργάτης] §~100; imenica kao dio imenskog predikata, Smyth 910
\item[ἔσῃ] εἰμί biti, 2. l. sg. fut. (med.); glagol kao kopula, dio imenskoga predikata, otvara mjesto imenskoj dopuni Smyth 909
\item[τῷ σώματι] §~123
\item[πονῶν] πονέω mučiti se, n. sg. m. r. ptc. prez. akt.
\item[κἀν] = καί + ἐν; kraza §~66 
\item[τούτῳ] §~213.2
\item[τὴν ἐλπίδα] §~123 
\item[ἅπασαν] §~193
\item[τοῦ βίου] §~82
\item[τεθειμένος] τίθημι staviti, polagati; n. sg. m. r. ptc. perf. medpas.
\item[ἀφανὴς] §~153
\item[ἀφανὴς μὲν\dots\ εὐτελὴς δὲ\dots] koordinacija rečeničnih članova pomoću čestica μέν\dots\ δέ\dots
\item[αὐτὸς] §~207
\item[ὤν] εἰμί biti; n. sg. m. r. ptc. prez. akt. §~315.2
\item[ὀλίγα] §~103
\item[ἀγεννῆ] §~153
\item[λαμβάνων] λαμβάνω uzeti; n. sg. m. r. ptc. prez. akt. §~231
\item[ταπεινὸς] §~103
\item[τὴν γνώμην] §~90, akuzativ obzira §~389
\item[εὐτελὴς] §~153
\item[τὴν πρόοδον] §~82, §~83, akuzativ obzira §~389
\item[φίλοις] §~103
\item[ἐπιδικάσιμος] §~103, §~106
\item[ἐχθροῖς] §~103
\item[φοβερὸς] §~103
\item[τοῖς πολίταις] §~100
\item[ζηλωτός] §~103
\item[τῶν ἐκ τοῦ πολλοῦ δήμου] ovisno o εἷς, genitiv partitivni §~395; član τῶν supstantivira prijedložni izraz ἐκ τοῦ πολλοῦ δήμου §~82, §~196; ἐκ + g. §~424
\item[εἷς] §~224
\item[προὔχοντα] προέχω nadvisivati; a. sg. m. r. ptc. prez. akt., koronida §~16, kraza §~66
\item[ὑποπτήσσων] ὑποπτήσσω τινά bojati se nekog, puzati pred nekim; n. sg. m. r. ptc. prez. akt. 
\item[τὸν δυνάμενον] δύναμαι moći; a. sg. m. r. ptc. prez. medpas., supstantivirani particip §~499.2
\item[λέγειν] λέγω govoriti; inf. prez. akt., dopuna participu δυνάμενον
\item[θεραπεύων] θεραπεύω τινά služiti nekog, ulagivati se nekom; n. sg. m. r. ptc. prez. akt. 
\item[λαγὼ] §~111
\item[βίον] §~82
\item[ζῶν] ζῶ (ζάω) živjeti; n. sg. m. r. ptc. prez. akt. §~244
\item[τοῦ κρείττονος] §~202, supstantiviranje članom §~373
\item[ἕρμαιον] §~82
\item[ὤν] εἰμί biti, n. sg. m. r. ptc. prez. akt. §~315, bilj. 2
\item[εἰ] veznik uvodi potencijalnu pogodbenu protazu §~477, apodoza je eventualna §~479
\item[Φειδίας] §~100
\item[ἢ] §~514
\item[Πολύκλειτος] §~82
\item[γένοιο] γίγνομαι postati, nastati; 2. l. sg. opt. aor. med. 
\item[πολλὰ] §~196
\item[θαυμαστὰ] §~103
\item[ἐξεργάσαιο] ἐξεργάζομαι izraditi; 2. l. sg. opt. aor. med.
\item[τὴν μὲν τέχνην\dots\ οὐκ ἔστι δὲ\dots] koordinacija rečeničnih članova pomoću čestica μέν\dots\ δέ\dots
\item[τὴν τέχνην] §~90
\item[ἅπαντες] §~193
\item[ἐπαινέσονται] ἐπαινέω hvaliti; 3. l. pl. ind. fut. med. s aktivnim značenjem 
\item[ἔστι] εἰμί postojati (ovdje u egzistencijalnom značenju), οὐκ ἔστι\dots\ ὅστις nema nikoga tko\dots; 3. l. sg. ind. prez. akt. §~315, bilj. 2
\item[ὅστις] §~217
\item[τῶν ἰδόντων] ὁράω gledati, g. pl. m. r. ptc. aor. akt.; genitiv partitivni §~395
\item[εἰ] veznik uvodi potencijalnu pogodbenu rečenicu §~477
\item[νοῦν] §~107, §~108.1a
\item[ἔχοι] ἔχω imati; 3. l. sg. opt. prez. akt. 
\item[εὔξαιτ'] εὔχομαι hvaliti se, dičiti se; 3. l. sg. opt. aor. med., elizija §~68
\item[γενέσθαι] γίγνομαι postati, nastati; inf. aor. med.
\item[οἷος] §~219
\item[γὰρ] postpozitivni usporedni uzročni veznik §~517
\item[ἂν] §~489, bilj. 4; §~486
\item[ᾖς] εἰμί biti, 2. l. sg. konj. prez. akt. §~315, bilj. 2
\item[βάναυσος] §~103, §~106
\item[χειρῶναξ] §~123
\item[ἀποχειροβίωτος] §~103, §~106
\item[νομισθήσῃ] νομίζω smatrati, 2. l. sg. ind. fut. pas., otvara mjesto imenskoj dopuni

\end{description}

%3


{\large
\begin{greek}
\noindent  Ἢν δ' ἐμοὶ πείθῃ, \\
\tabto{2em} πρῶτον μέν \\
\tabto{2em} σοι \\
\tabto{2em} πολλὰ \\
\tabto{4em} ἐπιδείξω \\
\tabto{6em} παλαιῶν ἀνδρῶν \\
\tabto{2em} ἔργα \\
\tabto{4em} καὶ πράξεις θαυμαστὰς καὶ λόγους αὐτῶν ἀπαγγελῶ, \\
\tabto{4em} καὶ πάντων \\
\tabto{6em} ὡς εἰπεῖν \\
\tabto{4em} ἔμπειρον ἀποφανῶ, \\
\tabto{4em} καὶ τὴν ψυχήν, \\
\tabto{6em} ὅπερ σοι κυριώτατόν ἐστι, \\
\tabto{4em} κατακοσμήσω \\
\tabto{6em} πολλοῖς καὶ ἀγαθοῖς κοσμήμασι\\
\tabto{6em} — σωφροσύνῃ, δικαιοσύνῃ, εὐσεβείᾳ, πρᾳότητι, ἐπιεικείᾳ, συνέσει, καρτερίᾳ, \\
\tabto{6em} τῷ \\
\tabto{8em} τῶν καλῶν \\
\tabto{6em} ἔρωτι, \\
\tabto{6em} τῇ \\
\tabto{8em} πρὸς τὰ σεμνότατα \\
\tabto{6em} ὁρμῇ· \\
\tabto{8em} ταῦτα γάρ ἐστιν \\
\tabto{10em} ὁ τῆς ψυχῆς ἀκήρατος \\
\tabto{12em} ὡς ἀληθῶς \\
\tabto{10em} κόσμος.\\

\end{greek}
}

\begin{description}[noitemsep]
\item[Ἢν] veznik uvodi eventualnu pogodbenu rečenicu, §~474, §~476
\item[δ'] usporedni suprotni veznik §~515.2; elizija §~68 
\item[ἐμοὶ] §~205
\item[πείθῃ] πείθω nagovarati, \textit{medijalno} biti nagovoren, vjerovati; 2. l. sg. konj. prez. medpas.
\item[πρῶτον μέν\dots\ λήσει δέ σε] koordinacija rečenica pomoću čestica μέν\dots\ δέ\dots
\item[πρῶτον] §~223, §~204
\item[σοι] §~205
\item[πολλὰ] §~196
\item[ἐπιδείξω] ἐπιδείκνυμι pokazati; 1. l. sg. ind. fut. akt. 
\item[παλαιῶν] §~103
\item[ἀνδρῶν] §~146, §~149
\item[ἔργα] §~82
\item[πράξεις] §~165
\item[θαυμαστὰς] §~103
\item[λόγους] §~82
\item[αὐτῶν] §~207
\item[ἀπαγγελῶ] ἀπαγγέλλω javljati, pripovijedati; 1. l. sg. ind.  fut. akt.  
\item[πάντων] §~193
\item[ὡς εἰπεῖν] tako reći \textit{(fraza);} λέγω reći, inf. aor. akt. §~327, §~496
\item[ἔμπειρον] §~103, §~106
\item[ἀποφανῶ] ἀποφαίνω pokazati, otkriti; 1. l. sg. ind. fut. akt.
\item[ὅπερ] §~216, odnosna zamjenica uvodi odnosnu rečenicu  §~481 
\item[σοι] §~205
\item[κυριώτατόν] §~197, za naglasak §~39, §~40
\item[ἐστι] εἰμί biti, 3. l. sg. ind. prez. akt. §~315, bilj. 2
\item[κατακοσμήσω] κατακοσμέω urediti; 1. l. sg. ind. fut. akt. 
\item[πολλοῖς] §~196
\item[ἀγαθοῖς] §~103
\item[κοσμήμασι] §~123
\item[σωφροσύνῃ] §~90
\item[δικαιοσύνῃ] §~90
\item[εὐσεβείᾳ] §~97
\item[πρᾳότητι] §~123
\item[ἐπιεικείᾳ] §~97
\item[συνέσει] §~165
\item[καρτερίᾳ] §~90
\item[τῷ ἔρωτι] §~123
\item[τῶν καλῶν] §~103, genitiv objekta §~394
\item[πρὸς τὰ σεμνότατα] §~197; πρὸς + a.\ §~435C; supstantiviranje članom §~373
\item[τῇ ὁρμῇ] §~90
\item[ταῦτα] §~213.2
\item[γάρ] usporedni uzročni veznik §~517
\item[ἐστιν] εἰμί biti, 3. l. sg. ind. prez. akt. §~315, bilj. 2
\item[ὁ κόσμος] §~82
\item[τῆς ψυχῆς] §~90
\item[ἀκήρατος] §~103, §~106
\item[ὡς] §~221
\item[ἀληθῶς] §~204

\end{description}

%4


{\large
\begin{greek}
\noindent λήσει δέ σε \\
\tabto{2em} οὔτε παλαιὸν οὐδὲν \\
\tabto{2em} οὔτε νῦν γενέσθαι δέον, \\
ἀλλὰ καὶ τὰ μέλλοντα προόψει μετ' ἐμοῦ, \\
\tabto{2em} καὶ ὅλως ἅπαντα \\
\tabto{4em} ὁπόσα ἐστί, \\
\tabto{6em} τά τε θεῖα τά τ' ἀνθρώπινα, \\
\tabto{2em} οὐκ εἰς μακράν \\
\tabto{4em} σε διδάξομαι.\\

\end{greek}
}

\begin{description}[noitemsep]
\item[λήσει] λανθάνω τινά biti skriven od nekoga; 3. l. sg. ind. fut. akt., otvara mjesto σε: neće ti promaknuti (fraza)
\item[σε] §~205
\item[οὔτε\dots\ οὔτε\dots] usporedni sastavni veznici §~513.4
\item[παλαιὸν] §~103
\item[οὐδὲν] §~224.2, §~510
\item[γενέσθαι] γίγνομαι postati, nastati; inf. aor. med.
\item[δέον] δεῖ treba (bezlično), otvara mjesto dopuni u infinitivu; n. sg. s. r. ptc. prez. akt. 
\item[ἀλλὰ] usporedni suprotni veznik §~515.1
\item[τὰ μέλλοντα ] μέλλω namjeravati, biti određen od sudbine;  n. pl. s. r. ptc. prez. akt., supstantivirani particip §~499.2
\item[προόψει] προοράω pred sobom vidjeti, predvidjeti; 2. l. sg. ind. fut. med. 
\item[μετ' ἐμοῦ] §~205, μετά + g. §~430, elizija §~68
\item[ὅλως] §~204
\item[ἅπαντα] §~193
\item[ὁπόσα] §~219
\item[ἐστί] εἰμί biti; 3. l. sg. ind. prez. akt. §~315, bilj. 2
\item[τά θεῖα] §~103, supstantiviranje članom §~373
\item[τά  ἀνθρώπινα] §~103, supstantiviranje članom §~373
\item[τε\dots\ τ'\dots] usporedni sastavni veznici §~513.2, elizija §~68
\item[εἰς μακράν] zadugo (priložni izraz); §~103; εἰς + a. §~419
\item[σε] §~205
\item[διδάξομαι] διδάσκω τινά τι podučavati nekoga nešto; 1. l. sg. ind. fut. med., §~32.9; §~386

\end{description}




%kraj
