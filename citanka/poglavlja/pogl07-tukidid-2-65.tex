% Unio ispravke NČ 2019-09-05
%\section*{O autoru}


\section*{O tekstu}

Na početku Peloponeskog rata, tijekom prve provale Spartanaca u Atiku, pod vodstvom Arhidama, u ljeto 431.\ pr.~Kr., Periklo se odlučio za nekonvencionalnu taktiku: umjesto da se sukobe sa Spartancima na otvorenom (tradicionalno ratovanje poljoprivrednih državica redovito se svodilo na provociranje uništavanjem usjeva te kratku i ubojitu odlučnu bitku nakon toga), Atenjani su se povukli unutar gradskih zidina i ostali ondje oslanjajući se na svoju financijsku, političku i morsku prevlast. Znali su da sve što su Spartanci uništili mogu lako nadoknaditi uz potporu saveznika i dopremajući namirnice brodovima, a da će atenska mornarica, zauzvrat, opustošiti Peloponez. Pa ipak, nije bilo lako gledati uništavanje vrijedne imovine. Zato je u zimu, kad su Atenjani svečano sahranili prve žrtve rata, Periklo održao slavni nadgrobni govor (Thuc. Hist. 2.35–46). Potom je, međutim, u Ateni izbila kuga, i nakon druge ljetne provale (430.) Atenjani su počeli očajavati, kritizirati Periklovu strategiju, slati poslanike Spartancima. Novim ih je govorom Periklo uspio barem načelno umiriti – no gorčina zbog privatnih gubitaka nije odmah nestala.

%\newpage

\section*{Pročitajte naglas grčki tekst.}

Thuc.\ Historiae 2.65.2

%Naslov prema izdanju

\medskip

\begin{greek}
{\large
{ \noindent Oἱ δὲ δημοσίᾳ μὲν τοῖς λόγοις ἀνεπείθοντο καὶ οὔτε πρὸς τοὺς Λακεδαιμονίους ἔτι ἔπεμπον ἔς τε τὸν πόλεμον μᾶλλον ὥρμηντο, ἰδίᾳ δὲ τοῖς παθήμασιν ἐλυποῦντο, ὁ μὲν δῆμος ὅτι ἀπ' ἐλασσόνων ὁρμώμενος ἐστέρητο καὶ τούτων, οἱ δὲ δυνατοὶ καλὰ κτήματα κατὰ τὴν χώραν οἰκοδομίαις τε καὶ πολυτελέσι κατασκευαῖς ἀπολωλεκότες, τὸ δὲ μέγιστον, πόλεμον ἀντ' εἰρήνης ἔχοντες. οὐ μέντοι πρότερόν γε οἱ ξύμπαντες ἐπαύσαντο ἐν ὀργῇ ἔχοντες αὐτὸν πρὶν ἐζημίωσαν χρήμασιν. ὕστερον δ' αὖθις οὐ πολλῷ, ὅπερ φιλεῖ ὅμιλος ποιεῖν, στρατηγὸν εἵλοντο καὶ πάντα τὰ πράγματα ἐπέτρεψαν, ὧν μὲν περὶ τὰ οἰκεῖα ἕκαστος ἤλγει ἀμβλύτεροι ἤδη ὄντες, ὧν δὲ ἡ ξύμπασα πόλις προσεδεῖτο πλείστου ἄξιον νομίζοντες εἶναι. ὅσον τε γὰρ χρόνον προύστη τῆς πόλεως ἐν τῇ εἰρήνῃ, μετρίως ἐξηγεῖτο καὶ ἀσφαλῶς διεφύλαξεν αὐτήν, καὶ ἐγένετο ἐπ' ἐκείνου μεγίστη, ἐπειδή τε ὁ πόλεμος κατέστη, ὁ δὲ φαίνεται καὶ ἐν τούτῳ προγνοὺς τὴν δύναμιν.

}
}
\end{greek}

\section*{Analiza i komentar}

%1

{\large
\begin{greek}
\noindent Oἱ δὲ δημοσίᾳ μὲν \\
\tabto{2em} τοῖς λόγοις ἀνεπείθοντο \\
καὶ οὔτε πρὸς τοὺς Λακεδαιμονίους \\
\tabto{2em} ἔτι ἔπεμπον \\
ἔς τε τὸν πόλεμον \\
\tabto{2em} μᾶλλον ὥρμηντο, \\
ἰδίᾳ δὲ \\
\tabto{2em} τοῖς παθήμασιν ἐλυποῦντο, \\
\tabto{4em} ὁ μὲν δῆμος \\
\tabto{6em} ὅτι ἀπ' ἐλασσόνων ὁρμώμενος \\
\tabto{6em} ἐστέρητο καὶ τούτων, \\
\tabto{4em} οἱ δὲ δυνατοὶ \\
\tabto{6em} καλὰ κτήματα \\
\tabto{6em} κατὰ τὴν χώραν \\
\tabto{6em} οἰκοδομίαις τε καὶ πολυτελέσι κατασκευαῖς \\
\tabto{6em} ἀπολωλεκότες, \\
\tabto{2em} τὸ δὲ μέγιστον, \\
\tabto{4em} πόλεμον ἀντ' εἰρήνης \\
\tabto{6em} ἔχοντες.\\

\end{greek}
}

\begin{description}[noitemsep]
\item[Oἱ δὲ ] §~370.2
\item[δημοσίᾳ μὲν\dots\  ἰδίᾳ δὲ\dots] koordinacija s pomoću čestica μὲν\dots\ δὲ\dots
\item[δημοσίᾳ ] §~103; prilog izveden od dativa pridjeva
\item[τοῖς λόγοις ] §~82; misli se na Periklov govor u skupštini
\item[ἀνεπείθοντο] ἀναπείθω uvjeriti; 3. l. pl. impf. medpas.
\item[πρὸς τοὺς Λακεδαιμονίους] §~435; §~103; §~373
\item[ἔπεμπον ] πέμπω slati; 3. l. pl. impf. akt.
\item[ἔς τε τὸν πόλεμον ] §~40; §~419; §~82
\item[μᾶλλον ] §~204.3
\item[ὥρμηντο] ὁρμάω εἰς πόλεμον poticati na rat; 3. l. pl. impf. medpas.
\item[ἰδίᾳ] §~103; prilog izveden od pridjeva
\item[τοῖς παθήμασιν ] §~123
\item[ἐλυποῦντο] λυπέω med. žalostiti se; 3. l. pl. impf. medpas.
\item[ὁ μὲν δῆμος\dots\  οἱ δὲ δυνατοὶ\dots\   τὸ δὲ μέγιστον\dots] koordinacija s pomoću čestica μὲν\dots\   δὲ\dots\   δὲ\dots
\item[δῆμος] §~82
\item[ὅτι ] uvodi izričnu rečenicu, dopunu uz ἐλυποῦντο
\item[ἀπ' ἐλασσόνων ] §~68; §~423; §~202
\item[ὁρμώμενος ] \textbf{ἀπ’ ἐλασσόνων} započeti s manje sredstava, LSJ ὁρμάω B.2.b; n. sg. m. r. ptc. prez. medpas.
\item[ἐστέρητο] στερέω biti lišen, biti opljačkan; 3. l. pl. impf. medpas.
\item[τούτων] §~213.2
\item[οἱ\dots\  δυνατοὶ ] §~103; §~373
\item[καλὰ κτήματα ] §~103; §~123
\item[κατὰ τὴν χώραν ] §~429; §~90
\item[οἰκοδομίαις ] §~90
\item[οἰκοδομίαις τε καὶ ] §~40; kombinacija sastavnih veznika τε καὶ naglašava korespondenciju (i\dots\  i\dots)
\item[πολυτελέσι ] §~153
\item[κατασκευαῖς ] §~90
\item[ἀπολωλεκότες] ἀπόλλυμι izgubiti; n. pl. m. r. ptc. perf. akt.
\item[τὸ\dots\  μέγιστον] §~200; §~373
\item[πόλεμον ] §~82
\item[ἀντ' εἰρήνης ] §~68; §~422; §~90
\item[ἔχοντες] ἔχω imati; ἐν ὀργῇ ἔχειν ljutiti se na koga, biti ljut na koga (LSJ ἔχω II); n. pl. m. r. ptc. prez. akt.


\end{description}

%2

{\large
\begin{greek}
\noindent οὐ μέντοι πρότερόν γε \\
οἱ ξύμπαντες ἐπαύσαντο \\
\tabto{2em} ἐν ὀργῇ ἔχοντες αὐτὸν \\
πρὶν ἐζημίωσαν \\
\tabto{2em} χρήμασιν.\\

\end{greek}
}

\begin{description}[noitemsep]
\item[οὐ μέντοι πρότερόν γε] §~40; kombinacija čestica μέντοι\dots\  γε naglašava suprotnu (adverzativnu) tvrdnju: međutim, nisu\dots
\item[οἱ ξύμπαντες ] atički oblik pridjeva σύμπας; §~193; §~373
\item[ἐπαύσαντο\dots\  ἔχοντες] παύω med. + ptc. prez. prestati; 3. l. pl. impf. medpas.; ἔχω imati, \textit{ovdje} osjećati što prema kome; n. pl. m. r. ptc. prez. akt.
\item[ἐν ὀργῇ] §~426; §~90
\item[αὐτὸν ] sc.\ Perikla; §~207
\item[ἐζημίωσαν ] ζημιόω globiti; 3. l. pl. ind. aor. akt.
\item[χρήμασιν] §~123


\end{description}

%3

{\large
\begin{greek}
\noindent ὕστερον δ' αὖθις οὐ πολλῷ, \\
ὅπερ φιλεῖ ὅμιλος \\
\tabto{2em} ποιεῖν, \\
στρατηγὸν εἵλοντο \\
καὶ πάντα τὰ πράγματα ἐπέτρεψαν, \\
\tabto{2em} ὧν μὲν \\
\tabto{4em} περὶ τὰ οἰκεῖα \\
\tabto{2em} ἕκαστος ἤλγει \\
\tabto{4em} ἀμβλύτεροι ἤδη ὄντες, \\
\tabto{2em} ὧν δὲ \\
\tabto{2em} ἡ ξύμπασα πόλις \\
\tabto{2em} προσεδεῖτο \\
\tabto{4em} \underline{πλείστου ἄξιον} \\
\tabto{6em} νομίζοντες \underline{εἶναι}.\\

\end{greek}
}

\begin{description}[noitemsep]
\item[δ' αὖθις ] §~68; δὲ izriče suprotnost u odnosu na prethodnu rečenicu
\item[πολλῷ] §~414.4
\item[ὅπερ] §~216.3
\item[φιλεῖ ] bezlično: „kako to obično bude'', „običava se''; 3. l. sg. ind. prez. akt.
\item[ὅμιλος ] §~82
\item[ποιεῖν] ποιέω činiti; inf. prez. akt.
\item[στρατηγὸν ] §~82
\item[εἵλοντο] αἱρέω τινά τι med. birati koga za što; 3. l. pl. ind. aor. med.
\item[πάντα τὰ πράγματα ] §~193; §~123; §~379.1
\item[ἐπέτρεψαν] ἐπιτρέπω povjeriti; 3. l. pl. ind. aor. akt.
\item[τὰ πράγματα\dots\  ὧν μὲν\dots\  ὧν δὲ] §~215; koordinacija s pomoću čestica  μὲν\dots\  δὲ; dijelni genitiv §~395
\item[περὶ τὰ οἰκεῖα ] §~433; §~103; §~373
\item[ἕκαστος ] §~103
\item[ἤλγει ] ἀλγέω patiti, osjećati bol; 3. l. sg. impf. akt.
\item[ἀμβλύτεροι] §~198
\item[ὄντες] εἰμί biti; n. pl. m. r. ptc. prez. akt.
\item[ἡ ξύμπασα πόλις] atički oblik pridjeva σύμπας; §~193; §~165; §~379.3
\item[προσεδεῖτο ] προσδέω τινός med. trebati što; 3. l. sg. impf. medpas.
\item[πλείστου ] §~202
\item[ἄξιον ] §~103
\item[νομίζοντες ] νομίζω smatrati; n. pl. m. r. ptc. prez. akt.; otvara mjesto A+I
\item[εἶναι] εἰμί biti; inf. prez. akt.


\end{description}

%4

{\large
\begin{greek}
\noindent ὅσον τε γὰρ χρόνον \\
προύστη \\
\tabto{2em} τῆς πόλεως \\
\tabto{2em} ἐν τῇ εἰρήνῃ, \\
μετρίως ἐξηγεῖτο \\
καὶ ἀσφαλῶς διεφύλαξεν \\
\tabto{2em} αὐτήν, \\
καὶ ἐγένετο \\
\tabto{2em} ἐπ' ἐκείνου \\
μεγίστη, \\
ἐπειδή τε ὁ πόλεμος κατέστη, \\
ὁ δὲ φαίνεται \\
\tabto{2em} καὶ ἐν τούτῳ \\
\tabto{2em} προγνοὺς \\
\tabto{4em} τὴν δύναμιν.\\

\end{greek}
}

\begin{description}[noitemsep]
\item[ὅσον τε] §~40
\item[ὅσον ] §~219
\item[τε γὰρ\dots\  ἐπειδή τε\dots] koordinacija s pomoću sastavnih veznika (čestica) τε
\item[χρόνον ] §~82
\item[προύστη ] προΐστημι τινός biti na čelu čega; 3. l. sg. ind. aor. akt.
\item[τῆς πόλεως ] §~165
\item[ἐν τῇ εἰρήνῃ] §~426; §~90
\item[μετρίως ] §~204.1
\item[ἐξηγεῖτο] ἐξηγέομαι biti vođa, upravljati; 3. l. sg. impf. medpas.
\item[ἀσφαλῶς ] §~204.1
\item[διεφύλαξεν ] διαφυλάσσω (atički διαφυλάττω) čuvati; 3. l. sg. ind. aor. akt.
\item[αὐτήν] §~207
\item[ἐγένετο ] γίγνομαι postati; 3. l. sg. ind. aor. med.
\item[ἐπ' ἐκείνου ] §~68; §~436; §~212
\item[μεγίστη] §~200
\item[ἐπειδή τε ] §~40
\item[ὁ πόλεμος ] §~82
\item[κατέστη] καθίστημι nastati; 3. l. sg. ind. aor. akt.
\item[ὁ δὲ ] §~370.2
\item[φαίνεται] φαίνω med. pokazati se; 3. l. sg. ind. prez. medpas.
\item[ἐν τούτῳ ] §~426; §~213.2
\item[προγνοὺς ] προγιγνώσκω predvidjeti; n. sg. m. r. ptc. aor. akt.
\item[τὴν δύναμιν] §~165


\end{description}


%kraj
