% Unesi korekture NČ i NZ, 2019-09-24
%\section*{O autoru}



\section*{O tekstu}

Sokrat je primijetio da Epigen, jedan od njegovih drugova, nije u dobroj formi. Epigen tvrdi da mu vježbanje ne treba jer nije sportaš. Na to Sokrat iznosi svoje mišljenje o prednostima dobre tjelesne kondicije u ratu, u svim ostalim ljudskim aktivnostima, pa i u duhovnom radu.

%\newpage

\section*{Pročitajte naglas grčki tekst.}
Xen. Memorabilia 3.12.4
%Naslov prema izdanju

\medskip

{\large
\begin{greek}

\noindent Καὶ πολλοὶ μὲν διὰ τοῦτο ἐκ τῶν πολεμικῶν ἀγώνων σῴζονταί τε εὐσχημόνως καὶ τὰ δεινὰ πάντα διαφεύγουσι, πολλοὶ δὲ φίλοις τε βοηθοῦσι καὶ τὴν πατρίδα εὐεργετοῦσι καὶ διὰ ταῦτα χάριτός τε ἀξιοῦνται καὶ δόξαν μεγάλην κτῶνται καὶ τιμῶν καλλίστων τυγχάνουσι καὶ διὰ ταῦτα τόν τε λοιπὸν βίον ἥδιον καὶ κάλλιον διαζῶσι καὶ τοῖς ἑαυτῶν παισὶ καλλίους ἀφορμὰς εἰς τὸν βίον καταλείπουσιν. οὔτοι χρή, ὅτι οὐκ ἀσκεῖ δημοσίᾳ ἡ πόλις τὰ πρὸς τὸν πόλεμον, διὰ τοῦτο καὶ ἰδίᾳ ἀμελεῖν, ἀλλὰ μηδὲν ἧττον ἐπιμελεῖσθαι. εὖ γὰρ ἴσθι ὅτι οὐδὲ ἐν ἄλλῳ οὐδενὶ ἀγῶνι οὐδὲ ἐν πράξει οὐδεμιᾷ μεῖον ἕξεις διὰ τὸ βέλτιον τὸ σῶμα παρεσκευάσθαι· πρὸς πάντα γὰρ ὅσα πράττουσιν ἄνθρωποι χρήσιμον τὸ σῶμά ἐστιν· ἐν πάσαις δὲ ταῖς τοῦ σώματος χρείαις πολὺ διαφέρει ὡς βέλτιστα τὸ σῶμα ἔχειν· ἐπεὶ καὶ ἐν ᾧ δοκεῖ ἐλαχίστη σώματος χρεία εἶναι, ἐν τῷ διανοεῖσθαι, τίς οὐκ οἶδεν ὅτι καὶ ἐν τούτῳ πολλοὶ μεγάλα σφάλλονται διὰ τὸ μὴ ὑγιαίνειν τὸ σῶμα; καὶ λήθη δὲ καὶ ἀθυμία καὶ δυσκολία καὶ μανία πολλάκις πολλοῖς διὰ τὴν τοῦ σώματος καχεξίαν εἰς τὴν διάνοιαν ἐμπίπτουσιν οὕτως ὥστε καὶ τὰς ἐπιστήμας ἐκβάλλειν.

\end{greek}

}

\section*{Analiza i komentar}


%1

{\large
\begin{greek}
\noindent Καὶ πολλοὶ μὲν \\
\tabto{4em} διὰ τοῦτο \\
\tabto{4em} ἐκ τῶν πολεμικῶν ἀγώνων \\
\tabto{2em} σῴζονταί τε εὐσχημόνως \\
\tabto{2em} καὶ τὰ δεινὰ πάντα διαφεύγουσι, \\
πολλοὶ δὲ \\
\tabto{2em} φίλοις τε βοηθοῦσι \\
\tabto{2em} καὶ τὴν πατρίδα εὐεργετοῦσι \\
\tabto{2em} καὶ διὰ ταῦτα \\
\tabto{4em} χάριτός τε ἀξιοῦνται \\
\tabto{4em} καὶ δόξαν μεγάλην κτῶνται \\
\tabto{4em} καὶ τιμῶν καλλίστων τυγχάνουσι \\
\tabto{2em} καὶ διὰ ταῦτα \\
\tabto{4em} τόν τε λοιπὸν βίον ἥδιον καὶ κάλλιον διαζῶσι \\
\tabto{4em} καὶ τοῖς ἑαυτῶν παισὶ \\
\tabto{6em} καλλίους ἀφορμὰς \\
\tabto{6em} εἰς τὸν βίον \\
\tabto{4em} καταλείπουσιν.\\

\end{greek}
}

\begin{description}[noitemsep]
\item[πολλοὶ] §~196
\item[πολλοὶ μὲν\dots\ πολλοὶ δὲ\dots] koordinacija rečeničnih članova parom čestica
\item[διὰ τοῦτο] §~213.2, §~428
\item[ἐκ τῶν πολεμικῶν ἀγώνων] §~424, §~103, §~131
\item[σῴζονταί] σῴζω spasiti, izvući; 3. l. pl. ind. prez. medpas.
\item[εὐσχημόνως] §~136
\item[τὰ δεινὰ] §~103
\item[πάντα] §~193
\item[διαφεύγουσι] διαφεύγω izbjeći; 3. l. pl. ind. prez. akt.
\item[πολλοὶ ] §~196
\item[φίλοις] §~103
\item[βοηθοῦσι] βοηθέω pomagati; 3. l. pl. ind. prez. akt.
\item[τὴν πατρίδα] §~123
\item[εὐεργετοῦσι] εὐεργετέω τινά činiti dobro komu; 3. l. pl. ind. prez. akt.
\item[διὰ ταῦτα] §~213.2, §~428
\item[χάριτός] §~123, §~129
\item[ἀξιοῦνται] ἀξιόω τινός smatrati vrijednim čega; 3. l. pl. ind. prez. akt.
\item[δόξαν] §~97
\item[μεγάλην ] §~196
\item[κτῶνται] κτάομαι steći; 3. l. pl. ind. prez. akt.
\item[τιμῶν καλλίστων] §~90
\item[τυγχάνουσι] τυγχάνω τινός domoći se čega, postići što; 3. l. pl. ind. prez. akt.
\item[διὰ ταῦτα] §~213.2, §~428
\item[τόν τε λοιπὸν βίον] §~103, §~82
\item[ἥδιον] §~200, §~204.3
\item[κάλλιον] §~200, §~204.3
\item[διαζῶσι] διαζάω proživjeti; 3. l. pl. ind. prez. akt.
\item[τοῖς ἑαυτῶν παισὶ] §~208, §~193
\item[καλλίους ἀφορμὰς] §~200, §~90, §~164
\item[εἰς τὸν βίον] §~82, §~419
\item[καταλείπουσιν] καταλείπω ostaviti; 3. l. pl. ind. prez. akt.

\end{description}

{\large
\begin{greek}
\noindent οὔτοι χρή, \\
ὅτι οὐκ ἀσκεῖ \\
\tabto{2em} δημοσίᾳ \\
ἡ πόλις \\
τὰ πρὸς τὸν πόλεμον, \\
διὰ τοῦτο \\
\tabto{2em} καὶ ἰδίᾳ \\
ἀμελεῖν, \\
ἀλλὰ μηδὲν ἧττον ἐπιμελεῖσθαι.\\

\end{greek}
}

\begin{description}[noitemsep]
\item[οὔτοι] zacijelo ne
\item[χρή] §~492; otvara mjesto infinitivima kao subjektima
\item[ὅτι οὐκ ἀσκεῖ\dots\ ἡ πόλις ]  uzročna rečenica §~518.5, §~468
\item[ἀσκεῖ] ἀσκέω vježbati; 3. l. sg. ind. prez. akt.; §~165
\item[δημοσίᾳ] prilog izveden od pridjeva δημόσιος
\item[τὰ πρὸς τὸν πόλεμον] §~435, §~82; poimeničenje članom §~373
\item[διὰ τοῦτο] §213.2 §~428
\item[ἰδίᾳ] prilog izveden od pridjeva ἴδιος
\item[ἀμελεῖν] ἀμελέω zanemarivati; inf. prez. akt.
\item[μηδὲν ] §~224.2
\item[ἧττον] §~202, §~137
\item[ἐπιμελεῖσθαι] ἐπιμελέομαι τι brinuti se za što, baviti se čime; inf. prez. medpas.

\end{description}
%3 itd

{\large
\begin{greek}
\noindent εὖ γὰρ ἴσθι \\
\tabto{2em} ὅτι \\
\tabto{4em} οὐδὲ ἐν ἄλλῳ οὐδενὶ ἀγῶνι \\
\tabto{4em} οὐδὲ ἐν πράξει οὐδεμιᾷ \\
\tabto{2em} μεῖον ἕξεις \\
\tabto{4em} διὰ τὸ βέλτιον τὸ σῶμα παρεσκευάσθαι·\\

\end{greek}
}

\begin{description}[noitemsep]
\item[ἴσθι ] znaj; οἶδα znati; 2. l. sg. impt. perf. akt., perfekt sa značenjem prezenta, §~317
\item[ὅτι] uvodi izričnu rečenicu §~518, §~467
\item[ἐν ἄλλῳ οὐδενὶ ἀγῶνι] §~212, §~224.2, §~131, §~426
\item[ἐν πράξει οὐδεμιᾷ] §~165, §~224.2, §~426
\item[μεῖον] §~202, §~137
\item[ἕξεις] ἔχω imati; μεῖον ἔχω biti manje vrijednosti; 2. l. sg. ind. fut. akt.
\item[διὰ τὸ βέλτιον] §~428, §~202, §~137
\item[τὸ\dots\ παρεσκευάσθαι] poimeničeni infinitiv §~497
\item[τὸ σῶμα] §~123
\item[παρεσκευάσθαι] παρασκευάζω pripremiti; inf. perf. medpas.

\end{description}

%4

{\large
\begin{greek}
\noindent πρὸς πάντα γὰρ \\
\tabto{2em} ὅσα πράττουσιν ἄνθρωποι \\
χρήσιμον τὸ σῶμά ἐστιν· \\

\end{greek}
}

\begin{description}[noitemsep]
\item[πρὸς πάντα] §~193, §~435
\item[ὅσα ] §~219
\item[πράττουσιν] πράττω činiti; 3. l. pl. ind. prez. akt.
\item[ἄνθρωποι] §~82
\item[χρήσιμον] sc.\ ἐστιν; §~103
\item[τὸ σῶμά] §~123
\item[ἐστιν] εἰμί biti; 3. l. sg. ind. prez. akt.
\item[χρήσιμον\dots\ ἐστιν] imenski predikat, Smyth 909

\end{description}

%5

{\large
\begin{greek}
\noindent ἐν πάσαις δὲ ταῖς τοῦ σώματος χρείαις \\
πολὺ διαφέρει \\
\tabto{4em} ὡς βέλτιστα \\
\tabto{2em} τὸ σῶμα ἔχειν·\\

\end{greek}
}

\begin{description}[noitemsep]
\item[ἐν πάσαις ταῖς τοῦ σώματος χρείαις] §~426, §~193, §~90
\item[πολὺ] §~196
\item[διαφέρει ] διαφέρω razlikovati se, biti važno; 3. l. sg. ind. prez. akt.; bezlično, neprijelazno značenje §~447
\item[ὡς βέλτιστα] §~202, §~204.3; ὡς uz superlativ Smyth 1086
\item[τὸ σῶμα] §~123
\item[ἔχειν] ἔχω imati; inf. prez. akt.

\end{description}

%6

{\large
\begin{greek}
\noindent ἐπεὶ καὶ \\
\tabto{2em} ἐν ᾧ δοκεῖ \\
\tabto{4em} ἐλαχίστη σώματος χρεία εἶναι, \\
\tabto{2em} ἐν τῷ διανοεῖσθαι, \\
τίς οὐκ οἶδεν ὅτι \\
\tabto{4em} καὶ ἐν τούτῳ \\
\tabto{2em} πολλοὶ μεγάλα σφάλλονται \\
\tabto{4em} διὰ τὸ μὴ ὑγιαίνειν τὸ σῶμα;\\

\end{greek}
}

\begin{description}[noitemsep]
\item[ἐπεὶ] §~518
\item[ἐν ᾧ ] §~215, §~426
\item[δοκεῖ] δοκέω \textit{bezlično} čini se, 3. l. sg. ind. prez. akt.
\item[ἐλαχίστη] §~202
\item[σώματος] §~123
\item[χρεία] §~90
\item[εἶναι] εἰμί biti; inf. prez. akt.
\item[ἐλαχίστη\dots\ εἶναι] imenski predikat, Smyth 909
\item[ἐν τῷ διανοεῖσθαι ] διανοέομαι imati na umu, razmišljati; inf. prez. medpas.; poimeničeni infinitiv §~497
\item[τίς] §~217
\item[οἶδεν] οἶδα znati; 3. l. sg. ind. perf. akt., perfekt sa značenjem prezenta, §~317
\item[ὅτι\dots\ σφάλλονται] izrična rečenica §~518
\item[σφάλλονται] σφάλλω varati, prevariti; 3. l. pl. ind. prez. medpas.
\item[ἐν τούτῳ] §~213.2, §~426
\item[πολλοὶ] §~196
\item[μεγάλα] §~196
\item[διὰ τὸ μὴ ὑγιαίνειν ] ὑγιαίνω biti zdrav; inf. prez. akt.; poimeničeni infinitiv §~497; negacija μὴ §~508, 509
\item[τὸ σῶμα] §~123

\end{description}


%7

{\large
\begin{greek}
\noindent καὶ λήθη δὲ \\
καὶ ἀθυμία \\
καὶ δυσκολία \\
καὶ μανία \\
\tabto{2em} πολλάκις \\
\tabto{2em} πολλοῖς \\
\tabto{2em} διὰ τὴν τοῦ σώματος καχεξίαν \\
\tabto{2em} εἰς τὴν διάνοιαν \\
ἐμπίπτουσιν \\
\tabto{2em} οὕτως ὥστε \\
\tabto{4em} καὶ τὰς ἐπιστήμας ἐκβάλλειν.\\

\end{greek}
}

\begin{description}[noitemsep]
\item[λήθη ] §~90
\item[ἀθυμία] §~90
\item[δυσκολία] §~90
\item[μανία] §~90
\item[πολλοῖς] §196
\item[διὰ τὴν τοῦ σώματος καχεξίαν] §~123, §~90, §~428
\item[εἰς τὴν διάνοιαν] §~97, §~419
\item[ἐμπίπτουσιν] ἐμπίπτω napasti, zahvatiti; 3. l. pl. ind. prez. akt.
\item[ὥστε καὶ τὰς ἐπιστήμας ἐκβάλλειν] §~90; posljedična rečenica s predikatom u infinitivu §~473
\item[ἐκβάλλειν] ἐκβάλλω izbaciti, odbaciti; inf. prez. akt.
\end{description}

%kraj
