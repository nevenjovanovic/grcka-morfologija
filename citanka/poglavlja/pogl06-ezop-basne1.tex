%Unesene korekcije NZ i NJ
%\section*{O autoru}



\section*{O tekstu}

Poučna priča o savezništvu orla i lisice, o izdaji i božjoj kazni, dio je zbirke Ezopovih basni \textgreek[variant=ancient]{(Μῦθοι);} mada su one prvi put u proznu zbirku okupljene oko 300.\ pr.~Kr, u obliku u kojem su do nas došle zapravo su prozne parafraze kasnijih Babrijevih jampskih pjesama (I./II.\ st.\ po Kr) i radovi škola retorike različitih vremena. 

Jednu je varijantu basne o orlu i lisici obradio već pjesnik Arhiloh \textgreek[variant=ancient]{(Ἀρχίλοχος,} oko 650.\ pr.~Kr), drugu donose latinski stihovi oslobođenog rimskog roba Fedra (Phaedrus, oko 15.\ pr.~Kr. – 50.\ po Kr).

%\newpage

\section*{Pročitajte naglas grčki tekst.}
Aesop. Fabulae 1
%Naslov prema izdanju

\medskip

{\large
\begin{greek}
\noindent ΑΕΤΟΣ ΚΑΙ ΑΛΩΠΗΞ

\noindent Ἀετὸς καὶ ἀλώπηξ φιλίαν πρὸς ἀλλήλους σπεισάμενοι πλησίον ἑαυτῶν οἰκεῖν διέγνωσαν βεβαίωσιν φιλίας τὴν συνήθειαν ποιούμενοι. καὶ δὴ ὁ μὲν ἀναβὰς ἐπί τι περίμηκες δένδρον ἐνεοττοποιήσατο, ἡ δὲ εἰς τὸν ὑποκείμενον θάμνον ἔτεκεν. ἐξελθούσης δέ ποτε αὐτῆς ἐπὶ νομὴν ὁ ἀετὸς ἀπορῶν τροφῆς καταπτὰς εἰς τὸν θάμνον καὶ τὰ γεννήματα ἀναρπάσας μετὰ τῶν αὑτοῦ νεοττῶν κατεθοινήσατο. ἡ δὲ ἀλώπηξ ἐπανελθοῦσα ὡς ἔγνω τὸ πραχθέν, οὐ μᾶλλον ἐπὶ τῷ τῶν νεοττῶν θανάτῳ ἐλυπήθη, ὅσον ἐπὶ τῆς ἀμύνης· χερσαία γὰρ οὖσα πετεινὸν διώκειν ἠδυνάτει. διόπερ πόρρωθεν στᾶσα, ὃ μόνον τοῖς ἀσθενέσιν καὶ ἀδυνάτοις ὑπολείπεται, τῷ ἐχθρῷ κατηρᾶτο. συνέβη δὲ αὐτῷ τῆς εἰς τὴν φιλίαν ἀσεβείας οὐκ εἰς μακρὰν δίκην ὑποσχεῖν. θυόντων γάρ τινων αἶγα ἐπ' ἀγροῦ καταπτὰς ἀπὸ τοῦ βωμοῦ σπλάγχνον ἔμπυρον ἀνήνεγκεν· οὗ κομισθέντος ἐπὶ τὴν καλιὰν σφοδρὸς ἐμπεσὼν ἄνεμος ἐκ λεπτοῦ καὶ παλαιοῦ κάρφους λαμπρὰν φλόγα ἀνῆψε. καὶ διὰ τοῦτο καταφλεχθέντες οἱ νεοττοὶ — καὶ γὰρ ἦσαν ἔτι ἀτελεῖς οἱ πτηνοὶ — ἐπὶ τὴν γῆν κατέπεσον. καὶ ἡ ἀλώπηξ προσδραμοῦσα ἐν ὄψει τοῦ ἀετοῦ πάντας αὐτοὺς κατέφαγεν. 

ὁ λόγος δηλοῖ, ὅτι οἱ φιλίαν παρασπονδοῦντες, κἂν τὴν τῶν ἠδικημένων ἐκφύγωσι κόλασιν, ἀλλ' οὖν γε τὴν ἐκ θεοῦ τιμωρίαν οὐ διακρούσονται.

\end{greek}
}

\section*{Analiza i komentar}

%1

{\large
\begin{greek}
\noindent ΑΕΤΟΣ ΚΑΙ ΑΛΩΠΗΞ

\end{greek}
}

\begin{description}[noitemsep]
\item[ΑΕΤΟΣ] §~82
\item[ΑΛΩΠΗΞ] §~115

\end{description}

{\large
\begin{greek}
\noindent Ἀετὸς καὶ ἀλώπηξ \\
\tabto{2em} φιλίαν \\
\tabto{3em} πρὸς ἀλλήλους \\
\tabto{2em} σπεισάμενοι \\
πλησίον ἑαυτῶν \\
\tabto{2em} οἰκεῖν \\
διέγνωσαν \\
\tabto{2em} βεβαίωσιν \\
\tabto{4em} φιλίας \\
\tabto{2em} τὴν συνήθειαν \\
\tabto{3em} ποιούμενοι.\\

\end{greek}
}

\begin{description}[noitemsep]
\item[Ἀετὸς] §~82
\item[ἀλώπηξ ] §~115
\item[φιλίαν ] §~90
\item[πρὸς ἀλλήλους ] §~435; §~212
\item[σπεισάμενοι ] σπένδω med. sklopiti savez; n. pl. m. r. ptc. aor. med.
\item[πλησίον ἑαυτῶν ] §~417; §~208
\item[οἰκεῖν ] οἰκέω stanovati, nastavati; inf. prez. akt.
\item[διέγνωσαν ] διαγιγνώσκω s inf. odlučiti nešto; 3. l. pl. ind. aor. akt.
\item[βεβαίωσιν ] §~165
\item[φιλίας ] §~90
\item[τὴν συνήθειαν ] §~90
\item[ποιούμενοι] ποιέω τινά τι koga ili što činiti nečim; n. pl. m. r. ptc. prez. medpas.

\end{description}


%3 itd
{\large
\begin{greek}
\noindent καὶ δὴ \\
ὁ μὲν \\
\tabto{2em} ἀναβὰς \\
\tabto{4em} ἐπί τι περίμηκες δένδρον \\
ἐνεοττοποιήσατο, \\
ἡ δὲ \\
\tabto{2em} εἰς τὸν ὑποκείμενον θάμνον \\
ἔτεκεν.\\

\end{greek}
}

\begin{description}[noitemsep]
\item[καὶ δὴ ] kombinacija čestica povezuje rečenicu s prethodnom, najavljujući nešto važno ili zanimljivo: i onda\dots
\item[ὁ μὲν\dots\, ἡ δὲ\dots] koordinacija subjekata pomoću čestica; §~370.1
\item[ἀναβὰς] ἀναβαίνω uzaći, popeti se, \textit{ovdje} uzletjeti; n. sg. m. r. ptc. aor. akt.
\item[ἐπί τι] §~40
\item[ἐπί\dots\ δένδρον] §~436; §~82
\item[τι ] §~217
\item[περίμηκες ] §~153
\item[ἐνεοττοποιήσατο] νεοσσοποιέω, atički νεοττοποιέω saviti gnijezdo, izleći jaja; 3. l. sg. ind. aor. med.
\item[εἰς τὸν\dots\ θάμνον] §~419; §~82; §~375
\item[ὑποκείμενον] ὑπόκειμαι ležati ispod, biti ispod; a. sg. m. r. ptc. prez. medpas.
\item[ἔτεκεν] τίκτω roditi, (za životinje) okotiti se; 3. l. sg. ind. aor. akt.

\end{description}

%4

{\large
\begin{greek}
\noindent \uuline{ἐξελθούσης} \\
\tabto{2em} δέ ποτε \\
\uuline{αὐτῆς} \\
\tabto{2em} ἐπὶ νομὴν \\
ὁ ἀετὸς \\
\tabto{2em} ἀπορῶν \\
\tabto{4em} τροφῆς \\
\tabto{2em} καταπτὰς \\
\tabto{4em} εἰς τὸν θάμνον \\
\tabto{2em} καὶ \\
\tabto{4em} τὰ γεννήματα \\
\tabto{2em} ἀναρπάσας \\
μετὰ τῶν αὑτοῦ νεοττῶν \\
κατεθοινήσατο. \\

\end{greek}
}

\begin{description}[noitemsep]
\item[ἐξελθούσης ] ἐξέρχομαι izaći; g. sg. ž. r. ptc. aor. akt.
\item[δέ ποτε ] §~40; čestica δέ povezuje rečenicu s prethodnom: a\dots
\item[αὐτῆς ] §~207
\item[ἐπὶ νομὴν ] §~436; §~90
\item[ὁ ἀετὸς ] §~82
\item[ἀπορῶν ] ἀπορέω τινός ne znati što da radi, nemati nešto; n. sg. m. r. ptc. prez. akt.
\item[τροφῆς ] §~90
\item[καταπτὰς ] καταπέτομαι sletjeti; n. sg. m. r. ptc. aor. akt.
\item[εἰς τὸν θάμνον] §~419; §~82
\item[τὰ γεννήματα ] §~123
\item[ἀναρπάσας ] ἀναρπάζω ugrabiti; n. sg. m. r. ptc. aor. akt.
\item[μετὰ τῶν\dots\ νεοττῶν] §~430; §~82; §~375
\item[αὑτοῦ] §~209.1
\item[κατεθοινήσατο] καταθοινάω (i med.) gostiti se; 3. l. sg. ind. aor. med.

\end{description}


%5

{\large
\begin{greek}
\noindent ἡ δὲ ἀλώπηξ \\
\tabto{2em} ἐπανελθοῦσα \\
\tabto{3em} ὡς ἔγνω \\
\tabto{4em} τὸ πραχθέν, \\
οὐ μᾶλλον \\
\tabto{2em} ἐπὶ τῷ τῶν νεοττῶν θανάτῳ \\
ἐλυπήθη, \\
ὅσον \\
\tabto{2em} ἐπὶ τῆς ἀμύνης· \\
χερσαία γὰρ οὖσα \\
\tabto{2em} πετεινὸν διώκειν \\
ἠδυνάτει.\\

\end{greek}
}

\begin{description}[noitemsep]
\item[ἡ δὲ ἀλώπηξ ] §~115; čestica δέ povezuje rečenicu s prethodnom i ističe promjenu subjekta (u odnosu na \textgreek[variant=ancient]{ὁ ἀετὸς)}: a\dots
\item[ἐπανελθοῦσα ] ἐπανέρχομαι vratiti se; n. sg. ž. r. ptc. aor. akt.
\item[ὡς ] veznik uvodi vremensku rečenicu; §~487
\item[ἔγνω] γιγνώσκω saznati, shvatiti; 3. l. sg. ind. aor. akt.
\item[τὸ πραχθέν] §~373; πράσσω, atički πράττω činiti; a. sg. sr. r. ptc. aor. pas.
\item[οὐ μᾶλλον\dots\, ὅσον\dots] koordinacija; §~204.3
\item[ἐπὶ τῷ\dots\ θανάτῳ] §~436; §~82; §~375
\item[τῶν νεοττῶν ] §~82
\item[ἐλυπήθη] λυπέω pas. biti žalostan; 3. l. sg. ind. aor. pas.
\item[ὅσον] §~219
\item[ἐπὶ τῆς ἀμύνης] §~436; §~90
\item[χερσαία ] §~103
\item[γὰρ ] čestica najavljuje iznošenje dokaza prethodne tvrdnje: naime\dots
\item[οὖσα ] εἰμί biti; n. sg. ž. r. ptc. prez. (akt.)
\item[πετεινὸν ] §~103
\item[διώκειν ] διώκω progoniti, slijediti; inf. prez. akt.
\item[ἠδυνάτει] ἀδυνατέω biti nemoguće, ne biti u stanju; 3. l. sg. impf. akt.

\end{description}


%6

{\large
\begin{greek}
\noindent διόπερ \\
\tabto{2em} πόρρωθεν \\
στᾶσα, \\
ὃ μόνον \\
\tabto{2em} τοῖς ἀσθενέσιν \\
\tabto{2em} καὶ ἀδυνάτοις \\
ὑπολείπεται, \\
\tabto{2em} τῷ ἐχθρῷ \\
κατηρᾶτο.\\

\end{greek}
}

\begin{description}[noitemsep]
\item[στᾶσα] ἵστημι intr. stajati; n. sg. ž. r. ptc. aor. akt.
\item[ὃ ] §~215
\item[μόνον ] §~204.2
\item[τοῖς ἀσθενέσιν ] §~373; §~153
\item[ἀδυνάτοις ] §~106
\item[ὑπολείπεται] ὑπολείπω pas. ostati; 3. l. sg. ind. prez. medpas.
\item[τῷ ἐχθρῷ ] §~373; §~103
\item[κατηρᾶτο] καταράομαι prokleti; 3. l. sg. impf. (medpas.)

\end{description}

%7

{\large
\begin{greek}
\noindent συνέβη δὲ \\
\tabto{2em} αὐτῷ \\
τῆς \\
\tabto{2em} εἰς τὴν φιλίαν \\
ἀσεβείας \\
οὐκ εἰς μακρὰν \\
\tabto{2em} δίκην ὑποσχεῖν.\\

\end{greek}
}

\begin{description}[noitemsep]
\item[συνέβη ] συμβαίνω dogoditi se; 3. l. sg. ind. aor. akt.
\item[δὲ ] čestica δέ povezuje rečenicu s prethodnom: a\dots
\item[αὐτῷ ] §~207
\item[τῆς\dots\ ἀσεβείας] §~90; §~375
\item[εἰς τὴν φιλίαν] §~419; §~90
\item[οὐκ εἰς μακρὰν] (sc.\ ὥραν) fraza: ne zadugo, ubrzo; §~419; §~103
\item[δίκην] §~90
\item[ὑποσχεῖν] ὑπέχω trpjeti; inf. aor. akt.

\end{description}

%8

{\large
\begin{greek}
\noindent \uuline{θυόντων} γάρ \uuline{τινων} \\
\tabto{2em} αἶγα \\
\tabto{2em} ἐπ' ἀγροῦ \\
καταπτὰς \\
\tabto{2em} ἀπὸ τοῦ βωμοῦ \\
σπλάγχνον \\
\tabto{2em} ἔμπυρον \\
ἀνήνεγκεν· \\
\uuline{οὗ κομισθέντος} \\
\tabto{2em} ἐπὶ τὴν καλιὰν \\
σφοδρὸς \\
\tabto{2em} ἐμπεσὼν \\
ἄνεμος \\
\tabto{2em} ἐκ λεπτοῦ καὶ παλαιοῦ κάρφους \\
λαμπρὰν φλόγα \\
ἀνῆψε.\\

\end{greek}
}

\begin{description}[noitemsep]
\item[θυόντων ] θύω žrtvovati, prinositi žrtvu; g. pl. m. r. ptc. prez. akt.
\item[γάρ τινων] §~40
\item[γάρ ] čestica najavljuje iznošenje dokaza prethodne tvrdnje: naime\dots
\item[τινων ] §~217
\item[αἶγα ] §~115
\item[ἐπ' ἀγροῦ ] §~68; §~436; §~82
\item[καταπτὰς ] καταπέτομαι sletjeti; n. sg. m. r. ptc. aor. akt.
\item[ἀπὸ τοῦ βωμοῦ ] §~423; §~82
\item[σπλάγχνον ] §~82
\item[ἔμπυρον ] §~106
\item[ἀνήνεγκεν] ἀναφέρω odnijeti; 3. l. sg. ind. aor. akt.
\item[οὗ ] §~215
\item[κομισθέντος ] κομίζω donijeti; g. sg. sr. r. ptc. aor. pas.
\item[ἐπὶ τὴν καλιὰν ] §~436; §~90
\item[σφοδρὸς ] §~103
\item[ἐμπεσὼν ] ἐμπίπτω pasti na nešto, raspiriti; n. sg. m. r. ptc. aor. akt.
\item[ἄνεμος ] §~82
\item[ἐκ\dots\ κάρφους] §~424; §~153
\item[λεπτοῦ\dots\ παλαιοῦ ] §~103
\item[λαμπρὰν ] §~103
\item[φλόγα ] §~115
\item[ἀνῆψε] ἀνάπτω zapaliti; 3. l. sg. ind. aor. akt.

\end{description}

%9

%\newpage

{\large
\begin{greek}
\noindent καὶ διὰ τοῦτο \\
καταφλεχθέντες \\
οἱ νεοττοὶ —\\
καὶ γὰρ \\
\tabto{2em} ἦσαν \\
\tabto{4em} ἔτι ἀτελεῖς \\
\tabto{2em} οἱ πτηνοὶ —\\
ἐπὶ τὴν γῆν \\
κατέπεσον. \\

\end{greek}
}

\begin{description}[noitemsep]
\item[διὰ τοῦτο ] §~428; §~213.2
\item[καταφλεχθέντες ] καταφλέγω pas. izgorjeti; n. pl. m. r. ptc. aor. pas.
\item[οἱ νεοττοὶ] §~82
\item[καὶ γὰρ ] §~517
\item[ἦσαν] εἰμί biti; 3. l. pl. impf. (akt.)
\item[ἀτελεῖς ] §~153
\item[οἱ πτηνοὶ] §~373; §~103
\item[ἐπὶ τὴν γῆν] §~436; §~106
\item[κατέπεσον] καταπίπτω pasti; 3. l. pl. ind. aor. akt.

\end{description}

%10

{\large
\begin{greek}
\noindent καὶ ἡ ἀλώπηξ \\
\tabto{2em} προσδραμοῦσα \\
\tabto{2em} ἐν ὄψει \\
\tabto{4em} τοῦ ἀετοῦ \\
πάντας αὐτοὺς \\
κατέφαγεν.\\

\end{greek}
}

\begin{description}[noitemsep]
\item[ἡ ἀλώπηξ ] §~115
\item[προσδραμοῦσα ] προστρέχω pritrčati; n. sg. ž. r. ptc. aor. akt.
\item[ἐν ὄψει ] §~426; §~165
\item[τοῦ ἀετοῦ ] §~82
\item[πάντας ] §~193
\item[αὐτοὺς ] §~207
\item[κατέφαγεν] κατεσθίω pojesti; 3. l. sg. ind. aor. akt.

\end{description}

%11

{\large
\begin{greek}
\noindent ὁ λόγος δηλοῖ, \\
\tabto{2em} ὅτι \\
\tabto{2em} οἱ φιλίαν παρασπονδοῦντες, \\
\tabto{2em} κἂν τὴν \\
\tabto{4em} τῶν ἠδικημένων \\
\tabto{2em} ἐκφύγωσι \\
\tabto{4em} κόλασιν, \\
\tabto{2em} ἀλλ' οὖν γε \\
\tabto{2em} τὴν \\
\tabto{4em} ἐκ θεοῦ \\
\tabto{2em} τιμωρίαν \\
\tabto{2em} οὐ διακρούσονται.\\

\end{greek}
}

\begin{description}[noitemsep]
\item[ὁ λόγος ] §~82; ovdje u značenju ``basna'', LSJ s.~v. V.1
\item[δηλοῖ] δηλόω pokazati; 3. l. sg. ind. prez. akt.
\item[ὅτι ] veznik uvodi izričnu rečenicu, §~467
\item[οἱ\dots\ παρασπονδοῦντες] §~373; παρασπονδέω prekršiti sporazum ili vjeru; n. pl. m. r. ptc. prez. akt.; §~375; supstantiviranje članom §~373
\item[φιλίαν] §~90
\item[κἂν\dots\ ἐκφύγωσι\dots, οὐ διακρούσονται] §~480; dopusna rečenica uvedena s καὶ εἰ (ἄν); ἐκφεύγω izbjeći; 3. l. pl. konj. aor. akt.; διακρούω med. izmicati; 3. l. pl. ind. fut. med.
\item[τὴν\dots\ κόλασιν] §~165; §~375
\item[τῶν ἠδικημένων] §~373; genitiv subjektni uz κόλασιν; ἀδικέω nanositi nepravdu; g. pl. m. r. ptc. perf. medpas.
\item[ἀλλ' οὖν γε] §~68; §~40; §~519.1
\item[τὴν\dots\ τιμωρίαν] §~90; §~375
\item[ἐκ θεοῦ] §~424; §~82

\end{description}


%kraj
