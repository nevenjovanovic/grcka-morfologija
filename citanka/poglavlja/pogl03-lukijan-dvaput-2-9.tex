% redaktura NZ
\section*{O autoru}

Lukijan \textgreek[variant=ancient]{(Λουκιανός,} Lucianus) bio je retoričar i prozni pisac (Samozata na rijeci Eufratu, danas Samsat u Turskoj, oko 115. – Atena, nakon 180.\ po Kr.). Životopis mu se može tek uvjetno rekonstruirati iz rijetkih autobiografskih očitovanja. Materinski mu je jezik najvjerojatnije bio aramejski, no izvrsno je ovladao grčkim i stekao zavidno retoričko i književno obrazovanje; nakon kraće odvjetničke karijere i mnogobrojnih putovanja, na kojima je, u tradiciji druge sofistike, držao javna predavanja, deklamirao i poučavao retoriku, nastanio se u Ateni i posvetio pisanju.

Očuvano je osamdesetak njegovih spisa (desetak ih je dvojbene autentičnosti). Žanrovski se obično dijele na retoričke spise, dijaloge, menipske satire, pamflete i pripovjedna djela. Danas su možda najpopularnije \textit{Istinite pripovijesti} \textgreek[variant=ancient]{(Ἀληϑῆ διηγήματα),} žanrovski hibrid parodije, menipske satire i znanstvenofantastične pripovijesti.

Skeptičan prema svakoj dogmi, nemilosrdan u izrugivanju ljudske gluposti, Lukijan je bio osobito rado čitan u humanizmu (Poggio Bracciolini, Giovanni Pontano, Ulrich von Hutten, Philipp Melanchthon); Erazmova \textit{Pohvala ludosti} izravan je plod oduševljenja Lukijanom.

\section*{O tekstu}

Dijalog \textgreek[variant=ancient]{Δὶς κατηγορούμενος} \textit{(Bis accusatus}, \textit{Dvaput optužen}) počinje Zeusovom invektivom upućenom filozofima koji smatraju da je bogovima lako. Zeus objašnjava da nipošto nije tako.

\newpage

\section*{Pročitajte naglas grčki tekst.}

Luc.\ Bis accusatus 2.9

\medskip

{\large
\begin{greek}
\noindent Ἐγὼ δὲ αὐτὸς ὁ πάντων βασιλεὺς καὶ πατὴρ ὅσας μὲν ἀηδίας ἀνέχομαι, ὅσα δὲ πράγματα ἔχω πρὸς τοσαύτας φροντίδας διῃρημένος· ᾧ πρῶτα μὲν τὰ τῶν ἄλλων θεῶν ἔργα ἐπισκοπεῖν ἀναγκαῖον ὁπόσοι τι ἡμῖν συνδιαπράττουσι τῆς ἀρχῆς, ὡς μὴ βλακεύωσιν ἐν αὐτοῖς, ἔπειτα δὲ καὶ αὐτῷ μυρία ἄττα πράττειν καὶ σχεδὸν ἀνέφικτα ὑπὸ λεπτότητος· οὐ γὰρ μόνον τὰ κεφάλαια ταῦτα τῆς  διοικήσεως, ὑετοὺς καὶ χαλάζας καὶ πνεύματα καὶ ἀστραπὰς αὐτὸς οἰκονομησάμενος καὶ διατάξας πέπαυμαι τῶν ἐπὶ μέρους φροντίδων ἀπηλλαγμένος, ἀλλά με δεῖ καὶ ταῦτα μὲν ποιεῖν ἀποβλέπειν δὲ κατὰ τὸν αὐτὸν χρόνον ἁπανταχόσε καὶ πάντα ἐπισκοπεῖν ὥσπερ τὸν ἐν τῇ Νεμέᾳ βουκόλον, τοὺς κλέπτοντας, τοὺς ἐπιορκοῦντας, τοὺς θύοντας, εἴ τις ἔσπεισε, πόθεν ἡ κνῖσα καὶ ὁ καπνὸς ἀνέρχεται, τίς νοσῶν ἢ πλέων ἐκάλεσεν, καὶ τὸ πάντων ἐπιπονώτατον, ὑφ' ἕνα καιρὸν ἔν τε ᾿Ολυμπίᾳ τῇ ἑκατόμβῃ παρεῖναι καὶ ἐν Βαβυλῶνι τοὺς πολεμοῦντας ἐπισκοπεῖν καὶ ἐν Γέταις χαλαζᾶν καὶ ἐν Αἰθίοψιν εὐωχεῖσθαι.

\end{greek}

}

\section*{Analiza i komentar}
\begin{greek}

{\large
\noindent Ἐγὼ δὲ αὐτὸς \\
\tabto{2em} ὁ πάντων βασιλεὺς καὶ πατὴρ \\
ὅσας μὲν ἀηδίας ἀνέχομαι, \\
ὅσα δὲ πράγματα ἔχω \\
\tabto{2em} πρὸς τοσαύτας φροντίδας \\
διῃρημένος· \\

}
\end{greek}

%komentar
%zašto glagoli upućuju na paragrafe?

\begin{description}[noitemsep]
\item[Ἐγὼ] §~205
\item[δὲ] čestica izražava suprotnost u odnosu na prethodnu (ovdje izostavljenu) rečenicu: a\dots
\item[αὐτὸς] §~207
\item[πάντων] §~193, ovisno o βασιλεὺς καὶ πατὴρ; atributni položaj §~375
\item[βασιλεὺς] §~175
\item[πατὴρ] §~146
\item[ὅσας μὲν\dots\ ὅσα δὲ\dots] koordinacija rečeničnih članova s pomoću para čestica: a\dots
\item[ὅσας] §~219, §~103
\item[ἀηδίας] §~90
\item[ἀνέχομαι] ἀνέχω podnositi; 1. l. sg. ind. prez. medpas.; §~327, §~447.1 
\item[ὅσα πράγματα] §~219, §~103, §~123
\item[ἔχω] ἔχω imati, πράγματα ἔχω (fraza) imati posla, imati gnjavaže; 1. l. sg. ind. prez. akt.; §~327; §~447.1 
\item[πρὸς τοσαύτας φροντίδας] §~123, §~435; πρός ovdje: na
\item[διῃρημένος] διαιρέω πρός τι dijeliti na što, među čime; n. sg. m. r. ptc. perf. medpas. §~327
\end{description}

\begin{greek}

{\large
\noindent ᾧ πρῶτα μὲν \\
\tabto{2em} τὰ τῶν ἄλλων θεῶν ἔργα \\
ἐπισκοπεῖν \\
\tabto{2em} ἀναγκαῖον \\
\tabto{4em} ὁπόσοι \\
\tabto{6em} τι \\
\tabto{4em} ἡμῖν \\
\tabto{4em} συνδιαπράττουσι \\
\tabto{6em} τῆς ἀρχῆς, \\
\tabto{6em} ὡς μὴ βλακεύωσιν ἐν αὐτοῖς, \\
ἔπειτα δὲ καὶ αὐτῷ \\
\tabto{2em} μυρία ἄττα \\
\tabto{2em} πράττειν \\
\tabto{2em} καὶ σχεδὸν ἀνέφικτα \\
\tabto{4em} ὑπὸ λεπτότητος· 

}
\end{greek}

%komentar

\begin{description}[noitemsep]
\item[ᾧ] §~215; odnosna zamjenica uvodi zavisnu odnosnu rečenicu, antecedent Ἐγὼ
\item[πρῶτα μὲν\dots\ ἔπειτα δὲ\dots] koordinacija rečeničnih članova s pomoću para čestica: a\dots
\item[πρῶτα] §~391
\item[τὰ τῶν ἄλλων θεῶν ἔργα] §~80, §~82; §~212; posvojni genitiv u atributnom položaju, §~375
\item[ἐπισκοπεῖν] ἐπισκοπέω nadgledati; inf. prez. akt., dopuna uz ἀναγκαῖον, a otvara mjesto zavisnoj relativnoj rečenici uvedenoj zamjenicom ὁπόσοι
\item[ἀναγκαῖον] §~103; ἀναγκαῖον τινί ἐστι: nekomu je nužno, netko mora; imenski predikat (s izostavljenom kopulom) Smyth 909; fraza otvara mjesto dopuni u infinitivu
\item[ὁπόσοι] §~219; složena odnosna zamjenica uvodi zavisnu relativnu rečenicu, §~443; antecedent τῶν ἄλλων θεῶν
\item[τι] §~217
\item[ἡμῖν] §~205
\item[συνδιαπράττουσι] συνδιαπράττω τινί zajedno s kime vršiti, pomoći komu vršiti (što); 3. l. pl. ind. prez. akt. 
\item[τῆς ἀρχῆς] §~90; dijelni genitiv ovisan o τι
\item[ὡς μὴ] veznik i negacija uvode (zanijekanu) zavisnu namjernu rečenicu (konjunktiv stoji iza glavnog vremena): da ne bi\dots
\item[βλακεύωσιν] βλακεύω zabušavati; 3. l. pl. konj. prez. akt.
\item[ἐν αὐτοῖς] sc.\ ἔργοις; §~207, §~426
\item[αὐτῷ] §~207, sc.\ ἀναγκαῖον αὐτῷ ἐμοί ἐστι; i ovdje (neizrečeno) ἀναγκαῖον otvara mjesto dopuni u infinitivu
\item[μυρία] pridjev, §~103
\item[ἄττα] §~217
\item[πράττειν] πράττω činiti; inf. prez. akt. 
\item[ἀνέφικτα] §~103, §~106; ovisno o ἄττα
\item[ὑπὸ λεπτότητος] §~123, §~437
\end{description}

\begin{greek}

{\large
\noindent οὐ γὰρ μόνον \\
\tabto{2em} τὰ κεφάλαια ταῦτα τῆς διοικήσεως, \\
\tabto{4em} ὑετοὺς καὶ χαλάζας καὶ πνεύματα καὶ ἀστραπὰς \\
\tabto{2em} αὐτὸς οἰκονομησάμενος καὶ διατάξας \\
\tabto{2em} πέπαυμαι \\
\tabto{6em} τῶν ἐπὶ μέρους φροντίδων \\
\tabto{4em} ἀπηλλαγμένος, \\
ἀλλά \\
\tabto{2em} \underline{με} δεῖ \\
\tabto{4em} καὶ ταῦτα μὲν \underline{ποιεῖν} \\
\tabto{4em} \underline{ἀποβλέπειν} δὲ \\
\tabto{6em} κατὰ τὸν αὐτὸν χρόνον \\
\tabto{4em} ἁπανταχόσε \\
\tabto{4em} καὶ πάντα \underline{ἐπισκοπεῖν} \\
\tabto{6em} ὥσπερ \underline{τὸν ἐν τῇ Νεμέᾳ βουκόλον}, \\
\tabto{8em} τοὺς κλέπτοντας, \\
\tabto{8em} τοὺς ἐπιορκοῦντας, \\
\tabto{8em} τοὺς θύοντας, \\
\tabto{8em} εἴ τις ἔσπεισε, \\
\tabto{8em} πόθεν ἡ κνῖσα καὶ ὁ καπνὸς ἀνέρχεται, \\
\tabto{8em} τίς νοσῶν ἢ πλέων ἐκάλεσεν, \\
\tabto{8em} καὶ τὸ πάντων ἐπιπονώτατον, \\
\tabto{10em} ὑφ' ἕνα καιρὸν \\
\tabto{8em} ἔν τε ᾿Ολυμπίᾳ τῇ ἑκατόμβῃ \underline{παρεῖναι} \\
\tabto{8em} καὶ ἐν Βαβυλῶνι τοὺς πολεμοῦντας \underline{ἐπισκοπεῖν} \\
\tabto{8em} καὶ ἐν Γέταις \underline{χαλαζᾶν} \\
\tabto{8em} καὶ ἐν Αἰθίοψιν \underline{εὐωχεῖσθαι}.

}
\end{greek}

%komentar

\begin{description}[noitemsep]
\item[οὐ\dots\ μόνον\dots\ ἀλλά\dots] koordinacija rečeničnih članova: ne samo\dots\ nego\dots
\item[γὰρ] čestica najavljuje iznošenje objašnjenja prethodne tvrdnje: naime\dots
\item[τὰ κεφάλαια ταῦτα] §~103, §~213.2
\item[τῆς διοικήσεως] §~165
\item[ὑετοὺς] §~82
\item[χαλάζας] §~97
\item[πνεύματα] §~123
\item[ἀστραπὰς] §~90
\item[αὐτὸς] sc.\ Ἐγὼ
\item[οἰκονομησάμενος] οἰκονομέω τι upravljati čime; ptc. aor. med. n. sg. m. r.; odnosi se na objekte koji prethode (ὑετοὺς καὶ χαλάζας καὶ πνεύματα καὶ ἀστραπὰς)
\item[διατάξας] διατάσσω posložiti; ptc. aor. akt. n. sg. m. r.  
\item[πέπαυμαι] παύω med. παύομαι odmarati se; 1. l. sg. ind. perf. medpas.
\item[τῶν ἐπὶ μέρους φροντίδων] §~153, §~123; ἐπὶ μέρους poseban, specijalan; prijedložni izraz u atributnom položaju §~373
\item[ἀπηλλαγμένος] ἀπαλλάσσω med. ἀπαλλάσσομαί τινος osloboditi se čega; ptc. perf. medpas. n. sg. m. r.
\item[με] §~205, imenski dio A+I
\item[δεῖ] δέω trebati; 3. l. sg. ind. prez.; bezlično δεῖ otvara mjesto A+I (§~492)
\item[ταῦτα] §~213.2; demonstrativ upućuje na (prethodno) navedeno
\item[ταῦτα μὲν\dots\ ἀποβλέπειν δὲ\dots] koordinacija rečeničnih članova s pomoću para čestica
\item[ποιεῖν] ποιέω činiti; inf. prez. akt., dio A+I
\item[ἀποβλέπειν] ἀποβλέπω odvratiti pogled, gledati drugamo; inf. prez. akt., dio A+I
\item[κατὰ τὸν αὐτὸν χρόνον] §~82, §~207, §~429
\item[πάντα] §~193
\item[ἐπισκοπεῖν] ἐπισκοπέω nadzirati; inf. prez. akt., dio A+I
\item[τὸν ἐν τῇ Νεμέᾳ βουκόλον] §~426, §~82, §~90; prijedložni izraz u atributnom položaju, §~375; aluzija na Ijina čuvara u Nemeji, stookog diva Arga
\item[τοὺς κλέπτοντας] κλέπτω krasti; ptc. prez. akt. a. pl. m. r.; poimeničenje članom §~373
\item[τοὺς ἐπιορκοῦντας] ἐπιορκέω krivo se zaklinjati; ptc. prez. akt. a. pl. m. r.; poimeničenje članom §~373
\item[τοὺς θύοντας] θύω prinositi žrtvu; ptc. prez. akt. a. pl. m. r.; poimeničenje članom §~373
\item[εἴ] veznik uvodi zavisnu upitnu rečenicu (s indikativom) §~469 
\item[τις] §~217
\item[ἔσπεισε] σπένδω prinijeti žrtvu ljevanicu, izliti žrtvu; 3. l. sg. ind. aor. akt.
\item[πόθεν] §~221; upitni prilog uvodi zavisnu upitnu rečenicu (s indikativom) §~446, §~469
\item[ἡ κνῖσα καὶ ὁ καπνὸς] §~82, §~90
\item[ἀνέρχεται] ἀνέρχομαι dizati se; 3. l. sg. ind. prez. med.
\item[τίς] §~217; upitna zamjenica  uvodi zavisnu upitnu rečenicu (s indikativom) §~446, §~469 
\item[νοσῶν ἢ πλέων] νοσέω biti bolestan, ptc. prez. akt. n. sg. m. r.; πλέω ploviti, ptc. prez. akt. n. sg. m. r.; adverbni particip §~503
\item[ἐκάλεσεν] καλέω zvati, 3. l. sg. ind. aor. akt.  
\item[τὸ πάντων ἐπιπονώτατον] §~193, §~197, dijelni genitiv §~395; atributni položaj §~375
\item[ὑφ' ἕνα καιρὸν] glasovne promjene §~68, §~74; §~82, §~437, §~224
\item[ἔν τε ᾿Ολυμπίᾳ\dots] \textbf{καὶ ἐν Βαβυλῶνι\dots\ καὶ ἐν Γέταις\dots\ καὶ ἐν Αἰθίοψιν\dots}\ koordinacija ostvarena nizom sastavnih veznika
\item[ἔν\dots\ ᾿Ολυμπίᾳ] §~90, §~426; priložna oznaka mjesta
\item[τῇ ἑκατόμβῃ] §~90; objekt παρεῖναι
\item[παρεῖναι\dots\ ἐπισκοπεῖν\dots\ ἐπισκοπεῖν\dots\ εὐωχεῖσθαι] svi su ti infinitivi dio A+I ovisnog o δεῖ (v.~gore)
\item[παρεῖναι] πάρειμί τινι biti prisutan na čemu; inf. prez. akt.
\item[ἐν Βαβυλῶνι] §~131; priložna oznaka mjesta
\item[τοὺς πολεμοῦντας] πολεμέω ratovati; ptc. prez. akt. a. pl. m. r.: one koji ratuju, poimeničenje članom §~373
\item[ἐπισκοπεῖν] ἐπισκοπέω nadgledati; inf. prez. akt.
\item[ἐν Γέταις] §~100
\item[χαλαζᾶν] χαλαζάω slati tuču; inf. prez. akt. 
\item[ἐν Αἰθίοψιν] §~115
\item[εὐωχεῖσθαι] εὐωχέω gostiti, hraniti; inf. prez. medpas.
\end{description}
