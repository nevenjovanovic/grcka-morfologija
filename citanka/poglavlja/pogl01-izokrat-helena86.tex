%Unesi korekture NZ, NJ 2019-08-09
%\section*{O autoru}



\section*{O tekstu}

Izokrat u \textit{Pohvali} nastoji umanjiti Heleninu krivicu za Trojanski rat. Uspoređujući Helenu s Tezejem, on ističe njegovo junaštvo i prvenstvo pred Heraklom, čime želi ukazati na to da je i žena koja je osvojila srce takvoga junaka također vrijedna pohvale. U ovom odlomku iznosi kako je Tezej Atenu učinio najvećim helenskim gradom i kako je građane potaknuo da se natječu u vrlinama, uvjeren da će sam još uvijek kvalitetama biti iznad njih, dok će mu, s druge strane, biti ljepše da ga slave produhovljeni građani nego oni ropskog duha. Pritom je toliko pazio da ne učini ništa protiv volje građana da im je na kraju  ponudio i da preuzmu vlast u državi, a oni su ipak, umjesto demokracije, izabrali da nad njima vlada Tezej.

%\newpage

\section*{Pročitajte naglas grčki tekst.}

Isoc.\ Helenae encomium 35

%Naslov prema izdanju

\medskip

\begin{greek}
{\large

\noindent καὶ πρῶτον μὲν τὴν πόλιν σποράδην καὶ κατὰ κώμας οἰκοῦσαν εἰς ταὐτὸν συναγαγὼν τηλικαύτην ἐποίησεν ὥστ' ἔτι καὶ νῦν ἀπ' ἐκείνου τοῦ χρόνου μεγίστην τῶν Ἑλληνίδων εἶναι· μετὰ δὲ ταῦτα, κοινὴν τὴν πατρίδα καταστήσας καὶ τὰς ψυχὰς τῶν συμπολιτευομένων ἐλευθερώσας, ἐξ ἴσου τὴν ἅμιλλαν αὐτοῖς περὶ τῆς ἀρετῆς ἐποίησεν, πιστεύων μὲν ὁμοίως αὐτῶν προέξειν ἀσκούντων ὥσπερ ἀμελούντων, εἰδὼς δὲ τὰς τιμὰς ἡδίους οὔσας τὰς παρὰ τῶν μέγα φρονούντων ἢ τὰς παρὰ τῶν δουλευόντων. Τοσούτου δ' ἐδέησεν ἀκόντων τι ποιεῖν τῶν πολιτῶν ὥσθ' ὁ μὲν τὸν δῆμον καθίστη κύριον τῆς πολιτείας, οἱ δὲ μόνον αὐτὸν ἄρχειν ἠξίουν, ἡγούμενοι πιστοτέραν καὶ κοινοτέραν εἶναι τὴν ἐκείνου μοναρχίαν τῆς αὑτῶν δημοκρατίας.


}
\end{greek}

\section*{Analiza i komentar}

%1

{\large
\begin{greek}
\noindent καὶ πρῶτον μὲν τὴν πόλιν \\
\tabto{2em} σποράδην καὶ κατὰ κώμας οἰκοῦσαν \\
εἰς ταὐτὸν συναγαγὼν \\
τηλικαύτην ἐποίησεν\\ 
\tabto{2em} ὥστ' ἔτι καὶ νῦν \\
\tabto{3em} ἀπ' ἐκείνου τοῦ χρόνου \\
\tabto{2em} \underline{μεγίστην} τῶν Ἑλληνίδων \underline{εἶναι}·\\

\end{greek}
}

\begin{description}[noitemsep]
\item[πρῶτον] §~223, §~82, §~203, §~204.2
\item[τὴν πόλιν] §~165
\item[κατὰ κώμας] §~90
\item[οἰκοῦσαν] οἰκέω biti smješten; a. sg.\ ž. r. ptc. prez. akt.
\item[εἰς ταὐτὸν] = εἰς τὸ αὐτὸν; §~207, §~16
\item[συναγαγὼν] συνάγω skupljati; n. sg.\ m. r. ptc. aor. akt.
\item[τηλικαύτην] §~219
\item[ἐποίησεν] ποιέω činiti; 3. l. sg.\ ind. aor. akt. (subjekt je neizrečen, radi se o Tezeju)
\item[ἀπ' ἐκείνου τοῦ χρόνου] ἀπ' = ἀπό; §~82, §~213.3
\item[μεγίστην τῶν Ἑλληνίδων] sc.\ τῶν πόλεων; §~200, §~123; pridjevi s jednim završetkom §~195; dijelni genitiv §~395
\item[εἶναι] εἰμί biti; inf. prez.; imenski predikat s pridjevom kao predikatnim imenom Smyth 910
\item[ὥστ'\dots\ εἶναι] posljedična rečenica konstruirana kao A+I; §~473

\end{description}

{\large
\begin{greek}
\noindent μετὰ δὲ ταῦτα, \\
κοινὴν τὴν πατρίδα \\
\tabto{2em} καταστήσας \\
καὶ τὰς ψυχὰς τῶν συμπολιτευομένων \\
\tabto{2em} ἐλευθερώσας,\\ 
ἐξ ἴσου \\
τὴν ἅμιλλαν \\
αὐτοῖς \\
περὶ τῆς ἀρετῆς \\
ἐποίησεν,\\
\tabto{2em} πιστεύων μὲν \\
\tabto{4em} ὁμοίως αὐτῶν προέξειν \\
\tabto{6em} ἀσκούντων \\
\tabto{6em} ὥσπερ ἀμελούντων,\\
\tabto{2em} εἰδὼς δὲ τὰς τιμὰς ἡδίους οὔσας \\
\tabto{4em} τὰς παρὰ τῶν μέγα φρονούντων \\
\tabto{4em} ἢ τὰς παρὰ τῶν δουλευόντων.\\

\end{greek}
}

\begin{description}[noitemsep]
\item[μετὰ δὲ ταῦτα] §~213
\item[κοινὴν τὴν πατρίδα] §~103, §~123
\item[καταστήσας] καθίστημι uspostaviti; n. sg. m. r. ptc. aor. akt.
\item[τὰς ψυχὰς] §~90
\item[τῶν συμπολιτευομένων] συμπολιτεύω s kime zajedno u državi živjeti; g. pl. m. r. ptc. prez. medpas.; poimeničenje članom §~373; posvojni genitiv §~393
\item[ἐλευθερώσας] ἐλευθερόω oslobađati; n. sg. m. r. ptc. aor. akt.
\item[ἐξ ἴσου] ravnomjerno, pod jednakim uvjetima, nepristrano; §~103
\item[τὴν ἅμιλλαν] §~97
\item[αὐτοῖς] §~207
\item[περὶ τῆς ἀρετῆς] §~90
\item[πιστεύων] πιστεύω vjerovati; n. sg. m. r. ptc. prez. akt.; glagol otvara mjesto dopuni u infinitivu (s vrijednošću izrične rečenice) πιστεύων\dots\ προέξειν vjerujući da će\dots; §~491
\item[ὁμοίως] §~204
\item[προέξειν] προέχω τινός biti nadmoćan nad kime; inf. fut. akt.
\item[ἀσκούντων] ἀσκέω vježbati; g. pl. m. r. ptc. prez. akt.
\item[ἀμελούντων] ἀμελέω zanemarivati; g. pl. m. r. ptc. prez. akt.
\item[εἰδὼς] οἶδα znati; n. sg. m. r. ptc. perf. akt.; u ovoj rečenici glagol otvara mjesto imenici i predikatnom participu u akuzativu (s vrijednošću izrične rečenice) τὰς τιμὰς ἡδίους οὔσας; §~502
\item[τὰς τιμὰς] §~90
\item[ἡδίους] §~200
\item[οὔσας] εἰμί biti; a. pl. ž. r. ptc. prez.; kao kopula, otvara mjesto pridjevu (imenskoj dopuni)
\item[παρὰ τῶν μέγα φρονούντων] μέγα φρονέω biti plemenitih nazora; g. pl. m. r. ptc. prez. akt.
\item[παρὰ τῶν δουλευόντων] δουλεύω robovati; g. pl. m. r. ptc. prez. akt.

\end{description}

%3 itd
{\large
\begin{greek}
\noindent Τοσούτου δ' ἐδέησεν \uuline{ἀκόντων} τι ποιεῖν \uuline{τῶν πολιτῶν} \\
\tabto{2em} ὥσθ' ὁ μὲν τὸν δῆμον \\
\tabto{4em} καθίστη \\
\tabto{6em} κύριον τῆς πολιτείας, \\
\tabto{2em} οἱ δὲ \underline{μόνον αὐτὸν ἄρχειν} ἠξίουν, \\
\tabto{4em} ἡγούμενοι \\
\tabto{6em} \underline{πιστοτέραν καὶ κοινοτέραν εἶναι τὴν} ἐκείνου \underline{μοναρχίαν} \\
\tabto{6em} τῆς αὑτῶν δημοκρατίας.\\

\end{greek}
}

\begin{description}[noitemsep]
\item[Τοσούτου] §~219; konektor najavljuje posljedičnu rečenicu
\item[ἐδέησεν] δέω nedostajati; 3. l. sg. ind. aor. akt.; otvara mjesto infinitivu, odnosno nominativu s infinitivom (nominativ je neizrečen, radi se o Tezeju)
\item[ἀκόντων] §~139
\item[τῶν πολιτῶν] §~100
\item[ἀκόντων τῶν πολιτῶν] GA; §~504
\item[ὥσθ'] veznik ὥστε uvodi posljedičnu rečenicu, a iza τοσούτου δέω s inf. upotrebljava se ὥστε s ind. §~473 (i ondje bilj. 3)
\item[ὁ μὲν\dots\ οἱ δὲ\dots] rečenični članovi iste službe (ovdje subjekti) koordiniraju se česticama, §~370.1, §~519.7
\item[τὸν δῆμον] §~82
\item[καθίστη] καθίστημι τινά τι učiniti koga čime; 3. l. sg. impf. akt.; akuzativ objekta i predikata uz glagole koji znače „činim koga čim'' §~388
\item[κύριον τῆς πολιτείας] §~82, §~90
\item[μόνον αὐτὸν] §~82, §~207
\item[ἄρχειν] ἄρχω vladati; inf. prez. akt.
\item[ἠξίουν] ἀξιόω smatrati primjerenim; 3. l. pl. impf. akt.; glagol otvara mjesto A+I
\item[ἡγούμενοι] ἡγέομαι smatrati; n. pl. m. r. ptc. prez. medpas.; glagol otvara mjesto A+I
\item[πιστοτέραν καὶ κοινοτέραν τὴν μοναρχίαν] §~90, §~103, §~197
\item[ἡγούμενοι πιστοτέραν καὶ κοινοτέραν εἶναι τὴν μοναρχίαν] A+I; §~491
\item[ὥσθ'\dots\ δημοκρατίας] posljedična rečenica §~473
\item[δημοκρατίας] genitiv usporedbe §~404

\end{description}

%kraj
