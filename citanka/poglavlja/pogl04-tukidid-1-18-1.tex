% Redaktura NZ
\section*{O autoru}

Povjesničar Tukidid \textgreek[variant=ancient]{(Θoυκυδίδης,} oko 460. – oko 396.\ pr.~Kr.) kao atenski vojskovođa sudjelovao je u prvom razdoblju Peloponeskoga rata (od 431.) sve do 424., kada je, zbog neuspjeha u obrani važnog savezničkog grada Amfipola, bio optužen za izdaju, te prognan.

Već po izbijanju rata počeo je raditi na \textit{Spisu o ratu Peloponežana i Atenjana} \textgreek[variant=ancient]{(Ξυγγραφὴ περὶ τoῦ πολέμoυ τῶν Пελoπoννησίων καὶ Ἀϑηναίων,} poznat i kao \textit{Povijest Peloponeskog rata}). Konačno ga je uobličio nakon atenskoga konačnog poraza. Djelo ima osam knjiga, ostalo je nedovršeno, a prikazuje događaje do 411.\ pr.~Kr.

\section*{O tekstu}

Na početku prve knjige \textit{Povijesti} Tukidid sažeto prikazuje stanje u Grčkoj od najranijeg doba. Odlomak tog pregleda donesen ovdje govori o Perzijskim ratovima (490.\ – 479.\ pr.~Kr.), tijekom kojih su Sparta i Atena postale najutjecajnije i vojnički najjače grčke države.

\newpage

\section*{Pročitajte naglas grčki tekst.}

Thuc. Historiae 1.18.1

\medskip

{\large
\begin{greek}
\noindent Μετὰ δὲ τὴν τῶν τυράννων κατάλυσιν ἐκ τῆς ῾Ελλάδος οὐ πολλοῖς ἔτεσιν ὕστερον καὶ ἡ ἐν Μαραθῶνι μάχη Μήδων πρὸς ᾿Αθηναίους ἐγένετο. δεκάτῳ δὲ ἔτει μετ' αὐτὴν αὖθις ὁ βάρβαρος τῷ μεγάλῳ στόλῳ ἐπὶ τὴν ῾Ελλάδα δουλωσόμενος ἦλθεν. καὶ μεγάλου κινδύνου ἐπικρεμασθέντος οἵ τε Λακεδαιμόνιοι τῶν ξυμπολεμησάντων ῾Ελλήνων ἡγήσαντο δυνάμει προύχοντες, καὶ οἱ ᾿Αθηναῖοι ἐπιόντων τῶν Μήδων διανοηθέντες ἐκλιπεῖν τὴν πόλιν καὶ ἀνασκευασάμενοι ἐς τὰς ναῦς ἐσβάντες ναυτικοὶ ἐγένοντο. κοινῇ τε ἀπωσάμενοι τὸν βάρβαρον, ὕστερον οὐ πολλῷ διεκρίθησαν πρός τε ᾿Αθηναίους καὶ Λακεδαιμονίους οἵ τε ἀποστάντες βασιλέως ῞Ελληνες καὶ οἱ ξυμπολεμήσαντες. δυνάμει γὰρ ταῦτα μέγιστα διεφάνη· ἴσχυον γὰρ οἱ μὲν κατὰ γῆν, οἱ δὲ ναυσίν. καὶ ὀλίγον μὲν χρόνον ξυνέμεινεν ἡ ὁμαιχμία, ἔπειτα διενεχθέντες οἱ Λακεδαιμόνιοι καὶ ᾿Αθηναῖοι ἐπολέμησαν μετὰ τῶν ξυμμάχων πρὸς ἀλλήλους.
\end{greek}

}

\section*{Analiza i komentar}

%1

{\large
\noindent Μετὰ δὲ τὴν τῶν τυράννων κατάλυσιν \\
\tabto{2em} ἐκ τῆς ῾Ελλάδος \\
οὐ πολλοῖς ἔτεσιν \\
\tabto{2em} ὕστερον \\
καὶ \\
ἡ ἐν Μαραθῶνι μάχη \\
\tabto{2em} Μήδων \\
\tabto{4em} πρὸς ᾿Αθηναίους \\
ἐγένετο.\\

}

\begin{description}[noitemsep]

\item[Μετὰ\dots\ τὴν\dots\ κατάλυσιν] §~430, §~165
\item[δὲ] čestica povezuje rečenicu s prethodnom: a\dots
\item[τῶν τυράννων] §~82
\item[τὴν τῶν τυράννων κατάλυσιν] §~375
\item[ἐκ τῆς ῾Ελλάδος] §~424, §~123
\item[πολλοῖς ἔτεσιν] §~196, §~153; \textit{dativus mensurae et differentiae} kod komparativnih pojmova (ὕστερον) §~414.4
\item[ἡ ἐν Μαραθῶνι μάχη] §~426, §~131, §~90, §~375
\item[Μήδων] §~82
\item[πρὸς ᾿Αθηναίους] §~435, §~103
\item[ἐγένετο] γίγνομαι dogoditi se; 3. l. sg. ind. aor. (med.)
\end{description}

{\large
\noindent δεκάτῳ δὲ ἔτει \\
\tabto{2em} μετ' αὐτὴν \\
αὖθις \\
ὁ βάρβαρος \\
\tabto{2em} τῷ μεγάλῳ στόλῳ \\
\tabto{2em} ἐπὶ τὴν ῾Ελλάδα \\
δουλωσόμενος \\
ἦλθεν.\\

}

\begin{description}[noitemsep]

\item[δεκάτῳ\dots\ ἔτει] §~223, §~153; \textit{dativus temporis} §~415.2
\item[δὲ] čestica povezuje rečenicu s prethodnom: a\dots
\item[μετ' αὐτὴν] §~68; §~430, §~207
\item[ὁ βάρβαρος] §~82
\item[τῷ μεγάλῳ στόλῳ ] §~196, §~82; \textit{dativus instrumenti} §~414.1
\item[ἐπὶ τὴν ῾Ελλάδα ] §~436, §~123
\item[δουλωσόμενος ] δουλόω porobiti, pretvoriti u robove; n. sg. m. ptc. fut. med.; particip izražava namjeru ``da\dots''
\item[ἦλθεν] ἔρχομαι ići; 3. l. sg. ind. aor. (akt.)
\end{description}

%3

{\large
\noindent καὶ \uuline{μεγάλου κινδύνου ἐπικρεμασθέντος} \\
οἵ τε Λακεδαιμόνιοι \\
τῶν ξυμπολεμησάντων ῾Ελλήνων \\
ἡγήσαντο \\
\tabto{2em} δυνάμει προύχοντες, \\
καὶ οἱ ᾿Αθηναῖοι \\
\uuline{ἐπιόντων τῶν Μήδων} \\
διανοηθέντες \\
\tabto{2em} ἐκλιπεῖν τὴν πόλιν \\
καὶ ἀνασκευασάμενοι \\
\tabto{2em} ἐς τὰς ναῦς ἐσβάντες \\
ναυτικοὶ ἐγένοντο. 

}

\begin{description}[noitemsep]

\item[μεγάλου κινδύνου] §~196, §~82
\item[ἐπικρεμασθέντος] ἐπικρεμάννυμι objesiti; pas. nadvijati se, prijetiti; g. sg. m. ptc. aor. pas.
\item[οἵ τε] §~40
\item[οἵ τε Λακεδαιμόνιοι\dots\ καὶ οἱ ᾿Αθηναῖοι] koordinacija surečenica pomoću veznika  τε\dots\  καὶ\dots
\item[Λακεδαιμόνιοι] §~103
\item[ξυμπολεμησάντων] συμπολεμέω (atički ξυμπολεμέω) pridružiti se u ratu; g. pl. m. ptc. aor. akt.
\item[τῶν\dots\ ῾Ελλήνων] §~131
\item[ἡγήσαντο] ἡγέομαι τινός voditi nekoga; 3. l. pl. ind. aor. med.
\item[δυνάμει ] §~165; \textit{dativus instrumenti} §~414.1
\item[προύχοντες] προέχω (stegnuto προὔχω) τινί isticati se nečim; n. pl. m. ptc. prez. akt.
\item[οἱ ᾿Αθηναῖοι ] §~103
\item[ἐπιόντων ] ἔπειμι napasti; g. pl. m. ptc. prez. akt.
\item[τῶν Μήδων ] §~82
\item[διανοηθέντες ] διανοέομαι namjeravati, odlučivati; n. pl. m. ptc. aor. pas.
\item[ἐκλιπεῖν ] ἐκλείπω napustiti; inf. aor. akt.
\item[τὴν πόλιν] §~165
\item[ἀνασκευασάμενοι ] ἀνασκευάζω spakirati; n. pl. m. ptc. aor. med.
\item[ἐς τὰς ναῦς ] §~419, §~180
\item[ἐσβάντες ] εἰσβαίνω ukrcati se; n. pl. m. ptc. aor. akt.
\item[ναυτικοὶ ] §~103
\item[ἐγένοντο] γίγνομαι postati; 3. l. pl. ind. aor. med.; glagol nepotpuna značenja otvara mjesto imenskoj dopuni
\end{description}

%4

{\large
\noindent κοινῇ τε ἀπωσάμενοι \\
\tabto{2em} τὸν βάρβαρον, \\
ὕστερον \\
οὐ πολλῷ διεκρίθησαν \\
\tabto{2em} πρός τε ᾿Αθηναίους καὶ Λακεδαιμονίους \\
οἵ τε ἀποστάντες \\
\tabto{2em} βασιλέως \\
῞Ελληνες \\
καὶ οἱ ξυμπολεμήσαντες. \\

}

\begin{description}[noitemsep]

\item[κοινῇ] zajedno, d. sg. f. u službi priložne oznake 
\item[κοινῇ τε ] §~40; Tukidid osobito često povezuje rečenice sastavnim veznikom τε
\item[ἀπωσάμενοι ] ἀπωθέω odbiti, otjerati; n. pl. m. ptc. aor. med.
\item[τὸν βάρβαρον] §~82
\item[πολλῷ] mnogo, d. sg. n. u službi priložne oznake
\item[διεκρίθησαν ] διακρίνω razdvojiti; 3. l. pl. ind. aor. pas.
\item[πρός τε ᾿Αθηναίους καὶ Λακεδαιμονίους] §~40, §~435, §~103; koordinacija parom sastavnih veznika τε\dots\ καὶ: niti\dots\ niti\dots (jer je rečenica niječna)
\item[οἵ τε ἀποστάντες ] §~40, §~373, ἀφίστημι pobuniti se protiv koga, odmetnuti se od koga, rekcija τινός; n. pl. m. ptc. aor. akt.
\item[οἵ τε ἀποστάντες\dots\ καὶ οἱ ξυμπολεμήσαντες] koordinacija parom sastavnih veznika τε\dots\ καὶ\dots
\item[βασιλέως ] §~175
\item[῞Ελληνες] §~131
\item[οἱ ξυμπολεμήσαντες] §~373, συμπολεμέω (atički ξυμπολεμέω) τινί (sc.\ βασιλεῖ) pridružiti se nekome u ratu; n. pl. m. ptc. aor. akt.
\end{description}

%5

{\large
\noindent δυνάμει γὰρ \\
ταῦτα \\
μέγιστα διεφάνη· \\
ἴσχυον γὰρ \\
οἱ μὲν \\
\tabto{2em} κατὰ γῆν, \\
οἱ δὲ \\
\tabto{2em} ναυσίν. \\

}

\begin{description}[noitemsep]

\item[δυνάμει ] §~165
\item[γὰρ ] čestica najavljuje iznošenje dokaza prethodne tvrdnje: naime\dots
\item[ταῦτα ] §~213.2, sc.\ \textgreek[variant=ancient]{οἵ τε Λακεδαιμόνιοι καὶ οἱ ᾿Αθηναῖοι}
\item[μέγιστα ] §~200
\item[διεφάνη] διαφαίνω pokazati kroz nešto, pas. rekcija τινι vidjeti se u nečemu; 3. l. sg. ind. aor. pas.
\item[ἴσχυον ] ἰσχύω biti snažan, imati nadmoć; 3. l. pl. impf. akt.
\item[γὰρ ] čestica najavljuje iznošenje dokaza prethodne tvrdnje: naime\dots
\item[οἱ μὲν\dots\ οἱ δὲ\dots] koordinacija: jedni\dots\ drugi\dots
\item[κατὰ γῆν] §~429, §~107-108
\item[ναυσίν] §~180
\end{description}

%6

{\large
\noindent καὶ ὀλίγον μὲν χρόνον \\
ξυνέμεινεν \\
ἡ ὁμαιχμία, \\
ἔπειτα \\
διενεχθέντες \\
οἱ Λακεδαιμόνιοι καὶ ᾿Αθηναῖοι \\
ἐπολέμησαν \\
\tabto{2em} μετὰ τῶν ξυμμάχων \\
\tabto{2em} πρὸς ἀλλήλους.\\

}

\begin{description}[noitemsep]

\item[ὀλίγον\dots\ χρόνον ] §~103, §~82
\item[μὲν\dots\ ἔπειτα\dots] koordinacija surečenica pomoću čestice i veznika
\item[ξυνέμεινεν ] συμμένω držati zajedno, vrijediti; 3. l. sg. ind. aor. akt.
\item[ἡ ὁμαιχμία] §~90
\item[διενεχθέντες ] διαφέρω medpas. svađati se; n. pl. m. ptc. aor. pas.
\item[οἱ Λακεδαιμόνιοι] §~103; §~373
\item[᾿Αθηναῖοι ] §~103
\item[ἐπολέμησαν ] πολεμέω ratovati; 3. l. pl. ind. aor. akt.
\item[μετὰ τῶν ξυμμάχων ] §~430, §~103, §~106; §~373
\item[πρὸς ἀλλήλους] §~435, §~212
\end{description}

%kraj
