% Unesi korekture NZ, NJ 2019-08-09
%\section*{O autoru}



\section*{O tekstu}

Izabrani odlomak iz šeste knjige Tukididova \textit{Spisa o ratu Peloponežana i Atenjana} \textgreek[variant=ancient]{(Ξυγγραφὴ περὶ τoῦ πολέμoυ τῶν Пελoπoννησίων καὶ Ἀϑηναίων)} govori o događanjima s početka 415.\ pr.~Kr., kad je sicilski grad Segesta \textgreek[variant=ancient]{(Ἔγεστα)} Atenjanima poslao zahtjev i mjesečnu novčanu naknadu za slanje šezdeset ratnih brodova da ih podupru u ratnim sukobima sa Selinuntom \textgreek[variant=ancient]{(Σελινοῦς).} Segešćani lažno predstavljaju svoju financijsku moć, no Atenjani odlučuju zahtjevu udovoljiti, a usput, ako im ratne prilike dopuste, pomoći i žiteljima sicilskog grada Leontina \textgreek[variant=ancient]{(Λεοντῖνοι).} To je bio početak znamenite i za Atenu pogubne Sicilske ekspedicije (415.\ – 413).

%\newpage

\section*{Pročitajte naglas grčki tekst.}

%Naslov prema izdanju

Thuc.\ Historiae 6.8.1

\medskip

{\large
\begin{greek}
\noindent Τοῦ δ' ἐπιγιγνομένου θέρους ἅμα ἦρι οἱ τῶν Ἀθηναίων πρέσβεις ἧκον ἐκ τῆς Σικελίας καὶ οἱ Ἐγεσταῖοι μετ' αὐτῶν ἄγοντες ἑξήκοντα τάλαντα ἀσήμου ἀργυρίου ὡς ἐς ἑξήκοντα ναῦς μηνὸς μισθόν, ἃς ἔμελλον δεήσεσθαι πέμπειν. καὶ οἱ Ἀθηναῖοι ἐκκλησίαν ποιήσαντες καὶ ἀκούσαντες τῶν τε Ἐγεσταίων καὶ τῶν σφετέρων πρέσβεων τά τε ἄλλα ἐπαγωγὰ καὶ οὐκ ἀληθῆ καὶ περὶ τῶν χρημάτων ὡς εἴη ἑτοῖμα ἔν τε τοῖς ἱεροῖς πολλὰ καὶ ἐν τῷ κοινῷ, ἐψηφίσαντο ναῦς ἑξήκοντα πέμπειν ἐς Σικελίαν καὶ στρατηγοὺς αὐτοκράτορας Ἀλκιβιάδην τε τὸν Κλεινίου καὶ Νικίαν τὸν Νικηράτου καὶ Λάμαχον τὸν Ξενοφάνους, βοηθοὺς μὲν Ἐγεσταίοις πρὸς Σελινουντίους, ξυγκατοικίσαι δὲ καὶ Λεοντίνους, ἤν τι περιγίγνηται αὐτοῖς τοῦ πολέμου, καὶ τἆλλα τὰ ἐν τῇ Σικελίᾳ πρᾶξαι ὅπῃ ἂν γιγνώσκωσιν ἄριστα Ἀθηναίοις.


\end{greek}

}


\section*{Analiza i komentar}



%1

{\large
\noindent \uuline{Τοῦ δ' ἐπιγιγνομένου θέρους} \\
ἅμα ἦρι \\
οἱ τῶν ᾿Αθηναίων πρέσβεις \\
ἧκον \\
\tabto{2em} ἐκ τῆς Σικελίας \\
καὶ οἱ ᾿Εγεσταῖοι \\
\tabto{2em} μετ' αὐτῶν\\
\tabto{2em} ἄγοντες ἑξήκοντα τάλαντα ἀσήμου ἀργυρίου\\
\tabto{4em} ὡς ἐς ἑξήκοντα ναῦς μηνὸς μισθόν,\\
\tabto{6em} ἃς ἔμελλον δεήσεσθαι πέμπειν.\\

}

\begin{description}[noitemsep]
\item[δ'] = δέ
\item[ἐπιγιγνομένου] ἐπιγίγνομαι nadolaziti; g. sg.\ s. r. ptc. prez. medpas.
\item[θέρους] §~153
\item[ἦρι] §~146
\item[Τοῦ δ' ἐπιγιγνομένου θέρους ἅμα ἦρι] u proljeće sljedeće sezone vojnih pohoda; GA; §~408
\item[οἱ τῶν ᾿Αθηναίων πρέσβεις] §~82; 165
\item[ἧκον] ἥκω doći; 3. l. pl.\ impf. akt.
\item[ἐκ τῆς Σικελίας] §~90
\item[οἱ ᾿Εγεσταῖοι ] §~82
\item[μετ' αὐτῶν] = μετὰ αὐτῶν; §~207
\item[ἄγοντες] ἄγω voditi; n. pl.\ m. r. ptc. prez. akt.; ovisno o ἧκον; §~500
\item[ἑξήκοντα τάλαντα] §~223; §~82
\item[ἀσήμου ἀργυρίου] §~82
\item[ἐς ἑξήκοντα ναῦς] §~182
\item[ὡς μηνὸς μισθόν] kao mjesečnu plaću; §~82
\item[ἃς] §~215; otvara mjesto odnosnoj rečenici, antecedent je ναῦς
\item[ἔμελλον] μέλλω namjeravati; 3. l. pl.\ impf. akt.
\item[δεήσεσθαι] δέω tražiti; inf. fut. med.
\item[πέμπειν] πέμπω slati; inf. prez. akt.; §~493.2

\end{description}

%\newpage


{\large
\noindent καὶ οἱ ᾿Αθηναῖοι \\
\tabto{2em} ἐκκλησίαν ποιήσαντες \\
\tabto{2em} καὶ ἀκούσαντες \\
\tabto{4em} τῶν τε ᾿Εγεσταίων καὶ τῶν σφετέρων πρέσβεων\\
\tabto{2em} τά τε ἄλλα ἐπαγωγὰ \\
\tabto{2em} καὶ οὐκ ἀληθῆ \\
\tabto{2em} καὶ περὶ τῶν χρημάτων\\
\tabto{4em}  ὡς εἴη ἑτοῖμα \\
\tabto{6em}  ἔν τε τοῖς ἱεροῖς πολλὰ \\
\tabto{6em} καὶ ἐν τῷ κοινῷ,\\
ἐψηφίσαντο

\tabto{2em} ναῦς ἑξήκοντα πέμπειν \\
\tabto{4em} ἐς Σικελίαν \\
\tabto{2em} καὶ στρατηγοὺς αὐτοκράτορας \\
\tabto{4em} ᾿Αλκιβιάδην τε τὸν Κλεινίου \\
\tabto{4em} καὶ Νικίαν τὸν Νικηράτου \\
\tabto{4em} καὶ Λάμαχον τὸν Ξενοφάνους, \\
\tabto{2em} βοηθοὺς μὲν ᾿Εγεσταίοις πρὸς Σελινουντίους,
\tabto{2em} ξυγκατοικίσαι δὲ καὶ Λεοντίνους, \\
\tabto{4em} ἤν τι περιγίγνηται αὐτοῖς \\
\tabto{6em} τοῦ πολέμου, \\
\tabto{3em} καὶ τἆλλα \\
\tabto{6em} τὰ ἐν τῇ Σικελίᾳ \\
\tabto{2em} πρᾶξαι \\
\tabto{4em} ὅπῃ ἂν γιγνώσκωσιν \\
\tabto{6em} ἄριστα ᾿Αθηναίοις.\\

}

\begin{description}[noitemsep]
\item[ἐκκλησίαν] §~82
\item[ποιήσαντες] ποιέω činiti; n. pl.\ m. r. ptc. aor. akt.
\item[ἀκούσαντες] ἀκούω τί τινος slušati što od koga; n. pl.\ m. r. ptc. aor. akt.
\item[τῶν τε ᾿Εγεσταίων καὶ τῶν σφετέρων πρέσβεων ] τε\dots\ καὶ ne samo\dots\ nego i; §~82; §~210,3; §~165; §~513
\item[τά τε ἄλλα ἐπαγωγὰ καὶ οὐκ ἀληθῆ] §~212; §~103; §~153; §~513
\item[περὶ τῶν χρημάτων] §~123
\item[εἴη] εἰμί biti; 3. l. sg.\ opt. prez.; glagol u jednini uz imenicu s. r. u množini (§~361)
\item[ἑτοῖμα] §~82
\item[ἔν τε τοῖς ἱεροῖς] §~82
\item[πολλὰ] §~196
\item[ἐν τῷ κοινῷ] u državnoj riznici; §~82
\item[ὡς\dots\  κοινῷ] zavisna izrična rečenica; §~467
\item[ἐψηφίσαντο] ψηφίζω glasovati; 3. l. pl.\ ind. aor. med.
\item[στρατηγοὺς αὐτοκράτορας] §~82; §~146
\item[Αλκιβιάδην] \textbf{τε τὸν Κλεινίου καὶ Νικίαν τὸν Νικηράτου} §~100; §~82
\item[Λάμαχον τὸν Ξενοφάνους] §~82; §~153
\item[βοηθοὺς μὲν\dots\ ξυγκατοικίσαι δὲ\dots] koordinacija rečeničnih članova s pomoću para suprotnih čestica
\item[βοηθοὺς μὲν ᾿Εγεσταίοις πρὸς Σελινουντίους] §~82
\item[ξυγκατοικίσαι] συγκατοικίζω skupa obnoviti naseobinu; inf. aor. akt.
\item[Λεοντίνους] §~82
\item[ἤν] = ἐάν 
\item[τι] otvara mjesto τοῦ πολέμου §~217
\item[περιγίγνηται] περιγίγνομαι nadvisivati, περιγίγνηταί τί τινι tko u čemu ostvaruje prednost; 3. l. sg.\ konj. prez. medpas.
\item[αὐτοῖς] §~207
\item[τοῦ πολέμου] §~82; §~395
\item[ἤν\dots\ πολέμου] protaza eventualne pogodbene rečenice; §~476
\item[τἆλλα] = τὰ ἄλλα; §~16; §~66
\item[πρᾶξαι ] πράττω (πράσσω) činiti; inf. aor. akt.
\item[ὅπῃ] kako; §~221
\item[γιγνώσκωσιν] γιγνώσκω prepoznati, prosuditi; 3. l. pl.\ konj. prez. akt.
\item[ἄριστα] §~202
\item[ὅπῃ\dots\ Αθηναίοις] eventualna odnosna rečenica; §~486,2
\end{description}

%kraj
