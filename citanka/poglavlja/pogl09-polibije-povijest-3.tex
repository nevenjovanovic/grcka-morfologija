% korektura NZ, NJ
\section*{O autoru}

Polibije \textgreek[variant=ancient]{(Πολύβιος)} rođen je u Megalopolu u Arkadiji 200.\ pr.~Kr. Pripadao je utjecajnoj obitelji i kao mladić se borio pod zapovjedništvom vojskovođe Filopemena. U dobi od 30 godina bio je jedan od vođa Ahejskog saveza, udruženja grčkih polisa koji su za vrijeme sukoba između Rima i Makedonije pokušavali ostati neutralni, no i ta neutralnost je imala svoju cijenu. Nakon bitke kod Pidne (168.\ pr.~Kr), u kojoj su Rimljani porazili makedonskog kralja Perzeja, Polibije je stigao u Rim kao jedan od tisuću ahejskih talaca. Zbližio se s krugom oko Emilija Paula i Publija Kornelija Scipiona Mlađeg. Potonjeg je pratio u Trećem punskom ratu (149.\ – 146.\ pr.~Kr) gdje je njegovo prethodno ratno iskustvo bilo od velike koristi (osmislio je sustav šifriranih signala). Kad su Rimljani konačno pokorili i Ahejski savez, Polibije se trudio isposlovati prihvatljive uvjete predaje za svoje sunarodnjake. U znak zahvalnosti mu je šest polisa podiglo spomenik. Umro je 118.\ pr.~Kr.

\section*{O tekstu}

U djelu \textit{Povijest} \textgreek[variant=ancient]{(Ἱστορίαι)} Polibije prikazuje vojni i politički uspon Rima. Zahvaljujući prijateljstvu s brojnim rimskim uglednicima, autor je imao pristup državnim dokumentima i arhivima. Za potrebe istraživanja razgovarao je i s velikim brojem sudionika u suvremenim povijesnim zbivanjima. Naposljetku, svojim očima je vidio kako su Rimljani Kartagu sravnili sa zemljom, te je \textit{Povijest} dijelom rezultat autopsije. Djelo se sastojalo od 40 knjiga od kojih je samo prvih pet sačuvano u cijelosti. Slijedi odlomak iz treće knjige u kojem se Hanibal obraća vojnicima prije bitke kod Kane u Apuliji, u jugoistočnoj Italiji (216.\ pr.~Kr); u ovoj će velikoj bici Hanibalove snage pobijediti nadmoćnu vojsku Rimske Republike.

%\newpage

\section*{Pročitajte naglas grčki tekst.}
Polyb.\ Historiae 3.111.3
%Naslov prema izdanju

\medskip

{\large
\begin{greek}
\noindent Πρῶτον μὲν τοῖς θεοῖς ἔχετε χάριν· ἐκεῖνοι γὰρ ἡμῖν συγκατασκευάζοντες τὴν νίκην εἰς τοιούτους τόπους ἤχασι τοὺς ἐχθρούς· δεύτερον δ' ἡμῖν, ὅτι καὶ μάχεσθαι τοὺς πολεμίους συνηναγκάσαμεν· οὐ γὰρ ἔτι δύνανται τοῦτο διαφυγεῖν· καὶ μάχεσθαι προφανῶς ἐν τοῖς ἡμετέροις προτερήμασι. τὸ δὲ παρακαλεῖν ὑμᾶς νῦν διὰ πλειόνων εὐθαρσεῖς καὶ προθύμους εἶναι πρὸς τὸν κίνδυνον οὐδαμῶς μοι δοκεῖ καθήκειν. ὅτε μὲν γὰρ ἀπείρως διέκεισθε τῆς πρὸς ῾Ρωμαίους μάχης, ἔδει τοῦτο ποιεῖν, καὶ μεθ' ὑποδειγμάτων ἐγὼ πρὸς ὑμᾶς πολλοὺς διεθέμην λόγους· ὅτε δὲ κατὰ τὸ συνεχὲς τρισὶ μάχαις τηλικαύταις ἐξ ὁμολογουμένου νενικήκατε ῾Ρωμαίους, ποῖος ἂν ἔτι λόγος ὑμῖν ἰσχυρότερον παραστήσαι θάρσος αὐτῶν τῶν ἔργων; διὰ μὲν οὖν τῶν πρὸ τοῦ κινδύνων κεκρατήκατε τῆς χώρας καὶ τῶν ἐκ ταύτης ἀγαθῶν κατὰ τὰς ἡμετέρας ἐπαγγελίας, ἀψευστούντων ἡμῶν ἐν πᾶσι τοῖς πρὸς ὑμᾶς εἰρημένοις· ὁ δὲ νῦν ἀγὼν ἐνέστηκεν περὶ τῶν πόλεων καὶ τῶν ἐν αὐταῖς ἀγαθῶν. οὗ κρατήσαντες κύριοι μὲν ἔσεσθε παραχρῆμα πάσης ᾿Ιταλίας, ἀπαλλαγέντες δὲ τῶν νῦν πόνων, γενόμενοι συμπάσης ἐγκρατεῖς τῆς ῾Ρωμαίων εὐδαιμονίας, ἡγεμόνες ἅμα καὶ δεσπόται πάντων γενήσεσθε διὰ ταύτης τῆς μάχης. διόπερ οὐκέτι λόγων ἀλλ' ἔργων ἐστὶν ἡ χρεία· θεῶν γὰρ βουλομένων ὅσον οὔπω βεβαιώσειν ὑμῖν πέπεισμαι τὰς ἐπαγγελίας.

\end{greek}
}

\newpage

\section*{Analiza i komentar}

%1

{\large
\begin{greek}
\noindent Πρῶτον μὲν \\
\tabto{2em} τοῖς θεοῖς ἔχετε χάριν·\\
\tabto{4em} ἐκεῖνοι γὰρ \\
\tabto{6em} ἡμῖν συγκατασκευάζοντες τὴν νίκην \\
\tabto{4em} εἰς τοιούτους τόπους \\
\tabto{4em} ἤχασι τοὺς ἐχθρούς·\\
δεύτερον δ' ἡμῖν, \\
\tabto{2em} ὅτι καὶ μάχεσθαι τοὺς πολεμίους συνηναγκάσαμεν·\\
\tabto{4em} οὐ γὰρ ἔτι δύνανται \\
\tabto{6em} τοῦτο διαφυγεῖν·\\
\tabto{6em} καὶ μάχεσθαι προφανῶς \\
\tabto{8em} ἐν τοῖς ἡμετέροις προτερήμασι.\\

\end{greek}
}

\begin{description}[noitemsep]
\item[Πρῶτον] priložno 
\item[τοῖς θεοῖς ] §~82
\item[ἔχετε ] ἔχω imati; 2. l. pl. imp. prez. akt.
\item[χάριν] §~129
\item[ἐκεῖνοι ] §~213.3
\item[ἡμῖν ] §~205
\item[συγκατασκευάζοντες ] συγκατασκευάζω pomoći u pripremi; n. pl. m. r. ptc. prez. akt.
\item[τὴν νίκην] §~90
\item[εἰς τοιούτους τόπους ] §~219, §~82, §~419
\item[ἤχασι] ἄγω voditi; 3. l. pl. ind. perf. akt. 
\item[τοὺς ἐχθρούς] §~82
\item[δεύτερον] priložno
\item[ἡμῖν] §~205
\item[ὅτι] §~518
\item[μάχεσθαι] μάχομαι boriti se; inf. prez. medpas. 
\item[τοὺς πολεμίους ] §~82
\item[συνηναγκάσαμεν] συναναγκάζω pomoći da se (netko) prisili; 1. l. pl. ind. aor. akt. 
\item[δύνανται ] δύναμαι moći; 3. l. pl. ind. perf. medpas; §~312
\item[τοῦτο ] §~213.2
\item[διαφυγεῖν ] διαφεύγω izbjeći; inf. aor. akt. 
\item[μάχεσθαι ] μάχομαι boriti se; inf. prez. medpas. 
\item[προφανῶς ] prilog od pridjeva προφανής
\item[ἐν τοῖς ἡμετέροις προτερήμασι ] §~426, §~210, §~123

\end{description}

{\large
\begin{greek}
\noindent τὸ δὲ παρακαλεῖν ὑμᾶς \\
\tabto{2em} νῦν \\
\tabto{2em} διὰ πλειόνων εὐθαρσεῖς \\
καὶ προθύμους εἶναι \\
\tabto{2em} πρὸς τὸν κίνδυνον \\
οὐδαμῶς μοι δοκεῖ \\
\tabto{6em} καθήκειν.\\

\end{greek}
}

\begin{description}[noitemsep]
\item[τὸ παρακαλεῖν] παρακαλέω pozivati, poticati; inf. prez. akt; supstantiviranje infinitiva članom §~497
\item[ὑμᾶς ] §~205
\item[διὰ πλειόνων ] §~428, §~202
\item[εὐθαρσεῖς] §~153
\item[προθύμους ] §~82
\item[εἶναι ] εἰμί biti; inf. prez. (kopula u imenskom predikatu)
\item[πρὸς τὸν κίνδυνον ] §~435, §~82
\item[μοι ] §~205
\item[δοκεῖ] δοκέω činiti se; 3. l. sg. ind. prez. akt. 
\item[καθήκειν] καθήκω priličiti, biti primjereno; inf. prez. akt. 

\end{description}
%3 itd

{\large
\begin{greek}
\noindent ὅτε μὲν γὰρ ἀπείρως διέκεισθε \\
\tabto{2em} τῆς πρὸς ῾Ρωμαίους μάχης, \\
ἔδει τοῦτο ποιεῖν, \\
\tabto{2em} καὶ μεθ' ὑποδειγμάτων \\
\tabto{4em} ἐγὼ πρὸς ὑμᾶς \\
\tabto{6em} πολλοὺς διεθέμην λόγους·\\ 
ὅτε δὲ κατὰ τὸ συνεχὲς \\
\tabto{2em} τρισὶ μάχαις τηλικαύταις \\
\tabto{4em} ἐξ ὁμολογουμένου \\
\tabto{6em} νενικήκατε ῾Ρωμαίους,\\ 
ποῖος ἂν ἔτι λόγος \\
\tabto{2em} ὑμῖν ἰσχυρότερον παραστήσαι θάρσος \\
\tabto{4em} αὐτῶν τῶν ἔργων;\\

\end{greek}
}

\begin{description}[noitemsep]
\item[ὅτε] §~487
\item[ἀπείρως ] §~204
\item[διέκεισθε ] διάκειμαι biti u (nekom) stanju (to je stanje ovdje izraženo prilogom); 2. l. pl. impf. (medpas.) 
\item[τῆς πρὸς ῾Ρωμαίους μάχης] §~103, §~90, §~435
\item[ἔδει ] δεῖ treba (bezličan glagol); 3. l. sg. impf. akt.
\item[τοῦτο ] §~213.2
\item[ποιεῖν] ποιέω činiti; inf. prez. akt. 
\item[μεθ' ὑποδειγμάτων ] §~430, §~123
\item[ἐγὼ ] §~205
\item[πρὸς ὑμᾶς ] §~435, §~205
\item[πολλοὺς ] §~196
\item[διεθέμην ] διατίθημι uputiti (riječi); 1. l. sg. ind. aor. med. 
\item[λόγους ] §~82
\item[ὅτε] §~487
\item[κατὰ τὸ συνεχὲς ] §~429, §~153; supstantiviranje članom §~373
\item[τρισὶ ] §~224
\item[μάχαις ] §~90
\item[τηλικαύταις ] §~219
\item[ἐξ ὁμολογουμένου] ὁμολογέω suglasiti se; ptc. prez. medpas. g. sg. s. r.
\item[νενικήκατε ] νικάω pobijediti; 2. l. pl. ind. perf. akt.
\item[῾Ρωμαίους] §~82
\item[ποῖος ] §~219
\item[ἂν ] §~489, uvodi optativ παραστήσαι; ἄν + optativ izriče mogućnost (potencijal sadašnji)
\item[λόγος ] §~82
\item[ὑμῖν ] §~205
\item[ἰσχυρότερον ] §~197
\item[παραστήσαι ] παρίστημι pribaviti, pobuditi; 3. l sg. opt. aor. akt. 
\item[θάρσος ] §~153
\item[αὐτῶν τῶν ἔργων ] §~207, §~82; genitivus comparationis §~404

\end{description}

%4

{\large
\begin{greek}
\noindent διὰ μὲν οὖν τῶν πρὸ τοῦ κινδύνων \\
\tabto{2em} κεκρατήκατε \\
\tabto{4em} τῆς χώρας \\
\tabto{4em} καὶ τῶν ἐκ ταύτης ἀγαθῶν \\
\tabto{6em} κατὰ τὰς ἡμετέρας ἐπαγγελίας, \\
\uuline{ἀψευστούντων ἡμῶν} \\
\tabto{2em} ἐν πᾶσι τοῖς πρὸς ὑμᾶς εἰρημένοις·\\ 
ὁ δὲ νῦν ἀγὼν ἐνέστηκεν \\
\tabto{2em} περὶ τῶν πόλεων \\
\tabto{4em} καὶ τῶν ἐν αὐταῖς ἀγαθῶν.\\

\end{greek}
}

\begin{description}[noitemsep]
\item[διὰ τῶν\dots\ κινδύνων ] §~428, §~82
\item[πρὸ τοῦ] §~370.3, §~425; prijedložni izraz kao atribut §~375
\item[κεκρατήκατε ] κρατέω τινός ovladati nečim, osvojiti nešto; 2. l. pl. ind. perf. akt. 
\item[τῆς χώρας ] §~90
\item[τῶν ἐκ ταύτης ἀγαθῶν ] §~103, §~424, §~213.2
\item[κατὰ τὰς ἡμετέρας ἐπαγγελίας]  §~429, §~210, §~90
\item[ἀψευστούντων ] ἀψευστέω govoriti istinu (ne lagati); g. pl. m. r. ptc. prez. akt. 
\item[ἡμῶν ] §~205
\item[ἐν πᾶσι ] §~193
\item[τοῖς εἰρημένοις] λέγω govoriti; d. pl. s. r. ptc. perf. med; §~103, §~327.7
\item[πρὸς ὑμᾶς] §~435, §~205
\item[ὁ νῦν ἀγὼν] §~131; prilog kao atribut §~375, νῦν prevedite pridjevom: sadašnji
\item[ἐνέστηκεν ] ἐνίστημι prijetiti, voditi se (o borbi); 3. l. sg. ind. perf. akt. 
\item[περὶ τῶν πόλεων] §~433, §~165
\item[τῶν\dots\ ἀγαθῶν] §~103
\item[ἐν αὐταῖς ] §~207; prijedložni izraz kao atribut §~375

\end{description}

%5

{\large
\begin{greek}
\noindent οὗ κρατήσαντες \\
κύριοι μὲν ἔσεσθε παραχρῆμα \\
\tabto{2em} πάσης ᾿Ιταλίας, \\
ἀπαλλαγέντες δὲ \\
\tabto{2em} τῶν νῦν πόνων, \\
γενόμενοι \\
\tabto{2em} συμπάσης ἐγκρατεῖς τῆς ῾Ρωμαίων εὐδαιμονίας, \\
ἡγεμόνες ἅμα καὶ δεσπόται πάντων γενήσεσθε \\
\tabto{2em} διὰ ταύτης τῆς μάχης.\\

\end{greek}
}

\begin{description}[noitemsep]
\item[οὗ ] §~215, zamjenica uvodi relativnu rečenicu pogodbenog smisla, antecedent je ὁ\dots\ νῦν ἀγὼν iz prethodne rečenice
\item[κρατήσαντες ] κρατέω τινός ovladati nečim, pobijediti u nečem; n. pl. m. r. ptc. aor. akt.
\item[κύριοι ] §~82
\item[ἔσεσθε ] εἰμί biti; 2. l. pl. ind. fut; (kopula, dio imenskog predikata) 
\item[πάσης ᾿Ιταλίας ] §~193, §~90
\item[ἀπαλλαγέντες ] ἀπαλλάσσω udaljiti, pas. osloboditi se, riješiti se; ptc. aor. pas. n. pl. m. r. 
\item[τῶν νῦν πόνων] §~82; prilog kao atribut §~375, νῦν prevedite pridjevom: sadašnji
\item[γενόμενοι ] γίγνομαι postati ; n. pl. m. r. ptc. aor. (med.); kopulativan glagol otvara mjesto imenskoj dopuni
\item[συμπάσης] §~193, §~379 
\item[ἐγκρατεῖς ] §~153
\item[τῆς ῾Ρωμαίων εὐδαιμονίας] §~90, §~103
\item[ἡγεμόνες ] §~131
\item[δεσπόται ] §~100, §~101
\item[πάντων ] §~193
\item[γενήσεσθε ] γίγνομαι postati; 2. l. pl. ind. fut. (med.);  kopulativan glagol otvara mjesto imenskoj dopuni
\item[διὰ ταύτης τῆς μάχης ] §~428, §~90

\end{description}
%6

{\large
\begin{greek}
\noindent διόπερ \\
οὐκέτι λόγων \\
\tabto{2em} ἀλλ' ἔργων \\
\tabto{4em} ἐστὶν ἡ χρεία·\\
\uuline{θεῶν} γὰρ \uuline{βουλομένων} \\
ὅσον οὔπω βεβαιώσειν ὑμῖν πέπεισμαι τὰς ἐπαγγελίας.\\

\end{greek}
}

\begin{description}[noitemsep]
\item[λόγων] §~82 
\item[ἔργων ] §~82
\item[ἐστὶν ] εἰμί biti; 3. l. sg. ind. prez. (kopula, dio imenskog predikata)
\item[ἡ χρεία ] §~90
\item[θεῶν ] §~82
\item[βουλομένων ] βούλομαι htjeti; ptc. prez. medpas. g. pl. m. r. 
\item[ὅσον ] §~219; ὅσον οὔπω: gotovo, odmah, smjesta
\item[βεβαιώσειν ] βεβαιόω učvrstiti; inf. fut. akt; subjekt infinitiva se ne izriče jer je sadržan u finitnom obliku πέπεισμαι
\item[ὑμῖν ] §~205
\item[πέπεισμαι] πείθω nagovoriti, uvjeriti; 1. l. sg. ind. perf. medpas. 
\item[τὰς ἐπαγγελίας] §~90

\end{description}


%kraj
