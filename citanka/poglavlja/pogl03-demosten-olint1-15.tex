% Unio korekture NZ, NJ 2019-08-08
%\section*{O autoru}



\section*{O tekstu}

Demosten je tijekom 349.~pr.~Kr.\ sastavio tri politička govora motivirana napadom makedonskoga kralja Filipa II. (Aleksandrova oca) na grad Olint, na poluotoku Halkidici. Kako je Olint u to vrijeme bio saveznik Atene, Demosten je u olintskim govorima poticao atenske političare da pomognu tom grčkom polisu.

U nekoliko navrata Olinćani su slali poklisare u Atenu, moleći za vojnu pomoć. No, Atenjani nisu bili voljni poduzeti vojni pohod jer je Olint bio predaleko. U prvom \textit{Olintskom govoru}, napisanom povodom dolaska prvog olintskog poslanstva, Demosten poziva Atenjane da po hitnom postupku izglasaju slanje vojske u Olint. Demosten u ovom izvatku upozorava da ne suprotstaviti se Filipu znači smrtnu opasnost.

%\newpage

\section*{Pročitajte naglas grčki tekst.}

Dem.\ Olynthiaca I 15
%Naslov prema izdanju

\medskip

{\large
\begin{greek}
\noindent Πρὸς θεῶν, τίς οὕτως εὐήθης ἐστὶν ὑμῶν ὅστις ἀγνοεῖ τὸν ἐκεῖθεν πόλεμον δεῦρ'  ἥξοντα, ἂν ἀμελήσωμεν; ἀλλὰ μήν, εἰ τοῦτο γενήσεται, δέδοικ', ὦ ἄνδρες ᾿Αθηναῖοι, μὴ τὸν αὐτὸν τρόπον ὥσπερ οἱ δανειζόμενοι ῥᾳδίως ἐπὶ τοῖς μεγάλοις [τόκοις] μικρὸν εὐπορήσαντες χρόνον ὕστερον καὶ τῶν ἀρχαίων ἀπέστησαν, οὕτω καὶ ἡμεῖς [ἂν] ἐπὶ πολλῷ φανῶμεν ἐρρᾳθυμηκότες, καὶ ἅπαντα πρὸς ἡδονὴν ζητοῦντες πολλὰ καὶ χαλεπὰ ὧν οὐκ ἐβουλόμεθ' ὕστερον εἰς ἀνάγκην ἔλθωμεν ποιεῖν, καὶ κινδυνεύσωμεν περὶ τῶν ἐν αὐτῇ τῇ χώρᾳ.

\end{greek}
}

\section*{Analiza i komentar}

%1

{\large
\begin{greek}
\noindent Πρὸς θεῶν,\\
τίς οὕτως εὐήθης ἐστὶν ὑμῶν \\
\tabto{2em} ὅστις ἀγνοεῖ \\
\tabto{3em} τὸν ἐκεῖθεν πόλεμον δεῦρ' ἥξοντα,\\
\tabto{3em} ἂν ἀμελήσωμεν;\\

\end{greek}
}

\begin{description}[noitemsep]
\item[Πρὸς θεῶν] uobičajena formula općenitog zaklinjanja: za ime božje; prijedložni izraz πρὸς + g. §~418, §~435.A
\item[τίς ] §~217, §~218.1-2
\item[οὕτως ] §~213.2; prilog izveden iz pokazne zamjenice
\item[εὐήθης ] pridjev kao imenski dio imenskoga predikata (εὐήθης ἐστὶν)
\item[ἐστὶν ] εἰμί biti; 3.l.sg. ind.\ prez.\ akt.; kopula kao dio imenskoga predikata (εὐήθης ἐστὶν)
\item[ὑμῶν ] §~205
\item[ὅστις ] §~217, §~218.5-7
\item[ἀγνοεῖ ] ἀγνοέω ne znati; 3. l. sg. ind. prez. akt. 
\item[ὅστις ἀγνοεῖ] tko stvarno ne bi znao; odnosna rečenica posljedičnog smisla, §~481, §~482, §~484
\item[τὸν ἐκεῖθεν πόλεμον] §~80, §~82; atributni položaj priloga §~375.5
\item[ἥξοντα] ἥκω doći, \textit{ovdje} približiti se; a. sg. m. r. ptc. fut. akt.; predikatni particip (§~502): da će se približiti\dots\ ako\dots
\item[ἂν ἀμελήσωμεν] ἀμελέω ne brinuti se za, zanemariti; 1. l. pl. konj. aor. akt.; ἂν ovdje = εἰ ἂν; konjunktiv s ἂν u protazi pogodbene eventualne (futurske) rečenice, §~476

\end{description}

{\large
\begin{greek}
\noindent ἀλλὰ μήν,\\
εἰ τοῦτο γενήσεται,\\
δέδοικ',\\
\tabto{2em} ὦ ἄνδρες ᾿Αθηναῖοι,\\
μὴ τὸν αὐτὸν τρόπον\\
\tabto{2em} ὥσπερ οἱ δανειζόμενοι \\
\tabto{4em} ῥᾳδίως \\
\tabto{4em} ἐπὶ τοῖς μεγάλοις [τόκοις] \\
\tabto{2em} μικρὸν εὐπορήσαντες χρόνον \\
\tabto{2em} ὕστερον καὶ τῶν ἀρχαίων ἀπέστησαν,\\
οὕτω καὶ ἡμεῖς [ἂν] \\
\tabto{2em} ἐπὶ πολλῷ \\
φανῶμεν ἐρρᾳθυμηκότες,\\
καὶ ἅπαντα πρὸς ἡδονὴν ζητοῦντες \\
\tabto{2em} πολλὰ καὶ χαλεπὰ \\
\tabto{3em} ὧν οὐκ ἐβουλόμεθ' \\
\tabto{2em} ὕστερον εἰς ἀνάγκην ἔλθωμεν ποιεῖν,\\
\tabto{2em} καὶ κινδυνεύσωμεν περὶ τῶν ἐν αὐτῇ τῇ χώρᾳ.\\

\end{greek}
}

\begin{description}[noitemsep]
\item[ἀλλὰ μήν] nego ipak
\item[εἰ\dots\ γενήσεται] umetnuta zavisna pogodbena rečenica, realna pogodba (§~474): ako se bude\dots\ dogodilo
\item[γενήσεται] γίγνομαι postati, zbiti se; 3. l. sg. ind. fut. med.
\item[τοῦτο ] §~213.2
\item[δέδοικ' = δέδοικα] §~68; δείδω bojati se; 1. l. sg. ind. perf. akt.
\item[ὦ ἄνδρες ᾿Αθηναῖοι] §~80, §~103, §~149
\item[τὸν αὐτὸν τρόπον] na isti način; adverbni akuzativ §~391
\item[δέδοικα μὴ\dots] bojim se da ćemo\dots, rečenica uz glagole bojanja; §~471
\item[ὥσπερ… ἀπέστησαν] poredbena rečenica: kao što\dots
\item[οἱ δανειζόμενοι] δανείζω pozajmiti; n. pl. m. r. ptc. prez. medpas.; supstantiviranje članom §~373
\item[ῥᾳδίως] ovdje: olako, lakomisleno; §~204
\item[ἐπὶ τοῖς μεγάλοις τόκοις] §~80, §~82, §~196; prijedložni izraz ἐπὶ + d.: po\dots,§~418, §~436.B; ὁ τόκος kamata; uglate zagrade označavaju tekst koji stoji u rukopisima, ali moderni priređivači smatraju da ga nije bilo u izvorniku (da ga se može isključiti)
\item[εὐπορήσαντες] εὐπορέω živjeti u obilju; n. pl. m. r. ptc. aor. akt.
\item[μικρὸν\dots\ χρόνον] §~80, §~82, §~103, §~202; akuzativ protezanja u vremenu §~390
\item[τῶν ἀρχαίων] §~80, §~103; §~373; τὸ ἀρχαῖον kapital
\item[ἀπέστησαν] ἀφίστημι ostati bez; 3. pl. ind. aor. akt.; gnomski aorist §~454
\item[οὕτω] §~221
\item[ἡμεῖς] §~205-206
\item[ἐπὶ πολλῷ] po visokoj cijeni; §~196; prijedložni izraz ἐπὶ + d.: po\dots; §~418, §~436.B
\item[φανῶμεν] φαίνομαι pokazati se; 1. l. pl. konj. aor. pas.; ovaj i svi sljedeći konjunktivi ovise o μὴ kao vezniku namjernih rečenica uvedenih s glagolom bojanja §~471
\item[ἐρρᾳθυμηκότες] ῥᾳθυμέω biti lakomislen; n. pl. m. r. ptc. perf. akt.
\item[ἅπαντα] §~379
\item[πρὸς ἡδονὴν] §~80, §~90; prijedložni izraz πρὸς + a.: radi\dots, §~418, §~435.C
\item[ζητοῦντες] sc.\ ποιεῖν; ζητέω težiti; n. pl. m. r. ptc. prez. akt.
\item[πολλὰ καὶ χαλεπὰ] §~103, §~196
\item[ὧν] §~215, §~216; = τούτων ἃ; asimilacija relativa §~444; partitivni genitiv ovisan o πολλὰ
\item[οὐκ ἐβουλόμεθ' = οὐκ ἐβουλόμεθα] βούλομαι htjeti, željeti; 1. l. pl. impf. medpas. 
\item[εἰς ἀνάγκην] §~80, §~90; prijedložni izraz εἰς + a.: u\dots; §~418, §~419
\item[ἔλθωμεν] ἔρχομαι doći; 1. l. pl. konj. aor. akt.
\item[εἰς ἀνάγκην ἔλθωμεν + inf.] = ἀναγκασθῶμεν, 1. l. pl. konj. aor. pas.; ἀναγκάζω prisiliti 
\item[ποιεῖν] ποιέω učiniti; inf. prez. akt. 
\item[εἰς ἀνάγκην ἔλθωμεν ποιεῖν] da ćemo biti prisiljeni učiniti (ovisno o δέδοικα μὴ\dots)
\item[κινδυνεύσωμεν περὶ] κινδυνεύω περί τινος dovesti (što) u opasnost; 1. l. pl. konj. aor. akt.
\item[περὶ τῶν ἐν αὐτῇ τῇ χώρᾳ] §~80, §~90, §~207; prijedložni izraz ἐν + d.: u\dots; supstantiviranje članom §~373

\end{description}


%kraj
