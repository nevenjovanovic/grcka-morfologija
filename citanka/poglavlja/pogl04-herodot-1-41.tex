% Unesi korekture NČ 2019-09-09
% Unio korekture NZ i NJ <2022-01-10 pon>
\section*{O autoru}

Grčki povjesničar Herodot (Ἡρόδοτος) iz Halikarnasa u Kariji (nakon 490.\ – Atena, nakon 430.\ pr.~Kr.) proputovao je velik dio Grcima poznatog svijeta, bio je prijatelj Perikla i Sofokla, sudjelovao u osnivanju atenske kolonije Turija u Tarantskom zaljevu u Velikoj Grčkoj (444.\ pr.~Kr.); doživio je početak Peloponeskog rata.

U prvome sačuvanom proznom djelu grčke književnosti, opsežnoj \textit{Povijesti} (Ἱστορίαι) o sukobu Grka i Perzijanaca do 479., Herodot je opisao uspon Perzijskog Carstva, pobunu u Joniji i dvije perzijske provale u Grčku, uz mnogo geografskih, etnografskih i mitoloških ekskursa. \textit{Povijest} je u biti niz zanimljivih i poučnih priča, često o slavnim i legendarnim likovima. Djelo je sastavio jonskim dijalektom. Bilo je namijenjeno javnom izvođenju kao svojevrsna prozna rapsodija; nije jasno je li sam autor oblikovao cjelovito izdanje. Do nas je stiglo podijeljeno u devet knjiga od kojih svaka nosi ime po jednoj muzi. Podjelu su proveli aleksandrijski filolozi u III.~st.\ pr.~Kr.

\section*{O tekstu}

U prvoj knjizi \textit{Povijesti, Klio,} Herodot pripovijeda o propasti kraljevstva lidijskog kralja Kreza (Κροῖσος; vladao 563.\ – 546.\ pr.~Kr.), fantastično bogatog Gigova nasljednika. Perzijski vladar Kir II.\ Veliki (staroperzijski Kūruš, grčki Κῦρος, između 590.\ i 580.\ – 529.\ pr.~Kr.) osvojio je Krezovu prijestolnicu Sard i zarobio samog kralja (koji je postao Kirov savjetnik). Prva knjiga \textit{Povijesti} završit će prikazom Kirove smrti. U ovdje donesenom odlomku Kir se priprema osvojiti grčke gradove u Maloj Aziji, u kojima žive Jonjani i Eoljani; ti su gradovi dotad priznavali Krezovu vlast.

%\newpage

\section*{Pročitajte naglas grčki tekst.}

Hdt.\ Historiae 1.141

%Naslov prema izdanju

\medskip

\begin{greek}
{\large
{ \noindent Ἴωνες δὲ καὶ Αἰολέες, ὡς οἱ Λυδοὶ τάχιστα κατεστράφατο ὑπὸ Περσέων, ἔπεμπον ἀγγέλους ἐς Σάρδις παρὰ Κῦρον, ἐθέλοντες ἐπὶ τοῖσι αὐτοῖσι εἶναι τοῖσι καὶ Κροίσῳ ἦσαν κατήκοοι. Ὁ δὲ ἀκούσας αὐτῶν τὰ προΐσχοντο ἔλεξέ σφι λόγον, ἄνδρα φὰς αὐλητὴν ἰδόντα ἰχθῦς ἐν τῇ θαλάσσῃ αὐλέειν, δοκέοντά σφεας ἐξελεύσεσθαι ἐς γῆν. Ὡς δὲ ψευσθῆναι τῆς ἐλπίδος, λαβεῖν ἀμφίβληστρον καὶ περιβαλεῖν τε πλῆθος πολλὸν τῶν ἰχθύων καὶ ἐξειρύσαι, ἰδόντα δὲ παλλομένους εἰπεῖν ἄρα αὐτὸν πρὸς τοὺς ἰχθῦς· ``Παύεσθέ μοι ὀρχεόμενοι, ἐπεὶ οὐδ' ἐμέο αὐλέοντος ἠθέλετε ἐκβαίνειν [ὀρχεόμενοι].'' 

Κῦρος μὲν τοῦτον τὸν λόγον τοῖσι Ἴωσι καὶ τοῖσι Αἰολεῦσι τῶνδε εἵνεκα ἔλεξε, ὅτι δὴ οἱ Ἴωνες πρότερον αὐτοῦ Κύρου δεηθέντος δι' ἀγγέλων ἀπίστασθαί σφεας ἀπὸ Κροίσου οὐκ ἐπείθοντο, τότε δὲ κατεργασμένων τῶν πρηγμάτων ἦσαν ἕτοιμοι πείθεσθαι Κύρῳ. 

Ὁ μὲν δὴ ὀργῇ ἐχόμενος ἔλεγέ σφι τάδε.  Ἴωνες δὲ ὡς ἤκουσαν τούτων ἀνενειχθέντων ἐς τὰς πόλις, τείχεά τε περιεβάλοντο ἕκαστοι καὶ συνελέγοντο ἐς Πανιώνιον οἱ ἄλλοι πλὴν Μιλησίων· πρὸς μούνους γὰρ τούτους ὅρκιον Κῦρος ἐποιήσατο ἐπ' οἷσί περ ὁ Λυδός· τοῖσι δὲ λοιποῖσι [Ἴωσι] ἔδοξε κοινῷ λόγῳ πέμπειν ἀγγέλους ἐς Σπάρτην δεησομένους Ἴωσι τιμωρέειν.
}
}
\end{greek}

\section*{Analiza i komentar}

%1

{\large
\begin{greek}
\noindent Ἴωνες δὲ καὶ Αἰολέες, \\
\tabto{2em} ὡς \\
\tabto{4em} οἱ Λυδοὶ \\
\tabto{6em} τάχιστα \\
\tabto{4em} κατεστράφατο \\
\tabto{6em} ὑπὸ Περσέων, \\
ἔπεμπον ἀγγέλους \\
\tabto{2em} ἐς Σάρδις \\
\tabto{2em} παρὰ Κῦρον, \\
ἐθέλοντες \\
\tabto{2em} ἐπὶ τοῖσι αὐτοῖσι εἶναι \\
\tabto{4em} τοῖσι καὶ Κροίσῳ ἦσαν κατήκοοι.\\

\end{greek}
}

\begin{description}[noitemsep]
\item[Ἴωνες ] §~131
\item[δὲ] a\dots; čestica povezuje rečenicu s (izostavljenom) prethodnom
\item[Αἰολέες] §~175 (nestegnuti oblik)
\item[ὡς\dots\ τάχιστα] „čim''; veznik uvodi zavisnu vremensku rečenicu §~487
\item[οἱ Λυδοὶ ] §~82
\item[τάχιστα]  §~204
\item[κατεστράφατο] καταστρέφομαι (med.) pokoriti; 3. l. pl. plpf. medpas. (subjekt: \textgreek[variant=ancient]{οἱ Λυδοὶ)}
\item[ὑπὸ Περσέων] §~437; §~100, nestegnuti oblik
\item[ἔπεμπον ] πέμπω poslati; 3. l. pl. impf. akt.
\item[ἀγγέλους ] §~82
\item[ἐς Σάρδις ] §~419; §~165
\item[παρὰ Κῦρον] §~434; §~82
\item[ἐθέλοντες] ἐθέλω željeti; n. pl. m. r. ptc. prez. akt.; otvara mjesto dopuni u infinitivu
\item[ἐπὶ τοῖσι αὐτοῖσι εἶναι] (sc. \textgreek[variant=ancient]{\textbf{κατήκοοι)\dots\ τοῖσι\dots}} pod istim uvjetima pod kojima\dots, §~89.3; §~207; §~413.3; §~436b.B.d; korelacija
\item[εἶναι] εἰμί ἐπί τινι biti u kojem odnosu; inf. prez. akt., obvezatna dopuna uz ἐθέλοντες
\item[τοῖσι] §~89.3; član je upotrijebljen kao odnosna zamjenica, čiji je antecedent τοῖσι αὐτοῖσι; uvodi zavisnu odnosnu rečenicu
\item[Κροίσῳ] §~82
\item[ἦσαν] εἰμί biti; 3. l. pl. impf. akt.
\item[κατήκοοι] §~106; rekcija τινι
\end{description}

%2

{\large
\begin{greek}
\noindent Ὁ δὲ \\
\tabto{2em} ἀκούσας αὐτῶν \\
\tabto{4em} τὰ προΐσχοντο \\
ἔλεξέ σφι λόγον, \\
\tabto{4em} \underline{ἄνδρα} \\
\tabto{2em} φὰς \\
\tabto{4em} \underline{αὐλητὴν} \\
\tabto{6em} \underline{ἰδόντα} ἰχθῦς ἐν τῇ θαλάσσῃ \\
\tabto{4em} \underline{αὐλέειν}, \\
\tabto{6em} \underline{δοκέοντά} \\
\tabto{8em} \underline{σφεας ἐξελεύσεσθαι} ἐς γῆν.\\

\end{greek}
}

\begin{description}[noitemsep]
\item[Ὁ δὲ ] §~370.2; čestica ističe promjenu subjekta
\item[ἀκούσας] ἀκούω τί τινος slušati što od koga; n. sg. m. r. ptc. aor. akt.
\item[αὐτῶν ] §~207
\item[τὰ ] član je upotrijebljen kao odnosna zamjenica; uvodi zavisnu odnosnu rečenicu; ta rečenica ovdje ima službu objekta uz ἀκούσας
\item[προΐσχοντο] προΐσχω med. tražiti, nuditi; 3. l. pl. impf. medpas.
\item[ἔλεξέ σφι] §~40
\item[ἔλεξέ ] λέγω govoriti; 3. l. sg. ind. aor. akt.
\item[ἔλεξε\dots\ φάς] čest pleonastički izraz kod Herodota (također susrećemo \textgreek[variant=ancient]{ἔφη\dots\ λέγων, ἔφασαν\dots\ φάμενοι)}
\item[σφι ] §~206.5 
\item[λόγον] §~82, ovdje znači otprilike što i \textgreek[variant=ancient]{μῦθον} „basna, priča'' (ova se slavna parabola nalazi i u Ezopovoj zbirci, v. LSJ λογοποιός; moguća aluzija u Novom zavjetu, Matej 11:17 i Luka 7:32 \textgreek[variant=ancient]{Ηὐλήσαμεν ὑμῖν καὶ οὐκ ὠρχήσασθε})
\item[ἄνδρα ] §~149
\item[φὰς] φημί govoriti; n. sg. m. r. ptc. prez. akt.
\item[αὐλητὴν ] §~100
\item[ἰδόντα ] ὁράω gledati; a. sg. m. r. ptc. aor. akt.; particip ovisan o ἄνδρα αὐλητὴν
\item[ἰχθῦς ] §~173
\item[ἐν τῇ θαλάσσῃ ] §~426; §~97
\item[αὐλέειν] αὐλέω svirati frulu (αὐλός); inf. prez. akt. (s vrijednošću prošlog vremena), nestegnuti oblik
\item[δοκέοντά σφεας] §~40
\item[δοκέοντά ] δοκέω smatrati, očekivati; a. sg. m. r. ptc. prez. akt., nestegnuti oblik; particip ovisan o ἄνδρα αὐλητὴν
\item[σφεας ] §~206.5 
\item[ἐξελεύσεσθαι ] ἐξέρχομαι izaći; inf. fut. med.
\item[ἐς γῆν] §~419; §~108

\end{description}

%3

{\large
\begin{greek}
\noindent Ὡς δὲ \\
\tabto{2em} \underline{ψευσθῆναι} τῆς ἐλπίδος, \\
\underline{λαβεῖν} \\
\tabto{2em} ἀμφίβληστρον \\
καὶ \underline{περιβαλεῖν} τε \\
\tabto{2em} πλῆθος πολλὸν τῶν ἰχθύων \\
καὶ \underline{ἐξειρύσαι}, \\
\underline{ἰδόντα} δὲ \\
\tabto{2em} παλλομένους \\
\underline{εἰπεῖν} ἄρα \underline{αὐτὸν} \\
\tabto{2em} πρὸς τοὺς ἰχθῦς· \\
``Παύεσθέ μοι ὀρχεόμενοι, \\
\tabto{2em} ἐπεὶ \\
\tabto{2em} οὐδ' \\
\tabto{4em} \uuline{ἐμέο αὐλέοντος} \\
\tabto{2em} ἠθέλετε \\
\tabto{4em} ἐκβαίνειν [ὀρχεόμενοι].''\\

\end{greek}
}

\begin{description}[noitemsep]
\item[Ὡς δὲ ] veznik uvodi zavisnu vremensku rečenicu §~487; čestica δέ povezuje rečenicu s prethodnom
\item[Ὡς\dots\ ψευσθῆναι] u neupravnom govoru infinitiv može doći kako iza relativa i vremenskih veznika tako i iza εἰ (neatički); subjekt se može dokučiti iz \textgreek[variant=ancient]{ἔλεξε} u prethodnoj rečenici
\item[ψευσθῆναι ] ψεύδω varati, med.\ ψεύδομαί τινος prevariti se u čemu; inf. aor. pas.
\item[τῆς ἐλπίδος] §~123
\item[λαβεῖν ] λαμβάνω uzeti; inf. aor. akt.
\item[ἀμφίβληστρον] §~82
\item[περιβαλεῖν τε] §~40
\item[περιβαλεῖν ] περιβάλλω okružiti; inf. aor. akt.
\item[περιβαλεῖν τε\dots\  καὶ ἐξειρύσαι] koordinacija sastavnim česticama  τε\dots\  καὶ\dots (drugi je član istaknutiji)
\item[πλῆθος πολλὸν] §~153; §~196 
\item[τῶν ἰχθύων] §~173
\item[ἐξειρύσαι] ἐξερύω, jonski ἐξειρύω izvući; inf. aor. akt.
\item[ἰδόντα] ὁράω gledati; a. sg. m. r. ptc. aor. akt.; particip ovisan o αὐτὸν
\item[λαβεῖν\dots\ ἰδόντα δὲ\dots] čestica povezuje surečenicu s prethodnom: a\dots
\item[παλλομένους ] πάλλω med. koprcati se; a. pl. m. r. ptc. prez. medpas.; predikatni particip kao dopuna glagolu \textgreek[variant=ancient]{ὁράω} §~502
\item[εἰπεῖν ] λέγω govoriti; inf. aor. akt.
\item[ἄρα ] čestica ističe zanimljivost onog što slijedi
\item[αὐτὸν ] §~207
\item[πρὸς τοὺς ἰχθῦς] §~435; §~173
\item[Παύεσθέ μοι] §~40
\item[Παύεσθέ] παύω (med. s participom) prestati što raditi; 2. l. pl. impt. prez. medpas.
\item[ὀρχεόμενοι] ὀρχέομαι plesati; n. pl. m. r. ptc. prez. med., predikatni particip uz Παύεσθέ
\item[μοι] §~205
\item[ἐπεὶ ] veznik uvodi zavisnu vremensku rečenicu §~487
\item[οὐδ' ἐμέο] §~68
\item[ἐμέο ] §~206.b
\item[αὐλέοντος ] αὐλέω svirati frulu; g. sg. m. r. ptc. prez. akt., nestegnuti oblik
\item[ἠθέλετε ] ἐθέλω željeti; 2. l. pl. impf. akt.; otvara mjesto dopuni u infinitivu
\item[ἐκβαίνειν] ἐκβαίνω izaći; inf. prez. akt.
\item[ὀρχεόμενοι] ὀρχέομαι plesati; n. pl. m. r. ptc. prez. medpas.; u kritičkim izdanjima, uglate zagrade označavaju riječi koje priređivači smatraju viškom (postoje u rukopisnoj predaji, ali ne bi trebale biti u tekstu)

\end{description}

%4

{\large
\begin{greek}
\noindent Κῦρος μὲν \\
τοῦτον τὸν λόγον \\
τοῖσι Ἴωσι καὶ τοῖσι Αἰολεῦσι \\
τῶνδε εἵνεκα \\
ἔλεξε, \\
\tabto{2em} ὅτι δὴ οἱ Ἴωνες \\
\tabto{2em} πρότερον \\
\tabto{4em} \uuline{αὐτοῦ Κύρου δεηθέντος} δι' ἀγγέλων \\
\tabto{6em} \underline{ἀπίστασθαί σφεας} ἀπὸ Κροίσου \\
\tabto{2em} οὐκ ἐπείθοντο, \\
\tabto{2em} τότε δὲ \\
\tabto{4em} \uuline{κατεργασμένων τῶν πρηγμάτων} \\
\tabto{2em} ἦσαν ἕτοιμοι \\
\tabto{4em} πείθεσθαι Κύρῳ.\\

\end{greek}
}

\begin{description}[noitemsep]
\item[Κῦρος μὲν\dots] čestica povezuje rečenicu s prethodnom, ističe uvođenje novog subjekta: a\dots
\item[τοῦτον τὸν λόγον] §~213.2; §~82
\item[τοῖσι Ἴωσι] §~89.3; §~131
\item[τοῖσι Αἰολεῦσι ] §~89.3; §~175 
\item[τῶνδε εἵνεκα ] §~417; §~213.1
\item[ἔλεξε] λέγω govoriti; 3. l. sg. ind. aor. akt.
\item[ὅτι δὴ] veznik uvodi zavisnu uzročnu rečenicu, a δὴ naglašava subjektivnost razloga: jer, po njemu\dots
\item[οἱ Ἴωνες] §~131
\item[πρότερον\dots\  τότε δὲ\dots] koordinacija surečenica česticom δὲ, koja ističe suprotnost (vremenskih okolnosti): prije\dots\  a tada\dots
\item[αὐτοῦ ] §~207
\item[Κύρου ] §~82
\item[δεηθέντος ] δέω medpas. moliti; g. sg. m. r. ptc. aor. pas.
\item[δι' ἀγγέλων ] §~68; §~428; §~82
\item[ἀπίστασθαί σφεας] §~40
\item[ἀπίστασθαί ] ἀφίστημι (ἀπίστημι) ἀπό τινος pobuniti se protiv koga; inf. prez. medpas.
\item[σφεας ] §~206.5 
\item[ἀπὸ Κροίσου] §~423; §~82
\item[ἐπείθοντο] πείθω med. poslušati, pristati; 3. l. pl. impf. medpas.
\item[κατεργασμένων ] κατεργάζομαι ostvariti, postići; g. pl. s. r. ptc. perf. medpas.
\item[τῶν πρηγμάτων ] §~123; πρᾶγμα, jonski πρῆγμα
\item[ἦσαν ἕτοιμοι] imenski predikat
\item[ἦσαν ] εἰμί biti; 3. l. pl. impf. akt.
\item[ἕτοιμοι ] §~103
\item[πείθεσθαι ] πείθω med. πείθομαί τινι pokoravati se komu; inf. prez. medpas., obvezatna dopuna predikatu (otvara mu mjesto ἕτοιμοι)
\item[Κύρῳ] §~82

\end{description}


%5

{\large
\begin{greek}
\noindent Ὁ μὲν δὴ \\
\tabto{2em} ὀργῇ ἐχόμενος \\
ἔλεγέ σφι τάδε.\\

\end{greek}
}

\begin{description}[noitemsep]
\item[Ὁ μὲν δὴ\dots\  Ἴωνες δὲ\dots] (na početku sljedeće rečenice) koordinacija rečenica česticama μὲν\dots\ δὲ\dots; μὲν je resumptivno, podsjeća na to da je riječ o Kiru (i najavljuje novu informaciju o njemu);  δὴ najavljuje zaključni dio prethodnog iskaza; sljedeća rečenica izvještava o reakciji na Kirovu izjavu; §~370.2
\item[ὀργῇ ] §~90
\item[ἐχόμενος ] ἔχω pas. ἔχομαί τινι biti svladan ili zahvaćen čime, kojim osjećajem; n. sg. m. r. ptc. prez. medpas.
\item[ἔλεγέ σφι] §~40
\item[ἔλεγέ ] λέγω govoriti; 3. l. sg. impf. akt.
\item[σφι ] §~206.5 
\item[τάδε] §~213.1

\end{description}


%6

{\large
\begin{greek}
\noindent Ἴωνες δὲ \\
\tabto{2em} ὡς ἤκουσαν \\
\tabto{4em} τούτων ἀνενειχθέντων \\
\tabto{6em} ἐς τὰς πόλις, \\
τείχεά τε περιεβάλοντο \\
ἕκαστοι \\
καὶ συνελέγοντο ἐς Πανιώνιον \\
οἱ ἄλλοι \\
\tabto{2em} πλὴν Μιλησίων· \\
\tabto{4em} πρὸς μούνους γὰρ τούτους \\
\tabto{4em} ὅρκιον Κῦρος ἐποιήσατο \\
\tabto{6em} ἐπ' οἷσί περ ὁ Λυδός· \\
\tabto{4em} τοῖσι δὲ λοιποῖσι [Ἴωσι] ἔδοξε \\
\tabto{6em} κοινῷ λόγῳ \\
\tabto{6em} πέμπειν ἀγγέλους ἐς Σπάρτην \\
\tabto{8em} δεησομένους Ἴωσι τιμωρέειν.\\

\end{greek}
}

\begin{description}[noitemsep]
\item[Ἴωνες] §~131
\item[Ἴωνες δὲ] čestica uvodi novi subjekt, ističe suprotnost u odnosu na subjekt prethodne rečenice
\item[ὡς ἤκουσαν] veznik uvodi zavisnu vremensku rečenicu §~487; ἀκούω τινός slušati o čemu; 3. l. pl. ind. aor. akt.
\item[τούτων ] §~213.2
\item[ἀνενειχθέντων] ἀναφέρω prenijeti, izvijestiti (donijeti natrag vijesti); g. pl. s. r. ptc. aor. pas.
\item[ἐς τὰς πόλις] §~419; §~165
\item[τείχεά τε] §~40
\item[τείχεά] §~153, nestegnuti oblik
\item[τείχεά τε περιεβάλοντο\dots\  καὶ συνελέγοντο\dots] koordinacija sastavnim česticama, pri čemu je drugi član istaknutiji
\item[περιεβάλοντο ] περιβάλλω med. περιβάλλομαί τι okružiti se čime, ograditi se čime; 3. l. pl. ind. aor. med.
\item[ἕκαστοι] §~103
\item[συνελέγοντο ] συλλέγω med. okupiti se; 3. l. pl. impf. medpas.
\item[ἐς Πανιώνιον ] §~419; §~82
\item[οἱ ἄλλοι ] §~212; §~371.1.b.4
\item[πλὴν Μιλησίων] §~417; §~103
\item[πρὸς μούνους\dots\  τούτους\dots\  τοῖσι δὲ λοιποῖσι] §~435; §~103; μόνος, jonski μούνος; §~213.2; §~89.3; koordinacija surečenica česticom  δὲ: a\dots
\item[γὰρ] čestica upozorava na to da se navodi razlog: naime\dots
\item[ὅρκιον\dots\  ἐποιήσατο] §~82; ποιέω činiti, med.\ ὅρκιον ποιέομαι sklopiti sporazum; 3. l. sg. ind. aor. med.
\item[Κῦρος ] §~82
\item[ἐπ' οἷσί περ] §~68; §~40
\item[ἐπ' οἷσί ] §~436; §~89.3; §~215
\item[ὁ Λυδός] sc.\ lidijski kralj, Krez; §~82
\item[τοῖσι\dots\  λοιποῖσι] \textbf{[Ἴωσι]\dots\ ἔδοξε} §~373; §~103; §~89.3; §~131; u kritičkim izdanjima, uglate zagrade označavaju riječi koje priređivači smatraju viškom (postoje u rukopisnoj predaji, ali ne bi trebale biti u tekstu); δοκεῖ τινι (δοκέω) netko je odlučio, otvara mjesto dopuni u infinitivu; 3. l. sg. ind. aor. akt.
\item[κοινῷ λόγῳ ] LSJ λόγος VII.4; §~103; §~82
\item[πέμπειν ] πέμπω poslati; inf. prez. akt., obvezatna dopuna uz ἔδοξε
\item[ἀγγέλους ] §~82
\item[ἐς Σπάρτην ] §~419; §~90
\item[δεησομένους ] δέω med. δέομαι moliti; otvara mjesto dopuni u infinitivu; a. pl. m. r. ptc. fut. med.; particip ovisan o ἀγγέλους, izriče namjeru „da zamole”
\item[Ἴωσι ] §~131
\item[τιμωρέειν] τιμωρέω τινί pomoći komu; inf. prez. akt., nestegnuti oblik; obvezatna dopuna uz δεησομένους

\end{description}

%kraj
