% Unio ispravke NČ 2019-09
\section*{O autoru}

Sofist Antifont \textgreek[variant=ancient]{(Ἀντιφῶν,} lat. Antiphon) iz Atene živio je u V.~st.\ pr.~Kr. Učenjaci se razilaze oko toga je li identičan s poznatim govornikom Antifontom iz atičke općine Ramnunta. Pripisuju mu se djela \textgreek[variant=ancient]{Περὶ ὁμονοίας} (\textit{O slozi}) i \textgreek[variant=ancient]{Περὶ ἀληθείας} (\textit{O istini}). Od drugospomenutog su djela sačuvani papirusni fragmenti (DK 44) koji kritiziraju konvencionalni moral s gledišta vlastitog interesa.

\section*{O tekstu}

Poruka je ovog ulomka, sačuvanog u Stobejevoj zbirci \textit{Florilegij} \textgreek[variant=ancient]{(Ἰωάννης ὁ Στοβαῖος,  Ἀνθολόγιον,} V.~st.\ po.~Kr; Stobaei Flor., XVI, 29) da se uspjeh ne sastoji u posjedovanju materijalnog bogatstva. Naime, oni koji gomilaju svoj novac, a da ga pritom ne koriste, nemaju veću korist od njega nego što bi imali od nekog kamena. Meta Antifontove kritike mogli bi biti ljudi ograničenog poimanja bogatstva, koji nisu sposobni upotrijebiti svoj um da bi shvatili što je pravo bogatstvo.

%\newpage

\section*{Pročitajte naglas grčki tekst.}

Antiph. Fr.14 (B54 DK)

%Naslov prema izdanju

\medskip

\begin{greek}
{\large
{ \noindent Ἰδὼν ἀνὴρ ἄνδρα ἕτερον ἀργύριον ἀναιρούμενον πολὺ ἐδεῖτό οἱ δανεῖσαι ἐπὶ τόκῳ, ὁ δ' οὐκ ἠθέλησεν, ἀλλ' ἦν οἷος ἀπιστεῖν τε καὶ μὴ ὠφελεῖν μηδένα, φέρων δ' ἀπέθετο ὅποι δή· καί τις καταμαθὼν τοῦτο ποιοῦντα ὑφείλετο, ὑστέρῳ δὲ χρόνῳ ἐλθὼν οὐχ ηὕρισκε τὰ χρήματα ὁ καταθέμενος. Περιαλγῶν οὖν τῇ συμφορᾷ τά τε ἄλλα καὶ ὅτι οὐκ ἔχρησε τῷ δεομένῳ, ὃ ἂν αὐτῷ καὶ σῶον ἦν καὶ ἕτερον προσέφερεν, ἀπαντήσας δὴ τῷ ἀνδρὶ τῷ τότε δανειζομένῳ ἀπωλοφύρετο τὴν συμφοράν, ὅτι ἐξήμαρτε καὶ ὅτι οἱ μεταμέλει οὐ χαρισαμένῳ, ἀλλ' ἀχαριστήσαντι, ὡς πάντως οἱ ἀπολόμενον τὸ ἀργύριον. ῾Ο δ' αὐτὸν ἐκέλευε μὴ φροντίζειν, ἀλλὰ νομίζειν αὑτῷ εἶναι καὶ μὴ ἀπολωλέναι, καταθέμενον λίθον εἰς τὸ αὐτὸ χωρίον· πάντως γὰρ οὐδ' ὅτε ἦν σοι ἐχρῶ αὐτῷ, ὅθεν μηδὲ νῦν νόμιζε στέρεσθαι μηδενός. ῞Οτῳ γάρ τις μὴ ἐχρήσατο μηδὲ χρήσεται, ὄντος ἢ μὴ ὄντος αὐτῷ οὐδὲν οὔτε πλέον οὔτε ἔλασσον βλάπτεται. ῞Οταν γὰρ ὁ θεὸς μὴ παντελῶς βούληται ἀγαθὰ διδόναι ἀνδρὶ, χρημάτων μὲν πλοῦτον παρασχών, τοῦ δὲ φρονεῖν καλῶς πένητα ποιήσας, τὸ ἕτερον ἀφελόμενος ἑκατέρων ἀπεστέρησεν.

}
}
\end{greek}

\section*{Analiza i komentar}

%1

{\large
\begin{greek}
\noindent Ἰδὼν ἀνὴρ \\
ἄνδρα ἕτερον \\
\tabto{2em} ἀργύριον ἀναιρούμενον \\
πολὺ ἐδεῖτό \\
\tabto{2em} οἱ δανεῖσαι ἐπὶ τόκῳ, \\
ὁ δ' οὐκ ἠθέλησεν, \\
ἀλλ' ἦν οἷος \\
\tabto{2em} ἀπιστεῖν τε καὶ μὴ ὠφελεῖν μηδένα, \\
φέρων δ' ἀπέθετο \\
\tabto{2em} ὅποι δή· \\
καί τις καταμαθὼν τοῦτο ποιοῦντα \\
ὑφείλετο, \\
ὑστέρῳ δὲ χρόνῳ ἐλθὼν \\
οὐχ ηὕρισκε τὰ χρήματα \\
ὁ καταθέμενος.\\

\end{greek}
}

\begin{description}[noitemsep]
\item[Ἰδὼν ἀνὴρ] ὁράω gledati; n. sg. m. r. ptc. aor. akt.; otvara mjesto za ἄνδρα ἕτερον;  §~149
\item[ἕτερον] §~219
\item[ἀργύριον ἀναιρούμενον] §~82; ἀργύριον je objekt participa ἀναιρούμενον; ἀναιρέομαι odnijeti; a. sg. m. r. ptc. prez. medpas.; predikatni particip uz \textit{verba sentiendi}, §~502.a
\item[πολὺ ἐδεῖτό] §~196; δέομαι moliti; 3. l. sg. impf. medpas.; properispomena ispred enklitike, §~40.d
\item[οἱ δανεῖσαι] οἱ ἑαυτῷ, §~206.5; δανείζω dati u zajam; inf. aor. akt.; subjekt infinitiva je ἀνὴρ
\item[ἐπὶ τόκῳ] §~436.B.c.δ; §~82
\item[ὁ δ'] čestica naznačava promjenu subjekta, §~370.2
\item[ἠθέλησεν] ἐθέλω voljan biti, htjeti; 3. l. sg. ind. aor. akt.
\item[ἀλλ' ἦν οἷος ] §~515.1; elizija §~68; εἰμί biti; 3. l. sg. impf. akt.; οἷος s inf. §~494.1.;
\item[ἀπιστεῖν\dots\ ὠφελεῖν] ἀπιστέω ne vjerovati; inf. prez. akt.; ὠφελέω pomagati, podupirati; inf. prez. akt.
\item[τε καὶ] §~513.2
\item[μηδένα] §~224.2; razlika među negacijama §~508
\item[φέρων δ' ἀπέθετο] φέρω nositi; n. sg. m. r. ptc. prez. akt.; §~515.2; ἀποτίθημι odložiti: 3. l. sg. ind. aor. med. 
\item[ὅποι δή] §~221; §~519.3
\item[καί τις] §~513.1; §~217; enklitike §~39, §~40
\item[καταμαθὼν] καταμανθάνω opaziti; n. sg. m. r. ptc. aor. akt.
\item[τοῦτο] §~213.2; objekt participa ποιοῦντα
\item[ποιοῦντα] ποιέω činiti; a. sg. m. r. ptc. prez. akt.; objekt participa καταμαθὼν
\item[ὑφείλετο] ὑφαιρέομαι krasti; 3. l. sg. ind. aor. med.
\item[ὑστέρῳ\dots\ χρόνῳ] §~203; §~82; \textit{dativus temporis} §~415
\item[ἐλθὼν] ἔρχομαι ići; n. sg. m. r. ptc. aor. akt.
\item[ηὕρισκε] εὑρίσκω naći; 3. l. sg. impf. akt.
\item[τὰ χρήματα] §~123
\item[ὁ καταθέμενος] κατατίθημι spremiti, položiti; n. sg. m. r. ptc. aor. med.; supstantiviranje članom §~373

\end{description}

%2

{\large
\begin{greek}
\noindent Περιαλγῶν οὖν τῇ συμφορᾷ \\
\tabto{2em} τά τε ἄλλα \\
\tabto{2em} καὶ ὅτι οὐκ ἔχρησε \\
\tabto{4em} τῷ δεομένῳ, \\
ὃ ἂν αὐτῷ καὶ σῶον ἦν \\
καὶ ἕτερον προσέφερεν, \\
ἀπαντήσας δὴ \\
\tabto{2em} τῷ ἀνδρὶ τῷ τότε δανειζομένῳ \\
ἀπωλοφύρετο τὴν συμφοράν, \\
\tabto{2em} ὅτι ἐξήμαρτε \\
\tabto{2em} καὶ ὅτι οἱ μεταμέλει \\
\tabto{4em} οὐ χαρισαμένῳ, \\
\tabto{4em} ἀλλ' ἀχαριστήσαντι, \\
\tabto{2em} ὡς πάντως οἱ ἀπολόμενον \\
\tabto{4em} τὸ ἀργύριον.\\

\end{greek}
}

\begin{description}[noitemsep]
\item[Περιαλγῶν οὖν] περιαλγέω bolom obuzet biti; n. sg. m. r. ptc. prez. akt.; §~516
\item[τῇ συμφορᾷ] §~90; \textit{dativus causae} §~414.2
\item[τά τε ἄλλα καὶ] osobito zato
\item[ὅτι οὐκ ἔχρησε] uzročne rečenice §~468; χράω davati, uzajmljivati; 3. l. sg. ind. aor. akt.
\item[τῷ δεομένῳ] δέομαι iskati, moliti; d. sg. m. r. ptc. prez. medpas.; supstantiviranje članom §~373
\item[ὃ ἂν\dots\ σῶον ἦν\dots\ προσέφερεν] §~215; relativ se odnosi na ἀργύριον; odnosne rečenice §~482
\item[αὐτῷ]  §~207
\item[σῶον ἦν] §~103; εἰμί biti; 3. l. sg. impf. akt. 
\item[ἕτερον προσέφερεν] §~219; προσφέρω donositi; 3. l. sg. impf. akt.
\item[ἀπαντήσας δὴ] ἀπαντάω τινί sresti koga; n. sg. m. r. ptc. aor. akt.; §~516.5
\item[τῷ ἀνδρὶ] §~149
\item[τῷ τότε δανειζομένῳ] atributni particip §~499; atributni položaj priloga \begin{greek}τότε\end{greek} §~375;  δανείζομαι uzeti u zajam; d. sg. m. r. ptc. prez. medpas.
\item[ἀπωλοφύρετο ] ἀπολοφύρομαι glasno jadikovati; 3. l. sg. impf. medpas.
\item[τὴν συμφοράν] §~90
\item[ὅτι ἐξήμαρτε] ὅτι kao veznik uzročnih rečenica §~468; ἐξαμαρτάνω pogriješiti; 3. l. sg. ind. aor. akt.
\item[οἱ μεταμέλει] οἱ = αὐτῷ, indirektni refleksiv §~439.b.2; μεταμέλει μοί τινος kajem se §~400.1
\item[οὐ χαρισαμένῳ] μεταμέλει μοί τινος s predikatnim participom §~501.e; \begin{greek}χαρίζομαι\end{greek} uslugu učiniti; d. sg. m. r. ptc. aor. med.
\item[ἀλλ' ἀχαριστήσαντι] §~515.1; ἀχαριστέω nezahvalan biti; d. sg. m. r. ptc. aor. med.
\item[ὡς πάντως] ὡς kao veznik uzročnih rečenica §~468; §~204
\item[οἱ ἀπολόμενον] οἱ = αὐτῷ, indirektni refleksiv; ἀπόλλυμαι propasti; n. sg. sr. r. ptc. aor. med.
\item[τὸ ἀργύριον] §~82

\end{description}


%3

{\large
\begin{greek}
\noindent ῾Ο δ' \\
αὐτὸν ἐκέλευε \\
\tabto{2em} μὴ φροντίζειν, \\
\tabto{2em} ἀλλὰ νομίζειν \\
\tabto{4em} αὑτῷ εἶναι \\
\tabto{4em} καὶ μὴ ἀπολωλέναι, \\
\tabto{6em} καταθέμενον λίθον \\
\tabto{8em} εἰς τὸ αὐτὸ χωρίον· \\
πάντως γὰρ \\
οὐδ' ὅτε ἦν σοι \\
ἐχρῶ αὐτῷ, \\
\tabto{2em} ὅθεν μηδὲ νῦν νόμιζε \\
\tabto{4em} στέρεσθαι μηδενός.\\

\end{greek}
}

\begin{description}[noitemsep]
\item[αὐτὸν ἐκέλευε ] §~207.3; κελεύω τινά + inf. pozivati koga na što; 3. l. sg. impf. akt.
\item[μὴ φροντίζειν] §~509.2; φροντίζω brinuti se; inf. prez. akt. 
\item[νομίζειν] νομίζω smatrati; inf. prez. akt.
\item[αὑτῷ εἶναι] §~208, §~209; εἰμί biti; inf. prez. akt.; posvojni dativ §~412.2; indirektni refleksiv §~439.b
\item[ἀπολωλέναι] ἀπόλλυμαι propasti; inf. perf. akt.
\item[καταθέμενον λίθον] κατατίθημι spremiti, položiti; a. sg. m. r. ptc. aor. med.; poveži s αὐτὸν; λίθον je objekt participa καταθέμενον, §~82
\item[εἰς τὸ αὐτὸ χωρίον] §~419.a; §~82; §~207; §~378.5
\item[γὰρ] §~517
\item[οὐδ' ὅτε ἦν σοι] οὐδ' < οὐδέ §~513.3; elizija §~82; ὅτε kao veznik vremenskih rečenica §~487; εἰμί biti, 3. l. sg. impf. akt.; §~205; posvojni dativ §~412.2
\item[ἐχρῶ ] χράομαι služiti se; 2. l. sg. impf. medpas.
\item[ὅθεν μηδὲ\dots\ νόμιζε] korelativni prilozi §~221; §~513.3; νομίζω smatrati; 2. l. sg. impt. prez. akt.
\item[στέρεσθαι μηδενός] στέρομαι τινός lišen biti nečega, ne imati nešto; inf. prez. medpas.; subjekt infinitiva naznačen u imperativu νόμιζε; §~224.2

\end{description}


%4

{\large
\begin{greek}
\noindent ῞Οτῳ γάρ τις \\
μὴ ἐχρήσατο μηδὲ χρήσεται, \\
\tabto{2em} ὄντος ἢ μὴ ὄντος αὐτῷ \\
οὐδὲν οὔτε πλέον οὔτε ἔλασσον \\
βλάπτεται.\\

\end{greek}
}

\begin{description}[noitemsep]
\item[῞Οτῳ ] ῞Οτῳ = ᾧτινι §~217, §~218.8
\item[ἐχρήσατο\dots\ χρήσεται] χράομαι služiti se; 3. l. sg. ind. aor. med.; drugi oblik istog glagola je 3. l. sg. ind. fut. med.
\item[ὄντος\dots\ αὐτῷ] εἰμί biti; g. sg. sr. r. ptc. prez. akt.; GA (uz ispušten pronominalni dio ὅτου = οὗτινος) s vrijednošću pogodbene rečenice; posvojni dativ §~412.2
\item[οὐδὲν\dots\ βλάπτεται] adverbni akuzativ §~391; βλάπτω štetiti; 3. l. sg. ind. prez. medpas.
\item[οὔτε\dots\ οὔτε\dots] §~513.4
\item[πλέον\dots\ ἔλασσον] komparacija od drugih osnova §~202

\end{description}

%5

{\large
\begin{greek}
\noindent ῞Οταν γὰρ ὁ θεὸς \\
μὴ παντελῶς βούληται \\
\tabto{2em} ἀγαθὰ διδόναι ἀνδρὶ, \\
χρημάτων μὲν πλοῦτον \\
\tabto{2em} παρασχών, \\
τοῦ δὲ φρονεῖν καλῶς \\
\tabto{2em} πένητα ποιήσας, \\
τὸ ἕτερον ἀφελόμενος \\
\tabto{2em} ἑκατέρων ἀπεστέρησεν. \\

\end{greek}
}

\begin{description}[noitemsep]
\item[῞Οταν\dots\ ὁ θεὸς μὴ\dots\ βούληται] ῞Οταν < ὅτε ἄν; vremenska rečenica sa značenjem pogodbene protaze eventualnog oblika §~488.2; §~82; βούλομαι htjeti, željeti; 3. l. sg. konj. prez. med.
\item[ἀγαθὰ διδόναι ἀνδρὶ] §~103; ἀγαθὰ je supstantivirani pridjev, član može izostati uz apstraktna imena; δίδωμι dati; inf. prez. akt.; §~149
\item[χρημάτων μὲν\dots\ τοῦ δὲ φρονεῖν καλῶς] §~123; φρονέω razborit biti; inf. prez. akt.; supstantiviranje članom §~373; \textit{genitivus inopiae} ovisan o πένητα §~403.2; koordinacija česticama μὲν\dots\ δὲ §~519.7; §~204
\item[πλοῦτον παρασχών] §~82; παρέχω pružati; n. sg. m. r. ptc. aor. akt.
\item[πένητα ποιήσας] jednozavršetni pridjevi §~195; ποιέω činiti; n. sg. m. r. ptc. aor. akt.
\item[τὸ ἕτερον ἀφελόμενος] korelativne zamjenice §~219; ἀφαιρέω oduzimati; n. sg. m. r. ptc. aor. med.
\item[ἀπεστέρησεν ] ἀποστερέω τινά τινος otimati od koga što; 3. l. sg. ind. aor. akt.

\end{description}


%kraj
