\section*{O autoru}

Plutarh (Πλούταρχος, Heroneja 46.–125.) bio je povjesničar, biograf i filozof. Smatra se posljednjim predstavnikom opće grčke obrazovanosti. Studirao je filozofiju, matematiku i govorništvo na atenskoj Akademiji. U Rimu je, pod zaštitom careva Trajana i Hadrijana, poučavao filozofiju, a potkraj života bio je Apolonov svećenik u rodnoj Heroneji. 

U filozofiji je nastojao pomiriti platonovske i aristotelovske teze. Pod imenom \textit{Moralia} (Ἠϑικά, zbirka od 78 eseja i govora) sačuvale su se njegove popularno-odgojne rasprave filozofskoga, religijskoga, političkoga, prirodoslovnoga, književnog i drugog sadržaja. Najpoznatiji je po \textit{Usporednim životopisima} \textgreek[variant=ancient]{(Βίοι παράλληλοι),} u kojima u parovima donosi biografije jednoga Grka i jednog Rimljanina (npr.\ Tezej i Romul, Likurg i Numa Pompilije) kako bi se istaknule njihove moralne vrijednosti i mane. Sačuvana su dvadeset tri para biografija, kao i četiri samostalne biografije. Tim izvanredno živo, anegdotski pisanim životopisima (Koriolan, Timon Atenjanin, Julije Cezar, Antonije, Kleopatra) služio se i Shakespeare u svojim dramama.

\section*{O tekstu}

Rasprava \textit{O brbljavosti} \textgreek[variant=ancient]{(Περὶ ἀδολεσχίας,} \textit{De garrulitate)} pripada zbirci \textit{Moralia}. Plutarh brbljavost prikazuje kao bolest za koju lijek pruža filozofija. U izabranom odlomku autor predstavlja tri načina odgovaranja na pitanja, na primjeru upita je li Sokrat kod kuće.

\newpage

\section*{Pročitajte naglas grčki tekst.}

%Naslov prema izdanju

Plut.\ De garrulitate 513A

\medskip

{\large
\begin{greek}
\noindent  Ἔστι τοίνυν τρία γένη τῶν πρὸς τὰς ἐρωτήσεις ἀποκρίσεων, τὸ μὲν ἀναγκαῖον τὸ δὲ φιλάνθρωπον τὸ δὲ περισσόν. οἷον πυθομένου τινὸς εἰ Σωκράτης ἔνδον, ὁ μὲν ὥσπερ ἄκων καὶ ἀπροθύμως ἀποκρίνεται τὸ ‘οὐκ ἔνδον’, ἐὰν δὲ βούληται λακωνίζειν, καὶ τὸ ‘ἔνδον’ ἀφελὼν αὐτὴν μόνην φθέγξεται τὴν ἀπόφασιν· ὡς ἐκεῖνοι, Φιλίππου γράψαντος εἰ δέχονται τῇ πόλει αὐτόν, εἰς χάρτην ‘οὐ’ μέγα γράψαντες ἀπέστειλαν. ὁ δὲ φιλανθρωπότερον ἀποκρίνεται ‘οὐκ ἔνδον ἀλλ' ἐπὶ ταῖς τραπέζαις’, κἂν βούληται προσεπιμετρῆσαι, ‘ξένους τινὰς ἐκεῖ περιμένων.’ ὁ δὲ περιττὸς καὶ ἀδολέσχης, ἄν γε δὴ τύχῃ καὶ τὸν Κολοφώνιον ἀνεγνωκὼς Ἀντίμαχον, ‘οὐκ ἔνδον’ φησίν ‘ἀλλ' ἐπὶ ταῖς τραπέζαις, ξένους ἀναμένων  Ἴωνας, ὑπὲρ ὧν αὐτῷ γέγραφεν Ἀλκιβιάδης περὶ Μίλητον ὢν καὶ παρὰ Τισσαφέρνῃ διατρίβων, τῷ τοῦ μεγάλου σατράπῃ βασιλέως, ὃς πάλαι μὲν ἐβοήθει Λακεδαιμονίοις, νῦν δὲ προστίθεται δι' Ἀλκιβιάδην Ἀθηναίοις· ὁ γὰρ Ἀλκιβιάδης ἐπιθυμῶν κατελθεῖν εἰς τὴν πατρίδα τὸν Τισσαφέρνην μετατίθησι’.

\end{greek}
}

%\newpage

\section*{Analiza i komentar}


%0

{\large
\begin{greek}
\noindent  Ἔστι τοίνυν τρία γένη \\
\tabto{2em} τῶν πρὸς τὰς ἐρωτήσεις ἀποκρίσεων, \\
τὸ μὲν ἀναγκαῖον, \\
τὸ δὲ φιλάνθρωπον, \\
τὸ δὲ περισσόν.\\

\end{greek}
}

\begin{description}[noitemsep] 
\item[ Ἔστι] εἰμί biti; 3. l. sg. ind. prez; naglašen oblik, LSJ εἰμί A.I: postojati
\item[τρία γένη] §~224, §~153
\item[τῶν πρὸς τὰς ἐρωτήσεις ἀποκρίσεων] §~165; prijedložni izraz u atributnom položaju §~375.4
\item[τὸ μὲν ἀναγκαῖον,] \textbf{τὸ δὲ φιλάνθρωπον, τὸ δὲ περισσόν} koordinacija rečeničnih članaka pomoću čestica; §~103; složeni pridjevi imaju samo dva završetka §~106

\end{description}
%1

{\large
\begin{greek}
\noindent οἷον \\
\tabto{2em} \uuline{πυθομένου τινὸς} \\
\tabto{4em} εἰ Σωκράτης ἔνδον,\\

\end{greek}
}

\begin{description}[noitemsep]
\item[οἷον] \textit{priložno} kao na primjer, LSJ οἷος A.V.2.b; §~82
\item[πυθομένου] πυνθάνομαι pitati; g. sg. m. r. ptc. aor. med.; kao \textit{verbum dicendi} otvara mjesto zavisnoj upitnoj rečenici
\item[τινὸς] §~217
\item[Σωκράτης] §~153
\item[εἰ] veznik uvodi zavisno upitnu rečenicu
\item[ἔνδον] imenski predikat s priložnom oznakom kao imenskim dijelom i neizrečenom kopulom; Smyth 909

\end{description}

%2

{\large
\begin{greek}
\noindent ὁ μὲν \\
\tabto{2em} ὥσπερ ἄκων καὶ ἀπροθύμως \\
ἀποκρίνεται τὸ ‘οὐκ ἔνδον’,\\
\tabto{2em} ἐὰν δὲ βούληται \\
\tabto{4em} λακωνίζειν,\\
καὶ τὸ ‘ἔνδον’ \\
\tabto{2em} ἀφελὼν \\
αὐτὴν μόνην \\
\tabto{2em} φθέγξεται \\
τὴν ἀπόφασιν·\\
ὡς ἐκεῖνοι, \\
\tabto{2em} \uuline{Φιλίππου γράψαντος} \\
\tabto{4em} εἰ δέχονται \\
\tabto{4em} τῇ πόλει \\
\tabto{4em} αὐτόν, \\
\tabto{2em} εἰς χάρτην \\
\tabto{2em} ‘οὐ’ μέγα \\
\tabto{2em} γράψαντες \\
ἀπέστειλαν.\\

\end{greek}
}

\begin{description}[noitemsep]
\item[ὁ μὲν ὥσπερ ἄκων\dots\ ὁ δὲ φιλανθρωπότερον\dots\ ὁ δὲ περιττὸς\dots] tri rečenice koje opisuju tri vrste onih koji odgovaraju koordinirane su česticama
\item[ἄκων] §~193
\item[ἀπροθύμως] §~106, §~204
\item[ἀποκρίνεται] ἀποκρίνω med. odgovarati (na pitanje); 3. l. sg. ind. prez. medpas.
\item[τὸ ‘οὐκ ἔνδον’] supstantiviranje članom §~373
\item[οὐκ ἔνδον] imenski predikat s priložnom oznakom kao imenskim dijelom i neizrečenom kopulom; Smyth 909
\item[ἐὰν] = εἰ ἄν; veznik uvodi protazu zavisne pogodbene rečenice, ovdje eventualnog futurskog oblika (konjunktiv u protazi, futur u apodozi); §~476
\item[ἐὰν δὲ\dots] čestica izražava suprotnost u odnosu na prethodnu surečenicu: a\dots
\item[βούληται] βούλομαι željeti, glagol nepotpuna značenja otvara mjesto dopuni u infinitivu; 3. l. sg. konj. prez. medpas.
\item[λακωνίζειν] λακωνίζω govoriti lakonski, tj.\ oponašati Spartance (Λάκωνες); inf. prez. akt.
\item[ἀφελὼν] ἀφαιρέω oduzimati; n. sg. m. r. ptc. aor. akt.; objekt participa je τὸ ἔνδον
\item[αὐτὴν μόνην\dots\ τὴν ἀπόφασιν] §~207, §~103, §~165
\item[φθέγξεται] φθέγγομαι izgovarati; 3. l. sg. ind. fut. med.
\item[ὡς] veznik uvodi poredbenu rečenicu: kao\dots
\item[ἐκεῖνοι] §~214
\item[Φιλίππου] §~82; Filip II, makedonski kralj od 359.\ pr.~Kr. (382.–336. pr.~Kr), otac Aleksandra III. Velikog; protiv Filipa je Demosten držao poznate govore. 
\item[γράψαντος] γράφω pisati; g. sg. m. r. ptc. aor. akt; GA ima vrijednost vremenske rečenice, §~504: kad\dots
\item[δέχονται] δέχομαι primati; 3. l. pl. ind. prez. medpas.
\item[τῇ πόλει] §~165, §~207; \textit{dativus loci} §~415
\item[εἰς χάρτην] §~90
\item[μέγα] §~196
\item[ἀπέστειλαν] ἀποστέλλω odašiljati; 3. l. pl. ind. aor. akt.
\end{description}

%4

%\newpage

{\large
\begin{greek}
\noindent ὁ δὲ φιλανθρωπότερον ἀποκρίνεται \\
\tabto{2em} ‘οὐκ ἔνδον \\
\tabto{4em} ἀλλ' ἐπὶ ταῖς τραπέζαις’, \\
κἂν βούληται \\
\tabto{2em} προσεπιμετρῆσαι, \\
\tabto{4em} ‘ξένους τινὰς ἐκεῖ περιμένων.’\\

\end{greek}

}

\begin{description}[noitemsep]
\item[φιλανθρωπότερον] §~197, 204.3
\item[ἀλλ'] = ἀλλά
\item[ἐπὶ ταῖς τραπέζαις] §~97; drugi dio imenskog predikata (s izostavljenom kopulom)
\item[κἂν] = καὶ ἐάν (usp. gore za pogodbenu rečenicu eventualnog oblika); §~16
\item[προσεπιμετρῆσαι] προσεπιμετρέω dodati preko mjere; inf. aor. akt., dopuna glagola nepotpunog značenja βούληται
\item[ξένους τινὰς] §~82, §~217
\item[περιμένων] περιμένω čekati; n. sg. m. r. ptc. prez. akt.
\end{description}

%5

{\large
\begin{greek}
\noindent ὁ δὲ περιττὸς καὶ ἀδολέσχης, \\
\tabto{2em} ἄν γε δὴ τύχῃ καὶ \\
\tabto{4em} τὸν Κολοφώνιον \\
\tabto{6em} ἀνεγνωκὼς \\
\tabto{4em} Ἀντίμαχον, \\
‘οὐκ ἔνδον’ φησίν \\
\tabto{2em} ‘ἀλλ' ἐπὶ ταῖς τραπέζαις, \\
\tabto{2em} ξένους \\
\tabto{4em} ἀναμένων \\
\tabto{2em}  Ἴωνας,\\
\tabto{4em} ὑπὲρ ὧν \\
\tabto{4em} αὐτῷ \\
\tabto{4em} γέγραφεν \\
\tabto{4em} Ἀλκιβιάδης \\
\tabto{6em} περὶ Μίλητον ὢν \\
\tabto{6em} καὶ παρὰ Τισσαφέρνῃ διατρίβων, \\
\tabto{8em} τῷ \\
\tabto{10em} τοῦ μεγάλου \\
\tabto{8em} σατράπῃ \\
\tabto{10em} βασιλέως, \\
\tabto{8em} ὃς πάλαι μὲν \\
\tabto{10em} ἐβοήθει \\
\tabto{10em} Λακεδαιμονίοις, \\
\tabto{8em} νῦν δὲ \\
\tabto{10em} προστίθεται \\
\tabto{12em} δι' Ἀλκιβιάδην \\
\tabto{10em} Ἀθηναίοις·
\tabto{12em} ὁ γὰρ Ἀλκιβιάδης \\
\tabto{12em} ἐπιθυμῶν \\
\tabto{14em} κατελθεῖν εἰς τὴν πατρίδα \\
\tabto{12em} τὸν Τισσαφέρνην \\
\tabto{12em} μετατίθησι.’\\

\end{greek}

}

\begin{description}[noitemsep]
\item[ὁ δὲ περιττὸς] usp. oblik περισσόν gore; §~57a
\item[ἀδολέσχης] §~100
\item[ἄν] = ἐάν (usp. gore za pogodbenu rečenicu eventualnog oblika)
\item[τύχῃ] τυγχάνω slučiti se; kopulativni glagol otvara mjesto predikatnom participu i prevodi se prilogom: slučajno\dots, §~501; 3. l. sg. konj. aor. akt.
\item[τὸν Κολοφώνιον\dots\ Ἀντίμαχον] §~82; Antimah iz Kolofona (oko 400.\ pr.~Kr.), pjesnik i gramatičar poznat po opširnosti i zamršenom stilu
\item[ἀνεγνωκὼς] ἀναγιγνώσκω čitati; n. sg. m. r. ptc. perf. akt.
\item[ἄν\dots\ Ἀντίμαχον] ako je slučajno čitao i Antimaha; §~501b
\item[φησίν] φημί govoriti; 3. l. sg. ind. prez. 
\item[ἀναμένων] isto značenje i oblik kao ranije περιμένων (v.~gore)
\item[Ἴωνας] §~131
\item[ὑπὲρ ὧν] o kojima, uvodi zavisnu odnosnu rečenicu (antecedent je ξένους); §~215
\item[αὐτῷ] §~207
\item[γέγραφεν] γράφω pisati; 3. l. sg. ind. perf. akt.
\item[Ἀλκιβιάδης] §~100; Alkibijad (oko 450.\ – 404.\ pr.~Kr.), slavni atenski političar, govornik i vojskovođa, Sokratov prijatelj i učenik, tijekom Peloponeskog rata višekratno mijenjao stranu, 412.\ pobjegao pod okrilje perzijskog namjesnika Tisaferna
\item[περὶ Μίλητον] §~82; Milet je antički grad u Joniji, blizu ušća rijeke Meandra (danas Menderes), najvažnija grčka naseobina u Maloj Aziji
\item[ὢν] εἰμί biti; n. sg. m. r. ptc. prez.
\item[παρὰ Τισσαφέρνῃ] §~100; Tisaferno, staroperzijski  Čiθrafarnah ili Čiçafarnah, umro nakon 395.\ pr.~Kr; namjesnik i vojni zapovjednik Perzijskog carstva, stolovao u lidijskom gradu Sardu
\item[διατρίβων] διατρίβω provoditi vrijeme; n. sg. m. r. ptc. prez. akt.
\item[τῷ τοῦ μεγάλου σατράπῃ βασιλέως] = τῷ σατράπῃ τοῦ μεγάλου βασιλέως; §~100, §~175, §~196; komplicirani hiperbat vjerojatno je parodija Antimahova zakučastog stila
\item[ὃς] §~215; odnosna zamjenica čiji je antecedent Τισσαφέρνῃ uvodi zavisnu odnosnu rečenicu
\item[ἐβοήθει] βοηθέω pomagati; 3. l. sg. impf. akt.
\item[Λακεδαιμονίοις\dots\ Ἀθηναίοις] §~82
\item[προστίθεται] προστίθημι svrstati se se uz nekog; 3. l. sg. ind. prez. medpas.
\item[δι'] = διά
\item[γὰρ] čestica najavljuje iznošenje dokaza prethodne tvrdnje: naime\dots
\item[ἐπιθυμῶν] ἐπιθυμέω žudjeti, silno željeti; glagol nepotpuna značenja otvara mjesto dopuni u infinitivu; n. sg. m. r. ptc. prez. akt.
\item[κατελθεῖν] κατέρχομαι vraćati se; inf. aor. akt.
\item[εἰς τὴν πατρίδα] §~123
\item[μετατίθησι] μετατίθημι navesti nekoga da promijeni mišljenje; 3. l. sg. ind. prez. akt.

\end{description}

%kraj
