%\section*{O autoru}
% pregledao i redigirao NJ, 14. 4. 2019. NZ (unio NJ 10. 5. 2019.)


\section*{O tekstu}

Ovaj je govor, deveti od 21 koji su stigli do nas, Izokratova nadgrobna pohvala ciparskog kralja Euagore (411.\ – 374. pr.~Kr.), oca kralja Nikokla (njemu je pak Izokrat posvetio svoj drugi i treći govor). U odabranom odlomku Izokrat uspoređuje pjesnike i govornike ističući da je govorniku naklonost slušatelja mnogo lakše pridobiti pjesničkim izrazom nego uobičajenim sredstvima žanra javnog govora.

%\newpage; NZ 25. 10. 2021.

\section*{Pročitajte naglas grčki tekst.}

%Naslov prema izdanju

Isoc.\ Evagoras 9

\medskip

{\large
\begin{greek}
\noindent Τοῖς μὲν γὰρ ποιηταῖς πολλοὶ δέδονται κόσμοι· καὶ γὰρ πλησιάζοντας τοὺς θεοὺς τοῖς ἀνθρώποις οἷόν τ' αὐτοῖς ποιῆσαι καὶ διαλεγομένους καὶ συναγωνιζομένους οἷς ἂν βουληθῶσιν, καὶ περὶ τούτων δηλῶσαι μὴ μόνον τοῖς τεταγμένοις ὀνόμασιν, ἀλλὰ τὰ μὲν ξένοις, τὰ δὲ καινοῖς, τὰ δὲ μεταφοραῖς, καὶ μηδὲν παραλιπεῖν, ἀλλὰ πᾶσιν τοῖς εἴδεσιν διαποικῖλαι τὴν ποίησιν· τοῖς δὲ περὶ τοὺς λόγους οὐδὲν ἔξεστιν τῶν τοιούτων, ἀλλ' ἀποτόμως καὶ τῶν ὀνομάτων τοῖς πολιτικοῖς μόνον καὶ τῶν ἐνθυμημάτων τοῖς περὶ αὐτὰς τὰς πράξεις ἀναγκαῖόν ἐστιν χρῆσθαι. Πρὸς δὲ τούτοις οἱ μὲν μετὰ μέτρων καὶ ῥυθμῶν ἅπαντα ποιοῦσιν, οἱ δ' οὐδενὸς τούτων κοινωνοῦσιν· ἃ τοσαύτην ἔχει χάριν ὥστ', ἂν καὶ τῇ λέξει καὶ τοῖς ἐνθυμήμασιν ἔχῃ κακῶς, ὅμως αὐταῖς ταῖς εὐρυθμίαις καὶ ταῖς συμμετρίαις ψυχαγωγοῦσιν τοὺς ἀκούοντας.

\end{greek}

}

\section*{Analiza i komentar}

%1

{\large
\noindent Τοῖς μὲν γὰρ ποιηταῖς \\
\tabto{2em} πολλοὶ\\ 
δέδονται\\ 
\tabto{2em} κόσμοι·\\
καὶ γὰρ \\
\tabto{2em} πλησιάζοντας τοὺς θεοὺς\\
\tabto{4em} τοῖς ἀνθρώποις \\
οἷόν τ' \\
\tabto{2em} αὐτοῖς \\
\tabto{4em} ποιῆσαι\\
\tabto{2em} καὶ διαλεγομένους \\
\tabto{2em} καὶ συναγωνιζομένους\\
\tabto{4em} οἷς ἂν βουληθῶσιν,\\
\tabto{2em} καὶ περὶ τούτων \\
\tabto{4em} δηλῶσαι \\
\tabto{6em} μὴ μόνον \\
\tabto{8em} τοῖς τεταγμένοις ὀνόμασιν, \\
\tabto{6em} ἀλλὰ \\
\tabto{8em} τὰ μὲν ξένοις, \\
\tabto{8em} τὰ δὲ καινοῖς, \\
\tabto{8em} τὰ δὲ μεταφοραῖς,\\
\tabto{2em} καὶ μηδὲν \\
\tabto{4em} παραλιπεῖν, \\
\tabto{2em} ἀλλὰ \\
\tabto{6em} πᾶσιν τοῖς εἴδεσιν \\
\tabto{4em} διαποικῖλαι \\
\tabto{6em} τὴν ποίησιν·

\noindent τοῖς δὲ \\
\tabto{2em} περὶ τοὺς λόγους \\
οὐδὲν ἔξεστιν \\
\tabto{2em} τῶν τοιούτων,\\
ἀλλ' ἀποτόμως \\
καὶ τῶν ὀνομάτων \\
\tabto{2em} τοῖς πολιτικοῖς \\
\tabto{4em} μόνον \\
καὶ τῶν ἐνθυμημάτων \\
\tabto{2em} τοῖς περὶ αὐτὰς τὰς πράξεις \\
ἀναγκαῖόν ἐστιν \\
\tabto{2em} χρῆσθαι.\\

}

\begin{description}[noitemsep]
\item[Τοῖς μὲν γὰρ ποιηταῖς\dots\ τοῖς δὲ περὶ τοὺς λόγους] koordinacija (suprotstavljenih) rečeničnih članova parom čestica
\item[γὰρ] čestica uvodi obrazloženje: naime\dots
\item[Τοῖς ποιηταῖς] §~100
\item[πολλοὶ κόσμοι] §~196; §~82; LSJ κόσμος II.1
\item[δέδονται] δίδωμι davati; 3. l. pl. ind. perf. medpas.
\item[πλησιάζοντας] πλησιάζω drugovati, rekcija τινι s kime; a. pl. m. r. ptc. prez. akt.; ovisan o τοὺς θεοὺς
\item[τοὺς θεοὺς τοῖς ἀνθρώποις] §~82
\item[οἷόν τ'\dots] \textbf{καὶ διαλεγομένους\dots\ καὶ συναγωνιζομένους\dots\ καὶ περὶ τούτων\dots} koordinacija rečeničnih članova sastavnim veznicima τε i καί
\item[οἷόν τ' αὐτοῖς] fraza αὐτοῖς οἷόν ἐστι moguće im je, oni mogu; otvara mjesto dopuni u infinitivu; τ' = τε; §~103; §~207
\item[ποιῆσαι] ποιέω činiti, pjesnički prikazivati; inf. aor. akt., ovisan o οἷόν τ' αὐτοῖς
\item[διαλεγομένους καὶ συναγωνιζομένους] participi ovisni o τοὺς θεοὺς
\item[διαλεγομένους] διαλέγομαι τινί razgovarati s kime; a. pl. m. r. ptc. prez. medpas.
\item[συναγωνιζομένους] συναγωνίζομαι τινί skupa s kime boriti se; a. pl. m. r. ptc. prez. medpas.  
\item[οἷς] §~215; uvodi zavisnu odnosnu rečenicu koja ima službu objekta (ovisna o συναγωνιζομένους)
\item[οἷς ἂν βουληθῶσιν] hipotetička odnosna rečenica (eventualni oblik); §~486: s kojima god\dots
\item[βουληθῶσιν] βούλομαι željeti; 3. l. pl. konj. aor. pas.
\item[περὶ τούτων] §~213, ovisno o δηλῶσαι
\item[δηλῶσαι] ovisno o αὐτοῖς οἷόν ἐστι; δηλόω razlagati; inf. aor. akt.
\item[μὴ μόνον\dots\ ἀλλὰ\dots] koordinacija: ne samo\dots\ nego\dots; μή stoji uz particip kojim se izriče mogućnost (pogodbeno), Smyth 2728
\item[τεταγμένοις] τάττω redati; τεταγμένος uobičajen; d. pl. s. r. ptc. perf. medpas.
\item[ὀνόμασιν] §~123
\item[τὰ μὲν\dots\ τὰ δὲ\dots\ τὰ δὲ] koordinacija česticama: jedno\dots\ drugo\dots\ treće\dots
\item[τὰ μὲν ξένοις, τὰ δὲ καινοῖς] §~103
\item[μεταφοραῖς] §~90
\item[καὶ μηδὲν\dots\ ἀλλὰ\dots] koordinacija veznicima: i ništa\dots\ nego\dots
\item[μηδὲν] §~224; μή negacija uz infinitiv, §~509c
\item[παραλιπεῖν] παραλείπω izostavljati; inf. aor. akt. ovisan o αὐτοῖς οἷόν ἐστι
\item[πᾶσιν τοῖς εἴδεσιν] sc.\ κόσμων, vrstama ukrašavanja; §~193; §~153
\item[διαποικῖλαι] διαποικίλλω učiniti raznolikim, šarolikim; inf. aor. akt. ovisan o αὐτοῖς οἷόν ἐστι
\item[τὴν ποίησιν] §~165
\item[τοῖς δὲ περὶ τοὺς λόγους]  poimeničenje članom svih vrsta riječi i sintagma §~373; §~82; \textgreek[variant=ancient]{οἱ περὶ τοὺς λόγους} govornici, tj. prozaici, nadovezuje se na διὰ λόγους (ἐγκωμιάζειν) spomenuto ranije, u 8.\ poglavlju
\item[οὐδὲν] §~224
\item[ἔξεστιν] ἔξεστι (bezlično) slobodno je, dopušteno je, moguće je, složenica εἰμί; rekcija τινι (sc.\ \textgreek[variant=ancient]{τοῖς περὶ τοὺς λόγους}); 3. l. sg. ind. prez.
\item[τῶν τοιούτων] §~213.4; ovisno o οὐδὲν; dijelni genitiv §~395
\item[ἀλλ'] = ἀλλά
\item[ἀποτόμως] prilog: apsolutno, strogo, sc.\ \textgreek[variant=ancient]{ἀναγκαῖόν ἐστιν}
\item[τῶν ὀνομάτων] §~123; §~395; ovisno o \textgreek[variant=ancient]{τοῖς πολιτικοῖς}
\item[τοῖς πολιτικοῖς] §~103; ovdje u prenesenom značenju „uobičajenim, u uobičajenom smislu''; sinonim gornjeg τεταγμένοις; usp. LSJ s. v. πολῑτῐκός V
\item[μόνον] priložno: samo
\item[τῶν ἐνθυμημάτων] \textit{ovdje} misao, ideja; §~123; §~395; ovisno o \textgreek[variant=ancient]{τοῖς περὶ αὐτὰς τὰς πράξεις}, tj.\ ideje koje se odnose na same činjenice
\item[τοῖς περὶ αὐτὰς τὰς πράξεις] §~207; §~165; sc.\ ἐνθυμήμασι
\item[ἀναγκαῖόν] §~103
\item[ἐστιν] εἰμί biti; 3. l. sg. ind. prez.
\item[ἀναγκαῖόν ἐστιν] imenski predikat; bezlični izraz otvara mjesto nužnoj dopuni u infinitivu
\item[χρῆσθαι] χράομαί τινι služiti se čime, upotrebljavati što; inf. prez. medpas. ovisan o \textgreek[variant=ancient]{ἀναγκαῖόν ἐστιν}

\end{description}

%7


{\large
\noindent Πρὸς δὲ τούτοις \\
οἱ μὲν \\
\tabto{2em} μετὰ μέτρων καὶ ῥυθμῶν \\
ἅπαντα\\
ποιοῦσιν, \\
οἱ δ' \\
οὐδενὸς \\
\tabto{2em} τούτων \\
κοινωνοῦσιν·

\noindent ἃ τοσαύτην ἔχει χάριν \\
\tabto{2em} ὥστ',\\
\tabto{4em} ἂν καὶ τῇ λέξει \\
\tabto{6em} καὶ τοῖς ἐνθυμήμασιν \\
\tabto{4em} ἔχῃ κακῶς,\\
\tabto{2em} ὅμως \\
\tabto{4em} αὐταῖς ταῖς εὐρυθμίαις \\
\tabto{4em} καὶ ταῖς συμμετρίαις \\
\tabto{2em} ψυχαγωγοῦσιν \\
\tabto{4em} τοὺς ἀκούοντας.

}

\begin{description}[noitemsep]

\item[Πρὸς δὲ τούτοις] §~213.2; čestica δέ označava nadovezivanje na prethodni iskaz
\item[οἱ μὲν\dots\ οἱ δ'] jedni (pjesnici)\dots\ a drugi (prozaici)\dots; δ' = δέ
\item[μετὰ μέτρων καὶ ῥυθμῶν] §~430, §~82
\item[ἅπαντα] = πάντα, §~193
\item[ποιοῦσιν] ποιέω činiti, sastavljati; 3. l. pl. ind. prez. akt.
\item[οὐδενὸς] §~224, ovisno o κοινωνοῦσιν
\item[τούτων] §~213.2, ovisno o οὐδενὸς
\item[κοινωνοῦσιν] κοινωνέω τινός imati udjela u čemu; 3. l. pl. ind. prez. akt.
\item[ἃ] §~215; odnosna zamjenica uvodi zavisnu odnosnu rečenicu, antecedent je τούτων, odnosno \textgreek[variant=ancient]{ἅπαντα (μετὰ μέτρων καὶ ῥυθμῶν)}
\item[τοσαύτην] §~213.4; pokazna zamjenica najavljuje posljedičnu rečenicu (τοσαύτην\dots\ ὥστε\dots)
\item[ἔχει] ἔχω imati; 3. l. sg. ind. prez. akt.; ἃ\dots\ ἔχει sročnost §~361
\item[χάριν] §~165
\item[ὥστ'] = ὥστε, veznik uvodi posljedičnu rečenicu, predikat je \textgreek[variant=ancient]{ψυχαγωγοῦσιν}; §~473
\item[ἂν] = ἐάν, uvodi eventualnu pogodbenu rečenicu, predikat je \textgreek[variant=ancient]{ἔχῃ κακῶς}; §~476
\item[καὶ τῇ λέξει καὶ τοῖς ἐνθυμήμασιν] §~165; §~123; λέξις ovdje u značenju: stil, način izražavanja
\item[ἔχῃ κακῶς] ἔχω κακῶς loše ići, ovdje bezlično: loše ide, loše stoji (tj.\ pjesnicima); 3. l. sg. konj. prez. akt.
\item[αὐταῖς ταῖς εὐρυθμίαις καὶ ταῖς συμμετρίαις] §~207; §~90
\item[ψυχαγωγοῦσιν] ψυχαγωγέω τινί τινα osvajati čime koga, zavoditi čime koga; 3. l. pl. ind. prez. akt.
\item[τοὺς ἀκούοντας] ἀκούω slušati; a. pl. m. r. ptc. prez. akt. (poimeničenje članom svih vrsta riječi i sintagma §~373)% član zastupa posvojnu zamjenicu §~371 - ne slažem se
\end{description}


%kraj
