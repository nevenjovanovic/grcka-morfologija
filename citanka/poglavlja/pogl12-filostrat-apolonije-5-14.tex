%unesi korekture NČ 2019-08-24
\section*{O autoru}

Filostrat \textgreek[variant=ancient]{(Φλαούιος Φιλόστρατος,} lat.\ Flavius Philostratus): na temelju nepovezanih obavijesti u natuknicama bizantskog leksikona \textit{Suda} danas poznajemo četiri člana iste obitelji koje skupno nazivamo Filostrati. Autor \textit{Života Apolonija iz Tijane} \textgreek[variant=ancient]{(τὰ εἰς τὸν Τυανέα Ἀπολλώνιον)} sin je prvog Filostrata, Verova sina, sofista čije se nijedno djelo nije sačuvalo. 

Lucije Flavije Filostrat (``Atenjanin'' – obnašao je dvije visoke dužnosti u Ateni) dospio je, zahvaljujući sofističkom umijeću i vezama, na dvor Septimija Severa (145.–211). Careva ga je žena Julija Domna, kako sam tvrdi, potakla da napiše romansiranu biografiju Apolonija iz Tijane u osam knjiga, objavljenu nakon 217. Kasnije je, možda 242, dovršio \textit{Živote sofista} posvećene caru Gordijanu. Mogući je autor i djela \textit{O kultu heroja, O tjelovježbi, Pisma, Slike} i \textit{Neron}. 

Zbog opsega i žanrovske raznolikosti djela, stilističke virtuoznosti u kratkim rečenicama, sklonosti metafori i paradoksu, Filostrata smatraju vodećom kreativnom osobnošću u grčkoj književnosti carskoga razdoblja. Bio je cijenjen i u Bizantu i Renesansi.

\section*{O tekstu}

\textit{Život Apolonija iz Tijane} svojevrstan je biografski eksperiment u kojem se prožimaju aspekti tada omiljenih ``idealnih'' romana s njihovim pretečom, Ksenofonotovom \textit{Kirupedijom}. Filostrat je Apolonija, oko čijeg su se lika rano počele raspredati čudesne pripovijesti, želio od vrača uzdići do ranga novopitagorskog asketa i čudotvorca \textgreek[variant=ancient]{(θεῖος ἀνήρ).} Povezujući s aretalogijom motive bajkovitih romana o putovanjima, dijelovima svoje pripovijesti, poput onog o mudračevu boravku u Indiji, dao je orijentalan ugođaj, kakav je posebno privlačio Filostratovu moćnu zaštitnicu Juliju Domnu. 

U ovom se odlomku tematizira upotreba i učinak mita, odnosno fikcije, kod Ezopa, nasuprot obradi kod uglednih pjesnika.

%\newpage

\section*{Pročitajte naglas grčki tekst.}

Philostr.\ Vita Apollonii 5.14

%Naslov prema izdanju

\medskip

\begin{greek}
{\large
{ \noindent  Ἄρξαι δ' αὐτῶν τὸν Ἀπολλώνιον ὧδε ἐρόμενον τοὺς ἑταίρους ``ἔστι τι μυθολογία;'' 

\noindent ``νὴ Δί'”, εἶπεν ὁ Μένιππος ``ἥν γε οἱ ποιηταὶ ἐπαινοῦσι''. 

\noindent ``τὸν δὲ δὴ Αἴσωπον τί ἡγῇ;'' 

\noindent ``μυθολόγον'' εἶπε ``καὶ λογοποιὸν πάντα''. 

\noindent ``πότεροι δὲ σοφοὶ τῶν μύθων;'' 

\noindent ``οἱ τῶν ποιητῶν'', εἶπεν ``ἐπειδὴ ὡς γεγονότες ᾄδονται''. 

\noindent ``οἱ δὲ δὴ Αἰσώπου τί;'' 

\noindent ``βάτραχοι'' ἔφη ``καὶ ὄνοι καὶ λῆροι γραυσὶν οἷοι μασᾶσθαι καὶ παιδίοις''. 

\noindent ``καὶ μὴν'' ἔφη ``ἐμοὶ'' ὁ Ἀπολλώνιος, ``ἐπιτηδειότεροι πρὸς σοφίαν οἱ τοῦ Αἰσώπου φαίνονται· οἱ μὲν γὰρ περὶ τοὺς ἥρωας, ὧν ποιητικὴ πᾶσα ἔχεται, καὶ διαφθείρουσι τοὺς ἀκροωμένους, ἐπειδὴ ἔρωτάς τε ἀτόπους οἱ ποιηταὶ ἑρμηνεύουσι καὶ ἀδελφῶν γάμους καὶ διαβολὰς ἐς θεοὺς καὶ βρώσεις παίδων καὶ πανουργίας ἀνελευθέρους καὶ δίκας, καὶ τὸ ὡς γεγονὸς αὐτῶν ἄγει καὶ τὸν ἐρῶντα καὶ τὸν ζηλοτυποῦντα καὶ τὸν ἐπιθυμοῦντα πλουτεῖν ἢ τυραννεύειν ἐφ' ἅπερ οἱ μῦθοι, Αἴσωπος δὲ ὑπὸ σοφίας πρῶτον μὲν οὐκ ἐς τὸ κοινὸν τῶν ταῦτα ᾀδόντων ἑαυτὸν κατέστησεν, ἀλλ' ἑαυτοῦ τινα ὁδὸν ἐτράπετο, εἶτα, ὥσπερ οἱ τοῖς εὐτελεστέροις βρώμασι καλῶς ἑστιῶντες, ἀπὸ σμικρῶν πραγμάτων διδάσκει μεγάλα, καὶ προθέμενος τὸν λόγον ἐπάγει αὐτῷ τὸ πρᾶττε ἢ μὴ πρᾶττε, εἶτα τοῦ φιλαλήθους μᾶλλον ἢ οἱ ποιηταὶ ἥψατο.''

}
}
\end{greek}

\section*{Analiza i komentar}

%1

{\large
\begin{greek}
\noindent \underline{ Ἄρξαι} δ' αὐτῶν \underline{τὸν Ἀπολλώνιον} \\
\tabto{2em} ὧδε ἐρόμενον τοὺς ἑταίρους \\
\tabto{4em} ``ἔστι τι μυθολογία;'' \\
``νὴ Δί'”, εἶπεν ὁ Μένιππος \\
\tabto{2em} ``ἥν γε \\
\tabto{2em} οἱ ποιηταὶ ἐπαινοῦσι''. \\
``τὸν δὲ δὴ Αἴσωπον \\
\tabto{2em} τί ἡγῇ;''\\
``μυθολόγον'' εἶπε ``καὶ λογοποιὸν πάντα''. \\
``πότεροι δὲ σοφοὶ \\
\tabto{2em} τῶν μύθων;''\\
``οἱ τῶν ποιητῶν'', εἶπεν \\
\tabto{2em} ``ἐπειδὴ \\
\tabto{4em} ὡς γεγονότες \\
\tabto{2em} ᾄδονται''. \\
``οἱ δὲ δὴ Αἰσώπου τί;''\\
``βάτραχοι” ἔφη ``καὶ ὄνοι καὶ λῆροι \\
\tabto{2em} γραυσὶν οἷοι μασᾶσθαι \\
\tabto{2em} καὶ παιδίοις''.\\

\end{greek}
}

\begin{description}[noitemsep]
\item[Ἄρξαι] ἄρχω započinjati; inf. aor. akt.
\item[δ'] δ' = δὲ §~515.2; elizija §~68
\item[αὐτῶν] §~207; partitivni genitiv §~395
\item[τὸν Ἀπολλώνιον] §~82 
\item[ἐρόμενον] εἴρομαι pitati; a. sg. m. r. ptc. aor. med. 
\item[τοὺς ἑταίρους] §~82
\item[ἔστι τι] εἰμί biti, postojati; 3. l. sg. ind. prez. akt., o naglasku §~315 bilj. 2; §~217; za enklizu §~40. c
\item[μυθολογία] §~90
\item[Δί'] §~Δί'= Δία §~178; elizija §~68
\item[εἶπεν] λέγω govoriti; 3. l. sg. ind. aor. 
\item[ὁ Μένιππος] §~82
\item[ἥν] §~215
\item[γε] §~39.5; §~519.1 
\item[οἱ ποιηταὶ] §~100 
\item[ἐπαινοῦσι] ἐπαινέω hvaliti; 3. l. pl. ind. prez. akt.
\item[τὸν\dots\ Αἴσωπον] §~82
\item[δὴ] §~516.5
\item[τί ἡγῇ] §~217; ἡγέομαι misliti, držati; 2. l. sg. ind. prez. medpas.
\item[μυθολόγον\dots\ καὶ λογοποιὸν πάντα] §~82; §~193
\item[πότεροι] §~219
\item[σοφοὶ τῶν μύθων] §~103; partitivni genitiv §~395
\item[οἱ τῶν ποιητῶν] sc.\ μῦθοι; supstantiviranje članom §~373; §~108 
\item[ἐπειδὴ\dots\ ᾄδονται] §~468; ᾄδω pjevati, pjesmom kazivati; 3. l. pl. ind. prez. medpas.
\item[ὡς γεγονότες] sc.\ μῦθοι; ὡς kao, kao da (prilog); γίγνομαι postajati, dogoditi se; n. pl. m. r. ptc. perf. akt. 
\item[οἱ\dots\ Αἰσώπου] sc.\ μῦθοι; supstantiviranje članom §~373; §~82
\item[βάτραχοι\dots\ ὄνοι\dots\ λῆροι] §~82
\item[ἔφη] φημί reći; 3. l. sg. impf. akt.
\item[γραυσὶν] §~183
\item[οἷοι] §~494
\item[μασᾶσθαι] μασάομαι žvakati, grickati; inf. prez. medpas.
\item[παιδίοις] §~82

\end{description}

%2

{\large
\begin{greek}
\noindent ``καὶ μὴν'' ἔφη ``ἐμοὶ'' ὁ Ἀπολλώνιος, \\
``ἐπιτηδειότεροι \\
\tabto{2em} πρὸς σοφίαν \\
οἱ τοῦ Αἰσώπου \\
φαίνονται· \\
οἱ μὲν γὰρ \\
\tabto{2em} περὶ τοὺς ἥρωας, \\
\tabto{4em} ὧν ποιητικὴ πᾶσα ἔχεται, \\
καὶ διαφθείρουσι τοὺς ἀκροωμένους, \\
\tabto{2em} ἐπειδὴ ἔρωτάς τε ἀτόπους \\
\tabto{2em} οἱ ποιηταὶ ἑρμηνεύουσι \\
καὶ ἀδελφῶν γάμους \\
καὶ διαβολὰς ἐς θεοὺς \\
καὶ βρώσεις παίδων \\
καὶ πανουργίας ἀνελευθέρους \\
καὶ δίκας, \\
καὶ τὸ \\
\tabto{2em} ὡς γεγονὸς αὐτῶν \\
ἄγει \\
καὶ τὸν ἐρῶντα \\
καὶ τὸν ζηλοτυποῦντα \\
καὶ τὸν ἐπιθυμοῦντα \\
\tabto{2em} πλουτεῖν ἢ τυραννεύειν \\
ἐφ' ἅπερ \\
\tabto{2em} οἱ μῦθοι,\\

\end{greek}
}

\begin{description}[noitemsep]
\item[καὶ μὴν] §~515.4
\item[ἐμοὶ\dots\ φαίνονται] §~205; φαίνω pokazivati; 3. l. pl. ind. prez. medpas.
\item[ἐπιτηδειότεροι] §~197
\item[πρὸς σοφίαν] §~435.C.c.γ; §~90
\item[οἱ μὲν γὰρ] sc.\ \textgreek[variant=ancient]{οἱ τῶν ποιητῶν}
\item[οἱ μὲν\dots\ Αἴσωπος δὲ\dots] koordinacija rečeničnih članova parom čestica: a\dots
\item[περὶ τοὺς ἥρωας] §~43.C.c; §~180; §~184
\item[ὧν ]  sc.\ \textgreek[variant=ancient]{τῶν τῶν ποιητῶν μύθων}; §~215
\item[ποιητικὴ πᾶσα ] §~90; §~193
\item[ἔχεται] ἔχω držati; 3. l. sg. ind. prez. medpas.
\item[διαφθείρουσι] διαφθείρω uništavati; 3. l. pl. ind. prez. akt.
\item[τοὺς ἀκροωμένους] ἀκροάομαι slušati; a. pl. m. r. ptc. prez. medpas.
\item[ἐπειδὴ\dots\ ἑρμηνεύουσι] uzročni veznik §~468; ἑρμηνεύω opisivati, pripovijedati; 3. l. pl. ind. prez. akt.
\item[ἔρωτάς\dots\ ἀτόπους] §~123; složeni pridjevi s dva završetka §~106
\item[τε\dots\ καὶ\dots] koordinacija rečeničnih članova sastavnim veznicima, pri čemu je drugi član istaknutiji
\item[ἀδελφῶν γάμους] §~82; genitiv objektni §~394
\item[διαβολὰς ἐς θεοὺς] §~90; §~419.c.β; §~82
\item[βρώσεις παίδων] §~165; §~127; genitiv objektni §~394
\item[πανουργίας ἀνελευθέρους\dots\ δίκας] §~90; složeni pridjevi s dva završetka §~106
\item[τὸ ὡς γεγονὸς αὐτῶν] njihov privid zbilje; γίγνομαι bivati, dogoditi se, n. sg. sr. r. ptc. perf. akt.; ὡς: kao da, prilog u atributnom položaju §~375.5; αὐτῶν, sc.\ τῶν ποιητῶν 
\item[ἄγει ] ἄγω voditi; 3. l. sg. ind. prez. akt.
\item[τὸν ἐρῶντα ] ἐράω žudjeti, strastveno ljubiti; a. sg. m. r. ptc. prez. akt.
\item[τὸν ζηλοτυποῦντα] ζηλοτυπέω ljubomoran biti; a. sg. m. r. ptc. prez. akt.
\item[τὸν ἐπιθυμοῦντα] ἐπιθυμέω težiti, nastojati; a. sg. m. r. ptc. prez. akt. 
\item[πλουτεῖν ἢ τυραννεύειν ] πλουτέω bogat biti; inf. prez. akt.; \textgreek[variant=ancient]{τυραννεύω} (neograničeno) vladati; inf. prez. akt.; rastavni veznik §~514 a
\item[ἐφ' ἅπερ] sc.\ ἄγουσι οἱ μῦθοι upravo onamo kamo (vode) mitovi; §~436.c.a; §~215; §~216.3; nadovezuje se na \textgreek[variant=ancient]{τὸ ὡς γεγονὸς αὐτῶν ἄγει\dots}

\end{description}

%3


{\large
\begin{greek}
\noindent Αἴσωπος δὲ \\
\tabto{2em} ὑπὸ σοφίας \\
πρῶτον μὲν \\
\tabto{2em} οὐκ ἐς τὸ κοινὸν \\
\tabto{4em} τῶν ταῦτα ᾀδόντων \\
ἑαυτὸν κατέστησεν, \\
ἀλλ' ἑαυτοῦ \\
\tabto{2em} τινα ὁδὸν \\
\tabto{4em} ἐτράπετο, \\
εἶτα, \\
\tabto{2em} ὥσπερ οἱ τοῖς εὐτελεστέροις βρώμασι \\
\tabto{4em} καλῶς ἑστιῶντες, \\
ἀπὸ σμικρῶν πραγμάτων \\
\tabto{2em} διδάσκει μεγάλα, \\
καὶ προθέμενος τὸν λόγον \\
\tabto{2em} ἐπάγει αὐτῷ \\
\tabto{4em} τὸ πρᾶττε ἢ μὴ πρᾶττε, \\
εἶτα τοῦ φιλαλήθους μᾶλλον \\
\tabto{2em} ἢ οἱ ποιηταὶ ἥψατο.''\\


\end{greek}
}

\begin{description}[noitemsep]
\item[ὑπὸ σοφίας ] §~437.A.β; §~90
\item[πρῶτον μὲν] koordinacija se nastavlja s εἶτα\dots\ εἶτα\dots
\item[τὸ κοινὸν] §~103; §~379
\item[τῶν ταῦτα ᾀδόντων ] ᾄδω pjevati, pjesmom kazivati; g. pl. m. r. ptc. prez. akt.; §~213.2
\item[ἑαυτὸν κατέστησεν] nije se poistovjetio \textgreek[variant=ancient]{(οὐκ ἐς τὸ κοινὸν ἑαυτὸν κατέστησεν)} s onima koji su pjesmom kazivali \textgreek[variant=ancient]{(ᾀδόντων)} to \textgreek[variant=ancient]{(ταῦτα,} mitove); §~208; καθίστημι postaviti; 3. l. pl. ind. aor. akt.
\item[ἑαυτοῦ τινα ὁδὸν] §~208; §~217; §~82; §~83
\item[ἐτράπετο] \textgreek[variant=ancient]{τρέπομαι (τὴν ὁδὸν)} putem udariti, krenuti; 3. l. sg. ind. aor. med.
\item[ὥσπερ] §~519.2
\item[οἱ\dots\ καλῶς ἑστιῶντες] ἑστιάω hraniti, gostiti; n. pl. m. r. ptc. prez. akt.; supstantiviranje članom §~373; prilog u atributnom položaju §~375
\item[τοῖς εὐτελεστέροις βρώμασι] §~123; §~153; §~198; \textit{dativus instrumenti} §~414.1 
\item[ἀπὸ σμικρῶν πραγμάτων] ἀπὸ izriče sredstvo §~423.c.γ; §~103; §~123
\item[διδάσκει μεγάλα] sc.\ πράγματα; διδάσκω podučavati; 3. l. sg. ind. prez. akt.; §~196
\item[προθέμενος τὸν λόγον] προτίθεμαι dati postaviti, poslužiti; n. sg. m. r. ptc. aor. med.; §~82; τὸν λόγον priču
\item[ἐπάγει αὐτῷ] ἐπάγω pridodati; 3. l. sg. ind. prez. akt.; §~207
\item[τὸ πρᾶττε ἢ μὴ πρᾶττε] πράττω činiti; 2. l. sg. impt. prez. akt.; supstantiviranje članom §~373
\item[τοῦ φιλαλήθους] τὸ φιλάληθες istinoljubivost; §~153; supstantiviranje članom §~373
\item[μᾶλλον ἢ] §~204.3; §~514.1.b
\item[ἥψατο] ἅπτομαι prionuti, privržen biti; 3. l. sg. ind. aor. akt.

\end{description}

%kraj
