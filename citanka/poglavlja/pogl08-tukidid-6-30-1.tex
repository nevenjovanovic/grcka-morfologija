% redaktura NZ (ne slažem se s ὅσον πλοῦν i δυνάμει ῾Ελληνικῇ), manja redaktura NJ
%\section*{O autoru}



\section*{O tekstu}

Sedamnaeste godine Peloponeskog rata (415.\ pr.\ Kr.) izglasan je pohod Atenjana i saveznika protiv gradova na Siciliji (tzv.\ sicilska ekspedicija) te se golema flota od preko stotinu trirema i nekoliko tisuća ljudi sprema isploviti iz luke Pirej.

%\newpage

\section*{Pročitajte naglas grčki tekst.}

%Naslov prema izdanju

Thuc.\ Historiae 6.30

\medskip

{\large
\begin{greek}
\noindent Αὐτοὶ δ' ᾿Αθηναῖοι καὶ εἴ τινες τῶν ξυμμάχων παρῆσαν, ἐς τὸν Πειραιᾶ καταβάντες ἐν ἡμέρᾳ ῥητῇ ἅμα ἕῳ ἐπλήρουν τὰς ναῦς ὡς ἀναξόμενοι. ξυγκατέβη δὲ καὶ ὁ ἄλλος ὅμιλος ἅπας ὡς εἰπεῖν ὁ ἐν τῇ πόλει καὶ ἀστῶν καὶ ξένων, οἱ μὲν ἐπιχώριοι τοὺς σφετέρους αὐτῶν ἕκαστοι προπέμποντες, οἱ μὲν ἑταίρους, οἱ δὲ ξυγγενεῖς, οἱ δὲ υἱεῖς, καὶ μετ' ἐλπίδος τε ἅμα ἰόντες καὶ ὀλοφυρμῶν, τὰ μὲν ὡς κτήσοιντο, τοὺς δ' εἴ ποτε ὄψοιντο, ἐνθυμούμενοι ὅσον πλοῦν ἐκ τῆς σφετέρας ἀπεστέλλοντο. καὶ ἐν τῷ παρόντι καιρῷ, ὡς ἤδη ἔμελλον μετὰ κινδύνων ἀλλήλους ἀπολιπεῖν, μᾶλλον αὐτοὺς ἐσῄει τὰ δεινὰ ἢ ὅτε ἐψηφίζοντο πλεῖν· ὅμως δὲ τῇ παρούσῃ ῥώμῃ, διὰ τὸ πλῆθος ἑκάστων ὧν ἑώρων, τῇ ὄψει ἀνεθάρσουν. οἱ δὲ ξένοι καὶ ὁ ἄλλος ὄχλος κατὰ θέαν ἧκεν ὡς ἐπ' ἀξιόχρεων καὶ ἄπιστον διάνοιαν. παρασκευὴ γὰρ αὕτη πρώτη ἐκπλεύσασα μιᾶς πόλεως δυνάμει ῾Ελληνικῇ πολυτελεστάτη δὴ καὶ εὐπρεπεστάτη τῶν ἐς ἐκεῖνον τὸν χρόνον ἐγένετο.

\end{greek}

}

\section*{Analiza i komentar}


%1

{\large
\noindent Αὐτοὶ δ' ᾿Αθηναῖοι \\
καὶ \\
\tabto{2em} εἴ τινες τῶν ξυμμάχων παρῆσαν, \\
ἐς τὸν Πειραιᾶ καταβάντες \\
\tabto{2em} ἐν ἡμέρᾳ ῥητῇ \\
\tabto{2em} ἅμα ἕῳ \\
ἐπλήρουν \\
τὰς ναῦς \\
\tabto{2em} ὡς ἀναξόμενοι. \\

}

\begin{description}[noitemsep]

\item[Αὐτοὶ] §~207
\item[δ' ᾿Αθηναῖοι] §~68; §~103; čestica δέ povezuje rečenicu s (ovdje izostavljenom) prethodnom: a\dots
\item[εἴ τινες] §~40
\item[τινες] §~217
\item[τῶν ξυμμάχων] §~103, §~106
\item[παρῆσαν] πάρειμι biti prisutan; 3. l. pl. impf. (akt.)
\item[ἐς τὸν Πειραιᾶ ] §~419; §~175; §~177
\item[καταβάντες ] καταβαίνω sići; n. pl. m. r. ptc. aor. akt.
\item[ἐν ἡμέρᾳ ῥητῇ] §~426; §~90; §~103; §~376
\item[ἅμα ἕῳ ] §~413.1.c; §~112.3
\item[ἐπλήρουν ] πληρόω ispuniti; 3. l. pl. impf. akt.
\item[τὰς ναῦς ] §~180
\item[ὡς ἀναξόμενοι] §~503.3, ἀνάγω medpas.\ isploviti; n. pl. m. r. ptc. fut. med.
\end{description}

{\large
\noindent ξυγκατέβη δὲ \\
καὶ ὁ ἄλλος ὅμιλος \\
\tabto{2em} ἅπας \\
\tabto{4em} ὡς εἰπεῖν \\
\tabto{2em} ὁ ἐν τῇ πόλει \\
\tabto{4em} καὶ ἀστῶν καὶ ξένων, \\
\tabto{4em} οἱ μὲν ἐπιχώριοι \\
\tabto{6em} τοὺς σφετέρους αὐτῶν \\
\tabto{4em} ἕκαστοι προπέμποντες, \\
\tabto{6em} οἱ μὲν ἑταίρους, \\
\tabto{6em} οἱ δὲ ξυγγενεῖς, \\
\tabto{6em} οἱ δὲ υἱεῖς, \\
\tabto{4em} καὶ μετ' ἐλπίδος τε ἅμα ἰόντες \\
\tabto{6em} καὶ ὀλοφυρμῶν, \\
\tabto{8em} τὰ μὲν ὡς κτήσοιντο, \\
\tabto{8em} τοὺς δ' εἴ ποτε ὄψοιντο, \\
\tabto{4em} ἐνθυμούμενοι \\
\tabto{6em} ὅσον πλοῦν \\
\tabto{8em} ἐκ τῆς σφετέρας \\
\tabto{4em} ἀπεστέλλοντο.\\

}

\begin{description}[noitemsep]

\item[ξυγκατέβη] συγκαταβαίνω (atički ξυγκαταβαίνω) sići zajedno s (kime); 3. l. sg. ind. aor.
\item[δὲ] čestica povezuje rečenicu s prethodnom: a\dots
\item[ὁ ἄλλος ὅμιλος ἅπας] §~212.a; §~82; §~193; §~379
\item[ὡς εἰπεῖν] λέγω govoriti; inf. aor. akt.; apsolutna uporaba infinitiva §~496
\item[ὁ ἐν τῇ πόλει] priložna oznaka kao atribut, u jače istaknutome atributnom položaju (iza imenice uz ponovljen član) §~375; §~426; §~165
\item[ἀστῶν] §~82
\item[ξένων] §~82
\item[οἱ μὲν ἐπιχώριοι\dots] §~103; §~373; μὲν najavljuje koordinaciju, ali drugi koordinirani član dolazi dvije rečenice kasnije: \textgreek[variant=ancient]{οἱ δὲ ξένοι}
\item[τοὺς σφετέρους αὐτῶν ἕκαστοι] §~211.3, §~207; §~103
\item[προπέμποντες] προπέμπω ispraćati; n. pl. m. r. ptc. prez. akt.
\item[οἱ μὲν\dots, οἱ δὲ\dots, οἱ δὲ\dots] koordinacija: jedni\dots\ drugi\dots\ treći\dots
\item[ἑταίρους] §~82
\item[ξυγγενεῖς] §~194.2
\item[υἱεῖς] §~172
\item[μετ' ἐλπίδος τε\dots\ καὶ ὀλοφυρμῶν] §~68; §~40; §~430; §~123; §~82; koordinacija parom sastavnih veznika
\item[ἰόντες] εἶμι ići; n. pl. m. r. ptc. (prez. akt.)
\item[τὰ μὲν\dots, τοὺς δ'\dots] sc.\ τὰ ἐν Σικελίᾳ, odgovara osjećaju μετ’ ἐλπίδος: u nadi da će steći\dots; τοὺς δ' sc.\ τοὺς σφετέρους, odgovara osjećaju μετ’ ὀλοφυρμῶν: uz jadanje hoće li svoje ikad više vidjeti\dots
\item[ὡς κτήσοιντο] veznik uvodi namjernu rečenicu kojoj mjesto otvara ἐλπίδος §~470; κτάομαι steći; 3. l. pl. opt. fut. (med.)
\item[δ' εἴ ποτε] §~68; §~40; veznik εἰ upotrijebljen iza izraza bojazni, što je ovdje ὀλοφυρμῶν
\item[ὄψοιντο] ὁράω gledati, vidjeti; 3. l. pl. opt. fut. (med.)
\item[ἐνθυμούμενοι ] ἐνθυμέομαι brinuti se, razmišljati; n. pl. m. r. ptc. prez. (med.)
\item[ὅσον πλοῦν] §~219; §~107
\item[ἐκ τῆς σφετέρας] sc.\ πόλεως; §~424; §~103
\item[ἀπεστέλλοντο] ἀποστέλλω odašiljati; 3. l. pl. impf. medpas.
\end{description}

%3 

{\large
\noindent καὶ ἐν τῷ παρόντι καιρῷ, \\
\tabto{2em} ὡς ἤδη ἔμελλον \\
\tabto{6em} μετὰ κινδύνων \\
\tabto{4em} ἀλλήλους \\
\tabto{4em} ἀπολιπεῖν, \\
μᾶλλον αὐτοὺς \\
ἐσῄει \\
τὰ δεινὰ \\
ἢ ὅτε ἐψηφίζοντο \\
\tabto{2em} πλεῖν· \\
ὅμως δὲ \\
\tabto{2em} τῇ παρούσῃ ῥώμῃ, \\
\tabto{2em} διὰ τὸ πλῆθος ἑκάστων \\
\tabto{4em} ὧν ἑώρων, \\
\tabto{2em} τῇ ὄψει \\
\tabto{2em} ἀνεθάρσουν.\\

}

\begin{description}[noitemsep]

\item[ἐν τῷ παρόντι καιρῷ] §~426; πάρειμι biti prisutan; d. sg. m. r. ptc. prez. akt. („sadašnji''); §~82; §~375
\item[ὡς\dots\ ἔμελλον] veznik uvodi zavisnu vremensku rečenicu §~487; μέλλω namjeravati, spremati se – otvara mjesto nužnoj dopuni u infinitivu; 3. l. pl. impf. akt.
\item[μετὰ κινδύνων] §~430; §~82
\item[ἀλλήλους ] §~212
\item[ἀπολιπεῖν] ἀπολείπω napuštati; inf. aor. akt.
\item[μᾶλλον\dots\ ἢ\dots] §~204.3; koordinacija: više\dots\ nego\dots
\item[αὐτοὺς ] §~207
\item[ἐσῄει ] εἴσειμι pasti na pamet; 3. l. sg. impf. (akt.)
\item[τὰ δεινὰ] §~103; §~373
\item[ὅτε ἐψηφίζοντο] §~487; ψηφίζω med. glasovanjem odlučiti – otvara mjesto dopuni u infinitivu; 3. l. pl. impf. medpas.
\item[πλεῖν] πλέω ploviti; inf. prez. akt.
\item[ὅμως δὲ] izražava suprotnost u odnosu na prvi dio rečenice: ali ipak\dots
\item[τῇ παρούσῃ ῥώμῃ] πάρειμι biti prisutan; d. sg. ž. r. ptc. prez. (akt.); §~90; §~375
\item[διὰ τὸ πλῆθος ] §~428; §~153
\item[ἑκάστων] §~103
\item[ὧν ] §~215; uvodi zavisnu odnosnu rečenicu, upotrijebljeno umjesto τούτων, ἃ\dots: asimilacija relativa §~444
\item[ἑώρων] ὁράω gledati; 3. l. pl. impf. akt.
\item[τῇ ὄψει] §~165
\item[ἀνεθάρσουν] ἀναθαρσέω povratiti hrabrost; 3. l. pl. impf. akt.
\end{description}

%4

{\large
\noindent οἱ δὲ ξένοι \\
καὶ ὁ ἄλλος ὄχλος \\
κατὰ θέαν \\
ἧκεν \\
\tabto{2em} ὡς ἐπ' ἀξιόχρεων καὶ ἄπιστον διάνοιαν. \\

}

\begin{description}[noitemsep]

\item[οἱ\dots\ ξένοι] §~82
\item[δὲ] koordinacija rečenica; odgovara gornjem \textgreek[variant=ancient]{οἱ μὲν ἐπιχώριοι}
\item[ὁ ἄλλος ὄχλος ] §~212.a; §~82; §~375
\item[κατὰ θέαν ] §~429; §~90
\item[ἧκεν ] ἥκω doći; 3. l. sg. impf. akt.
\item[ὡς ἐπ'\dots\  διάνοιαν] usporedba: kao na\dots; §~436
\item[ἐπ' ἀξιόχρεων] §~68; §~111
\item[ἄπιστον] §~106
\item[διάνοιαν] §~90
\end{description}

%5

%\newpage

{\large
\noindent παρασκευὴ γὰρ αὕτη πρώτη ἐκπλεύσασα \\
\tabto{2em} μιᾶς πόλεως \\
\tabto{4em} δυνάμει ῾Ελληνικῇ \\
\tabto{4em} πολυτελεστάτη δὴ καὶ εὐπρεπεστάτη \\
\tabto{6em} τῶν ἐς ἐκεῖνον τὸν χρόνον \\
ἐγένετο.\\

}

\begin{description}[noitemsep]
\item[παρασκευὴ\dots\ αὕτη πρώτη] §~90; §~213.2; §~223
\item[γὰρ] čestica najavljuje iznošenje dokaza prethodne tvrdnje: naime\dots
\item[ἐκπλεύσασα ] ἐκπλέω isploviti; n. sg. ž. r. ptc. aor. akt.
\item[μιᾶς πόλεως ] §~224; §~165
\item[δυνάμει ] §~165
\item[῾Ελληνικῇ ] §~103
\item[πολυτελεστάτη\dots\ εὐπρεπεστάτη] §~197
\item[δὴ] čestica naglašava superlativ (vrlo često kod Tukidida): i to\dots
\item[τῶν ἐς ἐκεῖνον τὸν χρόνον] poimeničenje prijedložnog izraza članom §~373; §~419.1.b; §~213.3; §~82
\item[ἐγένετο] γίγνομαι dogoditi se; 3. l. sg. ind. aor. (med.)

\end{description}


%kraj
