% Redaktura NJ; unio korekture NZ 2019-08-10
%\section*{O autoru}



\section*{O tekstu}

Oponašajući Ikara, filozof Menip izrađuje krila i leti u nebo do bogova, gdje saznaje da je Zeus odlučio uništiti sve filozofe jer su beskorisni. U izabranom odlomku Zeus u Menipovu društvu sjeda uz svojevrsne prozorske otvore kroz koje sa zemlje pristižu ljudske molbe upućene bogovima, da ih sasluša. Dio molbi, one pravedne, Zeus pripušta na nebo, dok one bezbožne i neispunjive otpuhuje ne dozvoljavajući da se nebu uopće približe.

%\newpage

\section*{Pročitajte naglas grčki tekst.}

%Naslov prema izdanju
Luc.\ Icaromenippus 25.3

\medskip

{\large
\begin{greek}
\noindent Θυρίδες δὲ ἦσαν ἑξῆς τοῖς στομίοις τῶν φρεάτων ἐοικυῖαι πώματα ἔχουσαι, καὶ παρ' ἑκάστῃ θρόνος ἔκειτο χρυσοῦς. καθίσας οὖν ἑαυτὸν ἐπὶ τῆς πρώτης ὁ Ζεὺς καὶ ἀφελὼν τὸ πῶμα παρεῖχε τοῖς εὐχομένοις ἑαυτόν· εὔχοντο δὲ πανταχόθεν τῆς γῆς διάφορα καὶ ποικίλα. συμπαρακύψας γὰρ καὶ αὐτὸς ἐπήκουον ἅμα τῶν εὐχῶν. ἦσαν δὲ τοιαίδε, ``῏Ω Ζεῦ, βασιλεῦσαί μοι γένοιτο·'' ``῏Ω Ζεῦ, τὰ κρόμμυά μοι φῦναι καὶ τὰ σκόροδα·'' ``῏Ω θεοί, τὸν πατέρα μοι ταχέως ἀποθανεῖν·'' ὁ δέ τις ἂν ἔφη, ``Εἴθε κληρονομήσαιμι τῆς γυναικός,'' ``Εἴθε λάθοιμι ἐπιβουλεύσας τῷ ἀδελφῷ,'' ``Γένοιτό μοι νικῆσαι τὴν δίκην,'' ``Δὸς στεφθῆναι τὰ ᾿Ολύμπια.'' τῶν πλεόντων δὲ ὁ μὲν βορέαν εὔχετο ἐπιπνεῦσαι, ὁ δὲ νότον, ὁ δὲ γεωργὸς ᾔτει ὑετόν, ὁ δὲ γναφεὺς ἥλιον. Ἐπακούων δὲ ὁ Ζεὺς καὶ τὴν εὐχὴν ἑκάστην ἀκριβῶς ἐξετάζων οὐ πάντα ὑπισχνεῖτο, ἀλλ' ἕτερον μὲν ἔδωκε πατήρ, ἕτερον δ' ἀνένευσε· τὰς μὲν γὰρ δικαίας τῶν εὐχῶν προσίετο ἄνω διὰ τοῦ στομίου καὶ ἐπὶ τὰ δεξιὰ κατετίθει φέρων, τὰς δὲ ἀνοσίους ἀπράκτους αὖθις ἀπέπεμπεν ἀποφυσῶν κάτω, ἵνα μηδὲ πλησίον γένοιντο τοῦ οὐρανοῦ. 
\end{greek}

}

%\newpage


\section*{Analiza i komentar}


%1

{\large
\noindent Θυρίδες δὲ ἦσαν ἑξῆς \\
\tabto{2em}  τοῖς στομίοις \\
\tabto{4em} τῶν φρεάτων \\
\tabto{2em} ἐοικυῖαι\\
\tabto{2em}  πώματα ἔχουσαι, \\
καὶ παρ' ἑκάστῃ \\
θρόνος \\
\tabto{2em} ἔκειτο \\
χρυσοῦς.\\

}

\begin{description}[noitemsep]
\item[Θυρίδες] §~123
\item[ἦσαν ἑξῆς] imenski predikat, Smyth 909
\item[ἦσαν] εἰμί biti; 3. l. pl.\ impf.
\item[ἐοικυῖαι] ἔοικα činiti se, rekcija τινι nalikovati; n. pl.\ ž. r. ptc. perf. akt; §~90
\item[Θυρίδες\dots\ ἐοικυῖαι\dots\ ἔχουσαι] participi dopunjuju imenicu
\item[τοῖς στομίοις] §~82
\item[τῶν φρεάτων] §~123, §~128; ovisno o τοῖς στομίοις
\item[πώματα] §~123
\item[ἔχουσαι] ἔχω imati; n. pl.\ ž. r. ptc. prez. akt; §~90; adverbni particip može označavati bilo kakvu popratnu okolnost, te ἔχων, ἄγων, φέρων, λαβών i χρώμενος mogu značiti ``sa'', Smyth 2068.a
\item[παρ' ἑκάστῃ] = παρὰ ἑκάστῃ; §~90
\item[θρόνος\dots\ χρυσοῦς] §~82, §~107-110
\item[ἔκειτο] κεῖμαι ležati, \textit{o predmetu} stajati (služi kao pasiv τίθημι §~315.4, bilj. 4); 3. l. sg.\ impf. medpas. 
\end{description}

%\newpage


{\large
\noindent καθίσας οὖν ἑαυτὸν \\
\tabto{2em} ἐπὶ τῆς πρώτης \\
ὁ Ζεὺς\\
καὶ ἀφελὼν \\
\tabto{2em} τὸ πῶμα\\
παρεῖχε \\
\tabto{2em} τοῖς εὐχομένοις \\
\tabto{2em} ἑαυτόν.\\

}

\begin{description}[noitemsep]
\item[καθίσας] καθίζω posjedati; n. sg.\ m. r. ptc. aor. akt.
\item[ἑαυτὸν] §~208
\item[ἐπὶ τῆς πρώτης] sc.\ θυρίδος; §~90
\item[ὁ Ζεὺς]  §~178
\item[ἀφελὼν] ἀφαιρέω dignuti; n. sg.\ m. r. ptc. aor. akt.
\item[παρεῖχε] παρέχω izložiti, prepustiti, LSJ s.~v.\ II; 3. l. sg.\ impf. akt.
\item[τοῖς εὐχομένοις] εὔχομαι moliti se; d. pl.\ m. r. ptc. prez. medpas.; supstantiviranje participa članom §~499
\end{description}

%3 

{\large
\noindent εὔχοντο δὲ \\
\tabto{2em} πανταχόθεν τῆς γῆς \\
διάφορα καὶ ποικίλα.\\

}

\begin{description}[noitemsep]
\item[εὔχοντο] εὔχομαι moliti se; 3. l. pl.\ impf. medpas.
\item[δὲ] čestica povezuje rečenicu s prethodnom: a\dots
\item[τῆς γῆς] §~108
\item[διάφορα καὶ ποικίλα] §~82; akuzativ unutarnjeg objekta ili sadržaja §~385.2
\end{description}

%4

{\large
\noindent συμπαρακύψας γὰρ καὶ αὐτὸς \\
ἐπήκουον ἅμα \\
\tabto{2em} τῶν εὐχῶν.\\

}

\begin{description}[noitemsep]
\item[συμπαρακύψας] συμπαρακύπτω nagnuti se, zajedno se sagibati pored nečeg; n. sg.\ m. r. ptc. aor. akt.
\item[γὰρ] čestica iskazuje nastavak pripovijedanja: a\dots
\item[αὐτὸς] §~207
\item[ἐπήκουον] ἐπακούω τινός slušati nešto; 3. l. pl.\ impf. akt.
\item[ἅμα] sc.\ τῷ Διί
\item[τῶν εὐχῶν] §~90
\end{description}

%\newpage


%5

{\large
\noindent ἦσαν δὲ τοιαίδε.\\
``῏Ω Ζεῦ, \\
\tabto{2em} βασιλεῦσαί μοι γένοιτο·''\\
``῏Ω Ζεῦ, \\
\tabto{2em} \underline{τὰ κρόμμυά} μοι \underline{φῦναι} \\
\tabto{2em} \underline{καὶ τὰ σκόροδα}·''\\
``῏Ω θεοί, \\
\tabto{2em} \underline{τὸν πατέρα} μοι ταχέως \underline{ἀποθανεῖν}·''\\

}

\begin{description}[noitemsep]
\item[ἦσαν\dots\ τοιαίδε] sc.\ εὐχαί; imenski predikat, Smyth 909
\item[δὲ] čestica povezuje rečenicu s prethodnom: a\dots
\item[τοιαίδε] §~213
\item[βασιλεῦσαι] βασιλεύω vladati; inf. aor. akt. kao dopuna γένοιτο
\item[μοι] §~205
\item[γένοιτο] γίγνεταί τινι ostvariti se, uspjeti nekome, LSJ γίγνομαι A.I.3, glagol nepotpuna značenja otvara mjesto dopuni u infinitivu; 3. l. sg.\ opt. aor. akt; optativ izriče želju §~464
\item[τὰ κρόμμυά\dots\ καὶ τὰ σκόροδα] §~82
\item[φῦναι] φύω rasti; inf. aor. akt.; A+I ovisi o neizrečenom γένοιτο
\item[῏Ω θεοί] §~82
\item[τὸν πατέρα] §~146
\item[ἀποθανεῖν] ἀποθνήσκω umrijeti; inf. aor. akt; A+I ovisi o neizrečenom γένοιτο kao u prethodnim rečenicima
\end{description}

%6

{\large
\noindent ὁ δέ τις ἂν ἔφη, \\
``Εἴθε \\
\tabto{2em} κληρονομήσαιμι τῆς γυναικός,''\\
``Εἴθε \\
\tabto{2em} λάθοιμι ἐπιβουλεύσας \\
\tabto{4em} τῷ ἀδελφῷ,''\\
``Γένοιτό μοι \\
\tabto{2em} νικῆσαι τὴν δίκην,'' \\
``Δὸς \\
\tabto{2em} στεφθῆναι τὰ ᾿Ολύμπια.''\\

}

\begin{description}[noitemsep]
\item[ὁ δέ τις] označava promjenu subjekta: a netko drugi\dots
\item[τις] §~217
\item[ἂν ἔφη] φημί govoriti; 3. l. sg.\ impf; iterativno ἂν + imperfekt izriče ponavljanje u prošlosti, odgovara hrvatskom kondicionalu sadašnjem §~462
\item[Εἴθε κληρονομήσαιμι] κληρονομέω τινός dobiti nasljedstvo od koga; 1. l. sg.\ opt. aor. akt.; optativ izriče želju §~464
\item[τῆς γυναικός] §~122
\item[λάθοιμι] λανθάνω ostati skriven, kopulativni glagol otvara mjesto predikatnom participu; 1. l. sg.\ opt. aor. akt; optativ izriče želju §~464: kad bi ostalo skriveno da ja\dots, kad bih bar neopazice\dots
\item[ἐπιβουλεύσας] ἐπιβουλεύω τινί raditi komu o glavi; n. sg.\ m. r. ptc. aor. akt; predikatni particip ovisan o λάθοιμι §~502
\item[τῷ ἀδελφῷ] §~82
\item[Γένοιτό μοι] γίγνεταί τινι usp.\ gore
\item[νικῆσαι] νικάω pobjeđivati, biti pobjednik (rezultativni prezent §~453.2); inf. aor. akt., služi kao dopuna riječi Γένοιτό
\item[τὴν δίκην] §~90; fraza \textgreek[variant=ancient]{νικάω τὴν δίκην} dobiti parnicu
\item[Δὸς] δίδωμι davati; 2. l. sg. impt. aor. akt.; otvara mjesto dopuni u infinitivu
\item[στεφθῆναι] στέφω ovjenčavati, pas. \textgreek[variant=ancient]{στέφομαι τί} pobijediti u nečemu; inf. aor. pas.
\item[τὰ ᾿Ολύμπια] akuzativ obzira ili, analogno s Ὀλύμπια νικᾶν, akuzativ unutrašnjeg objekta §~385.2; §~82; §~389
\end{description}

%7

{\large
\noindent τῶν πλεόντων δὲ \\
\tabto{2em} ὁ μὲν βορέαν εὔχετο ἐπιπνεῦσαι, \\
\tabto{2em} ὁ δὲ νότον, \\
\tabto{2em} ὁ δὲ γεωργὸς ᾔτει ὑετόν, \\
\tabto{2em} ὁ δὲ γναφεὺς ἥλιον.\\

}

\begin{description}[noitemsep] 
\item[τῶν πλεόντων] πλέω ploviti; g. pl.\ m. r. ptc. prez. akt.; dijelni genitiv, §~395; supstantiviranje participa članom §~499
\item[δὲ] čestica povezuje rečenicu s prethodnom: a\dots
\item[ὁ μὲν\dots] \textbf{ὁ δὲ\dots\ ὁ δὲ\dots\ ὁ δὲ\dots}\ koordinacija rečeničnih članova: jedan\dots\ drugi\dots\ a\dots\ a\dots
\item[βορέαν] §~100
\item[εὔχετο] εὔχομαι moliti se, otvara mjesto dopuni u infinitivu; 3. l. sg.\ impf. medpas.
\item[ἐπιπνεῦσαι] ἐπιπνέω puhati; inf. aor. akt.
\item[νότον] §~82
\item[γεωργὸς] §~82
\item[ᾔτει] αἰτέω tražiti; 3. l. sg.\ impf. akt.
\item[ὑετόν] §~82
\item[γναφεὺς] = κναφεύς; §~175
\item[ἥλιον] §~82

\end{description}

%8

{\large
\noindent Ἐπακούων δὲ ὁ Ζεὺς \\
καὶ τὴν εὐχὴν ἑκάστην \\
ἀκριβῶς \\
ἐξετάζων\\
\tabto{2em} οὐ πάντα ὑπισχνεῖτο,\\
ἀλλ' \\
\tabto{2em} ἕτερον μὲν ἔδωκε πατήρ,\\
\tabto{2em} ἕτερον δ' ἀνένευσε·\\

}

\begin{description}[noitemsep]
\item[Ἐπακούων] ἐπακούω slušati; n. sg.\ m. r. ptc. prez. akt.
\item[τὴν εὐχὴν ἑκάστην] §~90, §~103
\item[ἀκριβῶς] prilog od ἀκριβής; §~204
\item[ἐξετάζων] ἐξετάζω ispitivati; n. sg.\ m. r. ptc. prez. akt.
\item[πάντα] §~193
\item[ὑπισχνεῖτο] ὑπισχνέομαι obećavati; 3. l. sg.\ impf. medpas.
\item[ἀλλ'] = ἀλλά
\item[ἕτερον μὲν\dots\ ἕτερον δ'\dots] koordinacija rečeničnih članova parom čestica; riječi su citat Homera, Ilijada 16, 250
\item[ἕτερον] §~103
\item[ἔδωκε] δίδωμι davati; 3. l. sg.\ ind. aor. akt.
\item[ἀνένευσε] ἀνανεύω zanijekati; 3. l. sg.\ ind. aor. akt.
\end{description}

%9

{\large
\noindent τὰς μὲν γὰρ δικαίας \\
\tabto{2em} τῶν εὐχῶν \\
προσίετο ἄνω \\
\tabto{2em} διὰ τοῦ στομίου \\
καὶ \\
\tabto{2em} ἐπὶ τὰ δεξιὰ \\
κατετίθει φέρων,\\
τὰς δὲ ἀνοσίους ἀπράκτους \\
\tabto{2em} αὖθις ἀπέπεμπεν \\
\tabto{4em} ἀποφυσῶν \\
\tabto{2em} κάτω,\\
\tabto{4em} ἵνα μηδὲ πλησίον γένοιντο \\
\tabto{6em} τοῦ οὐρανοῦ.\\

}

\begin{description}[noitemsep]
\item[τὰς μὲν\dots\ τὰς δὲ\dots] koordinacija rečeničnih članova parom čestica
\item[γὰρ] čestica najavljuje iznošenje objašnjenja prethodne tvrdnje: naime\dots
\item[τὰς\dots\ δικαίας] §~103
\item[τῶν εὐχῶν] §~90, dijelni genitiv, §~395
\item[προσίετο] προσίημι med. pripuštati; 3. l. sg.\ impf. medpas.
\item[στομίου] §~82
\item[ἐπὶ τὰ δεξιὰ] na\dots\ (smjer); §~103 
\item[κατετίθει] κατατίθημι stavljati; 3. l. sg.\ impf. akt.
\item[φέρων] φέρω nositi; n. sg.\ m. r. ptc. prez. akt.
\item[τὰς\dots\ ἀνοσίους ἀπράκτους] §~103
\item[ἀπέπεμπεν] ἀποπέμπω odašiljati; 3. l. sg.\ impf. akt.
\item[ἀποφυσῶν] ἀποφυσάω otpuhivati; n. sg.\ m. r. ptc. prez. akt.
\item[ἵνα\dots] veznik uvodi zavisnu namjernu rečenicu (ovdje s optativom), §~470
\item[γένοιντο] γίγνομαι postati, kopulativni glagol otvara mjesto imenskoj dopuni; \textgreek[variant=ancient]{πλησίον τινὸς γίγνομαι} približiti se nečemu; 3. l. pl.\ opt. aor. med.
\item[πλησίον τοῦ οὐρανοῦ] blizu neba; πλησίον (nepravi) prijedlog izveden od πλήσιος §~103
\end{description}


%kraj
