% Unesi korekture NČ, 2019-08-18
\section*{O autoru}

Ahilej Tatije ili Tacije \textgreek[variant=ancient]{(Ἀχιλλεὺς Τάτıος)} autor je ljubavnog romana \textit{Zgode Leukipe i Klitofonta} \textgreek[variant=ancient]{(Tὰ κατὰ Λευκίππην καὶ Κλειτοφῶντα).} Živio je u 2.~st.\ po~Kr., vjerojatno u Aleksandriji. Navodi da se pod stare dane obratio na kršćanstvo i postao biskup najvjerojatnije nisu točni. Sudeći po broju sačuvanih papirusa, \textit{Zgode Leukipe i Klitofonta} bile su najpopularniji grčki ljubavni roman. Njegovi su glavni junaci dvoje mladih koji se ludo zaljube, ali ih okrutna sudbina razdvaja nizom nesretnih događaja poput brodoloma, otmice od strane gusara koji prodaju junake u roblje itd.

\section*{O tekstu}

Jedna od karakteristika grčkih ljubavnih romana jest izražen interes za osjećaje likova. Kad oteta Leukipa odbije udvaranje nasilnog Tersandra jer voli Klitofonta, Tersandar osjeća ljubav i bijes istovremeno. To je autoru povod za digresiju, kratko filozofsko razmatranje tih emocija i njihovog međusobnog odnosa. Ovakve digresije česte su u Tacijevu romanu. 

Odabrani se odlomak nalazi između Tersandrovog agresivnog pokušaja zavođenja i svađe koja uslijedi, a u kojoj ni Leukipa ni Tersandar ne biraju riječi.

\newpage

\section*{Pročitajte naglas grčki tekst.}

Ach. Tat. Leucippe et Clitophon 6.19.1

%Naslov prema izdanju

\medskip


{\large
{ 
\begin{greek}

\noindent Θυμὸς δὲ καὶ ἔρως δύο λαμπάδες· ἔχει γὰρ καὶ ὁ θυμὸς ἄλλο πῦρ, καὶ ἔστι τὴν μὲν φύσιν ἐναντιώτατον, τὴν δὲ βίαν ὅμοιον. ὁ μὲν γὰρ παροξύνει μισεῖν, ὁ δὲ ἀναγκάζει φιλεῖν· καὶ ἀλλήλων πάροικος ἡ τοῦ πυρός ἐστι πηγή. ὁ μὲν γὰρ εἰς τὸ ἧπαρ κάθηται, ὁ δὲ τῇ καρδίᾳ περιβέβληται.  ὅταν οὖν ἄμφω τὸν ἄνθρωπον καταλάβωσι, γίνεται μὲν αὐτοῖς ἡ ψυχὴ τρυτάνη, τὸ δὲ πῦρ ἑκατέρου ταλαντεύεται. μάχονται δὲ ἄμφω περὶ τῆς ῥοπῆς· καὶ τὰ πολλὰ μὲν ὁ ἔρως εἴωθε νικᾶν, ὅταν εἰς τὴν ἐπιθυμίαν εὐτυχῇ. ἢν δὲ αὐτὸν ἀτιμάσῃ τὸ ἐρώμενον, αὐτὸς τὸν θυμὸν εἰς συμμαχίαν καλεῖ.  κἀκεῖνος ὡς γείτων πείθεται, καὶ ἀνάπτουσιν ἄμφω τὸ πῦρ. ἂν δὲ ἅπαξ ὁ θυμὸς τὸν ἔρωτα παρ' αὑτῷ λάβῃ καὶ τῆς οἰκείας ἕδρας ἐκπεσόντα κατάσχῃ, φύσει γε ὢν ἄσπονδος, οὐχ ὡς φίλῳ πρὸς τὴν ἐπιθυμίαν συμμαχεῖ, ἀλλ' ὡς δοῦλον τῆς ἐπιθυμίας πεδήσας κρατεῖ· οὐκ ἐπιτρέπει δὲ αὐτῷ σπείσασθαι πρὸς τὸ ἐρώμενον, κἂν θέλῃ.  ὁ δὲ τῷ θυμῷ βεβαπτισμένος καταδύεται, καὶ εἰς τὴν ἰδίαν ἀρχὴν ἐκπηδῆσαι θέλων οὐκέτι ἐστὶν ἐλεύθερος, ἀλλὰ μισεῖν ἀναγκάζεται τὸ φιλούμενον. ὅταν δὲ ὁ θυμὸς καχλάζων γεμισθῇ καὶ τῆς ἐξουσίας ἐμφορηθεὶς ἀποβλύσῃ, κάμνει μὲν ἐκ τοῦ κόρου, καμὼν δὲ παρίεται, καὶ ὁ ἔρως ἀμύνεται καὶ ὁπλίζει τὴν ἐπιθυμίαν καὶ τὸν θυμὸν ἤδη καθεύδοντα νικᾷ. ὁρῶν δὲ τὰς ὕβρεις, ἃς κατὰ τῶν φιλτάτων ἐπαρῴνησεν, ἀλγεῖ καὶ πρὸς τὸ ἐρώμενον ἀπολογεῖται καὶ εἰς ὁμιλίαν παρακαλεῖ καὶ τὸν θυμὸν ἐπαγγέλλεται καταμαλάττειν ἡδονῇ.  τυχὼν μὲν οὖν ὧν ἠθέλησεν, ἵλεως γίνεται, ἀτιμούμενος δὲ πάλιν εἰς τὸν θυμὸν καταδύεται· ὁ δὲ καθεύδων ἐξεγείρεται καὶ τὰ ἀρχαῖα ποιεῖ· ἀτιμίᾳ γὰρ ἔρωτος σύμμαχός ἐστι θυμός.

\end{greek}
}
}

\vfill

\newpage

\section*{Analiza i komentar}

%1

{\large
\begin{greek}
\noindent Θυμὸς δὲ καὶ ἔρως \\
δύο λαμπάδες·\\

\end{greek}
}

\begin{description}[noitemsep]
\item[Θυμὸς ] §~82
\item[δὲ] usporedni suprotni veznik §~515.2, povezuje rečenicu s prethodnom
\item[ἔρως] §~123
\item[δύο λαμπάδες] imenski predikat, Smyth 909 (kopula je ovdje neizrečena)
\item[δύο] §~224
\item[λαμπάδες] §~123
\end{description}

%2


{\large
\begin{greek}
\noindent ἔχει γὰρ καὶ ὁ θυμὸς \\
\tabto{2em} ἄλλο πῦρ, \\
καὶ ἔστι \\
\tabto{2em} τὴν μὲν φύσιν ἐναντιώτατον, \\
\tabto{2em} τὴν δὲ βίαν ὅμοιον.\\

\end{greek}
}

\begin{description}[noitemsep]
\item[ἔχει] ἔχω imati; 3. l. sg. ind. prez. akt.
\item[γὰρ] postpozitivni usporedni uzročni veznik §~517
\item[ὁ θυμὸς] §~82
\item[ἄλλο] §~212
\item[πῦρ] §~146
\item[ἔστι\dots\ ὅμοιον] imenski predikat, Smyth 909
\item[ἔστι] εἰμί biti; 3. l. sg. ind. prez. akt.
\item[τὴν φύσιν] §~165, akuzativ obzira §~389 
\item[μὲν\dots\ δὲ\dots] koordinacija §~519.7: a\dots
\item[ἐναντιώτατον] §~103, §~197
\item[τὴν βίαν] §~90, akuzativ obzira  §~389 
\item[ὅμοιον] §~103

\end{description}

%3


{\large
\begin{greek}
\noindent ὁ μὲν γὰρ παροξύνει \\
\tabto{2em} μισεῖν, \\
ὁ δὲ ἀναγκάζει \\
\tabto{2em} φιλεῖν· \\
καὶ ἀλλήλων πάροικος \\
ἡ \\
\tabto{2em} τοῦ πυρός \\
ἐστι \\
\tabto{2em} πηγή.\\

\end{greek}
}

\begin{description}[noitemsep]
\item[ὁ μὲν\dots\ ὁ δὲ\dots] koordinacija, §~519.7: jedan\dots\ a drugi\dots
\item[γὰρ] postpozitivni usporedni uzročni veznik §~517
\item[παροξύνει] παροξύνω tjerati na nešto, poticati; 3. l. sg. ind. prez. akt; glagol otvara mjesto dopuni u infinitivu
\item[μισεῖν] μισέω mrziti; inf. prez. akt.
\item[ἀναγκάζει] ἀναγκάζω siliti; 3. l. sg. ind. prez. akt.
\item[φιλεῖν] φιλέω voljeti; inf. prez. akt.
\item[ἀλλήλων] §~212
\item[πάροικος] §~103, §~106
\item[ἡ\dots\ πηγή] §~90
\item[τοῦ πυρός] §~146
\item[ἐστι] εἰμί biti; 3. l. sg. ind. prez. akt.

\end{description}

%4


{\large
\begin{greek}
\noindent ὁ γὰρ \\
\tabto{2em} εἰς τὸ ἧπαρ \\
κάθηται, \\
ὁ δὲ \\
\tabto{2em} τῇ καρδίᾳ \\
περιβέβληται.\\

\end{greek}
}

\begin{description}[noitemsep]
\item[ὁ\dots\ ὁ δὲ\dots] koordinacija: jedan\dots\ a drugi\dots
\item[γὰρ] postpozitivni usporedni uzročni veznik §~517
\item[εἰς τὸ ἧπαρ] ἧπαρ, ᾰτος, τό; εἰς + a. §~419
\item[κάθηται] κάθημαι sjediti, biti skupljen; 3. l. sg. ind. prez. medpas.
\item[τῇ καρδίᾳ ] §~90
\item[περιβέβληται] περιβάλλω okretati, vrtjeti; 3. l. sg. ind. perf. medpas.

\end{description}

%5


{\large
\begin{greek}
\noindent ὅταν οὖν \\
\tabto{2em} ἄμφω \\
\tabto{2em} τὸν ἄνθρωπον \\
\tabto{2em} καταλάβωσι, \\
γίνεται μὲν \\
\tabto{2em} αὐτοῖς \\
ἡ ψυχὴ τρυτάνη, \\
τὸ δὲ πῦρ \\
\tabto{2em} ἑκατέρου \\
ταλαντεύεται.\\

\end{greek}
}

\begin{description}[noitemsep]
\item[ὅταν ] vremenski veznik, uvodi vremensku rečenicu §~487, §~488.2
\item[οὖν] usporedni zaključni veznik §~516.2
\item[ἄμφω] §~224.3
\item[τὸν ἄνθρωπον] §~82 
\item[καταλάβωσι] καταλαμβάνω spopasti, uhvatiti; 3. l. pl. konj. aor. akt.
\item[γίνεται] γίγνομαι postati, nastati; 3. l. sg. ind. prez. medpas., jonski i helenistički oblik; glagol otvara mjesto imenskoj dopuni
\item[μὲν\dots\ δὲ\dots] koordinacija §~519.7: a\dots
\item[αὐτοῖς ] §~207
\item[ἡ ψυχὴ] §~90
\item[τρυτάνη] §~90, dopuna uz γίνεται
\item[τὸ πῦρ] §~146
\item[ἑκατέρου] §~103, §~377.4
\item[ταλαντεύεται] ταλαντεύω pretegnuti; 3. l. sg. ind. prez. medpas.

\end{description}

%6


{\large
\begin{greek}
\noindent μάχονται δὲ ἄμφω \\
\tabto{2em} περὶ τῆς ῥοπῆς· \\
καὶ τὰ πολλὰ μὲν \\
ὁ ἔρως \\
εἴωθε \\
\tabto{2em} νικᾶν, \\
\tabto{2em} ὅταν \\
\tabto{4em} εἰς τὴν ἐπιθυμίαν \\
\tabto{2em} εὐτυχῇ.\\

\end{greek}
}

\begin{description}[noitemsep]
\item[μάχονται ] μάχομαι boriti se; 3. l. pl. ind. prez. medpas.
\item[δὲ] usporedni suprotni veznik §~515.2, povezuje rečenicu s prethodnom: a\dots
\item[ἄμφω ] §~224.3
\item[περὶ τῆς ῥοπῆς ] §~90, περὶ + g. §~433
\item[τὰ πολλὰ μὲν\dots\ ἢν δὲ αὐτὸν\dots] koordinacija rečenica parom čestica
\item[τὰ πολλὰ ] §~196, adverbni akuzativ §~391
\item[ὁ ἔρως ] §~123
\item[εἴωθε ] ἐθίζω privikavati, naviknuti se; 3. l. sg. ind. perf. akt. §~278.2, otvara mjesto dopuni u infinitivu
\item[νικᾶν ] νικάω pobijediti; inf. prez. akt.
\item[ὅταν ] vremenski veznik uvodi vremensku rečenicu §~487, §~488.2
\item[εἰς τὴν ἐπιθυμίαν ] §~90, εἰς + a. §~419
\item[εὐτυχῇ] εὐτυχέω postići cilj, ostvariti želju, 3. l. sg. konj. prez. akt.

\end{description}

%7


{\large
\begin{greek}
\noindent ἢν δὲ αὐτὸν \\
ἀτιμάσῃ \\
τὸ ἐρώμενον, \\
αὐτὸς \\
τὸν θυμὸν \\
\tabto{2em} εἰς συμμαχίαν \\
καλεῖ. \\

\end{greek}
}

\begin{description}[noitemsep]
\item[ἢν ] = ἐάν, pogodbeni veznik, uvodi pogodbenu rečenicu §~474
\item[αὐτὸν] §~207
\item[ἀτιμάσῃ ] ἀτιμάζω ne štovati, prezirati; 3. l. sg. konj. aor. akt.
\item[τὸ ἐρώμενον ] ἐράω žudjeti, čeznuti, voljeti; n. sg. sr. r. ptc. prez. medpas.; supstantiviranje participa članom §~499
\item[αὐτὸς ] §~207 
\item[τὸν θυμὸν ] §~82
\item[εἰς συμμαχίαν ] §~90 εἰς + a. §~419
\item[καλεῖ ] καλέω zvati; 3. l. sg. ind. prez. akt.

\end{description}

%8


{\large
\begin{greek}
\noindent κἀκεῖνος \\
\tabto{2em} ὡς γείτων \\
πείθεται, \\
καὶ ἀνάπτουσιν \\
ἄμφω \\
τὸ πῦρ.\\

\end{greek}
}

\begin{description}[noitemsep]
\item[κἀκεῖνος ] §~66 kraza, §~213.3
\item[ὡς γείτων ] §~131, §~221
\item[πείθεται] πείθω nagovarati; 3. l. sg. ind. prez. medpas.
\item[ἀνάπτουσιν ] ἀνάπτω zapaliti; 3. l. pl. ind. prez. akt.
\item[ἄμφω ] §~224.3
\item[τὸ πῦρ] §~146

\end{description}

%9


{\large
\begin{greek}
\noindent ἂν δὲ ἅπαξ \\
ὁ θυμὸς \\
τὸν ἔρωτα \\
\tabto{2em} παρ' αὑτῷ \\
λάβῃ \\
καὶ \\
\tabto{2em} τῆς οἰκείας \\
\tabto{2em} ἕδρας \\
ἐκπεσόντα \\
κατάσχῃ, \\
φύσει γε \\
ὢν \\
\tabto{2em} ἄσπονδος, \\
οὐχ \\
\tabto{2em} ὡς φίλῳ \\
\tabto{2em} πρὸς τὴν ἐπιθυμίαν \\
συμμαχεῖ, \\
ἀλλ' \\
\tabto{2em} ὡς δοῦλον \\
\tabto{2em} \tabto{2em} τῆς ἐπιθυμίας \\
πεδήσας \\
κρατεῖ· \\
οὐκ ἐπιτρέπει δὲ \\
\tabto{2em} αὐτῷ \\
\tabto{2em} σπείσασθαι \\
\tabto{2em} πρὸς τὸ ἐρώμενον, \\
κἂν θέλῃ.\\

\end{greek}
}

\begin{description}[noitemsep]
\item[ἂν] = ἐάν, §~474
\item[δὲ] usporedni suprotni veznik §~515.2, povezuje rečenicu s prethodnom: a\dots
\item[ἅπαξ ] §~223
\item[ὁ θυμὸς ] §~82
\item[τὸν ἔρωτα ] §~123
\item[παρ' αὑτῷ ] §~209 παρά + d. §~434B, elizija §~68 
\item[λάβῃ ] λαμβάνω uzeti, obuzeti; 3. l. sg. konj. aor. akt.
\item[τῆς οἰκείας ] §~103
\item[ἕδρας ] §~90
\item[ἐκπεσόντα ] ἐκπίπτω τινός ispadati iz čega, biti protjeran iz čega; a. sg. m. r. ptc. aor. akt.
\item[κατάσχῃ ] κατέχω držati, zadržavati; 3. l. sg. konj. aor. akt.
\item[φύσει ] §~165
\item[γε] čestica naglašava drugi dio izbora: ali\dots, međutim\dots
\item[ὢν ] εἰμί biti, n. m. r. sg. ptc. prez. akt.
\item[ἄσπονδος] §~103, §~106
\item[ὡς φίλῳ ] §~82, §~221
\item[πρὸς τὴν ἐπιθυμίαν ] §~90, πρὸς + a. §~435C
\item[συμμαχεῖ] συμμαχέω τινί komu u boju pomagati, zajedno s nekim se boriti; 3. l. sg. ind. prez. akt.
\item[ἀλλ' ὡς] elizija §~68
\item[ὡς δοῦλον] §~82, §~221
\item[τῆς ἐπιθυμίας] §~90
\item[πεδήσας ] πεδάω sputati; n. sg. m. r. ptc. aor. akt.
\item[κρατεῖ ] κρατέω τινός vladati nečim; 3. l. ind. prez. akt.
\item[ἐπιτρέπει] ἐπιτρέπω dopustiti; 3. l. sg. ind. prez. akt.
\item[δὲ] usporedni suprotni veznik §~515.2, povezuje surečenicu s prethodnima
\item[αὐτῷ ] §~207
\item[σπείσασθαι ] σπένδω med. sklopiti primirje; inf. aor. med.
\item[πρὸς τὸ ἐρώμενον ] ἐράω žudjeti, čeznuti, voljeti; a. sg. sr. r. ptc. prez. medpas.; supstantiviranje participa članom §~499; πρὸς + a. §~435.C
\item[κἂν ] = καὶ ἐάν, uvodi dopusnu rečenicu §~480, kraza §~66
\item[θέλῃ] θέλω (ἐθέλω) željeti; 3. l. sg. konj. prez. akt.

\end{description}

%10


{\large
\begin{greek}
\noindent ὁ δὲ \\
\tabto{2em} τῷ θυμῷ \\
βεβαπτισμένος \\
καταδύεται, \\
καὶ \\
\tabto{2em} εἰς τὴν ἰδίαν ἀρχὴν \\
\tabto{2em} ἐκπηδῆσαι \\
θέλων \\
οὐκέτι ἐστὶν ἐλεύθερος, \\
ἀλλὰ \\
\tabto{2em} μισεῖν \\
ἀναγκάζεται \\
\tabto{2em} τὸ φιλούμενον.\\

\end{greek}
}

\begin{description}[noitemsep]
\item[ὁ\dots\ βεβαπτισμένος ] βαπτίζω umočiti, potopiti; n. sg. m. r.  ptc. perf. medpas., supstantiviranje participa članom §~499
\item[δὲ] usporedni suprotni veznik §~515.2, povezuje rečenicu s prethodnom
\item[τῷ θυμῷ ] §~82
\item[καταδύεται] καταδύω potopiti; 3. l. sg. ind. prez. medpas.
\item[εἰς τὴν ἰδίαν ἀρχὴν ] §~90, §~103, εἰς + a. §~419
\item[ἐκπηδῆσαι ] ἐκπηδάω iskočiti, istrčati, provaliti; inf. aor. akt.
\item[θέλων ] θέλω (ἐθέλω) htjeti, n. sg. m. r. ptc. prez. akt. 
\item[ἐστὶν ] εἰμί biti, 3. l. sg. ind. prez. akt.
\item[ἐλεύθερος ] §~103
\item[ἀλλὰ] usporedni suprotni veznik §~515.1
\item[μισεῖν ] μισέω mrziti; inf. prez. akt.
\item[ἀναγκάζεται ] ἀναγκάζω siliti; 3. l. sg. ind. prez. medpas., otvara mjesto dopuni u infinitivu
\item[τὸ φιλούμενον] φιλέω voljeti; a. sg. sr. r. ptc. prez. medpas., supstantiviranje participa članom §~499

\end{description}

%11


{\large
\begin{greek}
\noindent ὅταν δὲ \\
ὁ θυμὸς \\
\tabto{2em} καχλάζων \\
γεμισθῇ \\
καὶ \\
\tabto{2em} τῆς ἐξουσίας \\
ἐμφορηθεὶς \\
ἀποβλύσῃ, \\
κάμνει μὲν \\
\tabto{2em} ἐκ τοῦ κόρου, \\
καμὼν δὲ \\
παρίεται, \\
καὶ ὁ ἔρως \\
ἀμύνεται \\
καὶ ὁπλίζει \\
τὴν ἐπιθυμίαν \\
καὶ \\
τὸν θυμὸν \\
\tabto{2em} ἤδη καθεύδοντα \\
νικᾷ.\\

\end{greek}
}

\begin{description}[noitemsep]
\item[ὅταν] vremenski veznik, uvodi vremensku rečenicu §~487, §~488.2
\item[ὁ θυμὸς ] §~82
\item[δὲ ] usporedni suprotni veznik §~515.2, povezuje rečenicu s prethodnom
\item[καχλάζων ] καχλάζω zapljuskivati, prskati; n. sg. m. r. ptc. prez. akt.
\item[γεμισθῇ ] γεμίζω napuniti; 3. l. sg. konj. aor. pas.
\item[τῆς ἐξουσίας ] §~90
\item[ἐμφορηθεὶς ] ἐμφορέω unositi; n. sg. m. r.  ptc. aor. pas.
\item[ἀποβλύσῃ] ἀποβλύζω izliti; 3. l. sg. konj. aor. akt.
\item[κάμνει ] κάμνω mučiti se; 3. l. sg. ind. prez. akt.
\item[κάμνει μὲν\dots\ καμὼν δὲ\dots] koordinacija rečeničnih članova česticama §~519.7: a\dots
\item[ἐκ τοῦ κόρου ] od sitosti, od obilja; §~82, ἐκ + g. §~424 
\item[καμὼν ] κάμνω mučiti se; n. sg. m. r. ptc. aor. akt.
\item[παρίεται] παρίημι popuštati, puštati; 3. l. sg. ind. prez. medpas.
\item[ὁ ἔρως ] §~123
\item[ἀμύνεται ] ἀμύνω odbijati; 3. l. sg. ind. prez. medpas.
\item[ὁπλίζει ] ὁπλίζω pripremati za bitku, naoružavati; 3. l. sg. ind. prez. akt.
\item[τὴν ἐπιθυμίαν ] §~90
\item[τὸν θυμὸν ] §~82
\item[καθεύδοντα ] καθεύδω spavati, počivati; a. sg. m. r. ptc. prez. akt.
\item[νικᾷ] νικάω pobjeđivati; 3. l. sg. ind. prez. akt.
\end{description}

%12


{\large
\begin{greek}
\noindent ὁρῶν δὲ \\
τὰς ὕβρεις, \\
\tabto{2em} ἃς \\
\tabto{4em} κατὰ τῶν φιλτάτων \\
\tabto{2em} ἐπαρῴνησεν, \\
ἀλγεῖ \\
καὶ \\
\tabto{2em} πρὸς τὸ ἐρώμενον \\
ἀπολογεῖται \\
καὶ \\
\tabto{2em} εἰς ὁμιλίαν \\
παρακαλεῖ \\
καὶ \\
τὸν θυμὸν \\
ἐπαγγέλλεται \\
\tabto{2em} καταμαλάττειν \\
\tabto{4em} ἡδονῇ.\\

\end{greek}
}

\begin{description}[noitemsep]
\item[ὁρῶν ] ὁράω vidjeti, gledati; n. sg. m. r. ptc. prez. akt.
\item[δὲ ] usporedni suprotni veznik §~515.2, povezuje rečenicu s prethodnom: a\dots
\item[τὰς ὕβρεις ] §~165
\item[ἃς] §~215, uvodi odnosnu rečenicu, §~481
\item[κατὰ τῶν φιλτάτων] §~197, bilj.\ 2; κατὰ + g. §~429.A
\item[ἐπαρῴνησεν] παροινέω τινά ponašati se prema nekome kao pijan čovjek, nanositi nekome zlo, loše postupati s kim; 3. l. sg. ind. aor. akt.
\item[ἀλγεῖ] ἀλγέω osjećati bol; 3. l. ind. prez. akt.
\item[πρὸς τὸ ἐρώμενον] ἐράω voljeti, žudjeti, čeznuti; a. sg. sr. r. ptc. prez. medpas., supstantiviranje participa članom §~499; πρὸς + a. §~435.C 
\item[ἀπολογεῖται ] ἀπολογέομαι πρός τινα braniti se pred nekim, opravdavati se pred nekim; 3. l. sg. ind. prez. medpas.
\item[εἰς ὁμιλίαν ] §~90, εἰς + a. §~419
\item[παρακαλεῖ ] παρακαλέω dozivati; 3. l. sg. ind. prez. akt.
\item[τὸν θυμὸν] §~82 
\item[ἐπαγγέλλεται ] ἐπαγγέλλω med. obećavati; 3. l. sg. ind. prez. medpas., otvara mjesto dopuni u infinitivu
\item[καταμαλάττειν ] καταμαλάττω udobrovoljiti, ublažiti; inf. prez. akt.
\item[ἡδονῇ  ] §~90, dativ sredstva §~414.1, dopuna uz καταμαλάττειν

\end{description}

%13


{\large
\begin{greek}
\noindent τυχὼν μὲν οὖν \\
\tabto{2em} ὧν ἠθέλησεν,\\
ἵλεως γίνεται, \\
ἀτιμούμενος δὲ \\
\tabto{2em} πάλιν \\
\tabto{2em} εἰς τὸν θυμὸν \\
καταδύεται· \\
ὁ δὲ καθεύδων \\
ἐξεγείρεται \\
καὶ τὰ ἀρχαῖα \\
ποιεῖ·\\
\tabto{2em} ἀτιμίᾳ γὰρ \\
ἔρωτος σύμμαχός ἐστι \\
θυμός.\\

\end{greek}
}

\begin{description}[noitemsep]
\item[τυχὼν ] τυγχάνω τινός pogađati, dobiti, postići nešto;  n. sg. m. r. ptc. aor. akt.
\item[μὲν…  δὲ] koordinacija §~519.7: a\dots
\item[οὖν] usporedni zaključni veznik §~516.2
\item[ὧν] §~215, uvodi odnosnu rečenicu §~481, partitivni genitiv uz glagole koji znače „gađati“ i „težiti“, „pogađati“ i „promašivati“ §~396 e
\item[ἠθέλησεν] ἐθέλω željeti; 3. l. sg. aor. akt.
\item[ἵλεως ] §~111
\item[γίνεται] γίγνομαι postati, nastati (jonski i helenistički oblik), 3. l. sg. ind. prez. medpas., otvara mjesto imenskoj dopuni
\item[ἀτιμούμενος ] ἀτιμόω osramotiti, pogrditi; n. sg. m. r. ptc. prez. medpas.
\item[εἰς τὸν θυμὸν ] §~82, εἰς + a. §~419
\item[καταδύεται ] καταδύω potopiti, med.\ potonuti, uvući se u što; 3. l. sg. ind. prez. medpas.
\item[ὁ δὲ] odnosi se na τὸν θυμὸν, član kao pokazna zamjenica §~370.2
\item[καθεύδων ] καθεύδω spavati; n. sg. m. r. ptc. prez. akt.
\item[ἐξεγείρεται ] ἐξεγείρω buditi; 3. l. sg. ind. prez. medpas.
\item[τὰ ἀρχαῖα ] §~103, supstantiviranje članom § 373
\item[ποιεῖ ] ποιέω činiti; 3. l. sg. ind. prez. akt.
\item[ἀτιμίᾳ] §~90 
\item[γὰρ ] postpozitivni usporedni uzročni veznik §~517
\item[ἔρωτος ] §~123
\item[σύμμαχός ] §~82
\item[ἐστι ] εἰμί biti, 3. l. sg. ind. prez. akt.
\item[θυμός] §~82

\end{description}

%kraj
