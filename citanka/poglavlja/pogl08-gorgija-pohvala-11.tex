% Unesene korekture NČ, NZ, NJ
\section*{O autoru}

Gorgija \textgreek[variant=ancient]{(Γοργίας),} grčki retor iz Leontina na Siciliji (oko 485.\ – oko 380.\ pr.~Kr), uz Protagoru najugledniji sofist. Kao poslanik Sirakuze 427.\ pr.~Kr.\ došao je u Atenu, gdje je osnovao i vodio školu govorništva; među njegovim učenicima bio je i Izokrat. Umro je u dubokoj starosti, u Tesaliji, na dvoru tiranina Jazona iz Fere.

Gorgija, veliki inovator retoričke prakse, svojim je artificijelnim i intelektualističkim stilom začetnik antičke umjetničke proze. Naglašena ritma, rima, asonancija, simetričnih članaka \textgreek[variant=ancient]{(κῶλα),} bogati antitezama, paralelizmima, metaforama i igrama riječi, njegovi tekstovi stoje između proze i poezije. U skladu sa sofističkim učenjima, Gorgija demonstrira neograničenu moć riječi, čak i onkraj racionalnog uvjeravanja.

\section*{O tekstu}

Uz nekoliko ulomaka, sačuvana su samo dva Gorgijina djela: \textit{Pohvala Helene} i \textit{Obrana Palameda}. U \textit{Pohvali Helene} Gorgija preuzima temu kojom se već bavio Stezihor (VII./VI.~st.\ pr.~Kr) u \textit{Palinodiji} (\textit{Opozivnoj pjesmi}), te dokazuje nevinost mitske ljepotice: njezin su odlazak u Troju prouzročili sudbina, bogovi, Parisove zavodljive riječi.

%\newpage

\section*{Pročitajte naglas grčki tekst.}
Gorg.\ Helenae encomium Fr.~11.~13
%Naslov prema izdanju

\medskip

{\large
\begin{greek}
\noindent Ὅτι μὲν οὖν φύσει καὶ γένει τὰ πρῶτα τῶν πρώτων ἀνδρῶν καὶ γυναικῶν ἡ γυνὴ περὶ ἧς ὅδε ὁ λόγος, οὐκ ἄδηλον οὐδὲ ὀλίγοις. δῆλον γὰρ ὡς μητρὸς μὲν Λήδας, πατρὸς δὲ τοῦ μὲν γενομένου θεοῦ, λεγομένου δὲ θνητοῦ, Τυνδάρεω καὶ Διός, ὧν ὁ μὲν διὰ τὸ εἶναι ἔδοξεν, ὁ δὲ διὰ τὸ φάναι ἠλέγχθη, καὶ ἦν ὁ μὲν ἀνδρῶν κράτιστος ὁ δὲ πάντων τύραννος. 

ἐκ τοιούτων δὲ γενομένη ἔσχε τὸ ἰσόθεον κάλλος, ὃ λαβοῦσα καὶ οὐ λαθοῦσα ἔσχε· πλείστας δὲ πλείστοις ἐπιθυμίας ἔρωτος ἐνειργάσατο, ἑνὶ δὲ σώματι πολλὰ σώματα συνήγαγεν ἀνδρῶν ἐπὶ μεγάλοις μέγα φρονούντων, ὧν οἱ μὲν πλούτου μεγέθη, οἱ δὲ εὐγενείας παλαιᾶς εὐδοξίαν, οἱ δὲ ἀλκῆς ἰδίας εὐεξίαν, οἱ δὲ σοφίας ἐπικτήτου δύναμιν ἔσχον· καὶ ἧκον ἅπαντες ὑπ' ἔρωτός τε φιλονίκου φιλοτιμίας τε ἀνικήτου. ὅστις μὲν οὖν καὶ δι' ὅτι καὶ ὅπως ἀπέπλησε τὸν ἔρωτα τὴν ῾Ελένην λαβών, οὐ λέξω· τὸ γὰρ τοῖς εἰδόσιν ἃ ἴσασι λέγειν πίστιν μὲν ἔχει, τέρψιν δὲ οὐ φέρει. τὸν χρόνον δὲ τῶι λόγωι τὸν τότε νῦν ὑπερβὰς ἐπὶ τὴν ἀρχὴν τοῦ μέλλοντος λόγου προβήσομαι, καὶ προθήσομαι τὰς αἰτίας, δι' ἃς εἰκὸς ἦν γενέσθαι τὸν τῆς ῾Ελένης εἰς τὴν Τροίαν στόλον.

\end{greek}

}

\section*{Analiza i komentar}

%1

{\large
\begin{greek}
\noindent Ὅτι μὲν οὖν \\
\tabto{2em} φύσει καὶ γένει \\
\tabto{2em} τὰ πρῶτα \\
\tabto{4em} τῶν πρώτων ἀνδρῶν καὶ γυναικῶν \\
ἡ γυνὴ \\
\tabto{2em} περὶ ἧς \\
\tabto{4em} ὅδε ὁ λόγος, \\
οὐκ ἄδηλον \\
\tabto{2em} οὐδὲ ὀλίγοις.\\

\end{greek}
}

\begin{description}[noitemsep]
\item[Ὅτι ] veznik uvodi zavisnu izričnu rečenicu; §~467
\item[μὲν οὖν ] kombinacija čestica označava prijelaz na novu misao: onda\dots
\item[φύσει καὶ γένει] §~165, §~153; dativus respectus, Smyth 1516
\item[τὰ πρῶτα ] §~373; §~223; sc.\ ἐστι (imenski predikat); slaže se (smislom) s ἡ γυνὴ
\item[τῶν πρώτων ἀνδρῶν καὶ γυναικῶν] §~149, §~223, §~122; genitiv partitivni uz kopulativne glagole §~396
\item[ἡ γυνὴ ] §~122
\item[περὶ ἧς ] §~433; §~215; prijedložni izraz s odnosnom zamjenicom (antecedent γυνὴ) uvodi zavisnu odnosnu rečenicu; ujedno je dio imenskog predikata (uz izostavljenu kopulu ἐστι)
\item[ὅδε ὁ λόγος] §~213, §~82
\item[ἄδηλον] §~106; izostavljena je kopula, \textgreek[variant=ancient]{οὐκ ἄδηλόν ἐστι;} ovo je glavna rečenica (zavisna je u inverziji); misao se iskazuje objema negacijama: \textgreek[variant=ancient]{οὐκ ἄδηλόν ἐστι οὐδὲ ὀλίγοις} (litota)
\item[ὀλίγοις] §~103

\end{description}

{\large
\begin{greek}
\noindent δῆλον γὰρ ὡς \\
\tabto{2em} μητρὸς μὲν \\
\tabto{4em} Λήδας, \\
\tabto{2em} πατρὸς δὲ \\
\tabto{4em} τοῦ μὲν γενομένου θεοῦ, \\
\tabto{4em} λεγομένου δὲ θνητοῦ, \\
\tabto{4em} Τυνδάρεω καὶ Διός, \\
\tabto{6em} ὧν \\
\tabto{6em} ὁ μὲν \\
\tabto{8em} διὰ τὸ εἶναι ἔδοξεν, \\
\tabto{6em} ὁ δὲ \\
\tabto{8em} διὰ τὸ φάναι ἠλέγχθη, \\
\tabto{4em} καὶ ἦν \\
\tabto{6em} ὁ μὲν \\
\tabto{8em} ἀνδρῶν κράτιστος \\
\tabto{6em} ὁ δὲ \\
\tabto{8em} πάντων τύραννος.\\

\end{greek}
}

\begin{description}[noitemsep]
\item[δῆλον ] §~103; izostavljena je kopula, δῆλόν ἐστι
\item[γὰρ ] čestica najavljuje iznošenje dokaza ili uzroka prethodne tvrdnje: naime\dots; §~517
\item[ὡς ] veznik uvodi zavisnu izričnu rečenicu; §~467
\item[μητρὸς μὲν\dots, πατρὸς δὲ] koordinacija pomoću (suprotnih) čestica  μὲν\dots\  δὲ\dots
\item[μητρὸς] sc.\ ἦν
\item[μητρὸς Λήδας] §~148; §~90; Λήδα, ας, ἡ iako je α impurum
\item[πατρὸς] §~148; genitiv subjektni ili posesivni, §~393
\item[τοῦ μὲν γενομένου\dots\, λεγομένου δὲ\dots] koordinacija pomoću (suprotnih) čestica  μὲν\dots\  δὲ\dots
\item[γενομένου] γίγνομαι postati, biti; g. sg. m. r. ptc. aor. (med.); ovdje ``pravog'', u suprotnosti s \textgreek[variant=ancient]{λεγομένου}
\item[θεοῦ] §~82
\item[λεγομένου] λέγω nazivati, imenovati; g. sg. n. ptc. prez. medpas.; ovdje ``navodnog'', u suprotnosti s \textgreek[variant=ancient]{γενομένου}
\item[θνητοῦ] §~103
\item[Τυνδάρεω καὶ Διός] §~111; §~178
\item[ὧν] §~215; partitivni genitiv: od kojih\dots; antecedent odnosne zamjenice imena su \textgreek[variant=ancient]{Τυνδάρεω καὶ Διός}
\item[ὁ μὲν\dots\ ὁ δὲ ] koordinacija pomoću (suprotnih) čestica  μὲν\dots\  δὲ\dots; §~370.1
\item[διὰ τὸ εἶναι] §~428; εἰμί biti; inf. prez. (akt.); supstantivirani infinitv u prijedložnom izrazu; §~373; suprotno διὰ τὸ φάναι
\item[ἔδοξεν] δοκεῖ τινί vrijediti za što (bezlično; za jednoga su verovali da je otac jer je to bio, a o drugom su govorili da je otac jer je to sam rekao); 3. l. ind. aor. akt.
\item[διὰ τὸ φάναι] sc.\ αὐτόν; §~428; φημί reći, inf. prez. akt; supstantivirani infinitiv u prijedložnom izrazu, §~373
\item[ἠλέγχθη] ἐλέγχω dokazati; 3. l. sg. ind. aor. pas; kasniji rukopisi donose prihvatljivije čitanje ἐλέχθη: o kojem se govorilo (da je otac) jer je tvrdio (da je otac)
\item[ἦν ] εἰμί biti; 3. l. sg. impf. (akt.)
\item[ὁ μὲν\dots\ ὁ δὲ\dots] koordinacija pomoću (suprotnih) čestica  μὲν\dots\  δὲ\dots; §~370.1
\item[ἀνδρῶν] §~149; partitivni genitiv ovisan o κράτιστος
\item[κράτιστος] §~202; dio imenskog predikata (uz ἦν)
\item[πάντων] §~193; objektni genitiv ovisan o τύραννος
\item[τύραννος] §~82; dio imenskog predikata (uz ἦν)

\end{description}

%3 itd
{\large
\begin{greek}
\noindent ἐκ τοιούτων δὲ \\
γενομένη \\
ἔσχε \\
τὸ ἰσόθεον κάλλος, \\
\tabto{2em} ὃ λαβοῦσα \\
\tabto{2em} καὶ οὐ λαθοῦσα \\
\tabto{2em} ἔσχε· \\
πλείστας δὲ \\
\tabto{2em} πλείστοις \\
ἐπιθυμίας \\
\tabto{2em} ἔρωτος \\
ἐνειργάσατο, \\
ἑνὶ δὲ σώματι \\
πολλὰ σώματα \\
συνήγαγεν \\
\tabto{2em} ἀνδρῶν \\
ἐπὶ μεγάλοις \\
\tabto{2em} μέγα φρονούντων, \\
\tabto{4em} ὧν \\
\tabto{4em} οἱ μὲν \\
\tabto{6em} πλούτου \\
\tabto{4em} μεγέθη, \\
\tabto{4em} οἱ δὲ \\
\tabto{6em} εὐγενείας παλαιᾶς \\
\tabto{4em} εὐδοξίαν, \\
\tabto{4em} οἱ δὲ \\
\tabto{6em} ἀλκῆς ἰδίας \\
\tabto{4em} εὐεξίαν, \\
\tabto{4em} οἱ δὲ \\
\tabto{6em} σοφίας ἐπικτήτου \\
\tabto{4em} δύναμιν \\
\tabto{4em} ἔσχον· \\
\tabto{4em} καὶ ἧκον \\
\tabto{4em} ἅπαντες \\
\tabto{6em} ὑπ' ἔρωτός τε φιλονίκου \\
\tabto{6em} φιλοτιμίας τε ἀνικήτου.\\

\end{greek}
}

\begin{description}[noitemsep]
\item[ἐκ τοιούτων ] §~424; §~213.2, sc.\ \textgreek[variant=ancient]{Τυνδάρεω καὶ Διός}
\item[δὲ ] čestica δέ povezuje rečenicu s prethodnom kao suprotni veznik: a\dots; §~515
\item[γενομένη ] γίγνομαι postati, roditi se; n. sg. ž. r. ptc. aor. (med.)
\item[ἔσχε ] ἔχω imati; 3. l. sg. ind. aor. akt.
\item[τὸ\dots\ κάλλος] §~153
\item[ἰσόθεον] §~106
\item[ὃ ] §~215; uvodi zavisnu odnosnu rečenicu, antecedent odnosne zamjenice je κάλλος
\item[λαβοῦσα] λαμβάνω dobiti; n. sg. ž. r. ptc. aor. akt.
\item[λαθοῦσα ] λανθάνω kriti; n. sg. ž. r. ptc. aor. akt; predikatni particip §~501: a kad ju je stekla sigurno nije bila tajna da je ima (veznik καί ovdje nije sastavno, nego emfatički upotrijebljen: a ne\dots); moderni se komentator čudi što je Gorgija ovdje, umjesto imperfekta, odabrao aorist
\item[ἔσχε] ἔχω imati; 3. l. sg. ind. aor. akt.
\item[πλείστας δὲ\dots\, ἑνὶ δὲ\dots] koordinacija pomoću čestice  δὲ (ima funkciju sastavnog veznika)
\item[πλείστας\dots\ ἐπιθυμίας ἔρωτος] §~202; §~90; §~123
\item[ἔρωτος] genitiv subjektni: ljubavne želje
\item[πλείστοις] §~202
\item[ἐνειργάσατο] ἐνεργάζομαι τινί τι proizvesti nešto u nekome; 3. l. sg. ind. aor. med.
\item[ἑνὶ σώματι] §~224; §~123; dativus instrumenti §~414.1
\item[πολλὰ σώματα\dots\ ἀνδρῶν] §~196; §~123; §~149
\item[συνήγαγεν] συνάγω okupiti (tj.\ u Sparti); 3. l. sg. ind. aor. akt.
\item[ἐπὶ μεγάλοις] §~436; §~196; o velikim stvarima, u pogledu velikih stvari
\item[μέγα φρονούντων] μέγα φρονέω biti hrabar, ohol, samouvjeren; g. pl. m. r. ptc. prez. akt.
\item[ὧν ] §~215; partitivni genitiv: od kojih\dots; antecedent odnosne zamjenice je ἀνδρῶν; uvodi zavisnu odnosnu rečenicu
\item[οἱ μὲν\dots\, οἱ δὲ\dots\, οἱ δὲ\dots\, οἱ δὲ\dots] koordinacija pomoću (suprotnih) čestica  μὲν\dots\  δὲ\dots\ δὲ\dots\ itd;  §~370.1; zajednički je predikat ἔσχον
\item[πλούτου μεγέθη] §~82; §~153
\item[εὐγενείας παλαιᾶς] §~90; §~103
\item[εὐδοξίαν] §~90
\item[ἀλκῆς ἰδίας] §~90; §~103
\item[εὐεξίαν] §~90
\item[σοφίας ἐπικτήτου] §~90; §~106
\item[δύναμιν ] §~165
\item[ἔσχον] ἔχω imati; 3. l. pl. ind. aor. akt.
\item[ἧκον ] ἥκω doći; 3. l. pl. impf. akt.
\item[ἅπαντες ] §~379
\item[ὑπ' ἔρωτός τε\dots\ φιλοτιμίας τε\dots] koordinacija pomoću (sastavnih) čestica  τε\dots\  τε\dots; §~513
\item[ὑπ' ἔρωτός φιλονίκου] §~68; §~123; §~437; §~106
\item[φιλοτιμίας ἀνικήτου] §~90; §~106

\end{description}

%4
{\large
\begin{greek}
\noindent ὅστις μὲν οὖν \\
\tabto{2em} καὶ δι' ὅτι \\
\tabto{2em} καὶ ὅπως \\
ἀπέπλησε τὸν ἔρωτα \\
\tabto{2em} τὴν ῾Ελένην λαβών, \\
οὐ λέξω· \\
τὸ γὰρ \\
\tabto{2em} τοῖς εἰδόσιν \\
\tabto{4em} ἃ ἴσασι \\
λέγειν \\
\tabto{2em} πίστιν μὲν ἔχει, \\
\tabto{2em} τέρψιν δὲ οὐ φέρει.\\

\end{greek}
}

\begin{description}[noitemsep]
\item[ὅστις μὲν\dots\ τὸν χρόνον δὲ\dots] koordinacija pomoću (suprotnih) čestica  μὲν\dots\  δὲ\dots
\item[ὅστις] §~217; odnosna zamjenica uvodi zavisnu odnosnu rečenicu, u inverziji; glavna rečenica je \textgreek[variant=ancient]{οὐ λέξω,} čiji je (neizrečeni) objekt antecedent odnosne zamjenice
\item[δι' ὅτι] §~68; §~428; §~217
\item[ἀπέπλησε ] ἀποπίμπλημι zadovoljiti; 3. l. sg. ind. aor. akt.
\item[τὸν ἔρωτα ] §~123
\item[τὴν ῾Ελένην ] §~90
\item[λαβών] λαμβάνω uzeti, oteti; n. sg. m. r. ptc. aor. akt.
\item[λέξω] λέγω govoriti; 1. l. sg. ind. fut. akt.
\item[τὸ\dots\ λέγειν ] λέγω govoriti; inf. prez. akt.; §~373; §~375
\item[γὰρ] čestica najavljuje iznošenje dokaza ili uzroka prethodne tvrdnje: naime\dots; §~517
\item[τοῖς εἰδόσιν ] οἶδα znam; d. pl. m. r. ptc. (prez. akt.); §~373
\item[ἃ ] §~215; odnosna zamjenica uvodi zavisnu odnosnu rečenicu, ujedno je objekt glagola λέγειν
\item[ἴσασι] οἶδα znam; 3. l. pl. ind. (prez. akt.)
\item[πίστιν μὲν\dots\, τέρψιν δὲ\dots] koordinacija pomoću (suprotnih) čestica  μὲν\dots\  δὲ\dots
\item[πίστιν] §~165; u retorici, πίστις kao tehnički termin označava uvjerljivost, τέρψις ugodu, dva cilja javnog govora
\item[ἔχει] ἔχω imati; 3. l. sg. ind. prez. akt.
\item[τέρψιν] §~165
\item[φέρει] φέρω nositi; 3. l. sg. ind. prez. akt.

\end{description}

%5

{\large
\begin{greek}
\noindent τὸν χρόνον δὲ \\
\tabto{2em} τῶι λόγωι \\
τὸν τότε \\
νῦν ὑπερβὰς \\
ἐπὶ τὴν ἀρχὴν \\
\tabto{2em} τοῦ μέλλοντος λόγου \\
προβήσομαι, \\
καὶ προθήσομαι \\
τὰς αἰτίας, \\
\tabto{2em} δι' ἃς \\
\tabto{2em} εἰκὸς ἦν \\
\tabto{4em} \underline{γενέσθαι} \\
\tabto{4em} \underline{τὸν} \\
\tabto{6em} τῆς ῾Ελένης \\
\tabto{6em} εἰς τὴν Τροίαν \\
\tabto{4em} \underline{στόλον}.\\

\end{greek}
}

\begin{description}[noitemsep]
\item[τὸν χρόνον\dots\ τὸν τότε] §~82; prilog kao atribut §~375: tadašnje
\item[τῶι λόγωι] §~82; \textit{iota adscriptum}, §~8; dativ izriče sredstvo
\item[ὑπερβὰς] ὑπερβαίνω preskočiti; n. sg. m. r. ptc. aor. akt.
\item[ἐπὶ τὴν ἀρχὴν ] §~436; §~90
\item[τοῦ\dots\ λόγου] §~82; §~375
\item[μέλλοντος] μέλλω bit ću; g. sg. m. r. ptc. prez. μέλλων budući
\item[προβήσομαι] προβαίνω napredovati; 1. l. sg. ind. fut. (med.)
\item[προθήσομαι ] προτίθημι med. prikazati; 1. l. sg. ind. fut. med.
\item[τὰς αἰτίας] §~90
\item[δι' ἃς ] §~68; §~428; §~215; odnosna zamjenica (u prijedložnom izrazu) uvodi zavisnu odnosnu rečenicu; antecedent je \textgreek[variant=ancient]{αἰτίας}
\item[εἰκὸς ] §~153; εἰκός ἐστι razumljivo je, shvatljivo je
\item[ἦν ] εἰμί biti; 3. l. sg. impf. (akt.)
\item[γενέσθαι ] γίγνομαι postati, doći do nečega; inf. aor. (med.)
\item[τὸν\dots\ στόλον] §~82; §~375
\item[τῆς ῾Ελένης ] §~90
\item[εἰς τὴν Τροίαν] §~419; §~90

\end{description}



%kraj
