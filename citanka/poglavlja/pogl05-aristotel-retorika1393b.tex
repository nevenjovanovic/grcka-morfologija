% 1. redaktura NJ, 2019-04-20; unesi NZ 2019-08-07
%\section*{O autoru}



\section*{O tekstu}

U drugoj knjizi \textit{Retorike} (djelo je nastajalo tijekom Aristotelovih dvaju boravaka u Ateni, 367. – 347.\ pr.~Kr., kad ga je Platon primio u Akademiju, i 335. – 322. pr.~Kr., kad je Aristotel vodio vlastitu školu, Licej) Aristotel izlaže načine kojima govornik može uvjeriti slušaoce i preduvjete za takvo uvjeravanje. Pošto je prikazao emocije i različitost emocija ovisno o dobi i društvenom statusu slušalaca, u 20.\ poglavlju govori o takozvanim τόποι, \textit{općim mjestima} kojima se dokazuje da se što može ili ne može dogoditi. U takve općenite dokaze, uz logičke, ulaze i primjeri \textgreek[variant=ancient]{(παραδείγματα);} njihova su posebna podvrsta pripovijesti \textgreek[variant=ancient]{(λόγοι),} poput prispodoba i basna. 

Ovaj tekst donosi Aristotelov primjer za primjenu basne u uvjeravanju; primjer je i sam priča, o Stezihoru \textgreek[variant=ancient]{(Στησίχορος),} grčkome korskom pjesniku arhajskog razdoblja (oko 640.\ – oko 555.\ pr.~Kr.), koji je živio u Himeri na Siciliji, i o Falaridu \textgreek[variant=ancient]{(Φάλαρις),} tiraninu sicilskog Akraganta između 570.\ i 555.\ pr.~Kr., vladaru čuvenom po okrutnosti.

\newpage

\section*{Pročitajte naglas grčki tekst.}

Arist.\ Rhetorica 1393b 10
%Naslov prema izdanju

\medskip

{\large
\begin{greek}
\noindent Στησίχορος μὲν γὰρ ἑλομένων στρατηγὸν αὐτοκράτορα τῶν ῾Ιμεραίων Φάλαριν καὶ μελλόντων φυλακὴν διδόναι τοῦ σώματος, τἆλλα διαλεχθεὶς εἶπεν αὐτοῖς λόγον ὡς ἵππος κατεῖχε λειμῶνα μόνος, ἐλθόντος δ' ἐλάφου καὶ διαφθείροντος τὴν νομὴν βουλόμενος τιμωρήσασθαι τὸν ἔλαφον ἠρώτα τινὰ ἄνθρωπον εἰ δύναιτ' ἂν μετ' αὐτοῦ τιμωρήσασθαι τὸν ἔλαφον, ὁ δ' ἔφησεν, ἐὰν λάβῃ χαλινὸν καὶ αὐτὸς ἀναβῇ ἐπ' αὐτὸν ἔχων ἀκόντια· συνομολογήσας δὲ καὶ ἀναβάντος ἀντὶ τοῦ τιμωρήσασθαι αὐτὸς ἐδούλευσε τῷ ἀνθρώπῳ. ``οὕτω δὲ καὶ ὑμεῖς'', ἔφη, ``ὁρᾶτε μὴ βουλόμενοι τοὺς πολεμίους τιμωρήσασθαι τὸ αὐτὸ πάθητε τῷ ἵππῳ· τὸν μὲν γὰρ χαλινὸν ἔχετε ἤδη, ἑλόμενοι στρατηγὸν αὐτοκράτορα· ἐὰν δὲ φυλακὴν δῶτε καὶ ἀναβῆναι ἐάσητε, δουλεύσετε ἤδη Φαλάριδι''.

\end{greek}
}

\section*{Analiza i komentar}

%1

{\large
\begin{greek}
\noindent Στησίχορος μὲν γὰρ \\
\tabto{2em} \uuline{ἑλομένων} στρατηγὸν αὐτοκράτορα \\
\tabto{2em} \uuline{τῶν ῾Ιμεραίων} \\
\tabto{3em} Φάλαριν \\
\tabto{2em} καὶ \uuline{μελλόντων} \\
\tabto{4em} φυλακὴν διδόναι \\
\tabto{6em} τοῦ σώματος, \\
τἆλλα διαλεχθεὶς \\
εἶπεν \\
αὐτοῖς \\
λόγον \\
\tabto{2em} ὡς \\
\tabto{2em} ἵππος \\
\tabto{4em} κατεῖχε \\
\tabto{4em} λειμῶνα \\
\tabto{2em} μόνος, \\
\tabto{2em} \uuline{ἐλθόντος} δ' \uuline{ἐλάφου} \\
\tabto{2em} καὶ \uuline{διαφθείροντος} τὴν νομὴν \\
\tabto{2em} βουλόμενος \\
\tabto{4em} τιμωρήσασθαι \\
\tabto{6em} τὸν ἔλαφον \\
\tabto{2em} ἠρώτα \\
\tabto{2em} τινὰ ἄνθρωπον \\
\tabto{4em} εἰ δύναιτ' ἂν \\
\tabto{6em} μετ' αὐτοῦ \\
\tabto{6em} τιμωρήσασθαι τὸν ἔλαφον, \\
\tabto{2em} ὁ δ' ἔφησεν, \\
\tabto{4em} ἐὰν λάβῃ \\
\tabto{6em} χαλινὸν \\
\tabto{4em} καὶ αὐτὸς \\
\tabto{4em} ἀναβῇ \\
\tabto{6em} ἐπ' αὐτὸν \\
\tabto{4em} ἔχων ἀκόντια· \\
\tabto{2em} συνομολογήσας δὲ \\
\tabto{4em} καὶ \uuline{ἀναβάντος} \\
\tabto{2em} \tabto{2em} ἀντὶ τοῦ τιμωρήσασθαι \\
\tabto{2em} \tabto{2em} αὐτὸς \\
\tabto{2em} ἐδούλευσε \\
\tabto{4em} τῷ ἀνθρώπῳ. \\

\end{greek}
}

\begin{description}[noitemsep]
\item[Στησίχορος ] §~90
\item[μὲν γὰρ] naime; kombinacija čestica osobito česta kod Aristotela
\item[ἑλομένων] αἱρέω med. izabrati; g. pl. m. r. ptc. aor. med.
\item[στρατηγὸν] §~82
\item[αὐτοκράτορα] §~146
\item[τῶν ῾Ιμεραίων ] §~103; §~373; stanovnici Himere, grčkog grada na Siciliji
\item[Φάλαριν] §~129
\item[μελλόντων] μέλλω namjeravati, glagol nepotpuna značenja otvara mjesto dopuni u infinitivu; g. pl. m. r. ptc. prez. akt.
\item[φυλακὴν ] §~90
\item[διδόναι ] δίδωμι dati; inf. prez. akt.
\item[τοῦ σώματος] §~123
\item[τἆλλα ] §~66; §~212; §~373
\item[διαλεχθεὶς ] διαλέγω (med.) raspravljati; n. m. r. sg. ptc. aor. (pas.)
\item[εἶπεν ] λέγω govoriti; 3. l. sg. ind. aor. akt.
\item[αὐτοῖς ] §~207
\item[λόγον ] §~82
\item[ἵππος ] §~82
\item[κατεῖχε ] κατέχω posjedovati; 3. l. sg. impf. akt.
\item[λειμῶνα ] §~131
\item[μόνος] §~103
\item[ἐλθόντος δ'] ἔρχομαι ići; g. sg. m. r. ptc. aor. akt.; čestica izražava suprotnost u odnosu na prethodnu surečenicu: a\dots
\item[δ' ἐλάφου ] §~68; §~82
\item[διαφθείροντος ] διαφθείρω pustošiti; g. sg. m. r. ptc. prez. akt.
\item[τὴν νομὴν ] §~90
\item[βουλόμενος ] βούλομαι željeti, glagol nepotpuna značenja otvara mjesto dopuni u infinitivu; n. sg. m. r. ptc. prez. (medpas.)
\item[τιμωρήσασθαι ] τιμωρέω med.\ τιμωρέομαί τινα osvetiti se komu; inf. aor. med.
\item[τὸν ἔλαφον ] §~82
\item[ἠρώτα] ἐρωτάω pitati, moliti (glagol otvara mjesto zavisnoj upitnoj rečenici); 3. l. sg. impf. akt.
\item[ἠρώτα τινὰ ] §~40
\item[ἄνθρωπον ] §~82
\item[ἠρώτα\dots\ εἰ\dots] veznik uvodi zavisnu upitnu rečenicu, §~469: pitao je\dots\ može li\dots
\item[δύναιτ' ἂν ] §~68; δύναμαι moći, glagol nepotpuna značenja otvara mjesto dopuni u infinitivu; 3. l. sg. opt. prez. (med.)
\item[μετ' αὐτοῦ ] §~68; §~430; §~207
\item[τιμωρήσασθαι ] za značenje, oblik i rekciju v.\ gore
\item[τὸν ἔλαφον] §~82
\item[ὁ δ'] §~370.2; član u kombinaciji s česticom δέ naznačuje promjenu subjekta
\item[δ' ἔφησεν] §~68
\item[ἔφησεν] φημί reći, \textit{ovdje} pristati; 3. l. sg. ind. aor. akt.
\item[ἔφησεν, ἐὰν λάβῃ ] zavisna pogodbena rečenica, kombinacija realne apodoze s eventualnom protazom, §~479; λαμβάνω uzeti; 3. l. sg. konj. aor. akt.
\item[χαλινὸν] §~82
\item[αὐτὸς ] §~207
\item[ἀναβῇ ] ἀναβαίνω uspeti se, \textit{ovdje} uzjahati; 3. l. sg. konj. aor. akt.
\item[ἐπ' αὐτὸν ] §~68; §~436; §~207
\item[ἔχων ] ἔχω imati, držati; n. sg. m. r. ptc. prez. akt.
\item[ἀκόντια] §~82
\item[συνομολογήσας ] συνομολογέω pristati, složiti se; n. sg. m. r. ptc. aor. akt.
\item[συνομολογήσας δὲ] čestica izražava suprotnost u odnosu na prethodnu surečenicu: a\dots
\item[ἀναβάντος] ἀναβαίνω uspeti se, uzjahati; g. sg. m. r. aor. akt.; participski dio GA (sc.\ τοῦ ἀνθρώπου)
\item[ἀντὶ τοῦ τιμωρήσασθαι ] §~422; za značenje, oblik i rekciju glagola v.\ gore; poimeničenje članom §~373
\item[αὐτὸς ] §~207
\item[ἐδούλευσε ] δουλεύω biti rob; 3. l. sg. ind. aor. akt.
\item[τῷ ἀνθρώπῳ] §~82

\end{description}


{\large
\begin{greek}
\noindent ``οὕτω δὲ καὶ ὑμεῖς'', \\
ἔφη, \\
``ὁρᾶτε \\
\tabto{2em} μὴ \\
\tabto{2em} βουλόμενοι \\
\tabto{4em} τοὺς πολεμίους τιμωρήσασθαι \\
\tabto{2em} τὸ αὐτὸ \\
\tabto{2em} πάθητε \\
\tabto{4em} τῷ ἵππῳ· \\
τὸν μὲν γὰρ χαλινὸν ἔχετε ἤδη, \\
ἑλόμενοι \\
\tabto{2em} στρατηγὸν αὐτοκράτορα· \\
ἐὰν δὲ \\
φυλακὴν δῶτε \\
καὶ ἀναβῆναι \\
\tabto{2em} ἐάσητε, \\
δουλεύσετε ἤδη \\
\tabto{2em} Φαλάριδι”.\\

\end{greek}
}

\begin{description}[noitemsep]
\item[οὕτω δὲ] čestica δέ povezuje rečenicu s prethodnom: a\dots
\item[ὑμεῖς] §~205
\item[ἔφη] φημί govoriti; 3. l. sg. impf. akt.
\item[ὁρᾶτε] ὁράω gledati, paziti da, otvara mjesto zavisnoj zahtjevnoj rečenici (veznik μή s konjunktivom); 2. l. pl. imp. prez. akt.
\item[μὴ\dots\ πάθητε ] πάσχω trpjeti, iskusiti, doživljavati; 2. l. pl. konj. aor. akt.; predikat zavisne zahtjevne rečenice: da ne\dots
\item[βουλόμενοι ] βούλομαι željeti, glagol nepotpuna značenja otvara mjesto dopuni u infinitivu; n. pl. m. r. ptc. prez. (medpas.)
\item[τοὺς πολεμίους ] §~103; §~373
\item[τιμωρήσασθαι] značenje, oblik i rekciju glagola v.~gore; dopuna uz βουλόμενοι
\item[τὸ αὐτὸ ] §~207; §~373
\item[τῷ ἵππῳ] §~82
\item[τὸν μὲν γὰρ χαλινὸν\dots\ ἐὰν δὲ φυλακὴν\dots] koordinacija objekata s pomoću čestica μὲν\dots\  δὲ\dots
\item[τὸν χαλινὸν ] §~82
\item[ἔχετε] ἔχω imati; 2. l. pl. ind. prez. akt.
\item[ἑλόμενοι ] αἱρέω med. izabrati; n. pl. m. r. ptc. aor. med.
\item[στρατηγὸν] §~82
\item[αὐτοκράτορα] §~146
\item[ἐὰν\dots\ δῶτε\dots\ ἐάσητε] ἐὰν = εἰ ἄν; veznik uvodi protazu zavisne pogodbene rečenice, ovdje eventualnog oblika (konjunktiv + ἄν u protazi, futur u apodozi), §~476; δίδωμι dati, 2. l. pl. konj. aor. akt.; ἐάω pustiti, dopustiti, glagol nepotpuna značenja otvara mjesto dopuni u infinitivu, 2. l. pl. konj. aor. akt.
\item[φυλακὴν] §~90
\item[ἀναβῆναι] ἀναβαίνω uspeti se, uzjahati; inf. aor. pas., dopuna ἐάσητε 
\item[δουλεύσετε] δουλεύω biti rob; 2. l. pl. ind. fut. akt.
\item[Φαλάριδι] §~123

\end{description}


%kraj
