% Unesi korekture NČ 2019-09-21
%\section*{O autoru}



\section*{O tekstu}

U \textit{Usporednim životopisima} Agesilajev je životopis u paru s Pompejevim. Agesilaj II. (oko 443.\ – 359./358.\ pr.~Kr.) bio je spartanski kralj (vladao 398.\ – 359./358.) i vrlo utjecajan grčki političar i vojskovođa. Niska rasta, kljast od rođenja, kraljem je postao ponešto neočekivano. Uspješno je ratovao u Maloj Aziji i tijekom Korintskog rata (395.\ – 387.). Agesilajev je prijatelj bio i povjesničar Ksenofont, koji je također sastavio njegov životopis. 

Ovdje odabrani odlomak prikazuje prvi Agesilajev pohod pošto je postao kralj, uz podršku spartanskog vojskovođe i admirala (i Agesilajeva ljubavnika) Lisandra (koji je 405.\ pobijedio Atenjane kod Egospotama, okončao Peloponeski rat, osigurao prevlast Sparte u Grčkoj).

Na Lisandrov nagovor, Agesilaj se 396.\ sprema prijeći u Aziju s vojskom sastavljenom od neodamoda (oslobođenih helota), saveznika i vrlo malog broja Spartanaca pod izlikom da želi grčke gradove u Aziji osloboditi perzijske vlasti.

%\newpage

\section*{Pročitajte naglas grčki tekst.}
Plut. Agesilaus 6.1
%Naslov prema izdanju

\medskip

{\large
\begin{greek}
\noindent Τοῦ δὲ Ἀγησιλάου τὴν βασιλείαν νεωστὶ παρειληφότος, ἀπήγγελλόν τινες ἐξ Ἀσίας ἥκοντες ὡς ὁ Περσῶν βασιλεὺς παρασκευάζοιτο μεγάλῳ στόλῳ Λακεδαιμονίους ἐκβαλεῖν τῆς θαλάσσης. ὁ δὲ Λύσανδρος ἐπιθυμῶν αὖθις εἰς Ἀσίαν ἀποσταλῆναι καὶ βοηθῆσαι τοῖς φίλοις, οὓς αὐτὸς μὲν ἄρχοντας καὶ κυρίους τῶν πόλεων ἀπέλιπε, κακῶς δὲ χρώμενοι καὶ βιαίως τοῖς πράγμασιν ἐξέπιπτον ὑπὸ τῶν πολιτῶν καὶ ἀπέθνησκον, ἀνέπεισε τὸν Ἀγησίλαον ἐπιθέσθαι τῇ στρατείᾳ καὶ προπολεμῆσαι τῆς Ἑλλάδος, ἀπωτάτω διαβάντα καὶ φθάσαντα τὴν τοῦ βαρβάρου παρασκευήν. ἅμα δὲ τοῖς ἐν Ἀσίᾳ φίλοις ἐπέστελλε πέμπειν εἰς Λακεδαίμονα καὶ στρατηγὸν Ἀγησίλαον αἰτεῖσθαι. παρελθὼν οὖν εἰς τὸ πλῆθος Ἀγησίλαος ἀνεδέξατο τὸν πόλεμον, εἰ δοῖεν αὐτῷ τριάκοντα μὲν ἡγεμόνας καὶ συμβούλους Σπαρτιάτας, νεοδαμώδεις δὲ λογάδας δισχιλίους, τὴν δὲ συμμαχικὴν εἰς ἑξακισχιλίους δύναμιν. συμπράττοντος δὲ τοῦ Λυσάνδρου πάντα προθύμως ἐψηφίσαντο, καὶ τὸν Ἀγησίλαον ἐξέπεμπον εὐθὺς ἔχοντα τοὺς τριάκοντα Σπαρτιάτας, ὧν ὁ Λύσανδρος ἦν πρῶτος, οὐ διὰ τὴν ἑαυτοῦ δόξαν καὶ δύναμιν μόνον, ἀλλὰ καὶ διὰ τὴν Ἀγησιλάου φιλίαν, ᾧ μεῖζον ἐδόκει τῆς βασιλείας ἀγαθὸν διαπεπρᾶχθαι τὴν στρατηγίαν ἐκείνην.

\end{greek}
}

\section*{Analiza i komentar}

%1

{\large
\begin{greek}
\noindent \uuline{Τοῦ δὲ Ἀγησιλάου} τὴν βασιλείαν νεωστὶ \uuline{παρειληφότος}, \\
ἀπήγγελλόν τινες ἐξ Ἀσίας ἥκοντες \\
\tabto{2em} ὡς ὁ Περσῶν βασιλεὺς παρασκευάζοιτο \\
\tabto{4em} μεγάλῳ στόλῳ \\
\tabto{4em} Λακεδαιμονίους ἐκβαλεῖν τῆς θαλάσσης.\\

\end{greek}
}

\begin{description}[noitemsep]
\item[Τοῦ Ἀγησιλάου ] §~82, §~374.2
\item[δὲ ] čestica povezuje ovu rečenicu s prethodnom: a\dots
\item[τὴν βασιλείαν] §~97
\item[παρειληφότος] παραλαμβάνω preuzeti; g. sg. m. r. ptc. perf. akt. 
\item[Τοῦ Ἀγησιλάου\dots] \textbf{παρειληφότος} GA ima vrijednost oznake vremena: pošto je\dots
\item[ἀπήγγελλόν ] ἀπαγγέλλω dojaviti; 3. pl. impf. akt.; službu objekta vrši izrična rečenica
\item[τινες] §~217; naglasak §~40
\item[ἐξ Ἀσίας] §~90, §~424
\item[ἥκοντες] ἥκω doći; n. pl. m. r. ptc. prez. akt.; τινες\dots\ ἥκοντες §~503 (Smyth 2054)
\item[ὡς ] uvodi izričnu rečenicu; u glavnoj je rečenici sporedno vrijeme §~467
\item[ὁ Περσῶν βασιλεὺς ] §~100, §~175
\item[παρασκευάζοιτο] παρασκευάζω + inf. spremati se, spreman biti; 3. l. sg. opt. prez. medpas.; optativ stoji zbog zavisne izrične rečenice §~467
\item[μεγάλῳ στόλῳ] §~196, §~82
\item[Λακεδαιμονίους ] §~82
\item[ἐκβαλεῖν ] ἐκβάλλω τινά τινός izbaciti, istjerati koga iz čega; inf. aor. akt.; objekt uz παρασκευάζοιτο §~493; Smyth 1989
\item[τῆς θαλάσσης] §~97

\end{description}

%8

{\large
\begin{greek}
\noindent ὁ δὲ Λύσανδρος \\
\tabto{2em} ἐπιθυμῶν αὖθις εἰς Ἀσίαν ἀποσταλῆναι \\
\tabto{4em} καὶ βοηθῆσαι τοῖς φίλοις, \\
\tabto{6em} οὓς αὐτὸς μὲν \\
\tabto{8em} ἄρχοντας καὶ κυρίους τῶν πόλεων \\
\tabto{6em} ἀπέλιπε, \\
\tabto{6em} κακῶς δὲ χρώμενοι καὶ βιαίως τοῖς πράγμασιν \\
\tabto{6em} ἐξέπιπτον \\
\tabto{8em} ὑπὸ τῶν πολιτῶν \\
\tabto{6em} καὶ ἀπέθνησκον, \\
ἀνέπεισε τὸν Ἀγησίλαον \\
\tabto{2em} ἐπιθέσθαι τῇ στρατείᾳ \\
\tabto{2em} καὶ προπολεμῆσαι τῆς  Ἑλλάδος, \\
ἀπωτάτω διαβάντα \\
καὶ φθάσαντα τὴν τοῦ βαρβάρου παρασκευήν.\\

\end{greek}
}

\begin{description}[noitemsep]
\item[ὁ\dots\ Λύσανδρος] §~82, §~374.2
\item[δὲ] čestica povezuje ovu rečenicu s prethodnom: a\dots
\item[ἐπιθυμῶν ] ἐπιθυμέω otvara mjesto dopuni u infinitivu: željeti učiniti što; n. sg. m. r. ptc. prez. akt.; ὁ Λύσανδρος\dots\ ἐπιθυμῶν §~503 (Smyth 2054)
\item[εἰς Ἀσίαν ] §~90, §~419
\item[ἀποσταλῆναι] ἀποστέλλω poslati; inf. aor. pas.; ovisi o ἐπιθυμῶν (što je Lisandar želio): da bude\dots\ §~493.2
\item[βοηθῆσαι] βοηθέω τινί doći u pomoć komu; inf. aor. akt.; ovisi o ἐπιθυμῶν (što je Lisandar želio) §~493.2
\item[τοῖς φίλοις] §~82, §~373
\item[οὓς ] §~215
\item[αὐτὸς ] §~207
\item[αὐτὸς μὲν\dots\ κακῶς δὲ χρώμενοι\dots] koordinacija rečeničnih članova s pomoću čestica: a\dots; §~515, §~519
\item[ἄρχοντας] §~139
\item[κυρίους τῶν πόλεων] §~82, §~165
\item[ἀπέλιπε] ἀπολείπω τινά τινα ostaviti za sobom koga kao što; 3. l. sg. ind. aor. akt.
\item[κακῶς χρώμενοι καὶ βιαίως] κακῶς χράομαι τοῖς πράγμασι loše upravljati; χρώμενοι n. pl. m. ptc. med.; βιαίως χράομαι τοῖς πράγμασι nasilno upravljati, zloupotrebljavati
\item[τοῖς πράγμασιν] §~123
\item[ἐξέπιπτον] ἐκπίπτω biti istjeran (doslovno: ispadati); 3. l. pl. impf. akt.
\item[ὑπὸ τῶν πολιτῶν] §~100, §~437
\item[ἀπέθνησκον] ἀποθνῄσκω umirati, biti pogubljen; 3. l. pl. impf. akt.
\item[ἀνέπεισε ] ἀναπείθω nagovoriti; 3. l. sg. ind. aor. akt.
\item[τὸν Ἀγησίλαον] §~82, §~374.2
\item[ἐπιθέσθαι ] ἐπιτίθημαι τινί staviti se na čelo čega, angažirati se u čemu; inf. aor. med.; ovisi o ἀνέπεισε (što je nagovorio Agesilaja) §~493.2
\item[τῇ στρατείᾳ] §~90
\item[προπολεμῆσαι ] προπολεμέω τινός voditi obrambeni rat za što; inf. aor. akt.; ovisi o predikatu ἀνέπεισε (što je nagovorio Agesilaja) §~493.2 
\item[τῆς  Ἑλλάδος] §~123
\item[διαβάντα] διαβαίνω prijeći (more); a. sg. m. r. ptc. aor. akt.; adverbijalni particip §~503
\item[φθάσαντα ] φθάνω preteći, preduhitriti; a. sg. m. r. ptc. aor. akt.; adverijalbni particip §~503
\item[τὴν τοῦ βαρβάρου παρασκευήν] §~82, §~90

\end{description}

%9

{\large
\begin{greek}
\noindent ἅμα δὲ τοῖς ἐν Ἀσίᾳ φίλοις ἐπέστελλε \\
\tabto{2em} πέμπειν εἰς Λακεδαίμονα \\
\tabto{2em} καὶ στρατηγὸν Ἀγησίλαον αἰτεῖσθαι.\\

\end{greek}
}

\begin{description}[noitemsep]
\item[δὲ] čestica povezuje ovu rečenicu s prethodnom: a\dots
\item[τοῖς\dots\ φίλοις] §~82, §~373
\item[ἐν Ἀσίᾳ ] §~90, §~426
\item[ἐπέστελλε] ἐπιστέλλω τινί τι: poručiti komu što; 3. l. sg. impf. akt.
\item[πέμπειν] πέμπω poslati (poslanstvo); inf. prez. akt. ovisi o predikatu ἐπέστελλε: da pošalje\dots\ §~493.2
\item[εἰς Λακεδαίμονα] §~131, §~419
\item[στρατηγὸν Ἀγησίλαον] §~82
\item[αἰτεῖσθαι] αἰτέω, \textit{ovdje} αἰτοῦμαι τινά τινα moliti, tražiti koga za koju dužnost; inf. prez. medpas.; ovisi o ἐπέστελλε: da\dots; §~493.2

\end{description}

%10

{\large
\begin{greek}
\noindent παρελθὼν οὖν εἰς τὸ πλῆθος \\
Ἀγησίλαος ἀνεδέξατο τὸν πόλεμον, \\
\tabto{2em} εἰ δοῖεν αὐτῷ \\
\tabto{4em} τριάκοντα μὲν ἡγεμόνας καὶ συμβούλους Σπαρτιάτας, \\
\tabto{4em} νεοδαμώδεις δὲ λογάδας δισχιλίους, \\
\tabto{4em} τὴν δὲ συμμαχικὴν εἰς ἑξακισχιλίους δύναμιν.\\

\end{greek}
}

\begin{description}[noitemsep]
\item[παρελθὼν] παρέρχομαι εἰς τι stupiti pred što; n. sg. m. r. ptc. aor. akt.
\item[εἰς τὸ πλῆθος] §~419, §~153
\item[ἀνεδέξατο] ἀναδείκνυμι πόλεμον objaviti rat; 3. l. sg. ind. aor. med.
\item[τὸν πόλεμον] §~82
\item[δοῖεν ] δίδωμι dati; 3. l. sg. opt. aor. akt.; optativ stoji jer je u glavnoj rečenici sporedno vrijeme, aorist
\item[αὐτῷ] §~207
\item[τριάκοντα] \textbf{μὲν\dots\ νεοδαμώδεις δὲ\dots\ τὴν δὲ συμμαχικὴν δύναμιν} koordinacija rečeničnih članova s pomoću čestica μὲν\dots\ δὲ\dots\ δὲ: \textit{a, b,} a \textit{c}\dots; §~515, §~519
\item[τριάκοντα] §~223
\item[ἡγεμόνας] §~131
\item[συμβούλους] §~82
\item[Σπαρτιάτας] §~100
\item[νεοδαμώδεις ] §~194, §~153
\item[λογάδας ] §~123
\item[δισχιλίους] §~223
\item[τὴν συμμαχικὴν\dots\ δύναμιν] §~103, §~165
\item[εἰς ἑξακισχιλίους ] §~223, §~419

\end{description}

%11

{\large
\begin{greek}
\noindent \uuline{συμπράττοντος δὲ τοῦ Λυσάνδρου} \\
πάντα προθύμως ἐψηφίσαντο, \\
καὶ τὸν Ἀγησίλαον ἐξέπεμπον εὐθὺς \\
\tabto{2em} ἔχοντα τοὺς τριάκοντα Σπαρτιάτας, \\
\tabto{4em} ὧν ὁ Λύσανδρος ἦν πρῶτος, \\
\tabto{4em} οὐ διὰ τὴν ἑαυτοῦ δόξαν καὶ δύναμιν μόνον, \\
\tabto{4em} ἀλλὰ καὶ διὰ τὴν Ἀγησιλάου φιλίαν, \\
\tabto{6em} ᾧ μεῖζον ἐδόκει τῆς βασιλείας ἀγαθὸν \\
\tabto{8em} \underline{διαπεπρᾶχθαι} \\
\tabto{10em} \underline{τὴν στρατηγίαν ἐκείνην}.\\

\end{greek}
}

\begin{description}[noitemsep]
\item[συμπράττοντος] συμπράττω pomagati, asistirati; g. sg. m. r. ptc. prez. akt.
\item[τοῦ Λυσάνδρου] §~82, §~374.2
\item[συμπράττοντος\dots\ τοῦ Λυσάνδρου] GA izriče uzrok: zato što\dots
\item[δὲ] čestica povezuje ovu rečenicu s prethodnom: a\dots
\item[πάντα] §~193
\item[ἐψηφίσαντο] ψηφίζομαι glasati; 3. l. pl. ind. aor. med.
\item[τὸν Ἀγησίλαον ] §~82, §~374.2
\item[ἐξέπεμπον] ἐκπέμπω odaslati, poslati na zadatak; 3. l. pl. impf. akt.
\item[εὐθὺς ] odmah (prilog)
\item[ἔχοντα ] ἔχω imati; a. sg. m. r. ptc. prez. akt.
\item[τοὺς τριάκοντα Σπαρτιάτας] §~223, §~100; §~375
\item[ὧν ] §~215
\item[ὁ Λύσανδρος ] §~82, §~374.2
\item[ἦν ] εἰμί biti; 3. l. sg. impf. akt.; kopula kao dio imenskog predikata Smyth 909
\item[πρῶτος] §~223
\item[οὐ\dots\ μόνον, ἀλλὰ καὶ\dots] ne samo\dots\ nego i\dots; koordinacija
\item[διὰ τὴν\dots\ δόξαν καὶ δύναμιν] §~97, §~165, §~428
\item[ἑαυτοῦ] §~208
\item[διὰ τὴν\dots\ φιλίαν] §~90, §~428
\item[Ἀγησιλάου] §~82
\item[ᾧ ] §~215
\item[μεῖζον\dots\ ἀγαθὸν ] §~200, §~103; poimeničeni pridjev §~373
\item[μεῖζον τῆς βασιλείας ἀγαθὸν] §~97, §~404
\item[ἐδόκει ] δοκέω činiti se, smatrati; δοκεῖ τινί τι nekomu se što čini kakvim; 3. l. sg. impf. akt.; glagol otvara mjesto obvezatnoj dopuni, ovdje A+I
\item[διαπεπρᾶχθαι] διαπράσσω, atički διαπράττω pribaviti; inf. perf. medpas.
\item[τὴν στρατηγίαν ἐκείνην] §~90, §~213.3

\end{description}


%kraj
