%\section*{O autoru}



\section*{O tekstu}

Na početku Izokratova govora \textit{Pohvala Helene} stoji duži uvod u kojem se autor osvrće na pomodnu, sofističku retoriku svojega vremena, zalažući se za govorništvo koje će odabirom vjerodostojnih tema i njihovim ozbiljnim tretmanom biti istinski korisno u izobrazbi mladih za građanske dužnosti. Sam panegirik Heleni kronološki slijedi njezin životopis: od priče o božanskom podrijetlu koje Helena vuče od Zeusa, preko Tezejeve otmice te povratka uz pomoć Kastora i Polideuka, do Parisova suda i otmice; nastavlja se pohvalom Helenine ljepote i stjecanja besmrtnosti, a završava njezinom simboličnom ulogom u ujedinjenju svih Grka i u njihovoj pobjedi nad barbarima. U ovdje odabranom odlomku prikazan je učinak Helenine ljepote na Tezeja, koji ju je, budući da je bila premlada za udaju, oteo i doveo iz Lakedemona u Atiku. Tim odlomkom započinje digresija u pohvalu Tezeja; on metaforički predstavlja veličinu i sjaj grada Atene.

%\newpage

\section*{Pročitajte naglas grčki tekst.}
Isoc. Helenae encomium 18
%Naslov prema izdanju

\medskip

{\large
\begin{greek}
\noindent Θησεὺς, ὁ λεγόμενος μὲν Αἰγέως γενόμενος δ' ἐκ Ποσειδῶνος, ἰδὼν αὐτὴν οὔπω μὲν ἀκμάζουσαν, ἤδη δὲ τῶν ἄλλων διαφέρουσαν, τοσοῦτον ἡττήθη τοῦ κάλλους, ὁ κρατεῖν τῶν ἄλλων εἰθισμένος, ὥσθ' ὑπαρχούσης αὐτῷ καὶ πατρίδος μεγίστης καὶ βασιλείας ἀσφαλεστάτης ἡγησάμενος οὐκ ἄξιον εἶναι ζῆν ἐπὶ τοῖς παροῦσιν ἀγαθοῖς ἄνευ τῆς πρὸς ἐκείνην οἰκειότητος, ἐπειδὴ παρὰ τῶν κυρίων οὐχ οἷός τ' ἦν αὐτὴν λαβεῖν, ἀλλ' ἐπέμενον τήν τε τῆς παιδὸς ἡλικίαν καὶ τὸν χρησμὸν τὸν παρὰ τῆς Πυθίας, ὑπεριδὼν τὴν ἀρχὴν τὴν Τυνδάρεω καὶ καταφρονήσας τῆς ῥώμης τῆς Κάστορος καὶ Πολυδεύκους καὶ πάντων τῶν ἐν Λακεδαίμονι δεινῶν ὀλιγωρήσας, βίᾳ λαβὼν αὐτὴν εἰς Ἄφιδναν τῆς Ἀττικῆς κατέθετο.

\end{greek}

}

\section*{Analiza i komentar}

%1

{\large
\begin{greek}
\noindent Θησεὺς, \\
\tabto{2em} ὁ λεγόμενος μὲν Αἰγέως \\
\tabto{2em} γενομένος δ' ἐκ Ποσειδῶνος, \\
ἰδὼν αὐτὴν \\
\tabto{2em} οὔπω μὲν ἀκμάζουσαν, \\
\tabto{2em} ἤδη δὲ τῶν ἄλλων διαφέρουσαν\\
τοσοῦτον ἡττήθη τοῦ κάλλους, \\
ὁ κρατεῖν τῶν ἄλλων εἰθισμένος, \\
\tabto{2em} ὥσθ' \\
\tabto{4em} \uuline{ὑπαρχούσης} αὐτῷ \\
\tabto{6em} \uuline{καὶ πατρίδος μεγίστης} \\
\tabto{6em} \uuline{καὶ βασιλείας ἀσφαλεστάτης} \\
\tabto{2em} ἡγησάμενος \\
\tabto{4em} \underline{οὐκ ἄξιον εἶναι ζῆν} \\
\tabto{6em} ἐπὶ τοῖς παροῦσιν ἀγαθοῖς \\
\tabto{6em} ἄνευ τῆς πρὸς ἐκείνην οἰκειότητος\\

\end{greek}
}

\begin{description}[noitemsep]
\item[Θησεὺς] §~175
\item[ὁ λεγόμενος] λέγω reći; n. sg. m. r. ptc. prez. medpas.
\item[Αἰγέως] §~175
\item[μὲν\dots\ δ(ὲ)] doduše\dots\ ali\dots; suprotni veznik §~515.2, §~519.7
\item[γενομένος] γίγνομαι postati, biti; n. sg. m. r. ptc. aor.
\item[ἐκ Ποσειδῶνος] §~131
\item[ἰδὼν] ὁράω vidjeti; n. sg. m. r. ptc. aor. akt.
\item[αὐτὴν] §~207; sc.\ Helenu
\item[οὔπω μὲν\dots\ ἤδη δὲ] još ne\dots\ a već\dots
\item[ἀκμάζουσαν] ἀκμάζω biti u cvijetu ljepote; a. sg. ž. r. ptc. prez. akt.
\item[διαφέρουσαν] διαφέρω τινός biti različit od koga, nadmašivati koga; a. sg. ž. r. ptc. prez. akt.
\item[τῶν ἄλλων] §~212
\item[τοσοῦτον] §~213.4; u korelaciji s ὥστε uvodi posljedičnu rečenicu
\item[ἡττήθη] ἡσσάομαι (ἡττάομαι) τινός biti svladan  od koga ili čega, ne moći odoljeti čemu; 3. l. sg. ind. aor. pas.
\item[τοῦ κάλλους] §~153
\item[ὁ\dots\ εἰθισμένος] ἐθίζω naviknuti; n. sg. m. r. ptc. perf. medpas. εἰθισμένος + inf. navikao na što
\item[κρατεῖν] κρατέω τινός vladati nad kime; inf. prez. akt.
\item[τῶν ἄλλων] §~212
\item[ὥσθ'] ὥστε, posljedični veznik §~473; aspiracija dentala zbog ispadanja vokala §~74; predikat posljedične rečenice nalazi se na samom kraju odlomka (κατέθετο)
\item[ὑπαρχούσης] ὑπάρχω τινί pripadati komu; g. sg. ž. r. ptc. prez. akt.; konstrukcija GA
\item[αὐτῷ] §~207
\item[πατρίδος] §~123
\item[μεγίστης] §~200
\item[βασιλείας] §~90
\item[ἀσφαλεστάτης] §~198
\item[ἡγησάμενος] ἡγέομαι voditi; misliti, držati; n. sg. m. r. ptc. aor. med.; glagol uvodi konstrukciju A+I
\item[ἄξιον] §~103
\item[εἶναι] εἰμί biti; inf. prez. akt.
\item[ἄξιον εἶναι] imenski predikat, Smyth 909
\item[ζῆν] ζάω živjeti; inf. prez. akt.
\item[ἐπί\dots\ ἀγαθοῖς] §~436.B; §~103
\item[τοῖς\dots\ ἀγαθοῖς] supstantiviranje članom §~373
\item[παροῦσιν] πάρειμι biti prisutan, biti u čijoj vlasti; d. pl. sr. r. ptc. prez. akt.
\item[ἄνευ τῆς\dots\ οἰκειότητος] §~417; §~123; οἰκειότης πρός τινα zajedništvo, veza s kim
\item[πρὸς ἐκείνην] §~213

\end{description}

%3 itd

{\large
\begin{greek}
\noindent ἐπειδὴ \\
\tabto{2em} παρὰ τῶν κυρίων \\
οὐχ οἷός τ' ἦν \\
\tabto{2em} αὐτὴν λαβεῖν, \\
ἀλλ' ἐπέμενον \\
\tabto{2em} τήν τε τῆς παιδὸς ἡλικίαν \\
\tabto{2em} καὶ τὸν χρησμὸν τὸν παρὰ τῆς Πυθίας\\
ὑπεριδὼν \\
\tabto{2em} τὴν ἀρχὴν τὴν Τυνδάρεω \\
καὶ καταφρονήσας \\
\tabto{2em} τῆς ῥώμης τῆς Κάστορος καὶ Πολυδεύκους \\
\tabto{2em} καὶ πάντων τῶν ἐν Λακεδαίμονι δεινῶν \\
ὀλιγωρήσας\\
βίᾳ λαβὼν αὐτὴν \\
\tabto{2em} εἰς Ἄφιδναν τῆς Ἀττικῆς κατέθετο.\\

\end{greek}
}

\begin{description}[noitemsep]
\item[ἐπειδή] veznik uvodi vremensku rečenicu §~487
\item[παρὰ τῶν κυρίων] §~434.A; §~82
\item[οἷός τ' ἦν] οἷός τε εἰμί moći
\item[ἦν] εἰμί biti; 3. l. sg. impf. akt.
\item[αὐτὴν] §~207
\item[λαβεῖν] λαμβάνω uzeti, ugrabiti, oteti; inf. aor. akt.
\item[ἐπέμενον] ἐπιμένω čekati; 3. l. pl. impf. akt.; promjena subjekta, subjekt su sad Helenini čuvari ili skrbnici (οἱ κύριοι)
\item[τήν\dots\ ἡλικίαν] §~90
\item[τῆς παιδὸς] §~127
\item[τὸν χρησμὸν] §~82
\item[παρὰ τῆς Πυθίας] §~434.A, §~90
\item[ὑπεριδὼν] ὑπεροράω prezreti; n. sg. m. r. ptc. aor. akt.
\item[τὴν ἀρχὴν] §~90
\item[Τυνδάρεω] §~111
\item[καταφρονήσας] καταφρονέω τινός ne obazirati se na što; n. sg. m. r. ptc. aor. akt.
\item[τῆς ῥώμης] §~90
\item[Κάστορος] §~146
\item[Πολυδεύκους] §~153
\item[τῆς Κάστορος καὶ Πολυδεύκους] atributni položaj §~375
\item[πάντων] §~193
\item[τῶν\dots\ δεινῶν] §~103; τὰ δεινά strahote, opasnosti; supstantiviranje članom §~373
\item[ἐν Λακεδαίμονι] §~131; atributni položaj §~375
\item[ὀλιγωρήσας] ὀλιγωρέω τινός ne mariti za što; n. sg. m. r. ptc. aor. akt.
\item[βίᾳ] §~90; instrumentalni dativ §~414
\item[λαβὼν] λαμβάνω uzeti, ugrabiti; n. sg. m. r. ptc. aor. akt.
\item[αὐτὴν] §~207
\item[εἰς Ἄφιδναν] §~90
\item[τῆς Ἀττικῆς] §~90; partitivni genitiv §~395
\item[κατέθετο] κατατίθημι smjestiti, skloniti; 3. l. sg. ind. aor. med.

\end{description}


%kraj
