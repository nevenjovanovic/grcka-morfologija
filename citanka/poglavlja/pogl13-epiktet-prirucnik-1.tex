\section*{O autoru}

Filozof Epiktet \textgreek[variant=ancient]{(Ἐπίκτητος)} rođen je oko 55.\ po Kr.\ u frigijskom Hijerapolu \textgreek[variant=ancient]{(Ἱεράπολις,} danas Pamukkale). Nije nam poznato njegovo pravo ime – \textgreek[variant=ancient]{Ἐπίκτητος} je nadimak izveden od glagola \textgreek[variant=ancient]{ἐπικτάομαι} ``steći''. Mladost je proveo u Rimu kao rob Neronova oslobođenika i tajnika Epafrodita, koji mu je dopustio da pohađa filozofsku školu utjecajnog stoika Muzonija Rufa. Epiktet je s vremenom i sam počeo predavati filozofiju, a Epafrodit ga je oslobodio. Kad je car Domicijan protjerao filozofe iz Italije (93.-94), Epiktet je napustio Rim i otišao u Nikopol \textgreek[variant=ancient]{(Νικόπολις),} bogati grad u Epiru. Epiktetova škola privlačila je mnoge učenike, među kojima je navodno bio i car Hadrijan. Epiktet je umro oko 135. 

Epiktet je, uz Seneku i Marka Aurelija, glavni predstavnik stoičke škole kasnog, rimskog perioda. Poput Sokrata i učitelja Muzonija, Epiktet nije zapisivao svoje učenje; zapisao ga je njegov učenik Flavije Arijan. Sačuvane su četiri od osam knjiga \textit{Razgovora} \textgreek[variant=ancient]{(Διατριβαί)} te \textit{Priručnik} \textgreek[variant=ancient]{(Ἐγχειρίδιον),} ``praktični vodič za početnike u stoičkoj filozofiji''. 

\textgreek[variant=ancient]{Ἐγχειρίδιον} je pisan jednostavnim jezikom bliskim govornoj \textgreek[variant=ancient]{κοινή,} grčkome standardu helenističkoga i carskoga doba. Bio je vrlo popularno djelo. U VI.~st.\ nastao je Simplicijev komentar, a 1497.\ je humanist Angelo Poliziano preveo djelo na latinski.

\section*{O tekstu}

Na početku \textit{Priručnika} Epiktet poučava da sreća ovisi isključivo o čovjeku samom i njegovoj percepciji. Moramo shvatiti što možemo kontrolirati \textgreek[variant=ancient]{(τὰ ἐφ' ἡμῖν:} npr.\ um, stavovi), a što ne \textgreek[variant=ancient]{(τὰ οὐκ ἐφ' ἡμῖν:} npr.\ tijelo, imetak, tuđa mišljenja, društveni status). U onome što ne možemo kontrolirati ne treba tražiti sreću.

%\newpage

\section*{Pročitajte naglas grčki tekst.}

Epict.\ Enchiridion 1.1

%Naslov prema izdanju

\medskip

\begin{greek}
{\large
{ \noindent Τῶν ὄντων τὰ μέν ἐστιν ἐφ' ἡμῖν, τὰ δὲ οὐκ ἐφ' ἡμῖν. ἐφ' ἡμῖν μὲν ὑπόληψις, ὁρμή, ὄρεξις, ἔκκλισις καὶ ἑνὶ λόγῳ ὅσα ἡμέτερα ἔργα· οὐκ ἐφ' ἡμῖν δὲ τὸ σῶμα, ἡ κτῆσις, δόξαι, ἀρχαὶ καὶ ἑνὶ λόγῳ ὅσα οὐχ ἡμέτερα ἔργα. καὶ τὰ μὲν ἐφ' ἡμῖν ἐστι φύσει ἐλεύθερα, ἀκώλυτα, ἀπαραπόδιστα, τὰ δὲ οὐκ ἐφ' ἡμῖν ἀσθενῆ, δοῦλα, κωλυτά, ἀλλότρια. μέμνησο οὖν, ὅτι, ἐὰν τὰ φύσει δοῦλα ἐλεύθερα οἰηθῇς καὶ τὰ ἀλλότρια ἴδια, ἐμποδισθήσῃ, πενθήσεις, ταραχθήσῃ, μέμψῃ καὶ θεοὺς καὶ ἀνθρώπους, ἐὰν δὲ τὸ σὸν μόνον οἰηθῇς σὸν εἶναι, τὸ δὲ ἀλλότριον, ὥσπερ ἐστίν, ἀλλότριον, οὐδείς σε ἀναγκάσει οὐδέποτε, οὐδείς σε κωλύσει, οὐ μέμψῃ οὐδένα, οὐκ ἐγκαλέσεις τινί, ἄκων πράξεις οὐδὲ ἕν, οὐδείς σε βλάψει, ἐχθρὸν οὐχ ἕξεις, οὐδὲ γὰρ βλαβερόν τι πείσῃ. τηλικούτων οὖν ἐφιέμενος μέμνησο, ὅτι οὐ δεῖ μετρίως  κεκινημένον ἅπτεσθαι αὐτῶν, ἀλλὰ τὰ μὲν ἀφιέναι παντελῶς, τὰ δ' ὑπερτίθεσθαι πρὸς τὸ παρόν. ἐὰν δὲ καὶ ταῦτ' ἐθέλῃς καὶ ἄρχειν καὶ πλουτεῖν, τυχὸν μὲν οὐδ' αὐτῶν τούτων τεύξῃ διὰ τὸ καὶ τῶν προτέρων ἐφίεσθαι, πάντως γε μὴν ἐκείνων ἀποτεύξῃ, δι' ὧν μόνων ἐλευθερία καὶ εὐδαιμονία περιγίνεται. εὐθὺς οὖν πάσῃ φαντασίᾳ τραχείᾳ μελέτα ἐπιλέγειν ὅτι ‘φαντασία εἶ καὶ οὐ πάντως τὸ φαινόμενον’. ἔπειτα ἐξέταζε αὐτὴν καὶ δοκίμαζε τοῖς κανόσι τούτοις οἷς ἔχεις, πρώτῳ δὲ τούτῳ καὶ μάλιστα, πότερον περὶ τὰ ἐφ' ἡμῖν ἐστιν ἢ περὶ τὰ οὐκ ἐφ' ἡμῖν· κἂν περί τι τῶν οὐκ ἐφ' ἡμῖν ᾖ, πρόχειρον ἔστω τὸ διότι ‘οὐδὲν πρὸς ἐμέ’.

}
}
\end{greek}

\vfill

\newpage

\section*{Analiza i komentar}

%1

{\large
\begin{greek}
\noindent Τῶν ὄντων \\
\tabto{2em} τὰ μέν ἐστιν ἐφ' ἡμῖν, \\
\tabto{2em} τὰ δὲ οὐκ ἐφ' ἡμῖν. \\

\end{greek}
}

\begin{description}[noitemsep]
\item[Τῶν ὄντων] εἰμί biti, g. pl. sr. r. ptc. prez. akt., genitiv partitivni, §~395; supstantivirani particip  §~499.2
\item[τὰ μέν\dots, τὰ δὲ\dots] član je upotrijebljen kao pokazna zamjenica §~370.1. Koordinacija μὲν\dots\ δὲ\dots: jedno\dots\ a drugo\dots
\item[ἐστιν ] εἰμί biti, 3. l. sg. ind. prez. akt.; enklitika §~39.3; kongruencija: predikat u jednini uz subjekt srednjeg roda u množini §~361
\item[ἐφ' ἡμῖν] §~205; §~436.B.c.β; elizija §~68, §~74

\end{description}

%2

{\large
\begin{greek}
\noindent ἐφ' ἡμῖν μὲν \\
\tabto{2em} ὑπόληψις, \\
\tabto{2em} ὁρμή, \\
\tabto{2em} ὄρεξις, \\
\tabto{2em} ἔκκλισις \\
\tabto{2em} καὶ \\
\tabto{4em} ἑνὶ λόγῳ \\
\tabto{2em} ὅσα ἡμέτερα ἔργα· \\
οὐκ ἐφ' ἡμῖν δὲ \\
\tabto{2em} τὸ σῶμα, \\
\tabto{2em} ἡ κτῆσις, \\
\tabto{2em} δόξαι, \\
\tabto{2em} ἀρχαὶ \\
\tabto{2em} καὶ \\
\tabto{4em} ἑνὶ λόγῳ \\
\tabto{2em} ὅσα οὐχ ἡμέτερα ἔργα. \\

\end{greek}
}

\begin{description}[noitemsep]
\item[ἐφ' ἡμῖν] §~74; §~205; §~436.B.c.β
\item[ἐφ' ἡμῖν μὲν\dots\ οὐκ ἐφ' ἡμῖν δὲ\dots] §~515.2, koordinacija pomoću para čestica
\item[ὑπόληψις ] §~165
\item[ὁρμή ] §~90
\item[ὄρεξις ] §~165
\item[ἔκκλισις ] §~165
\item[ἑνὶ λόγῳ ] §~224; §~82; dativ kao priložna oznaka
\item[ὅσα ] §~219
\item[ἡμέτερα ] §~210
\item[ἔργα ] §~82
\item[ἐφ' ἡμῖν] §~74; §~205; §436.B.c.β
\item[τὸ σῶμα] §~123
\item[ἡ κτῆσις ] §~165
\item[δόξαι ] §~97
\item[ἀρχαὶ] §~90
\item[ἑνὶ λόγῳ ] §~224; §~82
\item[ὅσα ] §~219
\item[οὐχ ἡμέτερα ] §~74; §~75.1; §~210
\item[ἔργα ] §~82

\end{description}

%3

{\large
\begin{greek}
\noindent καὶ τὰ μὲν ἐφ' ἡμῖν \\
ἐστι \\
\tabto{2em} φύσει \\
ἐλεύθερα, ἀκώλυτα, ἀπαραπόδιστα, \\
τὰ δὲ οὐκ ἐφ' ἡμῖν \\
ἀσθενῆ, δοῦλα, κωλυτά, ἀλλότρια. \\

\end{greek}
}

\begin{description}[noitemsep]
\item[τὰ μὲν ἐφ' ἡμῖν\dots\ τὰ δὲ οὐκ ἐφ' ἡμῖν\dots] koordinacija pomoću para čestica
\item[τὰ\dots\ ἐφ' ἡμῖν] §~74; §~205; §~436.B.c.β; supstantiviranje članom §~373
\item[ἐστι ] εἰμί biti, 3. l. sg. ind. prez. akt.
\item[φύσει ] obzirom na prirodu, po naravi; §~165; dativ obzira \textit{(dativus respectus)} 
\item[ἐλεύθερα ] §~103
\item[ἀκώλυτα] §~103
\item[ἀπαραπόδιστα] §~103
\item[τὰ\dots\ οὐκ ἐφ' ἡμῖν] supstantiviranje članom §~373
\item[ἀσθενῆ ] §~153
\item[δοῦλα ] §~103
\item[κωλυτά] §~103
\item[ἀλλότρια] §~103

\end{description}

%4

{\large
\begin{greek}
\noindent μέμνησο οὖν, \\
\tabto{2em} ὅτι, \\
\tabto{4em} ἐὰν τὰ φύσει δοῦλα ἐλεύθερα οἰηθῇς \\
\tabto{4em} καὶ τὰ ἀλλότρια ἴδια, \\
\tabto{2em} ἐμποδισθήσῃ, \\
\tabto{2em} πενθήσεις, \\
\tabto{2em} ταραχθήσῃ, \\
\tabto{2em} μέμψῃ καὶ θεοὺς καὶ ἀνθρώπους, \\
\tabto{4em} ἐὰν δὲ τὸ σὸν μόνον οἰηθῇς σὸν εἶναι, \\
\tabto{4em} τὸ δὲ ἀλλότριον, ὥσπερ ἐστίν, ἀλλότριον, \\
\tabto{2em} οὐδείς σε ἀναγκάσει οὐδέποτε, \\
\tabto{2em} οὐδείς σε κωλύσει, \\
\tabto{2em} οὐ μέμψῃ οὐδένα, \\
\tabto{2em} οὐκ ἐγκαλέσεις τινί, \\
\tabto{2em} ἄκων πράξεις οὐδὲ ἕν, \\
\tabto{2em} οὐδείς σε βλάψει, \\
\tabto{2em} ἐχθρὸν οὐχ ἕξεις, \\
\tabto{2em} οὐδὲ γὰρ βλαβερόν τι πείσῃ.\\

\end{greek}
}

\begin{description}[noitemsep]
\item[μέμνησο] μιμνήσκω sjetiti (se), 2. l. sg. impt. perf. medpas.
\item[ὅτι ] veznik uvodi izričnu rečenicu, §~467 
\item[ἐὰν] veznik uvodi eventualnu pogodbenu rečenicu, §~474-476
\item[ἐὰν\dots\ ἐὰν δὲ] koordinacija pomoću čestice δὲ, bez μὲν
\item[τὰ φύσει δοῦλα ] ono što je po prirodi ropsko; supstantiviranje članom §~373
\item[φύσει ] §~165
\item[δοῦλα ] §~103
\item[ἐλεύθερα ] §~103
\item[οἰηθῇς ] οἶμαι (οἴομαι) smatrati, 2. l. sg. konj. aor. pas. 
\item[τὰ ἀλλότρια ] §~103, supstantiviranje članom §~373
\item[ἴδια] §~103
\item[ἐμποδισθήσῃ] ἐμποδίζω ometati, 2. l. sg. ind. fut. pas.
\item[πενθήσεις] πενθέω žalostiti se, 2. l. sg. ind. fut. akt.
\item[ταραχθήσῃ] ταράσσω uzrujati, uznemiriti, 2. l. sg. ind. fut. pas.
\item[μέμψῃ ] μέμφομαι grditi, psovati; 2. l. sg. ind. fut. med.
\item[θεοὺς ] §~82
\item[ἀνθρώπους] §~82
\item[ἐὰν ] veznik uvodi eventualnu pogodbenu rečenicu, §~474-476
\item[τὸ σὸν ] §~210, supstantiviranje članom §~373
\item[τὸ σὸν\dots\ τὸ δὲ ἀλλότριον] koordinacija pomoću čestice δὲ, bez μὲν; supstantiviranje članom §~373
\item[μόνον ] §~378.4
\item[οἰηθῇς ] οἶμαι (οἴομαι) smatrati, 2. l. sg. konj. aor. pas., uvodi konstrukciju A+I
\item[σὸν ] §~210
\item[εἶναι ] εἰμί biti, inf. prez. akt., ovisi o οἰηθῇς
\item[τὸ\dots\ ἀλλότριον ] §~103, supstantiviranje članom §~373
\item[ὥσπερ ] §~519.2
\item[ἐστίν ] εἰμί 3. l. sg. ind. prez. akt.
\item[ἀλλότριον ] §~103
\item[οὐδείς ] §~224.2
\item[σε ] §~205
\item[ἀναγκάσει  ] ἀναγκάζω prisiljavati, 3. l. sg. ind. fut. akt.
\item[οὐδείς ] §~224.2
\item[σε ] §~205
\item[κωλύσει ] κωλύω ometati, sprečavati; 3. l. sg. ind. fut. akt.
\item[μέμψῃ ] μέμφομαι grditi, psovati; 2. l. sg. ind. fut. med.
\item[οὐδένα] §~224.2
\item[ἐγκαλέσεις ] ἐγκαλέω τινί optuživati nekoga; 2. l. sg. ind. fut. akt.
\item[τινί ] §~217
\item[ἄκων ] §~193
\item[πράξεις ] πράσσω (atički πράττω) činiti; 2. l. sg. ind. fut. akt.
\item[ἕν ] §~224
\item[οὐδείς ] §~224.2
\item[σε ] §~205
\item[βλάψει] βλάπτω τινά povrijediti nekoga; 3. l. sg. ind. fut. akt.]  
\item[ἐχθρὸν ] §~82
\item[ἕξεις ] ἔχω 2. l. sg. ind. fut. akt.
\item[βλαβερόν ] §~103
\item[τι ] §~217
\item[πείσῃ ] πάσχω trpjeti, 2. l. sg. ind. fut. med.

\end{description}

%5

{\large
\begin{greek}
\noindent τηλικούτων οὖν ἐφιέμενος \\
μέμνησο, \\
\tabto{2em} ὅτι οὐ δεῖ \\
\tabto{4em} μετρίως κεκινημένον \\
\tabto{4em} ἅπτεσθαι αὐτῶν, \\
\tabto{2em} ἀλλὰ τὰ μὲν \\
\tabto{4em} ἀφιέναι παντελῶς, \\
\tabto{2em} τὰ δ' \\
\tabto{4em} ὑπερτίθεσθαι πρὸς τὸ παρόν.\\

\end{greek}
}

\begin{description}[noitemsep]
\item[τηλικούτων] §~213.4
\item[ἐφιέμενος] ἐφίημι, \textit{ovdje} ἐφίημαι τινός težiti za nečim, čeznuti za nečim; n. sg. m. r. ptc. prez. medpas.
\item[μέμνησο] μιμνήσκω sjetiti (se), 2. l. sg. impt. perf. medpas.
\item[ὅτι] veznik uvodi izričnu rečenicu §~467
\item[δεῖ ] δέω, \textit{bezlično} δεῖ: treba, 3. l. sg. ind. prez. akt.; otvara mjesto infinitivima ἅπτεσθαι, ἀφιέναι, ὑπερτίθεσθαι
\item[μετρίως ] §~204
\item[κεκινημένον ] κινέω pokretati; a. sg. m. r. ptc. perf. medpas.; filozofski termin ὁ κεκινημένος uzbuđen, uznemiren
\item[ἅπτεσθαι ] ἅπτω vezati, \textit{ovdje} ἅπτομαι τινός vezati se za nešto, biti dirnut nečim; inf. prez. medpas. 
\item[αὐτῶν ] §~207
\item[τὰ μὲν\dots\ τὰ δ'] koordinacija: jedno\dots\ a drugo\dots
\item[ἀφιέναι ] ἀφίημι otpustiti, osloboditi (antiteza ἅπτω); inf. prez. akt.
\item[παντελῶς] prilog od παντελής, §~204
\item[ὑπερτίθεσθαι ] ὑπερτίθημι odložiti, odgoditi; inf. prez. medpas.
\item[πρὸς τὸ παρόν ] \textit{fraza:} zasad; §~435; πάρειμι biti prisutan, a. sg. s. r. ptc. prez. akt.; supstantivirani particip §~499.2

\end{description}

%6

{\large
\begin{greek}
\noindent ἐὰν δὲ καὶ ταῦτ' ἐθέλῃς \\
\tabto{2em} καὶ ἄρχειν \\
\tabto{2em} καὶ πλουτεῖν, \\
τυχὸν μὲν οὐδ' αὐτῶν τούτων τεύξῃ \\
\tabto{2em} διὰ τὸ καὶ τῶν προτέρων ἐφίεσθαι, \\
πάντως γε μὴν ἐκείνων ἀποτεύξῃ, \\
\tabto{2em} δι' ὧν μόνων ἐλευθερία καὶ εὐδαιμονία περιγίνεται.\\

\end{greek}
}

\begin{description}[noitemsep]
\item[ἐὰν] veznik uvodi eventualnu pogodbenu rečenicu, §~474-476
\item[ταῦτ'] §~213.2; elizija §~68
\item[ἐθέλῃς ] ἐθέλω (θέλω) htjeti; 2. l. sg. konj. prez. akt; otvara mjesto akuzativu ταῦτα, ali i (koordiniranim ponavljanjem καὶ) infinitivima ἄρχειν, πλουτεῖν
\item[ἄρχειν] ἄρχω vladati, inf. prez. akt.
\item[πλουτεῖν] πλουτέω obogatiti se, inf. prez. akt.
\item[τυχὸν ] τυγχάνω slučajno biti negdje, slučajno se dogoditi;  n. sg. s. r. ptc. aor. akt.; particip ovdje odgovara našem prilogu “možda“ (od ``može biti da''); τυχὸν μὲν i πάντως γε μὴν stoje u antitezi
\item[μὲν\dots\ γε μὴν] korelacija s adverzativnim (suprotnim) značenjem ``\dots\ ali\dots'', ili progresivno ``\dots\ a onda\dots''
\item[αὐτῶν] §~207
\item[τούτων ] §~213.2
\item[τεύξῃ] τυγχάνω τινός domoći se nečega, uspjeti u nečemu, 2. l. sg. ind. fut. med.
\item[διὰ τὸ ἐφίεσθαι] §~428, ἐφίημι inf. prez. medpas., supstantivirani infinitiv §~497, ἐφίημαι τινός težiti za nečim, čeznuti za nečim
\item[πάντως ] §~193, §~204
\item[ἐκείνων ] §~213.3
\item[ἀποτεύξῃ] ἀποτυγχάνω τινός izgubiti nešto; 2. l. sg. ind. fut. med.
\item[δι' ὧν μόνων] §~215; §~428; §~378.4; elizija §~68
\item[ἐλευθερία] §~90
\item[εὐδαιμονία] §~90
\item[περιγίνεται] περιγίνομαι (jonski i helenistički oblik glagola περιγίγνομαι) proizaći; 3. l. sg. ind. prez. medpas.

\end{description}

%7

{\large
\begin{greek}
\noindent εὐθὺς οὖν \\
\tabto{2em} πάσῃ φαντασίᾳ τραχείᾳ \\
μελέτα \\
\tabto{2em} ἐπιλέγειν ὅτι \\
\tabto{4em} ‘φαντασία εἶ \\
\tabto{4em} καὶ οὐ πάντως τὸ φαινόμενον’.\\

\end{greek}
}

\begin{description}[noitemsep]
\item[εὐθὺς ] vremenski prilog od εὐθὺς (§~191)
\item[πάσῃ ] §~193
\item[φαντασίᾳ ] §~90
\item[τραχείᾳ ] §~191
\item[μελέτα ] μελετάω truditi se, vježbati; 2. l. sg. impt. prez. akt; otvara mjesto infinitivu
\item[ἐπιλέγειν] ἐπιλέγω τινί obratiti se nekome, osloviti koga imenom, dodijeliti nekome; inf. prez. akt.
\item[ὅτι ] veznik uvodi zavisnu izričnu rečenicu, §~467, bilješka 2
\item[φαντασία ] §~90
\item[εἶ ] εἰμί 2. l. sg. ind. prez. akt.
\item[πάντως ] §~193, §~204
\item[τὸ φαινόμενον ] φαίνω pokazati, ukazati; n. sg. s. r. ptc. prez. medpas., supstantivirani particip §~499.2; τὸ φαινόμενον filozofski termin: ono što se osjetilima poima, ono što se čini

\end{description}

%8

{\large
\begin{greek}
\noindent ἔπειτα ἐξέταζε αὐτὴν \\
καὶ δοκίμαζε \\
\tabto{2em} τοῖς κανόσι τούτοις \\
\tabto{4em} οἷς ἔχεις, \\
\tabto{2em} πρώτῳ δὲ τούτῳ καὶ μάλιστα, \\
\tabto{4em} πότερον \\
\tabto{6em} περὶ τὰ ἐφ' ἡμῖν ἐστιν \\
\tabto{6em} ἢ περὶ τὰ οὐκ ἐφ' ἡμῖν·\\
κἂν \\
\tabto{2em} περί τι \\
\tabto{4em} τῶν οὐκ ἐφ' ἡμῖν ᾖ, \\
πρόχειρον ἔστω \\
\tabto{2em} τὸ διότι ‘οὐδὲν πρὸς ἐμέ’.\\

\end{greek}
}

\begin{description}[noitemsep]
\item[ἐξέταζε ] ἐξετάζω istražiti, 2. l. sg. impt. prez. akt.
\item[αὐτὴν] §~207
\item[δοκίμαζε ] δοκιμάζω ispitati, 2. l. sg. impt. prez. akt.
\item[τοῖς κανόσι ] §~131
\item[τοῖς κανόσι\dots\ πρώτῳ δὲ τούτῳ] koordinacija pomoću čestice δὲ, ovdje bez μὲν ili sličnog u prvom članu para
\item[τούτοις ] §~213.2
\item[οἷς ] §~215
\item[ἔχεις] ἔχω imati, 2. l. sg. ind. prez. akt.
\item[πρώτῳ ] §~223
\item[τούτῳ ] §~213.2
\item[πότερον\dots\ ἢ  ] uvodi koordinirane zavisno upitne surečenice §~469 
\item[περὶ τὰ ἐφ' ἡμῖν ] §~433.C; §~373
\item[ἐστιν ] εἰμί biti, 3. l. sg. ind. prez. akt.
\item[περὶ τὰ οὐκ ἐφ' ἡμῖν ] §~433.C; supstantiviranje članom §~373; §~205; §~436.B.c.β 
\item[κἂν ] §~66 kraza, καὶ ἄν (ἐάν); eventualna pogodbena rečenica §~474-476
\item[περί τι ] §~433.C
\item[τῶν οὐκ ἐφ' ἡμῖν ] §~373; partitivni genitiv §~395; supstantiviranje članom §~373; §~205 §~436.B.c.β
\item[ᾖ] εἰμί biti, 3. l. sg. konj. prez. akt.
\item[πρόχειρον ] §~103; kod Epikteta izraz često najavljuje savjet, formulu koju treba zapamtiti i usvojiti
\item[ἔστω ] εἰμί biti, imperativ prez. akt. 3. l. sg. 
\item[τὸ διότι ‘οὐδὲν πρὸς ἐμέ’] supstantiviranje članom §~373; ovdje je supstantivirana čitava surečenica
\item[διότι] ovdje upotrijebljeno kao izrični veznik: da\dots

\end{description}


%kraj
