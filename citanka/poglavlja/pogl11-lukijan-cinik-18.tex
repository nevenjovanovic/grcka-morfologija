% Unesi korekture NČ i NZ, 2019-09-24

\section*{O tekstu}

Spis pod naslovom \textit{Cinik} (Κυνικός) jedan je od tekstova kojem neki filolozi osporavaju Lukijanovo autorstvo, smatraju ga pseudolukijanskim. Djelo ima oblik dijaloga; glavna su lica Likin i neimenovani kinički filozof. Autorove stavove izlaže Likin. U ovom odlomku kinički filozof slikovito opisuje odnos ljudi i njihovih strasti.

%\newpage

\section*{Pročitajte naglas grčki tekst.}
Luc.\ Cynicus 18.8

%Naslov prema izdanju

\medskip

{\large
\begin{greek}
\noindent Πάσχετε δὲ παραπλήσιόν τι ὅ φασι παθεῖν τινα ἐφ' ἵππον ἀναβάντα μαινόμενον· ἁρπάσας γὰρ αὐτὸν ἔφερεν ἄρα ὁ ἵππος· ὁ δὲ οὐκέτι καταβῆναι τοῦ ἵππου θέοντος ἐδύνατο. καί τις ἀπαντήσας ἠρώτησεν αὐτὸν ποίαν ἄπεισιν; ὁ δὲ εἶπεν, Ὅπου ἂν τούτῳ δοκῇ, δεικνὺς τὸν ἵππον. καὶ ὑμᾶς ἄν τις ἐρωτᾷ, ποῖ φέρεσθε; τἀληθὲς ἐθέλοντες λέγειν ἐρεῖτε ἁπλῶς μέν, ὅπουπερ ἂν ταῖς ἐπιθυμίαις δοκῇ, κατὰ μέρος δέ, ὅπουπερ ἂν τῇ ἡδονῇ δοκῇ, ποτὲ δέ, ὅπου τῇ δόξῃ, ποτὲ δὲ αὖ, τῇ φιλοκερδίᾳ· ποτὲ δὲ ὁ θυμός, ποτὲ δὲ ὁ φόβος, ποτὲ δὲ ἄλλο τι τοιοῦτον ὑμᾶς ἐκφέρειν φαίνεται· οὐ γὰρ  ἐφ' ἑνός, ἀλλ' ἐπὶ πολλῶν ὑμεῖς γε ἵππων βεβηκότες ἄλλοτε ἄλλων, καὶ μαινομένων πάντων, φέρεσθε. τοιγαροῦν ἐκφέρουσιν ὑμᾶς εἰς βάραθρα καὶ κρημνούς. ἴστε δ' οὐδαμῶς πρὶν πεσεῖν ὅτι πεσεῖσθαι μέλλετε.

\end{greek}

}

\section*{Analiza i komentar}


%1

{\large
\begin{greek}
\noindent Πάσχετε δὲ \\
\tabto{2em} παραπλήσιόν τι \\
\tabto{4em} ὅ φασι \\
\tabto{6em} \underline{παθεῖν τινα} \\
\tabto{8em} ἐφ' ἵππον \underline{ἀναβάντα} μαινόμενον·\\

\end{greek}
}

\begin{description}[noitemsep]
\item[Πάσχετε] πάσχω pretrpjeti, \textit{ovdje} događa vam se; 2. l. pl. ind. prez. akt. 
\item[δὲ] čestica povezuje rečenicu s prethodnom: a\dots
\item[παραπλήσιόν] §~103 - §~106
\item[τι] §~217 - §~218.1-3
\item[ὅ] §~215; neodređena zamjenica τι (antecedent) otvara mjesto odnosnoj zamjenici ὅ
\item[φασι] φημί kazati; 3. l. pl. ind. prez. akt.; glagol govorenja otvara mjesto konstrukciji A+I
\item[παθεῖν] πάσχω pretrpjeti, \textit{ovdje} dogoditi se; inf. aor. akt.; infinitiv je dio konstrukcije A+I ovisne o glagolu φασι
\item[τινα ἀναβάντα] dva akuzativa dio su konstrukcije A+I ovisne o glagolu φασι
\item[τινα] §~217 - §~218.1-3
\item[ἀναβάντα] ἀναβαίνω penjati se; a. sg. m. r. ptc. aor. akt.; §~139
\item[ἐφ' (= ἐπί) ἵππον\dots\ μαινόμενον] §~82; prijedložni izraz: ἐπί + a.: na\dots; elizija §~68.c, aspiracija §~74; particip μαινόμενον ovisan je o imenici ἵππον
\item[μαινόμενον] μαίνομαι mahnitati; a. sg. m. r. ptc. prez. medpas.

\end{description}

{\large
\begin{greek}
\noindent ἁρπάσας γὰρ αὐτὸν \\
\tabto{2em} ἔφερεν ἄρα \\
\tabto{2em} ὁ ἵππος·\\
\tabto{2em} ὁ δὲ \\
\tabto{4em} οὐκέτι καταβῆναι \\
\tabto{6em} \uuline{τοῦ ἵππου θέοντος}\\
\tabto{4em} ἐδύνατο.\\

\end{greek}
}

\begin{description}[noitemsep]
\item[γὰρ ] čestica najavljuje iznošenje dokaza prethodne tvrdnje: naime\dots, jer\dots
\item[ἁρπάσας ] ἁρπάζω ugrabiti; n. sg. m. r. ptc. aor. akt.; §~139
\item[αὐτὸν ] §~207; riječ ima službu objekta participa i finitnog glagola istodobno
\item[ἔφερεν] φέρω nositi; odnijeti; 3. l. sg. impf. akt. 
\item[ἄρα] §~516.1
\item[ὁ ἵππος] §~80, §~82
\item[ὁ δὲ] a on\dots; član uz česticu δὲ na početku surečenice naznačuje nov subjekt §~370.2
\item[οὐκέτι\dots\ ἐδύνατο] δύναμαι moći; glagol otvara mjesto dopuni u infinitivu; 3. l. sg. impf. medpas.
\item[καταβῆναι] καταβαίνω sići, sjašiti; inf. aor. akt.
\item[τοῦ ἵππου θέοντος] §~80, §~82; GA; §~139
\item[θέοντος] θέω trčati; g. sg. m. r. ptc. prez. akt.

\end{description}

%3 itd

{\large
\begin{greek}
\noindent καί τις \\
\tabto{2em} ἀπαντήσας \\
ἠρώτησεν αὐτὸν\\
\tabto{2em} ποίαν \\
\tabto{4em} ἄπεισιν;\\
ὁ δὲ \\
εἶπεν, \\
\tabto{2em} Ὅπου ἂν \\
\tabto{4em} τούτῳ δοκῇ,\\
δεικνὺς τὸν ἵππον.\\

\end{greek}
}

\begin{description}[noitemsep]
\item[τις] §~217 - §~218.1-3
\item[ἀπαντήσας] ἀπαντάω sresti; n. sg. m. r. ptc. aor. akt.; §~139
\item[ἠρώτησεν] ἐρωτάω upitati; 3. l. sg. ind. aor. akt.
\item[αὐτὸν] §~207; riječ ima službu objekta participa i finitnog glagola istodobno
\item[ποίαν] sc.\ ὁδόν §~219; upitna zamjenica u neizravnom pitanju §~469
\item[ἄπεισιν] ἄπειμι odlaziti, ići; 3. l. sg. ind. prez. akt.; indikativ ima futursko značenje
\item[ὁ δὲ] a on\dots; član na početku surečenice naznačuje nov subjekt; §~370.2
\item[εἶπεν] εἶπον reći; 3. l. sg. ind. aor. akt.
\item[Ὅπου] kamo god; ovdje u neodređenome priložnom značenju §~220.3, §~221
\item[ἂν\dots\ δοκῇ] δοκεῖ μοι činiti se, \textit{ovdje} sviđati se; 3. l. sg. konj. prez. akt.
\item[τούτῳ] §~213.2
\item[δεικνὺς] δείκνυμι pokazati; n. sg. m. r. ptc. prez. akt. 
\item[τὸν ἵππον] §~80, §~82

\end{description}

%5

{\large
\begin{greek}
\noindent καὶ ὑμᾶς ἄν τις ἐρωτᾷ, \\
\tabto{2em} ποῖ φέρεσθε;\\
τἀληθὲς ἐθέλοντες λέγειν \\
\tabto{2em} ἐρεῖτε ἁπλῶς μέν,\\
\tabto{4em} ὅπουπερ ἂν \\
\tabto{6em} ταῖς ἐπιθυμίαις δοκῇ, \\
\tabto{2em} κατὰ μέρος δέ, \\
\tabto{4em} ὅπουπερ ἂν \\
\tabto{6em} τῇ ἡδονῇ δοκῇ, \\
\tabto{2em} ποτὲ δέ, \\
\tabto{4em} ὅπου \\
\tabto{6em} τῇ δόξῃ, \\
\tabto{2em} ποτὲ δὲ αὖ, \\
\tabto{6em} τῇ φιλοκερδίᾳ·\\
\tabto{2em} ποτὲ δὲ \\
\tabto{6em} \underline{ὁ θυμός},\\
\tabto{2em} ποτὲ δὲ \\
\tabto{6em} \underline{ὁ φόβος}, \\
\tabto{2em} ποτὲ δὲ \\
\tabto{6em} \underline{ἄλλο τι τοιοῦτον} \\
\tabto{8em} ὑμᾶς \underline{ἐκφέρειν} \\
\tabto{10em} φαίνεται·\\

\end{greek}
}

\begin{description}[noitemsep]
\item[ὑμᾶς] §~205
\item[τις] §~217 - §~218.1-3
\item[ἄν] = εἰ ἄν 
\item[ἐρωτᾷ] ἐρωτάω upitati; 3. l. sg. konj. prez. akt.
\item[ποῖ] kamo; upitni prilog mjesta §~221
\item[φέρεσθε] φέρω nositi, odnijeti; 2. l. pl. ind. prez. medpas.
\item[τἀληθὲς] §~153, kraza §~66; poimeničeni pridjev §~373
\item[ἐθέλοντες] ἐθέλω htjeti; glagol otvara mjesto dopuni u infinitivu; n. pl. m. r. ptc. prez. akt.; §~139
\item[λέγειν] λέγω kazati; inf. prez. akt. 
\item[ἐρεῖτε] εἴρω govoriti, kazati, reći; 2. l. pl. ind. fut. akt. 
\item[ἁπλῶς] §~204
\item[ἁπλῶς μέν\dots\ κατὰ μέρος δέ\dots] koordinacija rečeničnih članova s pomoću para suprotnih čestica: a\dots
\item[ταῖς ἐπιθυμίαις] §~80, §~90
\item[ἂν\dots\ δοκῇ] δοκέω činiti se, \textit{ovdje} sviđati se; 3. l. sg. konj. prez. akt.
\item[κατὰ μέρος] posebno (u opreci prema ἁπλῶς općenito)
\item[τῇ ἡδονῇ] §~80, §~90
\item[ποτὲ δέ\dots\ ποτὲ δέ\dots] koordinacija: sad\dots\ sad\dots
\item[τῇ δόξῃ] §~80, §~90
\item[τῇ φιλοκερδίᾳ] §~80, §~90
\item[ποτὲ δὲ\dots\ ποτὲ δὲ\dots\ ποτὲ δὲ\dots] koordinacija: sad\dots\ sad\dots\ sad\dots
\item[ὁ θυμός] §~80, §~82
\item[ὁ φόβος] §~80, §~82
\item[ἄλλο τι τοιοῦτον] §~212, §~217.1-2, §~213
\item[ὑμᾶς] §~205
\item[ἐκφέρειν] ἐκφέρω odnositi; inf. prez. akt. 
\item[ἄλλο τι τοιοῦτον\dots\ ἐκφέρειν] N+I, ovisi o glagolu φαίνεται: čini se da\dots
\item[φαίνεται] φαίνω činiti se; otvara mjesto dopuni (ovdje N+I); 3. l. sg. ind. prez. medpas.

\end{description}

%6

{\large
\begin{greek}
\noindent οὐ γὰρ ἐφ' ἑνός, \\
\tabto{2em} ἀλλ' ἐπὶ πολλῶν ὑμεῖς γε ἵππων βεβηκότες \\
\tabto{4em} ἄλλοτε ἄλλων, \\
\tabto{4em} καὶ μαινομένων πάντων, \\
\tabto{2em} φέρεσθε.\\

\end{greek}
}

\begin{description}[noitemsep]
\item[οὐ γὰρ\dots\ ἀλλ' (= ἀλλά)] ne\dots\ nego\dots; elizija §~68.c
\item[ἐφ' (= ἐπὶ) ἑνός]  (sc.\ ἵππου) §~223, §~224; prijedložni izraz ἐπὶ + g.: na\dots, §~418, §~436.A; elizija §~68.c, aspiracija suglasnika na kraju riječi §~74
\item[ἐπὶ πολλῶν\dots\ ἵππων\dots\ ἄλλων\dots\ μαινομένων πάντων] §~196, §~80, §~82; prijedložni izraz ἐπὶ + g.: na\dots, §~418, 436.A
\item[ὑμεῖς] §~205
\item[γε] baš\dots; čestica naglašava prethodnu riječ
\item[βεβηκότες] βαίνω ići, βαίνω ἐφ' ἵππων zajašiti konja; n. pl. m. r. ptc. perf. akt.; §~123
\item[μαινομένων πάντων] §~212, μαίνομαι mahnitati; g. pl. m. r. ptc. prez. medpas.; koji su svi\dots; §~193; §~82, §~90
\item[φέρεσθε] φέρω \textit{kao gore} kretati se, ići; odnijeti; 2. l. pl. ind. prez. medpas.


\end{description}

%7

{\large
\begin{greek}
\noindent τοιγαροῦν \\
ἐκφέρουσιν ὑμᾶς \\
\tabto{2em} εἰς βάραθρα καὶ κρημνούς.\\

\end{greek}
}

\begin{description}[noitemsep]
\item[ἐκφέρουσιν] ἐκφέρω odnositi; 3. l. pl. ind. prez. akt. 
\item[ὑμᾶς] §~205
\item[εἰς βάραθρα καὶ κρημνούς] §~82; §~418, §~419.1

\end{description}

%8

{\large
\begin{greek}
\noindent ἴστε δ' οὐδαμῶς \\
\tabto{2em} πρὶν πεσεῖν \\
\tabto{4em} ὅτι πεσεῖσθαι μέλλετε.\\

\end{greek}
}

\begin{description}[noitemsep]
\item[ἴστε δ' = ἴστε δέ] οἶδα znati; 2. l. pl. ind. perf. akt.; elizija §~68.c
\item[πρὶν] + infinitiv ima vrijednost zavisne rečenice vremenskog značenja: prije nego li\dots, §~487, §~488
\item[πεσεῖν] πίπτω pasti; inf. aor. akt.; subjekt je vidljiv iz ἴστε
\item[ὅτι] veznik uvodi izričnu rečenicu kojoj mjesto otvara glagol ἴστε
\item[μέλλετε] μέλλω namjeravati, \textit{ovdje} morati, trebati; 2. l. pl. ind. prez. akt.
\item[πεσεῖσθαι] πίπτω pasti; inf. fut. med.

\end{description}

%kraj
