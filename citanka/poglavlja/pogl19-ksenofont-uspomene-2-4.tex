%\section*{O autoru}



\section*{O tekstu}

\textit{Uspomene na Sokrata} \textgreek[variant=ancient]{(Ἀπομνημονεύματα Σωκράτους)} sastavljene su u četiri knjige i zapravo su skup razgovora i epizoda sa Sokratom u glavnoj ulozi. U drugoj knjizi Sokrat, između ostalog, govori i o prijateljstvu. U ovdje odabranom odjeljku Sokrat, podsjetivši na opće mišljenje da je dobar prijatelj najbolja stečevina, ipak primjećuje da se većina više brine o svemu drugome, nego o prijateljima.

%\newpage

\section*{Pročitajte naglas grčki tekst.}

%Naslov prema izdanju

Xen.\ Memorabilia 2.4.1

\medskip

{\large
\begin{greek}
\noindent  Ἤκουσα δέ ποτε αὐτοῦ καὶ περὶ φίλων διαλεγομένου ἐξ ὧν ἔμοιγε ἐδόκει μάλιστ' ἄν τις ὠφελεῖσθαι πρὸς φίλων κτῆσίν τε καὶ χρείαν. τοῦτο μὲν γὰρ δὴ πολλῶν ἔφη ἀκούειν, ὡς πάντων κτημάτων κράτιστον ἂν εἴη φίλος σαφὴς καὶ ἀγαθός· ἐπιμελομένους δὲ παντὸς μᾶλλον ὁρᾶν ἔφη τοὺς πολλοὺς ἢ φίλων κτήσεως. καὶ γὰρ οἰκίας καὶ ἀγροὺς καὶ ἀνδράποδα καὶ βοσκήματα καὶ σκεύη κτωμένους τε ἐπιμελῶς ὁρᾶν ἔφη καὶ τὰ ὄντα σῴζειν πειρωμένους, φίλον δέ, ὃ μέγιστον ἀγαθὸν εἶναί φασιν, ὁρᾶν ἔφη τοὺς πολλοὺς οὔτε ὅπως κτήσωνται φροντίζοντας οὔτε ὅπως οἱ ὄντες αὐτοῖς σῴζωνται. ἀλλὰ καὶ καμνόντων φίλων τε καὶ οἰκετῶν ὁρᾶν τινας ἔφη τοῖς μὲν οἰκέταις καὶ ἰατροὺς εἰσάγοντας καὶ τἆλλα τὰ πρὸς ὑγίειαν ἐπιμελῶς παρασκευάζοντας, τῶν δὲ φίλων ὀλιγωροῦντας, ἀποθανόντων τε ἀμφοτέρων ἐπὶ μὲν τοῖς οἰκέταις ἀχθομένους τε καὶ ζημίαν ἡγουμένους, ἐπὶ δὲ τοῖς φίλοις οὐδὲν οἰομένους ἐλαττοῦσθαι, καὶ τῶν μὲν ἄλλων κτημάτων οὐδὲν ἐῶντας ἀθεράπευτον οὐδ' ἀνεπίσκεπτον, τῶν δὲ φίλων ἐπιμελείας δεομένων ἀμελοῦντας.
\end{greek}

}

\section*{Analiza i komentar}

%1

{\large
\noindent ῎Ηκουσα δέ ποτε \\
\tabto{2em} αὐτοῦ \\
\tabto{3em} καὶ περὶ φίλων διαλεγομένου \\
ἐξ ὧν ἔμοιγε ἐδόκει \\
\tabto{2em} μάλιστ' ἄν τις ὠφελεῖσθαι \\
\tabto{3em} πρὸς φίλων κτῆσίν τε καὶ χρείαν. \\

}

\begin{description}[noitemsep]
\item[῎Ηκουσα] ἀκούω τινός τι slušati od koga što; 1. l. sg. ind. aor. akt.; poveži s αὐτοῦ\dots\  διαλεγομένου, predikatni particip uz \textit{verba sentiendi;} §~502.B 4
\item[αὐτοῦ] §~207
\item[περὶ φίλων] §~433; §~82
\item[διαλεγομένου] διαλέγομαι razgovarati; g. sg. m. r. ptc. prez. medpas.; ovisno o αὐτοῦ
\item[ἐξ ὧν] §~424; §~215, relativ se odnosi na neizrečeni objekt glagolskog oblika διαλεγομένου
\item[ἔμοιγε] §~205, §~206
\item[ἐδόκει] δοκέω  (intr.) činiti se; 3. l. sg. impf. akt.; δοκεῖ μοι čini mi se, mislim 
\item[μάλιστ'] §~204.3; elizija §~68 
\item[τις] §~217
\item[ἄν ὠφελεῖσθαι] ὠφελέω koristiti; inf. prez. med.; inf. uz ἄν zadobiva potencijalno značenje, tj. zamjenjuje rečenicu u kojoj bi bio optativ s ἄν §~506
\item[πρὸς\dots\  κτῆσίν τε καὶ χρείαν] §~435;  §~165;  §~90

\end{description}

{\large
\noindent τοῦτο μὲν γὰρ δὴ πολλῶν\\
ἔφη ἀκούειν, \\
\tabto{2em} ὡς \\
\tabto{3em} πάντων κτημάτων \\
\tabto{2em} κράτιστον ἂν εἴη \\
\tabto{2em} φίλος σαφὴς καὶ ἀγαθός· \\
ἐπιμελομένους δὲ \\
\tabto{2em} παντὸς μᾶλλον \\
ὁρᾶν ἔφη \\
\tabto{2em} τοὺς πολλοὺς \\
ἢ φίλων κτήσεως.\\

}

\begin{description}[noitemsep]

\item[τοῦτο\dots\  δὴ] §~213.2, δὴ dakle; §~516.5
\item[μὲν\dots\  δὲ] §~519.7
\item[γὰρ] §~517
\item[πολλῶν] §~196; genitiv odvajanja §~402
\item[ἔφη] φημί reći; 3. l. sg. impf. akt.
\item[ἀκούειν] ἀκούω čuti; inf. prez. akt.; subjekt infinitiva ne razlikuje se od subjekta glavne rečenice, pa se ne izriče §~491.2
\item[ὡς\dots\  κράτιστον ἂν εἴη]  objasnidbena apozicija uz τοῦτο
\item[ὡς] kao veznik izričnih zavisnih rečenica §~467
\item[ἂν εἴη] u izričnim rečenicama može doći i potencijal, §~467
\item[πάντων κτημάτων] §~193; §~123; dijelni genitiv §~395
\item[κράτιστον] §~202; pridjev kao dio imenskog predikata Smyth 909
\item[εἴη] εἰμί biti; 3. l. sg. opt.
\item[φίλος] §~82
\item[σαφὴς ] §~153
\item[ἀγαθός] §~103
\item[ἐπιμελομένους] ἐπιμέλομαί τινος brinuti se za što; a. pl. m. r. ptc. prez. medpas.; predikatni particip iza gl. ὁράω proteže se na objekt (τοὺς πολλοὺς) §~502 
\item[παντὸς μᾶλλον\dots\  ἢ] za sve više nego\dots\ §~193; §~204.3; §~514
\item[ὁρᾶν] ὁράω vidjeti; inf. prez. akt.; ὁρᾶν ἔφη, subjekt infinitiva ne razlikuje se od subjekta glavne rečenice, pa se ne izriče §~491.2
\item[τοὺς πολλοὺς] §~196; οἱ πολλοί ovdje: većina, §~370
\item[φίλων] §~82
\item[κτήσεως] §~165
\end{description}

{\large
\noindent καὶ γὰρ οἰκίας καὶ ἀγροὺς καὶ ἀνδράποδα καὶ βοσκήματα καὶ σκεύη \\
\tabto{2em} κτωμένους τε ἐπιμελῶς \\
ὁρᾶν ἔφη \\
καὶ τὰ ὄντα σῴζειν \\
\tabto{2em} πειρωμένους, \\
\underline{φίλον} δέ,\\
\tabto{2em} \underline{ὃ μέγιστον ἀγαθὸν εἶναί} \\
\tabto{3em} φασιν, \\
ὁρᾶν ἔφη \\
\tabto{2em} τοὺς πολλοὺς \\
\tabto{3em} οὔτε ὅπως κτήσωνται \\
\tabto{4em} φροντίζοντας \\
\tabto{3em} οὔτε ὅπως οἱ ὄντες αὐτοῖς \\
\tabto{4em} σῴζωνται. \\

}

\begin{description}[noitemsep]
\item[οἰκίας] §~90
\item[ἀγροὺς] §~82
\item[ἀνδράποδα] §~82
\item[βοσκήματα] §~123
\item[σκεύη] §~153
\item[κτωμένους] κτάομαι nastojati steći; a. pl. m. r. ptc. prez. medpas.
\item[ἐπιμελῶς] §~204; ovisno o κτωμένους
\item[ὄντα] εἰμί biti; a. pl. s. r. ptc.; τὰ ὄντα (sc.\ αὐτοῖς, posvojni dativ §~412.2): ono što imaju; objekt od σῴζειν
\item[σῴζειν] σῴζω sačuvati; inf. prez. akt.
\item[πειρωμένους] πειράομαι truditi se; a. pl. m. r. ptc. prez. medpas.
\item[φίλον] §~82
\item[ὃ] §~215; relativ je u rodu predikatnog imena, kao lat. \textit{amicum, quod bonum esse dicunt}
\item[μέγιστον] §~200
\item[ἀγαθὸν] §~103; ἀγαθὸν je ovdje poimeničeni pridjev, ali nema član jer je predikatno ime
\item[εἶναί] εἰμί biti; infinitiv; dio konstrukcije A+I
\item[φασιν] φημί reći; 3. l. ind. prez. akt.; enklitika, §~40
\item[οὔτε\dots\  οὔτε] §~513.4; sastavni veznik
\item[ὅπως κτήσωνται\dots\  ὅπως\dots\  σῴζωνται] u namjernim rečenicama koje ovise o \textit{verba curandi} (ovdje particip φροντίζοντας), osim indikativa futura, može iza glavnog vremena stajati konjunktiv §~472
\item[κτήσωνται] κτάομαι steći; 3. l. pl. konj. aor. med.
\item[φροντίζοντας] φροντίζω brinuti se; a. pl. m. r. ptc. prez. akt.; predikatni particip iza \textit{verba sentiendi} \textgreek[variant=ancient]{(ὁρᾶν\dots\ τοὺς πολλοὺς\dots\ φροντίζοντας)} §~502
\item[ὄντες] εἰμί biti; n. pl. m. r. ptc.; οἱ ὄντες poimeničenje članom §~373
\item[αὐτοῖς] §~207; οἱ ὄντες αὐτοῖς (sc.\ φίλοι) posvojni dativ §~412.2
\item[σῴζωνται] σῴζω sačuvati; 3. l. pl. konj. prez. medpas.

\end{description}

%\newpage

%3 

{\large
\noindent ἀλλὰ καὶ \uuline{καμνόντων φίλων τε καὶ οἰκετῶν} \\
ὁρᾶν τινας \\
ἔφη \\
\tabto{2em} τοῖς μὲν οἰκέταις \\
\tabto{2em} καὶ ἰατροὺς εἰσάγοντας \\
\tabto{2em} καὶ τἆλλα \\
\tabto{3em} τὰ πρὸς ὑγίειαν \\
\tabto{2em} ἐπιμελῶς παρασκευάζοντας, \\
\tabto{2em} τῶν δὲ φίλων ὀλιγωροῦντας, \\
\uuline{ἀποθανόντων} τε \uuline{ἀμφοτέρων} \\
\tabto{2em} ἐπὶ μὲν τοῖς οἰκέταις \\
\tabto{3em} ἀχθομένους τε καὶ ζημίαν ἡγουμένους, \\
\tabto{2em} ἐπὶ δὲ τοῖς φίλοις \\
\tabto{3em} οὐδὲν οἰομένους ἐλαττοῦσθαι, \\
καὶ τῶν μὲν ἄλλων κτημάτων \\
\tabto{2em} οὐδὲν ἐῶντας ἀθεράπευτον οὐδ' ἀνεπίσκεπτον, \\
τῶν δὲ φίλων \\
\tabto{2em} ἐπιμελείας δεομένων \\
ἀμελοῦντας. \\

}

\begin{description}[noitemsep]
\item[ἀλλὰ καὶ] štoviše (lat. \textit{quin etiam)}
\item[καμνόντων] κάμνω biti bolestan; g. pl. m. r. ptc. prez. akt.
\item[τε καὶ] §~513.2
\item[φίλων\dots\  οἰκετῶν] §~82; §~100; ovisno o καμνόντων; GA
\item[ὁρᾶν\dots\  ἔφη ] subjekt infinitiva ne razlikuje se od subjekta glavne rečenice, pa se ne izriče §~491.2
\item[τινας] §~217
\item[τοῖς\dots\  οἰκέταις] §~100; dativ interesa §~412
\item[ἰατροὺς] §~82
\item[εἰσάγοντας] εἰσάγω dovoditi; a. pl. m. r. ptc. prez. akt.; predikatni particip iza \textit{verba sentiendi} \textgreek[variant=ancient]{(ὁρᾶν\dots\  τινας\dots\ εἰσάγοντας\dots\  παρασκευάζοντας\dots\  ὀλιγωροῦντας)} §~502
\item[τἆλλα] = τὰ ἄλλα, kraza §~66; §~82; atributni položaj τὰ πρὸς ὑγίειαν §~375
\item[πρὸς ὑγίειαν] §~435.C; §~97
\item[παρασκευάζοντας] παρασκευάζω pripremati; a. pl. m. r. ptc. prez. akt.
\item[ὀλιγωροῦντας] ὀλιγορέω τινος zanemarivati koga; a. pl. m. r. ptc. prez. akt.
\item[ἀποθανόντων] ἀποθνῄσκω umirati; g. pl. m. r. ptc. aor. akt.; ἀποθανόντων\dots\  ἀμφοτέρων GA §~504
\item[ἀμφοτέρων] §~224.3
\item[ἀχθομένους] ἄχθομαι ἐπί τινι biti tužan zbog koga; a. pl. m. r. ptc. prez. medpas.; predikatni particip iza \textit{verba sentiendi} \textgreek[variant=ancient]{(ὁρᾶν\dots\  τινας\dots\  ἀχθομένους\dots\  ἡγουμένους\dots\  οἰομένους\dots\  ἐῶντας\dots\  ἀμελοῦντας)} §~502
\item[ζημίαν] §~90
\item[ἡγουμένους] ἡγέομαι (rekcija: τινά τινα) što za što držati; a. pl. m. r. ptc. prez. medpas.; ἡγουμένους ζημίαν (sc.\ τὸν θάνατον)
\item[οὐδὲν] §~391; priložni akuzativ (prilog): ništa, nikako \textit{(nihil)}
\item[οἰομένους] οἴομαι misliti; a. pl. m. r. ptc. prez. medpas.; dopunjuje ga  A+I, uz ispuštanje subjektnog akuzativa ako je subjekt isti
\item[ἐλαττοῦσθαι] ἐλαττόομαι ἐπί τινι trpjeti štetu zbog koga; inf. prez. medpas.
\item[τῶν\dots\  ἄλλων κτημάτων] §~123; §~212; §~371; dijelni genitiv §~395
\item[οὐδὲν ] §~224 
\item[ἐῶντας] ἐάω ostavljati; a. pl. m. r. ptc. prez. akt.
\item[ἀθεράπευτον\dots\  ἀνεπίσκεπτον] složeni pridjevi §~106; οὐδὲν ἀθεράπευτον\dots\ ἀνεπίσκεπτον dvostruka negacija pojačava učinak (litota)
\item[οὐδ'] οὐδ' = οὐδέ; elizija §~68
\item[ἐπιμελείας] §~97
\item[δεομένων] δέομαί τινος trebati što; g. pl. m. r. ptc. prez. medpas.; ovisno o ἐπιμελείας
\item[ἀμελοῦντας] ἀμελέω τινος zanemarivati koga; a. pl. m. r. ptc. prez. akt.; ovisno o τῶν\dots\  φίλων\dots\  δεομένων
\end{description}


%kraj
