% Unesi korekture NČ, NZ 2019-09-23
\section*{O tekstu}

Kada je 349.\ pr.~Kr.\ Filip Makedonski napao Olint, grad na Halkidici koji je bio sklopio savez s Atenom, Demosten je održao tri govora potičući Atenjane da interveniraju. Usprkos Demostenovu zauzimanju, pravodobna i priladna akcija Atene je izostala. Drugi olintski govor reakcija je na oklijevanje Atenjana, koji ne žele uputiti Olinćanima obećanu pomoć. Demosten ih ohrabruje i dokazuje da su Filipa napustili saveznici te da Filip sam nema više dovoljno snage. U ovom odlomku Demosten opisuje kako se Filip ponaša prema stranim vojnicima i suradnicima ističući kao negativan primjer državnog roba Kaliju.

%\newpage

\section*{Pročitajte naglas grčki tekst.}
Dem. Olynthiaca II 18
%Naslov prema izdanju

\medskip

{\large
\begin{greek}
\noindent Εἰ μὲν γάρ τις ἀνήρ ἐστιν ἐν αὐτοῖς οἷος ἔμπειρος πολέμου καὶ ἀγώνων, τούτους μὲν φιλοτιμίᾳ πάντας ἀπωθεῖν αὐτὸν ἔφη, βουλόμενον πάνθ' αὑτοῦ δοκεῖν εἶναι τἄργα (πρὸς γὰρ αὖ τοῖς ἄλλοις καὶ  τὴν φιλοτιμίαν ἀνυπέρβλητον εἶναι)· εἰ δέ τις σώφρων ἢ δίκαιος ἄλλως, τὴν καθ' ἡμέραν ἀκρασίαν τοῦ βίου καὶ μέθην καὶ κορδακισμοὺς οὐ δυνάμενος φέρειν, παρεῶσθαι καὶ ἐν οὐδενὸς εἶναι μέρει τὸν τοιοῦτον. λοιποὺς δὴ περὶ αὐτὸν εἶναι λῃστὰς καὶ κόλακας καὶ τοιούτους ἀνθρώπους οἵους μεθυσθέντας ὀρχεῖσθαι τοιαῦθ' οἷ' ἐγὼ νῦν ὀκνῶ πρὸς ὑμᾶς ὀνομάσαι. δῆλον δ' ὅτι ταῦτ' ἐστὶν ἀληθῆ· καὶ γὰρ οὓς ἐνθένδε πάντες ἀπήλαυνον ὡς πολὺ τῶν θαυματοποιῶν ἀσελγεστέρους ὄντας, Καλλίαν ἐκεῖνον τὸν δημόσιον καὶ τοιούτους ἀνθρώπους, μίμους γελοίων καὶ ποιητὰς αἰσχρῶν ᾀσμάτων, ὧν εἰς τοὺς συνόντας ποιοῦσιν εἵνεκα τοῦ γελασθῆναι, τούτους ἀγαπᾷ καὶ περὶ αὑτὸν ἔχει.

\end{greek}

}

\section*{Analiza i komentar}


%1

{\large
\begin{greek}
\noindent Εἰ μὲν γάρ \\
τις ἀνήρ ἐστιν \\
\tabto{2em} ἐν αὐτοῖς \\
οἷος \\
\tabto{2em} ἔμπειρος \\
\tabto{4em} πολέμου καὶ ἀγώνων, \\
τούτους μὲν \\
\tabto{2em} φιλοτιμίᾳ \\
πάντας \underline{ἀπωθεῖν αὐτὸν} \\
ἔφη, \\
\tabto{2em} \underline{βουλόμενον} \\
\tabto{4em} \underline{πάνθ'} αὑτοῦ \underline{δοκεῖν εἶναι τἄργα} \\
(πρὸς γὰρ αὖ τοῖς ἄλλοις \\
\tabto{2em} καὶ \underline{τὴν φιλοτιμίαν ἀνυπέρβλητον εἶναι})· \\
εἰ δέ τις σώφρων ἢ δίκαιος ἄλλως, \\
\tabto{2em} τὴν \\
\tabto{4em} καθ' ἡμέραν \\
\tabto{2em} ἀκρασίαν τοῦ βίου \\
\tabto{2em} καὶ μέθην \\
\tabto{2em} καὶ κορδακισμοὺς \\
οὐ δυνάμενος φέρειν, \\
\underline{παρεῶσθαι} \\
καὶ ἐν οὐδενὸς \underline{εἶναι} μέρει \\
\underline{τὸν τοιοῦτον}.\\

\end{greek}
}

\begin{description}[noitemsep]
\item[μὲν γάρ\dots\  δέ\dots] odnosno\dots, naime\dots; γάρ uz μὲν\dots\  δέ\dots\ označava objašnjavanje
\item[Εἰ μὲν\dots, τούτους μὲν\dots· εἰ δέ\dots] udvajanje  μὲν u koordinaciji  μὲν\dots\ δέ\dots; sadržaj prvog dijela koordinacije odviše je složen da bi „stao'' u jednu surečenicu
\item[γάρ τις] §~40; §~217
\item[ἀνήρ ἐστιν] §~40
\item[ἀνήρ ] §~149
\item[ἐστιν] εἰμί \textit{ovdje} ima, postoji; 3. l. sg. ind. prez. akt.
\item[ἐν αὐτοῖς] §~426; §~207
\item[οἷος ἔμπειρος ] nekako\dots, prilično\dots; οἷος modificira sljedeći pridjev; §~217; §~106
\item[πολέμου ] §~82
\item[ἀγώνων] §~131
\item[τούτους ] §~213.2
\item[φιλοτιμίᾳ ] §~90
\item[πάντας ] §~193
\item[ἀπωθεῖν ] ἀπωθέω odgurnuti, otjerati; inf. prez. akt.
\item[αὐτὸν ] sc.\ kralj Filip Makedonski; §~207
\item[ἔφη] φημί reći; 3. l. sg. impf. akt.; ovaj glagol otvara mjesto konstrukciji A+I; subjekt je naznačen neodređenom zamjenicom τινος u prethodnom poglavlju (dakle, netko je rekao)
\item[βουλόμενον ] βούλομαι htjeti; a. sg. m. r. ptc. prez. med.; glagol otvara mjesto dopuni u infinitivu
\item[πάνθ' αὑτοῦ ] §~68; §~193; §~209.1; subjektni ili posvojni genitiv uz εἶναι §~393
\item[δοκεῖν ] δοκέω činiti se; inf. prez. akt.; glagol otvara mjesto A+I
\item[εἶναι ] εἰμί biti; inf. prez. akt.
\item[τἄργα] §~66; §~82
\item[πρὸς\dots\ τοῖς ἄλλοις] sc.\ uz druge mane; §~435; §~212.a; §~373
\item[γὰρ ] naime\dots; čestica naglašava objašnjavanje
\item[τὴν φιλοτιμίαν ] §~90
\item[ἀνυπέρβλητον ] §~106
\item[εἶναι] εἰμί biti; inf. prez. akt.
\item[ἀνυπέρβλητον εἶναι] imenski predikat, Smyth~909
\item[δέ τις ] §~40; §~217
\item[σώφρων] §~131
\item[δίκαιος ] §~103
\item[ἄλλως] §~204
\item[τὴν καθ' ἡμέραν ἀκρασίαν ] §~68; §~429; §~90; atributni položaj prijedložnog izraza §~375
\item[τοῦ βίου] §~82
\item[μέθην] §~90
\item[κορδακισμοὺς ] §~82
\item[δυνάμενος ] δύναμαι moći; n. sg. m. r. ptc. prez. med.; glagol otvara mjesto dopuni u infinitivu
\item[φέρειν] φέρω nositi, podnositi; inf. prez. akt.
\item[παρεῶσθαι] παρωθέω odgurnuti, gurati u stranu; inf. perf. medpas.
\item[ἐν\dots\ μέρει] §~426; §~153
\item[οὐδενὸς] §~224.2
\item[εἶναι ] εἰμί biti; inf. prez. akt.
\item[ἐν οὐδενὸς εἶναι μέρει] imenski predikat, Smyth 909
\item[τὸν τοιοῦτον] §~213.4; §~373

\end{description}


{\large
\begin{greek}
\noindent \underline{λοιποὺς} δὴ \\
\tabto{2em} περὶ αὐτὸν \\
\underline{εἶναι} λῃστὰς \\
\tabto{2em} καὶ κόλακας \\
\tabto{2em} καὶ τοιούτους ἀνθρώπους \\
\tabto{4em} οἵους \\
\tabto{4em} \underline{μεθυσθέντας ὀρχεῖσθαι} \\
\tabto{6em} τοιαῦθ' \\
\tabto{6em} οἷ' \\
\tabto{8em} ἐγὼ νῦν ὀκνῶ \\
\tabto{10em} πρὸς ὑμᾶς \\
\tabto{8em} ὀνομάσαι.\\

\end{greek}
}

\begin{description}[noitemsep]
\item[λοιποὺς ] §~103
\item[δὴ ] čestica naglašava pridjev koji izražava (neodređenu) količinu
\item[περὶ αὐτὸν ] §~433; §~207
\item[εἶναι ] εἰμί biti; inf. prez. akt.; i dalje ovisno o ἔφη iz prošle rečenice
\item[λῃστὰς] §~100
\item[κόλακας] §~115
\item[τοιούτους ] §~213.4
\item[ἀνθρώπους ] §~82
\item[εἶναι] \textbf{λῃστὰς καὶ κόλακας καὶ τοιούτους ἀνθρώπους} imenski predikat, Smyth 909
\item[οἵους ] §~219; οἷος umjesto ὥστε uvodi posljedičnu rečenicu §~473 bilj. 4
\item[μεθυσθέντας ] μεθύω opijati se; a. pl. m. r. ptc. aor. pas.
\item[ὀρχεῖσθαι ] ὀρχέομαι plesati; inf. prez. medpas.
\item[τοιαῦθ' οἷ' ἐγὼ ] §~68
\item[τοιαῦθ' ] §~213.4
\item[οἷ' ] §~219
\item[ἐγὼ ] §~205
\item[ὀκνῶ ] ὀκνέω oklijevati; 1. l. sg. ind. prez. akt.
\item[πρὸς ὑμᾶς ] §~435; §~205
\item[ὀνομάσαι] ὀνομάζω imenovati; inf. aor. akt.

\end{description}

%3 itd

{\large
\begin{greek}
\noindent δῆλον δ' ὅτι \\
\tabto{2em} ταῦτ' ἐστὶν ἀληθῆ· \\
καὶ γὰρ \\
οὓς \\
\tabto{2em} ἐνθένδε \\
πάντες \\
ἀπήλαυνον \\
\tabto{2em} ὡς \\
\tabto{4em} πολὺ \\
\tabto{6em} τῶν θαυματοποιῶν \\
\tabto{4em} ἀσελγεστέρους ὄντας, \\
Καλλίαν \\
\tabto{2em} ἐκεῖνον τὸν δημόσιον \\
καὶ τοιούτους ἀνθρώπους, \\
\tabto{2em} μίμους γελοίων \\
\tabto{2em} καὶ ποιητὰς αἰσχρῶν ᾀσμάτων, \\
\tabto{4em} ὧν \\
\tabto{6em} εἰς τοὺς συνόντας \\
\tabto{4em} ποιοῦσιν \\
\tabto{6em} εἵνεκα τοῦ γελασθῆναι, \\
τούτους \\
ἀγαπᾷ \\
καὶ \\
\tabto{2em} περὶ αὑτὸν \\
ἔχει.\\

\end{greek}
}

\begin{description}[noitemsep]
\item[δῆλον δ' ὅτι ] §~68; §~103; čestica δέ povezuje rečenicu s prethodnom: a\dots; δῆλον ὅτι jasno je da\dots (fraza)
\item[ταῦτ' ἐστὶν ] §~40; §~68; §~213.2; εἰμί biti; 3. l. sg. ind. prez. akt.
\item[ἀληθῆ] §~153
\item[ἐστὶν ἀληθῆ] imenski predikat, Smyth 909
\item[καὶ γὰρ ] §~517
\item[οὓς] §~215
\item[πάντες ] §~193
\item[ἀπήλαυνον ] ἀπελαύνω otjerati; 3. l. pl. impf. akt.
\item[ὡς\dots\ ὄντας] ὡς uvodi adverbijalni particip s uzročnim značenjem §~503.2; εἰμί biti; a. pl. m. r. ptc. prez. akt.
\item[πολὺ ] §~196
\item[τῶν θαυματοποιῶν ἀσελγεστέρους] genitiv usporedbe §~404; §~106; §~197; §~373
\item[Καλλίαν ] §~100
\item[ἐκεῖνον τὸν δημόσιον] §~212; §~103; §~373
\item[τοιούτους ] §~213.4
\item[ἀνθρώπους] §~82
\item[μίμους ] §~82
\item[γελοίων] §~103
\item[ποιητὰς] §~100
\item[αἰσχρῶν ] §~103
\item[ᾀσμάτων] §~123
\item[ὧν] §~215; genitiv umjesto akuzativa, asimilacija relativa §~444
\item[εἰς τοὺς συνόντας ] §~419; σύνειμι biti dio društva, biti prisutan; a. pl. m. r. ptc. prez. akt.; poimeničenje članom §~373
\item[ποιοῦσιν ] ποιέω raditi, sastavljati; 3. l. pl. ind. prez.
\item[εἵνεκα τοῦ γελασθῆναι] §~417; γελάω smijati se; inf. aor. pas.; poimeničenje članom §~373
\item[τούτους ] §~213.2
\item[ἀγαπᾷ] ἀγαπάω voljeti; 3. l. sg. ind. prez. akt.
\item[περὶ αὑτὸν ] §~433; §~209.1
\item[ἔχει] ἔχω imati; 3. l. sg. ind. prez. akt.

\end{description}

%kraj
