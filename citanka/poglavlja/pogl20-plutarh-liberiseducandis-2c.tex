% Unesi korekture NČ 2019-09-20
%\section*{O autoru}



\section*{O tekstu}

Ovo je prvi esej u Plutarhovoj zbirci \textit{Moralia}. Na osnovi složene interne i eksterne argumentacije, filolozi osporavaju Plutarhovo autorstvo toga teksta. Osvrćući se na odgojne prilike svojeg doba, autor priznaje različitost talenata, ali inzistira na vrijednosti vježbanja: tjelesnoga, vojničkog, ali i vježbanja filozofije, koje je glavni cilj obrazovanja. Kritiziraju se roditelji koji ne mare za obrazovanje djece i oni koji ne žele platiti primjerene honorare za valjane učitelje. Upozorava se i na neobuzdanost mladića i na opasnost laskanja.

Ovdje doneseni odlomak dio je početne pohvale vježbanja. Tvrdnja da ono može kompenzirati i nedostatke (dok, s druge strane, nedostatak vježbe može upropastiti i najveći talent) potkrepljuje se usporedbama s pojavama iz prirodnog svijeta i procesima iz svakodnevnog života.

%\newpage

\section*{Pročitajte naglas grčki tekst.}
Plut. De liberis educandis 2c
%Naslov prema izdanju

\medskip

{\large
\begin{greek}
\noindent Εἰ δέ τις οἴεται τοὺς οὐκ εὖ πεφυκότας μαθήσεως καὶ μελέτης τυχόντας ὀρθῆς πρὸς ἀρετὴν οὐκ ἂν τὴν τῆς φύσεως ἐλάττωσιν εἰς τοὐνδεχόμενον ἀναδραμεῖν, ἴστω πολλοῦ, μᾶλλον δὲ τοῦ παντὸς διαμαρτάνων. φύσεως μὲν γὰρ ἀρετὴν διαφθείρει ῥᾳθυμία, φαυλότητα δ' ἐπανορθοῖ διδαχή· καὶ τὰ μὲν ῥᾴδια τοὺς ἀμελοῦντας φεύγει, τὰ δὲ χαλεπὰ ταῖς ἐπιμελείαις ἁλίσκεται. καταμάθοις δ' ἂν ὡς ἀνύσιμον πρᾶγμα καὶ τελεσιουργὸν ἐπιμέλεια καὶ πόνος ἐστίν, ἐπὶ πολλὰ τῶν γιγνομένων ἐπιβλέψας. σταγόνες μὲν γὰρ ὕδατος πέτρας κοιλαίνουσι, σίδηρος δὲ καὶ χαλκὸς ταῖς ἐπαφαῖς τῶν χειρῶν ἐκτρίβονται, οἱ δ' ἁρμάτειοι τροχοὶ πόνῳ καμφθέντες οὐδ' ἂν εἴ τι γένοιτο τὴν ἐξ ἀρχῆς δύναιντ' ἀναλαβεῖν εὐθυωρίαν· τάς γε μὴν καμπύλας τῶν ὑποκριτῶν βακτηρίας ἀπευθύνειν ἀμήχανον, ἀλλὰ τὸ παρὰ φύσιν τῷ πόνῳ τοῦ κατὰ φύσιν ἐγένετο κρεῖττον. καὶ μόνα ἆρα ταῦτα τὴν τῆς ἐπιμελείας ἰσχὺν διαδείκνυσιν; οὔκ, ἀλλὰ καὶ μυρί' ἐπὶ μυρίοις. ἀγαθὴ γῆ πέφυκεν· ἀλλ' ἀμεληθεῖσα χερσεύεται, καὶ ὅσῳ τῇ φύσει βελτίων ἐστί, τοσούτῳ μᾶλλον ἐξαργηθεῖσα δι' ἀμέλειαν ἐξαπόλλυται.

\end{greek}
}

\section*{Analiza i komentar}

%1

{\large
\begin{greek}
\noindent Εἰ δέ τις οἴεται \\
\tabto{2em} \underline{τοὺς οὐκ εὖ πεφυκότας} \\
\tabto{4em} μαθήσεως καὶ μελέτης \underline{τυχόντας} ὀρθῆς \\
\tabto{6em} πρὸς ἀρετὴν \\
\tabto{4em} οὐκ ἂν \\
\tabto{6em} τὴν τῆς φύσεως ἐλάττωσιν \\
\tabto{8em} εἰς τοὐνδεχόμενον \\
\tabto{6em} \underline{ἀναδραμεῖν}, \\
ἴστω \\
\tabto{2em} πολλοῦ, \\
\tabto{2em} μᾶλλον δὲ τοῦ παντὸς \\
\tabto{2em} διαμαρτάνων.\\

\end{greek}
}

\begin{description}[noitemsep]
\item[δέ τις] §~40; čestica δέ povezuje rečenicu s (ovdje izostavljenom) prethodnom: a\dots
\item[τις] §~217
\item[Εἰ δέ τις οἴεται\dots] \textbf{ἴστω πολλοῦ\dots}\ veznik uvodi realnu pogodbenu rečenicu
\item[οἴεται ] οἴομαι smatrati; 3. l. sg. ind. prez. medpas.; otvara mjesto dopuni u A+I
\item[τοὺς\dots\ πεφυκότας ] φύω \textit{ovdje} po prirodi biti talentiran (za nešto), LSJ φύω II.2; a. pl. m. r. ptc. perf. akt.
\item[μαθήσεως] §~165
\item[μελέτης ] §~90
\item[τυχόντας ] τυγχάνω τινός postići nešto; a. pl. m. r. ptc. aor. akt.
\item[ὀρθῆς] §~103
\item[πρὸς ἀρετὴν] §~435, §~90
\item[ἂν\dots\ ἀναδραμεῖν] ἀνατρέχω (figurativno) ispraviti; inf. aor. akt., dio A+I; ἂν označava mogućnost: da bi mogao\dots; §~506
\item[τὴν\dots\ ἐλάττωσιν ] §~165
\item[τῆς φύσεως] §~165
\item[εἰς τοὐνδεχόμενον] §~419, §~66; ἐνδέχομαι biti moguće; a. sg. s. r. ptc. prez. medpas.; εἰς τὸ ἐνδεχόμενον koliko je moguće (fraza); supstantiviranje članom §~373
\item[ἴστω] οἶδα znati; glagol otvara mjesto participu, v. LSJ οἶδα A.3, §~502.3; 3. l. sg. imp. prez. akt.
\item[πολλοῦ] §~196, §~407.1.B
\item[μᾶλλον δὲ] §~204.3; čestica δέ povezuje surečenicu s prethodnom: a\dots; §~396.e; supstantiviranje članom §~373
\item[τοῦ παντὸς] §~193, §~379
\item[διαμαρτάνων] διαμαρτάνω promašiti, pogriješiti; n. sg. m. r. ptc. prez. akt.

\end{description}


%2

{\large
\begin{greek}
\noindent φύσεως μὲν γὰρ ἀρετὴν \\
\tabto{2em} διαφθείρει ῥᾳθυμία, \\
φαυλότητα δ' \\
\tabto{2em} ἐπανορθοῖ διδαχή· \\
καὶ τὰ μὲν ῥᾴδια \\
\tabto{2em} τοὺς ἀμελοῦντας φεύγει, \\
τὰ δὲ χαλεπὰ \\
\tabto{2em} ταῖς ἐπιμελείαις ἁλίσκεται.\\

\end{greek}
}

\begin{description}[noitemsep]
\item[φύσεως] §~165
\item[μὲν\dots\ δ' ] §~68; koordinacija pomoću para čestica: a\dots\ §~515, §~519
\item[μὲν γὰρ\dots\ δ'\dots] naime\dots\ a\dots; γὰρ kao apozicija
\item[ἀρετὴν ] §~90
\item[διαφθείρει ] διαφθείρω uništavati; 3. l. sg. ind. prez. akt.
\item[ῥᾳθυμία] §~90
\item[φαυλότητα ] §~123
\item[ἐπανορθοῖ ] ἐπανορθόω popraviti; 3. l. sg. ind. prez. akt.
\item[διδαχή] §~90
\item[τὰ μὲν\dots\ τὰ δὲ\dots] koordinacija pomoću para čestica: ono što je\dots\ a ono što je\dots\ §~515, §~519
\item[τὰ\dots\ ῥᾴδια ] §~103; supstantiviranje članom §~373
\item[τοὺς ἀμελοῦντας ] ἀμελέω ne brinuti za, zanemariti; a. pl. m. r. ptc. prez. akt.; supstantiviranje članom §~373
\item[φεύγει] φεύγω τινά pobjeći nekome, izmaknuti nekome; 3. l. sg. ind. prez. akt.
\item[τὰ\dots\ χαλεπὰ] §~103; supstantiviranje članom §~373
\item[ταῖς ἐπιμελείαις ] §~90
\item[ἁλίσκεται] ἁλίσκομαι postizati, dosegnuti; 3. l. sg. ind. prez. medpas.

\end{description}


%3

{\large
\begin{greek}
\noindent καταμάθοις δ' ἂν \\
\tabto{2em} ὡς ἀνύσιμον πρᾶγμα \\
\tabto{4em} καὶ τελεσιουργὸν \\
\tabto{2em} ἐπιμέλεια καὶ πόνος ἐστίν, \\
ἐπὶ πολλὰ τῶν γιγνομένων \\
\tabto{2em} ἐπιβλέψας.\\

\end{greek}
}

\begin{description}[noitemsep]
\item[καταμάθοις δ' ἂν] možeš shvatiti\dots; καταμανθάνω shvatiti, razumjeti; 2. l. sg. opt. aor. akt., u značenju potencijala sadašnjega §~464, §~489.b.5
\item[δ' ] čestica povezuje rečenicu s prethodnom: a\dots
\item[ἀνύσιμον πρᾶγμα] §~103, §~123; imenski dio predikata
\item[τελεσιουργὸν ] §~103
\item[ἐπιμέλεια] §~90
\item[πόνος ἐστίν] §~40, §~82
\item[πρᾶγμα\dots\ ἐστίν] imenski predikat Smyth 909
\item[ἐστίν] εἰμί biti; 3. l. sg. ind. prez. akt.; kopula kao dio imenskog predikata
\item[ἐπὶ πολλὰ ] §~436, §~196
\item[τῶν γιγνομένων ] γίγνομαι dogoditi se; g. pl. s. r. ptc. prez. medpas.; \textit{supstantivirano} događaji, činjenice; supstantiviranje članom §~373
\item[ἐπιβλέψας] ἐπιβλέπω promotriti; n. sg. m. r. ptc. aor. akt.; particip u priložnom značenju: ako\dots

\end{description}

%12

{\large
\begin{greek}
\noindent σταγόνες μὲν γὰρ ὕδατος \\
\tabto{2em} πέτρας κοιλαίνουσι, \\
σίδηρος δὲ καὶ χαλκὸς \\
\tabto{2em} ταῖς ἐπαφαῖς τῶν χειρῶν ἐκτρίβονται, \\
οἱ δ' ἁρμάτειοι τροχοὶ \\
\tabto{2em} πόνῳ καμφθέντες \\
\tabto{4em} οὐδ' ἂν εἴ τι γένοιτο \\
\tabto{6em} τὴν ἐξ ἀρχῆς δύναιντ' ἀναλαβεῖν εὐθυωρίαν· \\
τάς γε μὴν καμπύλας τῶν ὑποκριτῶν βακτηρίας \\
\tabto{2em} ἀπευθύνειν ἀμήχανον, \\
ἀλλὰ τὸ παρὰ φύσιν \\
\tabto{2em} τῷ πόνῳ \\
τοῦ κατὰ φύσιν \\
ἐγένετο κρεῖττον.\\

\end{greek}
}

\begin{description}[noitemsep]
\item[σταγόνες ] §~131
\item[σταγόνες μὲν\dots] \textbf{σίδηρος δὲ\dots\ οἱ δ' ἁρμάτειοι\dots} koordinacija pomoću čestica: a\dots\ a\dots\ §~515, §~519
\item[μὲν γὰρ] naime\dots; γὰρ kao apozicija
\item[ὕδατος] §~128
\item[πέτρας ] §~90
\item[κοιλαίνουσι] κοιλαίνω dubiti; 3. l. pl. ind. prez. akt.
\item[σίδηρος] §~82
\item[χαλκὸς ] §~82
\item[ταῖς ἐπαφαῖς ] §~90
\item[τῶν χειρῶν ] §~150
\item[ἐκτρίβονται] ἐκτρίβω izlizati, izglačati; 3. l. pl. ind. prez. medpas.
\item[ἁρμάτειοι τροχοὶ ] §~103, §~82
\item[πόνῳ ] §~82
\item[καμφθέντες ] κάμπτω svinuti; n. pl. m. r. ptc. aor. pas.
\item[οὐδ' ἂν] §~68
\item[οὐδ' ἂν εἴ τι γένοιτο\dots] \textbf{(οἱ τροχοὶ) δύναιντ' ἀναλαβεῖν} pogodbena potencijalna rečenica
\item[ἂν\dots\ δύναιντ'] δύναμαι moći; otvara mjesto dopuni u infinitivu; 3. l. pl. opt. prez. medpas.; u značenju potencijala sadašnjega §~464, §~489.b.5
\item[εἴ τι] §~40
\item[τι ] §~217
\item[γένοιτο ] γίγνομαι dogoditi se; 3. l. sg. opt. aor. med.
\item[τὴν\dots\ εὐθυωρίαν] §~90
\item[ἐξ ἀρχῆς] §~424, §~90
\item[ἀναλαβεῖν] ἀναλαμβάνω vratiti, obnoviti; inf. aor. akt.; infinitiv kao obavezna dopuna glagola δύναμαι
\item[τάς γε ] §~40
\item[γε μὴν] a pogotovo\dots; ova kombinacija čestica progresivno uvodi sljedeći, još jači argument
\item[τάς\dots\ καμπύλας\dots\  βακτηρίας] §~103, §~90
\item[τῶν ὑποκριτῶν ] §~100
\item[ἀπευθύνειν ] ἀπευθύνω izravnati; inf. prez. akt.
\item[ἀμήχανον] §~103
\item[τὸ παρὰ φύσιν ] §~165, §~434; supstantiviranje članom §~373
\item[τῷ πόνῳ ] §~82
\item[τοῦ κατὰ φύσιν ] §~165, §~429; supstantiviranje članom §~373
\item[ἐγένετο ] γίγνομαι postati; 3. l. sg. ind. aor. med.
\item[κρεῖττον] §~202

\end{description}

%5

{\large
\begin{greek}
\noindent καὶ μόνα ἆρα ταῦτα \\
τὴν τῆς ἐπιμελείας ἰσχὺν \\
διαδείκνυσιν;\\

\end{greek}
}

\begin{description}[noitemsep]
\item[μόνα\dots\ ταῦτα] §~103, §~213.2
\item[τὴν\dots\ ἰσχὺν] §~173
\item[τῆς ἐπιμελείας] §~90
\item[διαδείκνυσιν] διαδείκνυμι pokazati; 3. l. sg. ind. prez. akt.

\end{description}

%6

{\large
\begin{greek}
\noindent οὔκ, ἀλλὰ καὶ μυρί' ἐπὶ μυρίοις.\\

\end{greek}
}

\begin{description}[noitemsep]
\item[μυρί' ἐπὶ] §~68
\item[μυρί'] §~223
\item[ἐπὶ μυρίοις] §~436, §~223

\end{description}

%7

{\large
\begin{greek}
\noindent ἀγαθὴ γῆ πέφυκεν· \\
ἀλλ' ἀμεληθεῖσα χερσεύεται, \\
καὶ ὅσῳ \\
\tabto{4em} τῇ φύσει \\
\tabto{2em} βελτίων ἐστί, \\
τοσούτῳ μᾶλλον \\
\tabto{2em} ἐξαργηθεῖσα \\
\tabto{4em} δι' ἀμέλειαν \\
\tabto{2em} ἐξαπόλλυται.\\

\end{greek}
}

\begin{description}[noitemsep]
\item[ἀγαθὴ γῆ] §~103, §~90
\item[πέφυκεν] φύω roditi, \textit{pasivno} rasti; 3. l. sg. ind. perf. akt.; aktivni oblik u pasivnom značenju
\item[ἀλλ' ἀμεληθεῖσα ] §~68
\item[ἀμεληθεῖσα] ἀμελέω zanemariti, ne njegovati; n. sg. ž. r. ptc. aor. pas.
\item[χερσεύεται] χερσεύω učiniti jalovim, \textit{pasivno} postati jalov; 3. l. sg. ind. prez. medpas.
\item[ὅσῳ\dots\ τοσούτῳ] koliko\dots\ toliko\dots\ (korelativno); §~219, §~213.4
\item[τῇ φύσει] §~165
\item[βελτίων ἐστί] §~40; imenski predikat Smyth 909
\item[βελτίων ] §~202
\item[ἐστί] εἰμί biti; 3. l. sg. ind. prez. akt.
\item[ἐξαργηθεῖσα] ἐξαργέω zamrijeti, \textit{pasivno} postati zapušten; n. sg. ž. r. ptc. aor. pas.
\item[δι' ἀμέλειαν] §~68, §~90, §~428
\item[ἐξαπόλλυται] ἐξαπόλλυμι potpuno uništiti, \textit{pasivno} potpuno propasti; 3. l. sg. ind. prez. medpas.

\end{description}

%kraj
