% redaktura NZ, unio NJ
%\section*{O autoru}



\section*{O tekstu}

Djelo \textit{Razgovori bogova} \textgreek[variant=ancient]{(Θεῶν διάλογοι)} zbirka je sastavljena od 26 samostalnih dijaloga grčkih bogova i heroja, koji su kod Lukijana svojim manama, prepirkama i skandalima kudikamo bliži običnim smrtnicima nego uzvišenim bogovima homerske i heziodovske pjesničke tradicije. 

Ovdje se donosi odlomak iz epizode o Parisovu sudu. Taj je dijalog tradicionalno dio zbirke \textit{Razgovori bogova}, ali ga se katkad, s obzirom na posebna obilježja, izdvaja i kao zasebno djelo pod nazivom \textit{Sud o božicama (Iudicium dearum).} U razgovoru sudjeluju Zeus, Hermes, Hera, Atena, Afrodita te Paris, koji, prema Zeusovoj odluci, mora presuditi kojoj će od triju olimpijskih božica pripasti Eridina jabuka s natpisom \textgreek[variant=ancient]{καλλίστῃ} („najljepšoj'').

%\newpage

\section*{Pročitajte naglas grčki tekst.}

%Naslov prema izdanju

Luc.\ Dialogi deorum 20.7

\medskip

{\large
\begin{greek}
\noindent Πῶς ἂν οὖν, ὦ δέσποτα  Ἑρμῆ, δυνηθείην ἐγὼ θνητὸς αὐτὸς καὶ ἀγροῖκος ὢν δικαστὴς γενέσθαι παραδόξου θέας καὶ μείζονος ἢ κατὰ βουκόλον; τὰ γὰρ τοιαῦτα κρίνειν τῶν ἁβρῶν μᾶλλον καὶ ἀστικῶν· τὸ δὲ ἐμόν, αἶγα μὲν αἰγὸς ὁποτέρα ἡ καλλίων καὶ δάμαλιν ἄλλης δαμάλεως, τάχ' ἂν δικάσαιμι κατὰ τὴν τέχνην· αὗται δὲ πᾶσαί τε ὁμοίως καλαὶ καὶ οὐκ οἶδ' ὅπως ἄν τις ἀπὸ τῆς ἑτέρας ἐπὶ τὴν ἑτέραν μεταγάγοι τὴν ὄψιν ἀποσπάσας· οὐ γὰρ ἐθέλει ἀφίστασθαι ῥᾳδίως, ἀλλ' ἔνθα ἂν ἀπερείσῃ τὸ πρῶτον, τούτου ἔχεται καὶ τὸ παρὸν ἐπαινεῖ· κἂν ἐπ' ἄλλο μεταβῇ, κἀκεῖνο καλὸν ὁρᾷ καὶ παραμένει, καὶ ὑπὸ τῶν πλησίον παραλαμβάνεται. καὶ ὅλως περικέχυταί μοι τὸ κάλλος αὐτῶν καὶ ὅλον περιείληφέ με καὶ ἄχθομαι, ὅτι μὴ καὶ αὐτὸς ὥσπερ ὁ Ἄργος ὅλῳ βλέπειν δύναμαι τῷ σώματι. δοκῶ δ' ἄν μοι καλῶς δικάσαι πάσαις ἀποδοὺς τὸ μῆλον.

\end{greek}

}

%\newpage

\section*{Analiza i komentar}



%1

{\large
\noindent Πῶς ἂν οὖν, ὦ δέσποτα  Ἑρμῆ, \\
δυνηθείην ἐγὼ \\
\tabto{2em} θνητὸς αὐτὸς καὶ ἀγροῖκος ὢν \\
δικαστὴς γενέσθαι \\
\tabto{2em} παραδόξου θέας καὶ μείζονος \\
\tabto{4em} ἢ κατὰ βουκόλον; \\

}

\begin{description}[noitemsep]

\item[Πῶς] §~221
\item[ἂν\dots\ δυνηθείην] δύναμαι moći, glagol nepotpuna značenja otvara mjesto dopuni u infinitivu; 1. l. sg. opt. aor. pas. §~328.2; ἄν + optativ izriče mogućnost (potencijal sadašnji) §~464.2
\item[οὖν] postpozitivna čestica: dakle, pa
\item[ὦ δέσποτα  Ἑρμῆ] §~100
\item[ἐγὼ] §~205
\item[θνητὸς\dots\ καὶ ἀγροῖκος] §~103
\item[αὐτὸς] §~207
\item[ὢν] adverbijalni particip s uzročnim značenjem (§~503.2): budući da sam\dots\ \textgreek[variant=ancient]{θνητὸς αὐτὸς καὶ ἀγροῖκος} (imenska dopuna)
\item[δικαστὴς] §~100
\item[γενέσθαι] γίγνομαι postati, kopulativni glagol (nepotpuna značenja, traži imensku dopunu); inf. aor. med.; ovo je dopuna uz πῶς ἂν\dots\ δυνηθείην
\item[παραδόξου] §~106; παράδοξος nevjerojatan, čudesan
\item[θέας] §~90, objektni genitiv uz δικαστὴς (§~394)
\item[μείζονος] §~200
\item[ἢ κατὰ βουκόλον] §~82, §~514.1.b; uz \textgreek[variant=ancient]{θέα μείζων}: prizor veći no što dolikuje govedaru

\end{description}

{\large
\noindent τὰ γὰρ τοιαῦτα κρίνειν\\
\tabto{2em} τῶν ἁβρῶν \\
μᾶλλον \\
\tabto{2em} καὶ ἀστικῶν·\\
τὸ δὲ ἐμόν, \\
\tabto{2em} αἶγα μὲν αἰγὸς \\
\tabto{4em} ὁποτέρα ἡ καλλίων \\
\tabto{2em} καὶ δάμαλιν ἄλλης δαμάλεως, \\
\tabto{2em} τάχ' ἂν δικάσαιμι\\
\tabto{4em} κατὰ τὴν τέχνην·\\
\tabto{2em} αὗται δὲ \\
\tabto{4em} πᾶσαί τε ὁμοίως καλαὶ \\
\tabto{4em} καὶ οὐκ οἶδ' ὅπως ἄν τις \\
\tabto{6em} ἀπὸ τῆς ἑτέρας \\
\tabto{6em} ἐπὶ τὴν ἑτέραν \\
\tabto{4em} μεταγάγοι \\
\tabto{6em} τὴν ὄψιν \\
\tabto{4em} ἀποσπάσας·\\

}

\begin{description}[noitemsep]

\item[τὰ\dots\ τοιαῦτα] §~213.4
\item[κρίνειν] κρίνω prosuđivati; inf. prez. akt.
\item[μᾶλλον] §~204.3
\item[τῶν ἁβρῶν καὶ ἀστικῶν] §~103, sc.\ ἐστίν: više priliči finim i gradskim ljudima (§~393.2); imenski predikat, Smyth 909 (kopula je ovdje neizrečena)
\item[τὸ\dots\ ἐμόν] što se mene tiče; §~210
\item[τὸ δὲ ἐμόν\dots] čestica izražava suprotnost u odnosu na prethodnu rečenicu: a\dots
\item[αἶγα μὲν\dots\ αὗται δὲ\dots] koordinacija rečeničnih članova ostvarena parom čestica (kod Lukijana dodatno komična jer suprotstavlja koze božicama)
\item[αἶγα\dots\ αἰγὸς] §~115, genitiv odvajanja §~402
\item[ὁποτέρα] §~219, odnosna zamjenica uvodi zavisnu upitnu rečenicu §~469
\item[ἡ καλλίων] §~200, sc.\ ἐστίν; imenski predikat, Smyth 909 (kopula je ovdje neizrečena)
\item[δάμαλιν ἄλλης δαμάλεως] §~165, §~212a, genitiv odvajanja §~402
\item[τάχ'] τάχα (elizija §~68), s ἂν + opt.: možda\dots
\item[ἂν δικάσαιμι] δικάζω razlučiti, presuditi; 1. l. sg. opt. aor. akt.; ἄν + optativ izriče mogućnost (potencijal sadašnji) §~464.2
\item[κατὰ τὴν τέχνην] §~90, §~429.B.c
\item[αὗται] §~213.2
\item[πᾶσαί τε\dots\ καὶ οὐκ οἶδ'\dots] koordinacija parom sastavnih veznika (drugi je član naglašeniji)
\item[πᾶσαί] §~193
\item[ὁμοίως] §~204
\item[καλαὶ] §~103; sc.\ εἰσίν; imenski predikat, Smyth 909 (kopula je ovdje neizrečena)
\item[οὐκ οἶδ'] οἶδα znati; 1. l. sg. ind. perf. s prezentskim značenjem; kao \textit{verbum sentiendi} otvara mjesto zavisnoj upitnoj rečenici
\item[ὅπως ἄν\dots\ μεταγάγοι] §~446; zavisna upitna rečenica (§~469) s optativom; μετάγω prenijeti, prebaciti; 3. l. sg. opt. aor. akt.
\item[τις] §~218.2.b
\item[ἀπὸ τῆς ἑτέρας] §~103; ἑτέρος\dots\ ἑτέρος: jedan\dots\ drugi; 423.a
\item[ἐπὶ τὴν ἑτέραν] §~103, §~419a
\item[τὴν ὄψιν] §~165; objekt glagola μεταγάγοι i ἀποσπάσας
\item[ἀποσπάσας] ἀποσπάω otrgnuti; n. sg. m. r. ptc. aor. akt.

\end{description}

%5

{\large
\noindent οὐ γὰρ ἐθέλει ἀφίστασθαι ῥᾳδίως,\\ 
ἀλλ' ἔνθα ἂν ἀπερείσῃ \\
τὸ πρῶτον,\\
\tabto{2em} τούτου ἔχεται \\
\tabto{2em} καὶ τὸ παρὸν ἐπαινεῖ·\\
κἂν ἐπ' ἄλλο μεταβῇ,\\
\tabto{2em} κἀκεῖνο καλὸν ὁρᾷ καὶ παραμένει,\\ 
\tabto{2em} καὶ ὑπὸ τῶν πλησίον παραλαμβάνεται.\\

}

\begin{description}[noitemsep]

\item[οὐ γὰρ ἐθέλει] ἐθέλω željeti (subjekt ἡ ὄψις), otvara mjesto dopuni u infinitivu; 3. l. sg. ind. prez. akt.
\item[ἀφίστασθαι] ἀφίστημι udaljiti se; inf. prez. medpas.
\item[ῥᾳδίως] §~204
\item[ἂν ἀπερείσῃ] ἀπερείδω uprijeti, upraviti (pogled na koga), 3. l. sg. konj. aor. akt.
\item[τὸ πρῶτον] §~223 (poimeničeni broj); priložno (oznaka vremena), LSJ πρότερος B III.3.a
\item[τούτου] §~213.2; objekt glagola ἔχεται
\item[ἔχεται] ἔχω med., rekcija τινός (čvrsto) držati se čega; 3. l. sg. ind. prez. medpas.
\item[τὸ παρὸν] πάρειμι biti tu, nazočan, pred očima; a. sg. s. r. ptc. prez.; poimeničenje članom §~373%; τὸ παρόν priložno: u tom trenutku, u sadašnjim okolnostima, LSJ πάρειμι (εἰμί sum) II.%možda?
\item[ἐπαινεῖ] ἐπαινέω hvaliti, odobravati; 3. l. sg. ind. prez. akt.
\item[κἂν\dots\ μεταβῇ] κἂν = καὶ ἐάν (kraza: §~66), uvodi pogodbenu rečenicu s konjunktivom; μεταβαίνω prijeći, preseliti se; 3. l. sg. konj. aor. akt. u pogodbenoj rečenici: a ako prijeđe\dots
\item[ἐπ' ἄλλο] ἐπί + ἄλλο, §~212
\item[κἀκεῖνο καλὸν] κἀκεῖνο = καὶ ἐκεῖνο (kraza: §~66), §~103, §~213.3; ἐκεῖνο καλὸν objekt je glagola ὁρᾷ
\item[ὁρᾷ] ὁράω gledati; 3. l. sg. ind. prez. akt.
\item[παραμένει] παραμένω ostati, zadržati se; 3. l. sg. ind. prez. akt.
\item[ὑπὸ τῶν πλησίον] genitiv lica, izriče vršitelja pasivne radnje §~449; τὰ πλησίον ono što je u blizini, poimeničenje članom §~373
\item[παραλαμβάνεται] παραλαμβάνω osvojiti; 3. l. sg. ind. prez. medpas.

\end{description}

%7

{\large
\noindent καὶ ὅλως περικέχυταί μοι \\
\tabto{2em} τὸ κάλλος \\
\tabto{4em} αὐτῶν \\
καὶ ὅλον περιείληφέ με καὶ ἄχθομαι, \\
\tabto{2em} ὅτι μὴ καὶ αὐτὸς \\
\tabto{4em} ὥσπερ ὁ  Ἄργος \\
\tabto{2em} ὅλῳ \\
\tabto{6em} βλέπειν \\
\tabto{4em} δύναμαι \\
\tabto{2em} τῷ σώματι.\\


}

\begin{description}[noitemsep]

\item[ὅλως] §~204
\item[περικέχυταί] περιχέω med. περιχεῖταί τι τινί nešto oblijeva koga; 3. l. sg. ind. perf. medpas.
\item[μοι] §~205
\item[τὸ κάλλος] §~153
\item[αὐτῶν] §~207, §~393
\item[ὅλον\dots\ με] §~103, §~205
\item[περιείληφέ] περιλαμβάνω obuhvatiti; 3. l. sg. ind. perf. akt.
\item[ἄχθομαι] ἄχθομαι žalostiti se; 1. l. sg. ind. prez.; \textit{verbum sentiendi} otvara mjesto zavisnoj uzročnoj rečenici
\item[ὅτι] veznik uvodi zavisnu uzročnu rečenicu (§~468); u kasnijih autora moguća je negacija μὴ umjesto οὐ
\item[καὶ αὐτὸς] §~207; καί ima ulogu isticanja: (također) i sâm, i ja
\item[ὥσπερ] §~519.2
\item[ὁ  Ἄργος] §~82; Ἄργος Πανόπτης, uvijek budan div iz grčkog mita, često prikazivan sa stotinu očiju
\item[ὅλῳ\dots\ τῷ σώματι] §~103, §~123; dativ sredstva §~414.1; hiperbatom se ističe pridjev
\item[βλέπειν] βλέπω gledati; inf. prez. akt.
\item[δύναμαι] δύναμαι moći, kao glagol nepotpuna značenja otvara mjesto dopuni u infinitivu; 1. l. sg. ind. prez.

\end{description}

%8

{\large
\noindent δοκῶ δ' ἄν μοι \\
\tabto{2em} καλῶς δικάσαι\\
\tabto{2em} πάσαις \\
ἀποδοὺς \\
\tabto{2em} τὸ μῆλον.\\

}

\begin{description}[noitemsep]

\item[δοκῶ\dots\ μοι] δοκέω: činiti se, rekcija τινί; 1. l. sg. ind. prez. akt.; otvara mjesto dopuni u infinitivu
\item[ἄν] naglašava da infinitiv ima potencijalno značenje, §~506
\item[καλῶς] §~204
\item[δικάσαι] δικάζω presuditi; inf. aor. akt. (dopuna uz δοκῶ\dots\ μοι)
\item[πάσαις] §~193; neizravni objekt glagola ἀποδοὺς §~411.1
\item[ἀποδοὺς] ἀποδίδωμι dati; n. sg. m. r. ptc. aor. akt.; §~503.4
\item[τὸ μῆλον] §~82

\end{description}




%kraj
