% ispravci NZ
\section*{O autoru}

Platon \textgreek[variant=ancient]{(Πλάτων,} 427.\ – 347.\ pr.~Kr.)  potječe iz ugledne atenske plemićke loze. Oko 407.\ postaje Sokratov učenik. Nakon učiteljeva smaknuća 399.\ napušta Atenu te poduzima brojna putovanja. S četrdesetak godina dolazi u Sirakuzu na Siciliji. Njegov nazor o nužnosti vladavine filozofa postaje preopasan tamošnjem vladaru, tiraninu Dioniziju I., pa oko 387.\ mora napustiti grad. Dva sljedeća putovanja u Sirakuzu (367.\ – 365.\ i 361.\ – 360.) k Dioniziju II.\ završila su također bez uspjeha. Oko 360.\ Platon se posvećuje isključivo vođenju Akademije, koju je u Ateni osnovao oko 387. (Akademija će, iako ne bez prekida, djelovati do 529.\ po Kr., kad ju je zatvorio car Justinijan.)

Spisateljsku djelatnost Platon je počeo nakon Sokratove smrti 399. Vjerojatno su sačuvani svi Platonovi spisi, tj.\ tridesetak dijaloga i niz pisama (manji se njihov dio smatra neautentičnim). Platon je djela sastavljao gotovo isključivo u dijaloškom obliku, u kojemu njegovo mišljenje kruži oko predmeta u otvorenom, ispitivačkom razgovoru. Tako misli sudionika u razgovoru postižu iznimno veliku zornost i životnost. Djela se obično dijele u tri skupine: 1.\ rani dijalozi (do prvoga sicilskog putovanja), među njima su \textit{Apologija, Protagora, Eutifron, Lahet, Država} I; 2.\ srednji dijalozi (do drugoga sicilskog putovanja), najvažniji su \textit{Gorgija, Kratil, Menon, Fedon, Simpozij, Država} II-X, \textit{Fedar}; 3.\ kasni dijalozi (nakon 365.): \textit{Teetet, Parmenid, Sofist, Timej, Kritija, Zakoni, Državnik, Fileb}. 

Tumačenje ovih dijaloga nameće brojne probleme. Na primjer, teško je od Platonova udjela razlučiti misaoni udio povijesnoga Sokrata, koji u gotovo svim dijalozima ima glavnu ulogu. Nadalje, zbog izrazito dugoga vremenskog raspona nastanka dijalozi odražavaju dinamiku razvoja Platonova nauka. U ranim se djelima s pomoću etičkih pojmovnih određenja provodi sokratovska metoda \textgreek[variant=ancient]{(μαιευτική,} majeutika, tj.\ primaljska vještina). Glavna je tema vrlina \textgreek[variant=ancient]{(ἀρετή),} a dijalozi većinom završavaju bez rezultata, u aporijama. U srednjim dijalozima Platon razvija nauk o idejama koje postaju temelj teorijama o čovjeku i idealnoj državi. U kasnim dijalozima ta se rasprava produbljuje, a nauk o idejama samokritički se podvrgava temeljitoj reviziji. 

Utjecaj Platona u duhovnoj povijesti teško je prenaglasiti. Dovoljno je navesti misao britanskog filozofa A.~N.\ Whiteheada: “The safest general characterization of the European philosophical tradition is that it consists of a series of footnotes to Plato. I do not mean the systematic scheme of thought which scholars have doubtfully extracted from his writings. I allude to the wealth of general ideas scattered through them.”

\section*{O tekstu}

\textit{Meneksen} \textgreek[variant=ancient]{(Μενέξενος)} pripada ranim Platonovim dijalozima (pretpostavlja se da je nastao između 386.\ i 380.\ pr.~Kr.). Sokrat i Meneksen razgovaraju u trenutku kada se u Ateni očekuje izbor govornika koji će pogrebnim govorom popratiti predstojeće javne pogrebne počasti. Potaknut Sokratovim ironičnim primjedbama o onodobnim govornicima, Meneksen ga izaziva da sam iznese uzoran pogrebni govor, a Sokrat nato izlaže onaj koji je dan prije čuo od Aspazije, čuvene atenske intelektualke za koju kaže da je govorništvu učila i njega i Perikla. U Sokratovu govoru u čast poginulim ratnicima, koji zauzima najveći dio \textit{Meneksena}, mnogi vide parodiju glasovitoga pogrebnog govora koji je palim Atenjanima održao Periklo, a koji prenosi Tukidid u \textit{Povijesti Peloponeskog rata}. 

Odabrani odlomak dio je Sokratova elogija Atike kao bogate i uspješne zemlje koja daje dobro podrijetlo i vrsnu naobrazbu.

%\newpage

\section*{Pročitajte naglas grčki tekst.}

Plat.\ Menex.\ 237c

\medskip

{\large
\begin{greek}
῎Εστι δὲ ἀξία ἡ χώρα καὶ ὑπὸ πάντων ἀνθρώπων ἐπαινεῖσθαι, οὐ μόνον ὑφ' ἡμῶν, πολλαχῇ μὲν καὶ ἄλλῃ, πρῶτον δὲ καὶ μέγιστον ὅτι τυγχάνει οὖσα θεοφιλής. μαρτυρεῖ δὲ ἡμῶν τῷ λόγῳ ἡ τῶν ἀμφισβητησάντων περὶ αὐτῆς θεῶν ἔρις τε καὶ κρίσις· ἣν δὴ θεοὶ ἐπῄνεσαν, πῶς οὐχ ὑπ' ἀνθρώπων γε συμπάντων δικαία ἐπαινεῖσθαι; δεύτερος δὲ ἔπαινος δικαίως ἂν αὐτῆς εἴη, ὅτι ἐν ἐκείνῳ τῷ χρόνῳ, ἐν ᾧ ἡ πᾶσα γῆ ἀνεδίδου καὶ ἔφυε ζῷα παντοδαπά, θηρία τε καὶ βοτά, ἐν τούτῳ ἡ ἡμετέρα θηρίων μὲν ἀγρίων ἄγονος καὶ καθαρὰ ἐφάνη, ἐξελέξατο δὲ τῶν ζῴων καὶ ἐγέννησεν ἄνθρωπον, ὃ συνέσει τε ὑπερέχει τῶν ἄλλων καὶ δίκην καὶ θεοὺς μόνον νομίζει.
\end{greek}

}

%\newpage

\section*{Analiza i komentar}

{\large
\noindent ῎Εστι δὲ ἀξία ἡ χώρα\\
\tabto{2em} καὶ ὑπὸ πάντων ἀνθρώπων \\
\tabto{4em} ἐπαινεῖσθαι, \\
\tabto{2em} οὐ μόνον ὑφ' ἡμῶν, \\
πολλαχῇ μὲν καὶ ἄλλῃ, \\
πρῶτον δὲ καὶ μέγιστον \\
\tabto{2em} ὅτι τυγχάνει \\
\tabto{4em} οὖσα θεοφιλής.\\

}

%komentar

\begin{description}[noitemsep]
\item[῎Εστι δὲ ἀξία] čestica izražava suprotnost u odnosu na prethodnu (ovdje izostavljenu) rečenicu: a\dots; imenski predikat s pridjevom kao imenskim dijelom; Smyth 909
\item[ἔστι] εἰμί biti, 3. l. sg. ind. prez. 
\item[ἀξία] §~103; pridjev otvara mjesto dopuni u infinitivu
\item[ἡ χώρα] §~90
\item[ὑπὸ πάντων ἀνθρώπων] §~193, §~82; priložna oznaka \textgreek[variant=ancient]{ὑπό τινος} uz pasiv \textgreek[variant=ancient]{(ἐπαινεῖσθαι)} iskazuje vršitelja radnje
\item[ἐπαινεῖσθαι] ἐπαινέω hvaliti; inf. prez. medpas.; dopuna uz ἀξία: vrijedna da bude hvaljena (§~492)
\item[οὐ μόνον] ne samo
\item[ὑφ' ἡμῶν] §~205; promjena suglasnika pred hakom, ὑπὸ ἡμῶν > ὑπ' ἡμῶν > ὑφ' ἡμῶν §~74; usp. gore \textgreek[variant=ancient]{ὑπὸ πάντων ἀνθρώπων}
\item[πολλαχῇ μὲν καὶ ἄλλῃ\dots\ πρῶτον δὲ\dots] koordinacija rečeničnih članova s pomoću para čestica μέν\dots\ δέ\dots
\item[πολλαχῇ\dots\ καὶ ἄλλῃ] §~212; i iz mnogo drugih razloga (sc.\ αἰτίᾳ ili slično; dativ kao priložna oznaka uzroka §~414.2)
\item[πρῶτον\dots\ καὶ μέγιστον] §~223, §~200; upotrijebljeno priložno: prvo i najviše
\item[ὅτι] veznik uvodi zavisnu uzročnu rečenicu, §~468
\item[τυγχάνει] τυγχάνω dogoditi se; 3. l. sg. ind. prez. akt.; kopulativni glagol otvara mjesto participu i prevodi se prilogom (a particip οὖσα finitnim oblikom): upravo\dots\ §~501
\item[οὖσα] εἰμί biti; ptc. prez. n. sg. ž. r.; pobliže opisuje χώρα i traži imensku dopunu, u ovom slučaju pridjev
\item[θεοφιλής] §~194.2
\end{description}

{\large
\noindent μαρτυρεῖ δὲ \\
\tabto{4em} ἡμῶν \\
\tabto{2em} τῷ λόγῳ \\
ἡ \\
\tabto{2em} τῶν ἀμφισβητησάντων \\
\tabto{4em} περὶ αὐτῆς \\
\tabto{2em} θεῶν \\
ἔρις τε καὶ κρίσις·\\
ἣν δὴ θεοὶ ἐπῄνεσαν,\\
πῶς οὐχ \\
\tabto{2em} ὑπ' ἀνθρώπων γε συμπάντων \\
δικαία \\
\tabto{2em} ἐπαινεῖσθαι;\\

}

%komentar

\begin{description}[noitemsep]
\item[δὲ] čestica izražava suprotnost u odnosu na prethodnu rečenicu: a\dots
\item[μαρτυρεῖ] μαρτυρέω τινί svjedočiti o čemu; 3. l. sg. ind. prez. akt.
\item[ἡμῶν] §~205, §~211.2.b; ovisno o τῷ λόγῳ
\item[τῷ λόγῳ] §~82
\item[ἡ\dots\ ἔρις] §~123
\item[τῶν ἀμφισβητησάντων περὶ αὐτῆς] ἀμφισβητέω περί τινος prisvajati što, svađati se oko čega; ptc. aor. akt. g. pl. m. r.; §~207, sc.\ χώρας; particip s imenicom kao atributna oznaka u atributnom položaju, §~375
\item[θεῶν] §~82
\item[ἔρις τε καὶ κρίσις] koordinacija ostvarena parom sastavnih veznika \dots\ τε\dots\ καὶ\dots: i\dots\ i\dots
\item[κρίσις] §~165
\item[ἣν] §~215, sc.\ χώραν; odnosna zamjenica uvodi odnosnu rečenicu
\item[δὴ] čestica naglašava zamjenicu: dakle, ona koju\dots; §~516.5
\item[θεοὶ] §~82
\item[ἐπῄνεσαν] ἐπαινέω hvaliti; 3. l. pl. ind. aor. akt.
\item[πῶς] §~221
\item[ὑπ' ἀνθρώπων γε συμπάντων] §~193, §~82; čestica γε ističe riječi između kojih stoji: baš\dots; §~519.1
\item[δικαία] §~103; sc.\ ἐστί, imenski dio imenskog predikata, kopula izostavljena; otvara mjesto infinitivu \textgreek[variant=ancient]{(ἐπαινεῖσθαι):} zaslužuje\dots
\item[ἐπαινεῖσθαι] ἐπαινέω hvaliti; inf. prez. medpas.
\end{description}



{\large
\noindent δεύτερος δὲ ἔπαινος \\
\tabto{2em} δικαίως \\
\tabto{2em} ἂν \\
αὐτῆς \\
\tabto{2em} εἴη, \\
\tabto{4em} ὅτι ἐν ἐκείνῳ τῷ χρόνῳ, \\
\tabto{4em} ἐν ᾧ ἡ πᾶσα γῆ \\
\tabto{6em} ἀνεδίδου καὶ ἔφυε \\
\tabto{6em} ζῷα παντοδαπά, \\
\tabto{8em} θηρία τε καὶ βοτά, \\
\tabto{4em} ἐν τούτῳ ἡ ἡμετέρα \\
\tabto{6em} θηρίων μὲν ἀγρίων \\
\tabto{4em} ἄγονος καὶ καθαρὰ ἐφάνη, \\
\tabto{6em} ἐξελέξατο δὲ τῶν ζῴων \\
\tabto{6em} καὶ ἐγέννησεν ἄνθρωπον, \\
\tabto{8em} ὃ συνέσει τε ὑπερέχει \\
\tabto{10em} τῶν ἄλλων \\
\tabto{8em} καὶ δίκην καὶ θεοὺς \\
\tabto{8em} μόνον \\
\tabto{8em} νομίζει.\\

}

%komentar

\begin{description}[noitemsep]
\item[δεύτερος\dots\ ἔπαινος\dots\ ἂν\dots\ εἴη] imenski predikat; Smyth 909
\item[δεύτερος\dots\ ἔπαινος] §~223, §~82
\item[δὲ] čestica izražava suprotnost u odnosu na prethodnu rečenicu: a\dots
\item[δικαίως] §~204
\item[ἂν\dots\ εἴη] εἰμί biti, 3. l. sg. opt., izriče mogućnost u sadašnjosti (§~464.2); otvara mjesto izričnoj rečenici koja ima ulogu subjekta: to što\dots
\item[αὐτῆς] sc.\ χώρας; §~207; objektni genitiv uz \textgreek[variant=ancient]{ἔπαινος} (§~394)
\item[ὅτι] veznik uvodi zavisnu izričnu rečenicu: to što\dots\ §~467
\item[ἐν ἐκείνῳ τῷ χρόνῳ] §~213, §~82; priložna oznaka vremena 
\item[ἐν ᾧ] §~215; odnosna zamjenica uvodi odnosnu rečenicu, antecedent je χρόνῳ
\item[ἡ πᾶσα γῆ] §~193, §~90
\item[ἀνεδίδου] ἀναδίδωμι iznijeti, donijeti; 3. l. sg. impf. akt.
\item[ἔφυε] φύω roditi; 3. l. sg. impf. akt.
\item[ζῷα παντοδαπά] §~82, §~103
\item[θηρία τε καὶ βοτά] koordinacija ostvarena parom sastavnih veznika; §~82
\item[ἐν τούτῳ] §~213, sc.\ τῷ χρόνῳ
\item[ἡ ἡμετέρα] §~210, sc.\ γῆ; Sokrat, kao Atenjanin, govori iz atičke perspektive
\item[θηρίων μὲν\dots\ ἐξελέξατο δὲ\dots] koordinacija rečeničnih članova s pomoću para čestica
\item[θηρίων\dots\ ἀγρίων] §~82, §~103; objektni genitiv uz ἄγονος (§~394) i genitiv odvajanja uz καθαρά (§~402.2)
\item[ἄγονος] τινός koji što ne rađa, §~103, §~106
\item[καθαρὰ] §~103
\item[ἐφάνη] φαίνω pokazati, glagol otvara mjesto imenskoj dopuni (pridjevima); 3. l. sg. ind. aor. pas.
\item[ἐξελέξατο] ἐκλέγω τινός odabrati od čega; 3. l. sg. ind. aor. med.
\item[τῶν ζῴων] §~82; dijelni genitiv §~395
\item[ἐγέννησεν] γεννάω roditi, stvoriti; 3. l. sg. ind. aor. akt.
\item[ἄνθρωπον] §~82
\item[ὃ] sc.\ ζῷον (antecedent je relativa τῶν ζῴων); §~215; odnosna zamjenica uvodi odnosnu rečenicu
\item[συνέσει] §~165
\item[ὑπερέχει] ὑπερέχω τινός nadmašiti koga; 3. l. sg. ind. prez. akt.
\item[τῶν ἄλλων] §~212; sc.\ ζῴων
\item[δίκην\dots\ θεοὺς] §~90, §~82
\item[μόνον] jedini (sc.\ ζῷον), predikatna uporaba uz νομίζει
\item[νομίζει] νομίζω priznavati, štovati; 3. l. sg. ind. prez. akt.
\end{description}
