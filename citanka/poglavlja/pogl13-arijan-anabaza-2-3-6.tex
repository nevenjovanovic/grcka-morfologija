% Redaktura NZ, unio NJ
\section*{O autoru}

Flavije Arijan \textgreek[variant=ancient]{(Φλάουιος Ἀρριανός,} Flavius Arrianus; Nikomedija u Bitiniji, oko 95. – Atena, oko 175.) bio je grčki povjesničar i filozof carskoga doba, učenik filozofa Epikteta (oko 55.–135). Za Hadrijana, oko 128.–129, Arijan je bio rimski konzul, a 131.–138. namjesnik (legatus Augusti pro praetore) u Kapadociji. Kao pedesetogodišnjak povukao se iz javnog života, boravio u Ateni (ἄρχων ἐπώνυμος 145./146). Književni mu je model bio Ksenofont (oko 430.–354. pr.~Kr); Ksenofontovo je ime Arijan nosio i kao nadimak.

Glavno je Arijanovo historiografsko djelo \textit{Aleksandrov pohod} \textgreek[variant=ancient]{(Ἀνάβασις Ἀλεξάνδρου)} ili \textit{Anabaza}; u sedam knjiga faktografski precizno i kritično iznio je povijest vladavine Aleksandra III. Velikog. Autor je i povijesno-geografskog \textit{Opisa Indije} \textgreek[variant=ancient]{(Ἰνδικὴ συγγραφή),} sastavljenog prema Aleksandrovim povjesničarima Megastenu i Nearhu; u \textit{Opisu} piše jonskim grčkim poput Hekateja i Herodota. Priredio je Epiktetove \textit{Rasprave} \textgreek[variant=ancient]{(Διατριβαί)} u osam knjiga (sačuvane samo prve četiri) – taj je filozofski angažman usporediv s Ksenofontovim \textgreek[variant=ancient]{Ἀπομνημονεύματα Σωκράτους} – te \textit{Priručnik} \textgreek[variant=ancient]{(Ἐγχειρίδιον)} za Epiktetovu etiku.

\section*{O tekstu}

U sljedećem odlomku Arijan iznosi dvije verzije legende o Aleksandru i gordijskom čvoru; drugu, manje uobičajenu, navodi prema povjesničaru Aristobulu (umro nakon 301. pr.~Kr), koji pripovijeda da čvor nije bio presječen, već razriješen, pošto je izvučen klin \textgreek[variant=ancient]{(ἕστωρ)} koji ga je držao. Riječ \textgreek[variant=ancient]{ἕστωρ} je homerska i u kasnijem grčkom neuobičajena; njome se koristi \textit{Ilijada} 24, 272, opisujući kako se klinom učvršćuje čvor. Po svemu sudeći, Gordijeva su kola bila vezana na homerski način, što je Aleksandar shvatio i prepoznao (obzirom da je Homera, kako je posvjedočeno, čitao pod Aristotelovim vodstvom). 

Stil teksta obilježen je za Arijanovo vrijeme ``starinskom'' atičkom jednostavnošću, i po tome je oprečan ``modernom'' grčkom Arijanova starijeg suvremenika Plutarha (oko 46. – oko 120).

%\newpage

\section*{Pročitajte naglas grčki tekst.}

%Naslov prema izdanju

Arr.\ Anabasis 2.3.6

\medskip

{\large
\begin{greek}
\noindent Πρὸς δὲ δὴ τούτοις καὶ τόδε περὶ τῆς ἁμάξης ἐμυθεύετο, ὅστις λύσειε τοῦ ζυγοῦ τῆς ἁμάξης τὸν δεσμόν, τοῦτον χρῆναι ἄρξαι τῆς Ἀσίας. ἦν δὲ ὁ δεσμὸς ἐκ φλοιοῦ κρανίας καὶ τούτου οὔτε τέλος οὔτε ἀρχὴ ἐφαίνετο. Ἀλέξανδρος δὲ ὡς ἀπόρως μὲν εἶχεν ἐξευρεῖν λύσιν τοῦ δεσμοῦ, ἄλυτον δὲ περιιδεῖν οὐκ ἤθελε, μή τινα καὶ τοῦτο ἐς τοὺς πολλοὺς κίνησιν ἐργάσηται, οἱ μὲν λέγουσιν, ὅτι παίσας τῷ ξίφει διέκοψε τὸν δεσμὸν καὶ λελύσθαι ἔφη· Ἀριστόβουλος δὲ λέγει ἐξελόντα τὸν ἕστορα τοῦ ῥυμοῦ, ὃς ἦν τύλος διαβεβλημένος διὰ τοῦ ῥυμοῦ διαμπάξ, ξυνέχων τὸν δεσμόν, ἐξελκύσαι ἔξω τοῦ ῥυμοῦ τὸ$\langle$ν$\rangle$ ζυγόν. ὅπως μὲν δὴ ἐπράχθη τὰ ἀμφὶ τῷ δεσμῷ τούτῳ Ἀλεξάνδρῳ οὐκ ἔχω ἰσχυρίσασθαι. ἀπηλλάγη  δ' οὖν ἀπὸ τῆς ἁμάξης αὐτός τε καὶ οἱ ἀμφ' αὐτὸν ὡς τοῦ λογίου τοῦ ἐπὶ τῇ λύσει τοῦ δεσμοῦ ξυμβεβηκότος. καὶ γὰρ καὶ τῆς νυκτὸς ἐκείνης βρονταί τε καὶ σέλας ἐξ οὐρανοῦ ἐπεσήμηναν· καὶ ἐπὶ τούτοις ἔθυε τῇ ὑστεραίᾳ Ἀλέξανδρος τοῖς φήνασι θεοῖς τά τε σημεῖα καὶ τοῦ δεσμοῦ τὴν λύσιν.
\end{greek}

}


%\newpage

\section*{Analiza i komentar}


%1

{\large
\noindent Πρὸς δὲ δὴ τούτοις \\
καὶ τόδε \\
\tabto{2em} περὶ τῆς ἁμάξης \\
ἐμυθεύετο, \\
\tabto{2em} ὅστις λύσειε \\
\tabto{4em} τοῦ ζυγοῦ\\
\tabto{6em} τῆς ἁμάξης \\
\tabto{2em} τὸν δεσμόν, \\
\underline{τοῦτον χρῆναι} \\
\tabto{2em} ἄρξαι τῆς Ἀσίας.\\

}

\begin{description}[noitemsep] 

\item[Πρὸς\dots\ τούτοις] §~418; §~213
\item[δὲ δὴ] kombinacija čestica povezuje ovu rečenicu s prethodnom: a\dots
\item[τόδε] §~213
\item[περὶ τῆς ἁμάξης] §~418; §~97
\item[ἐμυθεύετο] μυθεύω pripovijedati; 3. l. sg. impf. medpas.; glagol kao \textit{verbum dicendi} otvara mjesto A+I
\item[ὅστις] §~217; uvodi zavisnu odnosnu rečenicu, njegov je antecedent τοῦτον
\item[λύσειε] λύω razriješiti; 3. l. sg. opt. aor. akt.; hipotetička relativna rečenica realnog oblika; u indirektnom govoru može iza historijskog vremena umjesto indikativa ili konjunktiva stajati optativ \textit{(optativus obliquus)} § 486. B., §~489.2.b
\item[τοῦ ζυγοῦ] §~82; ovisno o τὸν δεσμόν
\item[τῆς ἁμάξης] §~97; ovisno o τοῦ ζυγοῦ
\item[τὸν δεσμόν] §~82
\item[τοῦτον] §~213
\item[ὅστις\dots\ τοῦτον] koordinacija ostvarena odnosnom zamjenicom (konektor) i pokaznom (antecedent)
\item[χρῆναι] χρή (bezlično) trebati, glagol nepotpuna značenja otvara mjesto dopuni u infinitivu; inf. prez. (akt.)
\item[τῆς Ἀσίας] §~90
\item[ἄρξαι] ἄρχω τινός vladati nečim; inf. aor. akt.
\end{description}

{\large
\noindent ἦν δὲ ὁ δεσμὸς \\
\tabto{2em} ἐκ φλοιοῦ \\
\tabto{4em} κρανίας \\
καὶ τούτου \\
\tabto{2em} οὔτε τέλος \\
\tabto{2em} οὔτε ἀρχὴ \\
\tabto{4em} ἐφαίνετο.\\

}

\begin{description}[noitemsep] 
\item[ἦν\dots\ ἐκ φλοιοῦ] imenski predikat (s prijedložnim izrazom kao imenskim dijelom) Smyth 909
\item[ἦν] εἰμί biti; 3. l. sg. impf. (akt.)
\item[δὲ] čestica povezuje rečenicu s prethodnom kao suprotni veznik: a\dots; §~515
\item[ὁ δεσμὸς ] §~82
\item[ἐκ φλοιοῦ κρανίας] §~418, §~82, §~97
\item[τούτου] (sc.\ δεσμοῦ) §~213
\item[οὔτε\dots\ οὔτε\dots] koordinacija ostvarena niječnim sastavnim veznicima
\item[τέλος] §~153
\item[ἀρχὴ] §~90
\item[ἐφαίνετο] φαίνω med. vidjeti se, pokazati se; 3. l. sg. impf. medpas.
\end{description}

%3 

{\large
\noindent Ἀλέξανδρος δὲ \\
\tabto{2em} ὡς ἀπόρως μὲν εἶχεν \\
\tabto{4em} ἐξευρεῖν \\
\tabto{6em} λύσιν \\
\tabto{8em} τοῦ δεσμοῦ, \\
\tabto{2em} ἄλυτον δὲ \\
\tabto{4em} περιιδεῖν \\
\tabto{2em} οὐκ ἤθελε, \\
μή τινα \\
καὶ τοῦτο \\
\tabto{2em} ἐς τοὺς πολλοὺς \\
κίνησιν \\
ἐργάσηται, \\
οἱ μὲν λέγουσιν, \\
\tabto{2em} ὅτι παίσας τῷ ξίφει \\
\tabto{4em} διέκοψε τὸν δεσμὸν \\
\tabto{4em} καὶ \underline{λελύσθαι} \\
\tabto{4em} ἔφη· \\
Ἀριστόβουλος δὲ λέγει \\
\tabto{2em} \underline{ἐξελόντα} τὸν ἕστορα \\
\tabto{4em} τοῦ ῥυμοῦ, \\
\tabto{6em} ὃς ἦν τύλος διαβεβλημένος \\
\tabto{8em} διὰ τοῦ ῥυμοῦ \\
\tabto{8em} διαμπάξ, \\
\tabto{4em} ξυνέχων τὸν δεσμόν, \\
\tabto{2em} \underline{ἐξελκύσαι} \\
\tabto{4em} ἔξω τοῦ ῥυμοῦ \\
\tabto{2em} τὸ$\langle$ν$\rangle$ ζυγόν.\\

}

\begin{description}[noitemsep] 

\item[Ἀλέξανδρος] §~82
\item[δὲ] čestica povezuje rečenicu s prethodnom kao suprotni veznik: a\dots; §~515
\item[ὡς] veznik uvodi zavisnu uzročnu rečenicu: jer\dots\ §~518
\item[ἀπόρως μὲν\dots\ ἄλυτον δὲ\dots] koordinacija rečeničnih članova pomoću para čestica: a\dots; §~515; §~519
\item[ἀπόρως\dots\ εἶχεν] §~204; ἀπόρως ἔχω + inf. ne biti u stanju nešto učiniti (traži dopunu u infinitivu), 3. l. sg. ind. aor. akt.
\item[ἐξευρεῖν] ἐξευρίσκω pronaći, inf. aor. akt.; dopuna uz ἀπόρως\dots\ εἶχεν
\item[λύσιν ] §~165
\item[τοῦ δεσμοῦ] §~82
\item[ἄλυτον] sc.\ τὸν δεσμόν, §~106
\item[περιιδεῖν] περιοράω τι dopustiti, pustiti nešto, inf. aor. akt.
\item[ἤθελε] ἐθέλω htjeti nešto činiti (traži dopunu u infinitivu); 3. l. sg. impf. akt.
\item[μή\dots\ ἐργάσηται] μή s konjunktivom izriče zavisnu namjernu rečenicu, §~470; ἐργάζομαι izazvati, 3. l. sg. konj. aor. med.
\item[τινα\dots\ κίνησιν] §~217; §~165; κίνησις ovdje u prenesenom značenju, LSJ s.~v.\ 5
\item[τοῦτο] §~213
\item[ἐς τοὺς πολλοὺς] §~418; §~372-373; §~196
\item[οἱ μὲν\dots\ Ἀριστόβουλος δὲ\dots] koordinacija značenjski suprotstavljenih rečeničnih članova pomoću čestica; §~370; §~82; za Aristobula v.~gore \textit{(O tekstu)}
\item[λέγουσιν] λέγω govoriti, reći; 3. l. pl. ind. prez. akt; \textit{verbum dicendi} otvara mjesto zavisnoj izričnoj rečenici
\item[ὅτι] veznik uvodi zavisnu izričnu rečenicu, §~518
\item[παίσας] παίω udarati; n. sg. m. r. ptc. aor. akt.
\item[τῷ ξίφει] §~153
\item[διέκοψε] διακόπτω presjeći; 3. l. sg. aor. akt.
\item[λελύσθαι] λύω razriješiti; inf. perf. medpas; dio A+I (ovdje neizrečeni akuzativ bio bi \textgreek[variant=ancient]{τὸν δεσμόν})
\item[ἔφη] φημί reći; 3. l. sg. impf. akt (subjekt je \textgreek[variant=ancient]{Ἀλέξανδρος);} \textit{verbum dicendi} otvara mjesto A+I
\item[λέγει] λέγω govoriti, reći; 3. l. sg. ind. prez. akt; \textit{verbum dicendi} otvara mjesto A+I (suprotno gornjem \textgreek[variant=ancient]{λέγουσιν,} iza kojeg slijedi izrični veznik ὅτι)
\item[ἐξελόντα\dots\ ἐξελκύσαι] ἐξαιρέω izvaditi; a. sg. m. r. ptc. aor. akt; ἐξέλκω izvući, inf. aor. akt.; A+I
\item[τὸν ἕστορα] §~146
\item[τοῦ ῥυμοῦ] §~82
\item[ὃς] §~215; zamjenica uvodi zavisnu odnosnu rečenicu, njezin je antecedent \textgreek[variant=ancient]{τὸν ἕστορα}
\item[ἦν τύλος] imenski predikat, Smyth 909
\item[τύλος] §~82
\item[διαβεβλημένος] διαβάλλω provući; n. sg. m. r. ptc. perf. medpas.
\item[διὰ τοῦ ῥυμοῦ] §~418; §~82
\item[ξυνέχων] συνέχω držati zajedno (ξυν- je atička varijanta prefiksa συν-); n. sg. m. r. ptc. prez. akt. 
\item[τὸ$\langle$ν$\rangle$] u kritičkim izdanjima dopune priređivača (emendacije i konjekture – ono čega nema u rukopisima, a trebalo je biti) označavaju se prelomljenim zagradama
\item[τὸ$\langle$ν$\rangle$ ζυγόν] §~82; riječ se javlja u dva oblika: ζυγόν, τό, i ζυγός, ὁ
\end{description}

%4

{\large
\noindent ὅπως μὲν δὴ ἐπράχθη \\
τὰ ἀμφὶ τῷ δεσμῷ τούτῳ \\
Ἀλεξάνδρῳ \\
οὐκ ἔχω \\
\tabto{2em} ἰσχυρίσασθαι.\\

}

\begin{description}[noitemsep] 
\item[ὅπως μὲν δὴ ἐπράχθη\dots\ ἀπηλλάγη δ' οὖν\dots] koordinacija dviju rečenica pomoću para čestica: a\dots
\item[ὅπως] relativni prilog uvodi zavisnu odnosnu rečenicu koja u glavnoj ima funkciju objekta (predikata \textgreek[variant=ancient]{ἰσχυρίσασθαι})
\item[δὴ] čestica upotrijebljena adverzativno: međutim, bilo kako bilo
\item[ἐπράχθη] πράττω činiti; 3. l. sg. ind. aor. pas.; vršitelj pasivne radnje izrečen je dativom
\item[τὰ ἀμφὶ τῷ δεσμῷ τούτῳ] supstantiviranje članom §~373; §~418; §~82; §~213
\item[Ἀλεξάνδρῳ] §~82; vršitelj pasivne radnje \textgreek[variant=ancient]{ἐπράχθη}
\item[ἔχω] ἔχω + inf. moći nešto učiniti, otvara mjesto dopuni ἰσχυρίσασθαι; 1. l. sg. ind. prez. akt.
\item[ἰσχυρίσασθαι] ἰσχυρίζομαι pouzdano reći; inf. aor. (med.)
\end{description}

%5

{\large
\noindent ἀπηλλάγη δ' οὖν \\
\tabto{2em} ἀπὸ τῆς ἁμάξης \\
αὐτός τε \\
καὶ οἱ ἀμφ' αὐτὸν \\
\tabto{2em} ὡς \uuline{τοῦ λογίου} \\
\tabto{4em} \uuline{τοῦ} ἐπὶ τῇ λύσει \\
\tabto{6em} τοῦ δεσμοῦ \\
\tabto{2em} \uuline{ξυμβεβηκότος}.\\

}

\begin{description}[noitemsep] 

\item[ἀπηλλάγη] ἀπαλλάσσω (atički ἀπαλλάττω) medpas. udaljiti se; 3. l. sg. ind. aor. pas.; neizrečeni subjekt je \textgreek[variant=ancient]{Ἀλέξανδρος}
\item[δ' οὖν] §~68; kombinacija čestica obilježava suprotnost i ujedno naglašava da je potonja ideja bitnija: u svakom slučaju\dots
\item[ἀπὸ τῆς ἁμάξης] §~418; §~97
\item[αὐτός τε] §~40, §~207
\item[αὐτός τε καὶ οἱ ἀμφ' αὐτὸν] koordinacija ostvarena parom sastavnih veznika: i\dots\ i \dots
\item[οἱ ἀμφ' αὐτὸν] §~68; §~207; supstantiviranje članom §~373; §~418
\item[ὡς τοῦ λογίου\dots\ ξυμβεβηκότος] GA §~504; veznik ὡς otvara mjesto adverbnom participu koji izriče (subjektivni) uzrok, §~503b: kao da\dots
\item[τοῦ λογίου] §~82
\item[τοῦ λογίου τοῦ ἐπὶ τῇ λύσει τοῦ δεσμοῦ] atributni položaj §~375; §~82; §~418; §~165
\item[ξυμβεβηκότος] συμβαίνω ostvariti se (ξυν- je atička varijanta prefiksa συν-); g. sg. s. r. ptc. perf. akt.
\end{description}

%6

{\large
\noindent καὶ γὰρ καὶ \\
\tabto{2em} τῆς νυκτὸς ἐκείνης \\
βρονταί τε \\
καὶ σέλας \\
\tabto{2em} ἐξ οὐρανοῦ \\
ἐπεσήμηναν· \\
καὶ \\
\tabto{2em} ἐπὶ τούτοις \\
ἔθυε \\
\tabto{2em} τῇ ὑστεραίᾳ \\
Ἀλέξανδρος \\
τοῖς φήνασι θεοῖς \\
\tabto{2em} τά τε σημεῖα \\
\tabto{2em} καὶ τοῦ δεσμοῦ τὴν λύσιν.\\

}

\begin{description}[noitemsep]

\item[καὶ γὰρ καὶ] pojačano καὶ γὰρ: i zaista\dots
\item[τῆς νυκτὸς ἐκείνης] §~123; §~213; genitiv izražava vrijeme §~408
\item[βρονταί τε] §~97; §~40
\item[βρονταί τε καὶ σέλας] koordinacija ostvarena kombinacijom sastavnih veznika
\item[σέλας] §~159
\item[ἐπεσήμηναν] ἐπισημαίνω pojaviti se; 3. l. pl. ind. aor. akt.
\item[ἐπὶ τούτοις] §~418; §~213; priložna oznaka uzroka
\item[ἔθυε] θύω žrtvovati, prinijeti žrtvu (u znak zahvalnosti), rekcija τινί; 3. l. sg. impf. akt. 
\item[τῇ ὑστεραίᾳ] (sc.\ ἡμέρᾳ) §~103; dativ izražava vrijeme §~415.2
\item[τοῖς φήνασι θεοῖς] §~82, §~193; φαίνω pokazati; d. pl. m. r. ptc. aor. akt. 
\item[φήνασι\dots\ τά τε σημεῖα καὶ\dots\ τὴν λύσιν] §~40, §~82, §~165; imenice su objekti participa; koordinacija ostvarena kombinacijom sastavnih veznika, pri čemu je drugi član para naglašeniji: i\dots
\item[τοῦ δεσμοῦ τὴν λύσιν] atributni položaj §~375 (imenica u genitivu naglašena Smyth 1161 c), §~82, §~165%NZ kaže da je predikatni, ne slažem se
\end{description}


%kraj
