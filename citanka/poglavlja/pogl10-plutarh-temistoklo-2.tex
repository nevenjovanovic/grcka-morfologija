%korektura NZ
%\section*{O autoru}



\section*{O tekstu}

Odlomak opisuje glasovitoga atenskog političara i vojskovođu Temistokla \textgreek[variant=ancient]{(Θεμιστοκλῆς}, oko 524.\ – 459.\ pr.~Kr.). Uočavamo osobine Plutarhova pristupa biografiji: prikazuju se događaji koji najbolje ocrtavaju karakterne osobine Temistokla još od doba dok je bio dječak. 

Temistoklo je bio zaslužan za izgradnju zida od Atene do Pireja i, u kontekstu perzijskog pritiska na Grčku, zagovarao je blisko vojno povezivanje sa Spartom. Sudjelovao je u bitci na Maratonskom polju (490.\ pr.~Kr.), a nakon poraza kod Termopila vodio je grčke snage do pobjede u bitci kod Salamine (480.\ pr.~Kr.). Oko 470.\ pr.~Kr.\ ostracizmom biva protjeran iz Atene. Umire kao namjesnik perzijskog kralja u Magneziji. 

U \textit{Usporednim životopisima} Temistoklova je biografija pridružena onoj rimskog političara Marka Furija Kamila (oko 446. – 365.\ pr.~Kr.). 

%\newpage

\section*{Pročitajte naglas grčki tekst.}
Plut. Themistocles 2.1
%Naslov prema izdanju

\medskip

{\large
\begin{greek}
\noindent ῎Ετι δὲ παῖς ὢν ὁμολογεῖται φορᾶς μεστὸς εἶναι, καὶ τῇ μὲν φύσει συνετός, τῇ δὲ προαιρέσει μεγαλοπράγμων καὶ πολιτικός. ἐν γὰρ ταῖς ἀνέσεσι καὶ σχολαῖς ἀπὸ τῶν μαθημάτων γιγνόμενος, οὐκ ἔπαιζεν οὐδ' ἐρρᾳθύμει καθάπερ οἱ πολλοὶ παῖδες, ἀλλ' εὑρίσκετο λόγους τινὰς μελετῶν καὶ συνταττόμενος πρὸς ἑαυτόν. ἦσαν δ' οἱ λόγοι κατηγορία τινὸς ἢ συνηγορία τῶν παίδων. ὅθεν εἰώθει λέγειν πρὸς αὐτὸν ὁ διδάσκαλος ὡς ‘οὐδὲν ἔσει, παῖ, σὺ μικρόν, ἀλλὰ μέγα πάντως ἀγαθὸν ἢ κακόν’. ἐπεὶ καὶ τῶν παιδεύσεων τὰς μὲν ἠθοποιοὺς ἢ πρὸς ἡδονήν τινα καὶ χάριν ἐλευθέριον σπουδαζομένας ὀκνηρῶς καὶ ἀπροθύμως ἐξεμάνθανε, τῶν δ' εἰς σύνεσιν ἢ πρᾶξιν † λεγομένων δῆλος ἦν ὑπερερῶν παρ' ἡλικίαν ὡς τῇ φύσει πιστεύων.

\end{greek}
}

\section*{Analiza i komentar}

%1

{\large
\begin{greek}
\noindent ῎Ετι δὲ \\
\tabto{2em} παῖς ὢν \\
ὁμολογεῖται \\
\tabto{2em} φορᾶς μεστὸς εἶναι, \\
καὶ\\
τῇ μὲν φύσει \\
\tabto{2em} συνετός, \\
τῇ δὲ προαιρέσει \\
\tabto{2em} μεγαλοπράγμων καὶ πολιτικός.\\

\end{greek}
}

\begin{description}[noitemsep]
\item[δὲ ] postpozitivna čestica, ovdje suprotnog značenja: a\dots
\item[παῖς ] §~127
\item[ὢν] §~141, §~498; εἰμί biti; n. sg. m. r. ptc. prez. akt.
\item[ὁμολογεῖται] ὁμολογέω slagati se, biti istog mišljenja; 3. l. sg. ind. prez. medpas.
\item[μεστὸς εἶναι] N+I, §~491.2
\item[φορᾶς] §~90
\item[φορᾶς μεστὸς] = μεστὸς + g., §~103-104 
\item[εἶναι] εἰμί biti; inf. prez. akt.
\item[συνετός] §~103-104
\item[τῇ μὲν φύσει] §~165, §~414.3; dativ načina ili obzira \textit{(dativus respectus,} Smyth 1516); „po prirodi''
\item[τῇ δὲ προαιρέσει] §~165, §~414.3; dativ načina; „po vlastitom odabiru'', „po vlastitoj odluci''
\item[μὲν\dots\ δὲ] korelacija čestica koja omogućava usporednu strukturu i kojom se nabrajaju različiti ili suprotni pojmovi; μὲν se može ostaviti neprevedeno, δὲ se može prevesti kao „a'' (§~515, §~519)
\item[μεγαλοπράγμων πολιτικός] §~194, §~131, §~103-104

\end{description}

{\large
\begin{greek}
\noindent ἐν γὰρ ταῖς ἀνέσεσι καὶ σχολαῖς \\
\tabto{2em} ἀπὸ τῶν μαθημάτων γιγνόμενος,\\
\tabto{4em} οὐκ ἔπαιζεν οὐδ' ἐρρᾳθύμει \\
\tabto{6em} καθάπερ οἱ πολλοὶ παῖδες, \\
\tabto{4em} ἀλλ' εὑρίσκετο λόγους τινὰς μελετῶν \\
\tabto{6em} καὶ συνταττόμενος πρὸς ἑαυτόν.\\

\end{greek}
}

\begin{description}[noitemsep]
\item[γὰρ] §~517; čestica, ovdje prati objašnjavanje: naime, baš
\item[ἐν ταῖς ἀνέσεσι καὶ σχολαῖς ] §~80, §~90, §~165, §~92; prijedložni izraz, ἐν + d.: u, §~426.1
\item[ἀπὸ τῶν μαθημάτων ] §~80; prijedložni izraz, ἀπὸ + g.: od, §~423.2; ovisno o γιγνόμενος
\item[γιγνόμενος] γίγνομαι postati, biti; n. sg. m. r. ptc. prez. medpas.
\item[ἔπαιζεν ] παίζω igrati se; 3. l. sg. impf. akt.
\item[οὐδ' = οὐδέ] sastavni veznik: niti
\item[ἐρρᾳθύμει] ῥᾳθυμέω ostaviti posao, biti nemaran; 3. l. sg. impf. akt. 
\item[οἱ πολλοὶ παῖδες] §~80, §~82, §~196, §~127 
\item[ἀλλ' = ἀλλά] §~68.c
\item[εὑρίσκετο] εὑρίσκω naći; 3. l. sg. impf. medpas.
\item[λόγους τινὰς] §~82, §~84-87, §~217, §~218.1-2 
\item[μελετῶν] μελετάω vježbati; n. sg. m. r. ptc. prez. akt.
\item[συνταττόμενος] συντάσσω slagati; n. sg. m. r. ptc. prez. medpas.
\item[πρὸς ἑαυτόν] §~208, prijedložni izraz; πρὸς + a.: za, §~435

\end{description}

%3 itd

{\large
\begin{greek}
\noindent ἦσαν δ' οἱ λόγοι \\
\tabto{2em} κατηγορία \\
\tabto{4em} τινὸς \\
\tabto{2em} ἢ συνηγορία\\
\tabto{4em} τῶν παίδων.\\

\end{greek}
}

\begin{description}[noitemsep]
\item[ἦσαν ] εἰμί biti; 3. l. sg. impf. akt.
\item[δ' = δὲ ] postpozitivna čestica, ovdje suprotnog značenja: a\dots; elizija, §~68.c
\item[οἱ λόγοι ] §~80, §~82, §~84-87 
\item[κατηγορία τινὸς] §~80, §~90, §~92-95, imenica otvara mjesto genitivu, §~217, 218.1-2  
\item[συνηγορία τῶν παίδων] §~80, 90, 92-95, imenica otvara mjesto genitivu, §~127

\end{description}

%4

{\large
\begin{greek}
\noindent ὅθεν εἰώθει λέγειν \\
\tabto{2em} πρὸς αὐτὸν \\
ὁ διδάσκαλος\\
\tabto{2em} ὡς \\
\tabto{4em} ‘οὐδὲν ἔσει, \\
\tabto{4em} παῖ, \\
\tabto{4em} σὺ μικρόν, \\
\tabto{4em} ἀλλὰ μέγα πάντως ἀγαθὸν ἢ κακόν’.\\

\end{greek}
}

\begin{description}[noitemsep]
\item[εἰώθει ] ἔθω običavati; 3. l. sg. ind. plpf. akt.
\item[λέγειν ] λέγω govoriti; λέγειν πρὸς + a. govoriti komu; inf. prez. akt.
\item[πρὸς αὐτὸν] §~207, §~435.C.c. 
\item[ὁ διδάσκαλος ] §~80, §~82, §~84-87
\item[οὐδὲν ] §~224.2
\item[ἔσει] εἰμί biti; 2. l. sg. ind. fut. med.
\item[παῖ] §~127
\item[σὺ ] §~205
\item[μικρόν] §~103-104
\item[μέγα ] §~196
\item[πάντως ] §~204
\item[ἀγαθὸν ἢ κακόν] §~103-104; pridjevi kao dio imenskog predikata, Smyth 910

\end{description}

%5

{\large
\begin{greek}
\noindent ἐπεὶ καὶ \\
\tabto{2em} τῶν παιδεύσεων τὰς μὲν ἠθοποιοὺς \\
\tabto{4em} ἢ πρὸς ἡδονήν τινα καὶ χάριν ἐλευθέριον σπουδαζομένας \\
\tabto{4em} ὀκνηρῶς καὶ ἀπροθύμως ἐξεμάνθανε, \\
\tabto{2em} τῶν δ' εἰς σύνεσιν ἢ πρᾶξιν † λεγομένων\\
\tabto{4em} δῆλος ἦν \\
\tabto{4em} ὑπερερῶν παρ' ἡλικίαν \\
\tabto{6em} ὡς τῇ φύσει πιστεύων.\\

\end{greek}
}

\begin{description}[noitemsep]
\item[τῶν παιδεύσεων] §~80, §~165; dijelni genitiv §~395
\item[μὲν\dots\ δ' (= δέ) ] korelacija čestica koja omogućava usporednu strukturu i kojom se nabrajaju različiti ili suprotni pojmovi: jedan\dots\ drugi\dots; elizija, §~68.c
\item[τὰς μὲν ἠθοποιοὺς] §~80, §~103
\item[πρὸς ἡδονήν τινα] §~80, §~90, §~92, §~217-218.1-2; prijedložni izraz; πρὸς + a. = za, §~435.C.c 
\item[τὰς σπουδαζομένας] σπουδάζω baviti se, biti posvećen čemu; a. pl. ž. r. ptc. prez. medpas.
\item[ἐξεμάνθανε] ἐκμανθάνω detaljno naučiti, dobro znati; ovdje: učiti; 3. l. sg. impf. akt.
\item[ὀκνηρῶς ἀπροθύμως] §~204
\item[εἰς σύνεσιν ἢ πρᾶξιν] §~165; prijedložni izraz; εἰς + a. = za, §~419
\item[τῶν\dots\ † λεγομένων] λέγω govoriti; g. pl. ptc. prez. medpas.; † \textit{crux (desperationis)} u kritičkim izdanjima označava mjesto koje priređivači smatraju nepopravljivo iskvarenim (nešto nije u redu, ali nije moguće dokučiti što je stajalo u ispravnoj inačici)
\item[ἦν] εἰμί biti; 3. l. sg. impf. akt.
\item[δῆλος ἦν ] konstrukcija s predikatnim participom: bilo je jasno\dots; §~500-501
\item[ὑπερερῶν = ὑπερορῶν] ὑπεροράω pokazivati prijezir; predikatni particip, §~500-501
\item[ὡς] kao da
\item[παρ' (= παρά) ἡλικίαν] §~68.c, §~80, §~97; prijedložni izraz; παρά + a.: mimo, §~434
\item[πιστεύων] πιστεύω pouzdavati se; n. sg. m. r. ptc. prez. akt.

\end{description}

%kraj
