% Unio ispravke NČ 2019-09
%\section*{O autoru}


\section*{O tekstu}

Izabrani odlomak govori o događaju iz 415.\ pr.~Kr., kad su u Ateni, tijekom noći uoči odlaska na Sicilsku ekspediciju, oskvrnute gotovo sve gradske herme, a to je u narodu izazvalo veliku uznemirenost jer je sve protumačeno kao loš znamen na početku ratnog pohoda. Za oskvrnuće je optužen Alkibijad; optužili su ga političari kojima je smetao da preuzmu vlast u gradu; isticali su i Alkibijadove navodne namjere da u Ateni ukine demokraciju.

%\newpage

\section*{Pročitajte naglas grčki tekst.}

Thuc.\ Historiae 6.27

%Naslov prema izdanju

\medskip

\begin{greek}
{\large
{ \noindent Ἐν δὲ τούτῳ, ὅσοι Ἑρμαῖ ἦσαν λίθινοι ἐν τῇ πόλει τῇ Ἀθηναίων (εἰσὶ δὲ κατὰ τὸ ἐπιχώριον, ἡ τετράγωνος ἐργασία, πολλοὶ καὶ ἐν ἰδίοις προθύροις καὶ ἐν ἱεροῖς), μιᾷ νυκτὶ οἱ πλεῖστοι περιεκόπησαν τὰ πρόσωπα. καὶ τοὺς δράσαντας ᾔδει οὐδείς, ἀλλὰ μεγάλοις μηνύτροις δημοσίᾳ οὗτοί τε ἐζητοῦντο καὶ προσέτι ἐψηφίσαντο, καὶ εἴ τις ἄλλο τι οἶδεν ἀσέβημα γεγενημένον, μηνύειν ἀδεῶς τὸν βουλόμενον καὶ ἀστῶν καὶ ξένων καὶ δούλων. καὶ τὸ πρᾶγμα μειζόνως ἐλάμβανον· τοῦ τε γὰρ ἔκπλου οἰωνὸς ἐδόκει εἶναι καὶ ἐπὶ ξυνωμοσίᾳ ἅμα νεωτέρων πραγμάτων καὶ δήμου καταλύσεως  γεγενῆσθαι. μηνύεται οὖν ἀπὸ μετοίκων τέ τινων καὶ ἀκολούθων περὶ μὲν τῶν Ἑρμῶν οὐδέν, ἄλλων δὲ ἀγαλμάτων περικοπαί τινες πρότερον ὑπὸ νεωτέρων μετὰ παιδιᾶς καὶ οἴνου γεγενημέναι, καὶ τὰ μυστήρια ἅμα ὡς ποιεῖται ἐν οἰκίαις ἐφ' ὕβρει· ὧν καὶ τὸν Ἀλκιβιάδην ἐπῃτιῶντο. καὶ αὐτὰ ὑπολαμβάνοντες οἱ μάλιστα τῷ Ἀλκιβιάδῃ ἀχθόμενοι ἐμποδὼν ὄντι σφίσι μὴ αὐτοῖς τοῦ δήμου βεβαίως προεστάναι, καὶ νομίσαντες, εἰ αὐτὸν ἐξελάσειαν, πρῶτοι ἂν εἶναι, ἐμεγάλυνον καὶ ἐβόων ὡς ἐπὶ δήμου καταλύσει τά τε μυστικὰ καὶ ἡ τῶν Ἑρμῶν περικοπὴ γένοιτο καὶ οὐδὲν εἴη αὐτῶν ὅ τι οὐ μετ' ἐκείνου ἐπράχθη, ἐπιλέγοντες τεκμήρια τὴν ἄλλην αὐτοῦ ἐς τὰ ἐπιτηδεύματα οὐ δημοτικὴν παρανομίαν. 

}
}
\end{greek}

\section*{Analiza i komentar}

%1

{\large
\begin{greek}
\noindent Ἐν δὲ τούτῳ, \\
ὅσοι Ἑρμαῖ ἦσαν λίθινοι \\
\tabto{2em} ἐν τῇ πόλει τῇ Ἀθηναίων \\
(εἰσὶ δὲ κατὰ τὸ ἐπιχώριον, \\
ἡ τετράγωνος ἐργασία, \\
πολλοὶ \\
καὶ ἐν ἰδίοις προθύροις \\
καὶ ἐν ἱεροῖς), \\
\tabto{2em} μιᾷ νυκτὶ \\
οἱ πλεῖστοι \\
περιεκόπησαν \\
\tabto{2em} τὰ πρόσωπα.\\

\end{greek}
}

\begin{description}[noitemsep]
\item[Ἐν δὲ τούτῳ] sc.\ καιρῷ (misli se na završne pripreme za odlazak na Sicilsku ekspediciju); §~213
\item[ὅσοι ] §~219
\item[Ἑρμαῖ λίθινοι] §~108; §~103
\item[ἦσαν] εἰμί biti; 3. l. pl. impf. akt.
\item[ἐν τῇ πόλει τῇ Ἀθηναίων] §~165; §~103; §~375
\item[εἰσὶ ] εἰμί biti; 3. l. pl. ind. prez. akt.
\item[κατὰ τὸ ἐπιχώριον] po običaju, uobičajeno; §~103
\item[ἡ τετράγωνος ἐργασία] §~103; §~90
\item[πολλοὶ] §~196
\item[ἐν ἰδίοις προθύροις καὶ ἐν ἱεροῖς] §~103; §~82
\item[μιᾷ νυκτὶ ] §~224; §~115
\item[οἱ πλεῖστοι] §~202
\item[περιεκόπησαν] περικόπτω okrnjiti; 3. l. pl. ind. aor. pas.
\item[τὰ πρόσωπα] §~82; akuzativ obzira §~389

\end{description}

%2

{\large
\begin{greek}
\noindent καὶ τοὺς δράσαντας \\
ᾔδει οὐδείς, \\
ἀλλὰ μεγάλοις μηνύτροις \\
\tabto{2em} δημοσίᾳ \\
οὗτοί τε ἐζητοῦντο \\
καὶ προσέτι ἐψηφίσαντο, \\
\tabto{2em} καὶ εἴ τις \\
\tabto{2em} ἄλλο τι οἶδεν \\
\tabto{2em} ἀσέβημα γεγενημένον, \\
\tabto{4em} \underline{μηνύειν} ἀδεῶς \underline{τὸν βουλόμενον} \\
\tabto{6em} καὶ ἀστῶν καὶ ξένων καὶ δούλων.\\

\end{greek}
}

\begin{description}[noitemsep]
\item[τοὺς δράσαντας] δράω činiti; a. pl. m. r. ptc. aor. akt.
\item[ᾔδει] οἶδα znati; 3. l. sg. ind. plpf. akt.
\item[οὐδείς] §~224
\item[μεγάλοις μηνύτροις] §~196; §~82
\item[δημοσίᾳ] o javnom trošku
\item[οὗτοί] §~213
\item[ἐζητοῦντο] ζητέω tražiti; 3. l. pl. impf. medpas.
\item[ἐψηφίσαντο] o njima je glasovanjem odlučeno\dots; ψηφίζω glasovati, odlučivati; 3. l. pl. ind. aor. medpas.; izraz otvara mjesto A+I \textgreek[variant=ancient]{μηνύειν τὸν βουλόμενον}
\item[τις ] §~217
\item[τι ἄλλο ἀσέβημα] §~217; §~212; §~123
\item[οἶδεν] οἶδα znati; 3. l. sg. ind. perf. akt.
\item[γεγενημένον] γίγνομαι postajem, nastajem; a. sg. sr. r. ptc. perf. medpas.
\item[μηνύειν ] μηνύω otkriti, prijaviti; inf. prez. akt.
\item[τὸν βουλόμενον] βούλομαι željeti; a. sg. m. r. ptc. prez. medpas. (v. bilj. uz generički član §~370) 
\item[ἀστῶν καὶ ξένων καὶ δούλων] §~82
\item[εἴ\dots\ γεγενημένον] realna pogodbena rečenica §~486

\end{description}

%3

{\large
\begin{greek}
\noindent καὶ τὸ πρᾶγμα \\
μειζόνως ἐλάμβανον·\\

\end{greek}
}

\begin{description}[noitemsep]
\item[τὸ πρᾶγμα] §~123
\item[μειζόνως] s većom ozbiljnošću; §~200; §~203
\item[ἐλάμβανον] λαμβάνω uzimati; 3. l. pl. impf. akt.

\end{description}


%4

{\large
\begin{greek}
\noindent τοῦ τε γὰρ ἔκπλου \\
\underline{οἰωνὸς} ἐδόκει \underline{εἶναι} \\
καὶ ἐπὶ ξυνωμοσίᾳ \\
\tabto{2em} ἅμα νεωτέρων πραγμάτων \\
\tabto{2em} καὶ δήμου καταλύσεως \\
\underline{γεγενῆσθαι}.\\

\end{greek}
}

\begin{description}[noitemsep]
\item[τοῦ ἔκπλου] §~107
\item[οἰωνὸς] §~82
\item[ἐδόκει] δοκεῖ τινι čini se nekome (bezlično); 3. l. sg. impf. akt.
\item[εἶναι] εἰμί biti; inf. prez. akt.
\item[οἰωνὸς ἐδόκει εἶναι καὶ γεγενῆσθαι] N+I (subjekt imperfekta \textgreek[variant=ancient]{ἐδόκει} je \textgreek[variant=ancient]{τὸ πρᾶγμα)}
\item[ἐπὶ ξυνωμοσίᾳ] §~90
\item[νεωτέρων πραγμάτων] §~197; §~123
\item[δήμου καταλύσεως] δῆμος ovdje u značenju ``demokracija''; §~82; §~165
\item[γεγενῆσθαι] γίγνομαι postajem, nastajem; inf. perf. medpas.

\end{description}


%5

{\large
\begin{greek}
\noindent μηνύεται οὖν \\
\tabto{2em} ἀπὸ μετοίκων τέ τινων \\
\tabto{4em} καὶ ἀκολούθων \\
\tabto{2em} περὶ μὲν τῶν Ἑρμῶν οὐδέν, \\
\tabto{2em} ἄλλων δὲ ἀγαλμάτων περικοπαί τινες \\
\tabto{4em} πρότερον \\
\tabto{4em} ὑπὸ νεωτέρων \\
\tabto{4em} μετὰ παιδιᾶς καὶ οἴνου \\
\tabto{2em} γεγενημέναι, \\
\tabto{2em} καὶ τὰ μυστήρια ἅμα \\
\tabto{4em} ὡς ποιεῖται ἐν οἰκίαις \\
\tabto{6em} ἐφ' ὕβρει·\\
\tabto{8em} ὧν καὶ τὸν Ἀλκιβιάδην \\
\tabto{8em} ἐπῃτιῶντο.\\

\end{greek}
}

\begin{description}[noitemsep]
\item[μηνύεται] μηνύω prijavljivati; 3. l. sg. ind. prez. medpas. slaže se sa subjektom περικοπαί §~361
\item[ἀπὸ μετοίκων τέ τινων καὶ ἀκολούθων] §~82
\item[ἄλλων ἀγαλμάτων] §~212; §~123
\item[περικοπαί] §~90
\item[πρότερον] sr. rod pridjeva upotrijebljen priloški
\item[ὑπὸ νεωτέρων] οἱ νεώτεροι mladići
\item[μετὰ παιδιᾶς καὶ οἴνου] §~90; §~82
\item[γεγενημέναι] γίγνομαι postajem, nastajem; n. pl. ž. r. ptc. perf. medpas. (ovisno o περικοπαί)
\item[τὰ μυστήρια] §~82
\item[ὡς] veznik uvodi izričnu rečenicu §~467; manje obilježeni poredak bio bi \textgreek[variant=ancient]{ὡς ποιεῖται τὰ μυστήρια}
\item[ποιεῖται] ποιέω činiti; 3. l. sg. ind prez. medpas., slaže se s \textgreek[variant=ancient]{τὰ μυστήρια} §~361
\item[ἐν οἰκίαις ] §~90
\item[ἐφ' ὕβρει] §~68, §~165
\item[ὧν] sc.\ πραγμάτων; §~215
\item[τὸν Ἀλκιβιάδην] §~100 
\item[ἐπῃτιῶντο] ἐπαιτιάομαι τινός optuživati za što; 3. l. pl. impf. medpas.

\end{description}


%6

{\large
\begin{greek}
\noindent καὶ αὐτὰ ὑπολαμβάνοντες \\
οἱ μάλιστα τῷ Ἀλκιβιάδῃ ἀχθόμενοι \\
\tabto{2em} ἐμποδὼν ὄντι σφίσι \\
\tabto{4em} μὴ αὐτοῖς τοῦ δήμου βεβαίως προεστάναι, \\
καὶ νομίσαντες, \\
\tabto{2em} εἰ αὐτὸν ἐξελάσειαν, \\
\underline{πρῶτοι ἂν εἶναι}, \\
ἐμεγάλυνον καὶ ἐβόων \\
\tabto{2em} ὡς ἐπὶ δήμου καταλύσει \\
\tabto{4em} τά τε μυστικὰ \\
\tabto{4em} καὶ ἡ τῶν Ἑρμῶν περικοπὴ \\
\tabto{2em} γένοιτο \\
\tabto{2em} καὶ οὐδὲν εἴη αὐτῶν \\
\tabto{4em} ὅ τι οὐ μετ' ἐκείνου ἐπράχθη, \\
ἐπιλέγοντες τεκμήρια \\
\tabto{2em} τὴν ἄλλην \\
\tabto{4em} αὐτοῦ \\
\tabto{4em} ἐς τὰ ἐπιτηδεύματα \\
\tabto{2em} οὐ δημοτικὴν παρανομίαν.\\

\end{greek}
}

\begin{description}[noitemsep]
\item[αὐτὰ] §~207
\item[ὑπολαμβάνοντες] ὑπολαμβάνω poduzimati; n. pl. m. r. ptc. prez. akt.
\item[οἱ ἀχθόμενοι] ἄχθομαι ljutiti se; n. pl. m. r. ptc. prez. medpas.
\item[ἐμποδὼν ] na putu
\item[ὄντι] εἰμί biti; d. sg. m. r. ptc. prez. akt.
\item[σφίσι] §~206.5
\item[προεστάναι] προίσταμαι τινός stajati komu na čelu; inf. perf. akt.; u rečenici mu mjesto otvara izraz \textgreek[variant=ancient]{ἐμποδὼν ὄντι}
\item[μὴ\dots\ προεστάναι] negirana posljedična rečenica (posljedica pomišljena) §~473
\item[νομίσαντες] νομίζω smatrati; n. pl. m. r. ptc. aor. akt.; \textit{verbum sentiendi} otvara mjesto N+I
\item[ἐξελάσειαν] ἐξελαύνω protjerivati; 3. l. pl. opt. aor. akt.
\item[εἰ αὐτὸν ἐξελάσειαν] potencijalna pogodbena protaza čija je apodoza uobličena u N+I; §~477 
\item[πρῶτοι ἂν εἶναι] N+I kojemu mjesto u rečenici otvara νομίσαντες (ἂν uvršten prema pravilu iz §~506); §~223
\item[ἐμεγάλυνον] μεγαλύνω uvećavati, naglašavati; 3. l. pl. impf. akt.
\item[ἐβόων] βοάω vikati; 3. l. pl. impf. akt.
\item[γένοιτο] γίγνομαι postajem, nastajem; 3. l. sg. opt. aor. medpas.
\item[εἴη] εἰμί biti; 3. l. sg. opt. prez. akt.
\item[ὡς\dots\ αὐτῶν] izrična rečenica §~467
\item[μετ' ἐκείνου ] u dogovoru s njim; §~68 
\item[ἐπράχθη] πράττω činiti; 3. l. sg. ind. aor. pas.
\item[ὅ τι\dots\ ἐπράχθη] odnosna rečenica; §~217; §~482
\item[ἐπιλέγοντες ] ἐπιλέγω izabirati; n. pl. m. r. ptc. prez. akt.
\item[τεκμήρια]  §~82
\item[τὴν ἄλλην αὐτοῦ οὐ δημοτικὴν παρανομίαν.] ἄλλος \textit{ovdje} loš, nevaljao; §~90
\item[ἐς τὰ ἐπιτηδεύματα] §~123

\end{description}



%kraj
