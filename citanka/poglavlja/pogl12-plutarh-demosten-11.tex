% Unesi korekture NČ i NZ, 2019-09-24



\section*{O tekstu}

Životopis Demostena (384.\ – 322.\ pr.~Kr.), najslavnijega grčkog govornika, Plutarh je pridružio životopisu najslavnijega rimskog govornika, Cicerona (106.\ – 44.\ pr.~Kr.). U izabranom odlomku pripovijeda se kako je Demosten vježbao govorničke vještine i pokušavao ispraviti vlastite nedostatke te koliko je cijenio govorničku izvedbu.

%\newpage

\section*{Pročitajte naglas grčki tekst.}
Plut.\ Demosthenes 11
%Naslov prema izdanju

\medskip

{\large
\begin{greek}

\noindent Τοῖς δὲ σωματικοῖς ἐλαττώμασι τοιαύτην ἐπῆγεν ἄσκησιν, ὡς ὁ Φαληρεὺς Δημήτριος ἱστορεῖ, λέγων αὐτοῦ Δημοσθένους ἀκοῦσαι πρεσβύτου γεγονότος· τὴν μὲν γὰρ ἀσάφειαν καὶ τραυλότητα τῆς γλώττης ἐκβιάζεσθαι καὶ διαρθροῦν εἰς τὸ στόμα ψήφους λαμβάνοντα καὶ ῥήσεις ἅμα λέγοντα, τὴν δὲ φωνὴν γυμνάζειν ἐν τοῖς δρόμοις καὶ ταῖς πρὸς τὰ σιμ' ἀναβάσεσι διαλεγόμενον καὶ λόγους τινὰς ἢ στίχους ἅμα τῷ πνεύματι πυκνουμένῳ προφερόμενον· εἶναι δ' αὐτῷ μέγα κάτοπτρον οἴκοι, καὶ πρὸς τοῦτο τὰς μελέτας ἱστάμενον ἐξ ἐναντίας περαίνειν. 

λέγεται δ' ἀνθρώπου προσελθόντος αὐτῷ δεομένου συνηγορίας καὶ διεξιόντος ὡς ὑπό του λάβοι πληγάς, ``ἀλλὰ σύ γε'', φάναι τὸν Δημοσθένην, ``τούτων ὧν λέγεις οὐδὲν πέπονθας.'' ἐπιτείναντος δὲ τὴν φωνὴν τοῦ ἀνθρώπου καὶ βοῶντος ``ἐγὼ Δημόσθενες οὐδὲν πέπονθα;'' ``νὴ Δία'' φάναι, ``νῦν ἀκούω φωνὴν ἀδικουμένου καὶ πεπονθότος.'' οὕτως ᾤετο μέγα πρὸς πίστιν εἶναι τὸν τόνον καὶ τὴν ὑπόκρισιν τῶν λεγόντων.

\end{greek}

}

\section*{Analiza i komentar}


%1

{\large
\begin{greek}
\noindent Τοῖς δὲ σωματικοῖς ἐλαττώμασι \\
\tabto{2em} τοιαύτην ἐπῆγεν ἄσκησιν,\\
\tabto{4em} ὡς ὁ Φαληρεὺς Δημήτριος ἱστορεῖ,\\
\tabto{6em} λέγων \\
\tabto{8em} αὐτοῦ Δημοσθένους \\
\tabto{6em} \underline{ἀκοῦσαι} \\
\tabto{8em} πρεσβύτου γεγονότος·\\

\end{greek}
}

\begin{description}[noitemsep]
\item[δὲ] čestica povezuje rečenicu s prethodnom: a\dots
\item[Τοῖς δὲ σωματικοῖς ἐλαττώμασι] §~103, §~123
\item[τοιαύτην ἄσκησιν] §~213, §~165
\item[ἐπῆγεν] ἐπάγω τί τινι primjenjivati što na što; 3. l. sg. impf. akt.
\item[ὁ Φαληρεὺς Δημήτριος] §~175, §~82
\item[ἱστορεῖ] ἱστορέω pripovijedati; 3. l. sg. ind. prez. akt.
\item[ὡς\dots\ ἱστορεῖ] načinska rečenica; načinski prilog ὡς u parentetičkoj uporabi (kao umetak) Smyth 2992; LSJ ὡς A.II.1
\item[λέγων] λέγω govoriti; n. sg. m. r. ptc. prez. akt.; kao \textit{verbum dicendi} otvara mjesto infinitivu u službi objekta
\item[ἀκοῦσαι] ἀκούω τινός slušati koga; inf. aor. akt.; dio konstrukcije N+I (§~491.2) ovisne o glagolu λέγω 
\item[γεγονότος] γίγνομαι postati; g. sg. m. r. ptc. perf. akt.
\item[αὐτοῦ Δημοσθένους πρεσβύτου γεγονότος] §~207, §~153, §~100, §~123, objekt u genitivu uz glagol ἀκούω (genitiv s predikatnim participom) §~502

\end{description}

{\large
\begin{greek}
\noindent τὴν μὲν γὰρ ἀσάφειαν \\
καὶ τραυλότητα \\
\tabto{2em} τῆς γλώττης \\
\underline{ἐκβιάζεσθαι καὶ διαρθροῦν} \\
\tabto{2em} εἰς τὸ στόμα ψήφους \underline{λαμβάνοντα} \\
\tabto{2em} καὶ ῥήσεις ἅμα \underline{λέγοντα},\\
τὴν δὲ φωνὴν \underline{γυμνάζειν} \\
\tabto{2em} ἐν τοῖς δρόμοις \\
\tabto{2em} καὶ ταῖς πρὸς τὰ σιμ' ἀναβάσεσι \\
\underline{διαλεγόμενον} \\
καὶ λόγους τινὰς ἢ στίχους \\
\tabto{2em} ἅμα τῷ πνεύματι πυκνουμένῳ \\
\underline{προφερόμενον}·\\

\end{greek}
}

\begin{description}[noitemsep]
\item[τὴν μὲν γὰρ ἀσάφειαν καὶ τραυλότητα τῆς γλώττης] §~97, §~123, §~90
\item[τραυλότητα] općenito: govorna mana, mucanje, izgovaranje \textit{r} umjesto \textit{l} (rotacizam)
\item[ἐκβιάζεσθαι] ἐκβιάζω tjerati; inf. prez. medpas.
\item[διαρθροῦν] διαρθρόω artikulirano izgovarati; inf. prez. akt.
\item[εἰς τὸ στόμα] §~123
\item[λαμβάνοντα] λαμβάνω uzimati; a. sg. m. r. ptc. prez. akt.
\item[λέγοντα] λέγω govoriti; a. sg. m. r. ptc. prez. akt.
\item[ἐκβιάζεσθαι καὶ διαρθροῦν\dots\ λαμβάνοντα καὶ λέγοντα] A+I, konstrukcija ovisna o participu λέγων iz prethodne cjeline; §~491 
\item[τὴν δὲ φωνὴν] §~90; uočite koordinaciju česticama μὲν\dots\ δὲ\dots
\item[ἐν τοῖς δρόμοις] \textbf{καὶ ταῖς πρὸς τὰ σιμ' ἀναβάσεσι} τὰ σιμ' = τὰ σιμά; §~82, §~103, §~165
\item[διαλεγόμενον] διαλέγω; διαλέγομαι razgovarati; a. sg. m. r. ptc. prez. medpas.
\item[καὶ λόγους τινὰς ἢ στίχους] §~82, §~217
\item[τῷ πνεύματι] §~123
\item[πυκνουμένῳ] u jednom dahu; πυκνόω stegnuti; d. sg. sr. r. ptc. prez. medpas.
\item[προφερόμενον] προφέρω iznositi, izgovarati; a. sg. m. r. ptc. prez. medpas.
\item[γυμνάζειν\dots\ διαλεγόμενον καὶ προφερόμενον] kao gore, A+I, konstrukcija ovisna o participu λέγων; §~491

\end{description}

%3 itd
{\large
\begin{greek}
\noindent \underline{εἶναι} δ' αὐτῷ \\
\underline{μέγα κάτοπτρον} \\
\tabto{2em} οἴκοι,\\
καὶ πρὸς τοῦτο \\
\tabto{2em} τὰς μελέτας \\
\underline{ἱστάμενον} ἐξ ἐναντίας \\
\underline{περαίνειν}.\\

\end{greek}
}

\begin{description}[noitemsep]
\item[εἶναι] εἰμί biti, εἶναι + d. imati (posvojni dativ); §~412.2; inf. prez. akt.
\item[αὐτῷ] §~207
\item[μέγα κάτοπτρον] §~196, §~82
\item[οἴκοι] razlikuj od οἶκοι
\item[εἶναι μέγα κάτοπτρον] A+I, konstrukcija u službi objekta participa λέγων iz prve cjeline \textit{(verbum dicendi)} §~491
\item[πρὸς τοῦτο] §~213
\item[τὰς μελέτας] §~90
\item[ἱστάμενον] ἵστημι med. postavljati se, stajati; a. sg. m. r. ptc. prez. medpas.
\item[ἐξ ἐναντίας] §~82
\item[περαίνειν] περαίνω recitirati, ponavljati od početka do kraja; inf. prez. akt.
\item[ἱστάμενον\dots\ περαίνειν] A+I, konstrukcija ovisna o obliku λέγων \textit{(verbum dicendi)} §~491


\end{description}

%4
{\large
\begin{greek}
\noindent λέγεται δ' \\
\uuline{ἀνθρώπου προσελθόντος} αὐτῷ \\
\tabto{2em} \uuline{δεομένου} συνηγορίας \\
\tabto{2em} καὶ \uuline{διεξιόντος} \\
\tabto{4em} ὡς \\
\tabto{6em} ὑπό του \\
\tabto{4em} λάβοι πληγάς,\\
``ἀλλὰ σύ γε'',\\
\underline{φάναι τὸν Δημοσθένην}, \\
\tabto{2em} ``τούτων ὧν λέγεις \\
\tabto{4em} οὐδὲν πέπονθας.''\\

\end{greek}
}

\begin{description}[noitemsep]
\item[λέγεται] λέγω govoriti; 3. l. sg. ind. prez. medpas.
\item[ἀνθρώπου] §~82
\item[προσελθόντος] προσέρχομαι dolaziti; g. sg. m. r. ptc. aor. akt.
\item[δεομένου] δέομαι τινός trebati; g. sg. m. r. ptc. prez. medpas.
\item[συνηγορίας] §~90
\item[διεξιόντος] διέξειμι prepričavati; g. sg. m. r. ptc. prez. akt.
\item[λάβοι] λαμβάνω dobivati; 3. l. sg. opt. aor. akt.
\item[ὑπό του] od koga; §~217
\item[πληγάς] §~90
\item[ὡς\dots\ λάβοι] izrični veznik uvodi zavisnu izričnu rečenicu: da\dots
\item[σύ] §~205
\item[φάναι] φημί govoriti, reći; inf. prez. akt.
\item[φάναι τὸν Δημοσθένην] A+I, konstrukcija ovisna o obliku λέγεται; §~491
\item[τούτων ὧν] §~215; ὧν uvodi odnosnu rečenicu, τούτων je antecedent; dijelni genitiv
\item[ὧν λέγεις] odnosna rečenica; umjesto τούτων ἃ λέγεις; asimilacija relativa §~444
\item[λέγεις] λέγω govoriti; 2. l. sg. ind. prez. akt.
\item[οὐδὲν] §~224
\item[πέπονθας] πάσχω trpjeti; 2. l. sg. ind. perf. akt.


\end{description}

%5
{\large
\begin{greek}
\noindent \uuline{ἐπιτείναντος} δὲ \\
\tabto{2em} τὴν φωνὴν \\
\uuline{τοῦ ἀνθρώπου} \\
καὶ \uuline{βοῶντος} \\
``ἐγὼ \\
\tabto{2em} Δημόσθενες \\
οὐδὲν πέπονθα;''\\
``νὴ Δία''\\
\tabto{2em} \underline{φάναι},\\
``νῦν ἀκούω \\
\tabto{2em} φωνὴν \\
\tabto{4em} ἀδικουμένου καὶ πεπονθότος.''\\

\end{greek}
}

\begin{description}[noitemsep]
\item[ἐπιτείναντος] ἐπιτείνω napinjati; g. sg. m. r. ptc. aor. akt.
\item[τὴν φωνὴν] §~90
\item[βοῶντος] βοάω vikati; g. sg. m. r. ptc. prez. akt.
\item[ἐγὼ] §~205
\item[πέπονθα] πάσχω trpjeti; 1. l. sg. ind. perf. akt.
\item[νὴ Δία] tako je, Zeusa mi; §~178
\item[ἀκούω] ἀκούω τινός čuti koga; 1. l. sg. ind. prez. akt.
\item[φάναι] A+I čiji je akuzativ, kao i gore, τὸν Δημοσθένην, konstrukcija u službi objekta glagolskog oblika λέγεται \textit{(verbum dicendi)} §~491
\item[ἀδικουμένου] ἀδικέω činiti nepravdu; g. sg. m. r. ptc. prez. medpas.
\item[πεπονθότος] πάσχω trpjeti; g. sg. m. r. ptc. perf. akt.

\end{description}

%6
{\large
\begin{greek}
\noindent  οὕτως ᾤετο μέγα \\
\tabto{2em} πρὸς πίστιν \\
\tabto{4em} \underline{εἶναι} \\
\tabto{4em} \underline{τὸν τόνον καὶ τὴν ὑπόκρισιν} \\
\tabto{6em} τῶν λεγόντων.\\

\end{greek}
}

\begin{description}[noitemsep]
\item[ᾤετο] οἴομαι smatrati; 3. l. sg. impf. medpas.; kao \textit{verbum sentiendi} otvara mjesto akuzativu s infinitivom u službi objekta
\item[πρὸς πίστιν\dots\ τὴν ὑπόκρισιν] §~165
\item[μέγα εἶναι] μέγα εἰμί mnogo vrijediti; inf. prez. akt.
\item[μέγα εἶναι] dio A+I u službi objekta, konstrukcija ovisna o obliku ᾤετο \textit{(verbum sentiendi)} §~491
\item[τὸν τόνον] τόνος tehnički termin u retorici: snaga, napetost, intenzitet
\item[τὴν ὑπόκρισιν] ὑπόκρισις tehnički termin u retorici: način na koji govornik izvodi govor
\item[τῶν λεγόντων] λέγω govoriti; g. pl. m. r. ptc. prez. akt.

\end{description}

%kraj
