%Unesi korekture NČ 2019-09-19
%\section*{O autoru}


\section*{O tekstu}

Aristotelova je \textit{Retorika} teoretski i praktični priručnik govorništva u tri knjige koji se bavi metodama uvjeravanja riječima. Uvelike se oslanja na Platonovo poimanje istinskog govorničkog umijeća izloženo u \textit{Fedru}. Tri su sredstva uvjeravanja: 1) osobnost govornika \textgreek[variant=ancient]{(ἦθος),} 2) sposobnost da se emocionalno djeluje na slušatelja \textgreek[variant=ancient]{(πάθος)} i 3) umijeće da se iskaz tako oblikuje da djeluje istinito ili vjerojatno \textgreek[variant=ancient]{(λόγος).}

Nakon što je razmatrao posebne dokaze, u ovom poglavlju druge knjige Aristotel se osvrće na dokaze koji su zajednički svim granama retorike: primjer \textgreek[variant=ancient]{(παράδειγμα)} i entimem \textgreek[variant=ancient]{(ἐνθύμημα).}
%\newpage

\section*{Pročitajte naglas grčki tekst.}

Arist.\ Rhetorica 1393a 28

%Naslov prema izdanju

\medskip

\begin{greek}
{\large
{ \noindent Παραδειγμάτων δὲ εἴδη δύο· ἓν μὲν γάρ ἐστιν παραδείγματος εἶδος τὸ λέγειν πράγματα προγεγενημένα, ἓν δὲ τὸ αὐτὸν ποιεῖν. τούτου δὲ ἓν μὲν παραβολὴ ἓν δὲ λόγοι, οἷον οἱ Αἰσώπειοι καὶ Λιβυκοί. ἔστιν δὲ τὸ μὲν πράγματα λέγειν τοιόνδε τι, ὥσπερ εἴ τις λέγοι ὅτι δεῖ πρὸς βασιλέα παρασκευάζεσθαι καὶ μὴ ἐᾶν Αἴγυπτον χειρώσασθαι· καὶ γὰρ πρότερον Δαρεῖος οὐ πρότερον διέβη πρὶν Αἴγυπτον ἔλαβεν, λαβὼν δὲ διέβη, καὶ πάλιν Ξέρξης οὐ πρότερον ἐπεχείρησεν πρὶν ἔλαβεν, λαβὼν δὲ διέβη, ὥστε καὶ οὗτος ἐὰν λάβῃ, διαβήσεται, διὸ οὐκ ἐπιτρεπτέον. παραβολὴ δὲ τὰ Σωκρατικά, οἷον εἴ τις λέγοι ὅτι οὐ δεῖ κληρωτοὺς ἄρχειν· ὅμοιον γὰρ ὥσπερ ἂν εἴ τις τοὺς ἀθλητὰς κληροίη μὴ οἳ δύνανται ἀγωνίζεσθαι ἀλλ' οἳ ἂν λάχωσιν, ἢ τῶν πλωτήρων ὅντινα δεῖ κυβερνᾶν κληρώσειεν, ὡς δέον τὸν λαχόντα ἀλλὰ μὴ τὸν ἐπιστάμενον. λόγος δέ, οἷος ὁ Στησιχόρου περὶ Φαλάριδος καὶ $\langle$ὁ$\rangle$ Αἰσώπου ὑπὲρ τοῦ δημαγωγοῦ.

}
}
\end{greek}

\section*{Analiza i komentar}

%1

{\large
\begin{greek}
\noindent Παραδειγμάτων δὲ εἴδη δύο· \\
\tabto{2em} ἓν μὲν γάρ ἐστιν παραδείγματος εἶδος \\
\tabto{4em} τὸ λέγειν πράγματα προγεγενημένα, \\
\tabto{2em} ἓν δὲ \\
\tabto{4em} τὸ αὐτὸν ποιεῖν. \\
τούτου δὲ \\
\tabto{2em} ἓν μὲν παραβολὴ \\
\tabto{2em} ἓν δὲ λόγοι, \\
\tabto{4em} οἷον οἱ Αἰσώπειοι καὶ Λιβυκοί.\\

\end{greek}
}

\begin{description}[noitemsep]
\item[Παραδειγμάτων] §~123; partitivni genitiv §~395
\item[δὲ] čestica δέ povezuje rečenicu s prethodnom kao suprotni veznik: a\dots; §~515
\item[εἴδη δύο] §~153, §~224; izostavljena kopula ἐστιν; imenski predikat Smyth 909
\item[ἓν ] §~224; ovisno o εἶδος 
\item[ἓν μὲν\dots\  ἓν δὲ\dots] koordinacija parom čestica §~519.7
\item[γάρ] jer; čestica ima funkciju uzročnog veznika §~517
\item[τὸ λέγειν] λέγω govoriti, navoditi; inf. prez. akt.; supstantiviranje članom §~373
\item[πράγματα] §~123
\item[προγεγενημένα] προγίγνομαι prije događati se; a. pl. sr. r. ptc. perf. medpas.
\item[ἓν] sc.\ εἶδος (na ovom mjestu i dalje)
\item[τὸ αὐτὸν ποιεῖν] §~207; ποιέω činiti, proizvoditi, iznalaziti; inf. prez. akt.; supstantivirani A+I §~373
\item[τούτου] §~213.2
\item[παραβολὴ] \textit{ovdje} usporedba, ilustracija, parabola; §~90; izostavljena kopula ἐστιν; imenski predikat Smyth 909
\item[λόγοι] \textit{ovdje} basne; §~82; izostavljena kopula ἐστιν; imenski predikat Smyth 909
\item[οἷον] kao, na primjer; §~219
\item[οἱ Αἰσώπειοι καὶ Λιβυκοί] sc.\ λόγοι, Ezopove i libijske; §~103
\end{description}

%2

{\large
\begin{greek}
\noindent ἔστιν δὲ τὸ μὲν πράγματα λέγειν \\
\tabto{2em} τοιόνδε τι, \\
\tabto{3em} ὥσπερ εἴ τις λέγοι \\
\tabto{4em} ὅτι δεῖ πρὸς βασιλέα παρασκευάζεσθαι \\
\tabto{4em} καὶ μὴ ἐᾶν Αἴγυπτον χειρώσασθαι· \\
καὶ γὰρ πρότερον \\
\tabto{2em} Δαρεῖος οὐ πρότερον διέβη \\
\tabto{3em} πρὶν Αἴγυπτον ἔλαβεν, \\
\tabto{4em} λαβὼν δὲ διέβη, \\
καὶ πάλιν \\
\tabto{2em} Ξέρξης οὐ πρότερον ἐπεχείρησεν \\
\tabto{3em} πρὶν ἔλαβεν, \\
\tabto{4em} λαβὼν δὲ διέβη, \\
ὥστε καὶ οὗτος \\
\tabto{2em} ἐὰν λάβῃ, \\
\tabto{3em} διαβήσεται, \\
διὸ οὐκ ἐπιτρεπτέον.\\

\end{greek}
}

\begin{description}[noitemsep]
\item[ἔστιν] εἰμί biti; 3. l. sg. ind. prez.; §~315, bilj. 2, 3; kopula je dio imenskog predikata
\item[δὲ] a\dots; čestica povezuje rečenicu s prethodnom kao suprotni veznik §~515
\item[τὸ μὲν πράγματα λέγειν] μὲν je resumptivno, podsjeća o čemu se govori (i najavljuje novu informaciju)
\item[τὸ\dots\  λέγειν] λέγω govoriti, navoditi; inf. prez. akt.; supstantiviranje članom §~373
\item[πράγματα] tj. προγεγενημένα
\item[τοιόνδε τι] §~213.4; §~217
\item[εἴ τις] enklitika iza proklitike §~40.e7
\item[εἴ τις λέγοι] §~217; λέγω govoriti, navoditi; 3. l. sg. opt. prez. akt.; εἰ s optativom §~476.2
\item[ὅτι] uvodi izričnu rečenicu §~467
\item[δεῖ\dots\  παρασκευάζεσθαι] \textit{bezlično} δεῖ treba; 3. l. sg. ind. prez. akt.; παρασκευάζω spremati; inf. prez. medpas.; bezlični izraz otvara mjesto dopuni u infinitivu §~492
\item[πρὸς βασιλέα] πρὸς + a. §~435.C; §~175
\item[μὴ ἐᾶν] ἐάω dopuštati; inf. prez. akt.; negacija μή, §~509; zanijekani infinitiv također ovisi o δεῖ
\item[Αἴγυπτον] §~82
\item[χειρώσασθαι] χειρόομαι u svoje ruke dobiti, pokoriti; inf. aor. med.; za potpunu konstrukciju A+I uz ἐᾶν zamisli izostavljeno αὐτὸν (tj.\ βασιλέα)
\item[καὶ γὰρ] naime\dots; §~517; ova kombinacija čestica uvodi objašnjenje
\item[πρότερον] §~203; prilog
\item[Δαρεῖος] §~82
\item[διέβη] sc.\ iz Azije u Grčku; διαβαίνω prijeći; 3. l. sg. ind. aor.
\item[πρὶν\dots\ ἔλαβεν] vremenska rečenica s veznikom πρὶν uz negativnu glavnu rečenicu §~488 bilj. 1 
\item[ἔλαβεν] λαμβάνω uzeti, osvojiti; 3. l. sg. ind. aor. akt.
\item[λαβὼν] λαμβάνω uzeti, osvojiti; n. sg. m. r. ptc. aor. akt.
\item[δὲ] a\dots; čestica povezuje surečenicu s prethodnom (Δαρεῖος οὐ πρότερον\dots) kao suprotni veznik §~515
\item[Ξέρξης] §~100
\item[ἐπεχείρησεν] sc.\ nas (Grke); ἐπιχειρέω napasti; 3. l. sg. ind. aor. akt.
\item[πρὶν ἔλαβεν] sc.\ Αἴγυπτον
\item[ὥστε] tako da\dots; otvara mjesto posljedičnoj rečenici
\item[οὗτος] §~213.2
\item[ἐὰν λάβῃ, διαβήσεται] eventualne futurske pogodbene rečenice §~476
\item[ἐὰν ] = εἰ ἄν; veznik uvodi protazu zavisne pogodbene rečenice
\item[λάβῃ] λαμβάνω uzeti, osvojiti; 3. l. sg. konj. aor. akt.
\item[διαβήσεται] διαβαίνω prijeći; 3. l. sg. ind. fut. med.
\item[διὸ]  = δι᾿ ὅ, διὰ τοῦτο; zato, zbog toga
\item[ἐπιτρεπτέον] ἐπιτρέπω dopuštati; nom sg. sr. r. gl. pridjeva §~300; izostavljena kopula ἐστί
\end{description}

%3

{\large
\begin{greek}
\noindent παραβολὴ δὲ τὰ Σωκρατικά, \\
οἷον εἴ τις λέγοι \\
\tabto{2em} ὅτι οὐ δεῖ \underline{κληρωτοὺς ἄρχειν}· \\
ὅμοιον γὰρ ὥσπερ \\
\tabto{2em} ἂν εἴ τις τοὺς ἀθλητὰς κληροίη \\
\tabto{3em} μὴ οἳ δύνανται ἀγωνίζεσθαι \\
\tabto{3em} ἀλλ' οἳ ἂν λάχωσιν, \\
ἢ τῶν πλωτήρων \\
\tabto{2em} ὅντινα δεῖ κυβερνᾶν κληρώσειεν, \\
\tabto{3em} ὡς δέον τὸν λαχόντα \\
\tabto{2em} ἀλλὰ μὴ τὸν ἐπιστάμενον. \\

\end{greek}
}

\begin{description}[noitemsep]
\item[παραβολὴ] §~80; izostavljena kopula ἐστί; imenski predikat Smyth 909
\item[τὰ Σωκρατικά] Sokratove izreke, misli; §~103; supstantiviranje članom §~373
\item[παραβολὴ δὲ τὰ Σωκρατικά]  za usporedbu su primjer Sokratove izreke
\item[δὲ] a\dots; čestica povezuje rečenicu s prethodnom kao suprotni veznik §~515
\item[κληρωτοὺς] §~103
\item[ἄρχειν] ἄρχω vladati; inf. prez. akt.
\item[ὅμοιον] §~103; izostavljena kopula ἐστί; imenski predikat Smyth 909
\item[ὥσπερ ἂν εἴ] kao kad bi\dots; uvodi poredbenu rečenicu; protaza jednoga oblika + apodoza drugoga oblika §~479
\item[τις] §~217
\item[τοὺς ἀθλητὰς] §~100
\item[κληροίη] κληρόω ždrijebom birati; 3. l. sg. opt. prez. akt.
\item[οἳ] §~215
\item[δύνανται] δύναμαι moći; 3. l. pl. ind. prez. medpas.
\item[ἀγωνίζεσθαι] ἀγωνίζομαι natjecati se; inf. prez. medpas.
\item[ἀλλ'] §~68
\item[λάχωσιν] λαγχάνω ždrijebom odabran biti; 3. l. pl. konj. aor. akt.
\item[τῶν πλωτήρων] §~146; partitivni genitiv §~395
\item[ὅντινα] §~217; 218
\item[κυβερνᾶν] κυβερνάω kormilar biti; inf. prez. akt.
\item[κληρώσειεν] κληρόω ždrijebom birati; 3. l. sg. opt. aor. akt.
\item[ὡς δέον] kao da bi trebao
\item[τὸν λαχόντα] λαγχάνω ždrijebom odabran biti; a. sg. m. r. ptc. aor. akt.
\item[τὸν ἐπιστάμενον] ἐπίσταμαι znati; a. sg. m. r. ptc. prez. medpas.; uz oba participa podrazumijeva se κυβερνᾶν
\end{description}

{\large
\begin{greek}
\noindent λόγος δέ, \\
\tabto{2em} οἷος ὁ Στησιχόρου περὶ Φαλάριδος \\
\tabto{2em} καὶ $\langle$ὁ$\rangle$ Αἰσώπου ὑπὲρ τοῦ δημαγωγοῦ.\\

\end{greek}
}

\begin{description}[noitemsep]
\item[λόγος] \textit{ovdje} basna; §~82
\item[δὲ] a\dots; čestica δέ povezuje rečenicu s prethodnom kao suprotni veznik §~515
\item[οἷος] §~103; izostavljena kopula ἐστί; imenski predikat Smyth 909
\item[ὁ Στησιχόρου] sc.\ λόγος; §~82
\item[περὶ Φαλάριδος] §~123; περὶ + g. §~433.A
\item[$\langle$ὁ$\rangle$ Αἰσώπου] sc.\ λόγος; §~82; prelomljene zagrade označavaju priređivačev dodatak – nešto čega u rukopisnoj predaji teksta nema, a trebalo je ondje stajati
\item[ὑπὲρ τοῦ δημαγωγοῦ] §~82; ὑπὲρ + g. §~431.A
\end{description}



%kraj
