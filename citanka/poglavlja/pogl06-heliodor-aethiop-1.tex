% unesi korekture NČ 2019-09-06
\section*{O autoru}

Heliodor \textgreek[variant=ancient]{(Ἡλιόδωρος),} grčki romanopisac (Emesa, danas Homs u Siriji, III. ili IV. st.), autor je najopsežnijega grčkog ljubavnoga romana \textit{Etiopske priče o Teagenu i Harikleji} \textgreek[variant=ancient]{(Αἰϑιοπικὰ τὰ περὶ Θεαγένην καὶ Χαρίκλειαν)} u deset knjiga. 

Djelo otkriva temeljito poznavanje žanrovske tradicije (motivima otmice, gusarskih prepada, lažne smrti, napasnih snubitelja), ali i znatnu inovativnost u pripovjednom oblikovanju (izokretanje fabularnog slijeda događaja pripovijedanjem u \textit{flash-backu}, pripovjedači s ograničenim znanjem, česte promjene tempa). 

Priča o djevojci koju je majka, tamnoputa etiopska kraljica, izložila zato što se rodila bijela, imala je snažan odjek i u bizantskoj književnosti (Teodor Prodrom, Niketa Eugenijan) i na Zapadu (Cervantes, Calderón, Tasso, Racine, Shakespeare), gdje je utjecala na barokni roman. Vjerojatno posljednji odjek bogatog nasljeđa Verdijeva je opera \textit{Aida}.

\section*{O tekstu}

U ranu zoru blizu ušća Nila u more egipatski razbojnici nailaze na neobičan prizor: teško natovaren brod vezan je uz obalu, ali bez mornara; svuda uokolo po obali leže mrtvaci i umirući. Netom je završila neobična bitka, u kojoj je oružje bio pribor za gozbu, no nije jasno gdje su pobjednici, kao što nije jasno ni zašto poraženi i njihov brod nisu opljačkani. Na stijeni pored broda, međutim, razbojnici opažaju nešto još neobičnije. Riječ je o prekrasnoj djevojci u punoj ratnoj opremi; u njezinu je krilu ranjeni mladić.

%\newpage

\section*{Pročitajte naglas grčki tekst.}

Heliod.\ Aeth.\ 1.2

%Naslov prema izdanju

\medskip

\begin{greek}
{\large
{ \noindent Κόρη καθῆστο ἐπὶ πέτρας, ἀμήχανόν τι κάλλος καὶ θεὸς εἶναι ἀναπείθουσα, τοῖς μὲν παροῦσι περιαλγοῦσα φρονήματος δὲ εὐγενοῦς ἔτι πνέουσα.  Δάφνῃ τὴν κεφαλὴν ἔστεπτο καὶ φαρέτραν τῶν ὤμων ἐξῆπτο καὶ τῷ λαιῷ βραχίονι τὸ τόξον ὑπεστήρικτο· ἡ λοιπὴ δὲ χεὶρ ἀφροντίστως ἀπῃώρητο. Μηρῷ δὲ τῷ δεξιῷ τὸν ἀγκῶνα θατέρας χειρὸς ἐφεδράζουσα καὶ τοῖς δακτύλοις τὴν παρειὰν ἐπιτρέψασα, κάτω νεύουσα καί τινα προκείμενον ἔφηβον περισκοποῦσα τὴν κεφαλὴν ἀνεῖχεν.  

Ὁ δὲ τραύμασι μὲν κατῄκιστο καὶ μικρὸν ἀναφέρειν ὥσπερ ἐκ βαθέος ὕπνου τοῦ παρ' ὀλίγον θανάτου κατεφαίνετο, ἤνθει δὲ καὶ ἐν τούτοις ἀνδρείῳ τῷ κάλλει καὶ ἡ παρειὰ καταρρέοντι τῷ αἵματι φοινιττομένη λευκότητι πλέον ἀντέλαμπεν. Ὀφθαλμοὺς δὲ ἐκείνου οἱ μὲν πόνοι κατέσπων, ἡ δὲ ὄψις τῆς κόρης ἐφ' ἑαυτὴν ἀνεῖλκε καὶ τοῦτο ὁρᾶν αὐτοὺς ἠνάγκαζεν, ὅτι ἐκείνην ἑώρων.  

Ὡς δὲ πνεῦμα συλλεξάμενος καὶ βύθιόν τι ἀσθμήνας λεπτὸν ὑπεφθέγξατο καὶ ``ὦ γλυκεῖα,'' ἔφη ``σῴζῃ μοι ὡς ἀληθῶς, ἢ γέγονας καὶ αὐτὴ τοῦ πολέμου πάρεργον, οὐκ ἀνέχῃ δὲ ἄλλως οὐδὲ μετὰ θάνατον ἀποστατεῖν ἡμῶν, ἀλλὰ φάσμα τὸ σὸν καὶ ψυχὴ τὰς ἐμὰς περιέπει τύχας;'', ``ἐν σοὶ'' ἔφη ``τὰ ἐμὰ'' ἡ κόρη ``σῴζεσθαί τε καὶ μή· τοῦτο γοῦν ὁρᾷς;'' δείξασα ἐπὶ τῶν γονάτων ξίφος, ``εἰς δεῦρο ἤργησεν ὑπὸ τῆς σῆς ἀναπνοῆς ἐπεχόμενον.''

}
}
\end{greek}

\section*{Analiza i komentar}

%1

{\large
\begin{greek}
\noindent Κόρη καθῆστο ἐπὶ πέτρας, \\
ἀμήχανόν τι κάλλος καὶ θεὸς εἶναι ἀναπείθουσα, \\
τοῖς μὲν παροῦσι περιαλγοῦσα \\
φρονήματος δὲ εὐγενοῦς ἔτι πνέουσα.\\

\end{greek}
}

\begin{description}[noitemsep]
\item[Κόρη] §~90
\item[καθῆστο] κάθημαι sjediti; jon. 3. l. sg. impf. medpas.
\item[ἐπὶ πέτρας] §~418; §~436.C; §~90
\item[ἀμήχανόν ] §~103; 106
\item[τι] §~217
\item[κάλλος] §~153
\item[θεὸς] §~82
\item[εἶναι] εἰμί biti, inf. prez. akt.
\item[ἀναπείθουσα] ἀναπείθω uvjeravati, n. sg. ž. r. ptc. prez. akt.
\item[τοῖς μὲν\dots\ φρονήματος δὲ\dots] koordinacija dvaju članaka s pomoću para čestica
\item[τοῖς παροῦσι] πάρειμι \textit{ovdje} događati se; d. pl. m. r. ptc. prez. akt.
\item[περιαλγοῦσα] περιαλγέω τινι osjećati bol zbog čega, patiti zbog čega; n. sg. ž. r. ptc. prez. akt.
\item[φρονήματος] §~123
\item[εὐγενοῦς] §~153
\item[πνέουσα] πνέω τινος puhati, \textit{ovdje} odisati čime; n. sg. ž. r. ptc. prez. akt.

\end{description}

%2

{\large
\begin{greek}
\noindent Δάφνῃ τὴν κεφαλὴν ἔστεπτο \\
καὶ φαρέτραν τῶν ὤμων ἐξῆπτο \\
καὶ τῷ λαιῷ βραχίονι τὸ τόξον ὑπεστήρικτο·\\
ἡ λοιπὴ δὲ χεὶρ \\
\tabto{2em} ἀφροντίστως ἀπῃώρητο.\\

\end{greek}
}

\begin{description}[noitemsep]
\item[Δάφνῃ] §~90
\item[τὴν κεφαλὴν] §~90
\item[ἔστεπτο] στέφω τί τινι ovjenčati što nečime; 3. l. sg. plpf. medpas.
\item[φαρέτραν] §~90
\item[τῶν ὤμων] §~82
\item[ἐξῆπτο] ἐξάπτω τινος objesiti o što; 3. l. sg. plpf. medpas.
\item[τῷ λαιῷ βραχίονι] §~103; §~131
\item[τὸ τόξον] §~82
\item[ὑπεστήρικτο] ὑποστηρίζω podržavati, držati ispod čega; 3. l. sg. plpf. medpas.
\item[δὲ] upozorava da je riječ o drugom dijelu para (jer djevojka ima dvije ruke)
\item[ἡ λοιπὴ χεὶρ] §~103; §~146; §~150
\item[ἀφροντίστως ] §~204
\item[ἀπῃώρητο] ἀπαιωρέομαι visjeti s čega; 3. l. sg. plpf. medpas.

\end{description}

%3

{\large
\begin{greek}
\noindent Μηρῷ δὲ τῷ δεξιῷ \\
\tabto{2em} τὸν ἀγκῶνα θατέρας χειρὸς ἐφεδράζουσα \\
καὶ τοῖς δακτύλοις \\
\tabto{2em} τὴν παρειὰν ἐπιτρέψασα, \\
κάτω νεύουσα \\
καί τινα προκείμενον ἔφηβον περισκοποῦσα \\
τὴν κεφαλὴν ἀνεῖχεν.\\

\end{greek}
}

\begin{description}[noitemsep]
\item[Μηρῷ τῷ δεξιῷ] §~82; §~103
\item[δὲ] u službi suprotnog veznika, povezuje s prethodnom rečenicom
\item[τὸν ἀγκῶνα] §~131
\item[θατέρας] §~103
\item[χειρὸς] §~146; §~150
\item[ἐφεδράζουσα] ἐφεδράζω τί τινι postaviti na što; n. sg. ž. r. ptc. prez. akt.
\item[τοῖς δακτύλοις] §~82
\item[τὴν παρειὰν] §~90
\item[ἐπιτρέψασα] ἐπιτρέπω τινί τι okrenuti čime što; n. sg. ž. r. ptc. aor. akt.
\item[νεύουσα] νεύω nagnuti, pognuti;  n. sg. ž. r. ptc. prez. akt.
\item[καί τινα] §~40
\item[τινα] §~217
\item[προκείμενον] πρόκειμαι ležati ispred, ležati izložen; a. sg. m. r. ptc. prez. medpas.
\item[ἔφηβον] §~82
\item[περισκοποῦσα] περισκοπέω promatrati; n. sg. ž. r. ptc. prez. akt.
\item[τὴν κεφαλὴν] §~90
\item[ἀνεῖχεν] ἀνέχω dizati; 3. l. sg. impf. akt.

\end{description}

%4
{\large
\begin{greek}
\noindent Ὁ δὲ τραύμασι μὲν κατῄκιστο \\
καὶ μικρὸν ἀναφέρειν \\
\tabto{2em} ὥσπερ ἐκ βαθέος ὕπνου \\
\tabto{2em} τοῦ παρ' ὀλίγον θανάτου \\
κατεφαίνετο, \\
ἤνθει δὲ καὶ ἐν τούτοις \\
\tabto{2em} ἀνδρείῳ τῷ κάλλει \\
καὶ ἡ παρειὰ \\
\tabto{2em} καταρρέοντι τῷ αἵματι φοινιττομένη \\
\tabto{2em} λευκότητι πλέον ἀντέλαμπεν.\\

\end{greek}
}

\begin{description}[noitemsep]
\item[Ὁ] §~370.2
\item[τραύμασι] §~123
\item[κατῄκιστο] καταικίζω izranjavati; 3. l. sg. plpf. medpas.
\item[μικρὸν] §~204
\item[ἀναφέρειν] ἀναφέρω \textit{ovdje} oporavljati se; inf. prez. akt.
\item[ἐκ βαθέος ὕπνου] §~167; §~82
\item[τοῦ θανάτου] §~82
\item[παρ' ὀλίγον] §~68; §~103; §~434
\item[κατεφαίνετο] καταφαίνω pas.\ činiti se, izgledati; 3. l. sg. impf. medpas.
\item[ἤνθει] ἀνθέω cvasti, \textit{preneseno} odlikovati se; 3. l. sg. impf. akt.
\item[δὲ] u službi suprotnog veznika, povezuje s prethodnom surečenicom
\item[ἐν τούτοις] §~426; §~213.2
\item[ἀνδρείῳ τῷ κάλλει] §~103; §~153
\item[ἡ παρειὰ] §~90
\item[καταρρέοντι] καταρρέω kapati, teći; d. sg. s. r. ptc. prez. akt.
\item[τῷ αἵματι] §~123
\item[φοινιττομένη] φοινίσσω, atički φοινίττω crvenjeti se; n. sg. ž. r. ptc. prez. medpas.; suprotnost \textgreek[variant=ancient]{φοινίσσω – λευκότης} ukazuje na dopusno značenje participa
\item[λευκότητι] §~123
\item[πλέον] §~202; §~204
\item[ἀντέλαμπεν] ἀντιλάμπω τινί sjajiti, blistati čime; 3. l. sg. impf. akt.

\end{description}
%5

{\large
\begin{greek}
\noindent Ὀφθαλμοὺς δὲ ἐκείνου \\
\tabto{2em} οἱ μὲν πόνοι κατέσπων, \\
\tabto{2em} ἡ δὲ ὄψις τῆς κόρης \\
\tabto{4em} ἐφ' ἑαυτὴν ἀνεῖλκε \\
\tabto{2em} καὶ τοῦτο ὁρᾶν αὐτοὺς ἠνάγκαζεν, \\
\tabto{4em} ὅτι ἐκείνην ἑώρων.\\

\end{greek}
}

\begin{description}[noitemsep]
\item[Ὀφθαλμοὺς ] §~82
\item[δὲ] u službi suprotnog veznika, povezuje s prethodnom rečenicom
\item[ἐκείνου] §~213.3
\item[οἱ πόνοι] §~82
\item[κατέσπων] κατασπάω vući dolje, u kolokaciji ὀφθαλμούς zatvarati; 3. l. pl. impf. akt.
\item[ἡ ὄψις] §~165
\item[τῆς κόρης] §~90
\item[ἐφ' ἑαυτὴν] §~68; §~208; §~436
\item[ἀνεῖλκε] ἀνέλκω vući gore, privlačiti; 3. l. sg. impf. akt.
\item[τοῦτο] §~213.2
\item[ὁρᾶν] ὁράω gledati; inf. prez. akt.
\item[αὐτοὺς] §~207
\item[ἠνάγκαζεν] ἀναγκάζω primoravati, siliti; 3. l. sg. impf. akt.
\item[ὅτι] §~468; veznik otvara mjesto uzročnoj rečenici
\item[ἐκείνην] §~213.3
\item[ἑώρων] ὁράω gledati; 3. l. pl. impf. akt., temporalni augment §~237

\end{description}
%6

{\large
\begin{greek}
\noindent Ὡς δὲ πνεῦμα συλλεξάμενος \\
καὶ βύθιόν τι ἀσθμήνας \\
λεπτὸν ὑπεφθέγξατο \\
καὶ ``ὦ γλυκεῖα,'' ἔφη \\
\tabto{2em} ``σῴζῃ μοι ὡς ἀληθῶς, \\
\tabto{2em} ἢ γέγονας καὶ αὐτὴ \\
\tabto{4em} τοῦ πολέμου πάρεργον, \\
οὐκ ἀνέχῃ δὲ ἄλλως \\
\tabto{2em} οὐδὲ μετὰ θάνατον \\
\tabto{2em} ἀποστατεῖν ἡμῶν, \\
ἀλλὰ φάσμα τὸ σὸν καὶ ψυχὴ \\
\tabto{2em} τὰς ἐμὰς περιέπει τύχας;'',\\
``ἐν σοὶ'' ἔφη ``τὰ ἐμὰ'' ἡ κόρη \\
\tabto{2em} ``σῴζεσθαί τε καὶ μή· \\
τοῦτο γοῦν ὁρᾷς;'' \\
\tabto{2em} δείξασα ἐπὶ τῶν γονάτων ξίφος,\\
``εἰς δεῦρο ἤργησεν \\
\tabto{2em} ὑπὸ τῆς σῆς ἀναπνοῆς ἐπεχόμενον.''

\end{greek}
}

\begin{description}[noitemsep]
\item[Ὡς ] §~487
\item[δὲ] u službi suprotnog veznika, povezuje s prethodnom rečenicom
\item[πνεῦμα] §~123
\item[συλλεξάμενος] συλλέγω skupiti; n. sg. m. r. ptc. aor. med.
\item[βύθιόν τι] §~39; §~40
\item[βύθιόν] §~103
\item[τι] §~217
\item[ἀσθμήνας] ἀσθμαίνω stenjati, hroptati; n. sg. m. r. ptc. aor. akt.
\item[λεπτὸν] §~204; oblik upotrijebljen priložno 
\item[ὑπεφθέγξατο] ὑποφθέγγομαι govoriti ispod glasa; 3. l. sg. ind. aor. med.
\item[ὦ γλυκεῖα] §~170
\item[ἔφη] φημί govoriti, reći; 3. l. sg. impf. akt.
\item[σῴζῃ μοι] §~39; §~40
\item[σῴζῃ] σῴζω spasiti; 2. l. sg. ind. prez. medpas.
\item[μοι] §~205
\item[ὡς ἀληθῶς] fraza, vidi LSJ ἀληθής III.b
\item[γέγονας] γίγνομαι postati, nastati (kopulativan glagol nepotpuna značenja, dio imenskog predikata, Smyth 909); 2. l. sg. ind. perf. akt.
\item[αὐτὴ] §~207
\item[τοῦ πολέμου] §~82
\item[πάρεργον] §~82
\item[ἀνέχῃ] ἀνέχω pas. prestati; 2. l. sg. ind. prez. medpas.
\item[ἄλλως] §~204
\item[μετὰ θάνατον] §~82
\item[ἀποστατεῖν] ἀποστατέω τινός biti udaljen, odsutan od koga; inf. prez. akt.
\item[ἡμῶν] §~205
\item[φάσμα τὸ σὸν] §~123; §~210
\item[ψυχὴ] §~90
\item[τὰς ἐμὰς τύχας] §~90; §~210
\item[περιέπει] περιέπω mučiti, maltretirati; 3. l. sg. ind. prez. akt.
\item[ἐν σοὶ] §~426; §~205
\item[ἔφη] φημί govoriti; 3. l. sg. impf. akt.
\item[τὰ ἐμὰ] §~210; poimeničenje članom §~373
\item[ἡ κόρη] §~90
\item[σῴζεσθαί] σῴζω spasiti; inf. prez. medpas.
\item[τοῦτο] §~213.2
\item[ὁρᾷς] ὁράω gledati; 2. l. ind. prez. akt.
\item[δείξασα] δείκνυμι pokazati; n. sg. ž. r. ptc. aor. akt.
\item[ἐπὶ τῶν γονάτων] §~128
\item[ξίφος] §~153
\item[εἰς δεῦρο] §~419
\item[ἤργησεν] ἀργέω biti nezaposlen, mirovati; 3. l. sg. ind. aor. akt.
\item[ὑπὸ τῆς σῆς ἀναπνοῆς] §~90; §~210; §~437
\item[ἐπεχόμενον] ἐπέχω zaustaviti, zadržati; n. sg. s. r. ptc. prez. medpas.

\end{description}


%kraj
