% Unesi korekture NČ, NZ 2019-09-23
%\section*{O autoru}



\section*{O tekstu}

Ova Ezopova basna pripovijeda o satiru koji je ostao zatečen vidjevši svojeg prijatelja čovjeka kako zbog velike hladnoće puše u prste da ih ugrije, ali odmah potom puše i u toplo jelo da ga rashladi. Kako mu je bilo nepojmljivo da iz istih usta izlaze i toplina i hladnoća, satir je naprasno prekinuo prijateljstvo s takvim čovjekom. Moralna pouka, koju je Erazmo Roterdamski u svojoj zbirci \textit{Adagia} sažeo u izreci \textit{Ex eodem ore calidum et frigidum efflare}, sugerirala je da se treba kloniti dvoličnih ljudi, onih koji jedno govore, a drugo misle te im se zbog toga ne može vjerovati. Kasnije, u doba prosvjetiteljstva, basnu su tumačili mislioci poput Voltairea i Lessinga upozoravajući na u njoj prisutnu nelogičnost: čovjek je doista, pušući u ruke i u jelo, činio ono što je u danom trenutku trebao činiti, koliko god to djelovalo proturječno. Zbog pomalo nategnute poruke pomišljalo se i na mogućnost da je basna naknadno smišljena kako bi dala okvir već postojećoj poslovici.

%\newpage

\section*{Pročitajte naglas grčki tekst.}
Aesop.\ Fabulae 35
%Naslov prema izdanju

\medskip

{\large
\begin{greek}
\noindent ΑΝΘΡΩΠΟΣ ΚΑΙ ΣΑΤΥΡΟΣ 

\noindent Ἄνθρωπόν ποτε λέγεται πρὸς σάτυρον φιλίαν σπείσασθαι. καὶ δὴ χειμῶνος καταλαβόντος καὶ ψύχους γενομένου ὁ ἄνθρωπος προσφέρων τὰς χεῖρας τῷ στόματι ἐπέπνει. τοῦ δὲ σατύρου τὴν αἰτίαν ἐρομένου δι' ἣν τοῦτο πράττει, ἔλεγεν, ὅτι θερμαίνει τὰς χεῖρας διὰ τὸ κρύος. ὕστερον δὲ παρατεθείσης αὐτοῖς τραπέζης καὶ προσφαγήματος θερμοῦ σφόδρα ὄντος ὁ ἄνθρωπος ἀναιρούμενος κατὰ μικρὸν τῷ στόματι προσέφερε καὶ ἐφύσα. πυνθανομένου δὲ πάλιν τοῦ σατύρου, τί τοῦτο ποιεῖ, ἔφασκε καταψύχειν τὸ ἔδεσμα, ἐπεὶ λίαν θερμόν ἐστι. κἀκεῖνος ἔφη πρὸς αὐτόν· ``ἀλλ' ἀποτάσσομαί σου τῇ φιλίᾳ, ὦ οὗτος, ὅτι ἐκ τοῦ αὐτοῦ στόματος τὸ θερμὸν καὶ τὸ ψυχρὸν ἐξιεῖς.''

ἀτὰρ οὖν καὶ ἡμᾶς περιφεύγειν δεῖ τὴν φιλίαν, ὧν ἀμφίβολός ἐστιν ἡ διάθεσις.

\end{greek}

}

\section*{Analiza i komentar}


%1

{\large
\begin{greek}
\noindent \underline{Ἄνθρωπόν} ποτε λέγεται \\
\tabto{2em} πρὸς σάτυρον \\
φιλίαν \underline{σπείσασθαι}.\\

\end{greek}
}

\begin{description}[noitemsep]
\item[Ἄνθρωπόν] §~82
\item[ποτε] jednom (vremenski prilog); prebacivanje naglaska s enklitike na prethodnu riječ §~39 i §~40
\item[λέγεται] λέγω reći, govoriti; 3. l. sg. ind. prez. medpas.; bezlično λέγεται može uvesti konstrukciju A+I ili N+I
\item[πρὸς σάτυρον ] §~82; πρός s, sa §~435.4.C.c
\item[φιλίαν] §~90
\item[σπείσασθαι] σπένδω dogovoriti, sklopiti; inf. aor. medpas.

\end{description}

{\large
\begin{greek}
\noindent καὶ δὴ \uuline{χειμῶνος καταλαβόντος} καὶ \uuline{ψύχους γενομένου} \\
ὁ ἄνθρωπος \\
\tabto{2em} προσφέρων \\
\tabto{4em} τὰς χεῖρας τῷ στόματι \\
ἐπέπνει.\\

\end{greek}
}

\begin{description}[noitemsep]
\item[χειμῶνος] §~131; GA ima vrijednost vremenske rečenice
\item[καταλαβόντος] καταλαμβάνω stići, nastati; g. sg. m. r. ptc. aor. akt.
\item[ψύχους] §~153
\item[γενομένου] γίγνομαι postati, biti; g. sg. m. r. ptc. aor. med.
\item[ὁ ἄνθρωπος] §~82
\item[προσφέρων] προσφέρω primaknuti; n. sg. m. r. ptc. prez. akt.
\item[τὰς χεῖρας] §~146
\item[τῷ στόματι] §~123
\item[ἐπέπνει] ἐπιπνέω puhnuti; 3. l. sg. impf. akt.

\end{description}
%3 itd
{\large
\begin{greek}
\noindent \uuline{τοῦ δὲ σατύρου τὴν αἰτίαν ἐρομένου} \\
\tabto{2em} δι' ἣν τοῦτο πράττει, \\
ἔλεγεν, \\
\tabto{2em} ὅτι θερμαίνει τὰς χεῖρας \\
\tabto{4em} διὰ τὸ κρύος.\\

\end{greek}
}

\begin{description}[noitemsep]
\item[τοῦ\dots\ σατύρου] §~82; GA ima vrijednost vremenske rečenice
\item[τὴν αἰτίαν] §~90
\item[ἐρομένου] εἴρομαι pitati; g. sg. m. r. ptc. aor. med.
\item[δι' ἣν] §~428.I.B; §~215; ἥν je ovisno o αἰτίαν
\item[τοῦτο] §~213
\item[πράττει] πράττω činiti, raditi; 3. l. sg. ind. prez. akt.
\item[ἔλεγεν] λέγω reći, govoriti: 3. l. sg. impf. akt.
\item[ὅτι] izrični veznik uz glagol govorenja §~518
\item[θερμαίνει] θερμαίνω grijati; 3. l. sg. ind. prez. akt.
\item[τὰς χεῖρας] §~146
\item[διὰ τὸ κρύος] §~153

\end{description}

%4

{\large
\begin{greek}
\noindent ὕστερον δὲ \\
\tabto{2em} \uuline{παρατεθείσης αὐτοῖς τραπέζης} \\
\tabto{2em} καὶ \uuline{προσφαγήματος θερμοῦ σφόδρα ὄντος} \\
ὁ ἄνθρωπος \\
\tabto{2em} ἀναιρούμενος κατὰ μικρὸν \\
\tabto{2em} τῷ στόματι \\
προσέφερε \\
καὶ ἐφύσα.\\

\end{greek}
}

\begin{description}[noitemsep]
\item[ὕστερον] poslije (vremenski prilog)
\item[παρατεθείσης] παρατίθημι postaviti; g. sg. ž. r. ptc. aor. pas.; GA ima vrijednost vremenske rečenice
\item[αὐτοῖς] §~207
\item[τραπέζης] §~90
\item[προσφαγήματος] §~123
\item[θερμοῦ] §~103
\item[σφόδρα] vrlo, jako
\item[ὄντος] εἰμί biti; g. sg. s. r. ptc. prez. akt.; prevedite prošlim vremenom
\item[ὁ ἄνθρωπος] §~82
\item[ἀναιρούμενος] ἀναιρέω podići; n. sg. m. r. ptc. prez. medpas.
\item[κατὰ μικρὸν] malo (priložna oznaka)
\item[τῷ στόματι] §~123
\item[προσέφερε] προσφέρω primaknuti; 3. l. sg. impf. akt.
\item[ἐφύσα] φυσάω puhnuti; 3. l. sg. impf. akt.

\end{description}

%5

{\large
\begin{greek}
\noindent \uuline{πυνθανομένου δὲ πάλιν τοῦ σατύρου}, \\
\tabto{2em} τί τοῦτο ποιεῖ, \\
ἔφασκε \\
\tabto{2em} \underline{καταψύχειν} τὸ ἔδεσμα, \\
\tabto{4em} ἐπεὶ λίαν θερμόν ἐστι.\\

\end{greek}
}

\begin{description}[noitemsep]
\item[πυνθανομένου] πυνθάνομαι pitati; g. sg. m. r. ptc. prez. medpas.; GA ima vrijednost vremenske rečenice
\item[τοῦ σατύρου] §~82
\item[τί] §~217
\item[τοῦτο] §~213
\item[ποιεῖ] ποιέω činiti; 3. l. sg. ind. prez. akt.
\item[ἔφασκε ] φάσκω (iterativni oblik glagola φημί) reći; 3. l. sg. impf. akt. 
\item[καταψύχειν] καταψύχω hladiti; inf. prez. akt.
\item[τὸ ἔδεσμα] §~123
\item[ἐπεὶ] veznik uvodi uzročnu rečenicu §~468
\item[θερμόν] §~103
\item[ἐστι] εἰμί biti; 3. l. sg. ind. prez.; prebacivanje naglaska s enklitike na prethodnu riječ §~39 i §~40

\end{description}

%6

{\large
\begin{greek}
\noindent κἀκεῖνος ἔφη \\
\tabto{2em} πρὸς αὐτόν· \\
``ἀλλ' ἀποτάσσομαί \\
\tabto{2em} σου τῇ φιλίᾳ, \\
ὦ οὗτος, \\
\tabto{2em} ὅτι \\
\tabto{4em} ἐκ τοῦ αὐτοῦ στόματος \\
\tabto{2em} τὸ θερμὸν καὶ τὸ ψυχρὸν \\
\tabto{2em} ἐξιεῖς.''\\

\end{greek}
}

\begin{description}[noitemsep]
\item[κἀκεῖνος] = καὶ ἐκεῖνος §~213; kraza §~66
\item[ἔφη] φημί reći; 3. l. sg. impf. akt.
\item[πρὸς αὐτόν] §~207
\item[ἀλλά] ali
\item[ἀποτάσσομαί] ἀποτάσσω τινί med. odreći se čega; 1. l. sg. ind. prez. medpas. 
\item[σου] §~205; objektni genitiv §~494 („prijateljstvo s tobom''); prebacivanje naglaska s enklitike na prethodnu riječ §~39 i §~40
\item[τῇ φιλίᾳ] §~90
\item[ὦ οὗτος] §~213
\item[ὅτι] veznik uvodi uzročnu rečenicu §~468
\item[ἐκ τοῦ\dots\  στόματος] §~123
\item[αὐτοῦ] isti; §~207
\item[τὸ θερμὸν] §~103; poimeničeni pridjev, zamjenjuje imenicu
\item[τὸ ψυχρὸν] §~103; poimeničeni pridjev, zamjenjuje imenicu
\item[ἐξιεῖς] ἐξίημι izbaciti, ispustiti; 2. l. sg. ind. prez. akt. (atička dubleta uz ἐξίης)

\end{description}

%7

{\large
\begin{greek}
\noindent ἀτὰρ οὖν καὶ \underline{ἡμᾶς} \\
\tabto{2em} \underline{περιφεύγειν} \\
δεῖ \\
\tabto{2em} τὴν φιλίαν, \\
\tabto{4em} ὧν ἀμφίβολός ἐστιν \\
\tabto{4em} ἡ διάθεσις.

\end{greek}
}

\begin{description}[noitemsep]
\item[ἡμᾶς] §~205; δεῖ otvara mjesto A+I §~492
\item[περιφεύγειν] περιφεύγω τί bježati od čega; inf. prez. akt.
\item[δεῖ] treba\dots, potrebno je da\dots
\item[τὴν φιλίαν] §~90
\item[ὧν] §~215
\item[ἀμφίβολός] §~103 i §~106
\item[ἐστιν] εἰμί biti; 3. l. sg. ind. prez.; prebacivanje naglaska s enklitike na prethodnu riječ §~39 i §~40
\item[ἡ διάθεσις] §~165

\end{description}

%kraj
