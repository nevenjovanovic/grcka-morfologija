% redaktura NZ (ne slažem se s autou gen. part.), mala red. NJ

\section*{O autoru}

Lizija (Λυσίας, Atena, oko 440.\ pr.~Kr.\ – oko 378.\ pr.~Kr.), govornik i pisac sudbenih govora \textgreek[variant=ancient]{(λογογράφος),} autor preko dvjesto takvih djela – u antici mu ih je pripisivano 425 – od kojih je sačuvano 31. Sin bogatog trgovca i poduzetnika iz Sirakuze, govorničku karijeru počeo je tek nakon pada tridesetorice tirana, 403. pr.~Kr.; govorom \textit{Protiv Eratostena}, \textgreek[variant=ancient]{Κατὰ Ἐρατοσϑένους,} jedinim koji je sam održao, optužio je jednog od oligarha za smrt svojeg brata. Sastavljao govore za najraznolikije parnice, od ubojstva i izdaje do preljuba i pronevjere, tako da su njegova djela bogat izvor podataka o atenskoj povijesti, pravu i običajima. Spominje ga Platon u dijalozima. Jedan je od desetorice kanonskih atičkih govornika.

\section*{O tekstu}

Agorat je bio profesionalni potkazivač čije su prijave skrivile smrt niza uglednih građana, među njima i izvjesnog Dionizodora. Nakon pada tridesetorice, Dionizodorov svak i rođak Agorata za to optužuje pred sudom 403.\ ili 402.\ pr.~Kr. Opsežno pripovijeda kako su Atenjani poraženi u Peloponeskom ratu i kako je Agorat radio za neprijatelje atenske slobode prijavljujući, između ostalih, one koji su za Peloponeskog rata bili zapovjednici – stratezi i taksijarsi (Dionizodor je bio taksijarh). Naš odlomak govori o rastanku osuđenika na smrt s rođakinjama i suprugama.

\newpage

\section*{Pročitajte naglas grčki tekst.}

%Naslov prema izdanju
Lys.\ In Agoratum 39.1

\medskip

{\large
\begin{greek}
\noindent ᾿Επειδὴ τοίνυν, ὦ ἄνδρες δικασταί, θάνατος αὐτῶν κατεγνώσθη καὶ ἔδει αὐτοὺς ἀποθνῄσκειν, μεταπέμπονται εἰς τὸ δεσμωτήριον ὁ μὲν ἀδελφήν, ὁ δὲ μητέρα, ὁ δὲ γυναῖκα, ὁ δ' ἥ τις ἦν ἑκάστῳ αὐτῶν προσήκουσα, ἵνα τὰ ὕστατα ἀσπασάμενοι τοὺς αὑτῶν οὕτω τὸν βίον τελευτήσειαν. καὶ δὴ καὶ Διονυσόδωρος μεταπέμπεται τὴν ἀδελφὴν τὴν ἐμὴν εἰς τὸ δεσμωτήριον, γυναῖκα ἑαυτοῦ οὖσαν· πυθομένη δ' ἐκείνη ἀφικνεῖται, μέλαν τε ἱμάτιον ἠμφιεσμένη \dots, ὡς εἰκὸς ἦν ἐπὶ τῷ ἀνδρὶ αὐτῆς τοιαύτῃ συμφορᾷ κεχρημένῳ. ἐναντίον δὲ τῆς ἀδελφῆς τῆς ἐμῆς Διονυσόδωρος τά τε οἰκεῖα τὰ αὑτοῦ διέθετο ὅπως αὐτῷ ἐδόκει, καὶ περὶ ᾿Αγοράτου τουτουὶ ἔλεγεν ὅτι $\langle$οἱ$\rangle$ αἴτιος ἦν τοῦ θανάτου, καὶ ἐπέσκηπτεν ἐμοὶ καὶ Διονυσίῳ τουτῳί, τῷ ἀδελφῷ τῷ αὑτοῦ, καὶ τοῖς φίλοις πᾶσι τιμωρεῖν ὑπὲρ αὑτοῦ ᾿Αγόρατον· καὶ τῇ γυναικὶ τῇ αὑτοῦ ἐπέσκηπτε, νομίζων αὐτὴν κυεῖν ἐξ αὑτοῦ, ἐὰν γένηται αὐτῇ παιδίον, φράζειν τῷ γενομένῳ ὅτι τὸν πατέρα αὐτοῦ ᾿Αγόρατος ἀπέκτεινε, καὶ κελεύειν τιμωρεῖν ὑπὲρ αὑτοῦ ὡς φονέα ὄντα.

\end{greek}

}

\section*{Analiza i komentar}


%1

{\large
\noindent ᾿Επειδὴ τοίνυν, \\
\tabto{2em} ὦ ἄνδρες δικασταί, \\
θάνατος \\
\tabto{2em} αὐτῶν \\
κατεγνώσθη \\
καὶ ἔδει \\
\tabto{2em} \underline{αὐτοὺς ἀποθνῄσκειν}, \\
μεταπέμπονται \\
\tabto{2em} εἰς τὸ δεσμωτήριον \\
ὁ μὲν ἀδελφήν, \\
ὁ δὲ μητέρα, \\
ὁ δὲ γυναῖκα, \\
ὁ δ' ἥ τις ἦν \\
\tabto{4em} ἑκάστῳ \\
\tabto{6em} αὐτῶν \\
\tabto{2em} προσήκουσα, \\
\tabto{2em} ἵνα τὰ ὕστατα ἀσπασάμενοι \\
\tabto{4em} τοὺς αὑτῶν \\
\tabto{2em} οὕτω τὸν βίον τελευτήσειαν.\\

}

\begin{description}[noitemsep]

\item[ὦ ἄνδρες δικασταί] §~149; §~100; fraza kojom se s poštovanjem oslovljavaju porotnici
\item[θάνατος] §~82
\item[αὐτῶν ] §~207
\item[κατεγνώσθη] καταγιγνώσκω θάνατόν τινος osuditi koga na smrt; 3. l. sg. ind. aor. pas.
\item[ἔδει ] δεῖ treba; (3. l. sg.) impf. (akt.); bezlični izraz otvara mjesto konstrukciji A+I
\item[αὐτοὺς ] §~207
\item[ἀποθνῄσκειν] ἀποθνῄσκω umrijeti; inf. prez. akt.
\item[μεταπέμπονται ] μεταπέμπω med. τινά poslati po koga, pozvati koga; 3. l. pl. ind. prez. medpas.
\item[εἰς τὸ δεσμωτήριον ] §~419; §~82
\item[ὁ μὲν\dots, ὁ δὲ\dots, ὁ δὲ\dots, ὁ δ'\dots] koordinacija rečeničnih članova česticama: jedan\dots\ drugi\dots\ treći\dots\ četvrti\dots; §~68
\item[ἀδελφήν] §~90
\item[μητέρα] §~148
\item[γυναῖκα] §~122
\item[δ' ἥ τις] §~68; §~40
\item[ἥ τις] §~218.5
\item[ἦν ] εἰμί biti; 3. l. sg. impf. (akt.)
\item[ἑκάστῳ αὐτῶν] §~103; §~207
\item[προσήκουσα] προσήκω pripadati, biti rođak; n. sg. ž. r. ptc. prez. akt.; poimeničeno članom \textgreek[variant=ancient]{ἡ προσήκουσα} rođakinja
\item[ἵνα\dots\ τελευτήσειαν] namjerna rečenica s optativom (u glavnoj je rečenici historijski prezent μεταπέμπονται), §~470; τελευτάω skončati; 3. l. pl. opt. aor. akt.
\item[τὰ ὕστατα ] §~373; §~203
\item[ἀσπασάμενοι ] ἀσπάζομαι pozdraviti, rastati se od; n. pl. m. r. ptc. aor. (med.)
\item[τοὺς αὑτῶν ] §~209.1; §~373
\item[τὸν βίον] §~82; individualni član zastupa posvojnu zamjenicu §~371 bilj. 1
\end{description}

{\large
\noindent καὶ δὴ καὶ \\
Διονυσόδωρος μεταπέμπεται \\
\tabto{2em} τὴν ἀδελφὴν τὴν ἐμὴν \\
εἰς τὸ δεσμωτήριον, \\
\tabto{2em} γυναῖκα ἑαυτοῦ οὖσαν· \\
πυθομένη δ' ἐκείνη ἀφικνεῖται, \\
\tabto{2em} μέλαν τε ἱμάτιον ἠμφιεσμένη \dots, \\
ὡς εἰκὸς ἦν \\
ἐπὶ τῷ ἀνδρὶ \\
\tabto{2em} αὐτῆς \\
\tabto{4em} τοιαύτῃ συμφορᾷ \\
\tabto{2em} κεχρημένῳ.\\

}

\begin{description}[noitemsep]

\item[καὶ δὴ καὶ] pa tako; kombinacija čestica označava prijelaz s općeg na posebno
\item[Διονυσόδωρος ] §~82
\item[μεταπέμπεται ] μεταπέμπω τινά poslati po koga, pozvati koga; 3. l. sg. ind. prez. medpas.
\item[τὴν ἀδελφὴν τὴν ἐμὴν ] §~90; §~210; §~375 
\item[εἰς τὸ δεσμωτήριον] §~419; §~82
\item[γυναῖκα ἑαυτοῦ οὖσαν] §~122; §~208; εἰμί biti; a. sg. ž. r. ptc. prez. (akt.)
\item[πυθομένη ] πυνθάνομαι saznati; n. sg. ž. r. ptc. aor. (med.)
\item[δ' ἐκείνη ] §~68; čestica povezuje surečenicu s prethodnom: a\dots; §~213.3
\item[ἀφικνεῖται] ἀφικνέομαι doći; 3. l. sg. ind. prez. (med.)
\item[μέλαν τε ἱμάτιον ] §~40; §~192; §~82
\item[ἠμφιεσμένη\dots] ἀμφιέννυμι ogrnuti, odjenuti; n. sg. ž. r. ptc. perf. pas.; tri točke označavaju da u rukopisima nešto nedostaje
\item[ὡς εἰκὸς ἦν ] fraza: kao što se moglo očekivati; ἔοικα činiti se; n. sg. s. r. ptc. perf. (akt.); εἰμί biti, 3. l. sg. impf. (akt.)
\item[ἐπὶ τῷ ἀνδρὶ ] §~436; §~149
\item[αὐτῆς] §~207
\item[τοιαύτῃ συμφορᾷ ] §~213.4; §~90
\item[κεχρημένῳ] χράομαί τινι koristiti se čime, doživjeti što; d. sg. m. r. ptc. perf. (med.)
\end{description}

%3 itd

{\large
\noindent ἐναντίον δὲ τῆς ἀδελφῆς τῆς ἐμῆς \\
Διονυσόδωρος \\
τά τε οἰκεῖα τὰ αὑτοῦ \\
διέθετο \\
\tabto{2em} ὅπως αὐτῷ ἐδόκει, \\
καὶ περὶ ᾿Αγοράτου τουτουὶ \\
ἔλεγεν \\
\tabto{2em} ὅτι \\
\tabto{4em} $\langle$οἱ$\rangle$ \\
\tabto{4em} αἴτιος ἦν \\
\tabto{6em} τοῦ θανάτου, \\
καὶ ἐπέσκηπτεν \\
\tabto{2em} ἐμοὶ \\
\tabto{2em} καὶ Διονυσίῳ τουτῳί, \\
\tabto{4em} τῷ ἀδελφῷ τῷ αὑτοῦ, \\
\tabto{2em} καὶ τοῖς φίλοις πᾶσι \\
τιμωρεῖν \\
\tabto{2em} ὑπὲρ αὑτοῦ \\
᾿Αγόρατον· \\
καὶ τῇ γυναικὶ τῇ αὑτοῦ ἐπέσκηπτε, \\
νομίζων \\
\tabto{2em} αὐτὴν κυεῖν \\
\tabto{4em} ἐξ αὑτοῦ, \\
ἐὰν γένηται αὐτῇ παιδίον, \\
\tabto{2em} φράζειν τῷ γενομένῳ \\
\tabto{4em} ὅτι τὸν πατέρα αὐτοῦ \\
\tabto{4em} ᾿Αγόρατος ἀπέκτεινε, \\
\tabto{2em} καὶ κελεύειν \\
\tabto{4em} τιμωρεῖν ὑπὲρ αὑτοῦ \\
\tabto{6em} ὡς φονέα ὄντα.\\

}

\begin{description}[noitemsep]

\item[ἐναντίον\dots\ τῆς ἀδελφῆς τῆς ἐμῆς] ἐναντίον τινός u čijoj prisutnosti; §~90; §~210; §~375 
\item[δὲ] čestica δέ povezuje rečenicu s prethodnom: a\dots
\item[Διονυσόδωρος ] §~82
\item[τά\dots\ οἰκεῖα τὰ αὑτοῦ ] §~103; §~209.1; §~373; §~375
\item[τά τε] §~40
\item[τά τε οἰκεῖα\dots\ καὶ περὶ ᾿Αγοράτου\dots] koordinacija parom sastavnih veznika
\item[διέθετο ] διατίθημι med. rasporediti, razdijeliti; 3. l. sg. ind. aor. med.
\item[αὐτῷ ] §~207
\item[ἐδόκει] δοκέω činiti se, činiti se dobrim, sviđati se; 3. l. sg. impf. akt.
\item[περὶ ᾿Αγοράτου τουτουὶ ] §~433; §~82; §~214.2
\item[ἔλεγεν] λέγω govoriti; 3. l. sg. impf. akt.
\item[$\langle$οἱ$\rangle$] §~206.5; u kritičkim izdanjima dopune priređivača (emendacije i konjekture – ono čega nema u rukopisima, a trebalo je biti) označavaju se prelomljenim zagradama
\item[αἴτιος ] §~103; rekcija τινος kriv za što
\item[ἦν ] εἰμί biti; 3. l. sg. impf. (akt.)
\item[τοῦ θανάτου] §~82
\item[ἐπέσκηπτεν ] ἐπισκήπτω τινί + inf. zaklinjati koga da što učini; 3. l. sg. impf. akt.
\item[ἐμοὶ] §~205
\item[Διονυσίῳ τουτῳί] §~82; §~214.2
\item[τῷ ἀδελφῷ τῷ αὑτοῦ] §~82; §~209.1; §~375 
\item[τοῖς φίλοις πᾶσι] §~103; §~193; §~373
\item[τιμωρεῖν ] τιμωρέω τινά osvetiti se kome; inf. prez. akt.
\item[τιμωρεῖν ] τιμωρέω τινά osvetiti se kome; inf. prez. akt.
\item[ὑπὲρ αὑτοῦ ] §~431; §~209.1 
\item[᾿Αγόρατον] §~82
\item[τῇ γυναικὶ τῇ αὑτοῦ ] §~122; §~207; §~375 
\item[ἐπέσκηπτε] ἐπισκήπτω τινί + inf. zaklinjati koga da što učini; 3. l. sg. impf. akt. (kao gore)
\item[νομίζων ] νομίζω smatrati; n. sg. m. r. ptc. prez. akt.
\item[αὐτὴν ] §~207
\item[κυεῖν ] κυέω ἔκ τινος začeti s kime, nositi čije dijete; inf. prez. akt.
\item[ἐξ αὑτοῦ] §~424; §~209.1 
\item[ἐὰν γένηται ] eventualna pogodbena rečenica (ἐὰν s konjunktivom, §~476); γίγνομαι roditi se; 3. l. sg. konj. aor. (med.)
\item[αὐτῇ ] §~207
\item[παιδίον] §~82
\item[φράζειν ] φράζω obavijestiti, reći; inf. prez. akt.
\item[τῷ γενομένῳ] γίγνομαι: roditi se; d. sg. m. r. ptc. aor. (med.)
\item[τὸν πατέρα ] §~148
\item[αὐτοῦ ] §~207
\item[᾿Αγόρατος ] §~82
\item[ἀπέκτεινε] ἀποκτείνω ubiti; 3. l. sg. aor. akt.
\item[κελεύειν] κελεύω zapovijedati; inf. prez. akt.
\item[τιμωρεῖν ] τιμωρέω τινὰ ὑπέρ τινος osvetiti se kome za koga; inf. prez. akt.
\item[ὑπὲρ αὑτοῦ ] §~431; §~209.1 
\item[ὡς φονέα ὄντα] particip ima vrijednost uzročne rečenice: jer\dots; εἰμί biti; a. sg. m. r. ptc. prez. (akt.)
\end{description}



%kraj
