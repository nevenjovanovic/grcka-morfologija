%Unesi korekture NČ 2019-09-09
%\section*{O autoru}


\section*{O tekstu}

U ovom odlomku Tukididove \textit{Povijesti} problematizira se početak naseljavanja Grčke. Trebalo je dugo vremena dok naseobine Grka nisu postale trajne.

%\newpage

\section*{Pročitajte naglas grčki tekst.}

Thuc.\ Historiae 1.2

%Naslov prema izdanju

\medskip

\begin{greek}
{\large
{ \noindent Φαίνεται γὰρ ἡ νῦν Ἑλλὰς καλουμένη οὐ πάλαι βεβαίως οἰκουμένη, ἀλλὰ μεταναστάσεις τε οὖσαι τὰ πρότερα καὶ ῥᾳδίως ἕκαστοι τὴν ἑαυτῶν ἀπολείποντες βιαζόμενοι ὑπό τινων αἰεὶ πλειόνων. τῆς γὰρ ἐμπορίας οὐκ οὔσης, οὐδ’ ἐπιμειγνύντες ἀδεῶς ἀλλήλοις οὔτε κατὰ γῆν οὔτε διὰ θαλάσσης, νεμόμενοί τε τὰ αὑτῶν ἕκαστοι ὅσον ἀποζῆν καὶ περιουσίαν χρημάτων οὐκ ἔχοντες οὐδὲ γῆν φυτεύοντες, ἄδηλον ὂν ὁπότε τις ἐπελθὼν καὶ ἀτειχίστων ἅμα ὄντων ἄλλος ἀφαιρήσεται, τῆς τε καθ’ ἡμέραν ἀναγκαίου τροφῆς πανταχοῦ ἂν ἡγούμενοι ἐπικρατεῖν, οὐ χαλεπῶς ἀπανίσταντο, καὶ δι’ αὐτὸ οὔτε μεγέθει πόλεων ἴσχυον οὔτε τῇ ἄλλῃ παρασκευῇ. μάλιστα δὲ τῆς γῆς ἡ ἀρίστη αἰεὶ τὰς μεταβολὰς τῶν οἰκητόρων εἶχεν, ἥ τε νῦν Θεσσαλία καλουμένη καὶ Βοιωτία Πελοποννήσου τε τὰ πολλὰ πλὴν Ἀρκαδίας, τῆς τε ἄλλης ὅσα ἦν κράτιστα.

}
}
\end{greek}

\section*{Analiza i komentar}

%1

{\large
\begin{greek}
\noindent Φαίνεται γὰρ \\
\tabto{2em} ἡ νῦν Ἑλλὰς καλουμένη \\
οὐ πάλαι βεβαίως οἰκουμένη,\\
ἀλλὰ μεταναστάσεις τε οὖσαι \\
\tabto{2em} τὰ πρότερα \\
καὶ ῥᾳδίως \\
\tabto{2em} ἕκαστοι τὴν ἑαυτῶν ἀπολείποντες \\
\tabto{4em} βιαζόμενοι \\
\tabto{6em} ὑπό τινων αἰεὶ πλειόνων.\\

\end{greek}
}

\begin{description}[noitemsep]
\item[Φαίνεται] φαίνομαι otvara mjesto participu: čini se da\dots; 3. l. sg. ind. prez. medpas.; o ovom predikatu ovise οἰκουμένη, οὖσαι i ἀπολείποντες, usp. §~501, bilj.
\item[γὰρ] naime, jer; čestica najavljuje iznošenje dokaza ili uzroka prethodne tvrdnje §~517
\item[καλουμένη] καλέω zvati, zvati se; n. sg. ž. r. ptc. prez. medpas.; §~90
\item[ἡ νῦν Ἑλλὰς καλουμένη] zemlja koja se sada zove Grčka; §~80; §~123; §~103; subjektni skup s prilogom vremena u atributnom položaju §~375.5
\item[οἰκουμένη] οἰκέω nastaniti; n. sg. ž. r. ptc. prez. medpas.; §~103; predikatni particip ovisi o glavnom glagolu
\item[μεταναστάσεις] §~165
\item[τε] sastavni veznik, stoji uvijek iza riječi koju povezuje, pa se u prijevodu uvrštava ispred nje
\item[οὖσαι] εἰμί biti; n. pl. ž. r. ptc. prez. akt.; §~97.β
\item[τὰ πρότερα] u prijašnja vremena; §~203
\item[ἕκαστοι] §~103
\item[τὴν] §~80, §~90
\item[ἑαυτῶν] §~208
\item[ἀπολείποντες] ἀπολείπω napustiti, ostaviti; n. pl. m. r. ptc. prez. akt.; §~139.α
\item[τὴν ἑαυτῶν ἀπολείποντες] §~80; §~90; §~208; atributni položaj zamjenice ἑαυτῶν §~375.2
\item[βιαζόμενοι] βιάζομαι biti prisiljen, biti natjeran; n. pl. m. r. ptc. prez. medpas.; §~103
\item[ὑπό τινων αἰεὶ πλειόνων] §~217; §~202.5; prijedložni izraz, ὑπό + g.: od\dots; §~418, §~437; atributivni položaj priloga αἰεὶ §~375.5

\end{description}

%2

{\large
\begin{greek}
\noindent \uuline{τῆς γὰρ ἐμπορίας οὐκ οὔσης}, \\
οὐδ’ ἐπιμειγνύντες ἀδεῶς ἀλλήλοις \\
\tabto{2em} οὔτε κατὰ γῆν \\
\tabto{2em} οὔτε διὰ θαλάσσης, \\
νεμόμενοί τε τὰ αὑτῶν ἕκαστοι \\
\tabto{2em} ὅσον ἀποζῆν \\
καὶ περιουσίαν χρημάτων οὐκ ἔχοντες \\
οὐδὲ γῆν φυτεύοντες, \\
ἄδηλον ὂν \\
\tabto{2em} ὁπότε τις ἐπελθὼν καὶ \\
\tabto{4em} \uuline{ἀτειχίστων ἅμα ὄντων} \\
\tabto{2em} ἄλλος ἀφαιρήσεται, \\
τῆς τε καθ’ ἡμέραν ἀναγκαίου τροφῆς \\
\tabto{2em} πανταχοῦ ἂν ἡγούμενοι \\
\tabto{4em} ἐπικρατεῖν, \\
οὐ χαλεπῶς ἀπανίσταντο, \\
καὶ δι’ αὐτὸ \\
\tabto{2em} οὔτε μεγέθει πόλεων ἴσχυον \\
\tabto{2em} οὔτε τῇ ἄλλῃ παρασκευῇ. \\

\end{greek}
}

\begin{description}[noitemsep]
\item[γὰρ] naime, jer; čestica, najavljuje iznošenje dokaza ili uzroka prethodne tvrdnje §~517
\item[τῆς ἐμπορίας οὐκ οὔσης] GA: u prijevodu imenica u genitivu postaje subjekt, a particip u genitivu predikat zavisne priložne rečenice; ne zaboravite: rečenica je niječna!
\item[οὐδ’ (οὐδέ)\dots\ οὔτε\dots\ οὔτε] i niti\dots\ niti\dots\ niti\dots; §~68; sastavni veznik
\item[ἐπιμειγνύντες] ἐπιμείγνυμι τινί miješati se s kim; n. pl. m. r. ptc. prez. akt.; §~139.β
\item[ἀλλήλοις] §~212.2
\item[κατὰ γῆν] §~90.a; prijedložni izraz κατὰ + a.: po\dots; §~418, §~429.B
\item[διὰ θαλάσσης] §~90.a, prijedložni izraz, διὰ + g.: preko\dots, §~418, §~428.A
\item[νεμόμενοί] νέμω dijeliti, posjedovati, uživati, nastavati;  n. pl. m. r. ptc. prez. medpas.; §~103
\item[αὑτῶν] §~209.1
\item[τὰ αὑτῶν] ono od samih sebe; §~373
\item[ἕκαστοι] §~103
\item[ὅσον ] §~103; ὅσον + inf.: dovoljno za\dots + imenica izvedena iz značenja glagola (u infinitivu)
\item[ἀποζῆν ] ἀποζάω imati dovoljno za život, biti dostatno za život; inf. prez. akt.
\item[οὐκ ἔχοντες] ἔχω imati; n. pl. m. r. ptc. prez. akt.; §~139.α
\item[περιουσίαν χρημάτων] §~90.a; §~123
\item[οὐδὲ φυτεύοντες] φυτεύω zasaditi; n. pl. m. r. ptc. prez. akt.; §~139.α
\item[γῆν] §~90.a
\item[ἄδηλον ὂν] budući da je nejasno; ἄδηλόν ἐστί nejasno je; apsolutni akuzativ u bezličnoj konstrukciji, bez očite sintaktičke veze s ostatkom rečenice
\item[ὁπότε] kada; vremenski veznik; §~221, §~487
\item[τις] §~217
\item[ἐπελθὼν ] ἐπέρχομαι pristupiti; n. sg. m. r. ptc. aor. akt.; §~139.α
\item[ὄντων] εἰμί biti; g. pl. m. r. ptc. prez. akt.; §~139.α
\item[ἀτειχίστων ὄντων] GA: u prijevodu imenica u genitivu postaje subjekt zavisne priložne rečenice, a particip u genitivu njezin predikat; §~103
\item[ἄλλος] §~212.a
\item[ἀφαιρήσεται] sc.\ \textbf{γῆν} ἀφαιρέω \textit{ovdje} uzeti za sebe; 3. l. sg. ind. fut. med. 
\item[καθ’ (κατά) ἡμέραν ] §~68; §~74; §~90
\item[ἀναγκαίου] §~103
\item[τῆς τροφῆς] §~80; §~90.a
\item[τῆς καθ’ ἡμέραν ἀναγκαίου τροφῆς] prijedložni izraz u atributnom položaju §~375.4
\item[ἡγούμενοι] ἡγέομαι misliti; n. pl. m. r. ptc. prez. medpas.; §~103
\item[ἂν ἐπικρατεῖν] ἐπικρατέω osvojiti, dobiti; ἂν uz infinitiv označava potencijalnost radnje (u neupravnom govoru zamjenjuje ono što je u upravnom govoru optativ + ἂν)
\item[ἀπανίσταντο] ἀπανίστημι ustati i otići, odseliti se; 3. l. pl. impf. medpas.; glavni glagol o kojem ovisi pet participa u n. pl., međusobno povezani različitim koordiniranim veznicima (potvrdnim i niječnim): \textgreek[variant=ancient]{οὐδ` ἐπιμειγνύντες, νεμόμενοί τε, καὶ\dots\ οὐκ ἔχοντες, οὐδὲ\dots\ φυτεύοντες, τῆς τε\dots\ ἡγούμενοι}
\item[δι’ (διά) αὐτὸ] §~68, §~207; prijedložni izraz, διά + a.: zbog\dots
\item[μεγέθει ] §~153
\item[πόλεων ] §~165
\item[ἴσχυον] ἰσχύω biti snažan, moćan; 3. l. pl. impf. akt. 
\item[τῇ ἄλλῃ παρασκευῇ] §~80; §~90; §~212

\end{description}

%3

{\large
\begin{greek}
\noindent μάλιστα δὲ \\
\tabto{2em} τῆς γῆς ἡ ἀρίστη \\
\tabto{4em} αἰεὶ τὰς μεταβολὰς \\
\tabto{5em} τῶν οἰκητόρων \\
\tabto{4em} εἶχεν, \\
\tabto{2em} ἥ τε νῦν Θεσσαλία καλουμένη\\
\tabto{2em} καὶ Βοιωτία \\
\tabto{2em} Πελοποννήσου τε τὰ πολλὰ \\
\tabto{4em} πλὴν Ἀρκαδίας, \\
\tabto{2em} τῆς τε ἄλλης \\
\tabto{4em} ὅσα ἦν κράτιστα.\\

\end{greek}
}

\begin{description}[noitemsep]
\item[δὲ] a\dots; čestica povezuje rečenicu s prethodnom
\item[ἡ ἀρίστη] §~80, §~90, §~202, §~103
\item[τῆς γῆς] §~80, §~90.a
\item[τὰς μεταβολὰς ] §~80, §~90.a
\item[τῶν οἰκητόρων] §~80, §~82, §~146
\item[εἶχεν] ἔχω imati; 3. l. sg. impf. akt.
\item[ἥ ] §~215
\item[Θεσσαλία ] §~90.α
\item[καλουμένη] καλέω zvati; n. sg. ž. r. ptc. prez. medpas.
\item[Βοιωτία ] §~90.α
\item[Πελοποννήσου] §~82, §~83
\item[τὰ πολλὰ] §~80, §~82; §~196
\item[Ἀρκαδίας] §~90.α
\item[τῆς ἄλλης] §~80, §~90; §~212
\item[ὅσα ] §~103, 219
\item[ἦν κράτιστα] imenski predikat Smyth 909
\item[ἦν ] εἰμί biti; 3. l. sg. impf. akt.; §~361
\item[κράτιστα] §~103, §~202

\end{description}


%kraj
