% Redaktura NČ, NJ; unio daljnje korekture NČ 2019-09-05
%\section*{O autoru}


\section*{O tekstu}

U korpusu od 21 sačuvanog Izokratova govora, \textit{Panegirik} zauzima posebno mjesto, kao iznimno uspješan početak autorove političko-književne publicistike (Aristotel ga na više mjesta u \textit{Retorici} uzima za primjer vrsnog govora). Djelo, na kojem je godinama radio, Izokrat je objavio 380.\ pr.~Kr. Formalno, to je govor pred svečanom panhelenskom skupštinom \textgreek[variant=ancient]{(πανήγυρις);} takve su govore u Olimpiji ranije održali i Gorgija i Lizija. Izokratov vjerojatno nije bio zaista izveden, već se širio samo u pisanom obliku. 

\textit{Panegirik} iznosi politički program, zahtjev za mir među Grcima i poziv na rat protiv Perzijanaca; oboje je moguće samo ako se dominantna Sparta sporazumije s tada slabijom Atenom. Zahtjev da Atena sudjeluje u vodstvu Grčke Izokrat argumentira dugom pohvalom svojeg rodnog grada, dokazujući kako porijeklom i zaslugama nadmašuje Spartu; jedna od zasluga Atene je i njegovanje lijepih umijeća i filozofije, o čemu se govori u odlomku koji slijedi.

\newpage

\section*{Pročitajte naglas grčki tekst.}

Isocr.\ Panegyricus 47

%Naslov prema izdanju

\medskip

\begin{greek}
{\large
{ \noindent Φιλοσοφίαν τοίνυν, ἣ πάντα ταῦτα συνεξεῦρε καὶ συγκατεσκεύασεν καὶ πρός τε τὰς πράξεις ἡμᾶς ἐπαίδευσεν καὶ πρὸς ἀλλήλους ἐπράϋνε καὶ τῶν συμφορῶν τάς τε δι' ἀμαθίαν καὶ τὰς ἐξ ἀνάγκης γιγνομένας διεῖλεν καὶ τὰς μὲν φυλάξασθαι, τὰς δὲ καλῶς ἐνεγκεῖν ἐδίδαξεν, ἡ πόλις ἡμῶν κατέδειξεν, καὶ λόγους ἐτίμησεν, ὧν πάντες μὲν ἐπιθυμοῦσιν, τοῖς δ' ἐπισταμένοις φθονοῦσιν, συνειδυῖα μὲν ὅτι τοῦτο μόνον ἐξ ἁπάντων τῶν ζῴων ἴδιον ἔφυμεν ἔχοντες καὶ διότι τούτῳ πλεονεκτήσαντες καὶ τοῖς ἄλλοις ἅπασιν αὐτῶν διηνέγκαμεν, ὁρῶσα δὲ περὶ μὲν τὰς ἄλλας πράξεις οὕτω ταραχώδεις οὔσας τὰς τύχας ὥστε πολλάκις ἐν αὐταῖς καὶ τοὺς φρονίμους ἀτυχεῖν καὶ τοὺς ἀνοήτους κατορθοῦν, τῶν δὲ λόγων τῶν καλῶς καὶ τεχνικῶς ἐχόντων οὐ μετὸν τοῖς φαύλοις, ἀλλὰ ψυχῆς εὖ φρονούσης ἔργον ὄντας, καὶ τούς τε σοφοὺς καὶ τοὺς ἀμαθεῖς δοκοῦντας εἶναι ταύτῃ πλεῖστον ἀλλήλων διαφέροντας, ἔτι δὲ τοὺς εὐθὺς ἐξ ἀρχῆς ἐλευθέρως τεθραμμένους ἐκ μὲν ἀνδρίας καὶ πλούτου καὶ τῶν τοιούτων ἀγαθῶν οὐ γιγνωσκομένους, ἐκ δὲ τῶν λεγομένων μάλιστα καταφανεῖς γιγνομένους, καὶ τοῦτο σύμβολον τῆς παιδεύσεως ἡμῶν ἑκάστου πιστότατον ἀποδεδειγμένον, καὶ τοὺς λόγῳ καλῶς χρωμένους οὐ μόνον ἐν ταῖς αὑτῶν δυναμένους, ἀλλὰ καὶ παρὰ τοῖς ἄλλοις ἐντίμους ὄντας.

}
}
\end{greek}

\vfill

\newpage

\section*{Analiza i komentar}

%1

{\large
\begin{greek}
\noindent Φιλοσοφίαν τοίνυν, \\
\tabto{2em} ἣ πάντα ταῦτα συνεξεῦρε \\
\tabto{2em} καὶ συγκατεσκεύασεν \\
\tabto{2em} καὶ πρός τε τὰς πράξεις \\
\tabto{6em} ἡμᾶς \\
\tabto{6em} ἐπαίδευσεν \\
\tabto{4em} καὶ πρὸς ἀλλήλους \\
\tabto{6em} ἐπράϋνε \\
\tabto{4em} καὶ τῶν συμφορῶν \\
\tabto{6em} τάς τε δι' ἀμαθίαν \\
\tabto{6em} καὶ τὰς ἐξ ἀνάγκης \\
\tabto{8em} γιγνομένας \\
\tabto{4em} διεῖλεν \\
\tabto{4em} καὶ τὰς μὲν \\
\tabto{6em} φυλάξασθαι, \\
\tabto{4em} τὰς δὲ \\
\tabto{6em} καλῶς ἐνεγκεῖν \\
\tabto{4em} ἐδίδαξεν,

\noindent ἡ πόλις ἡμῶν \\
κατέδειξεν, \\
καὶ λόγους \\
ἐτίμησεν, \\
\tabto{2em} ὧν \\
\tabto{4em} πάντες μὲν ἐπιθυμοῦσιν, \\
\tabto{4em} τοῖς δ' ἐπισταμένοις φθονοῦσιν, \\
συνειδυῖα μὲν ὅτι \\
\tabto{2em} τοῦτο μόνον \\
\tabto{4em} ἐξ ἁπάντων τῶν ζῴων \\
\tabto{2em} ἴδιον \\
\tabto{4em} ἔφυμεν \\
\tabto{2em} ἔχοντες \\
\tabto{2em} καὶ διότι \\
\tabto{4em} τούτῳ \\
\tabto{2em} πλεονεκτήσαντες \\
\tabto{2em} καὶ \\
\tabto{4em} τοῖς ἄλλοις ἅπασιν \\
\tabto{6em} αὐτῶν \\
\tabto{4em} διηνέγκαμεν,

\noindent  ὁρῶσα δὲ \\
\tabto{2em} περὶ μὲν τὰς ἄλλας πράξεις \\
\tabto{2em} οὕτω ταραχώδεις οὔσας τὰς τύχας \\
\tabto{4em} ὥστε πολλάκις \\
\tabto{4em} ἐν αὐταῖς \\
\tabto{4em} καὶ \underline{τοὺς φρονίμους ἀτυχεῖν} \\
\tabto{4em} καὶ \underline{τοὺς ἀνοήτους κατορθοῦν}, \\
\tabto{2em} τῶν δὲ λόγων \\
\tabto{4em} τῶν καλῶς καὶ τεχνικῶς ἐχόντων \\
\tabto{2em} οὐ μετὸν \\
\tabto{4em} τοῖς φαύλοις, \\
\tabto{2em} ἀλλὰ \\
\tabto{4em} ψυχῆς εὖ φρονούσης \\
\tabto{2em} ἔργον ὄντας, \\
\tabto{2em} καὶ τούς τε σοφοὺς\\
\tabto{2em} καὶ τοὺς ἀμαθεῖς \\
\tabto{4em} δοκοῦντας \underline{εἶναι} \\
\tabto{6em} ταύτῃ \\
\tabto{4em} πλεῖστον \\
\tabto{6em} ἀλλήλων \\
\tabto{4em} \underline{διαφέροντας},

\tabto{2em} ἔτι δὲ \\
\tabto{4em} τοὺς \\
\tabto{6em} εὐθὺς ἐξ ἀρχῆς \\
\tabto{4em} ἐλευθέρως τεθραμμένους \\
\tabto{6em} ἐκ μὲν ἀνδρίας \\
\tabto{8em} καὶ πλούτου \\
\tabto{8em} καὶ τῶν τοιούτων ἀγαθῶν \\
\tabto{6em} οὐ γιγνωσκομένους, \\
\tabto{8em} ἐκ δὲ τῶν λεγομένων \\
\tabto{6em} μάλιστα καταφανεῖς γιγνομένους, \\
\tabto{2em} καὶ τοῦτο \\
\tabto{4em} σύμβολον \\
\tabto{6em} τῆς παιδεύσεως \\
\tabto{10em} ἡμῶν \\
\tabto{8em} ἑκάστου \\
\tabto{4em} πιστότατον ἀποδεδειγμένον,

\tabto{2em} καὶ τοὺς \\
\tabto{4em} λόγῳ \\
\tabto{2em} καλῶς χρωμένους \\
\tabto{4em} οὐ μόνον \\
\tabto{6em} ἐν ταῖς αὑτῶν \\
\tabto{4em} δυναμένους, \\
\tabto{4em} ἀλλὰ καὶ \\
\tabto{6em} παρὰ τοῖς ἄλλοις \\
\tabto{4em} ἐντίμους ὄντας.\\

\end{greek}
}

\begin{description}[noitemsep]
\item[Φιλοσοφίαν ] §~90
\item[ἣ ] §~215
\item[ἣ\dots\ συνεξεῦρε] \textbf{\textgreek[variant=ancient]{καὶ συγκατεσκεύασεν καὶ\dots\ ἐπαίδευσεν καὶ\dots\ ἐπράϋνε καὶ\dots\ διεῖλεν καὶ\dots\ ἐδίδαξεν}} odnosna zamjenica uvodi niz odnosnih rečenica, njezin je antecedent \textgreek[variant=ancient]{Φιλοσοφίαν}
\item[πάντα ταῦτα] §~193, §~213.2; već u odjeljku 26 Izokrat je počeo opisivati atenske doprinose svijetu, \textgreek[variant=ancient]{ὅσων δὲ τοῖς ἄλλοις ἀγαθῶν αἴτιοι γεγόναμεν}, od poljoprivrede, preko koloniziranja otoka i Male Azije, državnog uređenja temeljenog na zakonima, otvorenosti strancima i trgovini, do festivala
\item[συνεξεῦρε] συνεξευρίσκω pomoći da se (nešto) pronađe; 3. l. sg. ind. aor. akt.
\item[συγκατεσκεύασεν] συγκατασκευάζω pomoći da se (nešto) uspostavi, organizira; 3. l. sg. ind. aor. akt.
\item[πρός τε τὰς πράξεις\dots] \textbf{καὶ πρὸς ἀλλήλους\dots}\ §~40; koordinacija rečeničnih članova pomoću sastavnih veznika
\item[πρός\dots\ τὰς πράξεις] §~435; §~165
\item[ἡμᾶς] §~205
\item[ἐπαίδευσεν] παιδεύω  πρός τι odgojiti za nešto, obrazovati za nešto; 3. l. sg. ind. aor. akt.
\item[πρὸς ἀλλήλους] §~435; §~212
\item[ἐπράϋνε] πραΰνω omekšati, pripitomiti; 3. l. sg. impf. akt.
\item[τῶν συμφορῶν] §~90; dijelni genitiv, §~395
\item[τάς τε\dots\ καὶ τὰς\dots] §~395; τὰς sc.\ συμφοράς; koordinacija rečeničnih članova pomoću sastavnih veznika
\item[τάς\dots\ δι' ἀμαθίαν\dots\ γιγνομένας ] §~373; §~68; §~428; §~90; γίγνομαι postati, nastati; a. pl. ž. r. ptc. prez. medpas.
\item[τὰς ἐξ ἀνάγκης γιγνομένας] §~373; §~424; §~90; γίγνομαι postati, nastati; a. pl. ž. r. ptc. prez. medpas.
\item[διεῖλεν] διαιρέω razlučiti, razlikovati; 3. l. sg. ind. aor. akt.
\item[τὰς μὲν\dots\ τὰς δὲ\dots] §~370; sc.\ \textgreek[variant=ancient]{συμφοράς δι' ἀμαθίαν – συμφοράς ἐξ ἀνάγκης}
\item[φυλάξασθαι] φυλάσσω, \textit{ovdje} med. φυλάσσομάι τι čuvati se nečega; inf. aor. med.
\item[καλῶς ἐνεγκεῖν] §~204; καλῶς φέρω dobro podnositi, junački trpjeti; inf. aor. akt.
\item[ἐδίδαξεν] διδάσκω poučavati; 3. l. sg. ind. aor. akt.
\item[ἡ πόλις ἡμῶν] sc.\ Ἀθῆναι
\item[ἡ πόλις] §~165
\item[ἡμῶν ] §~205
\item[κατέδειξεν] καταδείκνυμι otkriti, uvesti; 3. l. sg. ind. aor. akt.
\item[λόγους] §~82
\item[ἐτίμησεν] τιμάω poštovati, častiti; 3. l. sg. ind. aor. akt.
\item[ὧν\dots\ ἐπιθυμοῦσιν\dots\ φθονοῦσιν] §~215; odnosna zamjenica uvodi dvije zavisne odnosne rečenice, njezin je antecedent λόγους
\item[πάντες μὲν\dots, τοῖς δ' ἐπισταμένοις\dots] koordinacija rečeničnih članova pomoću (suprotnih) čestica; §~68
\item[πάντες] §~193
\item[ἐπιθυμοῦσιν] ἐπιθυμέω τινός željeti nešto; 3. l. pl. ind. prez. akt.
\item[τοῖς\dots\ ἐπισταμένοις ] §~373; ἐπίσταμαι znati; d. pl. m. r. ptc. med.
\item[φθονοῦσιν] φθονέω zavidjeti; 3. l. pl. ind. prez. akt.
\item[συνειδυῖα μὲν\dots\ ὁρῶσα δὲ\dots] koordinacija rečeničnih članova pomoću (suprotnih) čestica; participi su ovisni o \textgreek[variant=ancient]{ἡ πόλις}
\item[συνειδυῖα] σύνοιδα biti svjestan, dobro znati; kao \textit{verbum sentiendi} otvara mjesto izričnoj rečenici s veznikom ὅτι; n. sg. ž. r. ptc. prez. akt.
\item[τοῦτο] sc.\ λόγους; §~213.2
\item[μόνον] §~103
\item[ἐξ ἁπάντων ] §~424;  §~379
\item[τῶν ζῴων] §~82
\item[ἴδιον] §~103; opisuje τοῦτο
\item[ἔφυμεν] φύω pas. roditi se; 1. l. pl. aor. pas.
\item[ἔχοντες] ἔχω imati; n. pl. m. r. ptc. prez. akt.; otvara mjesto objektu τοῦτο
\item[τούτῳ ] sc.\ λόγοις; §~213.2
\item[πλεονεκτήσαντες] πλεονεκτέω τινι nadmašivati nečim; n. pl. m. r. ptc. aor. akt.
\item[τοῖς ἄλλοις ἅπασιν] §~373; §~379; §~212
\item[αὐτῶν ] sc.\ τῶν ζῴων; §~207
\item[διηνέγκαμεν] διαφέρω τινι razlikovati se od nekoga; 1. l. pl. ind. aor. akt.
\item[ὁρῶσα] sc.\ \textgreek[variant=ancient]{ἡ πόλις ἡμῶν; ὁράω} gledati, vidjeti; n. sg. ž. r. ptc. prez. akt.
\item[περὶ μὲν τὰς ἄλλας\dots\ τῶν δὲ λόγων\dots] koordinacija rečeničnih članova pomoću para suprotnih čestica; §~395; §~82
\item[περὶ\dots\ τὰς ἄλλας πράξεις] §~433; §~212; §~165
\item[ταραχώδεις] §~153
\item[οὔσας ] εἰμί biti; a. pl. ž. r. ptc. prez. akt.; kao kopulativni glagol ima nužnu imensku dopunu \textgreek[variant=ancient]{ταραχώδεις}
\item[τὰς τύχας ] §~90, u službi objekta participa ὁρῶσα, koji ovdje otvara mjesto dvama akuzativima \textgreek[variant=ancient]{ταραχώδεις οὔσας τὰς τύχας} da su igre slučaja nepredvidive\dots
\item[οὕτω\dots\ ὥστε\dots] prilog οὕτω najavljuje posljedičnu rečenicu koju uvodi veznik ὥστε tako da\dots; budući da su subjekti glavne i zavisne rečenice različiti, ὥστε otvara mjesto A+I, §~473
\item[ἐν αὐταῖς] §~426; §~207
\item[τοὺς φρονίμους ] supstantiviranje članom §~373; §~103
\item[ἀτυχεῖν] ἀτυχέω imati nesreću, doživjeti neuspjeh; inf. prez. akt.
\item[τοὺς ἀνοήτους ] supstantiviranje članom §~373; §~106
\item[κατορθοῦν] κατορθόω uspjeti; inf. prez. akt.
\item[τῶν\dots\ λόγων τῶν\dots\  ἐχόντων ] §~82; §~375; καλῶς ἔχω biti u dobrom stanju, τεχνικῶς ἔχω biti u skladu s pravilima umijeća; g. pl. m. r. ptc. prez. akt.
\item[μετὸν] sc.\ \textgreek[variant=ancient]{ἔργον; μέτειμι,} bezlično \textgreek[variant=ancient]{μέτεστί τινί τινος} netko ima udjela u nečemu; n. sg. s. r. ptc. prez. akt.
\item[τοῖς φαύλοις] supstantiviranje članom §~373; §~103
\item[ψυχῆς ] §~90
\item[εὖ φρονούσης] εὖ φρονέω dobro misliti, znati misliti (LSJ φρονέω A.I.2); g. sg. ž. r. ptc. prez. akt.
\item[ἔργον ] §~82
\item[ὄντας] sc.\ λόγους; εἰμί biti; a. pl. m. r. ptc. prez. akt., u službi objekta participa ὁρῶσα; imenski predikat je fraza ἔργον ἐστί τινος nečiji je posao, nečija je stvar (kojom se treba baviti)
\item[τούς τε\dots\ καὶ τοὺς\dots] koordinacija pomoću sastavnih čestica
\item[τούς\dots\ σοφοὺς] supstantiviranje članom §~373; §~103; u službi objekta participa ὁρῶσα, koji ovdje otvara mjesto dvama akuzativima  \textgreek[variant=ancient]{τούς τε σοφοὺς καὶ τοὺς ἀμαθεῖς δοκοῦντας} da i mudraci i neobrazovani smatraju da\dots
\item[τοὺς ἀμαθεῖς] supstantiviranje članom §~373;  §~153
\item[δοκοῦντας ] δοκέω smatrati; kao \textit{verbum sentiendi} otvara mjesto A+I; a. pl. m. r. ptc. prez. akt.
\item[εἶναι] εἰμί biti; kao kopulativni glagol ima nužnu imensku dopunu; inf. prez. akt.
\item[ταύτῃ ] time; §~213.2, LSJ οὗτος C.VIII.4.b
\item[πλεῖστον ] §~204.3
\item[ἀλλήλων ] §~212
\item[διαφέροντας] διαφέρω τινός razlikovati se od nekoga; a. pl. m. r. ptc. prez. akt.
\item[ἔτι δὲ ] uvodi dodatak prethodnome: a još i\dots
\item[τοὺς\dots\ τεθραμμένους ] τρέφω hraniti, odgajati; a. pl. m. r. ptc. perf. medpas.; supstantiviranje članom §~373; u službi objekta participa ὁρῶσα, koji ovdje otvara mjesto dvama akuzativima \textgreek[variant=ancient]{τοὺς\dots\ τεθραμμένους\dots\ γιγνωσκομένους\dots} da oni koji su odgojeni\dots\ nisu procjenjivani\dots
\item[εὐθὺς ] prilog s vremenskim značenjem
\item[ἐξ ἀρχῆς ] §~424; §~90
\item[ἐλευθέρως ] §~204
\item[ἐκ μὲν\dots\ ἐκ δὲ\dots] koordinacija rečeničnih članova pomoću para suprotnih čestica
\item[ἐκ\dots\ ἀνδρίας καὶ πλούτου καὶ τῶν τοιούτων ἀγαθῶν] §~424; §~90; §~82; §~373; §~213.4; §~103
\item[γιγνωσκομένους] γιγνώσκω procjenjivati; a. pl. m. r. ptc. prez. medpas.; u službi (drugog) objekta participa ὁρῶσα
\item[ἐκ\dots\ τῶν λεγομένων ] §~424; §~373; λέγω govoriti, reći; g. pl. s. r. ptc. prez. medpas.; supstantiviranje članom §~373
\item[μάλιστα ] §~204.3
\item[καταφανεῖς ] §~153
\item[γιγνομένους] γίγνομαι postati; kao kopulativan glagol otvara mjesto imenskoj dopuni καταφανεῖς; a. pl. m. r. ptc. prez. med.; u službi (drugog) objekta participa ὁρῶσα
\item[τοῦτο ] sc.\ λόγους; §~213.2; u službi objekta participa ὁρῶσα, koji ovdje otvara mjesto dvama akuzativima  \textgreek[variant=ancient]{τοῦτο σύμβολον\dots}\ da je to znak\dots
\item[σύμβολον ] §~82
\item[τῆς παιδεύσεως ] §~165
\item[ἡμῶν ] §~205
\item[ἑκάστου ] §~103
\item[πιστότατον ] §~204.3
\item[ἀποδεδειγμένον] ἀποδείκνυμι med. pokazati se kao; otvara mjesto imenskoj dopuni πιστότατον; n. sg. s. r. ptc. perf. medpas.
\item[τοὺς\dots\ χρωμένους] §~373; χράομαί τινι koristiti se nečim; a. pl. m. r. ptc. prez. med.; u službi objekta participa ὁρῶσα, koji ovdje otvara mjesto dvama akuzativima  \textgreek[variant=ancient]{τοὺς\dots\ χρωμένους\dots\ δυναμένους\dots}\ da su oni koji se služe moćni\dots
\item[λόγῳ ] §~82
\item[καλῶς] §~204
\item[οὐ μόνον\dots\ ἀλλὰ καὶ\dots] koordinacija rečeničnih članova: ne samo\dots\ nego i\dots
\item[ἐν ταῖς αὑτῶν] sc.\ πόλεσι; §~426; supstantiviranje članom §~373; §~207.4
\item[δυναμένους] δύναμαι moći, biti moćan; a. pl. m. r. ptc. prez. med.; u službi (drugog) objekta participa ὁρῶσα
\item[παρὰ τοῖς ἄλλοις ] §~434; supstantiviranje članom §~373; §~212
\item[ἐντίμους] §~106
\item[ὄντας] εἰμί biti; kao kopulativni glagol otvara mjesto imenskoj dopuni ἐντίμους; a. pl. m. r. ptc. prez. (akt.); u službi (drugog) objekta participa ὁρῶσα

\end{description}


%kraj
