%\section*{O autoru}



\section*{O tekstu}

U Plutarhovim \textit{Usporednim životopisima} \textgreek[variant=ancient]{(Βίοι παράλληλοι)} biografija Solona Atenjanina (639.\ – 559.\ pr.~Kr.) stoji uz životopis Rimljanina Publija Valerija Poplikole (ili Publikole, umro 503.\ pr.~Kr.).

Pošto je, na početku Solonova životopisa, kratko prikazao podrijetlo atenskog mudraca, njegovo prijateljstvo s kasnijim tiraninom Pizistratom i bavljenje trgovinom (koju Solon nije smatrao nedostojnom svojega plemenitog roda), Plutarh čini digresiju i pripovijeda o Talesu iz maloazijskog Mileta, prvom koji je filozofiju učinio spekulativnom, a ne isključivo praktičnom i pragmatičnom djelatnošću. 

I Tales i Solon, kao i Bijant iz Prijene, pripadali su sedmorici drevnih grčkih mudraca \textgreek[variant=ancient]{(οἱ ἑπτὰ σοφοί).} Oni su, prema Plutarhu, status mudraca potvrdili ponajprije u događaju s tronošcem \textgreek[variant=ancient]{(τρίπους)} o kojem pripovijeda odabrani odlomak.

%\newpage

\section*{Pročitajte naglas grčki tekst.}

%Naslov prema izdanju
Plut.\ Solon 4.3

\medskip

{\large
\begin{greek}
\noindent Κῴων γὰρ ὥς φασι καταγόντων σαγήνην, καὶ ξένων ἐκ Μιλήτου πριαμένων τὸν βόλον οὔπω φανερὸν ὄντα, χρυσοῦς ἐφάνη τρίπους ἑλκόμενος, ὃν λέγουσιν Ἐλένην πλέουσαν ἐκ Τροίας αὐτόθι καθεῖναι, χρησμοῦ τινος ἀναμνησθεῖσαν παλαιοῦ. γενομένης δὲ τοῖς ξένοις πρῶτον ἀντιλογίας πρὸς τοὺς ἁλιέας περὶ τοῦ τρίποδος, εἶτα τῶν πόλεων ἀναδεξαμένων τὴν διαφοράν, ἄχρι πολέμου προελθοῦσαν, ἀνεῖλεν ἀμφοτέροις ἡ Πυθία τῷ σοφωτάτῳ τὸν τρίποδα ἀποδοῦναι. καὶ πρῶτον μὲν ἀπεστάλη πρὸς Θαλῆν εἰς Μίλητον, ἑκουσίως τῶν Κῴων ἑνὶ δωρουμένων ἐκείνῳ περὶ οὗ πρὸς ἅπαντας ὁμοῦ Μιλησίους ἐπολέμησαν. Θάλεω δὲ Βίαντα σοφώτερον ἀποφαίνοντος αὑτοῦ, πρὸς ἐκεῖνον ἧκεν, ὑπ' ἐκείνου δ' αὖθις ἀπεστάλη πρὸς ἄλλον ὡς σοφώτερον. εἶτα περιιὼν καὶ ἀναπεμπόμενος, οὕτως ἐπὶ Θαλῆν τὸ δεύτερον ἀφίκετο, καὶ τέλος εἰς Θήβας ἐκ Μιλήτου κομισθεὶς τῷ Ἰσμηνίῳ Ἀπόλλωνι καθιερώθη.

\end{greek}

}

\section*{Analiza i komentar}


%1

{\large
\noindent \uuline{Κῴων} γὰρ \\
\tabto{2em} ὥς φασι \\
\uuline{καταγόντων} σαγήνην, \\
καὶ \uuline{ξένων} ἐκ Μιλήτου \uuline{πριαμένων}\\
\tabto{4em} τὸν βόλον \\
\tabto{6em} οὔπω φανερὸν ὄντα, \\
χρυσοῦς \\
ἐφάνη \\
τρίπους ἑλκόμενος, \\
\tabto{2em} ὃν λέγουσιν \\
\tabto{4em} \underline{῾Ελένην πλέουσαν} \\
\tabto{6em} ἐκ Τροίας \\
\tabto{6em} αὐτόθι\\
\tabto{4em} \underline{καθεῖναι}, \\
\tabto{6em} χρησμοῦ τινος\\
\tabto{4em} ἀναμνησθεῖσαν \\
\tabto{6em} παλαιοῦ.\\

}

\begin{description}[noitemsep]
\item[Κῴων] §~103
\item[ὥς φασι] §~40
\item[φασι] φημί govoriti; 3. l. pl. ind. prez. akt.
\item[καταγόντων] κατάγω donijeti s mora na kopno, izvlačiti; g. pl. m. r. ptc. prez. akt.; participski dio GA: pošto su\dots
\item[σαγήνην] §~90
\item[ξένων] §~82
\item[ἐκ Μιλήτου] §~424, §~82
\item[πριαμένων] ἐπριάμην kupiti (glagol ima samo aorist); g. pl. m. r. ptc. aor. med.; participski dio GA: pošto su\dots
\item[τὸν βόλον] §~82
\item[φανερὸν] §~103
\item[ὄντα] εἰμί biti; a. sg. m. r. ptc. prez. (akt.); glagol nepotpuna značenja (kopula) otvara mjesto nužnoj imenskoj dopuni
\item[χρυσοῦς] §~107
\item[ἐφάνη] φαίνω pas. pojaviti se; 3. l. sg. ind. aor. pas.
\item[τρίπους] §~126
\item[ἑλκόμενος] ἕλκω vući; n. sg. m. r. ptc. prez. medpas.; predikatni particip uz φαίνω §~502
\item[ὃν] §~215; antecedent je odnosne zamjenice τρίπους
\item[λέγουσιν ] λέγω govoriti; 3. l. pl. ind. prez. akt.; \textit{verbum dicendi} ovdje otvara mjesto dvama akuzativima, od kojih je drugi A+I (ὃν, ῾Ελένην\dots\ καθεῖναι): za koji kažu da ga je Helena\dots
\item[῾Ελένην ] §~90; imenski dio A+I
\item[πλέουσαν ] πλέω ploviti; a. sg. ž. r. ptc. prez. akt.; imenski dio A+I
\item[ἐκ Τροίας] §~424, §~90
\item[καθεῖναι] καθίημι ispustiti, baciti; inf. aor. akt.
\item[χρησμοῦ] §~82
\item[χρησμοῦ τινος] §~40
\item[τινος] §~217
\item[ἀναμνησθεῖσαν] ἀναμιμνήσκω podsjetiti, pas. sjetiti se, rekcija τινος čega; a. sg. ž. r. ptc. aor. pas.; odgovara hrvatskoj uzročnoj rečenici: zato što\dots
\item[παλαιοῦ] §~103

\end{description}

{\large
\noindent \uuline{γενομένης} δὲ τοῖς ξένοις \\
πρῶτον \\
\uuline{ἀντιλογίας} \\
\tabto{2em} πρὸς τοὺς ἁλιέας \\
\tabto{2em} περὶ τοῦ τρίποδος, \\
εἶτα \\
\uuline{τῶν πόλεων ἀναδεξαμένων} τὴν διαφοράν, \\
\tabto{4em} ἄχρι πολέμου \\
\tabto{2em} προελθοῦσαν, \\
ἀνεῖλεν ἀμφοτέροις \\
ἡ Πυθία \\
\tabto{2em} τῷ σοφωτάτῳ \\
\tabto{2em} τὸν τρίποδα ἀποδοῦναι.\\

}

\begin{description}[noitemsep]

\item[γενομένης ] γίγνομαι nastati; g. sg. ž. r. aor. med.; participski dio GA
\item[δὲ ] čestica povezuje rečenicu s prethodnom: a\dots
\item[τοῖς ξένοις] §~82
\item[πρῶτον\dots\ εἶτα] vremenski prilozi koordiniraju rečenične članove
\item[ἀντιλογίας] §~90; imenski dio GA
\item[πρὸς τοὺς ἁλιέας] §~435, §~175
\item[περὶ τοῦ τρίποδος] §~433, §~126
\item[τῶν πόλεων] §~165; imenski dio GA
\item[ἀναδεξαμένων ] ἀναδέχομαι preuzimati; g. pl. ž. r. aor. med.; participski dio GA
\item[τὴν διαφοράν] §~90
\item[πολέμου] §~82
\item[προελθοῦσαν] προέρχομαι napredovati, eskalirati; a. sg. ž. r. ptc. aor. akt.
\item[ἀνεῖλεν ] ἀναιρέω naređivati, određivati; 3. l. sg. ind. aor. akt.; glagol otvara mjesto dopuni u infinitivu
\item[ἀμφοτέροις ] §~224.3
\item[ἡ Πυθία] §~90; svećenica Apolonova proročišta u Delfima, u proročkom zanosu davala bi (često nejasne) odgovore na pitanja
\item[τῷ σοφωτάτῳ ] §~197; poimeničenje članom §~373
\item[τὸν τρίποδα] §~126
\item[ἀποδοῦναι] ἀποδίδωμι predati; inf. aor. akt.

\end{description}

%3 

{\large
\noindent καὶ πρῶτον μὲν ἀπεστάλη \\
\tabto{2em} πρὸς Θαλῆν \\
\tabto{2em} εἰς Μίλητον, \\
\tabto{2em} ἑκουσίως \uuline{τῶν Κῴων} ἑνὶ \uuline{δωρουμένων} ἐκείνῳ \\
\tabto{4em} περὶ οὗ \\
\tabto{4em} πρὸς ἅπαντας ὁμοῦ Μιλησίους \\
\tabto{4em} ἐπολέμησαν.\\

}

\begin{description}[noitemsep]
\item[πρῶτον μὲν\dots\ Θάλεω δὲ\dots] koordinacija dviju rečenica parom čestica
\item[ἀπεστάλη] sc.\ τρίπους; ἀποστέλλω poslati; 3. l. sg. ind. aor. pas.
\item[πρὸς Θαλῆν] §~435, §~100
\item[εἰς Μίλητον] §~419, §~82
\item[τῶν Κῴων ] §~103
\item[ἑνὶ\dots\ ἐκείνῳ] §~224; §~214.3 (Tales je iz Mileta)
\item[δωρουμένων] δωρέομαι poklanjati; g. pl. m. r. ptc. prez. medpas.
\item[περὶ οὗ] §~433, §~215; zamjenica uvodi odnosnu rečenicu, antecedent je neizrečen (τρίπους): poklanjali su ono zbog čega\dots
\item[πρὸς ἅπαντας\dots\ Μιλησίους] §~435, §~103
\item[ἐπολέμησαν] πολεμέω ratovati \textgreek[variant=ancient]{περί τινος πρός τινα} zbog čega protiv koga; 3. l. pl. ind. aor. akt.

\end{description}

%4

{\large
\noindent \uuline{Θάλεω} δὲ Βίαντα σοφώτερον \uuline{ἀποφαίνοντος} αὑτοῦ, \\
πρὸς ἐκεῖνον ἧκεν, \\
ὑπ' ἐκείνου δ' αὖθις ἀπεστάλη \\
\tabto{2em} πρὸς ἄλλον \\
\tabto{4em} ὡς σοφώτερον. \\

}

\begin{description}[noitemsep]

\item[Θάλεω ] §~100, §~102 (LSJ Θᾰλῆς, ὁ, g. Θάλεω, d. Θαλῇ, a. Θαλῆν; g. također Θαλοῦ); imenski dio GA
\item[δὲ] čestica označava nastavak koordinacije: a\dots
\item[Βίαντα] §~139
\item[σοφώτερον\dots\ αὑτοῦ] §~197; genitiv usporedbe \textit{(comparationis)} §~404
\item[ἀποφαίνοντος] ἀποφαίνω obznaniti; glagol otvara mjesto dvama akuzativima: ἀποφαίνω τινά τι obznanjujem da je tko kakav; g. sg. m. r. ptc. prez. akt.; participski dio GA
\item[πρὸς ἐκεῖνον] §~435, §~214.3
\item[ἧκεν] ἥκω doći (s vrijednošću perfekta); 3. l. sg. impf. akt. (s vrijednošću prošlog perfekta)
\item[ὑπ' ἐκείνου] §~437, §~214.3
\item[δ'] čestica označava nastavak koordinacije: a\dots
\item[δ' αὖθις ] §~68
\item[ἀπεστάλη ] ἀποστέλλω poslati; 3. l. sg. ind. aor. pas.
\item[πρὸς ἄλλον] §~435, §~212.1
\item[ὡς σοφώτερον] ὡς izražava subjektivan uzrok, kao kod adverbijalnog participa (§~503.2): kao da\dots, jer je navodno\dots; §~197

\end{description}

%5

{\large
\noindent εἶτα \\
\tabto{2em} περιιὼν \\
\tabto{2em} καὶ ἀναπεμπόμενος, \\
οὕτως \\
\tabto{2em} ἐπὶ Θαλῆν \\
\tabto{2em} τὸ δεύτερον \\
ἀφίκετο, \\
καὶ \\
\tabto{2em} τέλος \\
\tabto{2em} εἰς Θήβας \\
\tabto{2em} ἐκ Μιλήτου \\
κομισθεὶς \\
\tabto{2em} τῷ ᾿Ισμηνίῳ ᾿Απόλλωνι \\
καθιερώθη.\\

}

\begin{description}[noitemsep]
\item[περιιὼν] περιέρχομαι obilaziti; n. sg. m. r. ptc. prez. akt. (od osnove περίειμι)
\item[ἀναπεμπόμενος] ἀναπέμπω poslati natrag; n. sg. m. r. ptc. prez. medpas.
\item[ἐπὶ Θαλῆν] §~436, §~100
\item[τὸ δεύτερον] §~223; poimeničeni redni broj (§~373) upotrijebljen priložno odgovara na pitanje „koji put?''
\item[ἀφίκετο] ἀφικνέομαι doći; 3. l. sg. ind. aor. med.
\item[τέλος] §~153; ovdje upotrijebljeno priložno: naposljetku\dots
\item[εἰς Θήβας] §~419, §~90
\item[ἐκ Μιλήτου] §~424, §~82
\item[κομισθεὶς] κομίζω odnijeti, prenijeti; n. sg. m. r. ptc. aor. pas.
\item[τῷ ᾿Ισμηνίῳ ᾿Απόλλωνι] §~103, §~131; u Tebi Apolon ima atribut „Ismenijski'' jer je njegov hram (u kojem se nalazio kip boga od cedrova drveta iz VII./VI. st.) izvan grada, na niskom brijegu iznad rijeke Ismena (Ἰσμηνός); rijeka, prethodno zvana Ladon, preimenovana je po sinu Apolona i nimfe Melije; Pindarova 11. pitijska oda (4–6) kaže: \textgreek[variant=ancient]{ἴτε\dots\ πὰρ Μελίαν χρυσέων ἐς ἄδυτον τριπόδων / θησαυρόν, ὃν περίαλλ᾽ ἐτίμασε Λοξίας, / Ἰσμήνιον δ᾽ ὀνύμαξεν, ἀλαθέα μαντίων θῶκον\dots}
\item[καθιερώθη] καθιερόω posvetiti; 3. l. sg. ind. aor. pas.
\end{description}


%kraj
