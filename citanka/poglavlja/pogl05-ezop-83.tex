% Redaktura NZ
%\section*{O autoru}



\section*{O tekstu}

Uspješan javni nastup donio je majmunu položaj kralja životinja. No, lisica se s tim ne slaže.

%\newpage

\section*{Pročitajte naglas grčki tekst.}

%Naslov prema izdanju

Aesop. Fabulae 83

\medskip

{\large
\begin{greek}
\noindent ΑΛΩΠΗΞ ΚΑΙ ΠΙΘΗΚΟΣ 

\noindent Ἐν συνόδῳ τῶν ἀλόγων ζῴων πίθηκος ὀρχησάμενος καὶ εὐδοκιμήσας βασιλεὺς ὑπ' αὐτῶν ἐχειροτονήθη. ἀλώπηξ δὲ αὐτῷ φθονήσασα ὡς ἐθεάσατο ἔν τινι πάγῃ κρέας κείμενον, ἀγαγοῦσα αὐτὸν ἐνταῦθα ἔλεγεν, ὡς εὑροῦσα θησαυρὸν αὐτὴ μὲν οὐκ ἐχρήσατο, γέρας δὲ αὐτῷ τῆς βασιλείας τετήρηκε καὶ παρῄνει αὐτῷ λαβεῖν. τοῦ δὲ ἀμελετήτως ἐπελθόντος καὶ ὑπὸ τῆς παγίδος συλληφθέντος αἰτιωμένου τε τὴν ἀλώπεκα ὡς ἐνεδρεύσασαν αὐτῷ ἐκείνη ἔφη· ``ὦ πίθηκε, σὺ δὲ τοιαύτην ψυχὴν ἔχων τῶν ἀλόγων ζῴων βασιλεύεις;''


οὕτως οἱ τοῖς πράγμασιν ἀπερισκέπτως ἐπιχειροῦντες πρὸς τῷ δυστυχεῖν καὶ γέλωτα ὀφλισκάνουσιν. 
\end{greek}

}

\newpage


\section*{Analiza i komentar}

%T

\begin{description}[noitemsep]

\item[ΑΛΩΠΗΞ] §~115
\item[ΠΙΘΗΚΟΣ] §~82
\end{description}

%1

{\large
\noindent Ἐν συνόδῳ \\
\tabto{2em} τῶν ἀλόγων ζῴων \\
πίθηκος \\
\tabto{2em} ὀρχησάμενος καὶ εὐδοκιμήσας \\
βασιλεὺς \\
\tabto{2em} ὑπ' αὐτῶν \\
ἐχειροτονήθη.\\

}

\begin{description}[noitemsep]

\item[Ἐν συνόδῳ] §~82, §~83; prijedložni izraz u službi priložne oznake, ἐν + d.: na\dots, §~418, §~426 
\item[τῶν ἀλόγων ζῴων] §~80, §~103, §~82
\item[πίθηκος] §~82
\item[ὀρχησάμενος καὶ εὐδοκιμήσας] ὀρχέομαι plesati, n. sg. m.r. ptc. aor. medpas., §~82; εὐδοκιμέω dopasti se, n. sg. m. r. ptc. aor. akt., §~139
\item[βασιλεὺς] §~175
\item[ὑπ' αὐτῶν] = ὑπὸ αὐτῶν §~68, §~207; prijedložni izraz ὑπό + g.: od\dots, uz glagole u pasivu izražava vršitelja radnje
\item[ἐχειροτονήθη] χειροτονέω biti izabran, 3. l. sg. ind. aor. pas.
\end{description}

{\large
\noindent ἀλώπηξ δὲ \\
\tabto{2em} αὐτῷ φθονήσασα \\
ὡς ἐθεάσατο \\
\tabto{2em} ἔν τινι πάγῃ \\
κρέας κείμενον, \\
ἀγαγοῦσα αὐτὸν ἐνταῦθα \\
ἔλεγεν, \\
\tabto{2em} ὡς εὑροῦσα θησαυρὸν \\
\tabto{2em} αὐτὴ μὲν οὐκ ἐχρήσατο, \\
\tabto{2em} γέρας δὲ αὐτῷ τῆς βασιλείας τετήρηκε \\
\tabto{2em} καὶ παρῄνει αὐτῷ \\
\tabto{4em} λαβεῖν.\\

}

\begin{description}[noitemsep]
%ispravio
\item[δὲ] čestica suprotnoga značenja, povezuje ovu rečenicu s prethodnom: a\dots
\item[ἀλώπηξ] §~115
\item[αὐτῷ] §~207
\item[φθονήσασα] φθονέω τινί zavidjeti komu; n. sg. ž. r. ptc. aor. akt. §~90
\item[ὡς ἐθεάσατο] zavisna vremenska rečenica: kad je\dots
\item[ἐθεάσατο] θεάομαι vidjeti; 3. l. sg. ind. aor. med. 
\item[τινι] §~217-218
\item[πάγῃ] §~90
\item[ἔν τινι πάγῃ] prijedložni izraz ἐν + d.: u\dots; §~418, §~426 
\item[κρέας] §~159
\item[κείμενον] §~82; κεῖμαι biti postavljen, biti položen; a. sg. s. r. ptc. prez. medpas. 
\item[ἀγαγοῦσα] ἄγω voditi, odvesti; n. sg. ž. r. ptc. aor. akt. §~90
\item[αὐτὸν] §~207
\item[ἔλεγεν] λέγω govoriti; 3. l. sg. impf. akt. 
\item[ὡς] zavisna izrična rečenica: da…
\item[εὑροῦσα] εὑρίσκω naći; n. sg. ž. r. ptc. aor. akt. §~90
\item[θησαυρὸν] §~82; objekt oba glagola, εὑροῦσα i οὐκ ἐχρήσατο
\item[αὐτὴ] §~207
\item[οὐκ ἐχρήσατο] χράομαι iskoristiti; 3. l. sg. ind. aor. med. 
\item[αὐτὴ μὲν\dots\ γέρας δὲ\dots] koordinacija rečeničnih članova (ovdje izražavaju suprotne misli): a\dots
\item[γέρας] §~159
\item[αὐτῷ] §~207
\item[τῆς βασιλείας] §~80, §~90
\item[τετήρηκε] τηρέω sačuvati, 3. l. sg. ind. perf. akt.
\item[παρῄνει] παραινέω preporučiti; 3. l. sg. impf. akt. 
\item[λαβεῖν] λαμβάνω uzeti; inf. aor. akt.
\end{description}

%3 itd

{\large
\noindent \uuline{τοῦ} δὲ ἀμελετήτως \uuline{ἐπελθόντος} \\
καὶ \\
ὑπὸ τῆς παγίδος \uuline{συλληφθέντος} \\
\uuline{αἰτιωμένου} τε τὴν ἀλώπεκα \\
\tabto{2em} ὡς ἐνεδρεύσασαν \\
αὐτῷ ἐκείνη ἔφη·\\

}

\begin{description}[noitemsep]

\item[τοῦ δὲ ἐπελθόντος] ἐπέρχομαι približiti se; g. sg. m. r. ptc. aor. akt. §~138; GA odgovara nekoj od zavisnih adverbnih rečenica u kojoj imenica u genitivu odgovara subjektu, a particip u genitivu predikatu u vremenu participa: kad\dots
\item[ὑπὸ τῆς παγίδος ] §~123, prijedložni izraz, ὑπὸ + g.: od\dots; uz glagole u pasivu izražava vršitelja radnje
\item[συλληφθέντος] συλλαμβάνω zgrabiti, zarobiti; g. sg. m. r. ptc. aor. pas. §~139.β
\item[αἰτιωμένου] αἰτιάομαι kriviti; g. sg. m. r. ptc. prez. med. (stegnuti)
\item[τὴν ἀλώπεκα] §~115
\item[ἐνεδρεύσασαν] ἐνεδρεύω namamiti u klopku; a. sg. ž. r. ptc. aor. med. §~92
\item[ὡς ἐνεδρεύσασαν] ὡς + particip zamjenjuje izričnu rečenicu: da\dots
\item[αὐτῷ] §~207
\item[ἐκείνη] §~213.3, §~438.2
\item[ἔφη] φημί reći; 3. l. sg. impf. akt. 
\end{description}

%4

{\large
\noindent ``ὦ πίθηκε, \\
\tabto{2em} σὺ δὲ \\
\tabto{4em} τοιαύτην ψυχὴν ἔχων \\
\tabto{2em} τῶν ἀλόγων ζῴων \\
\tabto{2em} βασιλεύεις;''\\

}

\begin{description}[noitemsep]

\item[σὺ\dots\ βασιλεύεις;] upitna rečenica
\item[ὦ πίθηκε] §~80, §~82
\item[σὺ] §~205-206
\item[δὲ] čestica povezuje rečenice i ističe suprotnost: a\dots
\item[τοιαύτην ψυχὴν] §~213.4, §~90
\item[ἔχων] ἔχω imati; n. sg. m. r. ptc. prez. akt.
\item[τῶν ἀλόγων ζῴων] §~80, §~103, §~82
\item[βασιλεύεις] βασιλεύω vladati, rekcija τινος nad nekim; 2. l. sg. ind. prez. akt.
\end{description}


%5

{\large
\noindent οὕτως \\
οἱ \\
\tabto{2em} τοῖς πράγμασιν \\
\tabto{2em} ἀπερισκέπτως \\
ἐπιχειροῦντες \\
\tabto{2em} πρὸς τῷ δυστυχεῖν \\
καὶ γέλωτα \\
ὀφλισκάνουσιν.\\

}

\begin{description}[noitemsep]

\item[οἱ ἐπιχειροῦντες ] ἐπιχειρέω τινί naumiti nešto; n. pl. m. r. ptc. prez. akt.; supstantiviranje participa članom §~499
\item[τοῖς πράγμασιν] §~123
\item[πρὸς τῷ δυστυχεῖν] δυστυχέω biti nesretan; inf. prez. akt., supstantiviranje infinitiva članom §~497; prijedložni izraz, πρὸς + d.: osim\dots; §~418, §~435 %dodao
\item[γέλωτα] §~130
\item[ὀφλισκάνουσιν] ὀφλισκάνω zaslužiti, γέλωτα ὀφλισκάνω izlagati se poruzi; 3.~l.~pl. ind. prez. akt.
\end{description}




%kraj
