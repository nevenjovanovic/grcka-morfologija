% Unesi korekture NČ 2019-08-16
\section*{O autoru}

Rimski car Marko Aurelije (Marcus Aurelius, \textgreek[variant=ancient]{Μάρκος Αὐρήλιος}) rođen je 121.\ po Kr.\ u Rimu. Vladao je od 161. do smrti 180. Učitelji su mu bili Fronton i Herod Atički te stoici Junije Arulen Rustik i Apolonije iz Halkedona. Vladavinu Marka Aurelija obilježili su sukobi u Britaniji (162.), Germaniji i na dunavskom limesu (167.\ – 168.), Egiptu (172.) te višegodišnji rat s Partima (162.\ – 166.). Uspješno se suprotstavio prvom valu seobe naroda. Nasljednikom je proglasio sina Komoda te time prekinuo niz tzv.\ adoptivnih careva (Trajan – Hadrijan – Antonin Pio). 

Filozofski obrazovan, uz Seneku i Epikteta Marko Aurelije najpoznatiji je predstavnik kasne stoe.

Sačuvana su njegova pisma te filozofski spis \textit{Razgovori sa samim sobom} \textgreek[variant=ancient]{(Tὰ εἰς ἑαυτόν),} sastavljen na grčkom, kojim se car služio jednako kao i latinskim (dok je za njega latinski bio jezik poslovanja, upravljanja i donošenja odluka, grčki je bio jezik intimnih razmišljanja).


\section*{O tekstu}

Djelo \textgreek[variant=ancient]{Tὰ εἰς ἑαυτόν} pisano je jednostavnim jezikom bliskim svakodnevnom govoru tog doba \textgreek[variant=ancient]{(κοινή).} Autor mu vjerojatno nije sam dao naslov. U djelu nije sustavno prikazao cjelokupnost stoičke filozofije, već stoicizam kao način života, kao što su učinili i Seneka i Epiktet. Neke od tema kojima se bavi jesu božja providnost, kratkoća ljudskog života, tolerancija, zajedništvo svih ljudi, obveza svih da rade za opće dobro, dužnost i moralna odgovornost prema drugima.

U odabranom odlomku Marko Aurelije razmišlja o smrti i argumentira zašto ona nije strašna, kao što ni sve ostalo što se događa ljudima – i ono što se smatra dobrim i ono što se smatra lošim – zapravo nije ni dobro ni loše.

\section*{Pročitajte naglas grčki tekst.}

M. Aur. Ad se ipsum 2.11

%Naslov prema izdanju

\medskip


{\large
{ 
\begin{greek}

\noindent  Ὡς ἤδη δυνατοῦ ὄντος ἐξιέναι τοῦ βίου, οὕτως ἕκαστα ποιεῖν καὶ λέγειν καὶ διανοεῖσθαι. τὸ δὲ ἐξ ἀνθρώπων ἀπελθεῖν, εἰ μὲν θεοὶ εἰσίν, οὐδὲν δεινόν· κακῷ γάρ σε οὐκ ἂν περιβάλοιεν· εἰ δὲ ἤτοι οὐκ εἰσὶν ἢ οὐ μέλει αὐτοῖς τῶν ἀνθρωπείων, τί μοι ζῆν ἐν κόσμῳ κενῷ θεῶν ἢ προνοίας κενῷ; ἀλλὰ καὶ εἰσὶ καὶ μέλει αὐτοῖς τῶν ἀνθρωπείων καὶ τοῖς μὲν κατ' ἀλήθειαν κακοῖς ἵνα μὴ περιπίπτῃ ὁ ἄνθρωπος, ἐπ' αὐτῷ τὸ πᾶν ἔθεντο· τῶν δὲ λοιπῶν εἴ τι κακὸν ἦν, καὶ τοῦτο ἂν προείδοντο, ἵνα ἐπὶ παντὶ ᾖ τὸ μὴ περιπίπτειν αὐτῷ. (ὃ δὲ χείρω μὴ ποιεῖ ἄνθρωπον, πῶς ἂν τοῦτο βίον ἀνθρώπου χείρω ποιήσειεν;) οὔτε δὲ κατ' ἄγνοιαν οὔτε εἰδυῖα μέν, μὴ δυναμένη δὲ προφυλάξασθαι ἢ διορθώσασθαι ταῦτα ἡ τῶν ὅλων φύσις παρεῖδεν ἄν, οὔτ' ἂν τηλικοῦτον ἥμαρτεν ἤτοι παρ' ἀδυναμίαν ἢ παρ' ἀτεχνίαν, ἵνα τὰ ἀγαθὰ καὶ τὰ κακὰ ἐπίσης τοῖς τε ἀγαθοῖς ἀνθρώποις καὶ τοῖς κακοῖς πεφυρμένως συμβαίνῃ. θάνατος δέ γε καὶ ζωή, δόξα καὶ ἀδοξία, πόνος καὶ ἡδονή, πλοῦτος καὶ πενία, πάντα ταῦτα ἐπίσης συμβαίνει ἀνθρώπων τοῖς τε ἀγαθοῖς καὶ τοῖς κακοῖς, οὔτε καλὰ ὄντα οὔτε αἰσχρά. οὔτ' ἄρ' ἀγαθὰ οὔτε κακά ἐστι.


\end{greek}


}
}


\section*{Analiza i komentar}

%1

{\large
\begin{greek}
\noindent ῾Ως ἤδη \\
\tabto{2em} \uuline{δυνατοῦ ὄντος} \\
\tabto{4em} ἐξιέναι τοῦ βίου, \\
οὕτως \\
\tabto{2em} ἕκαστα \\
\tabto{4em} ποιεῖν καὶ λέγειν καὶ διανοεῖσθαι.\\

\end{greek}
}

\begin{description}[noitemsep]
\item[῾Ως] §~221
\item[δυνατοῦ] §~103, otvara mjesto dopuni u infinitivu
\item[ὄντος] εἰμί biti; g. sg. m. r. ptc. prez. akt.
\item[ἐξιέναι] ἔξειμι τινός izaći iz čega, napustiti što; inf. prez. akt.
\item[τοῦ βίου ] §~82
\item[οὕτως] §~221
\item[ἕκαστα] §~103
\item[ποιεῖν] ποιέω činiti; inf. prez. akt., ima vrijednost 2. l. sg. impt. prez. akt.
\item[λέγειν] λέγω govoriti; inf. prez. akt., ima vrijednost 2. l. sg. impt. prez. akt.
\item[διανοεῖσθαι] διανοέομαι misliti; inf. prez. medpas., ima vrijednost 2. l. sg. impt. prez. medpas.
\end{description}

%2

{\large
\begin{greek}
\noindent τὸ δὲ ἐξ ἀνθρώπων ἀπελθεῖν, \\
\tabto{2em} εἰ μὲν θεοὶ εἰσίν, \\
οὐδὲν δεινόν·\\
\tabto{2em} εἰ δὲ ἤτοι οὐκ εἰσὶν \\
\tabto{2em} ἢ οὐ μέλει αὐτοῖς \\
\tabto{4em} τῶν ἀνθρωπείων, \\
τί μοι ζῆν \\
\tabto{2em} ἐν κόσμῳ κενῷ θεῶν \\
\tabto{2em} ἢ προνοίας κενῷ;\\

\end{greek}
}

\begin{description}[noitemsep]
\item[τὸ\dots\ ἀπελθεῖν] ἀπέρχομαι otići; inf. aor. akt.; poimeničeni infinitiv §~497
\item[δὲ] čestica povezuje rečenicu s prethodnom: a\dots
\item[ἐξ ἀνθρώπων] §~82; ἐξ + g. §~424
\item[εἰ ] §~518; veznik uvodi realnu pogodbenu rečenicu
\item[εἰ μὲν\dots\ εἰ δὲ\dots] koordinacija dviju pogodbenih rečenica parom čestica
\item[θεοὶ] §~82
\item[εἰσίν] εἰμί \textit{ovdje} postojati; 3. l. pl. ind. prez. akt. 
\item[οὐδὲν] §~224.2
\item[δεινόν] podrazumijeva se kopula; §~103
\item[εἰσὶν ] εἰμί postojati; 3. l. pl. ind. prez. akt.
\item[μέλει] μέλει τινί τινος (bezlično) netko se brine za što; 3. l. sg. ind. prez. akt.
\item[αὐτοῖς] §~207
\item[τῶν ἀνθρωπείων] §~103; poimeničenje članom §~373
\item[τί] §~217
\item[μοι] §~205
\item[ζῆν] ζάω živjeti; inf. prez. akt.; stezanje §~244
\item[ἐν κόσμῳ ] §~82; ἐν + d. §~426
\item[κενῷ] §~103
\item[θεῶν] §~82; \textit{genitivus copiae et inopiae} §~403
\item[προνοίας] §~97; \textit{genitivus copiae et inopiae} §~403
\item[κενῷ] §~103
\end{description}

%3

{\large
\begin{greek}
\noindent κακῷ γάρ \\
σε \\
οὐκ ἂν περιβάλοιεν· \\
ἀλλὰ καὶ εἰσὶ \\
καὶ μέλει αὐτοῖς \\
\tabto{2em} τῶν ἀνθρωπείων \\
\tabto{2em} καὶ τοῖς μὲν \\
\tabto{4em} κατ' ἀλήθειαν κακοῖς \\
\tabto{2em} ἵνα μὴ περιπίπτῃ ὁ ἄνθρωπος, \\
\tabto{2em} ἐπ' αὐτῷ \\
\tabto{4em} τὸ πᾶν ἔθεντο· \\
\tabto{2em} τῶν δὲ λοιπῶν \\
\tabto{2em} εἴ τι κακὸν ἦν, \\
\tabto{2em} καὶ τοῦτο ἂν προείδοντο, \\
\tabto{2em} ἵνα ἐπὶ παντὶ ᾖ \\
\tabto{4em} τὸ μὴ περιπίπτειν αὐτῷ. \\

\end{greek}
}

\begin{description}[noitemsep]
\item[κακῷ] §~103
\item[σε] §~205
\item[ἂν] čestica uz optativ §~464.2, §~489.b
\item[περιβάλοιεν] περιβάλλω τινά okružiti koga; 3. l. pl. opt. aor. akt.
\item[εἰσὶ ] εἰμί \textit{ovdje} postojati; 3. l. pl. ind. prez. akt.
\item[μέλει] μέλει τινί τινος (bezlično) netko se brine za što; 3. l. sg. ind. prez. akt.
\item[αὐτοῖς] §~207
\item[τῶν ἀνθρωπείων] §~103; poimeničenje članom §~373
\item[κατ' ἀλήθειαν ] §~97; κατά + a. §~429.b; elizija §~68
\item[τοῖς κακοῖς] §~103; poimeničenje članom §~373
\item[τοῖς μὲν\dots\ τῶν δὲ λοιπῶν\dots] koordinacija rečeničnih članova parom čestica: a\dots
\item[ἵνα ] veznik uvodi namjernu rečenicu §~470 
\item[περιπίπτῃ] περιπίπτω τινί pasti, upasti ili dospjeti u što; 3. l. sg. konj. prez. akt.
\item[ὁ ἄνθρωπος] §~82
\item[ἐπ' αὐτῷ] §~207; elizija §~68
\item[τὸ πᾶν] §~193; poimeničenje članom §~373
\item[ἔθεντο] τίθημι staviti; 3. l. pl. aor. med.
\item[τῶν λοιπῶν] §~103; dijelni genitiv §~395; poimeničenje članom §~373
\item[εἴ ] veznik uvodi irealnu pogodbenu rečenicu §~478 
\item[τι] §~217
\item[κακὸν] §~103
\item[ἦν] εἰμί 3. l. sg. impf. akt.
\item[τοῦτο] §~213.2
\item[προείδοντο] προοράω predvidjeti; 3. l. ind. aor. med., §~489.b
\item[ἂν] čestica uz ireal §~489.b
\item[ἵνα] uvodi namjernu rečenicu §~470
\item[ᾖ] εἰμί biti; 3. l. sg. konj. prez. akt.
\item[ἐπὶ παντὶ] §~193, ἐπὶ + d. §~436.b
\item[τὸ μὴ περιπίπτειν] περιπίπτω τινί upasti u što; inf. prez. akt., poimeničeni infinitiv §~497, negiran negacijom μὴ
\item[αὐτῷ] §~207
\end{description}


%4

{\large
\begin{greek}
\noindent ὃ δὲ \\
χείρω \\
\tabto{2em} μὴ ποιεῖ \\
ἄνθρωπον, \\
πῶς ἂν τοῦτο \\
\tabto{2em} βίον ἀνθρώπου \\
\tabto{2em} χείρω ποιήσειεν;\\

\end{greek}
}

\begin{description}[noitemsep]
\item[ὃ] ono što\dots; §~215
\item[δὲ] čestica povezuje rečenicu s prethodnom: a\dots
\item[χείρω] §~202, §~137
\item[ποιεῖ] ποιέω činiti; 3. l. sg. ind. prez. akt.
\item[ἄνθρωπον] §~82
\item[πῶς] §~221
\item[τοῦτο] §213.2
\item[βίον] §~82
\item[ἀνθρώπου] §~82
\item[ἂν] čestica uz optativ, §~464.2, §~489.b
\item[χείρω] §~202, §~137
\item[ποιήσειεν] ποιέω činiti; 3. l. sg. opt. aor. akt.
\end{description}


%5

{\large
\begin{greek}
\noindent οὔτε δὲ \\
\tabto{2em} κατ' ἄγνοιαν \\
\tabto{4em} οὔτε εἰδυῖα μέν, \\
\tabto{4em} μὴ δυναμένη δὲ \\
\tabto{6em} προφυλάξασθαι ἢ διορθώσασθαι \\
\tabto{8em} ταῦτα \\
\tabto{4em} ἡ τῶν ὅλων φύσις \\
\tabto{4em} παρεῖδεν ἄν, \\
\tabto{4em} οὔτ' ἂν τηλικοῦτον ἥμαρτεν ἤτοι \\
\tabto{6em} παρ' ἀδυναμίαν \\
\tabto{6em} ἢ παρ' ἀτεχνίαν, \\
\tabto{4em} ἵνα τὰ ἀγαθὰ καὶ τὰ κακὰ \\
\tabto{6em} ἐπίσης \\
\tabto{8em} τοῖς τε ἀγαθοῖς ἀνθρώποις \\
\tabto{8em} καὶ τοῖς κακοῖς \\
\tabto{6em} πεφυρμένως συμβαίνῃ.\\

\end{greek}
}

\begin{description}[noitemsep]
\item[οὔτε\dots\ οὔτε\dots\ οὔτ'] koordinacija sastavnim veznicima §~513.4; elizija §~68
\item[δὲ] čestica povezuje rečenicu s prethodnom: a\dots
\item[κατ' ἄγνοιαν ] §~97; κατά + a. §~429.b; elizija §~68
\item[εἰδυῖα] οἶδα znati; n. sg. ž. r. ptc. perf. akt., perfekt sa značenjem prezenta, §~317
\item[εἰδυῖα μέν\dots\ μὴ δυναμένη δὲ\dots] koordinacija rečeničnih članova parom čestica: a\dots
\item[δυναμένη] δύναμαι moći; n. sg. ž. r. ptc. prez. medpas.; otvara mjesto dopuni u infinitivu
\item[προφυλάξασθαι] προφυλάσσω čuvati; inf. aor. med.
\item[διορθώσασθαι] διορθόω ispravljati; inf. aor. med.
\item[ταῦτα] §~213.2
\item[τῶν ὅλων ] §~103; poimeničenje članom §~373
\item[ἡ φύσις] §~165
\item[παρεῖδεν] παροράω previdjeti;  3. l. sg. ind. aor. akt.
\item[ἄν] čestica uz ireal, §~489.b
\item[τηλικοῦτον] §~219
\item[ἄν] čestica uz ireal, §~489.b
\item[ἥμαρτεν] ἁμαρτάνω promašiti, pogriješiti; 3. l. sg. ind. aor. akt.
\item[παρ' ἀδυναμίαν] §~90; παρά + a. §~434.C; elizija §~68
\item[παρ' ἀτεχνίαν] §~90; παρά + a. §~434.C; elizija §~68
\item[ἵνα] veznik uvodi namjernu rečenicu §~470
\item[συμβαίνῃ] συμβαίνω \textit{bezlično} dogoditi se; 3. l. sg. konj. prez. akt.
\item[τὰ ἀγαθὰ ] §~103; poimeničenje članom §~373
\item[τὰ κακὰ] §~103; poimeničenje članom §~373
\item[τοῖς ἀνθρώποις] §~82
\item[ἀγαθοῖς] §~103
\item[τοῖς κακοῖς] §~103; poimeničenje članom §~373
\item[πεφυρμένως] §~204
\end{description}


%6

{\large
\begin{greek}
\noindent θάνατος δέ γε καὶ ζωή, \\
δόξα καὶ ἀδοξία, \\
πόνος καὶ ἡδονή, \\
πλοῦτος καὶ πενία, \\
\tabto{2em} πάντα ταῦτα \\
\tabto{4em} ἐπίσης συμβαίνει \\
\tabto{6em} ἀνθρώπων \\
\tabto{4em} τοῖς τε ἀγαθοῖς καὶ τοῖς κακοῖς, \\
\tabto{2em} οὔτε καλὰ ὄντα \\
\tabto{2em} οὔτε αἰσχρά.\\

\end{greek}
}

\begin{description}[noitemsep]
\item[θάνατος] §~82
\item[δὲ] čestica povezuje rečenicu s prethodnom: a\dots
\item[ζωή] §~90
\item[δόξα] §~97
\item[ἀδοξία] §~90
\item[πόνος] §~82
\item[ἡδονή] §~90
\item[πλοῦτος] §~82
\item[πενία] §~90
\item[πάντα] §~193
\item[ταῦτα] §~213.2
\item[συμβαίνει] συμβαίνει τινί (bezlično) dogoditi se komu; 3. l. sg. ind. prez. akt.
\item[ἀνθρώπων] §~82, dijelni genitiv §~395.1
\item[τοῖς ἀγαθοῖς ] §~103; poimeničenje članom, §~373
\item[τοῖς κακοῖς] §~103; poimeničenje članom, §~373
\item[τε… καὶ] ne samo\dots\ nego i\dots; koordinacija sastavnim veznicima §~513.2
\item[οὔτε\dots\  οὔτε\dots] koordinacija sastavnim veznicima §~513.4
\item[καλὰ] §~103
\item[ὄντα] εἰμί biti; n. pl. s. r. ptc. prez. akt.
\item[αἰσχρά] §~103 
\end{description}

%7

{\large
\begin{greek}
\noindent οὔτ' ἄρ' ἀγαθὰ \\
οὔτε κακά \\
\tabto{2em} ἐστι.\\

\end{greek}
}

\begin{description}[noitemsep]
\item[οὔτ'\dots\ οὔτε\dots] koordinacija sastavnim veznicima §~513.4; elizija §~68
\item[ἄρ'] §~516.1 zaključni veznik; elizija §~68
\item[ἀγαθὰ] §~103
\item[κακά] §~103
\item[ἐστι] εἰμί biti; 3. l. sg. ind. prez. akt.
\end{description}



%kraj
