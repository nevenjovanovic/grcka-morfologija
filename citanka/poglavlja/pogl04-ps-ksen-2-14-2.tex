%\section*{O autoru}



\section*{O tekstu}
Ἀθηναίων πολιτεία (\textit{Ustav atenski}) u rukopisima je preneseno kao jedno od Ksenofontovih djela. Već od Diogena Laertija, a osobito od XIX.~stoljeća, Ksenofontovo je autorstvo prijeporno, te danas nepoznatog autora nazivamo ``Pseudo-Ksenofont'' ili, u angloameričkoj tradiciji, ``stari oligarh'' – ``stari'' zbog toga što se, po svemu sudeći, radi o najstarijem poznatom nam atičkom proznom djelu, a ne zbog autorove dobi.

Naslov sugerira da se radi o sustavnom prikazu organizacije atenske države, kakav će kasnije napisati Aristotel (a sam je Ksenofont autor jednako sustavnog \textit{Ustava lakedemonskog}), ali zapravo nije tako. Pseudo-Ksenofontovo je djelo kritička refleksija o političkom ustroju Atene iz doba Peloponeskog rata  (431.–404.\ pr.~Kr) . Razmatranje prednosti i nedostataka demokracije služi kao nekonvencionalan izazov, provokacija gospodarski i vojno manje sposobnim aristokratima, možda pripadnicima nekog ekskluzivnog ``kluba'' \textgreek[variant=ancient]{(ἑταιρεία).}

U drugom od tri poglavlja djela autor je nabrajao prednosti ``gospodarenja morem'' \textgreek[variant=ancient]{(θαλασσοκρατία);} sad ističe jedini nedostatak Atene kao gospodarice mora.


\newpage

\section*{Pročitajte naglas grčki tekst.}
Ps.-Xen.\ Atheniensium respublica 2.14.2
%Naslov prema izdanju

\medskip

{\large
\begin{greek}
\noindent Eἰ γὰρ νῆσον οἰκοῦντες θαλασσοκράτορες ἦσαν ᾿Αθηναῖοι, ὑπῆρχεν ἂν αὐτοῖς ποιεῖν μὲν κακῶς, εἰ ἐβούλοντο, πάσχειν δὲ μηδέν, ἕως τῆς θαλάττης ἦρχον, μηδὲ τμηθῆναι τὴν ἑαυτῶν γῆν μηδὲ προσδέχεσθαι τοὺς πολεμίους· νῦν δὲ οἱ γεωργοῦντες καὶ οἱ πλούσιοι ᾿Αθηναίων ὑπέρχονται τοὺς πολεμίους μᾶλλον, ὁ δὲ δῆμος, ἅτε εὖ εἰδὼς ὅτι οὐδὲν τῶν σφῶν ἐμπρήσουσιν οὐδὲ τεμοῦσιν, ἀδεῶς ζῇ καὶ οὐχ ὑπερχόμενος αὐτούς. πρὸς δὲ τούτοις καὶ ἑτέρου δέους ἀπηλλαγμένοι ἂν ἦσαν, εἰ νῆσον ᾤκουν, μηδέποτε προδοθῆναι τὴν πόλιν ὑπ' ὀλίγων μηδὲ πύλας ἀνοιχθῆναι μηδὲ πολεμίους ἐπεισπεσεῖν· πῶς γὰρ νῆσον οἰκούντων ταῦτ' ἂν ἐγίγνετο; μηδ' αὖ στασιάσαι τῷ δήμῳ μηδέν, εἰ νῆσον ᾤκουν· νῦν μὲν γὰρ εἰ στασιάσαιεν, ἐλπίδα ἂν ἔχοντες ἐν τοῖς πολεμίοις στασιάσειαν, ὡς κατὰ γῆν ἐπαξόμενοι· εἰ δὲ νῆσον ᾤκουν, καὶ ταῦτ' ἂν ἀδεῶς εἶχεν αὐτοῖς. ἐπειδὴ οὖν ἐξ ἀρχῆς οὐκ ἔτυχον οἰκήσαντες νῆσον, νῦν τάδε ποιοῦσι· τὴν μὲν οὐσίαν ταῖς νήσοις παρατίθενται, πιστεύοντες τῇ ἀρχῇ τῇ κατὰ θάλατταν, τὴν δὲ ᾿Αττικὴν γῆν περιορῶσι τεμνομένην, γιγνώσκοντες ὅτι εἰ αὐτὴν ἐλεήσουσιν, ἑτέρων ἀγαθῶν μειζόνων στερήσονται.

\end{greek}
}
\newpage


\section*{Analiza i komentar}
{\large
\begin{greek}
\noindent Eἰ γὰρ\\
νῆσον οἰκοῦντες\\
θαλασσοκράτορες ἦσαν \\
᾿Αθηναῖοι, \\
ὑπῆρχεν ἂν \\
\tabto{2em} αὐτοῖς \\
\tabto{2em} ποιεῖν μὲν κακῶς, \\
\tabto{4em} εἰ ἐβούλοντο, \\
\tabto{2em} πάσχειν δὲ μηδέν, \\
\tabto{4em} ἕως \\
\tabto{6em} τῆς θαλάττης \\
\tabto{4em} ἦρχον, \\
\tabto{4em} μηδὲ \\
\tabto{6em} \underline{τμηθῆναι} \\
\tabto{6em} \underline{τὴν} ἑαυτῶν \underline{γῆν} \\
\tabto{4em} μηδὲ \\
\tabto{6em} προσδέχεσθαι \\
\tabto{6em} τοὺς πολεμίους· \\
νῦν δὲ \\
οἱ γεωργοῦντες \\
\tabto{2em} καὶ οἱ πλούσιοι ᾿Αθηναίων \\
ὑπέρχονται \\
\tabto{2em} τοὺς πολεμίους \\
μᾶλλον, \\
ὁ δὲ δῆμος, \\
\tabto{2em} ἅτε \\
\tabto{2em} εὖ εἰδὼς \\
\tabto{4em} ὅτι οὐδὲν τῶν σφῶν \\
\tabto{4em} ἐμπρήσουσιν \\
\tabto{6em} οὐδὲ τεμοῦσιν, \\
ἀδεῶς ζῇ \\
\tabto{2em} καὶ οὐχ ὑπερχόμενος \\
\tabto{4em} αὐτούς.\\

\end{greek}
}

\begin{description}[noitemsep]
\item[γὰρ ] čestica najavljuje iznošenje dokaza prethodne tvrdnje: naime\dots
\item[νῆσον] §~82
\item[οἰκοῦντες ] οἰκέω stanovati, nastavati; n. pl. m. r. ptc. prez. akt.
\item[θαλασσοκράτορες ] §~146; imenica kao dio imenskog predikata Smyth 910
\item[ἦσαν ] εἰμί biti; 3. l. pl. impf. akt.
\item[᾿Αθηναῖοι] §~106
\item[Eἰ\dots\ θαλασσοκράτορες ἦσαν\dots\ ὑπῆρχεν ἂν ] ὑπάρχει τινί τι nešto je moguće nekome; 3. l. sg. impf. prez. akt.; irealni oblik pogodbene rečenice, §~478
\item[αὐτοῖς ] §~207
\item[ποιεῖν μὲν\dots\, πάσχειν δὲ] koordinacija pomoću čestica μὲν\dots\ δὲ: \dots\ a\dots
\item[ποιεῖν] ποιέω činiti; inf. prez. akt.
\item[κακῶς] §~204
\item[ὑπῆρχεν ἂν\dots\ εἰ ἐβούλοντο] βούλομαι htjeti; 3. l. pl. impf. (medpas.); irealni oblik pogodbene rečenice, §~478
\item[πάσχειν] πάσχω trpjeti; inf. prez. akt.
\item[μηδέν] §~224.2
\item[τῆς θαλάττης ] §~90
\item[ἦρχον] ἄρχω τινός vladati nečim; 3. l. pl. impf. akt.
\item[μηδὲ\dots\ μηδὲ\dots] koordinacija: niti\dots\ niti\dots
\item[τμηθῆναι] τέμνω sjeći, \textit{ovdje} pustošiti, LSJ s.~v.\ IV.3; inf. aor. pas.
\item[τὴν ἑαυτῶν γῆν] §~211.2.c; §~108.b; §~375
\item[προσδέχεσθαι] προσδέχομαι iščekivati, bojati se; inf. prez. (medpas.)
\item[τοὺς πολεμίους] §~106; §~373
\item[Eἰ γὰρ νῆσον οἰκοῦντες\dots\ νῦν δὲ\dots] koordinacija pomoću čestice δὲ, bez μὲν
\item[οἱ γεωργοῦντες] γεωργέω obrađivati zemlju, biti zemljoradnik; n. pl. m. r. ptc. prez. akt. §~373
\item[οἱ πλούσιοι ] §~106; §~373
\item[᾿Αθηναίων ] §~106
\item[ὑπέρχονται ] ὑπέρχομαί τινα umiljavati se nekome; 3. l. pl. ind. prez. (med.)
\item[τοὺς πολεμίους ] §~106; §~373
\item[μᾶλλον] §~204.3
\item[οἱ πλούσιοι\dots\ ὁ δὲ δῆμος] koordinacija pomoću čestice δὲ, bez μὲν
\item[δῆμος] §~82
\item[εἰδὼς ] οἶδα znati; n. sg. m. r. ptc. perf. akt.
\item[οὐδὲν ] §~224.2
\item[τῶν σφῶν ] §~106; §~373
\item[ἐμπρήσουσιν] ἐμπίμπρημι paliti; 3. l. pl. ind. fut. akt.
\item[τεμοῦσιν] τέμνω sjeći; 3. l. pl. ind. fut. akt.
\item[ἀδεῶς ] §~204
\item[ζῇ] ζάω živjeti; 3. l. sg. ind. prez. akt.
\item[ὑπερχόμενος] ὑπέρχομαί τινα umiljavati se nekome; n. sg. m. r. ptc. prez. (medpas.)
\item[αὐτούς] §~207

\end{description}


{\large
\begin{greek}
\noindent πρὸς δὲ τούτοις \\
καὶ ἑτέρου δέους \\
ἀπηλλαγμένοι ἂν ἦσαν, \\
\tabto{2em} εἰ νῆσον ᾤκουν, \\
μηδέποτε \underline{προδοθῆναι τὴν πόλιν} \\
\tabto{2em} ὑπ' ὀλίγων \\
μηδὲ \underline{πύλας ἀνοιχθῆναι} \\
μηδὲ \underline{πολεμίους ἐπεισπεσεῖν}· \\
πῶς γὰρ \\
\tabto{2em} νῆσον \uuline{οἰκούντων} \\
ταῦτ' ἂν ἐγίγνετο; \\
μηδ' αὖ \underline{στασιάσαι} \\
\tabto{2em} τῷ δήμῳ \\
μηδέν, \\
\tabto{2em} εἰ νῆσον ᾤκουν· \\
νῦν μὲν γὰρ \\
\tabto{2em} εἰ στασιάσαιεν, \\
ἐλπίδα ἂν ἔχοντες \\
\tabto{2em} ἐν τοῖς πολεμίοις \\
στασιάσειαν, \\
ὡς \\
\tabto{2em} κατὰ γῆν \\
ἐπαξόμενοι· \\
εἰ δὲ νῆσον ᾤκουν, \\
καὶ ταῦτ' ἂν \\
\tabto{2em} ἀδεῶς \\
εἶχεν \\
\tabto{2em} αὐτοῖς. \\

\end{greek}
}

\begin{description}[noitemsep]
\item[πρὸς δὲ τούτοις] §~435; čestica δέ povezuje rečenicu s prethodnom: a\dots; §~213.2
\item[ἑτέρου ] §~106
\item[δέους ] §~153
\item[ἀπηλλαγμένοι ἂν ἦσαν εἰ\dots\ ᾤκουν] ἀπαλλάσσω τινός osloboditi nečega; 3. l. pl. plpf. medpas.; οἰκέω stanovati, nastavati; 3. l. pl. impf. akt.; irealni oblik pogodbene rečenice, §~478
\item[νῆσον] §~82
\item[δέους\dots] \textbf{μηδέποτε προδοθῆναι\dots\ μηδὲ ἀνοιχθῆναι\dots\ μηδὲ ἐπεισπεσεῖν\dots\ μηδ' αὖ στασιάσαι\dots} uz δέους (bojazan od nečega) kao obavezna dopuna stoje akuzativi s infinitivom ili sami infinitivi
\item[προδοθῆναι ] προδίδωμι izdati; inf. aor. pas.
\item[τὴν πόλιν ] §~165
\item[ὑπ' ὀλίγων] §~68; §~106; οἱ ὀλίγοι oligarsi §~371. bilj. 4
\item[πύλας ] §~90
\item[ἀνοιχθῆναι] ἀνοίγνυμι otvoriti; inf. aor. pas.
\item[πολεμίους ] §~106
\item[ἐπεισπεσεῖν] ἐπεισπίπτω upasti; inf. aor. akt.
\item[γὰρ ] čestica najavljuje nastavak prethodnog razmišljanja: naime\dots
\item[νῆσον] §~82
\item[οἰκούντων ] οἰκέω stanovati, nastavati; g. pl. m. r. ptc. prez. akt.
\item[ταῦτ' ἂν] §~68; §~213.2
\item[ἂν ἐγίγνετο] γίγνομαι dogoditi se; 3. l. sg. impf. (medpas.)
\item[μηδ' αὖ ] §~68
\item[στασιάσαι ] στασιάζω τινί buniti se protiv nekoga, suprotstavljati se nekome; inf. aor. akt.
\item[τῷ δήμῳ ] §~82
\item[μηδέν] §~224.2
\item[εἰ\dots\ ᾤκουν] οἰκέω stanovati, nastavati; 3. l. pl. impf. akt.; ponovljena protaza irealne pogodbene rečenice, već izrečena gore
\item[νῆσον] §~82
\item[νῦν μὲν γὰρ] \textbf{εἰ\dots\ εἰ δὲ νῆσον ᾤκουν\dots}\ koordinacija dviju pogodbenih rečenica česticama  μὲν\dots\  δὲ\dots
\item[εἰ στασιάσαιεν\dots\ ἂν \dots\ στασιάσειαν] potencijalna pogodbena rečenica, §~477
\item[στασιάσαιεν] στασιάζω buniti se; 3. l. pl. opt. aor. akt.
\item[ἐλπίδα ] §~123
\item[ἔχοντες ] ἔχω imati; n. pl. m. r. ptc. prez. akt.
\item[ἐν τοῖς πολεμίοις] §~426; §~106; §~373
\item[ἂν\dots\ στασιάσειαν] στασιάζω buniti se; 3. l. pl. opt. aor. akt.
\item[ὡς\dots\ ἐπαξόμενοι] ἐπάγω dovesti; n. pl. m. r. ptc. fut. med.; §~503.3
\item[κατὰ γῆν] §~429; §~108.b
\item[εἰ\dots\ ᾤκουν\dots\ ἂν\dots\ εἶχεν ] irealni oblik pogodbene rečenice, §~478
\item[νῆσον ] §~82
\item[ᾤκουν] οἰκέω stanovati, nastavati; 3. l. pl. impf. akt.
\item[ταῦτ' ἂν] §~68; §~213.2
\item[ἀδεῶς] §~204
\item[εἶχεν] ἔχω + prilog + τινι netko je u nekom stanju; bezlični oblik glagola s prilogom kao fraza; 3. l. sg. impf. akt.
\item[αὐτοῖς] §~207
\end{description}


%3 itd

{\large
\begin{greek}
\noindent ἐπειδὴ οὖν \\
\tabto{2em} ἐξ ἀρχῆς \\
οὐκ ἔτυχον \\
\tabto{2em} οἰκήσαντες νῆσον, \\
νῦν \\
τάδε \\
ποιοῦσι· \\
τὴν μὲν οὐσίαν \\
\tabto{2em} ταῖς νήσοις \\
παρατίθενται, \\
πιστεύοντες \\
\tabto{2em} τῇ ἀρχῇ τῇ κατὰ θάλατταν, \\
τὴν δὲ ᾿Αττικὴν γῆν \\
περιορῶσι \\
τεμνομένην, \\
γιγνώσκοντες ὅτι \\
\tabto{2em} εἰ αὐτὴν ἐλεήσουσιν, \\
ἑτέρων ἀγαθῶν μειζόνων \\
στερήσονται.\\

\end{greek}
}

\begin{description}[noitemsep]
\item[ἐξ ἀρχῆς] §~424; §~90
\item[ἔτυχον οἰκήσαντες ] τυγχάνω + predikatni ptc.: slučajno + glagol (u finitnom obliku); ἔτυχον 3. l. pl. ind. aor. akt.; οἰκέω stanovati, nastavati; οἰκήσαντες n. pl. m. r. ptc. aor. akt.
\item[νῆσον] §~82
\item[τάδε ] §~213.1
\item[ποιοῦσι] ποιέω činiti; 3. l. pl. ind. prez. akt.
\item[τὴν μὲν οὐσίαν\dots, τὴν δὲ ᾿Αττικὴν γῆν\dots] koordinacija česticama μὲν\dots, δὲ: \dots\ a\dots
\item[τὴν\dots\ οὐσίαν ] §~90
\item[ταῖς νήσοις ] §~82
\item[παρατίθενται] παρατίθημι med. položiti, dati na čuvanje; 3. l. pl. ind. prez. medpas.
\item[πιστεύοντες ] πιστεύω τινί vjerovati u nešto, uzdati se u nešto; n. pl. m. r. ptc. prez. akt.
\item[τῇ ἀρχῇ τῇ κατὰ θάλατταν] §~90; §~373; §~375
\item[τὴν\dots\ ᾿Αττικὴν γῆν ] §~106; §~108.b; §~375
\item[περιορῶσι ] περιοράω promatrati, \textit{ovdje} mirno gledati, pustiti; 3. l. pl. ind. prez. akt.
\item[τεμνομένην] τέμνω sjeći, pustošiti (v. gore); a. sg. ž. r. ptc. prez. medpas.
\item[γιγνώσκοντες] γιγνώσκω znati; n. pl. m. r. ptc. prez. akt.
\item[αὐτὴν ] §~207
\item[ἐλεήσουσιν] ἐλεέω τι žaliti za nečim; 3. l. pl. ind. fut. akt.
\item[ἑτέρων ἀγαθῶν ] §~106
\item[μειζόνων ] §~200
\item[στερήσονται] στερέω τινός izgubiti nešto; 3. l. pl. ind. fut. med.

\end{description}

%kraj
