% redigirao NZ, unio NJ 10. 5. 2019.
%\section*{O autoru}



\section*{O tekstu}

\textit{Kritija} se ubraja među Platonove kasne dijaloge, iz razdoblja nakon filozofova drugog puta na Siciliju (367.). Dijalog je trebao biti središnji dio trilogije kojem bi prethodio \textit{Timej}, a za njim bi slijedio \textit{Hermokrat}, koji vjerojatno nikad nije napisan; i \textit{Kritija} je ostao nedovršen. 

Uz Sokrata, u razgovoru sudjeluju Atenjanin Kritija, pripadnik zloglasne tridesetorice tirana, zatim Hermokrat, vođa oligarhijske stranke u Sirakuzi, te filozof Timej iz Lokra u Italiji, naslovni lik dijaloga \textit{Timej}.  \textit{Kritija} je posvećen slavi drevne Atene, njezinih građana, državnog ustroja i postignuća i po tome je sličan dijalogu \textit{Meneksen}. Dok u \textit{Meneksenu} Sokrat u pohvalnom govoru slavi Atenjane jer su odbili pokušaj Perzijanaca da pokore Grčku, ovdje drevna Atena, zamišljena kao idealna država, u borbi za opstanak pobjeđuje moćno svjetsko carstvo Atlantidu. Ta je sila prijetila da će pokoriti cijelu Europu i Aziju, no, udaljivši se od bogova, iznenada je potonula u ocean, i to prije devet tisuća godina. 

Platon, tvorac mita o Atlantidi, nije mogao ni slutiti da će ljudi još tisućljećima kasnije s istom revnošću s kojom na mapama upisuju putanju Odisejevih putovanja tragati za potonulim divovskim otokom s one strane Heraklovih stupova. 

Nakon što je rečeno da je moćna Atlantida mogla računati na obilje dobara iz inozemstva, u odabranom se odlomku opisuju prirodni resursi toga otoka.

\newpage

\section*{Pročitajte naglas grčki tekst.}

%Naslov prema izdanju

Plat.\ Critias 114e

\medskip

{\large
\begin{greek}
\noindent Πλεῖστα δὲ ἡ νῆσος αὐτὴ παρείχετο εἰς τὰς τοῦ βίου κατασκευάς, πρῶτον μὲν ὅσα ὑπὸ μεταλλείας ὀρυττόμενα στερεὰ καὶ ὅσα τηκτὰ γέγονε, καὶ τὸ νῦν ὀνομαζόμενον μόνον — τότε δὲ πλέον ὀνόματος ἦν τὸ γένος ἐκ γῆς ὀρυττόμενον ὀρειχάλκου κατὰ τόπους πολλοὺς τῆς νήσου, πλὴν χρυσοῦ τιμιώτατον ἐν τοῖς τότε ὄν — καὶ ὅσα ὕλη πρὸς τὰ τεκτόνων διαπονήματα παρέχεται, πάντα φέρουσα ἄφθονα, τά τε αὖ περὶ τὰ ζῷα ἱκανῶς ἥμερα καὶ ἄγρια τρέφουσα. καὶ δὴ καὶ ἐλεφάντων ἦν ἐν αὐτῇ γένος πλεῖστον· νομὴ γὰρ τοῖς τε ἄλλοις ζῴοις, ὅσα καθ' ἕλη καὶ λίμνας καὶ ποταμούς, ὅσα τ' αὖ κατ' ὄρη καὶ ὅσα ἐν τοῖς πεδίοις νέμεται, σύμπασιν παρῆν ἅδην, καὶ τούτῳ κατὰ ταὐτὰ τῷ ζῴῳ, μεγίστῳ πεφυκότι καὶ πολυβορωτάτῳ. πρὸς δὲ τούτοις, ὅσα εὐώδη τρέφει που γῆ τὰ νῦν, ῥιζῶν ἢ χλόης ἢ ξύλων ἢ χυλῶν στακτῶν εἴτε ἀνθῶν ἢ καρπῶν, ἔφερέν τε ταῦτα καὶ ἔτρεφεν εὖ.
\end{greek}

}

\section*{Analiza i komentar}

%1

{\large
\noindent Πλεῖστα δὲ \\
ἡ νῆσος αὐτὴ \\
παρείχετο \\
\tabto{2em} εἰς τὰς \\
\tabto{4em} τοῦ βίου\\
\tabto{2em} κατασκευάς, \\
πρῶτον μὲν \\
\tabto{2em} ὅσα \\
\tabto{4em} ὑπὸ μεταλλείας \\
\tabto{2em} ὀρυττόμενα στερεὰ \\
\tabto{2em} καὶ ὅσα \\
\tabto{2em} τηκτὰ γέγονε, \\
\tabto{2em} καὶ τὸ νῦν ὀνομαζόμενον μόνον —\\
\tabto{4em} τότε δὲ \\
\tabto{6em} πλέον ὀνόματος ἦν \\
\tabto{6em} τὸ γένος \\
\tabto{8em} ἐκ γῆς ὀρυττόμενον \\
\tabto{6em} ὀρειχάλκου \\
\tabto{8em} κατὰ τόπους πολλοὺς \\
\tabto{10em} τῆς νήσου, \\
\tabto{8em} πλὴν χρυσοῦ \\
\tabto{6em} τιμιώτατον \\
\tabto{8em} ἐν τοῖς τότε \\
\tabto{6em} ὄν —\\
\tabto{2em} καὶ ὅσα ὕλη \\
\tabto{4em} πρὸς τὰ \\
\tabto{6em} τεκτόνων\\
\tabto{4em} διαπονήματα \\
\tabto{2em} παρέχεται, \\
\tabto{4em} πάντα φέρουσα ἄφθονα, \\
\tabto{6em} τά τε αὖ \\
\tabto{8em} περὶ τὰ ζῷα \\
\tabto{6em} ἱκανῶς \\
\tabto{8em} ἥμερα καὶ ἄγρια \\
\tabto{6em} τρέφουσα. \\

}

\begin{description}[noitemsep]

\item[Πλεῖστα] §~202
\item[ἡ νῆσος] §~82, §~83; sc.\ Ἀτλαντίς
\item[δὲ] čestica označava nadovezivanje na prethodni iskaz, §~515
\item[αὐτὴ] §~207, značenje u atributnom položaju §~378
\item[παρείχετο] παρέχομαι pružati; 3. l. sg. impf. medpas.
\item[εἰς] §~419 c
\item[τὰς\dots\ κατασκευάς] §~90
\item[τοῦ βίου] §~82, atributni položaj §~375
\item[πρῶτον μὲν\dots\ τότε δὲ\dots] koordinacija rečeničnih članova parom čestica: najprije\dots\ a tada\dots
\item[ὅσα] §~219; uvodi zavisnu odnosnu rečenicu, antecedent Πλεῖστα
\item[ὅσα\dots\ ὀρυττόμενα στερεὰ καὶ ὅσα τηκτὰ γέγονε] koordinacija (paralelizam) rečeničnih članova
\item[ὑπὸ] prijedlog s g. uz pasiv §~437 
\item[μεταλλείας] §~90
\item[ὀρυττόμενα] ὀρύττω kopati; n. pl. s. r. ptc. prez. medpas.
\item[στερεὰ\dots\ τηκτὰ] §~103  (minerali i kamenje – metali)
\item[γέγονε] γίγνομαι postajati, bivati; 3. l. sg. ind. perf. akt.; subjekt u pl. s. r., \textit{verbum finitum} u sg.; sročnost §~361
\item[τὸ νῦν ὀνομαζόμενον\dots\ τότε δὲ πλέον ὀνόματος\dots] koordinacija usporednih rečeničnih članaka: ono što se sad\dots\ a tada\dots
\item[ὀνομαζόμενον] ὀνομάζω nazivati; n. sg. s. r. ptc. prez. medpas.; ὀνομαζόμενον μόνον samo se naziva (= nije više od naziva)
\item[μόνον] priložno LSJ μόνος B.II
\item[πλέον\dots\ ἦν] imenski predikat, Smyth 909
\item[πλέον] §~202; komparativ otvara mjesto genitivu
\item[ὀνόματος] §~123; genitiv usporedbe §~404
\item[ἦν] εἰμί biti; 3. l. sg. impf.
\item[τὸ γένος] §~153
\item[γῆς] §~108
\item[ὀρυττόμενον] ὀρύττω kopati; n. sg. s. r. ptc. prez. medpas.
\item[ὀρειχάλκου] §~82; ovisno o τὸ γένος, adnominalni genitiv (Smyth 1290-1296): vrsta „orihalk''
\item[κατὰ] §~429. B.
\item[τόπους πολλοὺς] §~82, §~196
\item[τῆς νήσου] §~82, §~83
\item[πλήν] prijedlog s genitivom
\item[χρυσοῦ] §~107
\item[τιμιώτατον] §~197; ovisno o τὸ γένος i ὄν
\item[ἐν] §~426. a
\item[τοῖς τότε] poimeničenje članom §~373
\item[ὄν] εἰμί biti; n. sg. s. r. ptc.
\item[καὶ ὅσα ὕλη πρὸς τὰ τεκτόνων διαπονήματα\dots\ τά τε αὖ  περὶ τὰ ζῷα] koordinacija rečeničnih članaka sastavnim veznicima καὶ i τε, §~513.2
\item[ὅσα] §~219; uvodi zavisnu odnosnu rečenicu, antecedent Πλεῖστα
\item[ὕλη] §~90
\item[πρὸς] §~435 C. c. δ
\item[τὰ διαπονήματα] §~123
\item[τεκτόνων] §~131
\item[παρέχεται] παρέχομαι pružati; 3. l. sg. ind. prez. medpas.
\item[πάντα] §~193; objekt od φέρουσα
\item[φέρουσα] φέρω nositi; n. sg. ž. r. ptc. prez. akt.; ovisno o ὕλη
\item[ἄφθονα] §~106
\item[τά\dots\ περὶ τὰ ζῷα] poimeničenje članom §~373
\item[τὰ ζῷα] §~82
\item[ἥμερα\dots\ ἄγρια] §~106, §~103; sc.\ ζῷα; objekti participa τρέφουσα
\item[τρέφουσα] τρέφω hraniti; n. sg. ž. r. ptc. prez. akt.; ovisno o ὕλη
\end{description}

%3 

{\large
\noindent καὶ δὴ καὶ\\
\tabto{2em} ἐλεφάντων \\
ἦν \\
\tabto{2em} ἐν αὐτῇ \\
γένος πλεῖστον· \\
νομὴ γὰρ\\
\tabto{2em} τοῖς τε ἄλλοις ζῴοις, \\
\tabto{4em} ὅσα καθ' ἕλη καὶ λίμνας καὶ ποταμούς, \\
\tabto{4em} ὅσα τ' αὖ κατ' ὄρη \\
\tabto{4em} καὶ ὅσα ἐν τοῖς πεδίοις νέμεται, \\
\tabto{2em} σύμπασιν \\
παρῆν \\
ἅδην, \\
καὶ τούτῳ \\
\tabto{2em} κατὰ ταὐτὰ \\
τῷ ζῴῳ, \\
\tabto{2em} μεγίστῳ πεφυκότι καὶ πολυβορωτάτῳ. \\

}

\begin{description}[noitemsep]

\item[καὶ δὴ καὶ] pa i §~513
\item[ἐλεφάντων] §~139; ovisno o γένος (v. niže); adnominalni genitiv (Smyth 1290-1296): vrsta „slonovi''
\item[ἦν] εἰμί postojati, LSJ s.~v.\ A.I; 3. l. sg. impf.
\item[αὐτῇ] §~207.3; sc.\ τῇ νήσῳ
\item[γένος] §~153; otvara mjesto adnominalnom genitivu ἐλεφάντων
\item[πλεῖστον] §~202
\item[νομὴ] §~90
\item[γὰρ] čestica najavljuje iznošenje razloga ili objašnjenja: naime\dots, §~517
\item[τοῖς τε ἄλλοις ζῴοις\dots\ καὶ τούτῳ\dots\ τῷ ζῴῳ] koordinacija rečeničnih članaka s pomoću sastavnih veznika; dativima otvara mjesto predikat παρῆν
\item[τοῖς\dots\ ἄλλοις ζῴοις] §~202; §~82
\item[ὅσα\dots\ ὅσα τ' αὖ\dots\ καὶ ὅσα\dots] odnosna zamjenica uvodi zavisne odnosne rečenice koordinirane sastavnim veznicima, antecedent ζῴοις
\item[καθ' ἕλη]  καθ' ἕλη < κατ' ἕλη < κατὰ ἕλη; §~68
\item[ἕλη\dots\ λίμνας\dots\ ποταμούς] §~153, §~90, §~82
\item[τ' αὖ] = τε αὖ §~68
\item[ὄρη] §~153
\item[τοῖς πεδίοις] §~82
\item[νέμεται] νέμομαι na pašu ići, pasti; stanovati u čemu (na čemu); 3. l. sg. ind. prez. medpas.; subjekt u pl. s. r., \textit{verbum finitum} u sg.; §~361
\item[σύμπασιν] §~379
\item[παρῆν] πάρειμί τινι biti komu na raspolaganju, LSJ πάρειμι II; 3. l. sg. impf.
\item[τούτῳ] §~213.2
\item[κατὰ ταὐτὰ] jednako, na isti način; ταὐτὰ, §~16, §~66
\item[τῷ ζῴῳ] §~82
\item[μεγίστῳ] §~200
\item[πεφυκότι] φύομαι po prirodi biti; d. sg. s. r. ptc. perf. akt.
\item[πολυβορωτάτῳ] §~197
\end{description}



%4

{\large
\noindent πρὸς δὲ τούτοις, \\
ὅσα εὐώδη \\
\tabto{2em} τρέφει που \\
\tabto{2em} γῆ \\
\tabto{2em} τὰ νῦν, \\
\tabto{4em} ῥιζῶν ἢ χλόης ἢ ξύλων ἢ χυλῶν στακτῶν \\
\tabto{6em} εἴτε ἀνθῶν ἢ καρπῶν, \\
ἔφερέν τε \\
\tabto{2em} ταῦτα \\
καὶ ἔτρεφεν \\
\tabto{2em} εὖ.\\

}

\begin{description}[noitemsep]
\item[δὲ] čestica označava nadovezivanje na prethodni iskaz: a\dots
\item[πρὸς\dots\ τούτοις] osim toga
\item[τούτοις] §~213.2
\item[ὅσα] uvodi zavisnu odnosnu rečenicu (u inverziji), antecedent ταῦτα: sve ono što\dots
\item[εὐώδη] §~153
\item[τρέφει] τρέφω hraniti, rađati (LSJ s.~v. II.5); 3. l. sg. ind. prez. akt.
\item[γῆ] §~108
\item[τὰ νῦν] §~373; LSJ νῦν I.1
\item[ῥιζῶν\dots\ χλόης\dots\ ξύλων\dots\ χυλῶν] §~97, §~90, §~82; genitivi ovisni o ὅσα εὐώδη%materiae? pita NZ -- sintaksa, nije potrebno ovdje
\item[στακτῶν] στάζω kapati; gl. pridjev στακτός; g. pl. m. r. 
\item[εἴτε\dots\ ἢ\dots] koordinacija s pomoću rastavnih veznika: bilo\dots\ ili\dots
\item[ἀνθῶν\dots\ καρπῶν] §~153, §~82
\item[ἔφερέν τε\dots\ καὶ ἔτρεφεν\dots] koordinacija rečeničnih članaka s pomoću sastavnih veznika
\item[ἔφερέν τε] φέρω nositi; 3. l. sg. impf. akt.; enklitika §~39, §~40; subjekt je ἡ νῆσος
\item[ταῦτα] §~213.2; antecedent odnosne zamjenice ὅσα
\item[ἔτρεφεν] τρέφω hraniti; 3. l. sg. impf. akt.; subjekt je ἡ νῆσος
\end{description}




%kraj
