% Unio ispravke NČ 2019-09-04
%\section*{O autoru}


\section*{O tekstu}

Druga knjiga Herodotove povijesti \textgreek[variant=ancient]{(Εὐτέρπη)} posvećena je Egiptu. U izabranom odlomku Herodot opisuje prehrambene navike Egipćana, napose kako pripremaju ribu i ptice, kao i zanimljivost o običaju da na gozbama bogatih oko stolova kruži drveni kipić mrtvaca u koji, dok piju i zabavljaju se, gledaju i pomišljaju na vlastitu smrtnost.

%\newpage

\section*{Pročitajte naglas grčki tekst.}

Hdt.\ Historiae 2.77.12

%Naslov prema izdanju

\medskip

\begin{greek}
{\large
{ \noindent Ἀρτοφαγέουσι δὲ ἐκ τῶν ὀλυρέων ποιεῦντες ἄρτους, τοὺς ἐκεῖνοι κυλλήστις ὀνομάζουσι. Οἴνῳ δὲ ἐκ κριθέων πεποιημένῳ διαχρέωνται· οὐ γάρ σφί εἰσι ἐν τῇ χώρῃ ἄμπελοι. Ἰχθύων δὲ τοὺς μὲν πρὸς ἥλιον αὐήναντες ὠμοὺς σιτέονται, τοὺς δὲ ἐξ ἅλμης τεταριχευμένους· ὀρνίθων δὲ τούς τε ὄρτυγας καὶ τὰς νήσσας καὶ τὰ σμικρὰ τῶν ὀρνιθίων ὠμὰ σιτέονται προταριχεύσαντες· τὰ δὲ ἄλλα ὅσα ἢ ὀρνίθων ἢ ἰχθύων σφί ἐστι ἐχόμενα, χωρὶς ἢ ὁκόσοι σφι ἱροὶ ἀποδεδέχαται, τοὺς λοιποὺς ὀπτοὺς καὶ ἑφθοὺς σιτέονται. Ἐν δὲ τῇσι συνουσίῃσι τοῖσι εὐδαίμοσι αὐτῶν, ἐπεὰν ἀπὸ δείπνου γένωνται, περιφέρει ἀνὴρ νεκρὸν ἐν σορῷ ξύλινον πεποιημένον, μεμιμημένον ἐς τὰ μάλιστα καὶ γραφῇ καὶ ἔργῳ, μέγαθος ὅσον τε πάντῃ πηχυαῖον ἢ δίπηχυν, δεικνὺς δὲ ἑκάστῳ τῶν συμποτέων λέγει· ``Ἐς τοῦτον ὁρέων πῖνέ τε καὶ τέρπεο· ἔσεαι γὰρ ἀποθανὼν τοιοῦτος.'' Ταῦτα μὲν παρὰ τὰ συμπόσια ποιεῦσι.

}
}
\end{greek}

\section*{Analiza i komentar}

%1

{\large
\begin{greek}
\noindent Ἀρτοφαγέουσι δὲ \\
\tabto{2em} ἐκ τῶν ὀλυρέων \\
\tabto{4em} ποιεῦντες ἄρτους,
\tabto{6em} τοὺς ἐκεῖνοι \\
\tabto{6em} κυλλήστις ὀνομάζουσι.\\

\end{greek}
}

\begin{description}[noitemsep]
\item[Ἀρτοφαγέουσι] ἀρτοφαγέω jesti kruh; 3. l. pl. ind. prez. akt. (subjekt je \textgreek[variant=ancient]{Αἰγύπτιοι)}
\item[δὲ] a, pak; u korelaciji s μὲν u prethodnom odjeljku
\item[ἐκ τῶν ὀλυρέων] §~97; jonski oblik g. pl.
\item[ποιεῦντες] ποιέω činim; n. pl. m. r. ptc. prez. akt. (jonski umjesto \textgreek[variant=ancient]{ποιoῦντες)}

\end{description}

%2

{\large
\begin{greek}
\noindent Οἴνῳ δὲ \\
\tabto{2em} ἐκ κριθέων \\
πεποιημένῳ \\
διαχρέωνται·\\
οὐ γάρ σφί εἰσι \\
\tabto{2em} ἐν τῇ χώρῃ \\
ἄμπελοι.\\

\end{greek}
}

\begin{description}[noitemsep]
\item[Οἴνῳ] §~82
\item[ἐκ κριθέων] §~90; jonski oblik g. pl.
\item[πεποιημένῳ] ποιέω raditi; d. sg. m. r. ptc. perf. medpas.
\item[διαχρέωνται] διαχράομαι služiti se; 3. l. pl. ind. prez. medpas. (jonski umjesto \textgreek[variant=ancient]{διαχρέονται)}
\item[σφί] §~206.5-6 (jonski oblik osobne zamjenice za treće lice; u atičkom se umjesto nje obično upotrebljavaju oblici zamjenice αὐτός)
\item[εἰσι] εἰμί biti; 3. l. pl. ind. prez. akt; enklitike §~38.3 
\item[ἐν τῇ χώρῃ] §~90 (jonski oblik)
\item[ἄμπελοι] §~82

\end{description}


%3

{\large
\begin{greek}
\noindent Ἰχθύων δὲ \\
\tabto{2em} τοὺς μὲν \\
\tabto{4em} πρὸς ἥλιον αὐήναντες \\
\tabto{2em} ὠμοὺς σιτέονται,\\
\tabto{2em} τοὺς δὲ \\
\tabto{4em} ἐξ ἅλμης \\
\tabto{2em} τεταριχευμένους·\\
ὀρνίθων δὲ \\
\tabto{2em} τούς τε ὄρτυγας \\
\tabto{2em} καὶ τὰς νήσσας \\
\tabto{2em} καὶ τὰ σμικρὰ τῶν ὀρνιθίων ὠμὰ \\
σιτέονται προταριχεύσαντες·\\
τὰ δὲ ἄλλα ὅσα ἢ ὀρνίθων ἢ ἰχθύων \\
\tabto{2em} σφί ἐστι ἐχόμενα,\\
χωρὶς ἢ ὁκόσοι σφι ἱροὶ ἀποδεδέχαται,\\
τοὺς λοιποὺς \\
\tabto{2em} ὀπτοὺς καὶ ἑφθοὺς σιτέονται.\\

\end{greek}
}

\begin{description}[noitemsep]
\item[Ιχθύων] §~173
\item[τοὺς μὲν\dots\ τοὺς δὲ] koordinacija: jedne\dots\ a druge\dots
\item[πρὸς ἥλιον] §~82
\item[αὐήναντες] αὐαίνω sušiti; n. pl. m. r. ptc. aor. akt.
\item[ὠμοὺς] §~103
\item[σιτέονται] σιτέομαι jesti; 3. l. pl. ind. prez. medpas.
\item[ἐξ ἅλμης] §~90
\item[τεταριχευμένους] ταριχεύω zaštititi od propadanja; a. pl. m. r. ptc. perf. medpas.
\item[ὀρνίθων] §~129
\item[τε\dots\  καὶ\dots\  καὶ] koordinacija sastavnim veznicima; prevedite samo zadnje καὶ
\item[τούς ὄρτυγας] §~115
\item[τὰς νήσσας] jonski oblik za νῆττας; §~97
\item[τὰ σμικρὰ] jonski za τὰ μικρὰ;  §~103
\item[τῶν ὀρνιθίων] §~82
\item[ὠμὰ] §~103
\item[προταριχεύσαντεs] προταριχεύω usoliti; n. pl. m. r. ptc. aor. akt.
\item[τὰ δὲ ἄλλα] §~212
\item[ὅσα] §~219; §~482
\item[ἐστι] εἰμί biti; 3. l. sg. ind. prez. akt; enklitike §~38.3
\item[ἐχόμενα] ἔχω imati; n. pl. sr. r. ptc. prez. medpas.; slaganje subjekta srednjeg roda u množini s predikatom §~361
\item[ὁκόσοι] jonski umjesto ὁπόσοι;  §~219; §~482
\item[ἱροὶ] jonski umjesto ἱεροὶ; §~103
\item[ἀποδεδέχαται] ἀποδέχομαι vjerovati, razumijevati, prihvaćati (ovdje s pasivnim značenjem); 3. l. pl. ind. perf.  medpas. (jonski umjesto očekivanog atičkog oblika \textgreek[variant=ancient]{ἀποδεδεγμένοι εἰσί)}; §~297
\item[τοὺς λοιποὺς ὀπτοὺς καὶ ἑφθοὺς] §~103

\end{description}


%4

{\large
\begin{greek}
\noindent Ἐν δὲ τῇσι συνουσίῃσι \\
\tabto{2em} τοῖσι εὐδαίμοσι αὐτῶν,\\
\tabto{4em} ἐπεὰν ἀπὸ δείπνου γένωνται,\\
περιφέρει ἀνὴρ \\
νεκρὸν \\
\tabto{2em} ἐν σορῷ \\
ξύλινον πεποιημένον,\\
μεμιμημένον \\
\tabto{2em} ἐς τὰ μάλιστα \\
\tabto{2em} καὶ γραφῇ καὶ ἔργῳ,\\
μέγαθος \\
\tabto{2em} ὅσον τε πάντῃ πηχυαῖον ἢ δίπηχυν,\\
δεικνὺς δὲ \\
\tabto{2em} ἑκάστῳ τῶν συμποτέων \\
λέγει·\\

\end{greek}
}

\begin{description}[noitemsep]
\item[Ἐν δὲ τῇσι συνουσίῃσι] jonski umjesto \textgreek[variant=ancient]{Ἐν δὲ ταῖς συνουσίαις}; §~90
\item[τοῖσι εὐδαίμοσι] jonski umjesto τοῖς εὐδαίμοσι; §~131
\item[αὐτῶν] §~207
\item[ἐπεὰν] jonski umjesto ἐπήν; otvara mjesto konj. aor §~488.2
\item[ἀπὸ δείπνου]  §~82
\item[γένωνται] γίγνομαι postati (ovdje s prethodnim \textgreek[variant=ancient]{ἀπὸ δείπνου}: kad završe s jelom); 3. l. pl. konj. aor. med.
\item[περιφέρει] περιφέρω nositi uokolo; 3. l. sg. ind. prez. akt.
\item[ἀνὴρ] §~149
\item[νεκρὸν ἐν σορῷ ξύλινον] §~82; §~103
\item[πεποιημένον] ποιέω činiti, izrađivati; a. sg. m. r. ptc. perf. medpas.
\item[μεμιμημένον] μιμέομαι oponašati; a. sg. m. r. ptc. perf. medpas.
\item[ἐς τὰ μάλιστα] do krajnosti, u potpunosti; μάλιστα je ovdje upotrijebljeno pridjevski, inače se koristi kao superlativ priloga (§~204.3)
\item[γραφῇ καὶ ἔργῳ] bojom i oblikom; §~90; §~82
\item[μέγαθος] jonski za μέγεθος; akuzativ obzira; §~153; §~389
\item[ὅσον τε] otprilike
\item[πάντῃ] po širini i visini; §~193
\item[πηχυαῖον ἢ δίπηχυν] §~103; §~191; §~390
\item[δεικνὺς] δείκνυμι pokazivati; n. sg. m. r. ptc. prez. akt.
\item[ἑκάστῳ ] §~103
\item[τῶν συμποτέων] §~100 (jonski oblik g. pl.)
\item[λέγει] λέγω govoriti; 3. l. sg. ind. prez. akt.

\end{description}


%5

{\large
\begin{greek}
\noindent ``Ἐς τοῦτον \\
\tabto{2em} ὁρέων \\
πῖνέ τε καὶ τέρπεο· \\
ἔσεαι γὰρ \\
\tabto{2em} ἀποθανὼν\\
τοιοῦτος.''\\

\end{greek}
}

\begin{description}[noitemsep]
\item[Ἐς τοῦτον] §~213
\item[ὁρέων] ὁράω gledati; n. sg. m. r. ptc. prez. akt.(jonski nestegnuti oblik)
\item[πῖνέ ] πῖνω piti; 2. l. sg. impt. prez. akt.
\item[τέρπεο] τέρπομαι veseliti se; 2. l. sg. impt. prez medpas.
\item[ἔσεαι] εἰμί biti; 2. l. sg. fut. medpas. (jonski nestegnuti oblik)
\item[ἀποθανὼν ] ἀποθνῄσκω umirati; n. sg. m. r. ptc. aor. akt.
\item[τοιοῦτος] §~213.4

\end{description}


%6

{\large
\begin{greek}
\noindent Ταῦτα μὲν \\
\tabto{2em} παρὰ τὰ συμπόσια \\
ποιεῦσι.\\

\end{greek}
}

\begin{description}[noitemsep]
\item[Ταῦτα] §~213.2
\item[παρὰ τὰ συμπόσια] za vrijeme gozbi (pijanki); §~82
\item[ποιεῦσι] ποιέω činiti; 3. l. pl. ind. prez. akt. (jonski oblik umjesto atičkog ποιοῦσι)

\end{description}



%kraj
