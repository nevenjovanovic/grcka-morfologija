\documentclass[a4paper,12pt,twoside]{report}
\usepackage{polyglossia}
\setmainlanguage{croatian}
\setotherlanguage[variant=ancient]{greek}

%\usepackage{verse}
\defaultfontfeatures{Ligatures=TeX}
\setmainfont{Old Standard}
\newfontfamily\greekfont[Mapping=TeX]{Old Standard}
\newfontfamily\greekfontsf[Script=Greek]{Old Standard}
\newfontfamily\greekfontbf[Script=Greek]{Old Standard}
%\setmainfont{GFS Didot}
%\newfontfamily\greekfont[Mapping=TeX]{GFS Didot}
%\newfontfamily\greekfontsf[Script=Greek]{GFS Didot}
%\newfontfamily\greekfontbf[Script=Greek]{GFS Didot}

\usepackage{import}
\usepackage{sectsty}
\allsectionsfont{\rmfamily}
\usepackage{tabto}
\usepackage{ulem}
\usepackage{hyperref}
\usepackage{xcolor}
\hypersetup{
    colorlinks,
    linkcolor={red!50!black},
    citecolor={blue!50!black},
    urlcolor={blue!80!black}
}
\usepackage{enumitem}
%\usepackage{dtk-logos}
\usepackage[symbol]{footmisc}
%\usepackage[defaultlines=4,all]{nowidow}

\usepackage{fancyhdr}
\renewcommand{\chaptermark}[1]{\markboth{#1}{}}
\renewcommand{\sectionmark}[1]{\markright{#1}}
\pagestyle{fancy}
\fancyhf{}
\fancyhead[LE,RO]{\thepage}
\fancyhead[RE]{\itshape\nouppercase{Čitanka za Grčku morfologiju 2}}
\fancyhead[LO]{\nouppercase{\leftmark}}
\renewcommand{\headrulewidth}{0pt}
\setlength{\headheight}{15.39003pt}

\usepackage{titling}
\newcommand{\subtitle}[1]{%
  \posttitle{%
    \par\end{center}
    \begin{center}\large#1\end{center}
    \vskip0.5em}%
}


\hyphenation{δυσ-σέ-βει-αν βού-λεσ-θαί κα-τη-γο-ρού-σης τα-χέ-ως πε-πλημ-μέ-λη-κε νε-α-νί-ας αὐ-τῇ ἀ-φε-λό-με-νος ἀ-πή-γε-το ἐ-πέ-πληξ-άς ἠ-γό-μην ἤ-νεγ-κεν νο-μί-ζει ἄν-θρω-πον παν-το-δα-πά Αἰ-θί-οψ-ιν ἀν-έρ-χε-ται φρον-τί-δων αὐ-τὸς δι-ῃ-ρη-μέ-νος κλέπ-τον-τας ἑ-κα-τόμ-βῃ μέ-γισ-τον κιν-δύ-νου}

\hyphenation{Ελ-λή-νων ξυν-έ-μει-νεν ἐ-πι-όντ-ων ἀνα-σκευ-α-σά-με-νοι ἀ-πω-σά-με-νοι Λα-κε-δαι-μό-νι-οι Ελ-λη-νες δι-ε-φά-νη}

\hyphenation{εὐ-δο-κι-μή-σας ἐν-ταῦ-θα βα-σι-λεύ-εις ἐ-χρή-σα-το συλ-ληφ-θέν-τος}

\hyphenation{θε-ρά-πον-τες ἡ-γοῦ-μαι}

\hyphenation{ἐπι-φα-νέσ-τε-ρον το-σοῦ-τον ἔ-χον-τας συγ-γιγ-νο-μέ-νους μᾶλ-λον με-γίσ-την}

\hyphenation{ἐπ-έσ-κηπ-τε Θε-μισ-το-κλῆς}

\hyphenation{κοι-νω-νοῦ-σιν δη-λῶ-σαι δι-α-λε-γο-μέ-νους πλη-σι-ά-ζον-τας πα-ρα-λι-πεῖν}

\hyphenation{καρ-πῶν ὀ-νο-μα-ζό-με-νον τηκ-τὰ ὅ-σα}

\hyphenation{με-μά-θη-κας Μά-λισ-τα πο-λυ-μα-θής}

\hyphenation{δι-α-βε-βλη-μέ-νος Α-ρι-στό-βου-λος Σω-κρά-τους Δı-α-τ
ρι-βαί ἐξ-ελ-κύ-σαι Ἐγ-χει-ρί-δι-ον}

\hyphenation{δά-μα-λιν δύ-να-μαι}

\hyphenation{πυ-θο-μέ-νου ἀ-πο-κρί-νε-ται τρα-πέ-ζαις ἀ-πέσ-τει-λαν γρά-ψαν-τος με-τα-τί-θη-σι με-γά-λου Τισ-σα-φέρ-νῃ προσ-ε-πι-μετ-ρῆ-σαι Α-θη-ναί-οις Τισ-σα-φέρ-νην Αλ-κι-βι-ά-δην πα-τρί-δα}

\hyphenation{ἡ-γοῦν-ται ἀ-πο-ρω-τά-των πό-λεις Ras-pra-va Ἀ-πο-λο-γί-α Λα-κε-δαı-μο-νί-ων Κυ-νη-γε-τι-κός}

\hyphenation{πλε-όν-των γε-ωρ-γὸς ἕ-τε-ρον}

\hyphenation{κα-θεῖ-ναι προ-ελ-θοῦ-σαν ἀ-πο-δοῦ-ναι ἀ-πο-φαί-νον-τος ἀ-πεσ-τά-λη ἅ-παν-τας τρί-πο-δα τρί-πο-δος κα-θι-ερώ-θη}

\hyphenation{χρεί-αν παν-τὸς ἀν-επί-σκεπ-τον ἀ-πο-θα-νόν-των πα-ρα-σκευ-ά-ζον-τας φρον-τί-ζον-τας πολ-λοὺς κτω-μέ-νους ἐ-λατ-τοῦσ-θαι ἀ-με-λοῦν-τας ἀ-θε-ρά-πευ-τον κτη-μά-των πει-ρω-μέ-νους ὀ-λι-γω-ροῦν-τας δε-ο-μέ-νων ἐ-ῶν-τας ἔ-μοι-γε}

\hyphenation{αὐ-το-κρά-το-ρας Λά-μα-χον ποι-ή-σαν-τες ἑξή-κον-τα Σε-λι-νουν-τί-ους χρη-μά-των μισ-θόν Νι-κη-ρά-του πε-ρι-γίγ-νη-ται}

\renewcommand{\baselinestretch}{1.1} 

\begin{document}

\title{Čitanka za Grčku morfologiju 2}
%\subtitle{Vježbe 41–60}
\author{Nina Čengić, Neven Jovanović, Petra Matović,\\Vlado Rezar, Ninoslav Zubović}
\date{}
\maketitle

\clearpage

\tableofcontents

%\thispagestyle{empty}


%\frontmatter


\chapter*{Predgovor}
\label{chap:predgovor}
\addcontentsline{toc}{chapter}{Predgovor}
\chaptermark{Predgovor}

Ovaj izbor komentiranih tekstova čini propisanu literaturu obaveznog kolegija \textit{Grčka morfologija 2}. Kao i u \textit{Čitanci za Grčku morfologiju 1}, kratkim su odlomcima dodane tri vrste komentara. U uvodu su predstavljeni autor (kada se susreće prvi put), tekst iz kojeg odlomak potječe, i kontekst samog odlomka. Nakon teksta slijede rečenice, raščlanjene i složene tako da povećaju sintaktičku i semantičku preglednost, te gramatički komentar. Potonji je komentar, opet, dvojak. U skladu sa sadržajem ovog kolegija, za glagolske oblike donose se samo brojevi paragrafa koji upućuju studenta na relevantne dijelove \textit{Grčke gramatike} Augusta Musića i Nikole Majnarića, standardni školski priručnik za opis grčkog jezika. Ostatak komentara objašnjava gramatičke, sintaktičke i stvarne aspekte koji tekst čine teže razumljivima.

Zadatak je studenata da, uz pomoć komentara i referentne literature, kod kuće prirede svaki tekst, tako da ga na nastavi budu sposobni pročitati i prevesti na hrvatski, te identificirati i morfološki opisati glagolske oblike koje susreću. Također, očekuje se i da studenti nauče određeni broj riječi iz ovih tekstova, uz pomoć popisa i vježbi koje će naći na stranicama kolegija \textit{Grčka morfologija 2} Sustava učenja na daljinu Omega (Filozofski fakultet Sveučilišta u Zagrebu).

%\newpage

Izbor su sastavili i komentare priredili nastavnici Odsjeka za klasičnu filologiju Filozofskog fakulteta Sveučilišta u Zagrebu (abecednim redom): Nina Čengić, Neven Jovanović, Petra Matović, Vlado Rezar, Ninoslav Zubović. Čitanka je priređena računalnim programima za slaganje teksta LaTeX i XeLaTeX. Izvorni kod dostupan je u repozitoriju Github, na URL adresi \url{https://github.com/nevenjovanovic/grcka-morfologija}.

%\medskip

U Zagrebu, siječnja 2022.

\section*{Referentna literatura}

F. R. Adrados et al., \textit{Diccionario Griego-Español}, Instituto de Lenguas y Culturas del Mediterráneo y Oriente Próximo (ILC) del Centro de Ciencias Humanas y Sociales (CCHS) del CSIC (Madrid). (DGE; dostupno i u zbirci \textit{Logeion})\\
Georg Autenrieth, \textit{A Homeric Dictionary for Schools and Colleges}. New York, Harper and Brothers, 1891. (Dostupno i u zbirci \textit{Logeion})\\
John Dewar Denniston, Kenneth James Dover. \textit{The Greek Particles \dots\ Second Edition}. [Revised by Kenneth J. Dover.]. Clarendon Press: Oxford, 1954.\\
\textit{Grčka morfologija 2}, elektronska verzija kolegija, Sustav učenja na daljinu Omega, Zagreb: Filozofski fakultet Sveučilišta u Zagrebu, pristupljeno 23. kolovoza 2018. na adresi \url{https://omega.ffzg.hr/course/view.php?id=1615}\\
Henry George Liddell, Robert Scott, \textit{An Intermediate Greek-English Lexicon. Founded upon the seventh edition of Liddell \& Scott's Greek-English Lexicon}. Oxford, Clarendon Press, 1889. (LSJ; dostupno i u zbirci \textit{Logeion})\\
\textit{Logeion}. Pristupljeno 23. kolovoza 2018. na adresi: \\ \url{http://logeion.uchicago.edu/}\\
August Musić, Nikola Majnarić, \textit{Gramatika grčkoga jezika}, Zagreb (bilo koje izdanje)\\
Oton Gorski, Niko Majnarić, \textit{Grčko-hrvatski rječnik}, Zagreb (bilo koje izdanje)\\
Stjepan Senc, \textit{Grčko-hrvatski rječnik}, Zagreb (bilo koje izdanje)\\
Herbert Weir Smyth, \textit{A Greek Grammar for Colleges}, Perseus Digital Library. Pristupljeno 23. prosinca 2018. na adresi \url{http://www.perseus.tufts.edu}\\

\vspace*{\fill}

\noindent Ovo djelo je ustupljeno pod Creative Commons licencom Imenovanje 3.0 nelokalizirana licenca. Da biste vidjeli primjerak te licence, posjetite \url{http://creativecommons.org/licenses/by/3.0/} ili pošaljite pismo na Creative Commons, PO Box 1866, Mountain View, CA 94042, SAD.

\newpage


%\mainmatter


%\clearpage
%\thispagestyle{empty}
% start chapter numbering from 41:
%\setcounter{chapter}{40}

\chapter[Ἐπίκουρος Μενοικεῖ]{\textgreek[variant=ancient]{Ἐπιστολὴ Ἐπικούρου Μενοικεῖ,} 124–126}
%1
\import{grcmorf2poglavlja/}{tlg0537.tlg012:124-126.tex}

\chapter[Ἀνδοκίδου Κατὰ Ἀλκιβιάδου]{\textgreek[variant=ancient]{Ἀνδοκίδου Κατὰ Ἀλκιβιάδου,} 1–2}
%2
\import{grcmorf2poglavlja/}{tlg0027.tlg004.perseus-grc1:1-2.tex}

\chapter[Ξενοφῶντος Κύρου παιδεία]{\textgreek[variant=ancient]{Ξενοφῶντος Κύρου παιδεία Α} 2,8}
%3
\import{grcmorf2poglavlja/}{tlg0032.tlg007.perseus-grc2:1.2.8.tex}

\chapter[Πλάτωνος Πρωταγόρας]{\textgreek[variant=ancient]{Πλάτωνος Πρωταγόρας} 325c-326a}
%4
\import{grcmorf2poglavlja/}{tlg0059.tlg022.perseus-grc2:325-326.tex}

\chapter[Λουκιανοῦ Ἐνάλιος διάλογος]{\textgreek[variant=ancient]{Λουκιανοῦ Ἐνάλιος διάλογος} 2, 2}
%5
\import{grcmorf2poglavlja/}{tlg0062.tlg067.perseus-grc1:2.292-2.293.tex}

\chapter[Μάρκου Ἀντωνίνου Τῶν εἰς ἑαυτὸν Δ]{\textgreek[variant=ancient]{Μάρκου Ἀντωνίνου αὐτοκράτορος \\Τῶν εἰς ἑαυτὸν βιβλίον Δ} 48}
%6
\import{grcmorf2poglavlja/}{tlg0562.tlg001.perseus-grc1:4.48.tex}

\chapter[Διοδώρου Βιβλιοθήκης ἱστορικής ΙΕ]{\textgreek[variant=ancient]{Διοδώρου Σικελιώτου \\Βιβλιοθήκης ἱστορικῆς ΙΕ,} 6, 1–2}
%7
\import{grcmorf2poglavlja/}{tlg0060.tlg001.perseus-grc3:15.6.1-15.6.2.tex}

\chapter[Ἀριστοτέλους Ἠθικῶν Νικομαχείων Ι]{\textgreek[variant=ancient]{Ἀριστοτέλους \\Ἠθικῶν Νικομαχείων Ι,} \\1169b (9, 9)}
%8
\import{grcmorf2poglavlja/}{tlg0086.tlg010.perseus-grc1:1169b.tex}

\chapter[Αἰσώπειος μῦθος 279]{\textgreek[variant=ancient]{Αἰσώπειος μῦθος} 279 (349)}
%9
\import{grcmorf2poglavlja/}{tlg0096.tlg002.First1K-grc1:349.tex}

\chapter[Ἀνδοκίδου Περὶ τῶν μυστηρίων]{\textgreek[variant=ancient]{Ἀνδοκίδου Περὶ τῶν μυστηρίων,} 97}
\label{chap:andocides}
%10
\import{grcmorf2poglavlja/}{tlg0027.tlg001.perseus-grc1:97.tex}

\chapter[Πολυβίου Ἱστορίων ΛΗ]{\textgreek[variant=ancient]{Πολυβίου ἱστορίων ΛΗ,} 22.1–22.3}
%11
\import{grcmorf2poglavlja/}{tlg0543.tlg001:38.22.1-22.3.tex}

\chapter[Ἐγχειρίδιον Ἐπικτήτου 29]{\textgreek[variant=ancient]{Ἐγχειρίδιον Ἐπικτήτου} 29}
%12
\import{grcmorf2poglavlja/}{tlg0557.tlg002.perseus-grc1:29.tex}

\chapter[Ξενοφῶντος Ἀπομνημονεύματα Σωκράτους]{\textgreek[variant=ancient]{Ξενοφῶντος \\Ἀπομνημονεύματα Σωκράτους \\Δ} 3,3–3,6}
% 13
\import{grcmorf2poglavlja/}{tlg0032.tlg002.perseus-grc2:4.3.3-4.3.6.tex}

\chapter[Ἀππιανοῦ Ῥωμαϊκῶν ΙΔ]{\textgreek[variant=ancient]{Ἀππιανοῦ Ἀλεξανδρέως \\Ῥωμαϊκῶν βιβλίον ΙΔ, \\Ἐμφυλίων Α} 116–117}
% 14
\import{grcmorf2poglavlja/}{tlg0551.tlg017.perseus-grc2:1.14.116-1.14.117.tex}

\chapter[Ἀντιφῶντος Κατηγορία φαρμακείας\dots]{\textgreek[variant=ancient]{Ἀντιφῶντος Κατηγορία φαρμακείας κατὰ τῆς μητρυιᾶς} 26–27}
% 15
\import{grcmorf2poglavlja/}{tlg0028.tlg001.perseus-grc1:26-27.tex}

\chapter[Πλουτάρχου Ἀλκιβιάδης]{\textgreek[variant=ancient]{Πλουτάρχου Βίοι παράλληλοι, \\Ἀλκιβιάδης} 2.1–2.3}
% 16
\import{grcmorf2poglavlja/}{tlg0007.tlg015.perseus-grc2:2.1-2.3.tex}

\chapter[Λυσίου Ὑπὲρ τῶν Ἀριστοφάνους χρημάτων]{\textgreek[variant=ancient]{Λυσίου Ὑπὲρ τῶν Ἀριστοφάνους χρημάτων} 14–17}
% 17
\import{grcmorf2poglavlja/}{tlg0540.tlg019.perseus-grc2:14-17.tex}


\chapter[Δημοσθένους Ὀλυνθιακός Γ]{\textgreek[variant=ancient]{Δημοσθένους Ὀλυνθιακός Γ,} 24–26}
% 18
\import{grcmorf2poglavlja/}{tlg0014.tlg003.perseus-grc1:24-26.tex}

\chapter[Ἱπποκράτους Περὶ ἱερῆς νούσου]{\textgreek[variant=ancient]{Ἱπποκράτους Περὶ ἱερῆς νούσου} 1}
% 19
\import{grcmorf2poglavlja/}{tlg0627.tlg027.1st1K-grc1:1.tex}

\chapter[Ἀρριανοῦ Ἀναβάσεως Ἀλεξάνδρου Ζ]{\textgreek[variant=ancient]{Ἀρριανοῦ Ἀναβάσεως Ἀλεξάνδρου \\Ζ α,} 4–6}
% 20
\import{grcmorf2poglavlja/}{tlg0074.tlg001.perseus-grc1:7.1.4-7.1.6.tex}

\chapter[Πλάτωνος Συμπόσιον]{\textgreek[variant=ancient]{Πλάτωνος Συμπόσιον} 203b–203e}
% 21
\import{grcmorf2poglavlja/}{tlg0059.tlg011.perseus-grc2:203.tex}

\chapter[Λόγγου Τῶν κατὰ Δάφνιν καὶ Χλόην]{\textgreek[variant=ancient]{Λόγγου Τῶν κατὰ Δάφνιν καὶ Χλόην λόγος Α} 13.4–6}
% 22
\import{grcmorf2poglavlja/}{tlg0561.tlg001.perseus-grc2:1.13.4-1.13.6.tex}

\chapter[Ἀριστοτέλους Ῥητορική Γ]{\textgreek[variant=ancient]{Ἀριστοτέλους Ῥητορική Γ} 1408b}
% 23
\import{grcmorf2poglavlja/}{tlg0086.tlg038.perseus-grc1:1408b.tex}

\chapter[Θουκυδίδου Ἱστορίαι Ζ]{\textgreek[variant=ancient]{Θουκυδίδου Ἱστορίαι Ζ} 5, 1–4}
% 24
\import{grcmorf2poglavlja/}{tlg0003.tlg001.perseus-grc2:7.5.1-7.5.4.tex}

\chapter[Ἡλιοδώρου Αἰϑιοπικὰ Ι]{\textgreek[variant=ancient]{Ἡλιοδώρου Αἰϑιοπικὰ Ι} 27}
% 25
\import{grcmorf2poglavlja/}{tlg0658.tlg001.perseus-grc1:10.27.tex}

\chapter[Λυσίου Ἐπιτάφιος]{\textgreek[variant=ancient]{Λυσίου Ἐπιτάφιος} 4–6}
% 26
\import{grcmorf2poglavlja/}{tlg0540.tlg002.perseus-grc2:4-6.tex}

\chapter[Χαρίτωνος Τὰ περὶ Χαιρέαν καὶ Καλλιρόην]{\textgreek[variant=ancient]{Χαρίτωνος Ἀφροδισιέως Τῶν περὶ Χαιρέαν καὶ Καλλιρόην ἐρωτικῶν διηγημάτων λόγοι ὀκτώ} 2.2}
% 27
\import{grcmorf2poglavlja/}{tlg0554.tlg001.perseus-grc1:2.2.1-2.2.4.tex}

\chapter[Δίωνος Κασσίου Ῥωμαϊκὴ ἱστορία ΞΒ 16]{\textgreek[variant=ancient]{Δίωνος Κασσίου Κοκκηιανοῦ\\Ῥωμαϊκὴ ἱστορία ΞΒ} 16}
% 28
\import{grcmorf2poglavlja/}{tlg0385.tlg001.perseus-grc1:62b.16.tex}

\chapter[Πολυβίου Ἱστορίων Δ]{\textgreek[variant=ancient]{Πολυβίου ἱστορίων Δ,} 31}
% 29
\import{grcmorf2poglavlja/}{tlg0543.tlg001.perseus-grc2:4.31.3-4.31.8.tex}

\chapter[Πλουτάρχου Πομπήϊος]{\textgreek[variant=ancient]{Πλουτάρχου Πομπήϊος} 40}
% 30
\import{grcmorf2poglavlja/}{tlg0007.tlg045.perseus-grc2:40.1-40.3.tex}

\chapter[Πλάτωνος Θεαίτητος]{\textgreek[variant=ancient]{Πλάτωνος Θεαίτητος} 325c-326a}
%31
\import{grcmorf2poglavlja/}{tlg0059.tlg006.perseus-grc2:173-174.tex}

\chapter[Δίωνος Κασσίου Ῥωμαϊκὴ ἱστορία ΞΒ 16–17]{\textgreek[variant=ancient]{Δίωνος Κασσίου Κοκκηιανοῦ \\Ῥωμαϊκὴ ἱστορία ΞΒ} 16–17}
%32
\import{grcmorf2poglavlja/}{tlg0385.tlg001.perseus-grc1:62b.16.5-62b.17.tex}

\chapter[Πλουτάρχου Περὶ ἀδολεσχίας  509D]{\textgreek[variant=ancient]{Πλουτάρχου Περὶ ἀδολεσχίας} 509D}
%33
\import{grcmorf2poglavlja/}{tlg0007.tlg101.perseus-grc1:509c.tex}

\chapter[Πλάτωνος Πολιτεία]{\textgreek[variant=ancient]{Πλάτωνος Πολιτεία} 359-360}
%34
\import{grcmorf2poglavlja/}{tlg0059.tlg030.perseus-grc2:2.359-2.360.tex}

\chapter[Ἐπικούρου Κύριαι δόξαι]{\textgreek[variant=ancient]{Ἐπικούρου Κύριαι δόξαι,} 1–5}
%35
\import{grcmorf2poglavlja/}{tlg0537.tlg013:1.tex}

\chapter[Πλάτωνος Φαίδων]{\textgreek[variant=ancient]{Πλάτωνος Φαίδων} 113d-114c}
%36
\import{grcmorf2poglavlja/}{tlg0059.tlg004.perseus-grc2:113-114.tex}

\chapter[Ἰσοκράτους Πρὸς Δημόνικον]{\textgreek[variant=ancient]{Ἰσοκράτους Πρὸς Δημόνικον} 24–25}
%37
\import{grcmorf2poglavlja/}{tlg0010.tlg007.perseus-grc2:24-25.tex}

\chapter[Λουκιανοῦ Ἰκαρομένιππος]{\textgreek[variant=ancient]{Λουκιανοῦ \\Ἰκαρομένιππος ἢ Ὑπερνέφελος} 6}
%38
\import{grcmorf2poglavlja/}{tlg0062.tlg021.1st1K-grc1:6.tex}

\chapter[Πλουτάρχου Περὶ φιλοπλουτίας]{\textgreek[variant=ancient]{Πλουτάρχου Περὶ φιλοπλουτίας} \\524D-524F}
%39
\import{grcmorf2poglavlja/}{tlg0007.tlg103.perseus-grc1:524d.tex}

\chapter[Πλάτωνος Φαῖδρος]{\textgreek[variant=ancient]{Πλάτωνος Φαῖδρος} 259-260}
%40
\import{grcmorf2poglavlja/}{tlg0059.tlg012.perseus-grc2:259-260.tex}

\chapter[Ἰσοκράτους Ἀρεοπαγιτικός]{\textgreek[variant=ancient]{Ἰσοκράτους Ἀρεοπαγιτικός} 43–45}
%41
\import{grcmorf2poglavlja/}{tlg0010.tlg018.perseus-grc2:43-45.tex}


\chapter[Μάρκου Ἀντωνίνου Τῶν εἰς ἑαυτὸν Δ 49]{\textgreek[variant=ancient]{Μάρκου Ἀντωνίνου αὐτοκράτορος \\Τῶν εἰς ἑαυτὸν βιβλίον Δ} 49}
%42
\import{grcmorf2poglavlja/}{tlg0562.tlg001.perseus-grc1:4.49.tex}

\chapter[Θουκυδίδου Ἱστορίαι Α]{\textgreek[variant=ancient]{Θουκυδίδου Ἱστορίαι Α} 22, 1–4}
%43

\import{grcmorf2poglavlja/}{tlg0003.tlg001.perseus-grc2:1.22.1-1.22.4.tex}

\chapter[Πλουτάρχου Πῶς ἄν τις αἴσθοιτο ἑαυτοῦ\dots]{\textgreek[variant=ancient]{Πλουτάρχου Πῶς ἄν τις αἴσθοιτο ἑαυτοῦ προκόπτοντος ἐπ' ἀρετῇ} 77D}
%44 studentima je loše sjeo ovaj odlomak VR
\import{grcmorf2poglavlja/}{tlg0007.tlg071.perseus-grc1:77d.tex}

\chapter[Ἰσοκράτους Ἑλένης ἐγκώμιον 27]{\textgreek[variant=ancient]{Ἰσοκράτους Ἑλένης ἐγκώμιον} 27–28}
%45
\import{grcmorf2poglavlja/}{tlg0010.tlg009:27-28.tex}

\chapter[Μάρκου Ἀντωνίνου Τῶν εἰς ἑαυτόν Ε 1]{\textgreek[variant=ancient]{Μάρκου Ἀντωνίνου αὐτοκράτορος \\Τῶν εἰς ἑαυτὸν βιβλίον Ε} 1}
%46
\import{grcmorf2poglavlja/}{tlg0562.tlg001.perseus-grc1:5.1.tex}

\chapter[Δημοσθένους Περὶ τοῦ στεφάνου]{\textgreek[variant=ancient]{Δημοσθένους Ὑπὲρ Κτησιφῶντος \\περὶ τοῦ στεφάνου} 201–203}
%47
\import{grcmorf2poglavlja/}{tlg0014.tlg018.perseus-grc2:201-203.tex}

% 48 dodaj: Aristotel Poetika NČ
\chapter[Ἀριστοτέλους Περὶ ποιητικῆς]{\textgreek[variant=ancient]{Ἀριστοτέλους \\Περὶ ποιητικῆς,} 1451b}

\import{grcmorf2poglavlja/}{tlg0086.tlg034.1st1K-grc1:9.tex}

% 49 dodaj: Ksenofont Efeški NČ
\chapter[Ξενοφῶντος τοῦ Ἐφεσίου Ἐφεσιακῶν Α]{\textgreek[variant=ancient]{Ξενοφῶντος τοῦ Ἐφεσίου \\Ἐφεσιακῶν Α} 2,5–9}

\import{grcmorf2poglavlja/}{tlg0641.tlg001:1.2.tex}

% 50 Sekst Empirik NJ
\chapter[Σέξτου τοῦ Ἐμπειρικοῦ Πυρρώνειαι\dots\ Γ]{\textgreek[variant=ancient]{Σέξτου τοῦ Ἐμπειρικοῦ \\Πυρρώνειαι ὑποτυπώσεις Γ} 17–19}

\import{grcmorf2poglavlja/}{tlg0544.tlg001.opp-grc1:3.17-3.19.tex}

%51

\chapter[Ἀχιλλεὼς Τατίου Τῶν κατὰ Λευκίππην\dots]{\textgreek[variant=ancient]{Ἀχιλλεὼς Τατίου \\Τῶν κατὰ Λευκίππην καὶ\\ Κλειτοφῶντα Α,} 17}

\import{grcmorf2poglavlja/}{tlg0532.tlg001:1.17.tex}

%52


\chapter[Ἀριστοτέλους Ἠθικῶν Νικομαχείων Γ]{\textgreek[variant=ancient]{Ἀριστοτέλους \\Ἠθικῶν Νικομαχείων Γ,} \\1119a (11)}

\import{grcmorf2poglavlja/}{tlg0086.tlg010.perseus-grc1:1119a.tex}


%53

\chapter[Ἰσοκράτους Ἑλένης ἐγκώμιον 23]{\textgreek[variant=ancient]{Ἰσοκράτους Ἑλένης ἐγκώμιον} 23–25}

\import{grcmorf2poglavlja/}{tlg0010.tlg009:23-25.tex}

%54

\chapter[Ἐγχειρίδιον Ἐπικτήτου 31]{\textgreek[variant=ancient]{Ἐγχειρίδιον Ἐπικτήτου} 31}

\import{grcmorf2poglavlja/}{tlg0557.tlg002.perseus-grc1:31.tex}

%55

\chapter[Λουκιανοῦ Ἀληθῶν διηγημάτων Β]{\textgreek[variant=ancient]{Λουκιανοῦ Ἀληθῶν διηγημάτων \\Β} 14–16}

\import{grcmorf2poglavlja/}{tlg0062.tlg012.1st1K-grc1:2.14-2.16.tex}

%56

\chapter[Δημοκρίτου ἐκλογή ΡϞΑ]{\textgreek[variant=ancient]{Δημοκρίτου ἐκλογή ΡϞΑ \\(Περὶ εὐθυμίης)}}

\import{grcmorf2poglavlja/}{tlg2037.tlg001:191.tex}



%57

\chapter[Γοργίου Ἐπιτάφιος]{\textgreek[variant=ancient]{Γοργίου Ἐπιταφίου ἐκλογή Ϛ}}
\label{chap:gorgias}

\import{grcmorf2poglavlja/}{tlg0081.tlg007.1st1K-grc1:1-5.tex}

%58

\chapter[Ξενοφῶντος Συμπόσιον]{\textgreek[variant=ancient]{Ξενοφῶντος Συμπόσιον} 4, 30–32}

\import{grcmorf2poglavlja/}{tlg0032.tlg004.perseus-grc2:4.30-4.32.tex}


% 59 NČ
\chapter[Λυσίου Κατ' Ἀνδοκίδου ἀσεβείας]{\textgreek[variant=ancient]{Λυσίου Κατ' Ἀνδοκίδου ἀσεβείας,} 15.2–18.1}

\import{grcmorf2poglavlja/}{tlg0540.tlg006:15.2-18.1.tex}


% dodaj tekstove NZ

% 60 NJ
\chapter[Διοδώρου Βιβλιοθήκης ἱστορικής ΙΒ]{\textgreek[variant=ancient]{Διοδώρου Σικελιώτου \\Βιβλιοθήκης ἱστορικῆς ΙΒ,} 53, 1–5}

\import{grcmorf2poglavlja/}{tlg0060.tlg001.perseus-grc3:12.53.1-12.53.5.tex}

\end{document}

% kraj
[Δημοσθένους Ὀλυνθιακός Α]{\textgreek[variant=ancient]{Δημοσθένους Ὀλυνθιακός Α,} 14–15}

%\import{grcmorf2poglavlja/}{tlg0014.tlg001.perseus-grc1:14-15.tex}

