\documentclass[a4paper,12pt,twoside]{report}
\usepackage[quiet]{polyglossia}
\setdefaultlanguage{croatian}
\setotherlanguage[variant=ancient]{greek}

%\usepackage{verse}
\defaultfontfeatures{Ligatures=TeX}

\usepackage{import}
\usepackage[small,sf,bf]{titlesec}
\usepackage{tabto}
\usepackage{ulem}
\usepackage{hyperref}
\usepackage{enumitem}
\usepackage{dtk-logos}

\usepackage{graphicx}
\graphicspath{ {./slike/} }

\usepackage[all]{nowidow}

%\usepackage{titling}
%\newcommand{\subtitle}[1]{%
%  \posttitle{%
%    \par\end{center}
%    \begin{center}\large#1\end{center}
%    \vskip0.5em}%
%}
 
\setmainfont{Old Standard TT}
\setsansfont{Old Standard TT}

\usepackage{fancyhdr}
\pagestyle{fancy}
\fancyhf{}
\fancyhead[LE,RO]{\thepage}
\fancyhead[RE]{\itshape\nouppercase{Čitanka za Grčku morfologiju 1}}
\fancyhead[LO]{\nouppercase{\leftmark}}
\renewcommand{\headrulewidth}{0.4pt}

%\pagestyle{headings}

%\usepackage{fancyhdr}
%\pagestyle{fancy}
%\fancyhead{}
%\fancyhead[RO,LE]{\thepage}
%\fancyhead[CE]{Title of the book}
%\fancyhead[CO]{\leftmark}
%\fancyfoot{}
%\renewcommand{\chaptermark}[1]{\markboth{#1}{}}
%\renewcommand{\headrulewidth}{0.4pt}
%\renewcommand{\footrulewidth}{0pt}
%\renewcommand{\textgreek}{\relax}

%\usepackage{fancyhdr}
%\setlength{\headheight}{20.4pt}
%\pagestyle{fancy}
%\fancyhf{}
%\fancyhead[LE,RO]{Overleaf}
%\fancyhead[RE,LO]{Guides and tutorials}
%\fancyfoot[CE,CO]{\thepage}
%\fancyfoot[LE,RO]{\thepage}

%\usepackage{fancyhdr}
%\renewcommand{\chaptermark}[1]{\markboth{#1}{}}
%\renewcommand{\sectionmark}[1]{\markright{#1}}
%\pagestyle{fancy}
%\fancyhf{}
%\fancyhead[LE,RO]{\thepage}
%\lhead[\thepage]{}
%\rhead[]{\thepage}
%\chead[\MakeUppercase{Čitanka za Grčku morfologiju 1}]{\MakeUppercase{\nouppercase{\leftmark}}}
%\fancyhead[RE]{\itshape\nouppercase{Čitanka za Grčku morfologiju 1}}
%\fancyhead[LO]{\nouppercase{\leftmark}}
%\renewcommand{\headrulewidth}{0pt}
%\renewcommand{\headrulewidth}{0.4pt}

\hyphenation{δυσ-σέ-βει-αν βού-λεσ-θαί κα-τη-γο-ρού-σης τα-χέ-ως πε-πλημ-μέ-λη-κε νε-α-νί-ας αὐ-τῇ ἀ-φε-λό-με-νος ἀ-πή-γε-το ἐ-πέ-πληξ-άς ἠ-γό-μην ἤ-νεγ-κεν νο-μί-ζει ἄν-θρω-πον παν-το-δα-πά Αἰ-θί-οψ-ιν ἀν-έρ-χε-ται φρον-τί-δων αὐ-τὸς δι-ῃ-ρη-μέ-νος κλέπ-τον-τας ἑ-κα-τόμ-βῃ μέ-γισ-τον κιν-δύ-νου}

\hyphenation{Ελ-λή-νων ξυν-έ-μει-νεν ἐ-πι-όντ-ων ἀνα-σκευ-α-σά-με-νοι ἀ-πω-σά-με-νοι Λα-κε-δαι-μό-νι-οι Ελ-λη-νες δι-ε-φά-νη}

\hyphenation{εὐ-δο-κι-μή-σας ἐν-ταῦ-θα βα-σι-λεύ-εις ἐ-χρή-σα-το συλ-ληφ-θέν-τος}

\hyphenation{θε-ρά-πον-τες ἡ-γοῦ-μαι}

\hyphenation{ἐπι-φα-νέσ-τε-ρον το-σοῦ-τον ἔ-χον-τας συγ-γιγ-νο-μέ-νους μᾶλ-λον με-γίσ-την}

\hyphenation{ἐπ-έσ-κηπ-τε Θε-μισ-το-κλῆς}

\hyphenation{κοι-νω-νοῦ-σιν δη-λῶ-σαι δι-α-λε-γο-μέ-νους πλη-σι-ά-ζον-τας πα-ρα-λι-πεῖν}

\hyphenation{καρ-πῶν ὀ-νο-μα-ζό-με-νον τηκ-τὰ ὅ-σα}

\hyphenation{με-μά-θη-κας Μά-λισ-τα πο-λυ-μα-θής}

\hyphenation{δι-α-βε-βλη-μέ-νος Α-ρι-στό-βου-λος Σω-κρά-τους Δı-α-τ
ρι-βαί ἐξ-ελ-κύ-σαι Ἐγ-χει-ρί-δι-ον}

\hyphenation{δά-μα-λιν δύ-να-μαι}

\hyphenation{πυ-θο-μέ-νου ἀ-πο-κρί-νε-ται τρα-πέ-ζαις ἀ-πέσ-τει-λαν γρά-ψαν-τος με-τα-τί-θη-σι με-γά-λου Τισ-σα-φέρ-νῃ προσ-ε-πι-μετ-ρῆ-σαι Α-θη-ναί-οις Τισ-σα-φέρ-νην Αλ-κι-βι-ά-δην πα-τρί-δα}

\hyphenation{ἡ-γοῦν-ται ἀ-πο-ρω-τά-των πό-λεις Ras-pra-va Ἀ-πο-λο-γί-α Λα-κε-δαı-μο-νί-ων Κυ-νη-γε-τι-κός}

\hyphenation{πλε-όν-των γε-ωρ-γὸς ἕ-τε-ρον}

\hyphenation{κα-θεῖ-ναι προ-ελ-θοῦ-σαν ἀ-πο-δοῦ-ναι ἀ-πο-φαί-νον-τος ἀ-πεσ-τά-λη ἅ-παν-τας τρί-πο-δα τρί-πο-δος κα-θι-ερώ-θη}

\hyphenation{χρεί-αν παν-τὸς ἀν-επί-σκεπ-τον ἀ-πο-θα-νόν-των πα-ρα-σκευ-ά-ζον-τας φρον-τί-ζον-τας πολ-λοὺς κτω-μέ-νους ἐ-λατ-τοῦσ-θαι ἀ-με-λοῦν-τας ἀ-θε-ρά-πευ-τον κτη-μά-των πει-ρω-μέ-νους ὀ-λι-γω-ροῦν-τας δε-ο-μέ-νων ἐ-ῶν-τας ἔ-μοι-γε}

\hyphenation{αὐ-το-κρά-το-ρας Λά-μα-χον ποι-ή-σαν-τες ἑξή-κον-τα Σε-λι-νουν-τί-ους χρη-μά-των μισ-θόν Νι-κη-ρά-του πε-ρι-γίγ-νη-ται}

\hyphenation{cije-nio ok-lije-va-nje ne-kon-ven-ci-o-nal-nu}

\begin{document}

\title{Čitanka za Grčku morfologiju 1}

\author{Irena Bratičević, Nina Čengić, Neven Jovanović,\\Vlado Rezar, Petra Šoštarić, Ninoslav Zubović}
\date{}
%\secondpage


\clearpage

\tableofcontents

\thispagestyle{empty}


%\frontmatter


\chapter*{Predgovor}
\label{chap:predgovor}
\addcontentsline{toc}{chapter}{\nameref{chap:predgovor}}

\section*{O ovoj čitanci}

Ovaj izbor komentiranih tekstova namijenjen je upoznavanju studenata prve godine studija grčkog jezika i književnosti s izvornom starogrčkom prozom i dio je literature obvezatnog kolegija \textit{Grčka morfologija 1}. Izbor uključuje autore poput Herodota i Ezopa, Platona, Ksenofonta i Tukidida, Aristotela, Lizije, Izokrata i Demostena, pa sve do Polibija, Marka Aurelija, Filostrata, Ahileja Tacija i Plotina; zastupljeni su historiografija i filozofska proza, govorništvo, pripovijesti i basne.

Zadatak je studenata da samostalno prirede svaki tekst – da ga na nastavi budu sposobni pročitati i prevesti na hrvatski te prepoznati i morfološki opisati imenske oblike koje susreću.

Za uspješniji samostalni rad kratkim su odlomcima dodane tri vrste komentara. U uvodu su predstavljeni autor (kad se prvi put pojavljuje) i tekst iz kojeg odlomak potječe. Nakon grčkog odlomka slijede njegove rečenice, ovaj put raščlanjene i složene tako da povećaju sintaktičku i semantičku preglednost; posebno su istaknute konstrukcije genitiva apsolutnog i akuzativa, odnosno nominativa s infinitivom. Treći komentar donosi gramatička objašnjenja uz rečenice. Objašnjenja su ovdje dvojaka. Za imenske riječi (član, imenice, pridjevi, zamjenice) donose se samo brojevi paragrafa koji upućuju na relevantne dijelove \textit{Grčke gramatike} Augusta Musića i Nikole Majnarića, standardnoga školskog priručnika za opis grčkog jezika. Glagolski su oblici, međutim, identificirani vrlo detaljno – naveden je njihov rječnički oblik i opisane su sve gramatičke kategorije. 

Nepromjenjive riječi, poput priloga i veznika, najčešće nisu komentirane; studenti će ih naći u pomagalu poput \textit{Grčko-hrvatskog rječnika} Gorskog i Majnarića, ili na internetskome rječničkom portalu \textit{Logeion} Sveučilišta u Chicagu. U tim će referentnim pomagalima studenti tražiti i značenja imenskih riječi.

\newpage

Izbor su sastavili i komentare priredili nastavnici Odsjeka za klasičnu filologiju Filozofskog fakulteta Sveučilišta u Zagrebu (abecednim redom): Irena Bratičević, Nina Čengić, Neven Jovanović, Vlado Rezar, Petra Šoštarić, Ninoslav Zubović.

Čitanka je priređena računalnim programima za slaganje teksta \LaTeX\ i \XeLaTeX. Izvorni kod dostupan je u repozitoriju Github, na URL adresi \url{https://github.com/nevenjovanovic/grcka-morfologija}.

\medskip

U Zagrebu rujan 2018. – rujan 2019.

\newpage



\section*{Upute za rad}

\begin{enumerate}[label=\alph*)]
\item brojevi paragrafa (npr.\ §~127) upućuju na odgovarajući dio Musić-Majnarićeve gramatike koji obrađuje morfologiju imena ili sintaksu
\item glagoli su navedeni u rječničkom obliku i morfološki analizirani
\item članovi konstrukcije genitiva apsolutnog (GA) podcrtani su \\
\uuline{dvjema crtama}
\item članovi konstrukcije akuzativa s infinitivom (A+I) podcrtani su \\ \underline{jednom crtom}
\item nepromjenjive riječi (poput priloga i veznika) koje nisu komentirane potražite u rječniku
\end{enumerate}


\section*{Obvezatna referentna literatura}

Oton Gorski, Niko Majnarić, \textit{Grčko-hrvatski rječnik}, Zagreb (bilo koje izdanje)\\
\textit{Grčka morfologija 1}, elektronička inačica kolegija, Sustav učenja na daljinu Omega, Zagreb: Filozofski fakultet Sveučilišta u Zagrebu, pristupljeno 23. kolovoza 2018. na adresi \url{https://omega.ffzg.hr/course/view.php?id=1615}\\
\textit{Logeion}. Pristupljeno 23. kolovoza 2018. na adresi \url{http://logeion.uchicago.edu/}\\
August Musić, Nikola Majnarić, \textit{Gramatika grčkoga jezika}, Zagreb (bilo koje izdanje)\\
Stjepan Senc, \textit{Grčko-hrvatski rječnik}, Zagreb (bilo koje izdanje)\\
Herbert Weir Smyth, \textit{A Greek Grammar for Colleges}, Perseus Digital Library. Pristupljeno 23. prosinca 2018. na adresi \url{http://www.perseus.tufts.edu}\\


\newpage

%\mainmatter
\chapter[Αἰσώπειος μῦθος 216]{\textgreek[variant=ancient]{Αἰσώπειος μῦθος} 216}

\import{poglavlja/}{pogl01-ezop-216.tex}

\chapter[Πλάτωνος Μενέξενος]{\textgreek[variant=ancient]{Πλάτωνος Μενέξενος} 237c}

\import{poglavlja/}{pogl02-platon-meneksen237c.tex}

\chapter[Λουκιανοῦ Δὶς κατηγορούμενος ]{\textgreek[variant=ancient]{Λουκιανοῦ Δὶς κατηγορούμενος } 2, 9}

\import{poglavlja/}{pogl03-lukijan-dvaput-2-9.tex}


\chapter[Θουκυδίδου Ἱστορίαι Α]{\textgreek[variant=ancient]{Θουκυδίδου Ἱστορίαι Α} 18, 1}

\import{poglavlja/}{pogl04-tukidid-1-18-1.tex}


\chapter[Αἰσώπειος μῦθος 83]{\textgreek[variant=ancient]{Αἰσώπειος μῦθος} 83}

\import{poglavlja/}{pogl05-ezop-83.tex}

\chapter[Πλάτωνος Φαίδων]{\textgreek[variant=ancient]{Πλάτωνος Φαίδων} 85a}

\import{poglavlja/}{pogl06-platon-fedon85a.tex}

\chapter[Ἰσοκράτους Βούσιρις]{\textgreek[variant=ancient]{Ἰσοκράτους Βούσιρις} 28, 1}

\import{poglavlja/}{pogl07-izokrat-buzirid28.tex}

\chapter[Θουκυδίδου Ἱστορίαι Ϛ]{\textgreek[variant=ancient]{Θουκυδίδου Ἱστορίαι Ϛ} 30, 1}

\import{poglavlja/}{pogl08-tukidid-6-30-1.tex}

\chapter[Λυσίου Κατὰ Ἀγοράτου]{\textgreek[variant=ancient]{Λυσίου Κατὰ Ἀγοράτου} 39}

\import{poglavlja/}{pogl09-lizija-agorat-39-1.tex}

\chapter[Ἰσοκράτους Εὐαγόρας]{\textgreek[variant=ancient]{Ἰσοκράτους Εὐαγόρας} 9}

\import{poglavlja/}{pogl10-izokrat-euagora-9.tex}

\chapter[Πλάτωνος Κριτίας]{\textgreek[variant=ancient]{Πλάτωνος Κριτίας} 114e}

\import{poglavlja/}{pogl11-platon-kritija-114e.tex}

\chapter[Ἰσοκράτους Πρὸς Δημόνικον]{\textgreek[variant=ancient]{Ἰσοκράτους Πρὸς Δημόνικον} 16}

\import{poglavlja/}{pogl12-izokrat-demoniku-16.tex}

\chapter[Ἀρριανοῦ Ἀναβάσεως Ἀλεξάνδρου Β]{\textgreek[variant=ancient]{Ἀρριανοῦ Ἀναβάσεως Ἀλεξάνδρου Β,} 3, 6}

\import{poglavlja/}{pogl13-arijan-anabaza-2-3-6.tex}

\chapter[Λουκιανοῦ Θεῶν διάλογοι]{\textgreek[variant=ancient]{Λουκιανοῦ Θεῶν διάλογοι Κ} 7}

\import{poglavlja/}{pogl14-lukijan-razgovoribogova-20-7.tex}

\chapter[Πλουτάρχου Περὶ ἀδολεσχίας]{\textgreek[variant=ancient]{Πλουτάρχου Περὶ ἀδολεσχίας} 513a}

\import{poglavlja/}{pogl15-plutarh-obrbljavosti-513a.tex}

\chapter[Ξενοφῶντος Συμπόσιον Δ]{\textgreek[variant=ancient]{Ξενοφῶντος Συμπόσιον Δ} 34, 4}

\import{poglavlja/}{pogl16-ksenofont-simpozij-4-34-4.tex}

\chapter[Λουκιανοῦ Ἰκαρομένιππος]{\textgreek[variant=ancient]{Λουκιανοῦ Ἰκαρομένιππος \\ἢ Ὑπερνέφελος} 25, 3}

\import{poglavlja/}{pogl17-lukijan-ikaromenip-25-3.tex}

\chapter[Πλουτάρχου Σόλων]{\textgreek[variant=ancient]{Πλουτάρχου Βίοι παράλληλοι, Σόλων} 4, 3}

\import{poglavlja/}{pogl18-plutarh-solon-4-3.tex}

\chapter[Ξενοφῶντος Ἀπομνημονεύματα Β]{\textgreek[variant=ancient]{Ξενοφῶντος Ἀπομνημονεύματα Β} 4, 1}

\import{poglavlja/}{pogl19-ksenofont-uspomene-2-4.tex}



\chapter[Θουκυδίδου Ἱστορίαι Ϛ]{\textgreek[variant=ancient]{Θουκυδίδου Ἱστορίαι Ϛ} 8, 1}

\import{poglavlja/}{pogl20-tukidid-6-8-1.tex}

% 20

\chapter[Ἰσοκράτους Ἑλένης ἐγκώμιον]{\textgreek[variant=ancient]{Ἰσοκράτους Ἑλένης ἐγκώμιον} 35}

\import{poglavlja/}{pogl01-izokrat-helena86.tex}

\chapter[Λυσίου Ὑπὲρ τοῦ ἀδυνάτου]{\textgreek[variant=ancient]{Λυσίου Ὑπὲρ τοῦ ἀδυνάτου} 6, 2}

\import{poglavlja/}{pogl02-lizija-invalid-6-2.tex}

\chapter[Δημοσθένους Ὀλυνθιακός Α]{\textgreek[variant=ancient]{Δημοσθένους Ὀλυνθιακός Α,} 15}

\import{poglavlja/}{pogl03-demosten-olint1-15.tex}

\chapter[Ψευδο-Ξενοφῶντος Ἀθηναίων πολιτεία Β]{\textgreek[variant=ancient]{Ψευδο-Ξενοφῶντος Ἀθηναίων πολιτεία \\Β} 14, 12}

\import{poglavlja/}{pogl04-ps-ksen-2-14-2.tex}

\chapter[Ἀριστοτέλους Ῥητορική]{\textgreek[variant=ancient]{Ἀριστοτέλους Ῥητορική} 1393b}

\import{poglavlja/}{pogl05-aristotel-retorika1393b.tex}

\chapter[Αἰσώπειος μῦθος 1]{\textgreek[variant=ancient]{Αἰσώπειος μῦθος} 1}

\import{poglavlja/}{pogl06-ezop-basne1.tex}

\chapter[Ἀριστοτέλους Ῥητορική]{\textgreek[variant=ancient]{Ἀριστοτέλους Ῥητορική} 1358a}

\import{poglavlja/}{pogl07-aristotel-retorika1358a.tex}

\chapter[Γοργίου Ἑλένης ἐγκώμιον]{\textgreek[variant=ancient]{Γοργίου Ἑλένης ἐγκώμιον} 11, 13}

\import{poglavlja/}{pogl08-gorgija-pohvala-11.tex}

\chapter[Πολυβίου Ἱστορίων Γ]{\textgreek[variant=ancient]{Πολυβίου Ἱστορίων Γ,} 111, 3}

\import{poglavlja/}{pogl09-polibije-povijest-3.tex}

\chapter[Πλουτάρχου Θεμιστοκλῆς]{\textgreek[variant=ancient]{Πλουτάρχου Βίοι παράλληλοι, \\Θεμιστοκλῆς} 2}

\import{poglavlja/}{pogl10-plutarh-temistoklo-2.tex}

\chapter[Λουκιανοῦ Κυνικός]{\textgreek[variant=ancient]{Λουκιανοῦ Κυνικός} 18}

\import{poglavlja/}{pogl11-lukijan-cinik-18.tex}

\chapter[Πλουτάρχου Δημοσθένης]{\textgreek[variant=ancient]{Πλουτάρχου Βίοι παράλληλοι, \\Δημοσθένης} 11}

\import{poglavlja/}{pogl12-plutarh-demosten-11.tex}

\chapter[Ξενοφῶντος Ἀπομνημονεύματα Γ]{\textgreek[variant=ancient]{Ξενοφῶντος Ἀπομνημονεύματα Γ} 12, 4}

\import{poglavlja/}{pogl13-ksenofont-uspomene-3-12.tex}

\chapter[Ἰσοκράτους Ἑλένης ἐγκώμιον]{\textgreek[variant=ancient]{Ἰσοκράτους Ἑλένης ἐγκώμιον} 18}

\import{poglavlja/}{pogl14-izokrat-pohvalahelene-18.tex}

\chapter[Δημοσθένους Ὀλυνθιακός Β]{\textgreek[variant=ancient]{Δημοσθένους Ὀλυνθιακός Β,} 18}

\import{poglavlja/}{pogl15-demosten-olintski2-18.tex}

\chapter[Ἀριστοτέλους Ῥητορική]{\textgreek[variant=ancient]{Ἀριστοτέλους Ῥητορική} 1394a}

\import{poglavlja/}{pogl16-aristotel-retorika-1394a.tex}

\chapter[Αἰσώπειος μῦθος 35]{\textgreek[variant=ancient]{Αἰσώπειος μῦθος} 35}

\import{poglavlja/}{pogl17-ezop-35.tex}

\chapter[Γοργίου Ἑλένης ἐγκώμιον 11, 41]{\textgreek[variant=ancient]{Γοργίου Ἑλένης ἐγκώμιον} 11, 41}

\import{poglavlja/}{pogl18-gorgija-pohvalahelene-11-41.tex}

\chapter[Πλουτάρχου Ἀγεσίλαος]{\textgreek[variant=ancient]{Πλουτάρχου Βίοι παράλληλοι, \\Ἀγεσίλαος} 6}

\import{poglavlja/}{pogl19-plutarh-agesilaj-6.tex}

\chapter[Πλουτάρχου Περὶ παίδων ἀγωγῆς]{\textgreek[variant=ancient]{Πλουτάρχου Περὶ παίδων ἀγωγῆς} 9}

\import{poglavlja/}{pogl20-plutarh-liberiseducandis-2c.tex}

%40
\chapter[Διοδώρου Βιβλιοθήκης ἱστορικής Α]{\textgreek[variant=ancient]{Διοδώρου Σικελιώτου \\Βιβλιοθήκης ἱστορικής Α,} 8, 5}

\import{poglavlja/}{pogl01-diodor-knjiznica.tex}

\chapter[Ἀριστοτέλους Ῥητορική]{\textgreek[variant=ancient]{Ἀριστοτέλους Ῥητορική} 1393a}

\import{poglavlja/}{pogl02-aristotel-retorika-1393a.tex}

\chapter[Θουκυδίδου Ἱστορίαι Α 2]{\textgreek[variant=ancient]{Θουκυδίδου Ἱστορίαι Α} 2}

\import{poglavlja/}{pogl03-tukidid-1-2.tex}

\chapter[Ἡροδότου Ἱστορίαι Α 141]{\textgreek[variant=ancient]{Ἡροδότου Ἱστορίαι Α} 141}

\import{poglavlja/}{pogl04-herodot-1-41.tex}

\chapter[Αἰσώπειος μῦθος 9]{\textgreek[variant=ancient]{Αἰσώπειος μῦθος} 9}

\import{poglavlja/}{pogl05-ezop-9.tex}

\chapter[Ἡλιοδώρου Αἰϑιοπικὰ Α]{\textgreek[variant=ancient]{Ἡλιοδώρου Αἰϑιοπικὰ Α} 2}

\import{poglavlja/}{pogl06-heliodor-aethiop-1.tex}

\chapter[Θουκυδίδου Ἱστορίαι Β]{\textgreek[variant=ancient]{Θουκυδίδου Ἱστορίαι Β} 65}

\import{poglavlja/}{pogl07-tukidid-2-65.tex}

\chapter[Ἰσοκράτους Πανηγυρικός ]{\textgreek[variant=ancient]{Ἰσοκράτους Πανηγυρικός} 47}

\import{poglavlja/}{pogl08-izokrat-panegirik-47.tex}

\chapter[Ἡροδότου Ἱστορίαι Β]{\textgreek[variant=ancient]{Ἡροδότου Ἱστορίαι Β} 77, 12}

\import{poglavlja/}{pogl09-herodot-2-77.tex}

\chapter[Θουκυδίδου Ἱστορίαι Ϛ 27]{\textgreek[variant=ancient]{Θουκυδίδου Ἱστορίαι Ϛ} 27}

\import{poglavlja/}{pogl10-tukidid-6-27.tex}

\chapter[Λουκιανοῦ Περὶ τοῦ Ἐνυπνίου]{\textgreek[variant=ancient]{Λουκιανοῦ Περὶ τοῦ Ἐνυπνίου ἤτοι Βίος Λουκιανοῦ} 9}

\import{poglavlja/}{pogl20-lukijan-san-9.tex}

\chapter[Ἀντιφῶντος Περὶ ὁμονοίας]{\textgreek[variant=ancient]{Ἀντιφῶντος Περὶ ὁμονοίας} 14}

\import{poglavlja/}{pogl11-antifont-ulomak-14.tex}

\chapter[Φιλοστράτου Tὰ εἰς τὸν Τυανέα Ἀπολλώνιον Ε]{\textgreek[variant=ancient]{Φλαουίου Φιλοστράτου Tὰ εἰς τὸν Τυανέα Ἀπολλώνιον Ε} 14}

\import{poglavlja/}{pogl12-filostrat-apolonije-5-14.tex}

\chapter[Ἐγχειρίδιον Ἐπικτήτου]{\textgreek[variant=ancient]{Ἐγχειρίδιον Ἐπικτήτου} 1}

\import{poglavlja/}{pogl13-epiktet-prirucnik-1.tex}

\chapter[Ξενοφῶντος Ἀπομνημονεύματα Β]{\textgreek[variant=ancient]{Ξενοφῶντος Ἀπομνημονεύματα Β} 4, 5}

\import{poglavlja/}{pogl14-ksen-memor-2-4-5.tex}

\chapter[ Ἀχιλλέως Τατίου Tὰ κατὰ Λευκίππην καὶ Κλειτοφῶντα]{\textgreek[variant=ancient]{ Ἀχιλλέως Τατίου Tὰ κατὰ Λευκίππην καὶ Κλειτοφῶντα Α} 41}

\import{poglavlja/}{pogl15-ahilejt-6-19.tex}

\chapter[Λυσίου Ὀλυμπιακός]{\textgreek[variant=ancient]{Λυσίου Ὀλυμπιακός} 1}

\import{poglavlja/}{pogl16-lizija-olimp-1.tex}

\chapter[Μάρκου Ἀντωνίνου Τῶν εἰς ἑαυτὸν Β]{\textgreek[variant=ancient]{Μάρκου Ἀντωνίνου αὐτοκράτορος \\Τῶν εἰς ἑαυτὸν βιβλίον Β} 11}

\import{poglavlja/}{pogl17-markaur-2-11.tex}

\chapter[Πλωτίνου Ἐννεάδες Ϛ]{\textgreek[variant=ancient]{Πλωτίνου Ἐννεάδες Ϛ} 7, 34}

\import{poglavlja/}{pogl18-plotin-eneade-6-7-34.tex}

\chapter[Μάρκου Ἀντωνίνου Τῶν εἰς ἑαυτὸν Γ]{\textgreek[variant=ancient]{Μάρκου Ἀντωνίνου αὐτοκράτορος \\Τῶν εἰς ἑαυτὸν βιβλίον Γ} 1}

\import{poglavlja/}{pogl19-markaur-3-1.tex}




%%%%%%%%%%%%%%%%
%\backmatter




\end{document}
\clearpage





% kraj
