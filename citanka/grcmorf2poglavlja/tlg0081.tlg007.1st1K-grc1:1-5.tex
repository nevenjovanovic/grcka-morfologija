%\section*{O autoru}



\section*{O tekstu}

Slavni grčki retor i sofist iz Leontina na Siciliji (oko 485.\ – oko 380.\ pr.~Kr) artificijelnim je i intelektualističkim stilom začetnik antičke umjetničke proze. Ovdje donesen ulomak Gorgijina nadgrobnog govora – ne zna se kojim je u ratu junački palim Atenjanima bio namijenjen, možda se radilo naprosto o demonstraciji mogućnosti žanra – sačuvan je kao citat, reprezentativan za Gorgijine λόγοι ἐπιδεικτικοί, u djelu Dionizija Halikarnašanina (oko 60.\ – nakon 7.\ pr.~Kr) Περὶ λεκτικῆς Δημοσθένους δεινότητος, \textit{O snazi Demostenova stila}.

Gorgija hvali ljude kojima nijedna ljudska vrlina nije nedostajala, koji su štovali duh, a ne slovo zakona, koji su se odlikovali i znanjem i snagom, davali svakome što ga ide. Oni su umrli, ali čežnja za njima je besmrtna.


%\newpage

\section*{Pročitajte naglas grčki tekst.}

Gorg.\ Fr.\ 6

%Naslov prema izdanju

\medskip


{\large

\begin{greek}

\noindent τί γὰρ ἀπῆν τοῖς ἀνδράσι τούτοις, ὧν δεῖ ἀνδράσι προσεῖναι; τί δὲ προσῆν, ὧν δεῖ ἀπεῖναι; εἰπεῖν δυναίμην, ἃ βούλομαι, βουλοίμην δέ, ἃ δεῖ, λαθὼν μὲν τὴν θείαν νέμεσιν, φυγὼν δὲ τὸν ἀνθρώπινον φθόνον. οὗτοι γὰρ ἐκέκτηντο ἔνθεον μὲν τὴν ἀρετήν, ἀνθρώπινον δὲ τὸ θνητόν, πολλὰ μὲν δὴ τὸ παρὸν ἐπιεικὲς τοῦ αὐθάδους δικαίου προκρίνοντες, πολλὰ δὲ νόμου ἀκριβείας λόγων ὀρθότητα, τοῦτον νομίζοντες θειότατον καὶ κοινότατον νόμων τὸ δέον ἐν τῷ δέοντι καὶ λέγειν καὶ σιγᾶν καὶ ποιεῖν, δισσὰ ἀσκήσαντες μάλιστα ὧν δεῖ, γνώμην καὶ ῥώμην, τὴν μὲν βουλεύοντες τὴν δʼ ἀποτελοῦντες, θεράποντες μὲν τῶν ἀδίκως δυστυχούντων, κολασταὶ δὲ τῶν ἀδίκως εὐτυχούντων, αὐθάδεις πρὸς τὸ συμφέρον, εὐόργητοι πρὸς τὸ πρέπον, τῷ φρονίμῳ τῆς γνώμης παύοντες τὸ ἄφρον, ὑβρισταὶ εἰς τοὺς ὑβρίζοντας, κόσμιοι εἰς τοὺς κοσμίους, ἄφοβοι εἰς τοὺς ἀφόβους, δεινοὶ ἐν τοῖς δεινοῖς. μαρτύρια δὲ τούτων τρόπαια ἐστήσαντο τῶν πολεμίων Διὸς μὲν ἀγάλματα τούτων δὲ ἀναθήματα, οὐκ ἄπειροι οὔτε ἐμφύτου Ἄρεως οὔτε νομίμων ἐρώτων οὔτε ἐνοπλίου ἔριδος οὔτε φιλοκάλου εἰρήνης, σεμνοὶ μὲν πρὸς τοὺς θεοὺς τῷ δικαίῳ, ὅσιοι δὲ πρὸς τοὺς τοκέας τῇ θεραπείᾳ, δίκαιοι δὲ πρὸς τοὺς ἀστοὺς τῷ ἴσῳ, εὐσεβεῖς δὲ πρὸς τοὺς φίλους τῇ πίστει. τοιγαροῦν αὐτῶν ἀποθανόντων ὁ πόθος οὐ συναπέθανεν, ἀλλʼ ἀθάνατος ἐν οὐκ ἀθανάτοις σώμασι ζῇ οὐ ζώντων.

\end{greek}

}


\section*{Analiza i komentar}

%1

{\large
\begin{greek}
\noindent τί γὰρ ἀπῆν \\
τοῖς ἀνδράσι τούτοις \\
\tabto{2em} ὧν δεῖ \\
\tabto{4em} ἀνδράσι προσεῖναι;

\noindent τί δὲ καὶ προσῆν \\
\tabto{2em} ὧν οὐ δεῖ \\
\tabto{4em} προσεῖναι;\\
\end{greek}
}

\begin{description}[noitemsep]
\item[τί γὰρ] čestica γάρ uz upitnu zamjenicu u direktnom pitanju §~517
\item[ἀπῆν] §~315.2, složenica εἰμί
\item[ὧν δεῖ] odnosna zamjenica ὧν uvodi zavisnu odnosnu rečenicu i istovremeno funkcionira kao dijelni genitiv uz τί: od onoga što\dots
\item[δεῖ] §~244, osnove §~325.14, bezlično δεῖ otvara mjesto dopuni u infinitivu
\item[προσεῖναι] §~315.2, složenica εἰμί
\item[τί δὲ ] česticom se ostvaruje koordinacija s prethodnom rečenicom
\item[προσῆν] §~315.2, složenica εἰμί (uočite i rimu s τί γὰρ ἀπῆν)
\item[ὧν οὐ δεῖ] odnosna zamjenica kao gore
\item[δεῖ] kao gore
\item[προσεῖναι] kao gore

\end{description}

%3


{\large
\begin{greek}
\noindent εἰπεῖν δυναίμην \\
\tabto{2em} ἃ βούλομαι, \\
βουλοίμην δ' \\
\tabto{2em} ἃ δεῖ, \\
λαθὼν μὲν τὴν θείαν νέμεσιν, \\
φυγὼν δὲ τὸν ἀνθρώπινον φθόνον.\\

\end{greek}
}

\begin{description}[noitemsep]
\item[εἰπεῖν] §~254, osnove §~327.7
\item[δυναίμην] §~312.5, glagol nepotpuna značenja otvara mjesto dopuni u infinitivu; optativom se izriče mogućnost
\item[ἃ βούλομαι] odnosna zamjenica uvodi zavisnu odnosnu rečenicu i ujedno funkcionira kao objekt glavne: ono što\dots
\item[βούλομαι] §~231, osnove §~325.13
\item[βουλοίμην] §~231, osnove §~325.13
\item[δ'] čestica u adverzativnom značenju: a\dots
\item[ἃ δεῖ] odnosna zamjenica uvodi zavisnu odnosnu rečenicu, kao gore
\item[δεῖ] §~244, osnove §~325.14
\item[λαθὼν μὲν\dots\ φυγὼν δὲ\dots] rečenični dijelovi koordiniraju se parom čestica
\item[λαθὼν] §~254, osnove §~321.15, rekcija τινά §~382
\item[φυγὼν ] §~254, osnove s.~116, rekcija τινά §~382
\end{description}

%4


{\large
\begin{greek}
\noindent οὗτοι γὰρ ἐκέκτηντο \\
ἔνθεον μὲν τὴν ἀρετήν, \\
ἀνθρώπινον δὲ τὸ θνητόν, \\
πολλὰ μὲν δὴ \\
\tabto{2em} τὸ πρᾶον ἐπιεικὲς \\
\tabto{2em} τοῦ αὐθάδους δικαίου \\
\tabto{4em} προκρίνοντες, \\
πολλὰ δὲ \\
\tabto{2em} νόμου ἀκριβείας \\
\tabto{2em} λόγων ὀρθότητα, \\
τοῦτον\\
\tabto{2em} νομίζοντες \\
θειότατον καὶ κοινότατον νόμον, \\
\tabto{4em} τὸ δέον \\
\tabto{4em} ἐν τῷ δέοντι \\
\tabto{4em} καὶ λέγειν καὶ σιγᾶν καὶ ποιεῖν, \\
καὶ δισσὰ ἀσκήσαντες \\
\tabto{2em} μάλιστα ὧν δεῖ, \\
\tabto{4em} γνώμην καὶ ῥώμην, \\
\tabto{6em} τὴν μὲν βουλεύοντες \\
\tabto{6em} τὴν δ' ἀποτελοῦντες, \\
θεράποντες μὲν \\
\tabto{2em} τῶν ἀδίκως δυστυχούντων, \\
κολασταὶ δὲ \\
\tabto{2em} τῶν ἀδίκως εὐτυχούντων, \\
αὐθάδεις πρὸς τὸ συμφέρον, \\
εὐόργητοι πρὸς τὸ πρέπον, \\
\tabto{2em} τῷ φρονίμῳ \\
\tabto{4em} τῆς γνώμης \\
παύοντες \\
\tabto{2em} τὸ ἄφρον, \\
ὑβρισταὶ εἰς τοὺς ὑβριστάς, \\
κόσμιοι εἰς τοὺς κοσμίους, \\
ἄφοβοι εἰς τοὺς ἀφόβους, \\
δεινοὶ ἐν τοῖς δεινοῖς. \\

\end{greek}
}

\begin{description}[noitemsep]
\item[γὰρ] čestica γάρ ovdje ističe prethodnu riječ §~519.1
\item[ἐκέκτηντο] §~272
\item[ἔνθεον μὲν δὴ\dots\ ἀνθρώπινον δὲ\dots] rečenični članovi koordiniraju se parom čestica
\item[πολλὰ μὲν δὴ\dots\ πολλὰ δὲ\dots] rečenični članovi koordiniraju se parom čestica; δή označava nadovezivanje
\item[προκρίνοντες] §~231, rekcija τί τινος, osnove s.~118, složenica glagola κρίνω
\item[τοῦτον] zamjenica najavljuje dopunu koja slijedi (τὸ δέον\dots)
\item[νομίζοντες] §~231, otvara mjesto dvama akuzativima
\item[τὸ δέον] §~244; otvara mjesto dopuni u infinitivu; supstantiviranje participa članom §~499
\item[τῷ δέοντι] §~244
\item[λέγειν] §~231, osnove §~327.7
\item[καὶ σιγᾶν καὶ ποιεῖν] §~243
\item[ἀσκήσαντες] §~267
\item[ὧν δεῖ] odnosna zamjenica uvodi zavisnu odnosnu rečenicu; antecedent je δισσὰ
\item[δεῖ] §~244
\item[τὴν μὲν\dots\ τὴν δ'\dots] sc.\ γνώμην καὶ ῥώμην; rečenični članovi koordiniraju se parom čestica
\item[βουλεύοντες] §~231
\item[ἀποτελοῦντες] §~243, složenica τελέω
\item[θεράποντες μὲν\dots\ κολασταὶ δὲ\dots] suprotstavljeni rečenični članovi koordiniraju se parom čestica
\item[τῶν δυστυχούντων\dots\ τῶν εὐτυχούντων] §~243; supstantivirani particip  §~499.2
\item[τὸ συμφέρον] §~231; složenica φέρω; supstantivirani particip §~499. 2
\item[τὸ πρέπον] §~231; supstantivirani particip §~499.2
\item[παύοντες] §~231
\end{description}

%5

{\large
\begin{greek}
\noindent μαρτύρια δὲ τούτων \\
τρόπαια ἐστήσαντο \\
\tabto{2em} τῶν πολεμίων, \\
Διὸς μὲν ἀγάλματα, \\
ἑαυτῶν δὲ ἀναθήματα, \\
\tabto{2em} οὐκ ἄπειροι \\
\tabto{4em} οὔτε ἐμφύτου Ἄρεως\\
\tabto{4em} οὔτε νομίμων ἐρώτων \\
\tabto{4em} οὔτε ἐνοπλίου ἔριδος \\
\tabto{4em} οὔτε φιλοκάλου εἰρήνης, \\
σεμνοὶ μὲν \\
\tabto{2em} πρὸς τοὺς θεοὺς \\
\tabto{2em} τῷ δικαίῳ, \\
ὅσιοι δὲ \\
\tabto{2em} πρὸς τοὺς τοκέας \\
\tabto{2em} τῇ θεραπείᾳ, \\
δίκαιοι δὲ \\
\tabto{2em} πρὸς τοὺς ἀστοὺς \\
\tabto{2em} τῷ ἴσῳ, \\
εὐσεβεῖς δὲ \\
\tabto{2em} πρὸς τοὺς φίλους \\
\tabto{2em} τῇ πίστει.\\

\end{greek}
}

\begin{description}[noitemsep]
\item[δὲ] čestica ovdje s blagim adverzativnim značenjem: a\dots
\item[ἐστήσαντο] §~311.2
\item[Διὸς μὲν\dots\ ἑαυτῶν δὲ] rečenični članovi koordiniraju se česticama
\item[ἄπειροι] rekcija τινός; pridjev otvara mjesto dopunama u genitivu (objektni genitiv)
\item[οὔτε\dots\ οὔτε\dots\ οὔτε\dots\ οὔτε\dots] polisindeton
\item[σεμνοὶ μὲν\dots\ ὅσιοι δὲ\dots\ δίκαιοι δὲ\dots\ εὐσεβεῖς δὲ\dots] rečenični članovi koordiniraju se česticama; u ovakvoj koordinaciji δέ se može ponavljati više puta
\end{description}

%6

{\large
\begin{greek}
\noindent τοιγαροῦν αὐτῶν ἀποθανόντων \\
ὁ πόθος \\
οὐ συναπέθανεν, \\
ἀλλ' ἀθάνατος \\
\tabto{2em} οὐκ ἐν ἀθανάτοις σώμασι \\
ζῇ \\
\tabto{2em} οὐ ζώντων. \\

\end{greek}
}

\begin{description}[noitemsep]
\item[τοιγαροῦν] čestica zaključnoga značenja, §~516.4: stoga\dots
\item[ἀποθανόντων] §~254; osnove §~324.8
\item[οὐ συναπέθανεν] §~254, složenica ἀποθνῄσκω, osnove §~324.8
\item[ἀλλ'] veznik ἀλλά uvodi nezavisnu suprotnu rečenicu §~515
\item[ζῇ] §~243
\item[ζώντων] §~243; genitiv objektni ovisan o πόθος

\end{description}

%kraj
