% unesi ispravke NJ 30. 3. 2020.
% unio ispravke NZ <2021-12-30 čet>
%\section*{O autoru}

%TKTK


\section*{O tekstu}

U demokratskoj je Ateni najvažnije političke odluke donosila skupština \textgreek[variant=ancient]{(ἐκκλησία),} tijelo koje se sastajalo otprilike četrdeset puta godišnje. U njemu je mogao sudjelovati svaki odrasli atenski građanin; obično bi se na sjednicama okupljalo njih oko šest tisuća, približno četvrtina ukupnog broja. Skupštini se, istupajući protiv Filipa Makedonskog, Demosten obraća u osam nedvojbeno autentičnih govora. Svi oni potječu iz desetljeća 351.–341. Tri se bave sukobom Filipa s gradom Olintom, te se nazivaju \textit{Olintskim govorima} \textgreek[variant=ancient]{(Ὀλυνθιακοὶ λόγοι).}

Olint \textgreek[variant=ancient]{(Ὄλυνθος)} je bio grad na Halkidskom poluotoku, najvažniji član Halkidskog saveza, jedine značajne grčke sile na sjevernom Egejskom moru. To je područje za Atenu bilo strateški važno jer je onuda prolazio opskrbni put kojim se s obala Crnog mora uvozilo žito za atensko stanovništvo. 

Mada još uvijek najbogatiji i najmoćniji grad kontinentalne grčke, Atena je u Demostenovo doba patila od ozbiljnih problema – kako vojnih (sramotno je poražena u tzv.\ Savezničkom ratu 357.–355.), tako i financijskih (u četvrtom stoljeću više nije imala prihoda od pomorskog carstva kao u petom, te joj je kronično nedostajalo novaca za sve potrebe, osobito za vojne akcije).

Halkidski je savez isprva, od 357., bio na strani Filipa Makedonskog. No, odnosi saveza i Filipa pogoršali su se u drugoj polovici desetljeća. U bici na Krokrijskom polju \textgreek[variant=ancient]{(Κρόκριον,} 352.) Filip je teško porazio Fokiđane, a neprestano je ratovao u Trakiji. Olinćane je Filipovo jačanje brinulo te su se politički približavali Atenjanima. Tada je i Demosten počeo Filipa smatrati velikom opasnošću za Atenu; \textit{Prva Filipika} vjerojatno je održana 351. Demosten se zalagao za stvaranje jednog stalnog vojnog kontingenta spremnog za intervenciju, i za drugi kontingent koji bi trajno djelovao isključivo na području sjevernog Egejskog mora. Pokušavao je i podići moral Atenjana i potaknuti ih na jači borbeni angažman.

Ujesen 349.\ Filip je upao na područje Halkidske lige, osvojio Toronu i Mekibernu, olintsku luku, i počeo opsadu samog Olinta. Olinćani su zatražili pomoć od Atenjana. Ovi su triput poslali interventne snage; dvaput su bile nedostatne, a treći put već je bilo prekasno. Filipovi pristaše unutar Olinta izdali su savez i prepustili neprijatelju glavninu konjice. Filip je osvojio sve gradove na Halkidskom poluotoku te pobio ili u roblje prodao njihove građane. Sam je Olint razoren i nikad nije ponovo naseljen (tako da je danas, sa stajališta arheologa, najpotpunije očuvan grčki grad).

Sva tri \textit{Olintska govora} nastala su u godini Filipova pohoda na Olint, 349./348. Prva su dva sigurno održana prije atenskih intervencija. Iz posljednjeg, trećeg govora, jasno je da se odnos snaga promijenio na štetu Atenjana. U uvodu govora \textgreek[variant=ancient]{(προοίμιον)} Demosten upozorava sugrađane da su propustili priliku da uzvrate Filipu udarac, i sada moraju braniti sebe i svoje saveznike. U glavnom dijelu \textgreek[variant=ancient]{(ἀπόδειξις)} najprije pripovijeda  \textgreek[variant=ancient]{(διήγησις)} o propuštenim prilikama da se Filip neutralizira te opisuje sadašnju krizu; treba hitno poslati pomoć Olintu, inače će Filip napasti i Atenu. Pitanje je, međutim, kojim sredstvima pomoći Olintu. Demosten predlaže \textgreek[variant=ancient]{(πρόθεσις)} osnivanje posebnog povjerenstva \textgreek[variant=ancient]{(νομοθέται)} radi izmjene zakona o teoričkom fondu \textgreek[variant=ancient]{(θεωρικά).}\footnote{Uspostavljen vjerojatno oko 350., teorički je fond prvotno financirao ulaznice za siromašnije atenske građane na kazališnim predstavama tijekom dionizija i leneja; ubrzo su se sredstva počela koristiti i za druge socijalne namjene, postajući ideološki važna kao „vezivno tkivo atenske demokracije”. Zato je fond bio zaštićen nizom zakona, te mu je namjenu smjelo mijenjati isključivo posebno povjerenstvo. Demosten i njegovi pristaše smatrali su fond simbolom svega što je u Ateni „otišlo ukrivo”; trošilo se na nevažne stvari umjesto da se financirala obrana od Filipa.} Demosten argumentira korisnost svojeg prijedloga (πίστεις). Tu prvenstveno apelira na emocije publike; Filip je neprijatelj i barbarin, Atenjani se mogu sramiti zbog svega što su mu dopustili da postigne. Jednako je sramotno zanemariti ratovanje zbog nedostatka novaca kao i zanemariti državni interes radi zabave. Tako nisu postupali Atenjani slavnoga doba, doba Aristida, Nikije, Perikla; oni su se brinuli za javno dobro, a sami su živjeli skromno. Nasuprot tome, u ovom času ni vanjska ni unutrašnja politika Atene nije dobra. Vode je političari koji ugađaju narodu, a misle na vlastitu korist. U zaključku (ἐπίλογος) Demosten poziva skupštinu da promijene to stanje, da svaki građanin dobiva dio državnih sredstava kao naknadu za služenje državi, u miru i ratu, ili kao administrator ako nije sposoban za rat. To nije niti poticanje nerada, niti preusmjeravanje sredstava onih koji rade u džepove lijenčinama; treba se vratiti praksi prošlih generacija. Govor završava željom da bogovi nadahnu skupštinu da odabere najbolje rješenje.

U ovdje odabranom odlomku, Demosten opisuje kako su se ponašali slavni državnici prošlih generacija, kako su se brinuli za vanjskopolitičko i unutarnjepolitičko opće dobro, ali su sami živjeli skromno; u to se svatko može i danas uvjeriti, uspoređujući njihove kuće s kućama ostalih građana.


%\newpage

\section*{Pročitajte naglas grčki tekst.}

Dem. Olynthiaca III 24–26

%Naslov prema izdanju

\medskip


{\large

\begin{greek}

\noindent  ἐκεῖνοι τοίνυν, οἷς οὐκ ἐχαρίζονθ' οἱ λέγοντες οὐδ' ἐφίλουν αὐτοὺς ὥσπερ ὑμᾶς οὗτοι νῦν, πέντε μὲν καὶ τετταράκοντ' ἔτη τῶν Ἑλλήνων ἦρξαν ἑκόντων, πλείω δ' ἢ μύρια τάλαντ' εἰς τὴν ἀκρόπολιν ἀνήγαγον, ὑπήκουε δ' ὁ ταύτην τὴν χώραν ἔχων αὐτοῖς βασιλεύς, ὥσπερ ἐστὶ προσῆκον βάρβαρον Ἕλλησι, πολλὰ δὲ καὶ καλὰ καὶ πεζῇ καὶ ναυμαχοῦντες ἔστησαν τρόπαι' αὐτοὶ στρατευόμενοι, μόνοι δ' ἀνθρώπων κρείττω τὴν ἐπὶ τοῖς ἔργοις δόξαν τῶν φθονούντων κατέλιπον.

ἐπὶ μὲν δὴ τῶν Ἑλληνικῶν ἦσαν τοιοῦτοι· ἐν δὲ τοῖς κατὰ τὴν πόλιν αὐτὴν θεάσασθ' ὁποῖοι, ἔν τε τοῖς κοινοῖς κἀν τοῖς ἰδίοις. δημοσίᾳ μὲν τοίνυν οἰκοδομήματα καὶ κάλλη τοιαῦτα καὶ τοσαῦτα κατεσκεύασαν ἡμῖν ἱερῶν καὶ τῶν ἐν τούτοις ἀναθημάτων, ὥστε μηδενὶ τῶν ἐπιγιγνομένων ὑπερβολὴν λελεῖφθαι· ἰδίᾳ δ' οὕτω σώφρονες ἦσαν καὶ σφόδρ' ἐν τῷ τῆς πολιτείας ἤθει μένοντες, ὥστε τὴν Ἀριστείδου καὶ τὴν Μιλτιάδου καὶ τῶν τότε λαμπρῶν οἰκίαν εἴ τις ἄρ' οἶδεν ὑμῶν ὁποία ποτ' ἐστίν, ὁρᾷ τῆς τοῦ γείτονος οὐδὲν σεμνοτέραν οὖσαν\dots

\end{greek}

}


\section*{Analiza i komentar}

%1

{\large
\begin{greek}
\noindent ἐκεῖνοι τοίνυν, \\
\tabto{2em} οἷς οὐκ ἐχαρίζονθ' οἱ λέγοντες \\
\tabto{2em} οὐδ' ἐφίλουν αὐτοὺς \\
\tabto{4em} ὥσπερ ὑμᾶς οὗτοι νῦν, \\
πέντε μὲν καὶ τετταράκοντ' ἔτη \\
τῶν Ἑλλήνων ἦρξαν \\
\tabto{2em} ἑκόντων, \\
πλείω δ' ἢ μύρια τάλαντ' \\
\tabto{2em} εἰς τὴν ἀκρόπολιν \\
ἀνήγαγον, \\
ὑπήκουε δ' \\
\tabto{2em} ὁ \\
\tabto{4em} ταύτην τὴν χώραν \\
\tabto{2em} ἔχων \\
αὐτοῖς \\
\tabto{2em} βασιλεύς, \\
ὥσπερ ἐστὶ προσῆκον \\
\tabto{2em} βάρβαρον Ἕλλησι, \\
πολλὰ δὲ καὶ καλὰ \\
\tabto{2em} καὶ πεζῇ καὶ ναυμαχοῦντες \\
ἔστησαν τρόπαι' \\
αὐτοὶ στρατευόμενοι, \\
μόνοι δ' ἀνθρώπων \\
κρείττω \\
τὴν ἐπὶ τοῖς ἔργοις δόξαν \\
\tabto{2em} τῶν φθονούντων \\
κατέλιπον.\\

\end{greek}
}

\begin{description}[noitemsep]
\item[ἐκεῖνοι] sc. οἱ ὑμῶν πρόγονοι
\item[οὐκ ἐχαρίζονθ'\dots\ οὐδ' ἐφίλουν] koordinacija surečenica niječnim veznicima
\item[ἐχαρίζονθ'] §~232; χαρίζομαί τινι; LSJ χαρίζω A
\item[οἱ λέγοντες] ovdje ima značenje οἱ ῥήτορες; §~231; supstantivirani particip §~373
\item[ἐφίλουν] §~243; “mazili su”, značenje između uobičajenih „voljeti” i „ljubiti”
\item[πέντε μὲν\dots] \textbf{πλείω δ'\dots\ ὑπήκουε δ'\dots\ πολλὰ δὲ\dots\ μόνοι δ'\dots}\ koordinacija rečeničnih članova pomoću čestica μέν\dots\ δέ\dots
\item[πέντε μὲν καὶ τετταράκοντ' ἔτη] broj je zaokružen; Demosten misli na razdoblje između uspostave Delskog saveza (478./477.) i izbijanja Peloponeskog rata (Tukididovu \textgreek{πεντηκονταετία)}
\item[ἔστησαν] §~311, §~267
\item[ἦρξαν] §~267, §~269; s. 116; §~401.1%s genitivom
\item[ἀνήγαγον] §~254; složenica glagola ἄγω, §~257
\item[ὑπήκουε] §~231; §~235, §~238; ὑπακούω τινί
\item[ὑπήκουε δ'] \textbf{ὁ ταύτην τὴν χώραν ἔχων αὐτοῖς βασιλεύς} stilski obilježen, umjetnički red riječi (figura konstrukcije po imenu hiperbat, ὑπερβατόν): \textgreek[variant=ancient]{ὑπήκουε δ'\dots\ αὐτοῖς ὁ ταύτην τὴν χώραν ἔχων\dots\ βασιλεύς}
\item[ταύτην τὴν χώραν] sc.\ Μακεδονίαν; u Periklovo je doba makedonski kralj bio Perdika II.
\item[ὁ\dots\ ἔχων\dots\ βασιλεύς] hiperbat
\item[ἔχων] §~231; atributni particip, §~499
\item[ἐστὶ προσῆκον] §~315; §~231; particip kao dopuna predikatu (imenski dio predikata), Smyth 909
\item[βάρβαρον Ἕλλησι] sc.\ ὑπακούειν
\item[ἔστησαν] §~311, §~267
\item[πολλὰ δὲ καὶ καλὰ\dots\ τρόπαι'] hiperbat
\item[αὐτοὶ στρατευόμενοι] §~232; atributni particip, §~499; αὐτοί naglašava da Atenjani trebaju \textit{osobno} ratovati (u ovo su vrijeme inače redovno u intervencije slali plaćenike)
\item[κρείττω\dots\ τῶν φθονούντων] §~231; supstantivirani particip §~373; \textit{genetivus comparationis,} §~404.1; hiperbat
\item[τῶν φθονούντων] umjesto \textgreek{τοῦ φθόνου}; konkretan izraz umjesto apstraktne imenice
\item[κατέλιπον] §~254; složenica glagola λείπω, §~256; dva akuzativa (objekta i predikata), §~388
\end{description}

%2

{\large
\begin{greek}
\noindent ἐπὶ μὲν δὴ τῶν Ἑλληνικῶν \\
ἦσαν τοιοῦτοι· \\
ἐν δὲ τοῖς \\
\tabto{2em} κατὰ τὴν πόλιν αὐτὴν \\
θεάσασθ' ὁποῖοι, \\
ἔν τε τοῖς κοινοῖς \\
κἀν τοῖς ἰδίοις.\\

\end{greek}
}

\begin{description}[noitemsep]
\item[ἐπὶ μὲν δὴ\dots\ ἐν δὲ τοῖς\dots] koordinacija rečeničnih članova pomoću čestica \textgreek[variant=ancient]{μέν\dots\ δέ\dots}; česticom δή uvodi se zaključak, „dakle”, §~516.5
\item[ἐπὶ\dots\ τῶν Ἑλληνικῶν] “što se tiče”; rijetko značenje prijedloga ἐπί, prema izrazima poput ἐπὶ πολλῶν
\item[τῶν Ἑλληνικῶν] \textgreek[variant=ancient]{τὰ Ἑλληνικὰ (πράγματα)} „grčka politika”, usp.\ DGE / Logeion \textgreek[variant=ancient]{Ἑλληνικός} B.I.3
\item[ἦσαν τοιοῦτοι] §~315; zamjenica kao imenski dio predikata, Smyth 909
\item[τοιοῦτοι] odnosi se na prethodno spomenuto; τοιοῦτος je jači oblik zamjenice τοῖος
\item[τοιοῦτοι\dots\ ὁποῖοι] korelacija surečenica ostvarena pokaznim antecedentom i odnosnom zamjenicom
\item[τοῖς κατὰ τὴν πόλιν αὐτὴν] supstantiviran prijedložni izraz, §~373
\item[θεάσασθ'] §~267, §~269
\item[ὁποῖοι] sc.\ ἦσαν
\item[ἔν τε τοῖς κοινοῖς\dots\ κἀν τοῖς ἰδίοις] supstantivirani pridjevi, §~373; κἀν kraza καὶ ἐν; τε enklitični sastavni veznik, §~513.2

\end{description}

%3

{\large
\begin{greek}
\noindent δημοσίᾳ μὲν τοίνυν \\
\tabto{2em} οἰκοδομήματα \\
\tabto{4em} καὶ κάλλη τοιαῦτα \\
\tabto{4em} καὶ τοσαῦτα \\
\tabto{2em} κατεσκεύασαν \\
\tabto{4em} ἡμῖν \\
\tabto{4em} ἱερῶν \\
\tabto{4em} καὶ τῶν \\
\tabto{6em} ἐν τούτοις \\
\tabto{4em} ἀναθημάτων, \\
\tabto{6em} ὥστε \\
\tabto{8em} μηδενὶ \\
\tabto{10em} τῶν ἐπιγιγνομένων \\
\tabto{8em} ὑπερβολὴν λελεῖφθαι· \\
ἰδίᾳ δ' \\
\tabto{2em} οὕτω σώφρονες ἦσαν \\
\tabto{2em} καὶ σφόδρ' \\
\tabto{4em} ἐν τῷ τῆς πολιτείας ἤθει \\
\tabto{2em} μένοντες, \\
\tabto{4em} ὥστε \\
\tabto{4em} τὴν Ἀριστείδου \\
\tabto{4em} καὶ τὴν Μιλτιάδου \\
\tabto{4em} καὶ τῶν τότε λαμπρῶν \\
\tabto{4em} οἰκίαν \\
\tabto{6em} εἴ τις ἄρ' οἶδεν ὑμῶν \\
\tabto{8em} ὁποία ποτ' ἐστίν, \\
\tabto{6em} ὁρᾷ \\
\tabto{8em} τῆς τοῦ γείτονος \\
\tabto{6em} οὐδὲν σεμνοτέραν οὖσαν\dots\\

\end{greek}
}

\begin{description}[noitemsep]
\item[δημοσίᾳ μὲν\dots\ ἰδίᾳ δ'\dots] koordinacija rečeničnih članova pomoću čestica μέν\dots\ δέ\dots
\item[οἰκοδομήματα\dots\ ἱερῶν] hiperbat
\item[κάλλη] akuzativ obzira, §~389
\item[τοιαῦτα\dots\ καὶ τοσαῦτα\dots\ ὥστε\dots] pokazne zamjenice otvaraju mjesto zavisnoj posljedičnoj rečenici
\item[κατεσκεύασαν] §~267, §~269
\item[ὥστε μηδενὶ\dots\ ὑπερβολὴν λελεῖφθαι] §~272; §~278.2; λείπω τινί τι dativ indirektnog objekta §~411; ὥστε otvara mjesto zavisno posljedičnoj rečenici s infinitivom §~473
\item[τῶν ἐπιγιγνομένων] §~232; supstantivirani particip §~373
\item[οὕτω σώφρονες ἦσαν\dots] \textbf{καὶ\dots\ μένοντες\dots\ ὥστε\dots\ ὁρᾷ} §~315; pridjev i particip kao imenski dio predikata, Smyth 909; prilog οὕτω, u korelaciji s veznikom ὥστε, otvara mjesto zavisno posljedičnoj rečenici, čiji predikat ovdje \textit{nije} u infinitivu zbog pogodbene protaze εἴ\dots\ οἶδεν; za misao usp.\ Hor.~Od. 2, 15, 13: \textit{Privatus illis census erat brevis, Commune magnum.}
\item[ἐν τῷ τῆς πολιτείας ἤθει] Demosten se ovdje vjerojatno koristi riječju \textgreek[variant=ancient]{πολιτεία} u značenju LSJ s.~v.\ III.2, „republika”; τὸ τῆς πολιτείας ἦθος „duh naše države”, „osnovna načela naše države”
\item[μένοντες] §~231
\item[τῶν τότε λαμπρῶν] sc. ἀνδρῶν; LSJ λαμπρός A.II
\item[εἴ\dots\ οἶδεν\dots\ ὁρᾷ\dots] pogodbena zavisna rečenica, realna §~475; apodoza je ujedno i zavisno posljedična rečenica, v.~gore
\item[εἴ\dots\ ἄρ' οἶδεν\dots\ ὁποία\dots] §~317.4; ὁποῖος iza \textit{verbum sentiendi} otvara mjesto zavisno upitnoj rečenici, §~469; §~516.1; ἄρα daje živost pitanjima, Smyth 2793
\item[ὁποία ποτ' ἐστίν] §~315; zamjenica kao imenski dio predikata, Smyth 909; pitanje pojačano enklitikom, LSJ ποτε III.3, „kakva je uopće”
\item[ὁρᾷ] §~243
\item[ὁρᾷ\dots\ οὐδὲν σεμνοτέραν οὖσαν] §~315; pridjev kao imenski dio predikata, Smyth 909; predikatni particip uz \textit{verbum sentiendi} proteže se na objekt, §~502.a; LSJ σεμνός II.2 ili III.
\item[τῆς τοῦ γείτονος] sc.\ οἰκίας; \textit{genetivus comparationis} §~404.1

\end{description}


%kraj
