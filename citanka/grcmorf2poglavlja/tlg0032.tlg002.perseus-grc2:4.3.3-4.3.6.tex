% unesi ispravke NZ, 30. 3. 2020.
%\section*{O autoru}

%TKTK


\section*{O tekstu}

Atenski pisac Ksenofont – plaćeni vojnik, povjesničar i filozof – bio je u mladosti i Sokratov učenik. I na Ksenofonta je, kao i na mnoge druge (Platona među njima), Sokrat ostavio dubok dojam. Poslije Sokratove smrti (399.\ pr.~Kr.), u nizu djela Ksenofont s raznih aspekata razmatra Sokratovo suđenje. To su takozvani sokratski spisi: \textit{Obrana Sokratova} \textgreek[variant=ancient]{(Ἀπολογία Σωκράτους)}, \textit{Uspomene na Sokrata} \textgreek[variant=ancient]{(Ἀπομνημονεύματα Σωκράτους)}, \textit{Gozba} \textgreek[variant=ancient]{(Συμπόσιον)} i \textit{Rasprava o gospodarstvu} \textgreek[variant=ancient]{(Οἰκονομικός)}. Ksenofont nije osobno svjedočio sudskom procesu Sokratu; u to je vrijeme bio u Perziji, kao grčki plaćenik u vojnom pohodu Kira Mlađeg.

U četvrtoj knjizi \textit{Uspomena na Sokrata}, odakle potječe ovaj odlomak, Ksenofont Sokratove poglede na obrazovanje predstavlja kroz dijalog Sokrata i njegova učenika, lijepog Eutidema. Razumnost ili umjerenost, \textgreek[variant=ancient]{σωφροσύνη,} prva je stvar koju Sokrat zahtijeva, a potrebna je i u odnosu prema bogovima. Opis načina na koji su bogovi ljudima uredili svijet smatra se jednim od najranijih primjera teleološkog argumenta.

\newpage

\section*{Pročitajte naglas grčki tekst.}

Xen. Memorabilia 4.3.3

%Naslov prema izdanju

\medskip


{\large

\begin{greek}

\noindent εἰπέ μοι, ἔφη, ὦ Εὐθύδημε, ἤδη ποτέ σοι ἐπῆλθεν ἐνθυμηθῆναι ὡς ἐπιμελῶς οἱ θεοὶ ὧν οἱ ἄνθρωποι δέονται κατεσκευάκασι; καὶ ὅς, μὰ τὸν Δίʼ, ἔφη, οὐκ ἔμοιγε. 

ἀλλʼ οἶσθά γʼ, ἔφη, ὅτι πρῶτον μὲν φωτὸς δεόμεθα, ὃ ἡμῖν οἱ θεοὶ παρέχουσι; 

νὴ Δίʼ, ἔφη, ὅ γʼ εἰ μὴ εἴχομεν, ὅμοιοι τοῖς τυφλοῖς ἂν ἦμεν ἕνεκά γε τῶν ἡμετέρων ὀφθαλμῶν. 

ἀλλὰ μὴν καὶ ἀναπαύσεώς γε δεομένοις ἡμῖν νύκτα παρέχουσι κάλλιστον ἀναπαυτήριον.

πάνυ γʼ, ἔφη, καὶ τοῦτο χάριτος ἄξιον. 

οὐκοῦν καὶ ἐπειδὴ ὁ μὲν ἥλιος φωτεινὸς ὢν τάς τε ὥρας τῆς ἡμέρας ἡμῖν καὶ τἆλλα πάντα σαφηνίζει, ἡ δὲ νὺξ διὰ τὸ σκοτεινὴ εἶναι ἀσαφεστέρα ἐστίν, ἄστρα ἐν τῇ νυκτὶ ἀνέφηναν, ἃ ἡμῖν τῆς νυκτὸς τὰς ὥρας ἐμφανίζει, καὶ διὰ τοῦτο πολλὰ ὧν δεόμεθα πράττομεν; 

ἔστι ταῦτα, ἔφη. 

ἀλλὰ μὴν ἥ γε σελήνη οὐ μόνον τῆς νυκτός, ἀλλὰ καὶ τοῦ μηνὸς τὰ μέρη φανερὰ ἡμῖν ποιεῖ.

πάνυ μὲν οὖν, ἔφη. 

τὸ δʼ, ἐπεὶ τροφῆς δεόμεθα, ταύτην ἡμῖν ἐκ τῆς γῆς ἀναδιδόναι καὶ ὥρας ἁρμοττούσας πρὸς τοῦτο παρέχειν, αἳ ἡμῖν οὐ μόνον ὧν δεόμεθα πολλὰ καὶ παντοῖα παρασκευάζουσιν, ἀλλὰ καὶ οἷς εὐφραινόμεθα;

πάνυ, ἔφη, καὶ ταῦτα φιλάνθρωπα.

\end{greek}

}


\section*{Analiza i komentar}

%1


{\large
\begin{greek}
\noindent εἰπέ μοι, \\
\tabto{2em} ἔφη, \\
ὦ Εὐθύδημε, \\
ἤδη ποτέ \\
\tabto{2em} σοι ἐπῆλθεν \\
\tabto{4em} ἐνθυμηθῆναι \\
\tabto{6em} ὡς ἐπιμελῶς \\
\tabto{6em} οἱ θεοὶ \\
\tabto{8em} ὧν οἱ ἄνθρωποι δέονται \\
\tabto{6em} κατεσκευάκασι;\\

\end{greek}
}

\begin{description}[noitemsep]
\item[εἰπέ] §~254, §~333.C.13, §~327.7, §~460.3
\item[ἔφη] §~312.8, §~39.3, §~327.7, §~460.1, §~452
\item[ἤδη ποτέ] »kad već«
\item[ἐπῆλθεν] rekcija τίνι; §~254, §~333.C.13, §~327.2, §~460.1, §~454                                         
\item[ἐνθυμηθῆναι] §~296, §~243, §~490
\item[ὡς] upitni prilog (uočite da je rečenica upitna)
\item[κατεσκευάκασι] §~272, §~275.b, §~281, §~301.B (s. 118), §~460.1, §~456
\item[δέονται] rekcija τινος; §~243
\item[ὧν οἱ ἄνθρωποι δέονται] odnosna zamjenica ὧν uvodi odnosnu rečenicu i objekt je predikata δέονται; antecedent odnosne zamjenice je neizrečeno ταῦτα, ovisno o κατεσκευάκασι
\end{description}

%2

{\large
\begin{greek}
\noindent καὶ ὅς, \\
\tabto{2em} μὰ τὸν Δί', \\
ἔφη, \\
\tabto{2em} οὐκ ἔμοιγε.\\

\end{greek}
}

\begin{description}[noitemsep]
\item[ἔφη] §~312.8, §~39.3, §~327.7, §~460.1, §~452
\item[καὶ ὅς\dots\ ἔφη] §~214.5; kombinacija καὶ ὅς na početku označava promjenu subjekta/govornika, »a on\dots«
\item[μὰ τὸν Δί'] §~519.9
\item[ἔμοιγε] < ἔμοι γε, §~519

\end{description}

%3


{\large
\begin{greek}
\noindent ἀλλʼ οἶσθά γ',\\
\tabto{2em} ἔφη,\\
ὅτι πρῶτον μὲν \\
φωτὸς δεόμεθα, \\
\tabto{2em} ὃ \\
\tabto{2em} ἡμῖν \\
\tabto{2em} οἱ θεοὶ παρέχουσι;\\

\end{greek}
}

\begin{description}[noitemsep]
\item[οἶσθά] §~317.4
\item[γ'] < γε, §~519
\item[ἔφη] §~312.8, §~39.3, §~327.7, §~460.1, §~452
\item[δεόμεθα] rekcija τινος, »trebati što«, §~243
\item[ὅτι\dots\ δεόμεθα] izrični veznik otvara mjesto umetnutoj izričnoj rečenici: »da\dots«; rečenica je dopuna glagolu οἶσθά
\item[παρέχουσι] §~231
\item[ὃ\dots\ παρέχουσι] antecedent je odnosne zamjenice genitiv φωτὸς; zamjenica uvodi odnosnu rečenicu »što\dots«
\end{description}


%4


{\large
\begin{greek}
\noindent νὴ Δί',\\
ἔφη, \\
ὅ γ' \\
\tabto{2em} εἰ μὴ εἴχομεν, \\
\tabto{2em} ὅμοιοι \\
\tabto{4em} τοῖς τυφλοῖς \\
\tabto{2em} ἂν ἦμεν \\
\tabto{4em} ἕνεκά γε \\
\tabto{6em} τῶν ἡμετέρων ὀφθαλμῶν.\\

\end{greek}
}

\begin{description}[noitemsep]
\item[νὴ Δί'] §~519.9
\item[ἔφη] §~312.8, §~39.3, §~327.7, §~460.1, §~452
\item[ὅ γ'] ponovljeni relativ naglašava prethodni: \textit{to\dots}
\item[γ'] < γε, §~519
\item[εἴχομεν] §~231, §~234.b, §~236, §~327.13, §~460.1, §~452
\item[ἦμεν] §~315
\item[ἂν ἦμεν] imperfekt s česticom ἂν tvori ireal, način koji izražava nezbiljnost radnje; na hrvatski se prevodi kondicionalom I.
\item[ἦμεν ὅμοιοι] imenski predikat, Smyth 909
\item[εἰ μὴ εἴχομεν\dots\ ἂν ἦμεν ὅμοιοι] pogodbena irealna rečenica za sadašnjost, u hrvatskom odgovara kombinaciji »da« + prezent – kondicional I., §~478.1
\item[μὴ] negacija u pogodbenoj zavisnoj rečenici uvijek je μὴ, §~474
\item[ἕνεκά γε] »bar što se tiče\dots«; Senc s.~v.\ ἕνεκα, LSJ ἕνεκα A.2
\end{description}


%5


{\large
\begin{greek}
\noindent ᾿Αλλὰ μὴν \\
\tabto{2em} καὶ ἀναπαύσεώς γε δεομένοις ἡμῖν \\
νύκτα παρέχουσι \\
κάλλιστον ἀναπαυτήριον.\\

\end{greek}
}

\begin{description}[noitemsep]
\item[᾿Αλλὰ μὴν] »nego zbilja«
\item[δεομένοις] rekcija τινος; §~243
\item[παρέχουσι] rekcija τινι τινα; §~231
\item[ἀναπαυτήριον] apozicija uz νύκτα
\end{description}


%6


{\large
\begin{greek}
\noindent πάνυ γ', \\
\tabto{2em} ἔφη, \\
καὶ τοῦτο \\
\tabto{2em} χάριτος ἄξιον.\\

\end{greek}
}

\begin{description}[noitemsep]
\item[πάνυ γ'] < Πάνυ γε, čestica naglašava prilog: »svakako,«
\item[ἔφη] §~312.8
\item[τοῦτο\dots\ ἄξιον] u imenskom predikatu izostavljena kopula (ἐστιν)
\end{description}


%7


{\large
\begin{greek}
\noindent οὐκοῦν καὶ \\
ἐπειδὴ \\
ὁ μὲν ἥλιος \\
\tabto{2em} φωτεινὸς ὢν \\
τάς τε ὥρας τῆς ἡμέρας \\
\tabto{2em} ἡμῖν \\
καὶ τἆλλα πάντα \\
σαφηνίζει,\\
ἡ δὲ νὺξ \\
\tabto{2em} διὰ τὸ σκοτεινὴ εἶναι \\
ἀσαφεστέρα ἐστίν, \\
ἄστρα \\
\tabto{2em} ἐν τῇ νυκτὶ \\
ἀνέφηναν, \\
ἃ ἡμῖν \\
\tabto{6em} τῆς νυκτὸς \\
\tabto{4em} τὰς ὥρας \\
\tabto{2em} ἐμφανίζει, \\
καὶ διὰ τοῦτο \\
πολλὰ \\
\tabto{2em} ὧν δεόμεθα \\
πράττομεν;\\

\end{greek}
}

\begin{description}[noitemsep]
\item[οὐκοῦν\dots\ ἀνέφηναν] čestica uvodi pitanje; takva su pitanja karakteristična za diskurs filozofskog razgovora i kod Ksenofonta i kod Platona
\item[ὁ μὲν ἥλιος\dots\ ἡ δὲ νὺξ\dots] koordinacija parom čestica μὲν\dots\ δὲ\dots, izriče suprotnost 
\item[ὢν] §~315
\item[σαφηνίζει] §~321
\item[ἐπειδὴ\dots\ σαφηνίζει] uzročna rečenica, »budući da...«
\item[εἶναι] §~315, §~490
\item[διὰ τὸ σκοτεινὴ εἶναι] prijedložni izraz sa supstantiviranim infinitivom, §~497
\item[ἐστίν] §~315
\item[ἀσαφεστέρα ἐστίν] imenski predikat (Smyth §~909)
\item[ἀνέφηναν] §~267, §~234.a, §~238, 301.B (s. 118), §~460.1, §~454; glagol upotrijebljen kauzativno, »učiniti da što zasja«; objekt je ἄστρα, subjekt je neizrečen (οἱ θεοὶ); upitna čestica οὐκοῦν na početku ove rečenice otvara mjesto upravo ovom predikatu οὐκοῦν\dots\ ἀνέφηναν »Nisu li (bogovi) učinili da zvijezde sjaje noću\dots«
\item[ἐμφανίζει] rekcija τινι τινα; §~231
\item[ἄστρα\dots\ ἃ\dots\ ἐμφανίζει] antecedent je odnosne zamjenice riječ ἄστρα; zamjenica uvodi zavisnu odnosnu rečenicu
\item[δεόμεθα] rekcija τινος »trebati što«, §~243
\item[πολλὰ ὧν δεόμεθα] antecedent je πολλὰ; odnosna zamjenica ὧν uvodi odnosnu rečenicu
\item[πράττομεν] §~231
\end{description}


%8


{\large
\begin{greek}
\noindent ἔστι ταῦτα,\\
ἔφη.\\

\end{greek}
}

\begin{description}[noitemsep]
\item[ἔστι] §~315
\item[ἔστι ταῦτα] jedan od načina da se na grčkom kaže »da«, Smyth 2680
\item[ἔφη] §~312.8
\end{description}

%9


{\large
\begin{greek}
\noindent ἀλλὰ μὴν \\
ἥ γε σελήνη \\
\tabto{2em} οὐ μόνον τῆς νυκτός, \\
\tabto{2em} ἀλλὰ καὶ τοῦ μηνὸς \\
\tabto{4em} τὰ μέρη \\
\tabto{4em} φανερὰ \\
\tabto{6em} ἡμῖν \\
\tabto{4em} ποιεῖ.\\

\end{greek}
}

\begin{description}[noitemsep]
\item[ἀλλὰ μὴν ] »nego zbilja«
\item[οὐ μόνον\dots\ ἀλλὰ καὶ] §~513.1
\item[ποιεῖ] §~243
\end{description}


%10


{\large
\begin{greek}
\noindent πάνυ μὲν οὖν, \\
ἔφη.\\

\end{greek}
}

\begin{description}[noitemsep]
\item[πάνυ μὲν οὖν] §~521.3
\item[ἔφη] §~312.8, §~39.3, §~327.7, §~460.1, §~452
\end{description}

%11

{\large
\begin{greek}
\noindent τὸ δ',\\
\tabto{2em} ἐπεὶ \\
\tabto{4em} τροφῆς \\
\tabto{4em} δεόμεθα, \\
\tabto{2em} ταύτην \\
\tabto{2em} ἡμῖν \\
\tabto{4em} ἐκ τῆς γῆς \\
\tabto{2em} ἀναδιδόναι

\tabto{2em} καὶ ὥρας ἁρμοττούσας \\
\tabto{4em} πρὸς τοῦτο \\
\tabto{2em} παρέχειν,

\tabto{4em} αἳ ἡμῖν\\
\tabto{6em} οὐ μόνον \\
\tabto{8em} ὧν δεόμεθα \\
\tabto{4em} πολλὰ καὶ παντοῖα \\
\tabto{4em} παρασκευάζουσιν, \\
\tabto{6em} ἀλλὰ καὶ \\
\tabto{8em} οἷς εὐφραινόμεθα;\\

\end{greek}
}

\begin{description}[noitemsep]
\item[τὸ δ'] otvara mjesto sljedećem pitanju; slobodnije »a što kažeš na ovo\dots?«
\item[δεόμεθα] rekcija τινος »trebati što«, §~243
\item[ἐπεὶ\dots\ δεόμεθα] uzročna rečenica, »budući da\dots«
\item[ἀναδιδόναι] rekcija τινι τινα; §~305, §~311, §~490 
\item[ἀναδιδόναι] infinitiv, dio akuzativa s infinitivom; kao subjekt treba zamisliti neizrečeno \textgreek[variant=ancient]{τοὺς θεούς (ταύτην} je objekt \textgreek[variant=ancient]{ἀναδιδόναι)}
\item[πρὸς τοῦτο] τοῦτο se odnosi na \textgreek[variant=ancient]{ταύτην (τροφήν) ἡμῖν ἐκ τῆς γῆς ἀναδιδόναι}
\item[ἁρμοττούσας] §~250
\item[παρέχειν] §~231; dio akuzativa s infinitivom čiji su subjekt, kao i gore, neizrečeno \textgreek[variant=ancient]{τοὺς θεούς}
\item[παρασκευάζουσιν] §~231
\item[αἳ\dots\ παρασκευάζουσιν] antecedent je odnosne zamjenice αἳ riječ ὥρας; αἳ uvodi odnosnu rečenicu
\item[οὐ μόνον\dots\ ἀλλὰ καὶ] §~513.1
\item[ὧν\dots] \textbf{πολλὰ καὶ παντοῖα\dots\ οἷς\dots} \textgreek[variant=ancient]{πολλὰ καὶ παντοῖα} antecedent je obiju odnosnih zamjenica (i odnosnih rečenica koje one uvode)
\item[δεόμεθα] rekcija τινος; §~243
\item[εὐφραινόμεθα] rekcija τινι; §~231

\end{description}


%12

{\large
\begin{greek}
\noindent πάνυ, \\
ἔφη, \\
καὶ ταῦτα φιλάνθρωπα.\\

\end{greek}
}

\begin{description}[noitemsep]
\item[πάνυ] »dakako«
\item[ἔφη] §~312.8, §~39.3, §~327.7, §~460.1, §~452
\item[ταῦτα φιλάνθρωπα] u imenskom predikatu izostavljena kopula (ἐστιν)
\end{description}

%kraj
