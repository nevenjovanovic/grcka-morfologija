%\section*{O autoru}

%TKTK


\section*{O tekstu}

Atenski pisac Ksenofont – plaćeni vojnik, povjesničar i filozof – bio je u mladosti i Sokratov učenik. I na Ksenofonta je, kao i na mnoge druge (Platona među njima), Sokrat ostavio dubok dojam. Poslije Sokratove smrti (399.\ pr.~Kr.), u nizu djela Ksenofont s raznih aspekata razmatra Sokratovo suđenje. To su takozvani sokratski spisi: \textit{Obrana Sokratova} \textgreek[variant=ancient]{(Ἀπολογία Σωκράτους)}, \textit{Uspomene na Sokrata} \textgreek[variant=ancient]{(Ἀπομνημονεύματα Σωκράτους)}, \textit{Gozba} \textgreek[variant=ancient]{(Συμπόσιον)} i \textit{Rasprava o gospodarstvu} \textgreek[variant=ancient]{(Οἰκονομικός)}. Ksenofont nije osobno svjedočio sudskom procesu Sokratu; u to je vrijeme bio u Perziji, kao grčki plaćenik u vojnom pohodu Kira Mlađeg.

U četvrtoj knjizi \textit{Uspomena na Sokrata}, odakle potječe ovaj odlomak, Ksenofont Sokratove poglede na obrazovanje predstavlja kroz dijalog Sokrata i njegova učenika, lijepog Eutidema. Razumnost ili umjerenost, \textgreek[variant=ancient]{σωφροσύνη,} prva je stvar koju Sokrat zahtijeva, a potrebna je i u odnosu prema bogovima. Opis načina na koji su bogovi ljudima uredili svijet smatra se jednim od najranijih primjera teleološkog argumenta.

\newpage

\section*{Pročitajte naglas grčki tekst.}

Xen. Memorabilia 4.3.3

%Naslov prema izdanju

\medskip


{\large

\begin{greek}

\noindent εἰπέ μοι, ἔφη, ὦ Εὐθύδημε, ἤδη ποτέ σοι ἐπῆλθεν ἐνθυμηθῆναι ὡς ἐπιμελῶς οἱ θεοὶ ὧν οἱ ἄνθρωποι δέονται κατεσκευάκασι; καὶ ὅς, μὰ τὸν Δίʼ, ἔφη, οὐκ ἔμοιγε. 

ἀλλʼ οἶσθά γʼ, ἔφη, ὅτι πρῶτον μὲν φωτὸς δεόμεθα, ὃ ἡμῖν οἱ θεοὶ παρέχουσι; 

νὴ Δίʼ, ἔφη, ὅ γʼ εἰ μὴ εἴχομεν, ὅμοιοι τοῖς τυφλοῖς ἂν ἦμεν ἕνεκά γε τῶν ἡμετέρων ὀφθαλμῶν. 

ἀλλὰ μὴν καὶ ἀναπαύσεώς γε δεομένοις ἡμῖν νύκτα παρέχουσι κάλλιστον ἀναπαυτήριον.

πάνυ γʼ, ἔφη, καὶ τοῦτο χάριτος ἄξιον. 

οὐκοῦν καὶ ἐπειδὴ ὁ μὲν ἥλιος φωτεινὸς ὢν τάς τε ὥρας τῆς ἡμέρας ἡμῖν καὶ τἆλλα πάντα σαφηνίζει, ἡ δὲ νὺξ διὰ τὸ σκοτεινὴ εἶναι ἀσαφεστέρα ἐστίν, ἄστρα ἐν τῇ νυκτὶ ἀνέφηναν, ἃ ἡμῖν τῆς νυκτὸς τὰς ὥρας ἐμφανίζει, καὶ διὰ τοῦτο πολλὰ ὧν δεόμεθα πράττομεν; 

ἔστι ταῦτα, ἔφη. 

ἀλλὰ μὴν ἥ γε σελήνη οὐ μόνον τῆς νυκτός, ἀλλὰ καὶ τοῦ μηνὸς τὰ μέρη φανερὰ ἡμῖν ποιεῖ.

πάνυ μὲν οὖν, ἔφη. 

τὸ δʼ, ἐπεὶ τροφῆς δεόμεθα, ταύτην ἡμῖν ἐκ τῆς γῆς ἀναδιδόναι καὶ ὥρας ἁρμοττούσας πρὸς τοῦτο παρέχειν, αἳ ἡμῖν οὐ μόνον ὧν δεόμεθα πολλὰ καὶ παντοῖα παρασκευάζουσιν, ἀλλὰ καὶ οἷς εὐφραινόμεθα;

πάνυ, ἔφη, καὶ ταῦτα φιλάνθρωπα.

\end{greek}

}


\section*{Analiza i komentar}

%1


{\large
\begin{greek}
\noindent Εἰπέ μοι, \\
\tabto{2em} ἔφη, \\
ὦ Εὐθύδημε, \\
\tabto{2em} ἤδη ποτέ \\
\tabto{2em} σοι \\
\tabto{2em} ἐπῆλθεν \\
\tabto{4em} ἐνθυμηθῆναι \\
\tabto{6em} ὡς ἐπιμελῶς \\
\tabto{6em} οἱ θεοὶ \\
\tabto{8em} ὧν οἱ ἄνθρωποι δέονται \\
\tabto{6em} κατεσκευάκασι;\\

\end{greek}
}

\begin{description}[noitemsep]
\item[Εἰπέ] § 254, § 333.C.13, § 327.7, § 460.3
\item[ἔφη] § 312.8, § 39.3, § 327.7, § 460.1, § 452
\item[ἤδη ποτέ] vremenski: “već kada”
\item[ἐπῆλθεν] rekcija: τίνι; § 254, § 333.C.13, § 327.2, § 460.1, § 454                                         
\item[ἐνθυμηθῆναι] § 296, § 243, § 490
\item[ὡς ἐπιμελῶς] ὡς kao adverb u pokaznom značenju, “tako brižno” (Smyth § 2988)
\item[κατεσκευάκασι] § 272, § 275.b, § 281, § 301.B (s. 118), § 460.1, § 456
\item[δέονται] rekcija: τινος; § 243
\item[ὧν οἱ ἄνθρωποι δέονται] odnosna zamjenica ὧν uvodi odnosnu rečenicu, “od onoga što\dots”, dopunjava misao οἱ θεοὶ κατεσκευάκασι
\end{description}

%2

{\large
\begin{greek}
\noindent καὶ ὅς, \\
Μὰ τὸν Δί', \\
ἔφη, \\
οὐκ ἔμοιγε.\\

\end{greek}
}

\begin{description}[noitemsep]
\item[ἔφη] § 312.8, § 39.3, § 327.7, § 460.1, § 452
\item[καὶ ὅς\dots\ ἔφη] § 214.5; kombinacija καὶ ὅς na početku označava promjenu subjekta/govornika, “a on\dots”
\item[Μὰ τὸν Δί'] § 519.9
\item[ἔμοιγε] < ἔμοι γε, § 519

\end{description}

%3


{\large
\begin{greek}
\noindent ᾿Αλλ' οἶσθά γ',\\
\tabto{2em} ἔφη,\\
ὅτι πρῶτον μὲν \\
φωτὸς \\
δεόμεθα, \\
\tabto{2em} ὃ \\
\tabto{2em} ἡμῖν \\
\tabto{2em} οἱ θεοὶ \\
\tabto{2em} παρέχουσι;\\

\end{greek}
}

\begin{description}[noitemsep]
\item[οἶσθά] § 317.4
\item[γ'] < γε, § 519
\item[ἔφη] § 312.8, § 39.3, § 327.7, § 460.1, § 452
\item[δεόμεθα] rekcija: τινος, trebati što, § 243
\item[ὅτι\dots\ δεόμεθα] izrični veznik “da” otvara mjesto umetnutoj izričnoj rečenici, “da\dots”, dopuna glagolu οἶσθά
\item[παρέχουσι] § 231
\item[ὃ\dots\ παρέχουσι] odnosna zamjenica ὃ veže se na genitiv φωτὸς i uvodi odnosnu rečenicu, “što\dots”
\end{description}


%4


{\large
\begin{greek}
\noindent Νὴ Δί',\\
ἔφη, \\
ὅ γ' \\
\tabto{2em} εἰ μὴ εἴχομεν, \\
\tabto{2em} ὅμοιοι \\
\tabto{4em} τοῖς τυφλοῖς \\
\tabto{2em} ἂν ἦμεν \\
\tabto{4em} ἕνεκά γε \\
\tabto{6em} τῶν ἡμετέρων ὀφθαλμῶν.\\

\end{greek}
}

\begin{description}[noitemsep]
\item[Νὴ Δί'] § 519.9
\item[ἔφη] § 312.8, § 39.3, § 327.7, § 460.1, § 452
\item[γ'] < γε, § 519
\item[εἴχομεν] § 231, § 234.b, § 236, § 327.13, § 460.1, § 452
\item[ἦμεν] § 315
\item[ἂν ἦμεν ὅμοιοι] imperfekt glagola s česticom ἂν tvori ireal, način koji izražava nezbiljnost radnje; u hrvatskom se prevodi kondicionalom I.
\item[ἦμεν ὅμοιοι] imenski predikat, Smyth 909
\item[εἰ μὴ εἴχομεν\dots\ ἂν ἦμεν ὅμοιοι] pogodbena irealna rečenica za sadašnjost, u hrvatskom odgovara kombinaciji ``da'' + prezent – kondicional I., § 478.1
\item[μὴ] negacija u pogodbenoj zavisnoj rečenici uvijek je μὴ, § 474
\end{description}


%5


{\large
\begin{greek}
\noindent ᾿Αλλὰ μὴν καὶ \\
\tabto{2em} ἀναπαύσεώς γε \\
δεομένοις ἡμῖν \\
νύκτα \\
παρέχουσι \\
κάλλιστον ἀναπαυτήριον.\\

\end{greek}
}

\begin{description}[noitemsep]
\item[᾿Αλλὰ μὴν καὶ] § 515, kombinacija čestica izražava potvrđivanje prethodne misli novom mišlju, “i stvarno, doista”
\item[δεομένοις] rekcija: τινος; § 243
\item[παρέχουσι] rekcija: τινι τινα; § 231
\end{description}


%6


{\large
\begin{greek}
\noindent Πάνυ γ', \\
ἔφη, \\
καὶ τοῦτο \\
χάριτος ἄξιον.\\

\end{greek}
}

\begin{description}[noitemsep]
\item[Πάνυ γ'] < Πάνυ γε, čestica naglašava prilog: “svakako, bez sumnje”
\item[ἔφη] § 312.8
\item[τοῦτο\dots\ ἄξιον] u imenskom predikatu izostavljena kopula (ἐστιν)
\end{description}


%7


{\large
\begin{greek}
\noindent Οὐκοῦν καὶ \\
ἐπειδὴ \\
ὁ μὲν ἥλιος \\
\tabto{2em} φωτεινὸς ὢν \\
τάς τε ὥρας τῆς ἡμέρας \\
\tabto{2em} ἡμῖν \\
καὶ \\
τἆλλα πάντα \\
σαφηνίζει,\\
ἡ δὲ νὺξ \\
\tabto{2em} διὰ τὸ σκοτεινὴ εἶναι \\
ἀσαφεστέρα ἐστίν, \\
ἄστρα \\
\tabto{2em} ἐν τῇ νυκτὶ \\
ἀνέφηναν, \\
\tabto{2em} ἃ ἡμῖν \\
\tabto{4em} τῆς νυκτὸς \\
\tabto{6em} τὰς ὥρας \\
\tabto{2em} ἐμφανίζει, \\
καὶ διὰ τοῦτο \\
πολλὰ \\
\tabto{2em} ὧν δεόμεθα \\
πράττομεν;\\

\end{greek}
}

\begin{description}[noitemsep]
\item[ὁ μὲν ἥλιος\dots\ ἡ δὲ νὺξ\dots] koordinacija u izricanju suprotnosti ostvarena česticama μὲν... δὲ
\item[Οὐκοῦν] § 520.b, § 516.2; pitanja koja uvodi čestica Οὐκοῦν karakteristična su za diskurs filozofskog razgovora, osim kod Ksenofonta česta su kod Platona
\item[ὢν] § 315
\item[σαφηνίζει] § 321
\item[ἐπειδὴ... σαφηνίζει] uzročna rečenica, “budući da...”
\item[εἶναι] § 315, § 490
\item[διὰ τὸ σκοτεινὴ εἶναι] prijedložni izraz sa supstantiviranim infinitivom, § 497
\item[ἐστίν] § 315
\item[ἀσαφεστέρα ἐστίν] imenski predikat (Smyth § 909)
\item[ἀνέφηναν] § 267, § 234.a, § 238, 301.B (s. 118), § 460.1, § 454 
\item[ἐμφανίζει] rekcija: τινι τινα; § 231
\item[ἃ... ἐμφανίζει] odnosna zamjenica ἃ veže se na riječ ἄστρα  i uvodi zavisnu odnosnu rečenicu, “koje...”
\item[δεόμεθα] rekcija: τινος, trebati što, § 243
\item[ὧν δεόμεθα] odnosna zamjenica ὧν uvodi odnosnu rečenicu koja dopunjuje πολλὰ πράττομεν
\item[πράττομεν] § 231
\end{description}


%8


{\large
\begin{greek}
\noindent ῎Εστι ταῦτα,\\
ἔφη.\\

\end{greek}
}

\begin{description}[noitemsep]
\item[῎Εστι] § 315
\item[ἔφη] § 312.8
\end{description}

%9


{\large
\begin{greek}
\noindent ᾿Αλλὰ μὴν \\
ἥ γε σελήνη \\
\tabto{2em} οὐ μόνον τῆς νυκτός, \\
\tabto{2em} ἀλλὰ καὶ τοῦ μηνὸς \\
\tabto{4em} τὰ μέρη \\
\tabto{4em} φανερὰ \\
\tabto{6em} ἡμῖν \\
\tabto{4em} ποιεῖ.\\

\end{greek}
}

\begin{description}[noitemsep]
\item[᾿Αλλὰ μὴν ] § 515.4, “ali zasigurno”, ova kombinacija čestica označava povratak na prethodnu misao i njezin nastavak
\item[οὐ μόνον\dots\ ἀλλὰ καὶ] § 513.1
\item[ποιεῖ] § 243
\end{description}


%10


{\large
\begin{greek}
\noindent Πάνυ μὲν οὖν, \\
ἔφη.\\

\end{greek}
}

\begin{description}[noitemsep]
\item[Πάνυ μὲν οὖν] § 521.3
\item[ἔφη] § 312.8, § 39.3, § 327.7, § 460.1, § 452
\end{description}

%11

{\large
\begin{greek}
\noindent Τὸ δ',\\
\tabto{2em} ἐπεὶ \\
\tabto{4em} τροφῆς \\
\tabto{4em} δεόμεθα, \\
\tabto{2em} ταύτην \\
\tabto{2em} ἡμῖν \\
\tabto{4em} ἐκ τῆς γῆς \\
\tabto{2em} ἀναδιδόναι

\tabto{2em} καὶ ὥρας ἁρμοττούσας \\
\tabto{4em} πρὸς τοῦτο \\
\tabto{2em} παρέχειν,

\tabto{4em} αἳ ἡμῖν\\
\tabto{6em} οὐ μόνον \\
\tabto{8em} ὧν δεόμεθα \\
\tabto{4em} πολλὰ καὶ παντοῖα \\
\tabto{4em} παρασκευάζουσιν, \\
\tabto{6em} ἀλλὰ καὶ \\
\tabto{8em} οἷς εὐφραινόμεθα;\\

\end{greek}
}

\begin{description}[noitemsep]
\item[δεόμεθα] rekcija: τινος, trebati što, § 243
\item[ἐπεὶ... δεόμεθα] uzročna rečenica, “budući da...”
\item[ἀναδιδόναι ] rekcija: τινι τινα; § 305, § 311, § 490 
\item[ταύτην ἀναδιδόναι] akuzativ s infinitivom
\item[ἁρμοττούσας] § 250
\item[παρέχειν] § 231
\item[ὥρας παρέχειν] akuzativ s infinitivom
\item[παρασκευάζουσιν] § 231
\item[αἳ...παρασκευάζουσιν] odnosna zamjenica αἳ veže se na riječ ὥρας i uvodi odnosnu rečenicu, “koja...”
\item[οὐ μόνον... ἀλλὰ καὶ] § 513.1
\item[δεόμεθα] rekcija: τινος; § 243
\item[εὐφραινόμεθα] rekcija: τινι; § 231
\item[οἷς εὐφραινόμεθα] odnosna zamjenica οἷς veže se na riječ ὥρας i uvodi odnosnu rečenicu, “kojima...”

\end{description}


%12

{\large
\begin{greek}
\noindent Πάνυ, \\
ἔφη, \\
καὶ ταῦτα \\
φιλάνθρωπα.\\

\end{greek}
}

\begin{description}[noitemsep]
\item[Πάνυ] „dakako“
\item[ἔφη] § 312.8, § 39.3, § 327.7, § 460.1, § 452
\item[ταῦτα φιλάνθρωπα] u imenskom predikatu izostavljena kopula (ἐστιν)
\end{description}

%kraj
