%\section*{O autoru}

%TKTK


\section*{O tekstu}

Govor \textit{Za Ktezifonta, o vijencu} i danas se smatra jednim od najvećih dostignuća grčke proze, uzor snažne argumentacije i živog stila. Demosten ga je održao pred Atenjanima 330.\ pr.~Kr. Ubrzo nakon toga, govor se pojavio u pisanom obliku.

Ovo je Demostenovo zrelo djelo – do njegova se sastavljanja autor govorništvom bavio već preko dvadeset godina – što se ogleda u vještini argumentacije, narativnog prikaza i ljepoti domoljubnoga sentimenta. I u ovom se govoru otkrivaju posvećenost Ateni i njezinu političkom opstanku te otpor jačanju Filipa II. Makedonskog. 

Nakon bitke kod Heroneje 336.\ pr.~Kr.\ pitanje odnosa s Filipom postalo je pitanjem unutrašnje politike; Demosten se političkim stavovima suprotstavljao Eshinu. Kada je govornik Ktezifont, Demostenov pristalica, predložio da se Demostenu za predanost općem dobru Atene dodijeli počasni vijenac, Eshin je podigao tužbu (sačuvanu u Eshinovu govoru \textit{Protiv Ktezifonta}) u kojoj je žestoko napao Demostena. Demosten je svojim odgovorom pobijedio i Eshin je otišao u progonstvo.

Odlomak koji čitamo pripada posljednjoj trećini govora i ulomak je Demostenove obrane stavova iz 339.\ i 338.\ pr.~Kr. Tada se Demosten zalagao za savez s Tebom, koji Eshin nije podržavao. Demosten svoje ideje i postupke predstavlja kao vjerni odraz duha Atene koji se odvajkada opirao napadima na njezin suverenitet.

U odlomku se Demosten spominje događaja prije bitke kod Plateje, 479.\ pr.~Kr., koji prenosi Herodot u svojoj povijesti (VIII, 140 i IX, 4, 5). Nakon poraza kod Salamine Kserkso je otplovio u Aziju, ostavivši u Grčkoj vojnog zapovjednika Mardonija da nakon zime dovrši osvajanja. Mardonije se povukao u Tesaliju, a period zimskog mirovanja iskoristio je da Atenjanima ponudi mir pod uvjetom da prihvate perzijsku vlast. Atenjani su tada ponudu odbili zbog ljubavi prema slobodi. Daljnja ratna zbivanja donijela su Perziji konačan poraz.

%\newpage

\section*{Pročitajte naglas grčki tekst.}

Dem.\ De corona 201

%Naslov prema izdanju

\medskip


{\large

\begin{greek}

\noindent τίσι δʼ ὀφθαλμοῖς πρὸς Διὸς ἑωρῶμεν ἂν τοὺς εἰς τὴν πόλιν ἀνθρώπους ἀφικνουμένους, εἰ τὰ μὲν πράγματʼ εἰς ὅπερ νυνὶ περιέστη, ἡγεμὼν δὲ καὶ κύριος ᾑρέθη Φίλιππος ἁπάντων, τὸν δʼ ὑπὲρ τοῦ μὴ γενέσθαι ταῦτʼ ἀγῶνα ἕτεροι χωρὶς ἡμῶν ἦσαν πεποιημένοι, καὶ ταῦτα μηδεπώποτε τῆς πόλεως ἐν τοῖς ἔμπροσθεν χρόνοις ἀσφάλειαν ἄδοξον μᾶλλον ἢ τὸν ὑπὲρ τῶν καλῶν κίνδυνον ᾑρημένης.

τίς γὰρ οὐκ οἶδεν Ἑλλήνων, τίς δὲ βαρβάρων, ὅτι καὶ παρὰ Θηβαίων καὶ παρὰ τῶν ἔτι τούτων πρότερον ἰσχυρῶν γενομένων Λακεδαιμονίων καὶ παρὰ τοῦ Περσῶν βασιλέως μετὰ πολλῆς χάριτος τοῦτʼ ἂν ἀσμένως ἐδόθη τῇ πόλει, ὅ τι βούλεται λαβούσῃ καὶ τὰ ἑαυτῆς ἐχούσῃ τὸ κελευόμενον ποιεῖν καὶ ἐᾶν ἕτερον τῶν Ἑλλήνων προεστάναι;

ἀλλʼ οὐκ ἦν ταῦθʼ, ὡς ἔοικε, τοῖς Ἀθηναίοις πάτρια οὐδʼ ἀνεκτὰ οὐδʼ ἔμφυτα, οὐδʼ ἐδυνήθη πώποτε τὴν πόλιν οὐδεὶς ἐκ παντὸς τοῦ χρόνου πεῖσαι τοῖς ἰσχύουσι μέν, μὴ δίκαια δὲ πράττουσι προσθεμένην ἀσφαλῶς δουλεύειν, ἀλλʼ ἀγωνιζομένη περὶ πρωτείων καὶ τιμῆς καὶ δόξης κινδυνεύουσα πάντα τὸν αἰῶνα διατετέλεκε.

\end{greek}

}


\section*{Analiza i komentar}

%1

{\large
\begin{greek}
\noindent τίσι δ' ὀφθαλμοῖς \\
\tabto{2em} πρὸς Διὸς \\
ἑωρῶμεν ἂν \\
τοὺς εἰς τὴν πόλιν ἀνθρώπους ἀφικνουμένους, \\
\tabto{2em} εἰ τὰ μὲν πράγματ' \\
\tabto{4em} εἰς ὅπερ νυνὶ \\
\tabto{2em} περιέστη, \\
\tabto{2em} ἡγεμὼν δὲ καὶ κύριος ᾑρέθη Φίλιππος \\
\tabto{4em} ἁπάντων, \\
\tabto{2em} τὸν δ' \\
\tabto{4em} ὑπὲρ τοῦ μὴ γενέσθαι ταῦτ' \\
\tabto{2em} ἀγῶνα \\
\tabto{2em} ἕτεροι \\
\tabto{4em} χωρὶς ἡμῶν \\
\tabto{2em} ἦσαν πεποιημένοι, \\
\tabto{2em} καὶ ταῦτα\\
\tabto{4em} μηδεπώποτε \\
\tabto{2em} τῆς πόλεως \\
\tabto{4em} ἐν τοῖς ἔμπροσθεν χρόνοις \\
\tabto{2em} ἀσφάλειαν ἄδοξον \\
\tabto{2em} μᾶλλον ἢ τὸν ὑπὲρ τῶν καλῶν κίνδυνον \\
\tabto{2em} ᾑρημένης.\\

\end{greek}
}

\begin{description}[noitemsep]
\item[ἑωρῶμεν] §~243, augment §~237.2
\item[τίσι\dots\ ἑωρῶμεν ἂν] zamjenica uvodi zavisnu upitnu rečenicu koja je ujedno pogodbena apodoza
\item[ἀφικνουμένους] §~243
\item[εἰ\dots\ ᾑρημένης] pogodbeni veznik uvodi zavisnu pogodbenu protazu koju tvore tri dijela: \textgreek[variant=ancient]{εἰ περιέστη\dots\ ἡγεμὼν δὲ καὶ κύριος ᾑρέθη\dots\ τὸν δ'\dots\ ἀγῶνα\dots\ ἦσαν πεποιημένοι\dots}
\item[τὰ μὲν πράγματ'\dots] \textbf{ἡγεμὼν δὲ\dots\ τὸν δ'\dots\ ἀγῶνα\dots} rečenični dijelovi koordiniraju se pomoću čestica
\item[εἰς ὅπερ νυνὶ] u sadašnjem stanju (doslovno: u onome što je sada)
\item[περιέστη] §~306, složenica ἵστημι, LSJ περιίστημι II. 3.
\item[ᾑρέθη] §~296
\item[τοῦ μὴ γενέσθαι ταῦτ'] §~254, supstantivirani A+I u prijedložnom izrazu §~497 
\item[ἦσαν πεποιημένοι] §~287, objekt je \textgreek[variant=ancient]{τὸν δ’\dots\ ἀγῶνα}; znači isto što i \textgreek[variant=ancient]{ἀγωνιζομαι ὑπέρ} boriti se za
\item[ᾑρημένης] §~272, ovisno o τῆς πόλεως, otvara mjesto (objektnim) dopunama u akuzativu

\end{description}

%2


{\large
\begin{greek}
\noindent τίς γὰρ οὐκ οἶδεν ῾Ελλήνων, \\
τίς δὲ βαρβάρων, \\
\tabto{2em} ὅτι καὶ παρὰ Θηβαίων \\
\tabto{2em} καὶ παρὰ τῶν \\
\tabto{4em} ἔτι τούτων πρότερον \\
\tabto{2em} ἰσχυρῶν γενομένων Λακεδαιμονίων\\
\tabto{2em} καὶ παρὰ τοῦ Περσῶν βασιλέως \\
\tabto{4em} μετὰ πολλῆς χάριτος \\
\tabto{2em} τοῦτ' \\
\tabto{2em} ἂν ἀσμένως ἐδόθη \\
\tabto{2em} τῇ πόλει, \\
\tabto{4em} ὅ τι βούλεται \\
\tabto{2em} λαβούσῃ \\
\tabto{2em} καὶ τὰ ἑαυτῆς ἐχούσῃ \\
\tabto{2em} τὸ κελευόμενον ποιεῖν \\
\tabto{2em} καὶ ἐᾶν \\
\tabto{4em} ἕτερον \\
\tabto{4em} τῶν ῾Ελλήνων \\
\tabto{4em} προεστάναι; \\

\end{greek}
}

\begin{description}[noitemsep]
\item[γὰρ] čestica γάρ ovdje uvodi objašnjenje koje se veže za prethodnu rečenicu: naime\dots
\item[τίς\dots\ οὐκ οἶδεν] upitna zamjenica τίς uvodi nezavisnu upitnu rečenicu; οἶδεν §~317.4 otvara mjesto dopuni u obliku izrične rečenice s ὅτι
\item[τίς\dots\ ῾Ελλήνων, τίς δὲ βαρβάρων\dots] koordinacija paralelnih rečeničnih članova pomoću čestice
\item[ὅτι\dots\ ἐδόθη] veznik otvara mjesto zavisnoj izričnoj rečenici koja je dopuna glagolu οἶδεν
\item[τούτων πρότερον] genetivus comparationis uz komparativ (priloga) § 404.1
\item[ἰσχυρῶν γενομένων] §~254, glagol nepotpuna značenja otvara mjesto nužnoj imenskoj dopuni
\item[τοῦτ'\dots\ ποιεῖν καὶ ἐᾶν\dots] zamjenica je antecedent dvaju infinitiva: bilo bi dano to da\dots
\item[ἐδόθη] §~311
\item[ὅ\dots\ βούλεται] odnosna zamjenica uvodi zavisnu odnosnu rečenicu sa službom objektne dopune participu λαβούσῃ
\item[λαβούσῃ] §~254
\item[τὰ ἑαυτῆς] supstantiviranje članom §~373
\item[ἐχούσῃ] §~231
\item[τὸ κελευόμενον] §~231, supstantivirani particip §~499, objekt ποιεῖν; raditi po zapovijedi, pokoravati se zapovijedi (doslovno ``raditi ono što je naređeno''); izraz označava za Grke (Atenjane napose) užasavajuć gubitak slobode i prihvaćanje tuđe vlasti
\item[ποιεῖν] §~243
\item[ἐᾶν] §~243, otvara mjesto dopuni u infinitivu
\item[προεστάναι] §~ 311.2, složenica ἵστημι, LSJ προίστημι, B.II, dopuna ἐᾶν

\end{description}

%3


{\large
\begin{greek}
\noindent ἀλλ' οὐκ ἦν ταῦθ', \\
\tabto{2em} ὡς ἔοικε, \\
τοῖς ᾿Αθηναίοις \\
\tabto{2em} πάτρια οὐδ' ἀνεκτὰ οὐδ' ἔμφυτα,\\
οὐδ' ἐδυνήθη πώποτε \\
\tabto{2em} τὴν πόλιν \\
οὐδεὶς \\
\tabto{4em} ἐκ παντὸς τοῦ χρόνου \\
\tabto{2em} πεῖσαι \\
\tabto{4em} τοῖς ἰσχύουσι μέν, \\
\tabto{4em} μὴ δίκαια δὲ πράττουσι \\
\tabto{2em} προσθεμένην ἀσφαλῶς \\
\tabto{2em} δουλεύειν, \\
ἀλλ' ἀγωνιζομένη \\
\tabto{2em} περὶ πρωτείων καὶ τιμῆς καὶ δόξης \\
κινδυνεύουσα \\
\tabto{2em} πάντα τὸν αἰῶνα \\
διατετέλεκε.\\

\end{greek}
}

\begin{description}[noitemsep]
\item[ἦν] §~315, kopula kao glagolski dio imenskoga predikata otvara mjesto imenskim dopunama \textgreek[variant=ancient]{πάτρια οὐδ' ἀνεκτὰ οὐδ' ἔμφυτα}
\item[ὡς ἔοικε] umetnuta načinska (poredbena) rečenica, LSJ ἔοικα II.2
\item[οὐδ' ἐδυνήθη] §~312, otvara mjesto obaveznoj dopuni u infinitivu 
\item[πεῖσαι] §~267, §~329, otvara mjesto objektu u akuzativu \textgreek[variant=ancient]{(τὴν πόλιν)} i dopuni u infinitivu \textgreek[variant=ancient]{(δουλεύειν)}
\item[τοῖς ἰσχύουσι μέν\dots] \textbf{\textgreek[variant=ancient]{μὴ δίκαια δὲ\dots}} čestice koordiniraju suprotstavljene, ali sintaktički paralelne rečenične dijelove
\item[τοῖς ἰσχύουσι] §~231, supstantivirani particip §~499
\item[πράττουσι] §~231
\item[προσθεμένην] §~306, složenica τίθημι, LSJ προστίθημι B.
\item[δουλεύειν] §~231
\item[ἀλλ'\dots\ διατετέλεκε] suprotni veznik ἀλλά uvodi nezavisno složenu suprotnu rečenicu
\item[ἀγωνιζομένη\dots\ κινδυνεύουσα\dots] §~231, participi ovisni o novom subjektu, ἡ πόλις
\item[διατετέλεκε] §~272, složenica τελέω

\end{description}



%kraj
