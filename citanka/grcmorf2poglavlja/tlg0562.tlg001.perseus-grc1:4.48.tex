%Unio ispravke NZ, <2021-12-27 pon>


\section*{O tekstu}

Moderni tumači u djelu rimskoga cara \textgreek[variant=ancient]{Tὰ εἰς ἑαυτὸν} (\textit{Razgovori sa samim sobom}, 170.–180) vide jedini antički sačuvan privatni spis namijenjen samoanalizi i samodisciplini. Po tome je djelo bez premca u antičkoj filozofskoj književnosti. Dvanaest njegovih knjiga donosi kratke osvrte na najrazličitije teme, bez posebno sustavne organizacije.

Stalno samopreispitivanje bilo je za stoičku etiku, osobito onu carskoga doba, osnova svakog moralnog usavršavanja (i Seneka i Epiktet preporučuju ispit savjesti na kraju dana). Djelo Marka Aurelija nadilazi tu konkretnu, dnevnu samoanalizu formulirajući opća, posvuda primjenjiva načela, nastojeći premostiti jaz između moralne refleksije i svakodnevnog djelovanja, između filozofije i prakse. Istovremeno, refleksija \textit{Razgovora} nije samo racionalna, nego i emocionalna i ekspresivna; autor uvijek iznova pokušava prvenstveno \textit{samog sebe} uvjeriti u vrijednost filozofskih načela, osvijestiti vlastite slabosti, vježbati razumijevanje za tuđe nedostatke.

U ovdje odabranom odlomku četvrte knjige Marko Aurelije dotiče se dominantne teme svojeg djela i, unekoliko, svojeg doba: krhkosti i beznačajnosti života, trajne pripreme na smrt. Smrti su podložni i oni koje inače smatramo njeznim gospodarima – liječnici, proroci, filozofi, ratnici, vladari, pa i čitave organizacije – i ljudskom biću ostaje samo da bude zahvalno na onome što mu je dano.

Odlomak je i primjer osobnosti stila, po kojem se Marko Aurelije razlikuje od svakog poznatog nam pisca carskoga doba. Jednostavne se i svakodnevne rečenice smjenjuju s onim kompleksnijima i retorički razrađenijima, a dojam \textgreek[variant=ancient]{ἀφέλεια} (jednostavnog, utilitarnog stila) postignut je pomnim radom na tekstu.

%\newpage

\section*{Pročitajte naglas grčki tekst.}

M.~Aur.\ Ad se ipsum 4.48

%Naslov prema izdanju

\medskip


{\large

\begin{greek}

\noindent  Ἐννοεῖν συνεχῶς πόσοι μὲν ἰατροὶ ἀποτεθνήκασι, πολλάκις τὰς ὀφρῦς ὑπὲρ τῶν ἀρρώστων συσπάσαντες· πόσοι δὲ μαθηματικοί, ἄλλων θανάτους ὥς τι μέγα προειπόντες· πόσοι δὲ φιλόσοφοι, περὶ θανάτου ἢ ἀθανασίας μυρία διατεινάμενοι· πόσοι δὲ ἀριστεῖς, πολλοὺς ἀποκτείναντες· πόσοι δὲ τύραννοι, ἐξουσίᾳ ψυχῶν μετὰ δεινοῦ φρυάγματος ὡς ἀθάνατοι κεχρημένοι· πόσαι δὲ πόλεις ὅλαι, ἵν' οὕτως εἴπω, τεθνήκασιν, Ἑλίκη καὶ Πομπήιοι καὶ Ἡρκλᾶνον καὶ ἄλλαι ἀναρίθμητοι. ἔπιθι δὲ καὶ ὅσους οἶδας, ἄλλον ἐπ' ἄλλῳ· ὁ μὲν τοῦτον κηδεύσας εἶτα ἐξετάθη, ὁ δὲ ἐκεῖνον, πάντα δὲ ἐν βραχεῖ. τὸ γὰρ ὅλον, κατιδεῖν ἀεὶ τὰ ἀνθρώπινα ὡς ἐφήμερα καὶ εὐτελῆ καὶ ἐχθὲς μὲν μυξάριον, αὔριον δὲ τάριχος ἢ τέφρα. τὸ ἀκαριαῖον οὖν τοῦτο τοῦ χρόνου κατὰ φύσιν διελθεῖν καὶ ἵλεων καταλῦσαι, ὡς ἂν εἰ ἐλαία πέπειρος γενομένη ἔπιπτεν, εὐφημοῦσα τὴν ἐνεγκοῦσαν καὶ χάριν εἰδυῖα τῷ φύσαντι δένδρῳ.


\end{greek}

}


\section*{Analiza i komentar}

%1

{\large
\begin{greek}
\noindent Ἐννοεῖν συνεχῶς \\
\tabto{2em} πόσοι μὲν ἰατροὶ ἀποτεθνήκασι, \\
\tabto{4em} πολλάκις τὰς ὀφρῦς \\
\tabto{6em} ὑπὲρ τῶν ἀρρώστων \\
\tabto{4em} συσπάσαντες· \\
\tabto{2em} πόσοι δὲ μαθηματικοί, \\
\tabto{4em} ἄλλων θανάτους \\
\tabto{6em} ὥς τι μέγα \\
\tabto{4em} προειπόντες· \\
\tabto{2em} πόσοι δὲ φιλόσοφοι, \\
\tabto{6em} περὶ θανάτου ἢ ἀθανασίας \\
\tabto{4em} μυρία διατεινάμενοι·

\tabto{2em} πόσοι δὲ ἀριστεῖς, \\
\tabto{4em} πολλοὺς ἀποκτείναντες· \\
\tabto{2em} πόσοι δὲ τύραννοι, \\
\tabto{4em} ἐξουσίᾳ ψυχῶν \\
\tabto{6em} μετὰ δεινοῦ φρυάγματος \\
\tabto{6em} ὡς ἀθάνατοι \\
\tabto{4em} κεχρημένοι· \\
\tabto{2em} πόσαι δὲ πόλεις ὅλαι, \\
\tabto{4em} ἵν' οὕτως εἴπω, \\
\tabto{2em} τεθνήκασιν, \\
\tabto{4em} Ἑλίκη καὶ Πομπήιοι καὶ Ἡρκλᾶνον καὶ ἄλλαι ἀναρίθμητοι.\\

\end{greek}
}

\begin{description}[noitemsep]
\item[Ἐννοεῖν]  §~231; §~243; infinitiv ima vrijednost zapovijedi, Smyth 2013
\item[πόσοι μὲν\dots] \textbf{πόσοι δὲ\dots\ πόσοι δὲ\dots\ πόσοι δὲ\dots\ πόσοι δὲ\dots\ πόσοι δὲ\dots}\ koordinacija rečeničnih članova pomoću čestica μέν\dots\ δέ\dots
\item[ἀποτεθνήκασι]  §~272; §~324.8
\item[συσπάσαντες] §~267
\item[μαθηματικοί] μαθηματικός u značenju LSJ II.2.b
\item[προειπόντες] §~254; složenica glagola λέγω §~327.7
\item[μυρία] μυρίος (pazi na naglasak!) u pluralu srednjeg roda upotrijebljeno kao prilog, LSJ s.~v.\ A.4
\item[διατεινάμενοι] §~267; §~270
\item[ἀποκτείναντες] §~267; §~270
\item[κεχρημένοι] χράομαί τινι, ovdje otvara mjesto objektu u dativu
\item[ἵν'\dots\ εἴπω] §~254; §~327.7; ἵνα otvara mjesto zavisno namjernoj rečenici, ovdje s konjunktivom (zbog oblika predikata u glavnoj rečenici), §~470
\item[τεθνήκασιν]  §~272; §~324.8
\item[Ἑλίκη] grad u Ahaji, na sjevernom Peloponezu; u zimi 373.\ pr.~Kr.\ najvjerojatnije ga je potopio cunami nakon potresa. Ruševine grada posjetili su (i o tome pisali) Strabon, Pauzanija, Diodor Sicilski, Elijan i Ovidije. Ruševine su ponovno otkrivene tek 2001.
\item[Πομπήιοι καὶ Ἡρκλᾶνον] Pompeji i Herkulanej, dva od tri antička grada zatrpanih vulkanskim pepelom u provali Vezuva 79.\ po~Kr.\ (tom je prigodom poginuo Plinije Stariji, a Plinije Mlađi zbivanja je opisao u pismima Tacitu)
\end{description}

%2

{\large
\begin{greek}
\noindent ἔπιθι δὲ καὶ \\
ὅσους οἶδας, \\
\tabto{2em} ἄλλον ἐπ' ἄλλῳ· \\
\tabto{4em} ὁ μὲν τοῦτον κηδεύσας \\
\tabto{6em} εἶτα ἐξετάθη, \\
\tabto{4em} ὁ δὲ ἐκεῖνον, \\
\tabto{4em} πάντα δὲ ἐν βραχεῖ.\\

\end{greek}
}

\begin{description}[noitemsep]
\item[ἔπιθι] složenica od εἶμι, §~314.1; LSJ / Logeion ἔπειμι B.III.2
\item[δὲ] čestica δέ označava nadovezivanje na prethodnu rečenicu
\item[οἶδας] §~317.4, alternativni (i rjeđi) oblik za atičko οἶσθα
\item[ἄλλον ἐπ' ἄλλῳ] usp.\ Diccionario Griego-Español (DGE, dostupan u okviru zbirke Logeion), ἄλλος III.2
\item[ὁ μὲν\dots\ ὁ δὲ\dots] koordinacija: „ovaj\dots\ a onaj\dots”
\item[κηδεύσας] §~267
\item[ἐξετάθη] §~296; §~238; složenica glagola τείνω s. 118; ἐκτείνω ovdje u značenju „položiti na odar”, DGE s.~v.\ II.1
\item[ἐκεῖνον] sc.\ κηδεύσας
\item[ἐν βραχεῖ] za frazu kao priložnu oznaku vremena v.\ βραχύς u DGE
\end{description}

%3

{\large
\begin{greek}
\noindent τὸ γὰρ ὅλον, \\
κατιδεῖν ἀεὶ \\
τὰ ἀνθρώπινα \\
\tabto{2em} ὡς ἐφήμερα \\
\tabto{2em} καὶ εὐτελῆ \\
\tabto{2em} καὶ ἐχθὲς μὲν μυξάριον, \\
\tabto{4em} αὔριον δὲ τάριχος ἢ τέφρα.\\

\end{greek}
}

\begin{description}[noitemsep]
\item[τὸ γὰρ ὅλον] supstantivirani srednji rod pridjeva ὅλος, LSJ s.~v.\ A.4 „općenito”, „ukratko”, jednako kao i τὸ μὲν ὅλον, τὸ δ´ ὅλον
\item[κατιδεῖν] §~254; složenica ὁράω §~327.3; infinitiv izriče zapovijed, kao gore
\item[ἐχθὲς μὲν\dots\ αὔριον δὲ\dots] koordinacija rečeničnih članova pomoću čestica μέν\dots\ δέ\dots
\end{description}

%4

{\large
\begin{greek}
\noindent τὸ ἀκαριαῖον οὖν \\
τοῦτο τοῦ χρόνου \\
\tabto{2em} κατὰ φύσιν \\
διελθεῖν \\
καὶ ἵλεων καταλῦσαι, \\
ὡς ἂν εἰ \\
\tabto{2em} ἐλαία πέπειρος γενομένη ἔπιπτεν, \\
\tabto{4em} εὐφημοῦσα τὴν ἐνεγκοῦσαν \\
\tabto{4em} καὶ χάριν εἰδυῖα \\
\tabto{6em} τῷ φύσαντι δένδρῳ.\\

\end{greek}
}

\begin{description}[noitemsep]
\item[τοῦτο τοῦ χρόνου] partitivni genitiv, §~395
\item[κατὰ φύσιν] „u skladu s naravi” važno je stoičko načelo; ono što nije u našoj moći (npr.\ zdravlje, siromaštvo) vrijednosno je indiferentno (ἀδιάφορα), ali indiferentne stvari koje su κατὰ φύσιν su προηγμένα, „poželjne”: zdravlje, čast, užitak itd.\ korisni su kao sredstvo za napredovanje prema moralnosti. Začetnik stoičke škole, Zenon iz Kitija (oko 334.\ – oko 262.\ pr.~Kr.), napisao je djelo \textgreek{Περὶ τοῦ κατὰ φύσιν βίου,} \textit{O životu u skladu s naravi}; suprotnost su τὰ παρὰ φύσιν
\item[διελθεῖν] §~254; složenica ἔρχομαι §~327.2
\item[καταλῦσαι] §~267; složenica λύω, s. 108; doslovno „ispregnuti konje (nakon puta)”, preneseno „otići, napustiti život”
\item[ὡς ἂν εἰ] „baš kao da\dots” kombinacija veznika i čestica otvara mjesto zavisnoj rečenici koja je kombinacija poredbe i pogodbe; predikat je u indikativu imperfekta (kao ovdje), aorista, ili u optativu; u gramatikama je opisano kao (češće) ὥσπερ ἂν εἰ (ὡσπερανεί); §~479.3e, Smyth 4.53.139.149 2478-2480
\item[ἔπιπτεν] §~231
\item[πέπειρος γενομένη] §~254; §~325.11; kopulativni glagol otvara mjesto predikatnoj dopuni, koja je ovdje pridjev
\item[εὐφημοῦσα] §~231
\item[τὴν ἐνεγκοῦσαν] §~254; §~327.5; ovdje je vjerojatno ispuštena riječ γῆν
„zemlju”; \textgreek{ἐλαία,} plod masline, slavi zemlju koja ga je iznijela i zahvalan je stablu na kojem je izrastao; stablo se spominje u nastavku
\item[εἰδυῖα] §~317.4; χάριν εἰδέναι τινί „osjećati zahvalnost prema nekome”, LSJ s.~v.\ χάρις II.2
\item[φύσαντι] §~267
\end{description}


%kraj
