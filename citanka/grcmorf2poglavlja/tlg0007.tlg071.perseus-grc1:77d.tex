% Unio ispravke NZ <2022-01-04 uto>


\section*{O tekstu}

``Kako se može osjetiti vlastiti napredak u vrlini'' posljednji je esej u prvoj od 14 knjiga zbirke \textgreek[variant=ancient]{Ἠϑικά} (\textit{Moralia, Spisi o moralu}). Plutarh osporava dvije važne postavke stoika: prvo, da samo mudrac može imati udjela u vrlini (pri čemu se i mudrost i vrlina stječu odjednom, a ne postupno); drugo, u nekoj mjeri korolar prvog: ako netko nije savršen, tj.\ mudar, nije važno koliko je zapravo njegovo nesavršenstvo, jesu li njegovi nedostaci veliki ili mali. Plutarhov zdrav razum buni se protiv ovakvih ekstremnih učenja, i Plutarh pokazuje da je napredak u vrlini \textgreek[variant=ancient]{(προκοπὴ ἐπ' ἀρετῇ)} moguć, kao i da postoji obilje znakova toga napretka.

Esej je posvećen Kvintu Soziju Senecionu, jednome od brojnih Plutarhovih prijatelja Rimljana. Senecion je dvaput bio konzul na početku Trajanove vladavine (od 98.\ po Kr.). Na Senecionov je zahtjev Plutarh sastavio dijalog \textgreek[variant=ancient]{Συμποσιακὰ τῶν ἑπτὰ σοφῶν,} \textit{Gozbu sedam mudraca} (također dio \textgreek[variant=ancient]{Ἠϑικά}), a Senecionu su posvećeni i životopisi Tezeja i Romula, Demostena i Cicerona, Diona i Bruta. Istom je adresatu i Plinije Mlađi uputio dva pisma (1, 13 i 4, 4).

%\newpage

\section*{Pročitajte naglas grčki tekst.}

Plut.\ Quomodo quis suos in virtute sentiat profectus 77D

%Naslov prema izdanju

\medskip


{\large

\begin{greek}

\noindent τούτῳ δ᾽ ὁμοῦ τι ταὐτόν ἐστιν ἢ σύνεγγυς τὸ πρεσβύτατον δήλωμα προκοπῆς τοῦ Ἡσιόδου, μηκέτι προσάντη μηδ᾽ ὄρθιον ἄγαν ἀλλὰ ῥᾳδίαν καὶ λείαν καὶ δι᾽ εὐπετείας εἶναι τὴν ὁδόν, οἷον ἐκλεαινομένην τῇ ἀσκήσει καὶ φῶς ἐν τῷ φιλοσοφεῖν καὶ λαμπρότητα ποιοῦσαν ἐξ ἀπορίας καὶ πλάνης καὶ μεταμελειῶν, αἷς προστυγχάνουσιν οἱ φιλοσοφοῦντες  τὸ πρῶτον, ὥσπερ οἱ γῆν ἀπολιπόντες ἣν ἴσασι, μηδέπω δὲ καθορῶντες ἐφ᾽ ἣν πλέουσι. προέμενοι γὰρ τὰ κοινὰ καὶ συνήθη πρὶν ἢ τὰ βελτίονα γνῶναι καὶ λαβεῖν, ἐν μέσῳ περιφέρονται πολλάκις ὑποτρεπόμενοι. καθάπερ φασὶ Σέξτιον τὸν Ῥωμαῖον ἀφεικότα τὰς ἐν τῇ πόλει τιμὰς καὶ ἀρχὰς διὰ φιλοσοφίαν, ἐν δὲ τῷ φιλοσοφεῖν αὖ πάλιν δυσπαθοῦντα καὶ χρώμενον τῷ λόγῳ χαλεπῷ τὸ πρῶτον, ὀλίγου δεῆσαι καταβαλεῖν ἑαυτὸν ἔκ τινος διήρους. καὶ περὶ Διογένους ὅμοια τοῦ Σινωπέως ἱστοροῦσιν ἀρχομένου φιλοσοφεῖν, ὡς Ἀθηναίοις ἦν ἑορτὴ καὶ δεῖπνα δημοτελῆ καὶ θέατρα, καὶ συνουσίας μετ᾽ ἀλλήλων ἔχοντες ἐχρῶντο κώμοις καὶ παννυχίσιν, ὁ δ᾽ ἔν τινι γωνίᾳ συνεσπειραμένος ὡς καθευδήσων ἐνέπιπτεν εἰς λογισμοὺς τρέποντας αὐτὸν οὐκ ἀτρέμα καὶ θραύοντας, ὡς ἀπ᾽ οὐδεμιᾶς ἀνάγκης εἰς ἐπίπονον καὶ ἀλλόκοτον ἥκων βίον αὐτὸς ὑφ᾽ ἑαυτοῦ κάθηται τῶν ἀγαθῶν ἁπάντων ἐστερημένος.

\end{greek}

}


\section*{Analiza i komentar}

%1

{\large
\begin{greek}
\noindent τούτῳ δ᾽ ὁμοῦ τι\\
ταὐτόν ἐστιν \\
ἢ σύνεγγυς \\
τὸ πρεσβύτατον δήλωμα \\
\tabto{2em} προκοπῆς\\
\tabto{2em} τοῦ Ἡσιόδου, \\
\tabto{2em} μηκέτι προσάντη μηδ᾽ ὄρθιον ἄγαν \\
\tabto{2em} ἀλλὰ ῥᾳδίαν καὶ λείαν καὶ δι᾽ εὐπετείας \\
\tabto{2em} εἶναι \\
\tabto{2em} τὴν ὁδόν, \\
\tabto{4em} οἷον ἐκλεαινομένην \\
\tabto{6em} τῇ ἀσκήσει \\
\tabto{4em} καὶ φῶς \\
\tabto{6em} ἐν τῷ φιλοσοφεῖν \\
\tabto{4em} καὶ λαμπρότητα \\
\tabto{4em} ποιοῦσαν \\
\tabto{6em} ἐξ ἀπορίας καὶ πλάνης καὶ μεταμελειῶν, \\
\tabto{8em} αἷς προστυγχάνουσιν \\
\tabto{8em} οἱ φιλοσοφοῦντες \\
\tabto{8em} τὸ πρῶτον, \\
\tabto{10em} ὥσπερ οἱ γῆν ἀπολιπόντες \\
\tabto{12em} ἣν ἴσασι, \\
\tabto{10em} μηδέπω δὲ καθορῶντες \\
\tabto{12em} ἐφ᾽ ἣν πλέουσι. \\

\end{greek}
}

\begin{description}[noitemsep]
\item[τούτῳ\dots\ ὁμοῦ] prilog s d. LSJ ὁμοῦ II
\item[ταὐτόν ἐστιν ἢ σύνεγγυς] LSJ αὐτός III; imenski predikat, Smyth 909; imenski dijelovi koordinirani rastavnim veznikom
\item[δήλωμα] budući da izriče ideju blisku \textit{verbum sentiendi}, ova riječ otvara u rečenici mjesto A+I
\item[τοῦ Ἡσιόδου] član uz vlastita imena u značenju ``onaj poznati'', §~374. bilj.; Heziod, uz Homera najstariji grčki epski pjesnik, tvorac didaktičkog i genealoškog epa, iz Askre u Beotiji, VII.~st.\ pr.~Kr.
\item[τὸ πρεσβύτατον δήλωμα] \textbf{προκοπῆς τοῦ Ἡσιόδου} Hesiodus, Opera et dies 286-292: \begin{greek} Σοὶ δ' ἐγὼ ἐσθλὰ νοέων ἐρέω, μέγα νήπιε Πέρση· / τὴν μέν τοι κακότητα καὶ ἰλαδὸν ἔστιν ἑλέσθαι / ῥηιδίως· λείη μὲν ὁδός, μάλα δ' ἐγγύθι ναίει· / τῆς δ' ἀρετῆς ἱδρῶτα θεοὶ προπάροιθεν ἔθηκαν / ἀθάνατοι· μακρὸς δὲ καὶ ὄρθιος οἶμος ἐς αὐτὴν / καὶ τρηχὺς τὸ πρῶτον· ἐπὴν δ' εἰς ἄκρον ἵκηται, / ῥηιδίη δὴ ἔπειτα πέλει, χαλεπή περ ἐοῦσα. \end{greek}
\item[μηκέτι προσάντη μηδ᾽ ὄρθιον\dots] \textgreek{\textbf{ἀλλὰ ῥᾳδίαν καὶ λείαν καὶ δι᾽ εὐπετείας εἶναι}} §~315; imenski predikat, Smyth 909; imenski dijelovi koordinirani sastavnim i suprotnim veznicima; predikatni dio A+I
\item[οἷον] priložno, LSJ οἶος A.1: ništa drugo nego\dots, upravo\dots
\item[ἐκλεαινομένην] §~232
\item[ἐν τῷ φιλοσοφεῖν] §~243; supstantivirani infinitiv §~497
\item[ποιοῦσαν] §~243
\item[αἷς προστυγχάνουσιν] odnosna zamjenica uvodi zavisno odnosnu rečenicu, antecedent su \textgreek[variant=ancient]{ἀπορίας καὶ πλάνης καὶ μεταμελειῶν}; §~231; složenica τυγχάνω; rekcija τινί
\item[οἱ φιλοσοφοῦντες] §~243; supstantiviranje participa članom §~373
\item[οἱ γῆν ἀπολιπόντες] §~254; složenica λείπω, §~256; supstantiviranje participa (objekt participa u atributnom je položaju) članom §~373
\item[ὥσπερ οἱ\dots\ μηδέπω δὲ\dots] ὥσπερ uvodi poredbu; koordinacija dvaju suprotstavljenih rečeničnih članova ostvarena je česticom
\item[ἣν ἴσασι] antecedent odnosne zamjenice je γῆν; §~317.4
\item[καθορῶντες] §~243
\item[ἐφ᾽ ἣν] antecedent odnosne zamjenice je γῆν
\item[πλέουσι] §~231, s.~116
\end{description}

%2

{\large
\begin{greek}
\noindent προέμενοι γὰρ \\
τὰ κοινὰ καὶ συνήθη \\
\tabto{2em} πρὶν ἢ \\
\tabto{2em} τὰ βελτίονα \\
\tabto{2em} γνῶναι καὶ λαβεῖν, \\
\tabto{2em} ἐν μέσῳ \\
περιφέρονται \\
\tabto{2em} πολλάκις \\
ὑποτρεπόμενοι.\\
 
\end{greek}
}

\begin{description}[noitemsep]
\item[προέμενοι] složenica ἵημι §~306; med. LSJ προΐημι B.II
\item[γὰρ] čestica uvodi obrazloženje: naime\dots
\item[πρὶν ἢ\dots\ γνῶναι καὶ λαβεῖν] veznik πρίν otvara mjesto vremenskoj zavisnoj rečenici, slaže se s infinitivom ako je glavna rečenica pozitivna §~488.1
\item[γνῶναι] §~292, §~316.5
\item[λαβεῖν] §~254, §~321.14
\item[περιφέρονται] §~232, složenica φέρω
\item[ὑποτρεπόμενοι] §~232, složenica τρέπω

\end{description}
%3

{\large
\begin{greek}
\noindent καθάπερ φασὶ \\
\tabto{2em} Σέξτιον τὸν Ῥωμαῖον \\
\tabto{2em} ἀφεικότα \\
\tabto{2em} τὰς \\
\tabto{4em} ἐν τῇ πόλει \\
\tabto{2em} τιμὰς καὶ ἀρχὰς \\
\tabto{4em} διὰ φιλοσοφίαν, \\
\tabto{2em} ἐν δὲ τῷ φιλοσοφεῖν \\
\tabto{2em} αὖ πάλιν \\
\tabto{2em} δυσπαθοῦντα \\
\tabto{2em} καὶ χρώμενον \\
\tabto{4em} τῷ λόγῳ \\
\tabto{6em} χαλεπῷ \\
\tabto{8em} τὸ πρῶτον, \\
\tabto{2em} ὀλίγου δεῆσαι \\
\tabto{4em} καταβαλεῖν \\
\tabto{6em} ἑαυτὸν \\
\tabto{8em} ἔκ τινος διήρους. \\

\end{greek}
}

\begin{description}[noitemsep]
\item[φασὶ] §~312.8; kao \textit{verbum dicendi} otvara mjesto A+I
\item[Σέξτιον τὸν Ῥωμαῖον] Kvint Sekstije Niger, stoičko-pitagorejski filozof, odbio je položaj senatora koji mu je nudio Julije Cezar; Seneka, koji piše o Sekstiju, spominje i da se taj filozof na kraju svakog dana samoanalizirao, pitajući se ``Koje si se loše navike danas odučio? Kojem si se poroku odupro? Kako si postao bolji?'' (Seneca, De ira 3, 36)
\item[ἀφεικότα τὰς ἐν τῇ πόλει τιμὰς\dots] \textgreek{\textbf{ἐν δὲ τῷ φιλοσοφεῖν\dots}} koordinacija suprotstavljenih rečeničnih članova pomoću čestice
\item[ἀφεικότα] §~272; složenica ἵημι §~311
\item[τῷ φιλοσοφεῖν] §~243; supstantivirani infinitiv §~497
\item[δυσπαθοῦντα] §~243
\item[χρώμενον] §~243; rekcija τινι; LSJ χράομαι II, pojam kao što je λόγος zahtijeva u hrvatskom prijevodu drugačiji glagol (npr.\ baviti se)
\item[τὸ πρῶτον] priložno; LSJ πρότερος B.III.3
\item[ὀλίγου δεῆσαι] §~267, §~269; fraza LSJ δεῖ II.b; infinitiv iz A+I ovisan o φασὶ; i sama fraza otvara mjesto nužnoj dopuni u infinitivu
\item[καταβαλεῖν] §~243, nužna dopuna uz δεῆσαι

\end{description}
%4

{\large
\begin{greek}
\noindent καὶ περὶ Διογένους \\
ὅμοια \\
\tabto{2em} τοῦ Σινωπέως \\
ἱστοροῦσιν \\
\tabto{2em} ἀρχομένου \\
\tabto{4em} φιλοσοφεῖν, \\
ὡς Ἀθηναίοις ἦν\\
\tabto{2em} ἑορτὴ \\
\tabto{2em} καὶ δεῖπνα δημοτελῆ \\
\tabto{2em} καὶ θέατρα, \\
καὶ συνουσίας \\
\tabto{2em} μετ᾽ ἀλλήλων \\
ἔχοντες \\
ἐχρῶντο \\
\tabto{2em} κώμοις καὶ παννυχίσιν, \\
ὁ δ᾽ \\
\tabto{2em} ἔν τινι γωνίᾳ \\
συνεσπειραμένος \\
\tabto{2em} ὡς καθευδήσων \\
ἐνέπιπτεν \\
\tabto{2em} εἰς λογισμοὺς\\
\tabto{4em} τρέποντας \\
\tabto{6em} αὐτὸν \\
\tabto{6em} οὐκ ἀτρέμα \\
\tabto{4em} καὶ θραύοντας, \\
\tabto{6em} ὡς \\
\tabto{8em} ἀπ᾽ οὐδεμιᾶς ἀνάγκης \\
\tabto{8em} εἰς ἐπίπονον καὶ ἀλλόκοτον\\
\tabto{6em} ἥκων\\
\tabto{8em} βίον \\
\tabto{6em} αὐτὸς ὑφ᾽ ἑαυτοῦ \\
\tabto{6em} κάθηται \\
\tabto{8em} τῶν ἀγαθῶν ἁπάντων \\
\tabto{6em} ἐστερημένος.\\

\end{greek}
}

\begin{description}[noitemsep]
\item[ἱστοροῦσιν] §~243
\item[ἀρχομένου] particip u genitivu ovisi o prijedložnom izrazu \textgreek{περὶ Διογένους\dots\ τοῦ Σινωπέως}, otvara mjesto infinitivu; §~232
\item[φιλοσοφεῖν] §~243, dopuna uz ἀρχομένου
\item[ὡς Ἀθηναίοις ἦν] veznik uvodi zavisnu vremensku rečenicu §~487; imenski predikat Smyth 909; LSJ εἰμί C.III
\item[ἔχοντες] §~231
\item[ἐχρῶντο] §~243; rekcija τινι; LSJ χράομαι II, zabavne aktivnosti kao što su κῶμος i παννυχίς u hrvatskom su u kolokaciji s drugim glagolima (npr.\ organizirati, provoditi se u\dots)
\item[ὁ δ᾽] sc.\ Διογένης
\item[συνεσπειραμένος] §~272, §~274.3; složenica σπειράω, med. LSJ συσπειράω A.3
\item[ὡς καθευδήσων] §~258; particip futura upotrijebljen adverbno, s ὡς izražava namjeru §~503.3
\item[ἐνέπιπτεν] §~231, složenica πίπτω, LSJ ἐμπίπτω 4.b
\item[τρέποντας\dots\ θραύοντας] §~231
\item[ἀτρέμα] prilog; DGE s.~v.\ 2
\item[ὡς] otvara mjesto κάθηται, v.~niže
\item[ἥκων] adverbni particip; §~231; §~453.2
\item[αὐτὸς ὑφ᾽ ἑαυτοῦ] sam zbog sebe, sam si je kriv što\dots\ LSJ ὑπό A.II
\item[κάθηται] §~315.a
\item[ἐστερημένος] §~272, §~274.3; rekcija τινος

\end{description}

%kraj
