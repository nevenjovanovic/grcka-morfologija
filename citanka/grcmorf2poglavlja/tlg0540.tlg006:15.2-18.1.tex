% Priredila NČ <2022-01-08 sub>
% Unio, dopunio NJ <2022-01-08 sub>

\section*{O tekstu}

Lizijin šesti govor, \textit{Protiv Andokida optuženog za bezboštvo} \textgreek{(Κατ' Ἀνδοκίδου ἀσεβείας),} bavi se sudskim slučajem iz 400.\ ili 399.\ pr. Kr., potaknutim skandalom koji se u Ateni dogodio 415. U jeku Peloponeskoga rata i uoči kobnog sicilskog pohoda u Ateni su noću vandalizirane herme, kipovi posvećeni bogu Hermesu (s glavom božanstva i istaknutim falusom). Grad je učinio sve kako bi se otkrili počinitelji bezbožnoga čina; istraga je razotkrila još jedan bezbožan čin – izrugivanje Eleuzinskim misterijama izvođenjem obreda po privatnim kućama pred neiniciranima. Za oba bezboštva bio je optužen Andokid, pripadnik ugledne aristokratske obitelji (još poznatiji osumnjičeni bio je Alkibijad). Uhićen, Andokid je u zatvoru priznao sve što je znao, bio je osuđen po Izotimidovu ukazu, po kojem oni koji su počinili djela bezboštva ne smiju ulaziti na javna i religijska mjesta, ali je uspio sačuvati glavu. Otišao je iz Atene u svojevoljan egzil; dozvolu za povratak dobio je tek nakon Trazibulove amnestije 403.\ pr.~Kr. Ubrzo je bio optužen za kršenje Izotimidova ukaza; amnestija je, naime, bila samo politička, nije obuhvaćala kazne za činove bezbožnosti. Andokidov slučaj jedan je od rijetkih atenskih sudskih postupaka u kojima su sačuvani govori obiju strana, optužbe i obrane.\footnote{Usporedi \autoref{chap:andocides} ove čitanke.}

U odabranom se odlomku nužnost osude Andokida argumentira nizom analogija: zločin svetogrđa uspoređuje se s kaznenim djelom tjelesne ozljede, govornik upozorava kako strogo drugi Grci kažnjavaju svetogrđe te kako su sami Atenjani bili strogi prema bezbožnome strancu Dijagori. Govornik spominje Areopag \textgreek{(ἡ Βουλὴ ἡ τοῦ Ἀρείου πάγου),} najstariji sud u Ateni pred kojim se najčešće sudilo zbog ubojstva i ostalih krvnih delikata, te pojam atenskog prava \textgreek{ἔνδειξις,} vrsta pokretanja optužnice u slučajevima zloporaba građanskog prava. 


%\newpage

\section*{Pročitajte naglas grčki tekst.}

Lys.\ In Andocidem 15.2–18.1

%Naslov prema izdanju

\medskip


{\large

\begin{greek}

\noindent ἐὰν μέν τις ἀνδρὸς σῶμα τρώσῃ, κεφαλὴν ἢ πρόσωπον ἢ χεῖρας ἢ πόδας, οὗτος μὲν κατὰ τοὺς νόμους τοὺς ἐξ Ἀρείου ⟨πάγου⟩ φεύξεται τὴν τοῦ ἀδικηθέντος πόλιν, καὶ ἐὰν κατίῃ, ἐνδειχθεὶς θανάτῳ ζημιωθήσεται· ἐὰν δέ τις τὰ αὐτὰ ταῦτα ἀδικήσῃ τὰ ἀγάλματα τῶν θεῶν, οὐδ' αὐτῶν κωλύσετε τῶν ἱερῶν ἐπιβαίνειν ἢ εἰσιόντα ⟨οὐ⟩ τιμωρήσεσθε; καὶ μὲν δὴ τούτων καὶ δίκαιον καὶ ἀγαθόν ἐστιν ἐπιμελεῖσθαι, ὑφ' ὧν καὶ εὖ καὶ κακῶς δυνήσεσθε πάσχειν. φασὶ δὲ καὶ τῶν Ἑλλήνων πολλοὺς διὰ τὰ ἐνθάδε ἀσεβήματα ἐκ τῶν παρ' αὐτοῖς ἱερῶν ἐξείργειν. ὑμεῖς δὲ αὐτοὶ οἱ ἀδικηθέντες περὶ ἐλάττονος ποιεῖσθε τὰ παρ' ὑμῖν νόμιμα ἢ ἕτεροι τὰ ὑμέτερα. τοσοῦτο δ' οὗτος Διαγόρου τοῦ Μηλίου ἀσεβέστερος γεγένηται· ἐκεῖνος μὲν γὰρ λόγῳ περὶ τὰ ἀλλότρια ἱερὰ καὶ ἑορτὰς ἠσέβει, οὗτος δὲ ἔργῳ περὶ τὰ ἐν τῇ αὑτοῦ πόλει. ὀργίζεσθαι οὖν χρή, ὦ ἄνδρες Ἀθηναῖοι, τοῖς ἀστοῖς ἀδικοῦσι μᾶλλον ἢ τοῖς ξένοις περὶ ταῦτα [τὰ ἱερά]· τὸ μὲν γὰρ ὥσπερ ἀλλότριόν ἐστιν ἁμάρτημα, τὸ δ' οἰκεῖον.

\end{greek}

}

\newpage

\section*{Analiza i komentar}

%1

{\large
  
\begin{greek}
\noindent ἐὰν μέν τις ἀνδρὸς σῶμα τρώσῃ,\\
\tabto{2em} κεφαλὴν ἢ πρόσωπον ἢ χεῖρας ἢ πόδας,\\ 
οὗτος μὲν\\
\tabto{2em} κατὰ τοὺς νόμους τοὺς ἐξ Ἀρείου ⟨πάγου⟩\\
φεύξεται τὴν τοῦ ἀδικηθέντος πόλιν,\\
καὶ ἐὰν κατίῃ,\\
ἐνδειχθεὶς θανάτῳ ζημιωθήσεται·\\
ἐὰν δέ τις τὰ αὐτὰ ταῦτα ἀδικήσῃ τὰ ἀγάλματα τῶν θεῶν, \\
οὐδ' αὐτῶν κωλύσετε τῶν ἱερῶν ἐπιβαίνειν \\
ἢ εἰσιόντα ⟨οὐ⟩ τιμωρήσεσθε;\\

\end{greek}
}

%\newpage

\begin{description}[noitemsep]
\item[ἐὰν μέν τις\dots\ ἐὰν δέ τις\dots] koordinacija surečenica pomoću čestica μέν\dots\ δέ\dots (uočite i sadržajni paralelizam)
\item[ἐὰν] pogodbeni veznik ἐάν uvodi zavisnu eventualnu pogodbenu rečenicu, „ako\dots”, §~476.1
\item[οὗτος] sc.\ τρώσας τις
\item[τρώσῃ] §~267 
\item[τοὺς ἐξ Ἀρείου ⟨πάγου⟩] prijedložni izraz u (jače istaknutom) atributnom položaju §~375; prelomljene zagrade označavaju priređivačevu emendaciju (dodano je što je u rukopisnoj predaji izostavljeno)
\item[φεύξεται] LSJ φεύγω III, rekcija τι; §~265, §~301
\item[τοῦ ἀδικηθέντος] supstantivirani particip u atributnom položaju, §~296
\item[κατίῃ] složenica glagola εἶμι, §~314.1, 
\item[ἐνδειχθεὶς] LSJ ἐνδείκνυμι I.2 (u pravnom kontekstu); adverbni particip s hipotetičkim značenjem §~318; složenica glagola δείκνυμι, §~296.1, §~318
\item[ζημιωθήσεται] otvara mjesto dopuni u dativu \textit{(dativus instrumenti);} §~296.2
\item[τὰ αὐτὰ ταῦτα] „istim takvim zločinom”, akuzativ unutrašnjeg objekta §~385.2
\item[ἀδικήσῃ] rekcija τι; §~267
\item[κωλύσετε] §~258; otvara mjesto dopuni u infinitivu
\item[αὐτῶν] sc.\ τῶν θεῶν; atribut τῶν ἱερῶν
\item[ἐπιβαίνειν] rekcija τινος „stupiti pred što”; složenica glagola βαίνω; §~231; §~321.6
\item[οὐδ'\dots\ ἢ ⟨οὐ⟩] §~520; zavisno disjunktivno pitanje Smyth 2657; prelomljene zagrade označavaju priređivačevu emendaciju (dodano je što je u rukopisnoj predaji izostavljeno)
\item[εἰσιόντα] adverbni particip s hipotetičkim značenjem §~318; složenica glagola εἶμι, §~314.1
\item[τιμωρήσεσθε] §~258

\end{description}

% 2

\bigskip

{\large
  
\begin{greek}
\noindent καὶ μὲν δὴ τούτων καὶ δίκαιον καὶ ἀγαθόν ἐστιν ἐπιμελεῖσθαι,\\
\tabto{2em} ὑφ' ὧν καὶ εὖ καὶ κακῶς δυνήσεσθε πάσχειν.\\

\end{greek}
}

\begin{description}[noitemsep]
\item[καὶ μὲν δὴ] sveza čestica koje Lizija često koristi (kod drugih je autora češća sveza καὶ μήν), iskazuje prijelaz na novu misao (Denniston 395)
\item[καὶ δίκαιον καὶ ἀγαθόν ἐστιν] imenski predikat, Smyth 909; bezlični izraz otvara mjesto infinitivu (kao subjektu) §~492; imenski dijelovi koordinirani sastavnim veznicima
\item[ἐπιμελεῖσθαι] rekcija τινός; §~243
\item[ὑφ' ὧν] prijedložni izraz uvodi zavisnu odnosnu rečenicu, antecedent je τούτων
\item[δυνήσεσθε] glagol nepotpuna značenja otvara mjesto dopuni u infinitivu; §~258, §~312.5
\item[πάσχειν] rekcija τι ὑπό τινος; §~231, §~327.15

\end{description}

% 3

\bigskip

{\large
  
\begin{greek}
\noindent φασὶ δὲ καὶ τῶν Ἑλλήνων πολλοὺς\\
\tabto{2em} διὰ τὰ ἐνθάδε ἀσεβήματα\\
\tabto{2em} ἐκ τῶν παρ' αὐτοῖς ἱερῶν\\
ἐξείργειν.\\

\end{greek}
}

\begin{description}[noitemsep]
\item[φασὶ] \textit{verbum dicendi} otvara mjesto akuzativu s infinitivom; §~312.8
\item[τῶν Ἑλλήνων] genitiv partitivni, atribut akuzativa πολλοὺς (subjekta A+I)
\item[ἐξείργειν] rekcija ἔκ τινος, složenica glagola εἴργω (atički oblik, LSJ ἐξέργω); §~321
\item[ἐνθάδε] prilog u atributnom položaju
\item[παρ' αὐτοῖς] sc.\ παρὰ πολλοῖς τῶν Ἑλλήνων; prijedložni izraz u atributnom položaju
\end{description}

% 4

\bigskip

{\large
\begin{greek}
\noindent ὑμεῖς δὲ αὐτοὶ οἱ ἀδικηθέντες\\
\tabto{2em} περὶ ἐλάττονος ποιεῖσθε\\
\tabto{4em} τὰ παρ' ὑμῖν νόμιμα\\
\tabto{2em} ἢ ἕτεροι τὰ ὑμέτερα.\\

\end{greek}
}

\begin{description}[noitemsep]
\item[οἱ ἀδικηθέντες] supstantivirani particip (obratite pažnju na glagolsko stanje); §~296.1
\item[ποιεῖσθε] §~243
\item[περὶ ἐλάττονος ποιεῖσθε\dots\ ἢ\dots] LSJ ἐλάσσων A.3
\item[παρ' ὑμῖν] prijedložni izraz u atributnom položaju
\item[ἕτεροι τὰ ὑμέτερα] sc.\ νόμιμα ποιοῦσιν
\end{description}

% 5

\bigskip

{\large
\begin{greek}
\noindent τοσοῦτο δ' οὗτος Διαγόρου τοῦ Μηλίου ἀσεβέστερος γεγένηται·\\
ἐκεῖνος μὲν γὰρ λόγῳ\\
\tabto{2em} περὶ τὰ ἀλλότρια ἱερὰ καὶ ἑορτὰς ἠσέβει,\\
οὗτος δὲ ἔργῳ\\
\tabto{2em} περὶ τὰ ἐν τῇ αὑτοῦ πόλει.\\

\end{greek}
}

\begin{description}[noitemsep]
\item[οὗτος] sc.\ Ἀνδοκίδης; u atenskim sudskim govorima uobičajeno je referirati se na protivnika pokaznom zamjenicom
\item[Διαγόρου τοῦ Μηλίου ἀσεβέστερος] \textit{genetivus comparationis} uz komparativ pridjeva; Dijagora s Mela, pjesnik i filozof iz V.~st.\ pr.~Kr., slavan po radikalnom ateizmu; u Ateni su mu sudili zbog bezbožnosti možda 433./432.\ te opet u vrijeme vandaliziranja hermi, sklonio se u Pelenu, grad na strani Spartanaca, umro vjerojatno u Korintu; djela nisu sačuvana
\item[γεγένηται] u kopulativnoj funciji, otvara mjesto imenskoj dopuni; §~284, §~325.11
\item[ἐκεῖνος μὲν\dots\ οὗτος δὲ\dots] koordinacija rečeničnih članova; paralelizam se nastavlja izrazima \textgreek{λόγῳ\dots\ ἔργῳ, τὰ ἀλλότρια\dots\ τὰ ἐν τῇ αὑτοῦ πόλει}
\item[γὰρ] čestica γάρ najavljuje iznošenje razloga ili objašnjenja, „naime…“
\item[ἠσέβει] rekcija περί τι; §~243 
\item[ἐν τῇ αὑτοῦ πόλει] prijedložni izraz u atributnom položaju
\item[αὑτοῦ] atički stegnut oblik zamjenice ἑαυτοῦ
\end{description}

% 6

\bigskip

{\large
\begin{greek}
\noindent ὀργίζεσθαι οὖν χρή,\\
ὦ ἄνδρες ᾿Αθηναῖοι,\\
\tabto{2em} τοῖς ἀστοῖς ἀδικοῦσι \\
\tabto{2em} μᾶλλον ἢ τοῖς ξένοις\\
\tabto{4em} περὶ ταῦτα [τὰ ἱερά]·\\
τὸ μὲν γὰρ\\
\tabto{2em} ὥσπερ ἀλλότριόν ἐστιν ἁμάρτημα,\\
τὸ δ' οἰκεῖον.\\

\end{greek}
}

\begin{description}[noitemsep]
\item[χρή] otvara mjesto dopuni u infinitivu; §~231, §~315, Β. 4
\item[ὀργίζεσθαι] rekcija τινι; §~231
\item[οὖν] čestica zaključnog značenja: „dakle…“
\item[ἀδικοῦσι] rekcija περί τι, LSJ s.~v.; §~243
\item[[τὰ ἱερά]] uglate zagrade označavaju riječi koje po priređivačevu mišljenju nisu stajale u izvorniku, usprkos rukopisnoj predaji (takvo se izbacivanje naziva ateteza)
\item[τὸ μὲν\dots\ τὸ δ'\dots] sc.\ \textgreek{οἱ ξένοι ἀδικοῦντες\dots\ οἱ ἀστοὶ ἀδικοῦντες;} koordinacija rečeničnih članova česticama
\item[ἀλλότριόν ἐστιν\dots\ οἰκεῖον] §~315; imenski predikat, Smyth 909
\end{description}



%kraj
