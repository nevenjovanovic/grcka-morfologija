%\section*{O autoru}

% Unio ispravke NZ <2021-12-27 pon>


\section*{O tekstu}

Od tri Aristotelova spisa koji se bave etikom, Ἠθικὰ Νικομάχεια (\textit{Nikomahova etika}, deset knjiga) danas je najpoznatija, i smatra se da predstavlja ``najbolji izvor za razumijevanje Aristotelovih pogleda na probleme ljudskog djelovanja'' (πράττειν). Prema Aristotelu, najviši je cilj djelovanja sreća \begin{greek}(εὐδαιμονία)\end{greek}, shvaćena kao potpuno samoostvarenje, odnosno ostvarenje svih potencijala pojedinca. Sreća se postiže pomoću vrline, koja je μεσότης, sredina između krajnosti (hrabrost kao sredina između kukavičluka i neopreznosti, darežljivost kao sredina između škrtosti i rasipnosti); \begin{greek}μεσότης τις ἄρα ἐστὶν ἡ ἀρετή\end{greek}, Eth. Nic. 1106b.

Poput ostalih Aristotelovih djela koja su stigla do nas, Ἠθικὰ Νικομάχεια pripada ``ezoteričnim'' spisima koji, za razliku od ``egzoteričnih'', nisu bili namijenjeni široj cirkulaciji, već upotrebi unutar škole (Λύκειον je bio antički ekvivalent modernoga istraživačkog instituta). Radi se najvjerojatnije o bilješkama s predavanja, koje su priređivali studenti, a pregledao ih i dopunio sam Aristotel; te je ezoterične spise, koji su dugo vremena ostali zaboravljeni, u I.~st.\ pr.~Kr.\ priredio i objavio Andronik s Roda (Ἀνδρόνικος ὁ Ῥόδιος), σχολάρχης peripatetičke škole.

Knjige Θ i Ι (8 i 9) \textit{Nikomahove etike} razmatraju prijateljstvo (φιλία), koje je samo po sebi vrlina, ili uključuje vrlinu. Prijateljstvu kao temi tako je u čitavom djelu posvećeno najviše mjesta, i razmatranje te teme neposredno prethodi kraju čitave rasprave. Dok knjiga Θ govori o prijateljstvu kao o umanjenoj verziji društva, u kojem ljude povezuje spona jača od pravde, knjiga Ι vidi prijateljstvo kao proširenje vlastite osobnosti (ἔστι γὰρ ὁ φίλος ἄλλος αὐτός, Ι 1166a), mogućnost da se vlastiti potencijali razviju do najveće mjere.

U našem se odlomku izvještava o shvaćanju da onom tko je sretan \begin{greek}(εὐδαίμων)\end{greek} ne trebaju prijatelji, i iznose se prigovori protiv tog shvaćanja; prijateljstvo implicira djelovanje, ne samo u nesreći, nego i u sreći.

%\newpage

\section*{Pročitajte naglas grčki tekst.}

Arist. Ethica Nicomachea 1169b (9.9)

%Naslov prema izdanju

\medskip


{\large

\begin{greek}

\noindent  ἀμφισβητεῖται δὲ καὶ περὶ τὸν εὐδαίμονα, εἰ δεήσεται φίλων ἢ μή. οὐθὲν γάρ φασι δεῖν φίλων τοῖς μακαρίοις καὶ αὐτάρκεσιν· ὑπάρχειν γὰρ αὐτοῖς τἀγαθά· αὐτάρκεις οὖν ὄντας οὐδενὸς προσδεῖσθαι, τὸν δὲ φίλον, ἕτερον αὐτὸν ὄντα, πορίζειν ἃ δι' αὑτοῦ ἀδυνατεῖ· ὅθεν ``ὅταν ὁ δαίμων εὖ διδῷ, τί δεῖ φίλων;'' ἔοικε δ' ἀτόπῳ τὸ πάντ' ἀπονέμοντας τἀγαθὰ τῷ εὐδαίμονι φίλους μὴ ἀποδιδόναι, ὃ δοκεῖ τῶν ἐκτὸς ἀγαθῶν μέγιστον εἶναι. εἴ τε φίλου μᾶλλόν ἐστι τὸ εὖ ποιεῖν ἢ πάσχειν, καὶ ἔστι τοῦ ἀγαθοῦ καὶ τῆς ἀρετῆς τὸ εὐεργετεῖν, κάλλιον δ' εὖ ποιεῖν φίλους ὀθνείων, τῶν εὖ πεισομένων δεήσεται ὁ σπουδαῖος. διὸ καὶ ἐπιζητεῖται πότερον ἐν εὐτυχίαις μᾶλλον δεῖ φίλων ἢ ἐν ἀτυχίαις, ὡς καὶ τοῦ ἀτυχοῦντος δεομένου τῶν εὐεργετησόντων καὶ τῶν εὐτυχούντων οὓς εὖ ποιήσουσιν. ἄτοπον δ' ἴσως καὶ τὸ μονώτην ποιεῖν τὸν μακάριον· οὐδεὶς γὰρ ἕλοιτ' ἂν καθ' αὑτὸν τὰ πάντ' ἔχειν ἀγαθά· πολιτικὸν γὰρ ὁ ἄνθρωπος καὶ συζῆν πεφυκός. καὶ τῷ εὐδαίμονι δὴ τοῦθ' ὑπάρχει\dots

\end{greek}

}


\section*{Analiza i komentar}

%1

{\large
\begin{greek}
\noindent  ἀμφισβητεῖται δὲ καὶ \\
\tabto{2em} περὶ τὸν εὐδαίμονα, \\
εἰ δεήσεται φίλων \\
\tabto{2em} ἢ μή.\\

\end{greek}
}

\begin{description}[noitemsep]
\item[ἀμφισβητεῖται] §~243
\item[δὲ] čestica δέ označava nadovezivanje na prethodni iskaz
\item[εἰ\dots\ ἢ μή] čestica εἰ otvara mjesto zavisnoj upitnoj rečenici, ovdje kao disjunktivno (alternativno) pitanje, §~469, Smyth 2675 c
\item[δεήσεται] §~258, §~259; §~325.15, §~328.2; δέομαί τινος (\textit{genetivus copiae}) §~403.1.β

\end{description}

%2
{\large
\begin{greek}
\noindent  οὐθὲν γάρ φασι δεῖν \\
\tabto{2em} φίλων \\
τοῖς μακαρίοις καὶ αὐτάρκεσιν· \\
ὑπάρχειν γὰρ \\
\tabto{2em} αὐτοῖς \\
τἀγαθά· \\
αὐτάρκεις οὖν ὄντας \\
\tabto{2em} οὐδενὸς \\
προσδεῖσθαι, \\
τὸν δὲ φίλον, \\
\tabto{2em} ἕτερον αὐτὸν ὄντα, \\
πορίζειν \\
\tabto{2em} ἃ \\
\tabto{4em} δι' αὑτοῦ \\
\tabto{2em} ἀδυνατεῖ· \\
ὅθεν \\
\tabto{2em} ``ὅταν ὁ δαίμων \\
\tabto{2em} εὖ διδῷ, \\
\tabto{2em} τί δεῖ \\
\tabto{4em} φίλων;'' \\
ἔοικε δ' \\
\tabto{2em} ἀτόπῳ \\
τὸ \\
\tabto{2em} πάντ' ἀπονέμοντας τἀγαθὰ \\
\tabto{4em} τῷ εὐδαίμονι \\
\tabto{2em} φίλους \\
\tabto{2em} μὴ ἀποδιδόναι, \\
ὃ δοκεῖ \\
\tabto{2em} τῶν ἐκτὸς ἀγαθῶν \\
μέγιστον εἶναι.\\

\end{greek}
}

\begin{description}[noitemsep]
\item[οὐθὲν] alternativni (kasniji) oblik umjesto οὐδέν
\item[γάρ] čestica γάρ uvodi objašnjenje prethodne tvrdnje, ``naime''
\item[φασι] §~312.8; \textit{verbum dicendi} otvara mjesto akuzativima s infinitivom οὐθὲν\dots\ δεῖν, ὑπάρχειν\dots\ τἀγαθά, αὐτάρκεις οὖν ὄντας\dots\ προσδεῖσθαι, τὸν δὲ φίλον\dots\ πορίζειν
\item[δεῖν] §~231; §~325.14; bezličan glagol, δεῖ τινί τινος
\item[ὑπάρχειν] §~231; ὑπάρχει τινί τι
\item[γὰρ] čestica γάρ uvodi objašnjenje prethodne tvrdnje, ``naime''
\item[αὐτάρκεις\dots\ ὄντας] §~315; pridjev je nužna dopuna uz kopulativni glagol
\item[προσδεῖσθαι] §~243; složenica glagola δέομαι, §~325.15, §~328.2; rekcija kao kod δέομαι, usp. gore
\item[αὐτάρκεις οὖν\dots\ τὸν δὲ φίλον\dots] koordinacija rečeničnih članova pomoću čestice δέ
\item[ἕτερον αὐτὸν ὄντα] §~315; zamjenice su nužna dopuna uz kopulativni glagol; za ἕτερος αὐτός usp.\ latinski \textit{alter ego}
\item[πορίζειν] §~231
\item[ἀδυνατεῖ] §~243; sc. πορίζειν + neizrečeni subjekt τις ili ἄνθρωπος (``netko, čovjek'')
\item[ὅταν\dots\ διδῷ] §~305; veznik ὅταν (ὅτε ἄν) uvodi zavisnu vremensku rečenicu u značenju pogodbene protaze eventualnog oblika, s konjunktivom §~488.2; Aristotel citira stih (jampski trimetar) iz Euripidove tragedije Ὀρέστης, izvedene 408.\ pr.~Kr.\ (Eur.\ Orest.\ 665)
\item[δεῖ] §~243; bezličan glagol otvara mjesto za \textit{genetivus copiae}, δεῖ τινος, §~403.1.a
\item[ἔοικε] §~272; §~327b; ἔοικά τινι
\item[δ'] čestica δέ označava nadovezivanje na prethodni iskaz
\item[τὸ\dots\ ἀπονέμοντας\dots\ μὴ ἀποδιδόναι] supstantiviran akuzativ s infinitivom, §~497
\item[ἀπονέμοντας] §~231; ἀπονέμω τινί τι
\item[ἀποδιδόναι] složenica glagola δίδωμι, §~305; negacija uz infinitiv §~509c
\item[ὃ] odnosi se na τὸ\dots\ μὴ ἀποδιδόναι
\item[δοκεῖ] §~243; bezlični glagol otvara mjesto infinitivu, §~492
\item[τῶν ἐκτὸς ἀγαθῶν] prilog u atributnom položaju, §~375.5; prijateljstvo kao najveće od izvanjskih dobara \textgreek{(τὰ ἐκτὸς ἀγαθά)} nužan je uvjet sreće
\item[μέγιστον εἶναι] §~315; pridjevska dopuna kopulativnom glagolu (imenski predikat), Smyth 4.26 910
\end{description}

%3

{\large
\begin{greek}
\noindent  εἴ τε \\
φίλου \\
\tabto{2em} μᾶλλόν \\
ἐστι \\
τὸ εὖ ποιεῖν \\
\tabto{2em} ἢ πάσχειν, \\
καὶ ἔστι \\
τοῦ ἀγαθοῦ \\
\tabto{2em} καὶ τῆς ἀρετῆς \\
τὸ εὐεργετεῖν, \\
\tabto{2em} κάλλιον δ' \\
\tabto{2em} εὖ ποιεῖν\\
\tabto{4em} φίλους \\
\tabto{6em} ὀθνείων, \\
τῶν εὖ πεισομένων \\
δεήσεται ὁ σπουδαῖος.\\

\end{greek}
}

\begin{description}[noitemsep]
\item[εἴ\dots\ φίλου\dots] \textbf{καὶ ἔστι τοῦ ἀγαθοῦ\dots, δεήσεται} realna pogodbena rečenica s dvije protaze i jednom apodozom, §~475
\item[τε] Aristotel se često koristi ovom česticom da bi uveo zaključak ili još jači argument u odnosu na ono što prethodi; ``pa''
\item[φίλου\dots\ ἐστι] §~315; kopulativni glagol otvara mjesto posvojnom genitivu koji izriče \textit{kome} je nešto svojstveno, §~393.2, Smyth  4.42.93.83, 1304 1305
\item[μᾶλλόν\dots\ ἢ\dots] koordinacija rečeničnih članova pomoću priloga i čestice: ``više\dots\ nego\dots''
\item[τὸ\dots\ ποιεῖν] §~243; supstantivirani infinitiv, §~497
\item[πάσχειν] §~231; sc. εὖ πάσχειν
\item[καὶ ἔστι τοῦ ἀγαθοῦ] paralela s φίλου\dots\ ἐστι, v.\ gore
\item[τὸ εὐεργετεῖν] §~243; supstantivirani infinitiv, §~497
\item[κάλλιον] sc. κάλλιόν ἐστι (izostavljen kopulativni glagol); komparativ otvara mjesto za \textit{genetivus comparationis}, §~404.1
\item[δ'] čestica δέ označava nadovezivanje na prethodni iskaz
\item[εὖ ποιεῖν] §~243; akuzativ izvanjeg objekta εὖ ποιέω τινά §~381.1
\item[τῶν\dots\ πεισομένων] §~258; §~327.15; supstantiviranje, §~373
\item[δεήσεται] §~258, §~259; §~325.15, §~328.2; δέομαί τινος (\textit{genetivus copiae}) §~403.1.β
\end{description}


%4

{\large
\begin{greek}
\noindent  διὸ καὶ ἐπιζητεῖται \\
\tabto{2em} πότερον \\
\tabto{4em} ἐν εὐτυχίαις \\
\tabto{2em} μᾶλλον δεῖ φίλων \\
\tabto{2em} ἢ ἐν ἀτυχίαις, \\
\tabto{4em} ὡς καὶ τοῦ ἀτυχοῦντος \\
\tabto{4em} δεομένου \\
\tabto{6em} τῶν εὐεργετησόντων \\
\tabto{4em} καὶ τῶν εὐτυχούντων \\
\tabto{6em} οὓς εὖ ποιήσουσιν.\\

\end{greek}
}

\begin{description}[noitemsep]
\item[ἐπιζητεῖται] §~243
\item[πότερον\dots\ ἢ\dots] koordinacija rečeničnih članova pomoću čestica koje uvode disjunktivno pitanje: ``da li\dots\ ili\dots''
\item[δεῖ] §~243; bezličan glagol otvara mjesto za \textit{genetivus copiae}, δεῖ τινος, §~403.1.a
\item[ὡς] ovdje otvara mjesto genitivu apsolutnom s funkcijom priložne oznake uzroka, §~504
\item[τοῦ ἀτυχοῦντος] §~243; supstantiviranje, §~373
\item[δεομένου] §~232; δέομαί τινος (\textit{genetivus copiae}) §~403.1.β
\item[τῶν εὐεργετησόντων] §~258; §~259; supstantiviranje, §~373
\item[τῶν εὐτυχούντων] §~243; supstantiviranje, §~373; sc. \textgreek{δεομένων τούτων}
\item[οὓς\dots\ ποιήσουσιν] §~258; §~259; akuzativ izvanjeg objekta \textgreek{εὖ ποιέω τινά} §~381.1
\end{description}


%5

{\large
\begin{greek}
\noindent  ἄτοπον δ' ἴσως καὶ \\
τὸ μονώτην ποιεῖν τὸν μακάριον· \\
οὐδεὶς γὰρ ἕλοιτ' ἂν \\
\tabto{2em} καθ' αὑτὸν \\
\tabto{2em} τὰ πάντ' ἔχειν ἀγαθά· \\
πολιτικὸν γὰρ ὁ ἄνθρωπος \\
καὶ συζῆν πεφυκός.\\

\end{greek}
}

\begin{description}[noitemsep]
\item[ἄτοπον] sc. ἐστι
\item[δ'] čestica δέ označava nadovezivanje na prethodni iskaz
\item[τὸ\dots\ ποιεῖν] §~243; supstantivirani infinitiv, §~497
\item[μονώτην ποιεῖν τὸν μακάριον] ποιέω τινά τινα – glagol otvara mjesta dvama akuzativima, objekta i predikata, §~388; predikatna je dopuna apstraktna imenica, Smyth 1612
\item[γὰρ] čestica γάρ uvodi objašnjenje prethodne tvrdnje, ``naime''
\item[ἕλοιτ' ἂν] §~254; §~327.1; potencijal sadašnji u nezavisnoj rečenici, §~464.2; αἱρέομαι otvara mjesto infinitivu, LSJ αἱρέω B.II.b
\item[καθ' αὑτὸν] ``biti sam''
\item[τὰ πάντ'\dots\ ἀγαθά] hiperbat
\item[ἔχειν] §~231
\item[πολιτικὸν\dots\ ὁ ἄνθρωπος] sc.\ \textgreek{πολιτικόν ἐστι} (izostavljen kopulativni glagol); predikatna dopuna je u srednjem rodu iako se slaže s ὁ ἄνθρωπος, podrazumijeva se ``nešto'' ili ``biće''; usp. \textgreek{ὁ ἄνθρωπος φύσει πολιτικὸν ζῷον,} Aristot. Pol. 1.1253a, i \textgreek{φύσει πολιτικὸν ὁ ἄνθρωπος} Nic.\ Eth.\ 1097b
\item[συζῆν] §~243; složenica glagola ζῶ (ζάω)
\item[πεφυκός] §~272; otvara mjesto infinitivu koji izriče svrhu §~495, LSJ φύω B.II.2

\end{description}


%6

{\large
\begin{greek}
\noindent  καὶ τῷ εὐδαίμονι \\
δὴ \\
τοῦθ' ὑπάρχει\dots\\

\end{greek}
}

\begin{description}[noitemsep]
\item[δὴ] čestica naglašava imenicu koja joj prethodi: ``dakle\dots''
\item[ὑπάρχει] §~231; ὑπάρχει τινί τι
\end{description}



%kraj
