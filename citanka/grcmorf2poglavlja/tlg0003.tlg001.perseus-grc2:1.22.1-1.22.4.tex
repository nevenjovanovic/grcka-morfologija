% Unio ispravke NZ <2022-01-04 uto>

\section*{O tekstu}

Tekst koji čitamo ulomak je iz djela \textit{Spis o ratu Peloponežana i Atenjana} \textgreek[variant=ancient]{(Ξυγγραφὴ περὶ τoῦ πολέμoυ τῶν Пελoπoννησίων καὶ Ἀϑηναίων,} ili jednostavnije \textit{Povijest Peloponeskog rata}) atenskog povjesničara Tukidida (460.\ pr.~Kr.\ – oko 396.\ pr.~Kr). Tukidid je sam sudjelovao u ratu, na njegovu početku, ali je zbog optužbi za izdaju kraj rata dočekao u progonstvu. 

Djelo se sastoji od osam knjiga, a prikazuje tijek rata do 411.\ pr.~Kr. Tukidid je snažno utjecao na potonju grčku historiografiju – Ksenofont piše \textit{Grčku povijest} kao nastavak Tukididova djela – i na rimske povjesničare (Salustije, Tacit).

Odabrani je ulomak iz prve knjige; Tukidid objašnjava svoju metodu. U pristupu povijesnim zbivanjima on želi biti objektivan i zato podatke prikuplja sa svih (zaraćenih) strana, a posebno ističe kako zaključke izvodi iz kritički promišljenih dokaza. Razlikuje i dvije formalno različite komponente povjesnice. Jedan su dio govori, a drugi dio pripovijedanje događaja. Govore je sastavljao po onome što je sam čuo i vidio, ali i po podacima koje je sakupio. U rekonstruiranju događaja, u narativnim dijelovima, nije se pouzdavao samo u svoje dojmove, a i tuđe je podatke podvrgnuo najstrožim provjerama. Sve to čini njegovo djelo možda neprivlačnim izvedbeno \textgreek[variant=ancient]{(ἐς ἀκρόασιν)}, ali osigurava ``stečevinu za sva vremena'', \textgreek[variant=ancient]{κτῆμα ἐς αἰεί.}

\newpage

\section*{Pročitajte naglas grčki tekst.}

Thuc.\ Historiae 1.22.1–4

%Naslov prema izdanju

\medskip


{\large

\begin{greek}

\noindent καὶ ὅσα μὲν λόγῳ εἶπον ἕκαστοι ἢ μέλλοντες πολεμήσειν ἢ ἐν αὐτῷ ἤδη ὄντες, χαλεπὸν τὴν ἀκρίβειαν αὐτὴν τῶν λεχθέντων διαμνημονεῦσαι ἦν ἐμοί τε ὧν αὐτὸς ἤκουσα καὶ τοῖς ἄλλοθέν ποθεν ἐμοὶ ἀπαγγέλλουσιν· ὡς δ' ἂν ἐδόκουν ἐμοὶ ἕκαστοι περὶ τῶν αἰεὶ παρόντων τὰ δέοντα μάλιστ' εἰπεῖν, ἐχομένῳ ὅτι ἐγγύτατα τῆς ξυμπάσης γνώμης τῶν ἀληθῶς λεχθέντων, οὕτως εἴρηται.

\noindent τὰ δ' ἔργα τῶν πραχθέντων ἐν τῷ πολέμῳ οὐκ ἐκ τοῦ παρατυχόντος πυνθανόμενος ἠξίωσα γράφειν, οὐδ' ὡς ἐμοὶ ἐδόκει, ἀλλ' οἷς τε αὐτὸς παρῆν καὶ παρὰ τῶν ἄλλων ὅσον δυνατὸν ἀκριβείᾳ περὶ ἑκάστου ἐπεξελθών.

\noindent ἐπιπόνως δὲ ηὑρίσκετο, διότι οἱ παρόντες τοῖς ἔργοις ἑκάστοις οὐ ταὐτὰ περὶ τῶν αὐτῶν ἔλεγον, ἀλλ' ὡς ἑκατέρων τις εὐνοίας ἢ μνήμης ἔχοι.

\noindent καὶ ἐς μὲν ἀκρόασιν ἴσως τὸ μὴ μυθῶδες αὐτῶν ἀτερπέστερον φανεῖται· ὅσοι δὲ βουλήσονται τῶν τε γενομένων τὸ σαφὲς σκοπεῖν καὶ τῶν μελλόντων ποτὲ αὖθις κατὰ τὸ ἀνθρώπινον τοιούτων καὶ παραπλησίων ἔσεσθαι, ὠφέλιμα κρίνειν αὐτὰ ἀρκούντως ἕξει. κτῆμά τε ἐς αἰεὶ μᾶλλον ἢ ἀγώνισμα ἐς τὸ παραχρῆμα ἀκούειν ξύγκειται.

\end{greek}

}


\section*{Analiza i komentar}

%1

{\large
\begin{greek}
\noindent Καὶ ὅσα μὲν \\
\tabto{2em} λόγῳ \\
εἶπον \\
ἕκαστοι\\
\tabto{2em} ἢ μέλλοντες \\
\tabto{4em} πολεμήσειν\\
\tabto{2em} ἢ \\
\tabto{4em} ἐν αὐτῷ \\
\tabto{4em} ἤδη \\
\tabto{2em} ὄντες, \\
χαλεπὸν \\
τὴν ἀκρίβειαν αὐτὴν \\
\tabto{2em} τῶν λεχθέντων \\
διαμνημονεῦσαι ἦν \\
\tabto{2em} ἐμοί τε\\
\tabto{4em} ὧν \\
\tabto{4em} αὐτὸς \\
\tabto{4em} ἤκουσα \\
\tabto{2em} καὶ τοῖς \\
\tabto{4em} ἄλλοθέν ποθεν \\
\tabto{4em} ἐμοὶ \\
\tabto{2em} ἀπαγγέλλουσιν· \\
ὡς δ' ἂν ἐδόκουν \\
ἐμοὶ \\
\tabto{2em} ἕκαστοι \\
\tabto{4em} περὶ τῶν αἰεὶ παρόντων \\
\tabto{4em} τὰ δέοντα \\
\tabto{2em} μάλιστ' εἰπεῖν,\\
ἐχομένῳ \\
ὅτι ἐγγύτατα \\
\tabto{2em} τῆς ξυμπάσης γνώμης \\
\tabto{4em} τῶν ἀληθῶς λεχθέντων, \\
οὕτως εἴρηται.\\

\end{greek}
}

\begin{description}[noitemsep]
\item[ὅσα\dots\ εἶπον] odnosna zamjenica uvodi zavisnu odnosnu rečenicu; ὅσα: što god\dots
\item[εἶπον] §~254
\item[μέλλοντες] §~231, perifrastični futur §~455. bilješka, otvara mjesto obaveznoj dopuni u infinitivu
\item[πολεμήσειν] §~258, dopuna participa μέλλοντες
\item[ἐν αὐτῷ] sc.\ πολεμεῖν
\item[ὄντες] §~315
\item[τῶν λεχθέντων] §~296
\item[διαμνημονεῦσαι] §~267, složenica μνημονεύω, dopuna imenskom predikatu
\item[ἦν] §~315; s dopunom u infinitivu: LSJ εἰμί A.VI
\item[ἤκουσα] §~267; rekcija τινος
\item[ὧν\dots\ ἤκουσα] odnosna zamjenica uvodi zavisnu odnosnu rečenicu koja u glavnoj ima funkciju objekta (predikat je \textgreek[variant=ancient]{ἦν διαμνημονεῦσαι):} ono što sam\dots
\item[ἀπαγγέλλουσιν] sc.\ ὧν ἤκουσαν; §~231, složenica ἀγγέλλω; rekcija τινι
\item[ἐδόκουν] §~243
\item[ὡς δ' ἂν ἐδόκουν ἐμοὶ ἕκαστοι\dots\ εἰπεῖν\dots\ οὕτως εἴρηται] zavisna poredbena rečenica u inverziji (ὡς u korelaciji s οὕτως); Smyth 2475; modalna čestica ἄν otvara mjesto infinitivu εἰπεῖν, izriče mogućnost
\item[ἐμοὶ] dopuna je ἐχομένῳ
\item[τῶν\dots\ παρόντων] §~315, složenica εἰμί
\item[τὰ δέοντα] §~244, supstantivirani particip §~499: ono što je potrebno, ono što je prikladno
\item[εἰπεῖν] §~254; infinitiv ovisan o ἐδόκουν
\item[ἐχομένῳ] §~231; medijalno LSJ ἔχω C.2 (i 3): držati se što bliže\dots
\item[ὅτι ἐγγύτατα] adverbno, što je bliže moguće; LSJ ἐγγύς IV
\item[τῶν\dots\ λεχθέντων] §~296, supstantivirani particip §~499; genitiv ima funkciju objekta uz \textgreek[variant=ancient]{τῆς ξυμπάσης γνώμης}
\item[εἴρηται] sc.\ ἐμοὶ; §~272

\end{description}

%2
{\large
\begin{greek}
\noindent τὰ δ' ἔργα \\
\tabto{2em} τῶν πραχθέντων \\
\tabto{2em} ἐν τῷ πολέμῳ \\
οὐκ \\
\tabto{2em} ἐκ τοῦ παρατυχόντος \\
πυνθανόμενος \\
ἠξίωσα \\
\tabto{2em} γράφειν, \\
οὐδ' ὡς \\
\tabto{2em} ἐμοὶ \\
\tabto{2em} ἐδόκει, \\
ἀλλ' οἷς τε \\
\tabto{2em} αὐτὸς \\
\tabto{2em} παρῆν \\
καὶ παρὰ τῶν ἄλλων \\
\tabto{4em} ὅσον δυνατὸν \\
\tabto{2em} ἀκριβείᾳ \\
\tabto{2em} περὶ ἑκάστου \\
ἐπεξελθών.\\

\end{greek}
}

\begin{description}[noitemsep]
\item[τῶν πραχθέντων] §~296, supstantivirani particip
\item[οὐκ ] negira πυνθανόμενος
\item[οὐκ ἐκ τοῦ\dots\ οὐδ' ὡς\dots\ ἀλλ' οἷς\dots] koordinacija rečeničnih članova pomoću sastavnih i suprotnih veznika
\item[τοῦ παρατυχόντος] §~254, složenica τυγχάνω
\item[πυνθανόμενος] §~231
\item[ἠξίωσα] §~267, augment §~235, otvara mjesto dopuni u infinitivu
\item[γράφειν] §~231, dopuna predikatu
\item[οὐδ' ὡς ἐμοὶ ἐδόκει] §~243, (negirana) zavisno složena poredbena rečenica, Smyth 2475
\item[οἷς\dots\ παρῆν] odnosna zamjenica οἷς uvodi zavisnu odnosnu rečenicu čiji je antecedent τὰ\dots\ ἔργα
\item[παρῆν] §~315, složenica εἰμί
\item[ἐπεξελθών] §~254, složenica glagola ἔρχομαι, LSJ ἐπεξέρχομαι II.3
\item[ὅσον δυνατὸν] relativna zamjenica ὅσον uvodi zavisnu odnosnu rečenicu, može se prevesti adverbno: koliko je moguće\dots

\end{description}
%3
{\large
\begin{greek}
\noindent ἐπιπόνως δὲ ηὑρίσκετο, \\
διότι οἱ παρόντες \\
\tabto{2em} τοῖς ἔργοις ἑκάστοις \\
οὐ ταὐτὰ \\
\tabto{2em} περὶ τῶν αὐτῶν \\
ἔλεγον, \\
ἀλλ' ὡς ἑκατέρων \\
\tabto{2em} τις \\
\tabto{4em} εὐνοίας ἢ μνήμης \\
\tabto{2em} ἔχοι.\\

\end{greek}
}

\begin{description}[noitemsep]
\item[ηὑρίσκετο] §~231
\item[οἱ παρόντες] §~315, složenica εἰμί, supstantivirani particip §~499
\item[ἔλεγον] §~231
\item[ἑκατέρων] genitiv objektni ovisan o \textgreek[variant=ancient]{εὐνοίας}
\item[ἔχοι] §~231, iterativni optativ §~476.2; rekcija τινος: LSJ ἔχω B.II.2.b
\item[ὡς\dots\ ἔχοι] veznik uvodi zavisnu načinsku rečenicu, Smyth 2475

\end{description}

%4
{\large
\begin{greek}
\noindent καὶ \\
ἐς μὲν ἀκρόασιν \\
ἴσως \\
τὸ μὴ μυθῶδες \\
\tabto{2em} αὐτῶν \\
ἀτερπέστερον φανεῖται·\\
ὅσοι δὲ βουλήσονται \\
\tabto{2em} τῶν τε γενομένων\\
\tabto{4em} τὸ σαφὲς \\
\tabto{4em} σκοπεῖν \\
\tabto{2em} καὶ τῶν μελλόντων \\
\tabto{4em} ποτὲ αὖθις \\
\tabto{4em} κατὰ τὸ ἀνθρώπινον \\
\tabto{4em} τοιούτων καὶ παραπλησίων ἔσεσθαι, \\
\tabto{2em} ὠφέλιμα κρίνειν \\
\tabto{2em} αὐτὰ \\
ἀρκούντως ἕξει.\\

\end{greek}
}

\begin{description}[noitemsep]
\item[ἐς] kod Tukidida uvijek umjesto εἰς
\item[ἐς μὲν ἀκρόασιν\dots\ ὅσοι δὲ βουλήσονται\dots] koordinacija rečeničnih članova pomoću para čestica, izražava kontrast između dva vida zabave i korisnosti 
\item[φανεῖται] §~258; glagol nepotpuna značenja (kopulativni), ima imensku dopunu ἀτερπέστερον
\item[αὐτῶν] odnosi se, kao i malo niže αὐτὰ, na Tukididovo djelo
\item[βουλήσονται] §~258, otvara mjesto dopuni u infinitivu \textgreek[variant=ancient]{σκοπεῖν}; objekt je te dopune \textgreek[variant=ancient]{τὸ σαφὲς}
\item[ὅσοι\dots\ βουλήσονται] odnosna zamjenica uvodi zavisnu odnosnu rečenicu u službi subjekta: tko god\dots
\item[τῶν τε γενομένων\dots\ καὶ τῶν μελλόντων\dots] koordinacija rečeničnih članova pomoću sastavnih veznika
\item[τὸ σαφὲς] Tukidid često umjesto apstraktnih imenica koristi supstantivirani srednji rod pridjeva (ili participa); takva se supstantivirana tvorba često modificira i dopunjava genitivom \textgreek[variant=ancient]{(τῶν τε γενομένων\dots\ καὶ τῶν μελλόντων\dots):} spoznati jasnu istinu o onome što\dots
\item[τῶν\dots\ γενομένων] §~254, §~325.11; supstantivirani particip ovisan o \textgreek[variant=ancient]{τὸ σαφὲς}
\item[σκοπεῖν] §~243
\item[τῶν μελλόντων] genitiv ovisan o τὸ σαφὲς; otvara mjesto dopuni u infinitivu
\item[ποτὲ αὖθις] opet jednom (tj.\ ako se povijest nekad ponovi)
\item[ἔσεσθαι] §~315, dopuna uz \textgreek[variant=ancient]{τῶν μελλόντων;} kao kopulativni glagol, oblik ima imenske dopune \textgreek[variant=ancient]{τοιούτων καὶ παραπλησίων}
\item[κρίνειν] sc.\ τούτους (neizrečeni antecedent od ὅσοι δὲ βουλήσονται); §~231; dopuna uz \textgreek[variant=ancient]{ἕξει}; uz glagol stoje dva akuzativa, \textgreek[variant=ancient]{ὠφέλιμα\dots\ αὐτὰ}, ``procijeniti da je što kakvo''
\item[ἕξει] §~258; ovdje kao dopunu ima infintiv
\item[ἀρκούντως ἕξει] ἔχω s prilozima u perifrastičnoj konstrukciji preuzima značenje glagola ``biti'': bit će dovoljno da\dots

\end{description}

%5
{\large
\begin{greek}
\noindent κτῆμά τε \\
\tabto{2em} ἐς αἰεὶ \\
μᾶλλον ἢ ἀγώνισμα \\
\tabto{2em} ἐς τὸ παραχρῆμα ἀκούειν \\
ξύγκειται.\\

\end{greek}
}

\begin{description}[noitemsep]
\item[τε] ``i tako''
\item[ἐς] specifičnost Tukididova izričaja, uvijek umjesto εἰς
\item[αἰεὶ] specifičnost Tukididova izričaja, uvijek umjesto ἀεί
\item[μᾶλλον ἢ] suprotstavlja dva para ideja \textgreek[variant=ancient]{(κτῆμά – ἀγώνισμα, ἐς αἰεὶ – ἐς τὸ παραχρῆμα ἀκούειν)}
\item[ἀκούειν] §~231; supstantivirani infinitiv §~497
\item[ξύγκειται] §~315.a; složenica κεῖμαι; Tukidid uvijek koristi prijedlog ξύν umjesto σύν, pa tako i u složenicama glagola

\end{description}



%kraj
