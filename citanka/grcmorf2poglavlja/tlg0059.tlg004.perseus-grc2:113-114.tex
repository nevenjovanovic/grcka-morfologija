% Unio ispravke NZ <2022-01-03 pon>



\section*{O tekstu}

Dijalog \textit{Fedon} pripada srednjem Platonovu stvaralačkom razdoblju, nastao je oko 385. To je četvrti i posljednji od dijaloga u kojima Platon opisuje posljednje dane Sokratova života. Pripovjedač je filozof Fedon iz Elide, Sokratov učenik; u peloponeskom gradu Flijuntu on skupini slušalaca opisuje Sokratovu smrt, koja se zbila prije nekoliko mjeseci (u proljeće 399).

Na vrhuncu dijaloga, neposredno pred smaknuće, Sokrat predlaže Simiji da mu ``ispriča lijepu i slušanja vrijednu priču'' \textgreek[variant=ancient]{(μῦθον λέγειν καλόν, ἄξιον ἀκοῦσαι)} i potom opisuje čitav svijet, uključujući i svijet mrtvih. U odlomku koji slijedi Sokrat pripovijeda kako se mrtve duše raspoređuju po odgovarajućim dijelovima podzemnog svijeta. Najbolje lokacije za život poslije smrti rezervirane su, dakako, za filozofe.


%\newpage

\section*{Pročitajte naglas grčki tekst.}

Plat.\ Phaedo 113d-114c

%Naslov prema izdanju

\medskip


{\large

\begin{greek}

\noindent ἐπειδὰν ἀφίκωνται οἱ τετελευτηκότες εἰς τὸν τόπον οἷ ὁ δαίμων ἕκαστον κομίζει, πρῶτον μὲν διεδικάσαντο οἵ τε καλῶς καὶ ὁσίως βιώσαντες καὶ οἱ μή. καὶ οἳ μὲν ἂν δόξωσι μέσως βεβιωκέναι, πορευθέντες ἐπὶ τὸν Ἀχέροντα, ἀναβάντες ἃ δὴ αὐτοῖς ὀχήματά ἐστιν, ἐπὶ τούτων ἀφικνοῦνται εἰς τὴν λίμνην, καὶ ἐκεῖ οἰκοῦσί τε καὶ καθαιρόμενοι τῶν τε ἀδικημάτων διδόντες δίκας ἀπολύονται, εἴ τίς τι ἠδίκηκεν, τῶν τε εὐεργεσιῶν τιμὰς φέρονται κατὰ τὴν ἀξίαν ἕκαστος· οἳ δ ἂν δόξωσιν ἀνιάτως ἔχειν διὰ τὰ μεγέθη τῶν ἁμαρτημάτων, ἢ ἱεροσυλίας πολλὰς καὶ μεγάλας ἢ φόνους ἀδίκους καὶ παρανόμους πολλοὺς ἐξειργασμένοι ἢ ἄλλα ὅσα τοιαῦτα τυγχάνει ὄντα, τούτους δὲ ἡ προσήκουσα μοῖρα ῥίπτει εἰς τὸν Τάρταρον, ὅθεν οὔποτε ἐκβαίνουσιν.

\noindent (\dots) οἳ δὲ δὴ ἂν δόξωσι διαφερόντως πρὸς τὸ ὁσίως βιῶναι, οὗτοί εἰσιν οἱ τῶνδε μὲν τῶν τόπων τῶν ἐν τῇ γῇ ἐλευθερούμενοί τε καὶ ἀπαλλαττόμενοι ὥσπερ δεσμωτηρίων, ἄνω δὲ εἰς τὴν καθαρὰν οἴκησιν ἀφικνούμενοι καὶ ἐπὶ γῆς οἰκιζόμενοι. τούτων δὲ αὐτῶν οἱ φιλοσοφίᾳ ἱκανῶς καθηράμενοι ἄνευ τε σωμάτων ζῶσι τὸ παράπαν εἰς τὸν ἔπειτα χρόνον, καὶ εἰς οἰκήσεις ἔτι τούτων καλλίους ἀφικνοῦνται\dots

\end{greek}

}


\section*{Analiza i komentar}

%1

{\large
\begin{greek}
\noindent ἐπειδὰν ἀφίκωνται οἱ τετελευτηκότες \\
\tabto{2em} εἰς τὸν τόπον \\
\tabto{4em} οἷ ὁ δαίμων ἕκαστον κομίζει,\\
πρῶτον μὲν διεδικάσαντο \\
\tabto{2em} οἵ τε καλῶς καὶ ὁσίως βιώσαντες\\
\tabto{2em} καὶ οἱ μή. \\

\end{greek}
}

\begin{description}[noitemsep]
\item[ἐπειδὰν] uvodi zavisnu vremensku rečenicu s načinima eventualne protaze §~488.2
\item[ἀφίκωνται] §~254, §~321.8
\item[οἱ τετελευτηκότες] §~272, supstantivirani particip, LSJ τελευτάω A.3
\item[οἷ] relativni prilog mjesta uvodi zavisnu relativnu rečenicu §~481, antecedent \textgreek[variant=ancient]{τὸν τόπον}
\item[κομίζει] §~231, LSJ κομίζω A.II.6
\item[πρῶτον μὲν] μὲν naglašava prethodnu riječ (Smyth 2897)
\item[διεδικάσαντο] složenica δικάζω, §~267, LSJ 3, gnomski aorist §~454, bilješka 2
\item[οἵ τε\dots\ καὶ οἱ\dots] koordinacija dvaju rečeničnih članova (u ovom slučaju supstantiviranih participa) sastavnim veznicima
\item[οἵ\dots\ βιώσαντες] supstantivirani particip osnove slabog aorista (v.\ LSJ βιόω; usp.\ §~327.9)
\item[οἱ μή] sc.\ \textgreek[variant=ancient]{καλῶς καὶ ὁσίως βιώσαντες; μή} kao negacija uz particip §~509.2

\end{description}

%2
{\large
\begin{greek}
\noindent καὶ οἳ μὲν ἂν δόξωσι \\
\tabto{2em} μέσως βεβιωκέναι, \\
πορευθέντες \\
\tabto{2em} ἐπὶ τὸν ᾿Αχέροντα, \\
ἀναβάντες \\
\tabto{2em} ἃ δὴ αὐτοῖς ὀχήματά ἐστιν, \\
ἐπὶ τούτων \\
ἀφικνοῦνται \\
\tabto{2em} εἰς τὴν λίμνην, \\
καὶ ἐκεῖ οἰκοῦσί τε \\
καὶ καθαιρόμενοι \\
\tabto{2em} τῶν τε ἀδικημάτων \\
\tabto{2em} διδόντες δίκας \\
\tabto{2em} ἀπολύονται, \\
\tabto{4em} εἴ τίς τι ἠδίκηκεν, \\
\tabto{2em} τῶν τε εὐεργεσιῶν \\
\tabto{2em} τιμὰς \\
\tabto{2em} φέρονται \\
\tabto{4em} κατὰ τὴν ἀξίαν \\
\tabto{2em} ἕκαστος·

\noindent οἳ δ' ἂν δόξωσιν \\
\tabto{2em} ἀνιάτως ἔχειν \\
\tabto{2em} διὰ τὰ μεγέθη \\
\tabto{4em} τῶν ἁμαρτημάτων, \\
\tabto{2em} ἢ ἱεροσυλίας πολλὰς καὶ μεγάλας \\
\tabto{2em} ἢ φόνους ἀδίκους καὶ παρανόμους πολλοὺς \\
\tabto{2em} ἐξειργασμένοι \\
\tabto{2em} ἢ ἄλλα \\
\tabto{4em} ὅσα τοιαῦτα τυγχάνει ὄντα,\\
τούτους δὲ \\
ἡ προσήκουσα μοῖρα \\
ῥίπτει \\
\tabto{2em} εἰς τὸν Τάρταρον,\\
\tabto{4em} ὅθεν οὔποτε ἐκβαίνουσιν. \\

\end{greek}
}

\begin{description}[noitemsep]
\item[οἳ μὲν\dots\ οἳ δ'\dots] koordinacija surečenica parom čestica koje izražavaju kontrast
\item[ἂν] §~489.b.4
\item[δόξωσι] §~325.2, §~267; \textit{verbum sentiendi} otvara mjesto dopuni u infinitivu
\item[βεβιωκέναι] §~327.9
\item[πορευθέντες] §~296
\item[τὸν ᾿Αχέροντα] rijeka u svijetu mrtvih
\item[ἀναβάντες] složenica βαίνω §~321.6, §~316
\item[ἃ] uvodi zavisnu relativnu rečenicu §~481; antecedent (ὀχήματά) ovdje je uključen u zavisnu rečenicu, te se u padežu slaže s relativom, a u glavnoj rečenici umjesto njega stoji pokazna zamjenica (τούτων); Smyth 2536, 2538
\item[δὴ] čestica koja služi pojašnjavanju: već\dots
\item[αὐτοῖς ἐστιν] §~315; imenski predikat; ἐστί τινι LSJ εἰμί C.III
\item[ἀφικνοῦνται] §~321.8, §~243
\item[οἰκοῦσί τε\dots] \textbf{καὶ\dots\ ἀπολύονται\dots\ φέρονται} kombinacija sastavnih veznika τε\dots\ καὶ\dots\ ističe drugi član kombinacije
\item[οἰκοῦσί] §~231, §~243
\item[καθαιρόμενοι] §~232
\item[τῶν τε ἀδικημάτων\dots] \textbf{τῶν τε εὐεργεσιῶν\dots}\ koordinacija rečeničnih članova pomoću para sastavnih veznika
\item[τῶν\dots\ ἀδικημάτων] objekt fraze διδόντες δίκας
\item[διδόντες] §~305; fraza δίκας διδόναι τινός, LSJ δίκη IV.3
\item[ἀπολύονται] složenica λύω, §~232
\item[εἴ] veznik uvodi realnu pogodbenu rečenicu, §~475
\item[ἠδίκηκεν] §~272
\item[τῶν\dots\ εὐεργεσιῶν] genitiv objektni §~394, ovisan o τιμὰς
\item[φέρονται] §~232, §~327.5, LSJ φέρω A.VI.3
\item[ἕκαστος] slaže se s predikatom u množini, Smyth 951
\item[ἂν] §~489.b.4
\item[δόξωσιν] §~325.2, §~267; \textit{verbum sentiendi} otvara mjesto dopuni u infinitivu
\item[ἔχειν] §~231, §~327.13
\item[ἀνιάτως ἔχειν] fraza: biti neizlječiv
\item[ἢ ἱεροσυλίας\dots] \textbf{ἢ φόνους\dots\ ἢ ἄλλα\dots}\ koordinacija pomoću para veznika
\item[ἐξειργασμένοι] složenica ἐργάζομαι, §~272, reduplikacija §~275
\item[τυγχάνει] §~321.19, §~231; otvara mjesto predikatnom participu §~501.b
\item[τοιαῦτα ὄντα] §~315, predikatni particip (s imenskom dopunom) §~501.b
\item[δὲ] označava nadovezivanje na prethodni iskaz (pritom je τούτους antecedent relativa οἳ)
\item[προσήκουσα] složenica ἥκω, §~231, LSJ προσήκω III.2
\item[ῥίπτει] §~231
\item[τὸν Τάρταρον] Tartar, najdublji dio podzemnog svijeta
\item[ὅθεν] relativni prilog uvodi zavisnu relativnu rečenicu §~481
\item[ἐκβαίνουσιν] složenica βαίνω §~321.6, §~231

\end{description}
%3
{\large
\begin{greek}
\noindent οἳ δὲ δὴ ἂν δόξωσι\\
\tabto{2em} διαφερόντως \\
\tabto{2em} πρὸς τὸ ὁσίως \\
βιῶναι, \\
οὗτοί εἰσιν \\
\tabto{2em} οἱ τῶνδε μὲν τῶν τόπων \\
\tabto{4em} τῶν ἐν τῇ γῇ \\
\tabto{2em} ἐλευθερούμενοί τε καὶ ἀπαλλαττόμενοι \\
\tabto{4em} ὥσπερ δεσμωτηρίων, \\
\tabto{2em} ἄνω δὲ \\
\tabto{4em} εἰς τὴν καθαρὰν οἴκησιν\\
\tabto{2em} ἀφικνούμενοι \\
\tabto{2em} καὶ \\
\tabto{4em} ἐπὶ γῆς \\
\tabto{2em} οἰκιζόμενοι. \\

\end{greek}
}

\begin{description}[noitemsep]
\item[δὲ δὴ] kombinacija čestica izriče prelazak na novu informaciju, Smyth 2839
\item[ἂν] §~489.b.4
\item[δόξωσι] §~325.2, §~267; \textit{verbum sentiendi} otvara mjesto dopuni u infinitivu
\item[πρὸς τὸ ὁσίως] LSJ πρός C.III
\item[βιῶναι] §~327.9, §~316.c; infinitiv ovisan o δόξωσι
\item[εἰσιν οἱ\dots] \textbf{ἐλευθερούμενοί τε καὶ ἀπαλλαττόμενοι} §~315; imenski predikat
\item[τῶνδε\dots] \textbf{τῶν τόπων τῶν ἐν τῇ γῇ} \textit{genetivus separationis} §~402; u rečenici mu mjesto otvaraju participi \textgreek[variant=ancient]{ἐλευθερούμενοί τε καὶ ἀπαλλαττόμενοι}
\item[οἱ τῶνδε μὲν\dots\ ἄνω δὲ\dots] koordinacija rečeničnih članova parom čestica koji izražava kontrast
\item[οἱ ἐλευθερούμενοί] rekcija τινος; §~232; supstantivirani particip
\item[ἐλευθερούμενοί τε καὶ ἀπαλλαττόμενοι] koordinacija rečeničnih članova parom sastavnih veznika, pri čemu je drugi član istaknutiji
\item[ἀπαλλαττόμενοι] složenica ἀλλάσσω, atički ἀλλάττω; rekcija τινος; §~232
\item[ἀφικνούμενοι] §~232
\item[οἰκιζόμενοι] §~232

\end{description}
%4
{\large
\begin{greek}
\noindent τούτων δὲ αὐτῶν \\
οἱ \\
\tabto{2em} φιλοσοφίᾳ \\
\tabto{2em} ἱκανῶς \\
καθηράμενοι \\
\tabto{2em} ἄνευ τε σωμάτων \\
ζῶσι \\
\tabto{2em} τὸ παράπαν \\
\tabto{2em} εἰς τὸν ἔπειτα χρόνον, \\
καὶ \\
\tabto{2em} εἰς οἰκήσεις \\
\tabto{2em} ἔτι τούτων καλλίους \\
ἀφικνοῦνται.\\

\end{greek}
}

\begin{description}[noitemsep]
\item[δὲ] čestica označava nadovezivanje na prethodni iskaz
\item[τούτων αὐτῶν] genitiv partitivni §~395
\item[οἱ καθηράμενοι] s.~118, §~267, supstantivirani particip
\item[ἄνευ τε σωμάτων\dots] \textbf{καὶ εἰς οἰκήσεις} koordinacija rečeničnih članova pomoću para sastavnih veznika, pri čemu je drugi član istaknutiji
\item[ζῶσι] §~327.9, §~243, §~244.2
\item[τούτων] \textit{genetivus comparationis} §~404.1  
\item[ἀφικνοῦνται] §~321.8, §~243

\end{description}


%kraj
