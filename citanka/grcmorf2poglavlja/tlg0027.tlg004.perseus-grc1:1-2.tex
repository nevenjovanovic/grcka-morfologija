\section*{O autoru}

Andokid (Ἀνδοκίδης, oko 440.\ – nakon 391.\ pr.~Kr.), rodom iz ugledne i bogate atenske obitelji, pripadao je „zlatnoj mladeži” Atene; mnogi od njegovih drugova javljaju se kao protagonisti u Platonovim dijalozima. Kao pripadnik aristokratskih, prema demokraciji neprijateljskih društava bio je tijekom Peloponeskog rata 415, zajedno s Alkibijadom, upleten u skandal oskvrnuća hermi uoči Sicilske ekspedicije; bogohulni je čin teško uvrijedio običan atenski narod, osobito zato što se sumnjalo da se Alkibijad, zajedno s prijateljima, u svojoj kući izrugivao i Eleuzinskim misterijama.

Andokid je završio u zatvoru (Alkibijad, koji je otplovio za Siciliju, bio je osuđen u odsutnosti) te je na savjet rođaka Harmida – onog po kojem se zove jedan Platonov dijalog – prokazao četiri člana svojeg aristokratskog udruženja. Potom je osuđen na progonstvo na Cipar. Vratio se u Atenu 407.\ i pokušao govorom ishoditi oprost, ali neuspješno; pomilovan je tek u okviru opće amnestije 403. Ponovo se počeo baviti politikom i opet je optužen za bezbožnost; branio se govorom \textit{O misterijama} (399). Bio je član poslanstva koje je 391.\ pregovaralo sa Spartom o miru, no, zbog držanja tijekom poslanstva optužen je za veleizdaju (branio se govorom \textit{O miru sa Spartancima}) i ponovno osuđen na progonstvo. Umro je izvan Atene, ne zna se kada.

Andokidovi govori dragocjen su povijesni izvor, a pokazuju i dijalektičku sposobnost i dobro poznavanje atenskoga sudskog stila. Andokid, koji nije bio profesionalni govornik, riječi nalazi u životu, ne u školi.

\section*{O tekstu}

Govor Κατὰ Ἀλκιβιάδου (\textit{Protiv Alkibijada}) već se u antičko doba nije smatrao Andokidovim djelom. Prema navodima u tekstu i onome što o povijesti znamo, govor bi bio održan 417, kada je ὀστρακοφορία (glasovanje ostrakama) trebala odlučiti hoće li prognan biti Alkibijad ili Nikija; demagog Hiperbol pokušao se riješiti barem jednog od dvojice utjecajnih oponenata, ali Alkibijad se nagodio s Nikijom, tako da je ostrakizmom prognan sam Hiperbol. Andokid je tada imao oko dvadeset godina, te se teško može s njim povezati spominjanje uspješne političke karijere koja je uključivala šest diplomatskih poslanstava u zapadnoj Grčkoj i na Siciliji, a o takvoj karijeri on ne govori u autentičnim djelima. Postoje i pravno-proceduralni problemi; budući da ostrakizam nije bio sudski postupak, nije uključivao govore optužbe i obrane.

Situacija, međutim, nije posve izmišljena. Plutarh, koji o ostrakizmu 417.\ pripovijeda na više mjesta, daje naslutiti da je upleten bio i vođa treće stranke, Feak (Φαίαξ). Zato je bilo pokušaja da se govor pripiše Feaku; ovo, pak, otežava spominjanje osvajanja Mela (koje se dogodilo tek 416). 

Najvjerojatnije je, stoga, da se radi o književnoj vježbi, možda „iz uloge” Feaka. Vježba mora da je nastala kad detalji procedure ostrakizma više nisu bili posve jasni, vjerojatno u ranom IV.~st.

U prvom dijelu govora govornik dokazuje svoje zasluge i nedužnost, ističe da, mada je četiri puta bio tužen zbog političkih prijestupa, nikad nije osuđen. U drugom se dijelu napada Alkibijadovo javno i privatno djelovanje. Uspoređuju se obitelji govornika i Alkibijada (oba su Alkibijadova djeda bila dvaput ostrakizmom prognana), ističe se spremnost govornika da i na sudu odgovara za svoje postupke (što Alkibijad nikad nije htio učiniti), odbija se prigovor da su govornika oslobodili zbog nesposobnosti tužitelja. Naposljetku govornik upozorava da će Alkibijad pokušati pridobiti suosjećanje publike, ali u javnom je interesu da se progna njega, a ne govornika, koji je u prošlosti mnogim djelima zadužio Atenu.

Odabrani odlomak stoji na samom početku govora i razmatra načelne opasnosti bavljenja politikom, kojih je govornik itekako svjestan. Opasnostima se izložio u interesu općeg dobra, uz podršku sebi sklonih, čak i po cijenu sukoba s moćnim neprijateljima. Formulira glavni problem govora: koga od trojice treba prognati na deset godina.

%\newpage

\section*{Pročitajte naglas grčki tekst.}

And.\ In Alcibiadem [Sp.] 1-2

%Naslov prema izdanju

\medskip


{\large

\begin{greek}

\noindent οὐκ ἐν τῷ παρόντι μόνον γιγνώσκω τῶν πολιτικῶν πραγμάτων ὡς σφαλερόν ἐστιν ἅπτεσθαι, ἀλλὰ καὶ πρότερον χαλεπὸν ἡγούμην, πρὶν τῶν κοινῶν ἐπιμελεῖσθαί τινος. πολίτου δὲ ἀγαθοῦ νομίζω προκινδυνεύειν ἐθέλειν τοῦ πλήθους, καὶ μὴ καταδείσαντα τὰς ἔχθρας τὰς ἰδίας ὑπὲρ τῶν δημοσίων ἔχειν ἡσυχίαν· διὰ μὲν γὰρ τοὺς τῶν ἰδίων ἐπιμελουμένους οὐδὲν αἱ πόλεις μείζους καθίστανται, διὰ δὲ τοὺς τῶν κοινῶν μεγάλαι καὶ ἐλεύθεραι γίγνονται.

ὧν [τῶν ἀγαθῶν] εἷς ἐγὼ βουληθεὶς ἐξετάζεσθαι μεγίστοις περιπέπτωκα κινδύνοις, προθύμων μὲν καὶ ἀγαθῶν ἀνδρῶν ὑμῶν τυγχάνων, δι' ὅπερ σῴζομαι, πλείστοις δὲ καὶ δεινοτάτοις ἐχθροῖς χρώμενος, ὑφ' ὧν διαβάλλομαι. ὁ μὲν οὖν ἀγὼν ὁ παρὼν οὐ στεφανηφόρος, ἀλλ' εἰ χρὴ μηδὲν ἀδικήσαντα τὴν πόλιν δέκα ἔτη φεύγειν· οἱ δ' ἀνταγωνιζόμενοι περὶ τῶν ἄθλων τούτων ἐσμὲν ἐγὼ καὶ Ἀλκιβιάδης καὶ Νικίας, ὧν ἀναγκαῖον ἕνα τῇ συμφορᾷ περιπεσεῖν.

\end{greek}

}


\section*{Analiza i komentar}

%1

{\large
\begin{greek}
\noindent οὐκ \\
\tabto{2em} ἐν τῷ παρόντι \\
μόνον \\
γιγνώσκω \\
\tabto{6em} τῶν πολιτικῶν πραγμάτων \\
\tabto{2em} ὡς σφαλερόν ἐστιν \\
\tabto{4em} ἅπτεσθαι, \\
ἀλλὰ καὶ \\
\tabto{2em} πρότερον \\
χαλεπὸν ἡγούμην, \\
\tabto{2em} πρὶν \\
\tabto{6em} τῶν κοινῶν \\
\tabto{2em} ἐπιμελεῖσθαί \\
\tabto{4em} τινος.\\

\end{greek}
}

\begin{description}[noitemsep]
\item[οὐκ\dots\ μόνον\dots\ ἀλλὰ καὶ\dots] koordinacija rečeničnih članova: „ne samo\dots\ nego i\dots”
\item[ἐν τῷ παρόντι] LSJ / Logeion πάρειμι II.
\item[γιγνώσκω] §~231
\item[ὡς σφαλερόν ἐστιν] §~315; pridjevska dopuna uz kopulu Smyth 4.26 910; \textit{verbum sentiendi} otvara mjesto zavisnoj izričnoj rečenici koju uvodi veznik ὡς, §~467
\item[ἅπτεσθαι] §~232; ἅπτεσθαί τινος % s genitivom
\item[χαλεπὸν ἡγούμην] §~235, §~243; ἡγέομαι otvara mjesto dvama akuzativima (objekta i predikata), §~388; ovdje bi objektni akuzativ bio ἅπτεσθαι % imenska dopuna
\item[πρὶν\dots\ ἐπιμελεῖσθαί] §~243; ἐπιμελέομαί τινος; πρίν otvara mjesto zavisnoj vremenskoj rečenici s infinitivom, §~488.1 %s infinitivom; s genitivom
\item[τῶν κοινῶν\dots\ τινος] potonji je genitiv objekt uz ἐπιμελέομαι, prvi je partitivni, §~395
\end{description}

%2

{\large
\begin{greek}
\noindent πολίτου δὲ ἀγαθοῦ \\
νομίζω \\
προκινδυνεύειν ἐθέλειν \\
\tabto{2em} τοῦ πλήθους, \\
καὶ μὴ \\
καταδείσαντα \\
\tabto{2em} τὰς ἔχθρας \\
\tabto{4em} τὰς ἰδίας \\
\tabto{2em} ὑπὲρ τῶν δημοσίων \\
ἔχειν ἡσυχίαν· \\
διὰ μὲν γὰρ τοὺς \\
\tabto{2em} τῶν ἰδίων \\
ἐπιμελουμένους \\
οὐδὲν \\
αἱ πόλεις \\
μείζους καθίστανται, \\
διὰ δὲ τοὺς \\
\tabto{2em} τῶν κοινῶν \\
μεγάλαι καὶ ἐλεύθεραι γίγνονται.\\

\end{greek}
}

\begin{description}[noitemsep]
\item[δὲ] čestica δέ označava nadovezivanje na prethodni iskaz
\item[πολίτου\dots\ ἀγαθοῦ νομίζω\dots] \textbf{ἐθέλειν\dots\ καὶ μὴ καταδείσαντα\dots\ ἔχειν ἡσυχίαν} §~231; νομίζω kao kopulativni glagol i \textit{verbum sentiendi} otvara mjesto ne samo objektima (ovdje infinitiv i akuzativ s infinitivom), nego i genitivu koji izriče \textit{kome} je nešto svojstveno, §~393.2, Smyth  4.42.93.83, 1304 1305
\item[προκινδυνεύειν] §~231; προκινδυνεύω τινός
\item[ἐθέλειν] §~231; glagol ἐθέλω otvara mjesto infinitivu % s infinitivom
\item[καταδείσαντα] §~267; složenica glagola δείδω; dio akuzativa s infinitivom, §~491 % ak s inf
\item[τὰς ἔχθρας τὰς ἰδίας] jače istaknut atributni položaj, §~375 % položaj člana
\item[ὑπὲρ τῶν δημοσίων] ὑπέρ ovdje: „što se tiče”, LSJ ὑπέρ III
\item[τῶν δημοσίων] supstantiviranje, §~373; DGE / Logeion δημόσιος IV
\item[ἔχειν] §~231
\item[διὰ μὲν γὰρ τοὺς\dots] \textbf{διὰ δὲ τοὺς\dots} koordinacija rečeničnih članova (supstantiviranih participa) pomoću čestica μέν\dots\ δέ\dots%koordinacija
\item[τοὺς\dots\ ἐπιμελουμένους] §~243; ἐπιμελέομαί τινος; §~373 % s genitivom
\item[μείζους] stegnuti oblik, Smyth 293 d
\item[καθίστανται] §~305, složenica glagola ἵστημι, u mediju znači „postati” i ima nužnu (ovdje atributnu) dopunu % dopuna
\item[τῶν κοινῶν] sc. ἐπιμελουμένους
\item[γίγνονται] §~232, kao kopulativni glagol ima nužnu (ovdje atributnu) dopunu % s imenskom dopunom
\end{description}


%3

{\large
\begin{greek}
\noindent ὧν [τῶν ἀγαθῶν] \\
εἷς ἐγὼ \\
\tabto{2em} βουληθεὶς ἐξετάζεσθαι \\
\tabto{2em} μεγίστοις \\
περιπέπτωκα \\
\tabto{2em} κινδύνοις, \\
προθύμων μὲν \\
\tabto{2em} καὶ ἀγαθῶν \\
ἀνδρῶν \\
\tabto{2em} ὑμῶν \\
τυγχάνων, \\
\tabto{2em} δι' ὅπερ σῴζομαι, \\
πλείστοις δὲ \\
\tabto{2em} καὶ δεινοτάτοις ἐχθροῖς \\
χρώμενος, \\
\tabto{2em} ὑφ' ὧν διαβάλλομαι.\\

\end{greek}
}

\begin{description}[noitemsep]
\item[ὧν] \textbf{ [τῶν ἀγαθῶν]} genitiv partitivni, §~395; uglatim zagradama označene su riječi koje postoje u rukopisnoj predaji, ali, po mišljenju priređivača, nisu stajale u izvorniku % uglate zagrade, genitivi
\item[βουληθεὶς] §~296; §~325.13, §~328.2; glagol otvara mjesto infinitivu % otvara mjesto infinitivu
\item[ἐξετάζεσθαι] §~232
\item[περιπέπτωκα] §~272; §~275b; složenica glagola πίπτω, §~327.17; περιπίπτω τινί % otvara mjesto dativu
\item[τυγχάνων] §~231; τυγχάνω τινός % otvara mjesto genitivu
\item[δι' ὅπερ] paralela s ὑφ' ὧν
\item[σῴζομαι] §~232 % rekcija swzomai
\item[χρώμενος] §~243; χρῆσθαί τινι; LSJ χράομαι IV.b % otvara mjesto dativu
\item[διαβάλλομαι] §~232 % pasiv s upo
\end{description}

%4
{\large
\begin{greek}
\noindent ὁ μὲν οὖν ἀγὼν \\
\tabto{2em} ὁ παρὼν \\
\tabto{2em} οὐ στεφανηφόρος, \\
ἀλλ' \\
\tabto{2em} εἰ χρὴ \\
\tabto{2em} μηδὲν ἀδικήσαντα \\
\tabto{4em} τὴν πόλιν \\
\tabto{2em} δέκα ἔτη φεύγειν· \\
οἱ δ' ἀνταγωνιζόμενοι \\
\tabto{2em} περὶ τῶν ἄθλων τούτων \\
ἐσμὲν ἐγὼ καὶ Ἀλκιβιάδης καὶ Νικίας, \\
\tabto{2em} ὧν ἀναγκαῖον \\
\tabto{2em} ἕνα \\
\tabto{4em} τῇ συμφορᾷ\\
\tabto{2em} περιπεσεῖν.\\

\end{greek}
}

\begin{description}[noitemsep]
\item[ὁ μὲν οὖν ἀγὼν\dots] \textbf{οἱ δ' ἀνταγωνιζόμενοι\dots} koordinacija rečeničnih članova (supstantiviranih participa) pomoću čestica μέν\dots\ δέ\dots
\item[ὁ\dots\ ἀγὼν ὁ παρὼν] jače istaknut atributni položaj, §~375; LSJ πάρειμι II.% položaj participa, značenje "sadašnji"
\item[οὐ στεφανηφόρος] sc. ἐστί
\item[εἰ] veznik εἰ uvodi zavisnu upitnu rečenicu, §~469
\item[χρὴ] χρή je bezličan glagol; otvara mjesto nužnoj dopuni, ovdje akuzativu s infinitivom % upitni veznik, otvara mjesto ak s inf
\item[ἀδικήσαντα] §~267, §~269; αδικέω τινά §~381.1 % s dva akuzativa
\item[φεύγειν] §~231
\item[οἱ\dots\ ἀνταγωνιζόμενοι] §~232; ἀνταγωνίζομαι περί τινος; supstantiviranje, §~373 % supstantivirani particip, rekcija peri
\item[ἐσμὲν] §~315; kopulativni glagol otvara mjesto nužnoj predikatnoj dopuni (imenski predikat, Smyth 910) % kopulativni plus ant.?
%\item[ὧν] % partitivni gen
\item[ἀναγκαῖον] sc. ἐστί; otvara mjesto nužnoj dopuni, ovdje akuzativu s infinitivom % sc. esti; otvara mjesto ak s inf
\item[περιπεσεῖν] §~254; složenica glagola πίπτω; περιπίπτω τινί, LSJ περιπίπτω II.3 % s dativom

\end{description}

%kraj
