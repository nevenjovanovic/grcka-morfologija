\section*{O autoru}

Ahilej Tacije (ili Tatije), Ἀχιλλεὺς Τάτιος, iz Aleksandrije, iz II. stoljeća, autor grčkoga ljubavnog romana (uz Longa, Heliodora iz Emese, Haritona i Ksenofonta Efeškog). Jedino su njegovo sačuvano djelo \textit{Zgode Leukipe i Klitofonta} (Τὰ κατὰ Λευκίππην καὶ Κλειτοφῶντα).

Autor romana u prvom licu isprepleće priču o ljubavi suočenoj s mnoštvom neprilika: brodoloma, otmica, susreta s piratima i razbojnicima. U tradiciji grčkoga ljubavnog romana, ljubav na kraju pobjeđuje i sve sretno završava.

Jezično, roman je u duhu aticizma (oponašanja atičke proze klasičnoga razdoblja), s pomno probranim leksikom. Rečenice su kratke, a česte su deklamacije i rasprave antitetičke strukture.

O popularnosti romana svjedoče egipatski papirusni fragmenti (najstariji nalaz je iz II.~st.). Djelo je poslužilo i kao glavni izvor bizantskom autoru Eustatiju Makrembolitu (XII. st) koji je prema \textit{Klitofontu i Leukipi} sastavio roman \textit{Priča o Hizmni i Hizminiji}.


\section*{O tekstu}

Neimenovani pripovjedač u prvom poglavlju romana opisuje sliku otmice Europe, izloženu kao zavjetni dar u hramu božice Ištar. Pripovjedač ondje susreće mladića Klitofonta, koji mu priča o svojim pustolovinama, kao potvrdi moći prave ljubavi. Klitofont je bio zaručen za polusestru Kaligonu, ali se na prvi pogled zaljubio u svoju rođakinju Leukipu. Ona mu je uzvratila ljubav, a Klitofont se oslobodio obaveze kad je Kaligonu oteo Kalisten, mladić iz Bizantija. No, Kalisten je u stvari htio oteti Leukipu, o čijoj se ljepoti nadaleko pričalo.

Klitofont i Leukipa odlučili su brodom pobjeći iz Tira, ali su u oluji doživjeli brodolom. Završili su u Egiptu, gdje su ih zarobili razbojnici s Nila. Klitofont se spasio, ali Leukipu su namjeravali žrtvovati. Misao da je Leukipa mrtva potakla je Klitofonta da se ubije na njezinu grobu. No Leukipa je bila živa – njezino žrtvovanje bilo je varka Klitofontovih zarobljenih prijatelja.

Uskoro je egipatska vojska spasila družinu, ali se u Leukipu zaljubio egipatski general. Ona je popila ljubavni napitak namijenjen drugome, te je skoro poludjela, ali ju je spasio protuotrov koji joj je dao stranac Hereja. Kad su već mislili da su spašeni i da se vraćaju u sigurnost Aleksandrije, Leukipu je oteo Hereja.

Uvjeren da je Leukipa mrtva, Klitofont se vratio u Aleksandriju. U njega se zaljubila udovica Melita. Otputovali su Efez, odakle je bila Melita, a Klitofont je ondje otkrio da je Leukipa živa. No živ je bio i Melitin muž. On je pokušao silovati Leukipu te se riješiti Klitofonta, optuživši ga za preljub i ubojstvo. Klitofont je dokazao nevinost te se konačno oženio Leukipom u njezinu Bizantiju. 

U odlomku iz prve knjige romana čitamo dio razgovora Klitofonta i njegova kućnog roba Satira u prelijepom vrtu Klitofontove kuće. Leukipa šeće vrtom sa svojom robinjom, a Klitofont se nada da će ljubavna tema o kojoj govori i u Leukipi pobuditi ljubavne osjećaje.


%\newpage

\section*{Pročitajte naglas grčki tekst.}

Ach.~Tat.\ Leucippe et Clitophon 1.17

%Naslov prema izdanju

\medskip


{\large

\begin{greek}

\noindent ἦ γὰρ ὁ ἔρως, ἔφη, τοσαύτην ἔχει τὴν ἰσχύν, ὡς καὶ μέχρις ὀρνίθων πέμπειν τὸ πῦρ; οὐ μέχρις ὀρνίθων, ἔφην, τοῦτο γὰρ οὐ θαυμαστόν, ἐπεὶ καὶ αὐτὸς ἔχει πτερόν, ἀλλὰ καὶ ἑρπετῶν καὶ φυτῶν, ἐγὼ δὲ δοκῶ μοι, καὶ λίθων. Ἐρᾷ γοῦν ἡ Μαγνησία λίθος τοῦ σιδήρου· κἂν μόνον ἴδῃ καὶ θίγῃ, πρὸς αὑτὴν εἵλκυσεν, ὥσπερ ἐρωτικόν τι ἔνδον ἔχουσα. Καὶ μή τι τοῦτό ἐστιν ἐρώσης λίθου καὶ ἐρωμένου σιδήρου φίλημα; περὶ δὲ τῶν φυτῶν λέγουσι παῖδες σοφῶν· καὶ μῦθον ἔλεγον ἂν τὸν λόγον εἶναι, εἰ μὴ καὶ παῖδες ἔλεγον γεωργῶν. Ὁ δὲ λόγος· ἄλλο μὲν ἄλλου φυτὸν ἐρᾷν, τῷ δὲ φοίνικι τὸν ἔρωτα μᾶλλον ἐνοχλεῖν· λέγουσι δὲ τὸν μὲν ἄρρενα τῶν φοινίκων, τὸν δὲ θῆλυν.

\noindent Ὁ ἄρρην οὖν τοῦ θήλεος ἐρᾷ· κἂν ὁ θῆλυς ἀπῳκισμένος ᾖ τῇ τῆς φυτείας στάσει, ὁ ἐραστὴς ὁ ἄρρην αὐαίνεται. Συνίησιν οὖν ὁ γεωργὸς τὴν λύπην τοῦ φυτοῦ, καὶ εἰς τὴν τοῦ χωρίου περιωπὴν ἀνελθὼν ἐφορᾷ ποῖ νένευκε· κλίνεται γὰρ εἰς τὸ ἐρώμενον· καὶ μαθὼν θεραπεύει τοῦ φυτοῦ τὴν νόσον· πτόρθον γὰρ τοῦ θήλεος φοίνικος λαβὼν εἰς τὴν τοῦ ἄρρενος καρδίαν ἐντίθησι. Καὶ ἀνέψυξε μὲν τὴν ψυχὴν τοῦ φυτοῦ, τὸ δὲ σῶμα ἀποθνῆσκον πάλιν ἀνεζωπύρησε καὶ ἐξανέστη, χαῖρον ἐπὶ τῇ τῆς ἐρωμένης  συμπλοκῇ. Καὶ τοῦτό ἐστὶ γάμος φυτῶν.

\end{greek}

}


\section*{Analiza i komentar}

%1

{\large
\begin{greek}
\noindent ῏Η γὰρ ὁ ῎Ερως,\\
\tabto{2em} ἔφη, \\
τοσαύτην ἔχει τὴν ἰσχύν, \\
ὡς καὶ \\
\tabto{2em} μέχρις ὀρνίθων \\
πέμπειν τὸ πῦρ;\\

\end{greek}
}

\begin{description}[noitemsep]
\item[ἔφη] §~312.8, osnove §~327.7; govori Satir, Klitofontov rob
\item[῏Η γὰρ] upitna čestica ἦ uvodi direktno pitanje; u pitanjima se može povezati s γάρ §~520.b
\item[ἔχει] §~231
\item[ὡς καὶ\dots\ πέμπειν] veznik ὡς uvodi zavisnu posljedičnu rečenicu, ovdje s infinitivom, §~473
\end{description}

%2

{\large
\begin{greek}
\noindent Οὐ μέχρις ὀρνίθων,\\
\tabto{2em} ἔφην, \\
τοῦτο γὰρ οὐ θαυμαστόν, \\
\tabto{2em} ἐπεὶ καὶ αὐτὸς ἔχει πτερόν, \\
ἀλλὰ καὶ \\
\tabto{2em} ἑρπετῶν καὶ φυτῶν, \\
ἐγὼ δὲ δοκῶ μοι, \\
\tabto{2em} καὶ λίθων.\\

\end{greek}
}

\begin{description}[noitemsep]
\item[ἔφην] §~312.8, osnove §~327.7; Klitofont pripovijeda svoje doživljaje u prvom licu, pa ovdje prepričava i svoj odgovor
\item[Οὐ μέχρις\dots] sc.\ πέμπει; s ovom surečenicom u koordinaciji stoji nezavisna suprotna surečenica \textgreek[variant=ancient]{ἀλλὰ καὶ\dots}
\item[οὐ θαυμαστόν] sc.\ \textgreek[variant=ancient]{ἐστιν;} imenski predikat Smyth §~909
\item[ἐπεὶ\dots\ ἔχει] veznik uvodi zavisnu uzročnu rečenicu
\item[ἔχει] §~231
\item[ἑρπετῶν\dots\ φυτῶν\dots\ λίθων] sc.\ \textgreek[variant=ancient]{μέχρις}
\item[ἐγὼ δὲ δοκῶ μοι] čestica ovdje s funkcijom suprotnog veznika §~515.2
\item[δοκῶ] §~243; rekcija τινι%doradi ego doko moi

\end{description}

%3

{\large
\begin{greek}
\noindent ἐρᾷ γοῦν \\
ἡ Μαγνησία λίθος \\
τοῦ σιδήρου·\\
κἂν μόνον ἴδῃ καὶ θίγῃ, \\
\tabto{2em} πρὸς αὑτὴν εἵλκυσεν, \\
\tabto{4em} ὥσπερ ἐρωτικὸν \\
\tabto{6em} ἔνδον ἔχουσα \\
\tabto{8em} πῦρ. \\

\end{greek}
}

\begin{description}[noitemsep]
\item[ἐρᾷ] §~243, rekcija τινος, §~396.e
\item[γοῦν] pojačana čestica \textgreek[variant=ancient]{γε (γε οὖν)} ističe prethodnu riječ 
\item[ἡ Μαγνησία λίθος] LSJ \textgreek[variant=ancient]{λίθος} II.
\item[κἂν ἴδῃ καὶ θίγῃ] veznička riječ κἄν uvodi zavisnu dopusnu rečenicu §~480
\item[ἴδῃ] §~254, osnove §~327.3
\item[θίγῃ] §~254, oblik glagola \textgreek[variant=ancient]{θιγγάνω}
\item[εἵλκυσεν] §~267, §~301
\item[ὥσπερ ἔχουσα] §~231; §~519.2; poredbeni prilog \textgreek[variant=ancient]{ὥσπερ} otvara mjesto neobaveznoj dopuni u participu

\end{description}

%4

{\large
\begin{greek}
\noindent καὶ μή τι τοῦτό ἐστιν \\
\tabto{2em} ἐρώσης λίθου \\
\tabto{2em} καὶ ἐρωμένου σιδήρου \\
φίλημα; \\

\end{greek}
}

\begin{description}[noitemsep]
\item[ἐστιν] §~315.2; kopulativni glagol \textgreek[variant=ancient]{εἰμί,} kao dio imenskog predikata, otvara mjesto dopuni, Smyth §~909%me ti touto je sumnjivo
\item[ἐρώσης] §~243
\item[ἐρωμένου] §~243
\end{description}

%5

{\large
\begin{greek}
\noindent περὶ δὲ φυτῶν \\
λέγουσι \\
παῖδες σοφῶν·\\
καὶ μῦθον \\
\tabto{2em} ἔλεγον ἂν \\
τὸν λόγον \\
εἶναι, \\
\tabto{2em} εἰ μὴ καὶ παῖδες ἔλεγον \\
\tabto{4em} γεωργῶν. \\

\end{greek}
}

\begin{description}[noitemsep]
\item[λέγουσι] §~231
\item[παῖδες σοφῶν\dots] \textgreek[variant=ancient]{\textbf{παῖδες γεωργῶν} παῖς} upotrijebljeno u perifrazi, »potomci nekih ljudi« u značenju »neki ljudi«; usp. LSJ \textgreek[variant=ancient]{παῖς} I.3
\item[ἔλεγον] §~231; \textit{verbum dicendi} otvara mjesto obaveznoj dopuni, ovdje A+I
\item[εἶναι] §~315, kopula otvara mjesto imenskoj dopuni kao dijelu imenskoga predikata, Smyth §~909
\item[μῦθον τὸν λόγον εἶναι] A+I
\item[εἰ μὴ καὶ\dots\ ἔλεγον] pogodbeni veznik εἰ μὴ καί uvodi zavisnu pogodbenu rečenicu irealnog oblika za sadašnjost §~474
\item[ἔλεγον] §~231
\end{description}

%6

{\large
\begin{greek}
\noindent ὁ δὲ λόγος· \\
\tabto{2em} ἄλλο μὲν \\
\tabto{4em} ἄλλου \\
\tabto{2em} φυτὸν ἐρᾶν, \\
\tabto{2em} τῷ δὲ φοίνικι \\
\tabto{2em} τὸν ἔρωτα \\
\tabto{2em} μᾶλλον ἐνοχλεῖν.\\

\end{greek}
}

\begin{description}[noitemsep]
\item[ὁ δὲ λόγος] čestica δέ ovdje u eksplanatornoj i narativnoj funkciji: a\dots
\item[λόγος] sc.\ ἐστίν
\item[ἐρᾶν] §~243, rekcija τινος, §~396.e; ovdje i dalje infinitivi su dopuna zamišljenom \textit{verbum dicendi}, indirektni govor §~489
\item[ἄλλο φυτὸν ἐρᾶν] A+I
\item[ἐνοχλεῖν] §~243; rekcija τινι
\item[τὸν ἔρωτα ἐνοχλεῖν] A+I
\end{description}

%7

{\large
\begin{greek}
\noindent λέγουσι δὲ \\
\tabto{2em} τὸν μὲν ἄρρενα \\
\tabto{4em} τῶν φοινίκων, \\
\tabto{2em} τὸν δὲ θῆλυν.\\

\end{greek}
}

\begin{description}[noitemsep]
\item[λέγουσι] §~231; \textit{verbum dicendi} otvara mjesto A+I
\item[δὲ] čestica δέ ovdje u narativnoj funkciji, povezuje misao s prethodnom rečenicom
\item[τὸν μὲν\dots\ τὸν δὲ\dots] rečenični dijelovi koordiniraju se pomoću čestica
\item[τὸν μὲν ἄρρενα\dots\ τὸν δὲ θῆλυν] akuzativi kao obavezne dopune uz \textit{verbum dicendi}, infinitiv je neizrečen (kopula εἶναι)

\end{description}

%8

{\large
\begin{greek}
\noindent ὁ ἄρρην οὖν \\
τοῦ θήλεος \\
ἐρᾷ·\\
κἂν ὁ θῆλυς \\
ἀπῳκισμένος ᾖ \\
\tabto{2em} τῇ τῆς φυτείας στάσει, \\
ὁ ἐραστὴς ὁ ἄρρην \\
αὐαίνεται.\\

\end{greek}
}

\begin{description}[noitemsep]
\item[ἐρᾷ] §~243, rekcija τινος, §~396.e
\item[οὖν] čestica ovdje u funkciji potvrđivanja misli: dakle\dots
\item[κἂν ἀπῳκισμένος ᾖ\dots] veznička riječ κἄν uvodi zavisnu pogodbenu rečenicu eventualnog oblika §~476
\item[ἀπῳκισμένος ᾖ] §~272, složenica οἰκίζω; primijenjen na biljke, glagol je očito upotrijebljen metaforički; nastavak metafore za presađivanje jest i \textgreek[variant=ancient]{ἡ τῆς φυτείας στάσις} (kao što u gradu dođe do sukoba, pa se jedna stranka odseli i osnuje novi grad, tako i vrtlari\dots)
\item[αὐαίνεται] §~231, apodoza pogodbene rečenice

\end{description}

%9

{\large
\begin{greek}
\noindent συνίησιν οὖν \\
ὁ γεωργὸς \\
τὴν λύπην \\
\tabto{2em} τοῦ φυτοῦ, \\
καὶ εἰς τὴν τοῦ χωρίου περιωπὴν \\
ἀνελθὼν \\
ἐφορᾷ \\
\tabto{2em} ποῦ νένευκε· \\
κλίνεται γὰρ \\
\tabto{2em} εἰς τὸ ἐρώμενον.\\

\end{greek}
}

\begin{description}[noitemsep]
\item[συνίησιν] §~305; složenica ἵημι, LSJ συνίημι II.2
\item[εἰς τὴν τοῦ χωρίου περιωπὴν] \textgreek[variant=ancient]{περιωπή} je mjesto s kojeg se može dobro osmatrati uokolo, zato je glagol \textgreek[variant=ancient]{ἀνέρχομαι} (mjesto je uzvišeno); LSJ \textgreek[variant=ancient]{περιωπή} A
\item[ἀνελθὼν] §~254, osnove §~327.2; složenica \textgreek[variant=ancient]{ἔρχομαι}
\item[ἐφορᾷ] §~243, osnove §~327.3; složenica \textgreek[variant=ancient]{ὁράω}
\item[ποῦ νένευκε] upitna riječ ποῦ uvodi zavisnu upitnu rečenicu
\item[νένευκε] §~272
\item[κλίνεται] §~231
\item[γὰρ] čestica γάρ ovdje u eksplanatornom značenju: jer naime\dots
\item[τὸ ἐρώμενον] §~243, supstantivirani particip §~499
\end{description}

%10

{\large
\begin{greek}
\noindent καὶ μαθὼν \\
θεραπεύει \\
\tabto{2em} τοῦ φυτοῦ \\
τὴν νόσον· \\
πτόρθον γὰρ \\
\tabto{2em} τοῦ θήλεος φοίνικος \\
λαβὼν \\
\tabto{2em} εἰς τὴν τοῦ ἄρρενος καρδίαν \\
ἐντίθησι.\\

\end{greek}
}

\begin{description}[noitemsep]
\item[μαθὼν] §~254, osnove §~321.17
\item[θεραπεύει] §~231
\item[λαβὼν] §~254, osnove §~321.14
\item[ἐντίθησι] §~305, složenica τίθημι

\end{description}

%11

{\large
\begin{greek}
\noindent καὶ ἀνέψυξε μὲν \\
ἡ ψυχὴ \\
\tabto{2em} τοῦ φυτοῦ, \\
τὸ δὲ σῶμα \\
\tabto{2em} ἀποθνῇσκον \\
πάλιν ἀνεζωπύρησε καὶ ἐξανέστη, \\
χαῖρον \\
\tabto{2em} ἐπὶ τῇ τῆς ἐρωμένης συμπλοκῇ. \\

\end{greek}
}

\begin{description}[noitemsep]
\item[ἀνέψυξε] §~267, složenica glagola ψύξω
\item[ἀνέψυξε μὲν ἡ ψυχὴ\dots\ τὸ δὲ σῶμα\dots] rečenični članovi koordiniraju se parom čestica
\item[ἀποθνῇσκον] §~231
\item[ἀνεζωπύρησε] §~267, složenica ζωπυρέω
\item[ἐξανέστη] §~306, složenica ἵστημι
\item[χαῖρον] §~231, rekcija ἐπί τινι
\item[τῆς ἐρωμένης] §~243
\end{description}

%12

{\large
\begin{greek}
\noindent καὶ τοῦτό ἐστι γάμος φυτῶν.

\end{greek}
}

\begin{description}[noitemsep]
\item[ἐστι] §~315.2, kopula otvara mjesto imenskoj dopuni, ovdje imenica, Smyth §~909 
\end{description}



%kraj
