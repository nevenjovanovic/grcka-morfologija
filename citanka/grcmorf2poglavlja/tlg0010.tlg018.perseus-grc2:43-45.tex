% Unio ispravke NZ <2022-01-04 uto>

%TKTK


\section*{O tekstu}

\textit{Areopaški govor} (Ἀρεοπαγιτικός, sedmi po redu u zbirci 21 sačuvanih) Izokrat je sastavio najvjerojatnije 354.\ p.~n.~e. U govoru, koji je i pohvalan i deliberativan, nostalgično i utopijski poziva na povratak demokraciji kakva je nekad bila, dok je, po Solonovim i Klistenovim odredbama, Atenom upravljalo areopaško vijeće, te je grad bio sretan i uspješan. Uspoređujući sadašnji trenutak s prošlošću u javnom i vjerskom kao i u privatnom životu Izokrat se zalaže za umjerenu institucionalnu reformu – ni u duhu oligarhije, ni radikalno demokratski – kojom bi Areopag opet dobio političku moć (461.\ su Efijalt i Periklo ograničili djelokrug ovog tijela na krvne delikte).

U odabranom odlomku Izokrat pokazuje kako je Areopag, kada je djelovao na svim područjima života, pozitivno utjecao i na odgoj omladine.

%\newpage

\section*{Pročitajte naglas grčki tekst.}

Isocr.\ Areopagiticus 43–45

%Naslov prema izdanju

\medskip


{\large

\begin{greek}

\noindent  ἁπάντων μὲν οὖν ἐφρόντιζον τῶν πολιτῶν, μάλιστα δὲ τῶν νεωτέρων. ἑώρων γὰρ τοὺς τηλικούτους ταραχωδέστατα διακειμένους καὶ πλείστων γέμοντας ἐπιθυμιῶν, καὶ τὰς ψυχὰς αὐτῶν μάλιστα δαμασθῆναι δεομένας ἐπιμελείαις καλῶν ἐπιτηδευμάτων καὶ πόνοις ἡδονὰς ἔχουσιν· ἐν μόνοις γὰρ ἂν τούτοις ἐμμεῖναι τοὺς ἐλευθέρως τεθραμμένους καὶ μεγαλοφρονεῖν εἰθισμένους.

\noindent  ἅπαντας μὲν οὖν ἐπὶ τὰς αὐτὰς ἄγειν διατριβὰς οὐχ οἷόν τ´ ἦν, ἀνωμάλως τὰ περὶ τὸν βίον ἔχοντας· ὡς δὲ πρὸς τὴν οὐσίαν ἥρμοττεν, οὕτως ἑκάστοις προσέταττον. τοὺς μὲν γὰρ ὑποδεέστερον πράττοντας ἐπὶ τὰς γεωργίας καὶ τὰς ἐμπορίας ἔτρεπον, εἰδότες τὰς ἀπορίας μὲν διὰ τὰς ἀργίας γιγνομένας, τὰς δὲ κακουργίας διὰ τὰς ἀπορίας· ἀναιροῦντες οὖν τὴν ἀρχὴν τῶν κακῶν ἀπαλλάξειν ᾤοντο καὶ τῶν ἄλλων ἁμαρτημάτων τῶν μετ´ ἐκείνην γιγνομένων. τοὺς δὲ βίον ἱκανὸν κεκτημένους περὶ τὴν ἱππικὴν καὶ τὰ γυμνάσια καὶ τὰ κυνηγέσια καὶ τὴν φιλοσοφίαν ἠνάγκασαν διατρίβειν, ὁρῶντες ἐκ τούτων τοὺς μὲν διαφέροντας γιγνομένους, τοὺς δὲ τῶν πλείστων κακῶν ἀπεχομένους.
\end{greek}

}

%\newpage

\section*{Analiza i komentar}

%1

{\large
\begin{greek}
\noindent ῾Απάντων μὲν οὖν \\
\tabto{2em} ἐφρόντιζον \\
τῶν πολιτῶν, \\
μάλιστα δὲ \\
τῶν νεωτέρων.\\

\end{greek}
}

\begin{description}[noitemsep]
\item[῾Απάντων\dots\ τῶν πολιτῶν\dots\ τῶν νεωτέρων] \textit{genetivus memoriae (curae)} §~400.1
\item[῾Απάντων μὲν\dots\ μάλιστα δὲ τῶν νεωτέρων] koordinacija parom čestica izražava blagi kontrast
\item[οὖν] zaključno: dakle\dots, nadovezivanje na ono što je rečeno ranije
\item[ἐφρόντιζον] subjekt su ranije spomenuti "naši preci" \textgreek{ἡμῶν οἱ πρόγονοι;} rekcija τινός; §~231

\end{description}

%2

{\large
\begin{greek}
\noindent ῾Εώρων γὰρ \\
\tabto{2em} τοὺς τηλικούτους \\
\tabto{4em} ταραχωδέστατα \\
\tabto{2em} διακειμένους \\
\tabto{2em} καὶ πλείστων \\
\tabto{4em} γέμοντας \\
\tabto{2em} ἐπιθυμιῶν, \\
\tabto{2em} καὶ τὰς ψυχὰς \\
\tabto{4em} αὐτῶν \\
\tabto{4em} μάλιστα \\
\tabto{4em} παιδευθῆναι \\
\tabto{2em} δεομένας \\
\tabto{4em} ἐπιμελείαις \\
\tabto{6em} καλῶν ἐπιτηδευμάτων \\
\tabto{4em} καὶ πόνοις \\
\tabto{6em} ἡδονὰς \\
\tabto{4em} ἔχουσιν·

\tabto{2em} ἐν μόνοις γὰρ ἂν τούτοις \\
\tabto{4em} ἐμμεῖναι \\
\tabto{4em} τοὺς \\
\tabto{6em} ἐλευθέρως \\
\tabto{4em} τεθραμμένους \\
\tabto{4em} καὶ μέγα φρονεῖν \\
\tabto{4em} εἰθισμένους.\\

\end{greek}
}

\begin{description}[noitemsep]
\item[῾Εώρων] §~327.3, za augment v. §~237.2; \textit{verbum sentiendi} otvara mjesto i akuzativima (npr.\ τηλικούτους) ali i A+I (ἐμμεῖναι τοὺς\dots\ τεθραμμένους καὶ\dots\ εἰθισμένους)
\item[γὰρ] uvodi objašnjenje: naime\dots
\item[τοὺς διακειμένους] supstantivirani particip, složenica glagola κεῖμαι §~315.a, LSJ διάκειμαι A.II
\item[πλείστων\dots\ ἐπιθυμιῶν] \textit{genetivus copiae et inopiae} §~403
\item[γέμοντας] §~231, rekcija τινός
\item[αὐτῶν] posvojni genitiv §~393.1
\item[παιδευθῆναι] §~296
\item[δεομένας] glagol nepotpuna značenja otvara mjesto dopuni u infinitivu, LSJ δέω (B) A.II.1.b, §~232
\item[ἐπιμελείαις] dativ sredstva ovisan o παιδευθῆναι §~414.1
\item[καλῶν ἐπιτηδευμάτων] genitiv objektni §~394
\item[πόνοις ἔχουσιν] dativ sredstva ovisan o παιδευθῆναι §~414.1
\item[ἔχουσιν] §~327.13, §~231
\item[γὰρ] uvodi objašnjenje: naime\dots
\item[ἂν] čestica uz infinitiv u značenju potencijala §~489.b.6
\item[ἐμμεῖναι] složenica glagola μένω §~325.7, §~270, §~268. Bilj. 1.; predikatni dio A+I
\item[τοὺς\dots\ τεθραμμένους] supstantivirani particip, s. 118, perf. med. §~285; imenski dio A+I
\item[φρονεῖν] §~243
\item[μέγα φρονεῖν] LSJ φρονέω A.II.2.b
\item[εἰθισμένους] §~275.1, §~236, ovdje otvara mjesto dopuni u infinitivu
\item[ἐμμεῖναι τοὺς\dots\ τεθραμμένους καὶ\dots\ εἰθισμένους] §~491

\end{description}

%3

{\large
\begin{greek}
\noindent ῞Απαντας μὲν οὖν \\
\tabto{2em} ἐπὶ τὰς αὐτὰς \\
ἄγειν \\
\tabto{2em} διατριβὰς\\
οὐχ οἷόν τ' ἦν, \\
\tabto{2em} ἀνωμάλως \\
\tabto{2em} τὰ περὶ τὸν βίον \\
ἔχοντας·\\
ὡς δὲ \\
\tabto{2em} πρὸς τὴν οὐσίαν \\
ἥρμοττεν, \\
οὕτως \\
ἑκάστοις \\
προσέταττον. \\

\end{greek}
}

\begin{description}[noitemsep]
\item[μὲν οὖν] kombinacija čestica služi povezivanju s prošlim navodom i upućivanje na antitezu, Smyth 2901c
\item[῞Απαντας μὲν\dots\ ὡς δὲ πρὸς τὴν οὐσίαν\dots] koordinacija rečeničnih članova parom čestica
\item[ἐπὶ τὰς αὐτὰς\dots\ διατριβὰς] atributni položaj §~207.2
\item[ἄγειν] §~231
\item[ἦν] §~315
\item[οἷόν τ' ἦν] otvara mjesto dopuni u infinitivu; LSJ οἷος III.2
\item[ἔχοντας] §~327.13, §~231
\item[ὡς\dots\ οὕτως] kako\dots\ tako; Smyth 2990: u poredbenim rečenicama ὡς često dolazi u korelaciji s οὕτως
\item[ἥρμοττεν] LSJ ἁρμόζω II.2; §~231, §~235
\item[προσέταττον] §~231, §~238; složenica τάσσω (atički τάττω)

\end{description}

%4

{\large
\begin{greek}
\noindent Τοὺς μὲν γὰρ \\
\tabto{2em} ὑποδεέστερον \\
πράττοντας \\
\tabto{2em} ἐπὶ τὰς γεωργίας \\
\tabto{2em} καὶ τὰς ἐμπορίας \\
ἔτρεπον, \\
εἰδότες \\
τὰς ἀπορίας μὲν \\
\tabto{2em} διὰ τὰς ἀργίας \\
γιγνομένας, \\
τὰς δὲ κακουργίας \\
\tabto{2em} διὰ τὰς ἀπορίας· \\
ἀναιροῦντες οὖν \\
\tabto{2em} τὴν ἀρχὴν \\
\tabto{4em} τῶν κακῶν\\
\tabto{2em} ἀπαλλάξειν \\
ᾤοντο \\
\tabto{2em} καὶ τῶν ἄλλων ἁμαρτημάτων \\
\tabto{2em} τῶν \\
\tabto{4em} μετ' ἐκείνην \\
\tabto{2em} γιγνομένων.\\

\end{greek}
}

\begin{description}[noitemsep]
\item[Τοὺς μὲν\dots] koordinacija će se nastaviti u sljedećoj rečenici: Τοὺς δὲ\dots
\item[Τοὺς\dots\  πράττοντας] supstantivirani particip, §~231, LSJ πράσσω A.II otvara mjesto priložnoj dopuni
\item[γὰρ] uvodi objašnjenje: naime\dots
\item[ὑποδεέστερον] prilog
\item[ἔτρεπον] §~231
\item[εἰδότες] §~317.4
\item[τὰς ἀπορίας μὲν\dots\ τὰς δὲ κακουργίας] koordinacija rečeničnih članova parom čestica
\item[γιγνομένας] §~325.11
\item[ἀναιροῦντες] složenica αἱρέω §~327.1, §~243
\item[οὖν] zaključno: dakle\dots
\item[τῶν κακῶν] genitiv subjektni §~393
\item[ἀπαλλάξειν] Kad infinitiv ima isti subjekt kao glavni glagol (ᾤοντο) nije ga potrebno navoditi; složenica glagola ἀλλάσσω, rekcija τινός; LSJ ἀπαλλάσσω A.II
\item[ᾤοντο] §~232; \textit{verbum sentiendi} otvara mjesto dopuni u infinitivu
\item[τῶν ἄλλων ἁμαρτημάτων] \textit{genetivus separationis} §~402.1
\item[τῶν μετ' ἐκείνην γιγνομένων] prijedložni izraz i particip kao atributne oznake u atributnom položaju, §~375
\item[γιγνομένων] §~325.11

\end{description}

%5

{\large
\begin{greek}
\noindent Τοὺς δὲ \\
\tabto{2em} βίον ἱκανὸν \\
κεκτημένους\\
\tabto{4em} περί τε τὴν ἱππικὴν \\
\tabto{4em} καὶ τὰ γυμνάσια \\
\tabto{4em} καὶ τὰ κυνηγέσια\\
\tabto{4em} καὶ τὴν φιλοσοφίαν \\
ἠνάγκασαν \\
\tabto{2em} διατρίβειν, \\
ὁρῶντες \\
\tabto{2em} ἐκ τούτων \\
τοὺς μὲν \\
\tabto{2em} διαφέροντας \\
γιγνομένους, \\
τοὺς δὲ \\
\tabto{2em} τῶν πλείστων κακῶν \\
ἀπεχομένους.\\

\end{greek}
}

\begin{description}[noitemsep]
\item[Τοὺς\dots\ κεκτημένους] supstantivirani particip, §~274.4 (izuzeci)
\item[δὲ] čestica označava nadovezivanje na prethodni navod
\item[ἠνάγκασαν] §~267; otvara mjesto dopuni u infinitivu
\item[διατρίβειν] περί τι, LSJ s.~v.\ A.II; §~231
\item[ὁρῶντες] §~243, §~327.3
\item[τοὺς μὲν\dots\ τοὺς δὲ\dots] koordinacija rečeničnih članova parom čestica: da jedni\dots\ a da drugi\dots
\item[διαφέροντας] složenica φέρω, LSJ διαφέρω A.III.4, §~327.5, §~231; imenska dopuna uz γιγνομένους
\item[γιγνομένους] §~325.11, §~232; glagol nepotpuna značenja upotrijebljen kopulativno, Smyth 917 
\item[τῶν πλείστων κακῶν] \textit{genetivus separationis} §~402
\item[ἀπεχομένους] složenica ἔχω, LSJ ἀπέχω A.II.2 rekcija τινος; §~327.13, §~232
\end{description}



%kraj
