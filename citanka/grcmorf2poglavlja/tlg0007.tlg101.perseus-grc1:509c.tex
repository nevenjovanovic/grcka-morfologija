% Unio ispravke NZ <2022-01-02 ned>


\section*{O tekstu}

Esej \textit{O brbljavosti (De garrulitate)} pripada golemoj zbirci Plutarhovih spisa neprecizno nazvanoj \textit{Moralia}. Po opsegu (78 ogleda i govora) ona ne zaostaje za slavnim \textit{Usporednim životopisima}. U tradicionalnom poretku koji se oslanja na tzv.\ Lamprijin katalog – antički popis Plutarhu pripisanih djela – ovaj esej zauzima 36. mjesto. 

Brbljavost \textgreek[variant=ancient]{(ἀδολεσχία),} rašireni oblik antisocijalnog ponašanja, Plutarh tretira kao bolest koju treba prepoznati i potom liječiti; terapija je filozofija. U grčkoj se predaji tema javlja već u staroj komediji s tipskim likom brbljavca \textgreek[variant=ancient]{(ἀδολέσχης)}, a povezuje se i sa Sokratom – brbljavcem \textit{par excellence.} I u Teofrastovoj zbirci \textit{Karakteri} kratkim skicama ocrtan je \textgreek[variant=ancient]{ἀδολέσχης.} Međutim, tek je Plutarh uspio uzdići pojam \textgreek[variant=ancient]{ἀδολεσχία} s razine pučke etike na razinu ozbiljne filozofske etike. Izvorno društveni problem urbane kulture, kojoj je Plutarh pripadao, postaje književnom temom.

Odabrani odlomak pokazuje kako je brbljavost došla glave sudionika u pljački hrama ``Stanovnice mjedene kuće'' \textgreek[variant=ancient]{(ἡ Χαλκίοικος),} tj.\ spartanske Atene. Naime, počinitelj je pred zgroženim mnoštvom nesmotreno iznio silno zamršenu, a ujedno krajnje uvjerljivu, pretpostavku o okolnostima pljačke i time privukao na sebe pažnju koju svakako nije želio.

%\newpage

\section*{Pročitajte naglas grčki tekst.}

Plut.\ De garrulitate 509D

%Naslov prema izdanju

\medskip


{\large

\begin{greek}

\noindent ἐν Λακεδαίμονι τῆς Χαλκιοίκου τὸ ἱερὸν ὤφθη σεσυλημένον, καὶ κειμένη ἔνδον κενὴ λάγυνος. ἦν οὖν ἀπορία πολλῶν συνδεδραμηκότων, καί τις τῶν παρόντων `εἰ βούλεσθ᾽' εἶπεν `ἐγὼ φράσω ὑμῖν ὁ μοι παρίσταται περὶ τῆς λαγύνου· νομίζω γάρ' ἔφη `τοὺς ἱεροσύλους ἐπὶ τηλικοῦτον ἐλθεῖν κίνδυνον, κώνειον ἐμπιόντας καὶ κομίζοντας οἶνον· ἵν᾽ εἰ μὲν αὐτοῖς λαθεῖν ἐγγένοιτο, τῷ ἀκράτῳ ποθέντι σβέσαντες καὶ διαλύσαντες τὸ φάρμακον ἀπέλθοιεν ἀσφαλῶς· εἰ δ᾽ ἁλίσκοιντο, πρὸ τῶν βασάνων ὑπὸ τοῦ φαρμάκου ῥᾳδίως καὶ ἀνωδύνως ἀποθάνοιεν.' ταῦτ᾽ εἰπόντος αὐτοῦ, τὸ πρᾶγμα πλοκὴν ἔχον καὶ περινόησιν τοσαύτην οὐχ ὑπονοοῦντος ἀλλ᾽ εἰδότος ἐφαίνετο. καὶ περιστάντες αὐτὸν ἀνέκριναν ἄλλοθεν ἄλλος `τίς εἶ;' καί `τίς σ᾽ οἶδε;' καί `πόθεν ἐπίστασαι ταῦτα;' καὶ τὸ πέρας ἐλεγχόμενος οὕτως ὡμολόγησεν εἷς εἶναι τῶν ἱεροσύλων.

\end{greek}

}


\section*{Analiza i komentar}

%1

{\large
\begin{greek}
\noindent ἐν Λακεδαίμονι \\
\tabto{2em} τῆς Χαλκιοίκου \\
τὸ ἱερὸν \\
ὤφθη \\
σεσυλημένον, \\
καὶ κειμένη ἔνδον \\
κενὴ λάγυνος.\\
 
\end{greek}
}

\begin{description}[noitemsep]
\item[ὤφθη] ὁράω §~327.3; slaba pasivna osnova §~296
\item[σεσυλημένον] predikatni particip uz \textit{verba sentiendi} §~502.2
\item[κειμένη] §~315.4
\item[κειμένη\dots\ κενὴ λάγυνος] sc.\ ὤφθη; predikatni particip uz \textit{verba sentiendi} §~502.2; λάγυνος, ἡ čutura, ploska
\end{description}



{\large
\begin{greek}
\noindent ἦν οὖν ἀπορία \\
\tabto{2em} πολλῶν συνδεδραμηκότων, \\
καί τις \\
\tabto{2em} τῶν παρόντων \\
‘εἰ βούλεσθ', \\
εἶπεν\\
ἐγὼ φράσω ὑμῖν \\
\tabto{2em} ὅ μοι παρίσταται \\
\tabto{4em} περὶ τῆς λαγύνου· \\
νομίζω γάρ’ \\
\tabto{2em} ἔφη \\
‘τοὺς ἱεροσύλους \\
\tabto{2em} ἐπὶ τηλικοῦτον \\
ἐλθεῖν \\
\tabto{2em} κίνδυνον \\
κώνειον ἐμπιόντας \\
καὶ κομίζοντας οἶνον, \\
ἵν', \\
\tabto{2em} εἰ μὲν \\
\tabto{4em} αὐτοῖς \\
\tabto{4em} λαθεῖν \\
\tabto{2em} ἐγγένοιτο, \\
τῷ ἀκράτῳ ποθέντι σβέσαντες καὶ διαλύσαντες τὸ φάρμακον \\
\tabto{2em} ἀπέλθοιεν ἀσφαλῶς· \\
\tabto{2em} εἰ δ' ἁλίσκοιντο, \\
\tabto{4em} πρὸ τῶν βασάνων \\
\tabto{4em} ὑπὸ τοῦ φαρμάκου \\
\tabto{4em} ῥᾳδίως καὶ ἀνωδύνως \\
\tabto{2em} ἀποθάνοιεν.’\\

\end{greek}
}

\begin{description}[noitemsep]
\item[ἦν] §~315.2
\item[οὖν] §~516.2
\item[συνδεδραμηκότων] συντρέχω §~327.4; perfektna osnova §~272
\item[τῶν παρόντων] πάρειμι §~315; supstantivirani particip §~373; genitiv partitivni §~395
\item[εἰ βούλεσθ'] pogodbena protaza realnog oblika §~474; §~325.13; §~232
\item[εἶπεν] ἀγορεύω / λέγω / φημί §~327.7
\item[φράσω] futurska osnova §~258
\item[ὅ μοι παρίσταται] relativne rečenice §~481; §~482; §~305; παρίσταταί μοι pada mi na um
\item[περὶ τῆς λαγύνου] §~433.Α
\item[νομίζω γάρ’ ἔφη] §~232; \textit{verbum sentiendi} νομίζω otvara mjesto A+I §~493; §~517; §~312.8
\item[ἐλθεῖν] ἔρχομαι §~327.2; jaka aorisna osnova §254; naglasak §~255
\item[ἐπὶ τηλικοῦτον κίνδυνον] §~436.C.c
\item[ἐμπιόντας] ἐμπίνω §~327.16; jaka aorisna osnova §254; naglasak §~255
\item[κομίζοντας] §~232; §~301, s.~118; oba su participa slažu s \textgreek[variant=ancient]{τοὺς ἱεροσύλους} i dio su A+I
\item[ἵν'\dots\ ἀπέλθοιεν] u namjernim rečenicama veznik ἵνα iza sporednih vremena otvara mjesto kosom optativu §~470; ἀπέρχομαι §~327.2; jaka aorisna osnova §~254
\item[εἰ μὲν αὐτοῖς\dots\ ἐγγένοιτο] εἰ + opt., pogodbena potencijalna protaza §~477; jaka aorisna osnova §254; ἐγγίγνεταί τινι + inf. bezlično: moguće je, ukazuje se prilika komu; εἰ μέν\dots\ εἰ δέ §~519
\item[λαθεῖν] λανθάνω §~321.15; jaka aorisna osnova §~254; naglasak §~255
\item[τῷ ἀκράτῳ ποθέντι] \textit{dativus instrumenti;} ὁ ἄκρατος (i bez οἶνος) čisto vino (tj. bez vode); πίνω §~327.16, ποθείς, part. aor. pas.: popijen
\item[σβέσαντες καὶ διαλύσαντες] §~319.6; prelazno i neprelazno značenje §~329; §~301.2; slabi aor. §~267; adverbni particip §~503 
\item[εἰ δ' ἁλίσκοιντο] εἰ + opt., pogodbena potencijalna protaza §~477; §~324.5; §~232;
\item[πρὸ τῶν βασάνων] §~425.b; βάσανος, ἡ (također u množini): istraga mučenjem, tortura
\item[ὑπὸ τοῦ φαρμάκου] §~437.b.β
\item[ἀποθάνοιεν] §~324.8; jaka aorisna osnova §~254; pripada namjernoj rečenici uvedenoj veznikom ἵνα (ἵν')

\end{description}

%2

{\large
\begin{greek}
\noindent ταῦτ' \\
\tabto{2em} εἰπόντος αὐτοῦ \\
τὸ πρᾶγμα \\
\tabto{2em} πλοκὴν ἔχον καὶ περινόησιν τοσαύτην \\
οὐχ ὑπονοοῦντος ἀλλ' εἰδότος \\
\tabto{2em} ἐφαίνετο· \\
καὶ περιστάντες \\
\tabto{2em} αὐτὸν \\
ἀνέκριναν \\
\tabto{2em} ἀλλαχόθεν ἄλλος \\
‘τίς εἶ;’ \\
καί ‘τίς σ' οἶδε;’ \\
καί ‘πόθεν ἐπίστασαι ταῦτα;’ \\
καὶ \\
\tabto{2em} τὸ πέρας \\
\tabto{2em} ἐλεγχόμενος οὕτως \\
ὡμολόγησεν \\
\tabto{2em} εἷς εἶναι \\
\tabto{4em} τῶν ἱεροσύλων. \\

\end{greek}
}

\begin{description}[noitemsep]
\item[εἰπόντος αὐτοῦ] GA §~504; ἀγορεύω / λέγω / φημί §~327.7
\item[ἔχον] §~232; predikatni particip uz kopulativne glagole §~501.b
\item[περινόησιν] pronicljivost, oštroumnost
\item[ὑπονοοῦντος] §~243
\item[εἰδότος] §~317.4
\item[ἐφαίνετο] §~232
\item[περιστάντες] περιίσταμαί τινα okruživati koga; jaki aorist §~306; adverbni particip §~503
\item[ἀνέκριναν] §~301 (s.~118); osnova slabog aorista §~267  
\item[εἶ] §~315.2
\item[οἶδε] §~317.4
\item[ἐπίστασαι] §~312.6, Bilj.~2
\item[τὸ πέρας] priložno = τὸ τέλος; adverbni akuzativ §~391
\item[ἐλεγχόμενος] §~232; adverbni particip §~503
\item[ὡμολόγησεν] osnova slabog aorista §~267; sigmatski aorist kod \textit{verba vocalia} §~269
\item[εἷς εἶναι] N+I §~491.2; §~315.2
\item[τῶν ἱεροσύλων] genitiv partitivni §~395

\end{description}


%kraj
