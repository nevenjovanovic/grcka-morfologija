% Unesi ispravke NZ <2022-01-02 ned>

\section*{O tekstu}

\textgreek[variant=ancient]{Κύριαι δόξαι,} \textit{Glavni nauci}, zbirka je četrdeset Epikurovih načela koju Diogen Laertije (možda nešto prije 250.\ n.~e) uključuje u desetu knjigu svojeg enciklopedijskog djela \textgreek[variant=ancient]{Βίοι καὶ γνῶμαι τῶν ἐν φιλοσοφίᾳ εὐδοκιμησάντων} \textit{(Životopisi i misli znamenitih filozofa)}. Ta deseta, i posljednja, knjiga u cijelosti je posvećena Epikuru.

Ovaj odlomak donosi prvih pet nauka; četiri od njih naknadno su sažeti još i u τετραφάρμακος (``lijek od četiri sastojka''), nađen na jednom od papirusa iz Herkulaneja (pap.\ Herc.\ 1005, 4.9–14). Prvi se nauk, u rečenici unutar zagrada, poziva i na Epikurovu teoriju spoznaje, prema kojoj na ljudska osjetila (pa i na um) djeluju ``sličice'', εἴδωλα, tanke opne koje predmeti emitiraju. Peti nauk uči da nema života u ugodi bez razuma, ljepote i pravednosti, niti obratno.



%\newpage

\section*{Pročitajte naglas grčki tekst.}

Epicur.\ Ratae sententiae 1.1

%Naslov prema izdanju

\medskip


{\large

\begin{greek}

\noindent ΚΥΡΙΑΙ ΔΟΞΑΙ

\noindent I. Τὸ μακάριον καὶ ἄφθαρτον οὔτε αὐτὸ πράγματα ἔχει οὔτε ἄλλῳ παρέχει, ὥστε οὔτε ὀργαῖς οὔτε χάρισι συνέχεται· ἐν ἀσθενεῖ γὰρ πᾶν τὸ τοιοῦτον. (ἐν ἄλλοις δέ φησι τοὺς θεοὺς λόγῳ θεωρητούς, οὓς μὲν κατ' ἀριθμὸν ὑφεστῶτας, οὓς δὲ καθ' ὁμοείδειαν ἐκ τῆς συνεχοῦς ἐπιρρύσεως τῶν ὁμοίων εἰδώλων ἐπὶ τὸ αὐτὸ ἀποτετελεσμένων, ἀνθρωποειδεῖς.)

\noindent II. Ὁ θάνατος οὐδὲν πρὸς ἡμᾶς· τὸ γὰρ διαλυθὲν ἀναισθητεῖ· τὸ δ' ἀναισθητοῦν οὐδὲν πρὸς ἡμᾶς.

\noindent III. Ὅρος τοῦ μεγέθους τῶν ἡδονῶν ἡ παντὸς τοῦ ἀλγοῦντος ὑπεξαίρεσις. ὅπου δ' ἂν τὸ ἡδόμενον ἐνῇ, καθ' ὃν ἂν χρόνον ᾖ, οὐκ ἔστι τὸ ἀλγοῦν ἢ τὸ λυπούμενον ἢ τὸ συναμφότερον.

\noindent IV. Οὐ χρονίζει τὸ ἀλγοῦν συνεχῶς ἐν τῇ σαρκί, ἀλλὰ τὸ μὲν ἄκρον τὸν ἐλάχιστον χρόνον πάρεστι, τὸ δὲ μόνον ὑπερτεῖνον τὸ ἡδόμενον κατὰ σάρκα οὐ πολλὰς ἡμέρας συμμένει. αἱ δὲ πολυχρόνιοι τῶν ἀρρωστιῶν πλεονάζον ἔχουσι τὸ ἡδόμενον ἐν τῇ σαρκὶ ἤπερ τὸ ἀλγοῦν.

\noindent V. Οὐκ ἔστιν ἡδέως ζῆν ἄνευ τοῦ φρονίμως καὶ καλῶς καὶ δικαίως, $\langle$οὐδὲ φρονίμως καὶ καλῶς καὶ δικαίως$\rangle$ ἄνευ τοῦ ἡδέως. ὅτῳ δὲ τοῦτο μὴ ὑπάρχει ἐξ οὗ ζῆν φρονίμως, καὶ καλῶς καὶ δικαίως ὑπάρχει, οὐκ ἔστι τοῦτον ἡδέως ζῆν.

\end{greek}

}


\section*{Analiza i komentar}

%1

{\large
\begin{greek}
\noindent ΚΥΡΙΑΙ ΔΟΞΑΙ\\
Τὸ μακάριον καὶ ἄφθαρτον \\
οὔτε αὐτὸ πράγματα ἔχει \\
οὔτε ἄλλῳ παρέχει, \\
ὥστε \\
\tabto{2em} οὔτε ὀργαῖς \\
\tabto{2em} οὔτε χάρισι \\
συνέχεται· \\
ἐν ἀσθενεῖ γὰρ \\
πᾶν τὸ τοιοῦτον.\\

\end{greek}
}

\begin{description}[noitemsep]
\item[Τὸ μακάριον καὶ ἄφθαρτον] supstantivirani pridjevi, §~373
\item[οὔτε αὐτο\dots\ οὔτε ἄλλῳ\dots] koordinacija pomoću para (niječnih) sastavnih veznika
\item[πράγματα ἔχει] LSJ πρᾶγμα III.5
\item[παρέχει] sc.\ πράγματα; §~231; složenica ἔχω
\item[ὥστε] veznik uvodi posljedičnu rečenicu
\item[συνέχεται] složenica ἔχω; §~232; LSJ συνέχω I.5
\item[γὰρ] čestica najavljuje iznošenje dokaza prethodne tvrdnje: naime\dots
\item[ἐν ἀσθενεῖ] sc.\ ἐστι, imenski predikat, Smyth 910 (kopula, tj.\ infinitiv εἶναι, ovdje je neizrečena)

\end{description}

%2

{\large
\begin{greek}
\noindent (ἐν ἄλλοις δέ φησι \\
\tabto{2em} τοὺς θεοὺς λόγῳ θεωρητούς, \\
\tabto{4em} οὓς μὲν \\
\tabto{6em} κατ' ἀριθμὸν \\
\tabto{4em} ὑφεστῶτας, \\
\tabto{4em} οὓς δὲ \\
\tabto{6em} κατὰ ὁμοείδειαν \\
\tabto{6em} ἐκ τῆς συνεχοῦς ἐπιρρύσεως \\
\tabto{8em} τῶν ὁμοίων εἰδώλων \\
\tabto{8em} ἐπὶ τὸ αὐτὸ \\
\tabto{8em} ἀποτετελεσμένων, \\
\tabto{4em} ἀνθρωποειδεῖς.) \\

\end{greek}
}

\begin{description}[noitemsep]
\item[ἐν ἄλλοις] drugdje
\item[φησι] §~312.8; \textit{verbum dicendi} otvara mjesto A+I
\item[τοὺς θεοὺς\dots\ θεωρητούς] A+I, imenski predikat, Smyth 910 (kopula je ovdje neizrečena)
\item[οὓς μὲν\dots\ οὓς δε\dots] sc.\ τοὺς θεοὺς; koordinacija surečenica pomoću para čestica
\item[ὑφεστῶτας] složenica ἵστημι; §~311; LSJ ὑφίστημι B.IV.2
\item[τῶν\dots\ εἰδώλων] u Epikurovoj filozofiji, osjete izazivaju tanke opne, ``sličice'' \textgreek[variant=ancient]{(εἴδωλα)} koje odašilju predmeti, a primaju ih ljudska osjetila; neke od tih sličica dovoljno su istančane da mogu doprijeti izravno do uma (koji se nalazi u prsima), i preko njih \textgreek[variant=ancient]{(κατὰ ὁμοείδειαν)} možemo zamisliti objekte poput bogova.
\item[ἐπὶ τὸ αὐτὸ] izriče svrhu: ``u istu tu svrhu''
\item[ἀποτετελεσμένων] složenica τελέω; §~272
\item[ἀνθρωποειδεῖς] sc.\ τοὺς θεοὺς; pridjev je imenski dio predikata (izostavljena je kopula)

\end{description}

%3

{\large
\begin{greek}
\noindent  Ὁ θάνατος \\
\tabto{2em} οὐδὲν πρὸς ἡμᾶς· \\
τὸ γὰρ διαλυθὲν \\
\tabto{2em} ἀναισθητεῖ· \\
τὸ δ' ἀναισθητοῦν \\
\tabto{2em} οὐδὲν πρὸς ἡμᾶς.\\

\end{greek}
}

\begin{description}[noitemsep]
\item[οὐδὲν πρὸς ἡμᾶς] sc.\ οὐδέν ἐστι; πρός LSJ C.III.1: u odnosu na\dots\ što se tiče\dots
\item[τὸ\dots\ διαλυθὲν] §~296; složenica glagola λύω (osnove s.~108); supstantivirani particip, §~373
\item[τὸ γὰρ\dots\ τὸ δ'\dots] koordinacija rečeničnih članaka pomoću čestice δέ; γὰρ: naime\dots
\item[ἀναισθητεῖ] §~243
\item[τὸ\dots\ ἀναισθητοῦν] §~243; supstantivirani particip, §~373

\end{description}

%4

{\large
\begin{greek}
\noindent  Ὅρος \\
\tabto{2em} τοῦ μεγέθους \\
\tabto{4em} τῶν ἡδονῶν \\
ἡ \\
\tabto{2em} παντὸς τοῦ ἀλγοῦντος \\
ὑπεξαίρεσις. \\

\end{greek}
}

\begin{description}[noitemsep]
\item[Ὅρος] sc.\ ἐστι, imenski predikat, Smyth 910 (kopula je ovdje neizrečena)
\item[τοῦ ἀλγοῦντος] §~243; supstantivirani particip, §~373

\end{description}

%5

{\large
\begin{greek}
\noindent ὅπου δ' ἂν \\
τὸ ἡδόμενον \\
ἐνῇ, \\
καθ’ ὃν ἂν χρόνον ᾖ, \\
οὐκ ἔστι \\
\tabto{2em} τὸ ἀλγοῦν \\
\tabto{2em} ἢ λυπούμενον \\
\tabto{2em} ἢ τὸ συναμφότερον.\\

\end{greek}
}

\begin{description}[noitemsep]
\item[ὅπου δ' ἂν\dots\ ἐνῇ] hipotetička relativna rečenica, eventualni oblik, §~486: gdje god\dots
\item[τὸ ἡδόμενον] §~232; supstantivirani particip, §~373
\item[ἐνῇ] složenica εἰμί, §~315
\item[καθ’ ὃν ἂν χρόνον ᾖ] imenski predikat, Smyth 910; hipotetička relativna rečenica, eventualni oblik, §~486: u koje god\dots
\item[οὐκ ἔστι] §~315; egzistencijalno značenje, LSJ εἰμί A.II.2
\item[τὸ ἀλγοῦν] §~243; supstantivirani particip, §~373
\item[λυπούμενον] §~243; supstantivirani particip, §~373
\item[τὸ συναμφότερον] supstantivirani pridjev, §~373

\end{description}

%6

{\large
\begin{greek}
\noindent Οὐ χρονίζει \\
τὸ ἀλγοῦν \\
\tabto{2em} συνεχῶς\\
\tabto{2em} ἐν τῇ σαρκί, \\
ἀλλὰ \\
\tabto{2em} τὸ μὲν ἄκρον \\
\tabto{4em} τὸν ἐλάχιστον χρόνον \\
\tabto{2em} πάρεστι, \\
\tabto{2em} τὸ δὲ μόνον ὑπερτεῖνον τὸ ἡδόμενον \\
\tabto{2em} κατὰ σάρκα \\
\tabto{2em} οὐ πολλὰς ἡμέρας \\
\tabto{2em} συμμένει· \\
\tabto{2em} αἱ δὲ πολυχρόνιοι \\
\tabto{4em} τῶν ἀρρωστιῶν \\
\tabto{2em} πλεονάζον\\
\tabto{4em} ἔχουσι\\
\tabto{2em} τὸ ἡδόμενον \\
\tabto{4em} ἐν τῇ σαρκὶ \\
\tabto{2em} ἤπερ τὸ ἀλγοῦν.\\

\end{greek}
}

\begin{description}[noitemsep]
\item[χρονίζει] §~231
\item[τὸ μὲν ἄκρον\dots\ τὸ δὲ μόνον ὑπερτεῖνον\dots] koordinacija rečeničnih članova pomoću para čestica
\item[τὸ μὲν ἄκρον] sc.\ τὸ ἀλγοῦν
\item[τὸν ἐλάχιστον χρόνον] akuzativ pokazuje protezanje u vremenu §~390, ima službu priložne oznake vremena
\item[πάρεστι] složenica εἰμί, §~315
\item[τὸ\dots\ ὑπερτεῖνον] §~231; složenica τείνω; supstantiviranje participa članom §~373
\item[τὸ ἡδόμενον] §~232; supstantiviranje participa članom §~373
\item[πολλὰς ἡμέρας] akuzativ pokazuje protezanje u vremenu §~390, ima službu priložne oznake vremena
\item[συμμένει] §~231; složenica μένω
\item[πλεονάζον] §~231
\item[ἔχουσι] §~231
\item[ἤπερ] pojačan veznik ἤ, ovisan o komparativu (ideja usporedbe sadržana je u glagolu πλεονάζω)

\end{description}

%7

{\large
\begin{greek}
\noindent Οὐκ ἔστιν ἡδέως ζῆν \\
ἄνευ τοῦ φρονίμως \\
\tabto{2em} καὶ καλῶς \\
\tabto{2em} καὶ δικαίως \\
$\langle$οὐδὲ \\
\tabto{2em} φρονίμως \\
\tabto{2em} καὶ καλῶς \\
\tabto{2em} καὶ δικαίως$\rangle$\\
ἄνευ τοῦ ἡδέως· \\
ὅτῳ δὲ \\
τοῦτο \\
μὴ ὑπάρχει, \\
ἐξ οὗ ζῆν \\
\tabto{2em} φρονίμως καὶ καλῶς καὶ δικαίως \\
ὑπάρχει,
\tabto{2em} οὐκ ἔστι \\
\tabto{4em} τοῦτον \\
\tabto{4em} ἡδέως ζῆν.\\

\end{greek}
}

\begin{description}[noitemsep]
\item[Οὐκ ἔστιν] §~315; u egzistencijalnom značenju otvara mjesto infinitivu, LSJ εἰμί A.VI
\item[ζῆν] §~243
\item[τοῦ φρονίμως\dots] \textbf{καὶ καλῶς καὶ δικαίως} sc.\ ζῆν (supstantivirani infinitiv §~497)
\item[ὑπάρχει] LSJ ὑπάρχω III, rekcija τινί
\item[οὐκ ἔστι] §~315; u egzistencijalnom značenju otvara mjesto A+I, LSJ εἰμί A.VI

\end{description}


%kraj
