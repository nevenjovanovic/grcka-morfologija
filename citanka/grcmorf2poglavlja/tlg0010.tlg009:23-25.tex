%\section*{O autoru}

%TKTK


\section*{O tekstu}

U ovom odlomku iz Izokratova epideiktičkoga govora \textit{Pohvala Helene} (uobičajeno se datira oko 370.\ pr.~Kr.), ujedno i lijepa primjera retoričke vježbe na danu temu, čitamo dio pohvale Tezeja (proteže se od 18. do 38. poglavlja). 

Tezej (Θησεύς) bio je atenski nacionalni heroj, i njegov spomen u djelu atenskoga autora ne treba posebno opravdanje; no, izrazito dug iskorak iz pohvale ``glavnog lika'' epideiktičkog govora, Helene same, u pohvalu nekog drugoga otvorio je prostor za negativne kritike. Već sam autor upozorava (29.) na rizik kojem se izložio takvim udaljavanjem od teme.
 
No, postoji dobar razlog da se u priči o Heleni govori o Tezeju. On je, naime, po jednoj varijanti mita oteo mladu Helenu.  Njome se namjeravao i oženiti, no naposljetku su je iz njegova zatočeništva spasili Dioskuri.

%\newpage

\section*{Pročitajte naglas grčki tekst.}

Isocr.\ Helenae encomium 23–25

%Naslov prema izdanju

\medskip


{\large

\begin{greek}

\noindent κάλλιστον μὲν οὖν ἔχω περὶ Θησέως τοῦτʼ εἰπεῖν, ὅτι κατὰ τὸν αὐτὸν χρόνον Ἡρακλεῖ γενόμενος ἐνάμιλλον τὴν αὑτοῦ δόξαν πρὸς τὴν ἐκείνου κατέστησεν. οὐ γὰρ μόνον τοῖς ὅπλοις ἐκοσμήσαντο παραπλησίοις, ἀλλὰ καὶ τοῖς ἐπιτηδεύμασιν ἐχρήσαντο τοῖς αὐτοῖς, πρέποντα τῇ συγγενείᾳ ποιοῦντες. ἐξ ἀδελφῶν γὰρ γεγονότες, ὁ μὲν ἐκ Διός, ὁ δʼ ἐκ Ποσειδῶνος, ἀδελφὰς καὶ τὰς ἐπιθυμίας ἔσχον. μόνοι γὰρ οὗτοι τῶν προγεγενημένων ὑπὲρ τοῦ βίου τοῦ τῶν ἀνθρώπων ἀθληταὶ κατέστησαν. συνέβη δὲ τὸν μὲν ὀνομαστοτέρους καὶ μείζους, τὸν δʼ ὠφελιμωτέρους καὶ τοῖς Ἕλλησιν οἰκειοτέρους ποιήσασθαι τοὺς κινδύνους. τῷ μὲν γὰρ Εὐρυσθεὺς προσέταττε τάς τε βοῦς τὰς ἐκ τῆς Ἐρυθείας ἀγαγεῖν καὶ τὰ μῆλα τὰ τῶν Ἑσπερίδων ἐνεγκεῖν καὶ τὸν Κέρβερον ἀναγαγεῖν καὶ τοιούτους ἄλλους πόνους, ἐξ ὧν ἤμελλεν οὐ τοὺς ἄλλους ὠφελήσειν ἀλλʼ αὐτὸς κινδυνεύσειν· ὁ δʼ αὐτὸς αὑτοῦ κύριος ὢν τούτους προῃρεῖτο τῶν ἀγώνων ἐξ ὧν ἤμελλεν ἢ τῶν Ἑλλήνων ἢ τῆς αὑτοῦ πατρίδος εὐεργέτης γενήσεσθαι.

\end{greek}

}


\section*{Analiza i komentar}

%1

{\large
\begin{greek}
\noindent κάλλιστον μὲν οὖν ἔχω \\
\tabto{2em} περὶ Θησέως \\
τοῦτ' εἰπεῖν, \\
\tabto{2em} ὅτι κατὰ τὸν αὐτὸν χρόνον \\
\tabto{4em} ῾Ηρακλεῖ \\
\tabto{2em} γενόμενος \\
\tabto{2em} ἐνάμιλλον τὴν αὐτοῦ δόξαν \\
\tabto{4em} πρὸς τὴν ἐκείνου \\
\tabto{2em} κατέστησεν. \\

\end{greek}
}

\begin{description}[noitemsep]
\item[μὲν οὖν] kombinacija čestica označava prelaz i rezimiranje: dakle\dots
\item[ἔχω] § 231, osnove § 327.13, Smyth 2000a, otvara mjesto dopuni u obliku izrične rečenice
\item[εἰπεῖν] § 254, osnove § 327.7
\item[τοῦτ'\dots\ ὅτι\dots\ κατέστησεν] zamjenica najavljuje, a veznik uvodi zavisnu izričnu rečenicu
\item[κατὰ τὸν αὐτὸν χρόνον] rekcija τινί, LSJ αὐτός III
\item[ἐνάμιλλον] rekcija πρός τι, LSJ s.~v.
\item[γενόμενος] § 254, osnove § 325.11
\item[κατέστησεν] § 311.2, složenica glagola ἵστημι, osnove § 311.1

\end{description}

%2


{\large
\begin{greek}
\noindent οὐ γὰρ μόνον \\
\tabto{2em} τοῖς ὅπλοις \\
ἐκοσμήσαντο \\
\tabto{2em} παραπλησίοις, \\
ἀλλὰ καὶ τοῖς ἐπιτηδεύμασιν \\
ἐχρήσαντο \\
\tabto{2em} τοῖς αὐτοῖς, \\
\tabto{4em} πρέποντα \\
\tabto{6em} τῇ συγγενείᾳ \\
\tabto{4em} ποιοῦντες. \\

\end{greek}
}

\begin{description}[noitemsep]
\item[οὐ γὰρ μόνον\dots\ ἀλλὰ καὶ\dots] koordinacija rečeničnih članova § 513.1; γάρ u eksplanatornoj funkciji: naime\dots
\item[ἐκοσμήσαντο] § 267
\item[ἐχρήσαντο] § 267, rekcija τινί
\item[πρέποντα] § 231; particip s rekcijom τινί LSJ πρέπω III.2
\item[ποιοῦντες] § 243

\end{description}

%3

{\large
\begin{greek}
\noindent ἐξ ἀδελφῶν γὰρ γεγονότες, \\
\tabto{2em} ὁ μὲν ἐκ Διὸς, \\
\tabto{2em} ὁ δ' ἐκ Ποσειδῶνος, \\
ἀδελφὰς καὶ τὰς ἐπιθυμίας \\
ἔσχον. \\

\end{greek}
}

\begin{description}[noitemsep]
\item[γεγονότες] § 272, osnove § 325.11
\item[γὰρ] čestica u eksplanatornoj funkciji: naime\dots
\item[ὁ μὲν\dots\ ὁ δ'\dots] rečenični članovi iste funkcije koordiniraju se česticama, § 370.1
\item[ἔσχον] § 231, osnove § 327.13

\end{description}

%4

{\large
\begin{greek}
\noindent μόνοι γὰρ οὗτοι \\
\tabto{2em} τῶν προγεγενημένων \\
\tabto{2em} ὑπὲρ τοῦ βίου \\
\tabto{4em} τοῦ τῶν ἀνθρώπων \\
ἀθληταὶ κατέστησαν. \\

\end{greek}
}

\begin{description}[noitemsep]
\item[τῶν προγεγενημένων] supstantivirani particip § 499
\item[κατέστησαν] § 306, § 311.2, složenica glagola ἵστημι, osnove § 311.1, s imenskom dopunom LSJ καθίστημι II.B.5

\end{description}

%5

{\large
\begin{greek}
\noindent συνέβη δὲ \\
\tabto{2em} τὸν μὲν ὀνομαστοτέρους καὶ μείζους, \\
\tabto{2em} τὸν δ' ὠφελιμωτέρους καὶ τοῖς ῞Ελλησιν οἰκειοτέρους \\
\tabto{2em} ποιήσασθαι \\
\tabto{4em} τοὺς κινδύνους. \\

\end{greek}
}

\begin{description}[noitemsep]
\item[συνέβη] § 316, složenica glagola βαίνω, osnove § 321.6; bezlično LSJ συμβαίνω III, otvara mjesto dopuni u A+I
\item[δὲ] čestica ovdje u adverzativnom značenju: a\dots
\item[τὸν μὲν\dots\ τὸν δ'\dots] rečenični članovi iste funkcije koordiniraju se česticama
\item[τὸν μὲν\dots\ τὸν δ'\dots\ ποιήσασθαι] A+I
\item[ποιήσασθαι] § 267%kindunous?

\end{description}

%6

{\large
\begin{greek}
\noindent τῷ μὲν γὰρ \\
Εὐρυσθεὺς \\
προσέταττεν \\
\tabto{2em} τάς τε βοῦς\\
\tabto{4em} τὰς ἐκ τῆς ᾿Ερυθείας \\
\tabto{2em} ἀγαγεῖν\\
\tabto{2em} καὶ τὰ μῆλα \\
\tabto{4em} τὰ τῶν ῾Εσπερίδων \\
\tabto{2em} ἐνεγκεῖν\\
\tabto{2em} καὶ τὸν Κέρβερον \\
\tabto{2em} ἀναγαγεῖν\\
\tabto{2em} καὶ τοιούτους ἄλλους πόνους, \\
\tabto{4em} ἐξ ὧν ἤμελλεν \\
\tabto{6em} οὐ τοὺς ἄλλους ὠφελήσειν \\
\tabto{6em} ἀλλ' αὐτὸς κινδυνεύσειν· \\
ὁ δ', αὐτὸς αὑτοῦ κύριος ὢν, \\
\tabto{2em} τούτους \\
προῃρεῖτο \\
\tabto{2em} τῶν ἀγώνων \\
\tabto{4em} ἐξ ὧν ἤμελλεν \\
\tabto{6em} ἢ τῶν ῾Ελλήνων \\
\tabto{6em} ἢ τῆς αὑτοῦ πατρίδος \\
\tabto{6em} εὐεργέτης γενήσεσθαι.\\

\end{greek}
}

\begin{description}[noitemsep]
\item[τῷ μὲν\dots\ ὁ δ'\dots] čestice uspostavljaju koordinaciju između rečeničnih članova
\item[γὰρ] čestica u eksplanatornoj funkciji: naime\dots
\item[προσέταττεν] § 231, složenica glagola τάττω, otvara mjesto dopunama u infinitivu
\item[ἀγαγεῖν] § 254, osnove s.~116
\item[ἐνεγκεῖν] § 254, osnove § 327.5
\item[ἀναγαγεῖν] § 254, složenica ἄγω, osnove s.~116
\item[ἐξ ὧν\dots\ ἤμελλεν] zamjenica u prijedložnom izrazu uvodi zavisnu odnosnu rečenicu, antecedent je τοιούτους ἄλλους πόνους
\item[ἤμελλεν] § 231, osnove § 325.15, otvara mjesto dopunama u infinitivu (futura)
\item[ἀλλ'] veznik ἀλλά uvodi nezavisnu suprotnu rečenicu
\item[ὠφελήσειν] § 258, rekcija τινα: koristiti kome
\item[κινδυνεύσειν] § 258 
\item[κύριος ὢν] § 315.2, adverbni particip § 503
\item[προῃρεῖτο] § 231, složenica glagola αἱρέω, osnove § 327.1
\item[ἐξ ὧν\dots\ ἤμελλεν] kao gore
\item[ἤμελλεν] kao gore
\item[ἢ\dots\ ἢ\dots] koordinacija rastavnim veznicima
\item[ἢ τῶν ῾Ελλήνων] sc.\ εὐεργέτης γενήσεσθαι
\item[γενήσεσθαι] § 258, osnove § 325.11
\end{description}


%kraj
