% Unesi ispravke NJ, 7. 4. 2020.
% Unio ispravke NZ, <2022-01-01 sub>
%\section*{O autoru}


\section*{O tekstu}

Od Tukidida saznajemo za atenski običaj godišnjih javnih sahrana onih koji su pali u borbi za državu; tom bi prigodom neki ugledan građanin održao javni nadgrobni govor. Znamenit je primjer Periklova govora (Thuc.~2, 35–46). Takav je govor i drugi tekst sačuvan u Lizijinu korpusu (Lys.~2). Lizijin je govor održan u čast palih u Korintskom ratu; rat je izbio 395., kad se Atena pridružila Korintu, Argu i Tebi u pobuni protiv Sparte (rat je trajao do 387., kad su, uz perzijsku podršku, Spartanci situaciju uspjeli okrenuti u svoju korist).

Lys.~2 jedan je od dva Lizijina govora koji nisu namijenjeni sudskim procesima te od glavnine korpusa odudara i stilom. Zato su filolozi dovodili u pitanje Lizijino autorstvo (označava se atributom \textit{spurius}, ``dvojbena porijekla''). Osim toga, Lizija, koji nije bio atenski građanin, teško da je govor držao osobno. Malo je vjerojatno i da bi ga pisao za drugoga; atenski građanin dovoljno ugledan (to znači, dovoljno politički istaknut) da govori u ovakvoj prigodi vjerojatno bi govor i sam mogao sastaviti. Djelo je, stoga, možda retorička vježba s početka IV.~st.\ pr.~n.~e, ili u to vrijeme nastao pamflet namijenjen širenju u pisanom obliku. Nemamo dovoljno podataka da bilo koju od ovih teorija opovrgnemo ili potvrdimo.

Važan je dio nadgrobnih govora veličanje postignuća predaka. U tom kontekstu, odmah na početku govora, Lizija prikazuje napad Amazonki na Atenu; drevni Atenjani dokazali su da su ove zastrašujuće ratnice bile ipak samo žene, i nijedna se od njih nije vratila u domovinu da bi izvukla pouku iz ovog nepromišljenog čina.


\newpage

\section*{Pročitajte naglas grčki tekst.}

Lys.\ Epitaphius [Sp.] 4–6

%Naslov prema izdanju

\medskip


{\large

\begin{greek}

\noindent Ἀμαζόνες γὰρ Ἄρεως μὲν τὸ παλαιὸν ἦσαν θυγατέρες, οἰκοῦσαι δὲ παρὰ τὸν Θερμώδοντα ποταμόν, μόναι μὲν ὡπλισμέναι σιδήρῳ τῶν περὶ αὐτάς, πρῶται δὲ τῶν πάντων ἐφʼ ἵππους ἀναβᾶσαι, οἷς ἀνελπίστως διʼ ἀπειρίαν τῶν ἐναντίων ᾕρουν μὲν τοὺς φεύγοντας, ἀπέλειπον δὲ διώκοντας· ἐνομίζοντο δὲ διὰ τὴν εὐψυχίαν μᾶλλον ἄνδρες ἢ διὰ τὴν φύσιν γυναῖκες· πλέον γὰρ ἐδόκουν τῶν ἀνδρῶν ταῖς ψυχαῖς διαφέρειν ἢ ταῖς ἰδέαις ἐλλείπειν.

ἄρχουσαι δὲ πολλῶν ἐθνῶν, καὶ ἔργῳ μὲν τοὺς περὶ αὐτὰς καταδεδουλωμέναι, λόγῳ δὲ περὶ τῆσδε τῆς χώρας ἀκούουσαι κλέος μέγα, πολλῆς δόξης καὶ μεγάλης ἐλπίδος χάριν παραλαβοῦσαι τὰ μαχιμώτατα τῶν ἐθνῶν ἐστράτευσαν ἐπὶ τήνδε τὴν πόλιν. τυχοῦσαι δʼ ἀγαθῶν ἀνδρῶν ὁμοίας ἐκτήσαντο τὰς ψυχὰς τῇ φύσει, καὶ ἐναντίαν τὴν δόξαν τῆς προτέρας λαβοῦσαι μᾶλλον ἐκ τῶν κινδύνων ἢ ἐκ τῶν σωμάτων ἔδοξαν εἶναι γυναῖκες.

μόναις δʼ αὐταῖς οὐκ ἐξεγένετο ἐκ τῶν ἡμαρτημένων μαθούσαις περὶ τῶν λοιπῶν ἄμεινον βουλεύσασθαι, οὐδʼ οἴκαδε ἀπελθούσαις ἀπαγγεῖλαι τήν τε σφετέραν αὐτῶν δυστυχίαν καὶ τὴν τῶν ἡμετέρων προγόνων ἀρετήν· αὐτοῦ γὰρ ἀποθανοῦσαι, καὶ δοῦσαι δίκην τῆς ἀνοίας, τῆσδε μὲν τῆς πόλεως διὰ τὴν ἀρετὴν ἀθάνατον τὴν μνήμην ἐποίησαν, τὴν δὲ ἑαυτῶν πατρίδα διὰ τὴν ἐνθάδε συμφορὰν ἀνώνυμον κατέστησαν. ἐκεῖναι μὲν οὖν τῆς ἀλλοτρίας ἀδίκως ἐπιθυμήσασαι τὴν ἑαυτῶν δικαίως ἀπώλεσαν.

\end{greek}

}

\newpage

\section*{Analiza i komentar}

%1

{\large
\begin{greek}
\noindent Ἀμαζόνες γὰρ \\
\tabto{2em} Ἄρεως μὲν \\
\tabto{4em} τὸ παλαιὸν ἦσαν θυγατέρες, \\
\tabto{2em} οἰκοῦσαι δὲ \\
\tabto{4em} παρὰ τὸν Θερμώδοντα ποταμόν, \\
\tabto{2em} μόναι μὲν \\
\tabto{4em} ὡπλισμέναι σιδήρῳ \\
\tabto{6em} τῶν περὶ αὐτάς, \\
\tabto{2em} πρῶται δὲ \\
\tabto{4em} τῶν πάντων \\
\tabto{6em} ἐφʼ ἵππους ἀναβᾶσαι, \\
\tabto{8em} οἷς ἀνελπίστως \\
\tabto{8em} διʼ ἀπειρίαν τῶν ἐναντίων \\
\tabto{8em} ᾕρουν μὲν \\
\tabto{10em} τοὺς φεύγοντας, \\
\tabto{8em} ἀπέλειπον δὲ \\
\tabto{10em} διώκοντας·

ἐνομίζοντο δὲ \\
\tabto{2em} διὰ τὴν εὐψυχίαν \\
\tabto{4em} μᾶλλον ἄνδρες \\
\tabto{2em} ἢ διὰ τὴν φύσιν \\
\tabto{4em} γυναῖκες·

\tabto{2em} πλέον γὰρ ἐδόκουν \\
\tabto{4em} τῶν ἀνδρῶν \\
\tabto{6em} ταῖς ψυχαῖς διαφέρειν \\
\tabto{6em} ἢ ταῖς ἰδέαις ἐλλείπειν.\\

\end{greek}
}

\begin{description}[noitemsep]
\item[γὰρ] čestica najavljuje iznošenje objašnjenja (tvrdnje iz prethodne rečenice): naime\dots
\item[Ἄρεως μὲν\dots\ οἰκοῦσαι δὲ\dots] koordinacija rečeničnih članaka
\item[τὸ παλαιὸν] supstantivirani prilog §~373
\item[ἦσαν θυγατέρες] §~315; kopulativni glagol otvara mjesto nužnoj predikatnoj dopuni (imenski predikat, Smyth 910)
\item[οἰκοῦσαι] §~231, §~243
\item[μόναι μὲν\dots\ πρῶται δὲ\dots] koordinacija rečeničnih članova
\item[μόναι] u rečenici otvara mjesto (dijelnom) genitivu τῶν περὶ αὐτάς
\item[ὡπλισμέναι] §~267; glagolske osnove na -ιζω §~261, §~269
\item[τῶν περὶ αὐτάς] supstantivirani prijedložni izraz §~373: od onih\dots
\item[πρῶται] u rečenici otvara mjesto (dijelnom) genitivu τῶν πάντων
\item[τῶν πάντων] supstantivirani pridjev §~373
\item[ἀναβᾶσαι] §~267; složenica glagola βαίνω §~321.6
\item[οἷς] odnosna zamjenica uvodi zavisnu odnosnu rečenicu (antecedent ἵππους); dativ instrumentalni §~414: pomoću njih\dots
\item[ᾕρουν μὲν\dots\ ἀπέλειπον δὲ\dots] koordinacija rečeničnih članova
\item[ᾕρουν] §~243, §~235, §~327.1; LSJ αἱρέω II
\item[τοὺς φεύγοντας] §~231, supstantivirani particip §~373
\item[ἀπέλειπον] §~231, složenica glagola λείπω
\item[διώκοντας] §~231
\item[ἐνομίζοντο δὲ\dots] \textbf{διὰ τὴν φύσιν γυναῖκες} ratnička je hrabrost Amazonki \textgreek[variant=ancient]{(εὐψυχία)} suprotstavljena ograničenjima njihova spola \textgreek[variant=ancient]{(φύσις)}; antički su Grci žene smatrali lošijima i podređenima te ``nedovršenim muškarcima'', \textgreek[variant=ancient]{ἔτι δὲ τὸ ἄρρεν πρὸς τὸ θῆλυ φύσει τὸ μὲν κρεῖττον τὸ δὲ χεῖρον, καὶ τὸ μὲν ἄρχον τὸ δ᾽ ἀρχόμενον,} Aristotel, Politika 1, 1254b; \textgreek[variant=ancient]{τὸ γὰρ θῆλυ ὥσπερ ἄρρεν ἐστὶ πεπηρωμένον}, Aristotel, De generatione animalium, 2, 3 (737a)
\item[ἐνομίζοντο] §~231; \textit{verbum sentiendi} otvara mjesto imenskoj dopuni (ἄνδρες\dots\ γυναῖκες)
\item[μᾶλλον\dots\ ἢ\dots] koordinacija rečeničnih članova
\item[πλέον γὰρ ἐδόκουν\dots] \textbf{ἢ ταῖς ἰδέαις ἐλλείπειν} slijedi objašnjenje razloga zbog kojeg su se Amazonke činile više muškarcima nego ženama; hrabrošću \textgreek[variant=ancient]{(ταῖς ψυχαῖς)} su više nadvisivale \textgreek[variant=ancient]{(διαφέρειν)} muškarce nego što su po slabosti svojeg spola \textgreek[variant=ancient]{(ταῖς ἰδέαις)} za muškarcima zaostajale \textgreek[variant=ancient]{(ἐλλείπειν)}
\item[πλέον\dots\ ἢ\dots] koordinacija rečeničnih članova
\item[γὰρ] čestica najavljuje iznošenje objašnjenja: naime\dots
\item[ἐδόκουν] §~243; \textit{verbum sentiendi} otvara mjesto dopuni u infinitivu (διαφέρειν\dots\ ἐλλείπειν)%otvara mjesto infinitivu
\item[διαφέρειν] §~231; rekcija τινός τινι ovdje: nadvisivati nekoga nečim, LSJ διαφέρω III.4
\item[ἐλλείπειν] §~231; složenica λείπω; LSJ ἐλλείπω A.5, rekcija τινός τινι u odnosu na nekog po nečemu
\end{description}

%2
\newpage


{\large
\begin{greek}
\noindent ἄρχουσαι δὲ \\
\tabto{2em} πολλῶν ἐθνῶν, \\
καὶ ἔργῳ μὲν \\
\tabto{2em} τοὺς περὶ αὐτὰς \\
\tabto{4em} καταδεδουλωμέναι, \\
λόγῳ δὲ \\
\tabto{2em} περὶ τῆσδε τῆς χώρας \\
\tabto{4em} ἀκούουσαι κλέος μέγα, \\
πολλῆς δόξης \\
καὶ μεγάλης ἐλπίδος \\
\tabto{2em} χάριν \\
\tabto{4em} παραλαβοῦσαι \\
τὰ μαχιμώτατα \\
\tabto{2em} τῶν ἐθνῶν \\
\tabto{4em} ἐστράτευσαν \\
\tabto{6em} ἐπὶ τήνδε τὴν πόλιν.\\

\end{greek}
}

\begin{description}[noitemsep]
\item[ἄρχουσαι] §~231, rekcija τινός
\item[ἔργῳ μὲν\dots\ λόγῳ δὲ\dots] koordinacija rečeničnih članaka pomoću para čestica
\item[καταδεδουλωμέναι] §~272, složenica glagola δουλόω
\item[περὶ τῆσδε τῆς χώρας] rečeno iz pozicije govornika (Lizije), koji govori u Ateni (na groblju u Keramiku)
\item[ἀκούουσαι] §~231
\item[δόξης] LSJ δόξα A, osnovno značenje »očekivanje«, ovdje u sličnom značenju kao \textgreek[variant=ancient]{ἐλπίς}; Amazonke su imale velika očekivanja od napada na bogatu i slavnu Atenu
\item[παραλαβοῦσαι] §~254, složenica glagola λαμβάνω; LSJ παραλαμβάνω A.4
\item[ἐστράτευσαν] §~267, §~269; rekcija ἐπί τι na nešto, protiv nečega

\end{description}

%3


{\large
\begin{greek}
\noindent τυχοῦσαι δʼ \\
\tabto{2em} ἀγαθῶν ἀνδρῶν \\
\tabto{4em} ὁμοίας ἐκτήσαντο τὰς ψυχὰς \\
\tabto{6em} τῇ φύσει, \\
\tabto{4em} καὶ ἐναντίαν τὴν δόξαν \\
\tabto{6em} τῆς προτέρας \\
\tabto{4em} λαβοῦσαι \\
\tabto{6em} μᾶλλον ἐκ τῶν κινδύνων \\
\tabto{6em} ἢ ἐκ τῶν σωμάτων \\
\tabto{8em} ἔδοξαν \\
\tabto{10em} εἶναι γυναῖκες.\\

\end{greek}
}

\begin{description}[noitemsep]
\item[τυχοῦσαι] §~254; §~321.19; rekcija (objekta) τινός
\item[ἀγαθῶν ἀνδρῶν] ἀγαθός ovdje u značenju: hrabar
\item[ὁμοίας ἐκτήσαντο] \textbf{τὰς ψυχὰς τῇ φύσει} u skladu s gore rečenim, u sukobu s junačkim Atenjanima hrabrost \textgreek[variant=ancient]{(αἱ ψυχαί)} Amazonki postala je jednaka njihovoj (manje vrijednoj) prirodi \textgreek[variant=ancient]{(τῇ φύσει)}
\item[ὁμοίας] rekcija τινί
\item[ἐκτήσαντο] §~267, §~269
\item[ἐναντίαν] rekcija τινός
\item[λαβοῦσαι] §~254; §~321.14
\item[τῶν κινδύνων] κίνδυνος rizik (koji može uključivati i izlaganje ratnoj sreći u bici), usp. LSJ s.~v.; moguće je također i da autor smatra da su se Amazonke pokazale ženama jer su \textit{pogrešno} procijenile i nešto nerealno očekivale
\item[ἔδοξαν] §~267; §~325.2; kao \textit{verbum sentiendi}, glagol otvara u rečenici mjesto dopuni, ovdje je to infinitiv
\item[εἶναι γυναῖκες] §~315; kopulativni glagol otvara mjesto nužnoj predikatnoj dopuni (imenski predikat, Smyth 910)

\end{description}

%4


{\large
\begin{greek}
\noindent μόναις δʼ αὐταῖς \\
οὐκ ἐξεγένετο \\
\tabto{2em} ἐκ τῶν ἡμαρτημένων μαθούσαις \\
\tabto{2em} περὶ τῶν λοιπῶν \\
\tabto{4em} ἄμεινον βουλεύσασθαι, \\
\tabto{2em} οὐδʼ οἴκαδε ἀπελθούσαις \\
\tabto{4em} ἀπαγγεῖλαι \\
\tabto{6em} τήν τε σφετέραν αὐτῶν δυστυχίαν \\
\tabto{6em} καὶ τὴν τῶν ἡμετέρων προγόνων ἀρετήν·

αὐτοῦ γὰρ ἀποθανοῦσαι, \\
καὶ δοῦσαι δίκην \\
\tabto{2em} τῆς ἀνοίας, \\
τῆσδε μὲν τῆς πόλεως \\
\tabto{2em} διὰ τὴν ἀρετὴν \\
\tabto{4em} ἀθάνατον τὴν μνήμην ἐποίησαν, \\
τὴν δὲ ἑαυτῶν πατρίδα \\
\tabto{2em} διὰ τὴν ἐνθάδε συμφορὰν \\
\tabto{4em} ἀνώνυμον κατέστησαν.\\

\end{greek}
}

\begin{description}[noitemsep]
\item[ἐξεγένετο] §~254; bezlično, složenica γίγνομαι, LSJ ἐκγίγνομαι III; glagol u rečenici otvara mjesto dativu osobe (τινί) i infinitivu (ovdje ih ima više)
\item[τῶν ἡμαρτημένων] §~272; supstantivirani particip (s.~r.) §~373; §~321.10
\item[μαθούσαις] §~254; §~321.17
\item[βουλεύσασθαι] §~267, §~269, LSJ βουλεύω B.1
\item[ἀπελθούσαις] §~254, složenica glagola ἔρχομαι
\item[ἀπαγγεῖλαι] §~267, složenica glagola ἀγγέλλω, s. 118
\item[δυστυχίαν\dots\ ἀρετήν] antiteza
\item[τήν τε\dots\ καὶ τὴν] koordinacija sastavnih veznika, pri čemu je drugi član para naglašeniji
\item[αὐτοῦ] prilog mjesta, LSJ s.\ v.; sc.\ na mjestu bitke
\item[γὰρ] čestica najavljuje iznošenje objašnjenja: jer\dots
\item[ἀποθανοῦσαι] §~254; §~324.8
\item[δοῦσαι δίκην] §~306; δίκας διδόναι τινός LSJ δίκη IV.3
\item[τῆσδε μὲν\dots\ τὴν δὲ\dots] koordinacija rečeničnih članova
\item[ἀθάνατον\dots\ ἐποίησαν] §~267, §~269; ποιῆσαί τινα s pridjevom LSJ ποιέω III; \textbf{\textgreek[variant=ancient]{ἀθάνατον τὴν μνήμην\dots\ ἀνώνυμον}} antiteza
\item[τὴν ἐνθάδε συμφορὰν] prilog u atributnom položaju §~375.5
\item[ἀνώνυμον κατέστησαν] §~267, glagol je složenica ἵστημι; rekcija τινά s pridjevom LSJ καθίστημι II.4

\end{description}


{\large
\begin{greek}
\noindent ἐκεῖναι μὲν οὖν \\
\tabto{2em} τῆς ἀλλοτρίας \\
\tabto{4em} ἀδίκως ἐπιθυμήσασαι \\
\tabto{2em} τὴν ἑαυτῶν \\
\tabto{4em} δικαίως ἀπώλεσαν.\\

\end{greek}
}

\begin{description}[noitemsep]
\item[τῆς ἀλλοτρίας] sc.\ χώρας
\item[ἐπιθυμήσασαι] §~267, §~269, rekcija τινός
\item[ἀδίκως\dots\ δικαίως] antiteza
\item[ἀπώλεσαν] §~267, §~319.15; LSJ ἀπόλλυμι II
\end{description}

%kraj
