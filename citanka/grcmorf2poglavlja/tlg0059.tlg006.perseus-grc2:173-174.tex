% Unio ispravke NZ <2022-01-02 ned>

%TKTK


\section*{O tekstu}

\textit{Teetet} je dijalog iz srednje ili kasnije faze Platonova stvaralaštva. Smatra se da je napisan oko 369.\ pr.~Kr. te da je posvećen uspomeni na Teeteta (koji je oko te godine umro).
 
U središtu razmatranja nalazi se pitanje što je znanje, ali se – kao i u drugim dijalozima – spominju i ostale teme koje su zaokupljale Platona: retorički i filozofski život, Protagorin relativizam, heraklitska promjena, sokratska dijalektika. Dijalog i danas postavlja pred filozofe i komentatore neke probleme u interpretaciji (npr. kako se slika Sokrata kao filozofa kojeg ne zanimaju svakodnevni poslovi grada odnosi prema Sokratu iz drugih dijaloga, gdje je mu je kao filozofu izrazito važna dobrobit njegovih sugrađana). 

Sokrat razgovara s dva matematičara, mladim Teetetom i njegovim učiteljem matematike Teodorom. U uvodu dijaloga, pak, razgovaraju Euklid, filozof iz Megare, te njegov prijatelj i Sokratov učenik Terpsion; Teetet je blizu smrti, govore, te se Euklid sjeća kako je jednom čuo i zapisao njegov razgovor sa Sokratom. Onda Euklid i Terpsion slušaju roba koji im čita Euklidov zapis; ovo je jedini platonski dijalog koji čita rob.

Iz razrade glavne teme, o naravi znanja (ἐπιστήμη), jasno se vidi što je Sokratova majeutika. Sokrat, naime, prepoznaje misaonu ``trudnoću'' i zatim osobi pomaže da započne misaoni ``porod''. Kada se Teetet zaplete u objašnjenjima, Sokrat mu, poput primalje, asistira, a pokazuje i jesu li tako rođene ideje ``zakonita djeca''. No Sokrat nije samo primalja, već i, poput zemljoradnika, zna koja je zemlja najbolja za koje sjeme te kada je najbolje vrijeme za žetvu.

Potaknut majeutičkom pomoći Teetet iznosi drugu definiciju znanja: znanje je percepcija. Sokrat tu misao povezuje s Protagorinim glasovitim riječima o čovjeku kao mjeri stvari (relativizam). Razmatrajući Protagorin stav na ontološkoj razini, Sokrat u njemu otkriva heraklitovsku konstantnu promjenu i sve podvrgava detaljnom kritičkom razmatranju. Nakon što se dokazalo da znanje nije percepcija, dijalog kritički razmatra treću definiciju znanja po kojoj je ono ``opravdano istinito vjerovanje''. Uz pomoć triju određenja pojma \textit{logos} Sokrat zaključuje da znanje ne odgovara ni četvrtoj razmotrenoj definiciji. Teetet ostaje bez ideja, dijalog završava bez novijeg ili točnijeg određenja znanja. Ipak, Sokrat smatra da je razgovor bio važan jer se Teetet lišio nekoliko krivih uvjerenja; zbog toga će njegove daljnje teorije, bude li ih bilo, biti bolje.

Ovaj Platonov dijalog ostavio je znatan utjecaj na suvremenu filozofiju, od Descartesa (Sokrat o ludilu u raspravi o znanju kao percepciji) do Wittgensteina (u nekim tezama iz djela \textit{Tractatus logico-philosophicus}).

Dio koji čitamo pripada Sokratovoj digresiji na prijelazu iz drugog u treći argument protiv Protagorine doktrine (172a-177c). Sudstvo i filozofija dvije su domene, kaže Sokrat. Stručnjaci za pravna pitanja slijepo prate pravne odredbe, no um filozofa nije ograničen prostorom ili vremenom te se može baviti razmatranjem biti pravde. Ona je za sud praktična, određena vremenom, no za filozofa je ona bezvremena, misaona, božanska. 

Filozof kakvog nam Sokrat dočarava u ovom dijelu dijaloga jest netko koga ne zanimaju svakodnevni poslovi grada. Filozofa se ne tiče ni gdje su glavna javna mjesta Atene niti tko je kojeg porijekla niti koliko je tko bogat. To nije samo njihova poza, objašnjava Sokrat; filozofi su u gradu samo tijelom, dok se njihov um slobodno kreće na drugim razinama, ispod zemlje i iznad zvijezda, nikada se ne baveći onime što je na dohvat ruke.


\newpage

\section*{Pročitajte naglas grčki tekst.}

Plat.\ Theaetetus 173c-174a

%Naslov prema izdanju

\medskip


{\large

\begin{greek}

\noindent ΣΩ. Λέγωμεν δή, ὡς ἔοικεν, ἐπεὶ σοί γε δοκεῖ, περὶ τῶν κορυφαίων· τί γὰρ ἄν τις τούς γε φαύλως διατρίβοντας ἐν φιλοσοφίᾳ λέγοι; οὗτοι δέ που ἐκ νέων πρῶτον μὲν εἰς ἀγορὰν οὐκ ἴσασι τὴν ὁδόν, οὐδὲ ὅπου δικαστήριον ἢ βουλευτήριον ἤ τι κοινὸν ἄλλο τῆς πόλεως συνέδριον· νόμους δὲ καὶ ψηφίσματα λεγόμενα ἢ γεγραμμένα οὔτε ὁρῶσιν οὔτε ἀκούουσι· σπουδαὶ δὲ ἑταιριῶν ἐπʼ ἀρχὰς καὶ σύνοδοι καὶ δεῖπνα καὶ σὺν αὐλητρίσι κῶμοι, οὐδὲ ὄναρ πράττειν προσίσταται αὐτοῖς. εὖ δὲ ἢ κακῶς τις γέγονεν ἐν πόλει, ἤ τί τῳ κακόν ἐστιν ἐκ προγόνων γεγονὸς ἢ πρὸς ἀνδρῶν ἢ γυναικῶν, μᾶλλον αὐτὸν λέληθεν ἢ οἱ τῆς θαλάττης λεγόμενοι χόες. καὶ ταῦτα πάντʼ οὐδʼ ὅτι οὐκ οἶδεν, οἶδεν· οὐδὲ γὰρ αὐτῶν ἀπέχεται τοῦ εὐδοκιμεῖν χάριν, ἀλλὰ τῷ ὄντι τὸ σῶμα μόνον ἐν τῇ πόλει κεῖται αὐτοῦ καὶ ἐπιδημεῖ, ἡ δὲ διάνοια, ταῦτα πάντα ἡγησαμένη σμικρὰ καὶ οὐδέν, ἀτιμάσασα πανταχῇ πέτεται κατὰ Πίνδαρον ``τᾶς τε γᾶς ὑπένερθε'' καὶ τὰ ἐπίπεδα γεωμετροῦσα, ``οὐρανοῦ θʼ ὕπερ'' ἀστρονομοῦσα, καὶ πᾶσαν πάντῃ φύσιν ἐρευνωμένη τῶν ὄντων ἑκάστου ὅλου, εἰς τῶν ἐγγὺς οὐδὲν αὑτὴν συγκαθιεῖσα.

\end{greek}

}


\section*{Analiza i komentar}

%1

{\large
\begin{greek}
  \noindent  ΣΩ. Λέγωμεν δή, ὡς ἔοικεν, ἐπεὶ σοί γε δοκεῖ, \\
  περὶ τῶν κορυφαίων· \\
  τί γὰρ ἄν τις \\
τούς γε φαύλως διατρίβοντας \\
\tabto{2em} ἐν φιλοσοφίᾳ \\
λέγοι;\\

\end{greek}
}

\begin{description}[noitemsep]
\item[Λέγωμεν] \textit{coniunctivus adhortativus} §~463.1
\item[ὡς ἔοικεν] LSJ ἔοικα II.2
\item[ἐπεὶ σοί γε δοκεῖ] LSJ δοκέω II.3
\item[περὶ τῶν κορυφαίων] κορυφαῖοι: glavni ili vodeći filozofi
\item[τί\dots\ λέγοι] upitna zamjenica (v.\ naglasak!) otvara upitnu rečnicu (direktno pitanje)
\item[γὰρ] ovdje u eksplanatornom značenju: naime\dots
\item[ἄν λέγοι] §~231, vremenske osnove §~327.7
\item[γε] čestica ističe značenje riječi na koju se odnosi
\item[τούς\dots\ διατρίβοντας] §~231; rekcija: διατρίβω ἔν τινι; složenica glagola τρίβω; supstantivirani particip §~499.2
\item[φαύλως] adverbna dopuna participu

\end{description}

%2

{\large
\begin{greek}
\noindent οὗτοι δέ που \\
\tabto{2em} ἐκ νέων \\
πρῶτον μὲν \\
\tabto{2em} εἰς ἀγορὰν \\
οὐκ ἴσασι \\
\tabto{2em} τὴν ὁδόν, \\
οὐδὲ ὅπου \\
\tabto{2em} δικαστήριον ἢ βουλευτήριον \\
\tabto{2em} ἤ τι κοινὸν ἄλλο \\
\tabto{4em} τῆς πόλεως \\
\tabto{2em} συνέδριον· \\
νόμους δὲ \\
καὶ ψηφίσματα \\
\tabto{4em} λεγόμενα ἢ γεγραμμένα \\
\tabto{2em} οὔτε ὁρῶσιν \\
\tabto{2em} οὔτε ἀκούουσι· \\
σπουδαὶ δὲ \\
\tabto{2em} ἑταιριῶν \\
\tabto{4em} ἐπ' ἀρχὰς \\
καὶ σύνοδοι καὶ δεῖπνα \\
καὶ σὺν αὐλητρίσι κῶμοι, \\
οὐδὲ ὄναρ πράττειν \\
\tabto{2em} προσίσταται \\
\tabto{4em} αὐτοῖς.\\

\end{greek}
}

\begin{description}[noitemsep]
\item[οὗτοι δέ] čestica δέ ponavlja se kao veznik za koordinirano redanje misli, označava nadovezivanje na prethodni iskaz; οὗτοι su οἱ κορυφαῖοι koje Sokrat spominje u prethodnoj rečenici prije
\item[ἐκ νέων] ``od djetinjstva''
\item[πρῶτον μὲν\dots\ νόμους δὲ\dots] koordinacija rečeničnih članova
\item[οὐκ ἴσασι] §~317.4
\item[ἢ\dots\ ἢ\dots] koordinacija pomoću rastavnih veznika: ili\dots\ ili\dots
\item[λεγόμενα ἢ γεγραμμένα] atributni particip
\item[οὔτε\dots\ οὔτε\dots ] sastavni veznici koordiniraju surečenice: niti\dots\ niti\dots
\item[ὁρῶσιν] §~243, osnove §~327.3
\item[ἀκούουσι] ἀκούω τινα slušati što; §~231, osnove s.~116
\item[καὶ\dots\ καὶ\dots\ καὶ\dots] koordinacija surečenica uz pomoć sastavnog veznika καί (nabrajanje)
\item[ὄναρ] adverbno; LSJ s. v.: \textit{kao prilog} u snu, \textit{pa} οὐδʼ ὄ. \textit{ni u snu}
\item[πράττειν] §~231, osnove s.~116
\item[προσίσταται] §~305, osnove §~306, §~311.2, složenica glagola ἵστημι; za rekciju προσίσταταί τινι v.\ LSJ προσίστημι, II.2

\end{description}

%3

{\large
\begin{greek}
\noindent εὖ δὲ ἢ κακῶς \\
\tabto{2em} τις γέγονεν \\
\tabto{4em} ἐν πόλει, \\
ἤ τί τῳ κακόν ἐστιν \\
\tabto{2em} ἐκ προγόνων \\
\tabto{4em} γεγονὸς \\
ἢ πρὸς ἀνδρῶν \\
ἢ γυναικῶν, \\
\tabto{2em} μᾶλλον αὐτὸν λέληθεν \\
\tabto{2em} ἢ οἱ \\
\tabto{4em} τῆς θαλάττης \\
\tabto{2em} λεγόμενοι \\
\tabto{4em} χόες.\\

\end{greek}
}

\begin{description}[noitemsep]
\item[γέγονεν] §~272, osnove §~325.11
\item[ἐστιν γεγονὸς] §~272; particip perfekta kombinira se s glagolom εἰμί u perifrastični (opisni oblik), posebno ako particip može preuzeti funkciju predikatnog pridjeva
\item[λέληθεν] §~272, osnove §~321.15
\item[οἱ λεγόμενοι] §~231, atributni particip §~499
\item[χόες] LSJ χοῦς I.2 \textit{poslovično o pokušajima da se izmjeri neizmjerivo} οἱ τῆς θαλάττης λεγόμενοι χόες Pl. Tht. 173d (referira se upravo na ovo mjesto u \textit{Teetetu})
\end{description}

%4

{\large
\begin{greek}
\noindent καὶ ταῦτα πάντ' \\
\tabto{2em} οὐδ' \\
\tabto{4em} ὅτι οὐκ οἶδεν, \\
\tabto{2em} οἶδεν· \\
\tabto{2em} οὐδὲ γὰρ \\
\tabto{4em} αὐτῶν ἀπέχεται \\
\tabto{6em} τοῦ εὐδοκιμεῖν χάριν, \\
\tabto{2em} ἀλλὰ \\
\tabto{2em} τῷ ὄντι \\
\tabto{2em} τὸ σῶμα μόνον \\
\tabto{4em} ἐν τῇ πόλει \\
\tabto{2em} κεῖται \\
\tabto{4em} αὐτοῦ \\
\tabto{2em} καὶ ἐπιδημεῖ, \\
\tabto{2em} ἡ δὲ διάνοια, \\
\tabto{4em} ταῦτα πάντα \\
\tabto{2em} ἡγησαμένη \\
\tabto{4em} σμικρὰ καὶ οὐδέν, \\
\tabto{2em} ἀτιμάσασα πανταχῇ \\
\tabto{2em} πέτεται \\
\tabto{4em} κατὰ Πίνδαρον \\
\tabto{2em} ``τᾶς τε γᾶς ὑπένερθε''\\
\tabto{2em} καὶ τὰ ἐπίπεδα \\
\tabto{4em} γεωμετροῦσα, \\
\tabto{2em} ``οὐρανοῦ θ' ὕπερ''\\
\tabto{4em} ἀστρονομοῦσα, \\
\tabto{2em} καὶ πᾶσαν πάντῃ φύσιν \\
\tabto{4em} ἐρευνωμένη \\
\tabto{6em} τῶν ὄντων \\
\tabto{8em} ἑκάστου ὅλου, \\
\tabto{2em} εἰς τῶν ἐγγὺς \\
\tabto{4em} οὐδὲν \\
\tabto{4em} αὑτὴν \\
\tabto{4em} συγκαθιεῖσα.\\

\end{greek}
}

\begin{description}[noitemsep]
\item[ὅτι οὐκ οἶδεν] izrična rečenica: da\dots
\item[οἶδεν] §~317.4; glagol otvara mjesto dopuni u objektu, ovdje u obliku izrične rečenice
\item[οὐδὲ γὰρ\dots\ ἀλλὰ\dots\ ἡ δὲ\dots] koordinacija surečenica: čestica γὰρ eksplanatorno: jer\dots; suprotni veznik: nego\dots; čestica δὲ adverzativno: a\dots
\item[ἀπέχεται] §~231, složenica glagola ἔχω
\item[τοῦ εὐδοκιμεῖν] §~243, supstantivirani infinitiv §~497
\item[τῷ ὄντι] §~315.2, supstantivirani particip §~499, upotrijebljen kao priložna oznaka ``zapravo'', v. DGE / Logeion εἰμί; LSJ εἰμί A. III τῶι ὄντι, \textit{Lat.} revera, zapravo, Plat. 
\item[κεῖται] §~315.4
\item[ἐπιδημεῖ] §~243
\item[ἡγησαμένη] §~267, glagol otvara mjesto dvjema dopunama u akuzativu, ταῦτα πάντα i σμικρὰ καὶ οὐδέν, akuzativ objekta i predikata §~388
\item[ἀτιμάσασα] §~267
\item[πέτεται] §~231, osnove §~325.20
\item[``τᾶς τε γᾶς ὑπένερθε''] Platon citira Pindarove stihove, fr. 292 u Snellovu izdanju
\item[γεωμετροῦσα] §~243
\item[ἀστρονομοῦσα] §~243
\item[ἐρευνωμένη] §~243
\item[τῶν ὄντων] §~315.2
\item[συγκαθιεῖσα] §~305, složenica glagola ἵημι, συγκαθίημι ἐμαυτόν spustiti se dolje

\end{description}


%kraj
