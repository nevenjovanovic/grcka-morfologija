% Unesi ispravke NZ <2021-12-30 čet>
\section*{O autoru}

Ἱπποκράτης ὁ Κῷος, Hipokrat s Kosa (oko 460.\ – oko 370.\ pr.~Kr.) najznamenitiji je grčki liječnik, jedna od najvažnijih ličnosti povijesti medicine; često se naziva i ocem medicine. Smatra se utemeljiteljem škole koja je liječništvo utemeljila kao zasebnu znanstvenu disciplinu, odvojivši je s jedne strane od filozofije i s druge strane od magijskih i nadriliječničkih praksi. 

Hipokratovo učenje sačuvano je u spisima zbirke \textit{Corpus Hippocraticum}, šezdesetak djela uglavnom nepoznatih autora; autorstvo samoga Hipokrata nije dokazano ni za jedan spis. Zbirka okuplja djela takozvane koške škole, središta stjecanja i prakticiranja medicinskog znanja s otoka Kosa, a nastala je u drugoj polovici V.~st.\ pr.~Kr.

Tekstovi hipokratskog korpusa pripadaju jonskoj prozi; u počecima formiranja proznog izraza nekoliko značajnih medicinskih autora bilo je porijeklom iz jonske Grčke, te je njihov dijalekt postao kanonskim za medicinske tekstove. 

\section*{O tekstu}

Odlomak koji donosimo dio je spisa \textgreek[variant=ancient]{Περὶ ἱερῆς νούσου,} \textit{O svetoj bolesti}, posvećenog epilepsiji (ili nizu neuroloških poremećaja s epilepsiji sličnim simptomima) za koju se u starini smatralo da je uzrokuju i ljudima daju božanstva. Osporavajući takva poimanja hipokratski autor već na početku iznosi stav da je epilepsija božanska utoliko koliko su i sve ostale bolesti, uzrokovane prirodnim elementima, božanske. Sveta se bolest prema hipokratskom autoru može liječiti, ali to nikako ne smije biti uz pomoć magijskih ili drugih nemedicinskih, odnosno neznanstvenih postupaka.

Od obilježja jonskog dijalekta, uz uobičajeno jonsko eta ovdje ćemo naići i na nestegnute oblike \textgreek[variant=ancient]{(καλεομένης, θεοσεβέες, δοκέουσιν\dots),} na kapa refleks umjesto pi oblika \textgreek[variant=ancient]{(ὅκως} umjesto \textgreek[variant=ancient]{ὅπως),} na jonske oblike deklinacije \textgreek[variant=ancient]{(προφάσιος)} i drugo \textgreek[variant=ancient]{(νούσημα} umjesto \textgreek[variant=ancient]{νόσημα).}

%\newpage

\section*{Pročitajte naglas grčki tekst.}

Hippoc.\ De morbo sacro 1

%Naslov prema izdanju

\medskip


{\large

\begin{greek}

\noindent Περὶ μὲν τῆς ἱερῆς νούσου καλεομένης ὧδ’ ἔχει· οὐδέν τί μοι δοκέει τῶν ἄλλων θειοτέρη εἶναι νούσων οὐδὲ ἱερωτέρη, ἀλλὰ φύσιν μὲν ἔχει ἣν καὶ τὰ λοιπὰ νουσήματα, ὅθεν γίνεται. Φύσιν δὲ αὐτῇ καὶ πρόφασιν οἱ ἄνθρωποι ἐνόμισαν θεῖόν τι πρῆγμα εἶναι ὑπὸ ἀπειρίης καὶ θαυμασιότητος, ὅτι οὐδὲν ἔοικεν ἑτέρῃσι νούσοισιν· καὶ κατὰ μὲν τὴν ἀπορίην αὐτοῖσι τοῦ μὴ γινώσκειν τὸ θεῖον αὐτῇ διασώζεται, κατὰ δὲ τὴν εὐπορίην τοῦ τρόπου τῆς ἰήσιος ᾧ ἰῶνται, ἀπόλλυται, ὅτι καθαρμοῖσί τε ἰῶνται καὶ ἐπαοιδῇσιν. Εἰ δὲ διὰ τὸ θαυμάσιον θεῖον νομιεῖται, πολλὰ τὰ ἱερὰ νουσήματα ἔσται καὶ οὐχὶ ἓν, ὡς ἐγὼ ἀποδείξω ἕτερα οὐδὲν ἧσσον ἐόντα θαυμάσια οὐδὲ τερατώδεα, ἃ οὐδεὶς νομίζει ἱερὰ εἶναι. Τοῦτο μὲν γὰρ οἱ πυρετοὶ οἷ ἀμφημερινοὶ καὶ οἱ τριταῖοι καὶ οἱ τεταρταῖοι οὐδὲν ἧσσόν μοι δοκέουσιν ἱεροὶ εἶναι καὶ ὑπὸ θεοῦ γίνεσθαι ταύτης τῆς νούσου, ὧν οὐ θαυμασίως γ’ ἔχουσιν· τοῦτο δὲ ὁρέω μαινομένους ἀνθρώπους καὶ παραφρονέοντας ἀπὸ μηδεμιῆς προφάσιος ἐμφανέος, καὶ πολλά τε καὶ ἄκαιρα ποιέοντας, ἔν τε τῷ ὕπνῳ οἶδα πολλοὺς οἰμώζοντας καὶ βοῶντας, τοὺς δὲ πνιγομένους, τοὺς δὲ καὶ ἀναΐσσοντάς τε καὶ φεύγοντας ἔξω καὶ παραφρονέοντας μέχρις ἂν ἐπέγρωνται, ἔπειτα δὲ ὑγιέας ἐόντας καὶ φρονέοντας ὥσπερ καὶ πρότερον, ἐόντας τ’ αὐτέους ὠχρούς τε καὶ ἀσθενέας, καὶ ταῦτα οὐχ ἅπαξ, ἀλλὰ πολλάκις, ἄλλα τε πολλά ἐστι καὶ παντοδαπὰ ὧν περὶ ἑκάστου λέγειν πουλὺς ἂν εἴη λόγος. Ἐμοὶ δὲ δοκέουσιν οἱ πρῶτοι τοῦτο τὸ νόσημα ἀφιερώσαντες τοιοῦτοι εἶναι ἄνθρωποι οἷοι καὶ νῦν εἰσι μάγοι τε καὶ καθάρται καὶ ἀγύρται καὶ ἀλαζόνες, ὁκόσοι δὴ προσποιέονται σφόδρα θεοσεβέες εἶναι καὶ πλέον τι εἰδέναι.

\end{greek}

}


\section*{Analiza i komentar}

%1

{\large
\begin{greek}
\noindent Περὶ μὲν τῆς ἱερῆς νούσου καλεομένης \\
ὧδ' ἔχει· \\
οὐδέν τί μοι δοκέει \\
\tabto{2em} \tabto{2em} τῶν ἄλλων \\
\tabto{2em} θειοτέρη εἶναι \\
\tabto{2em} \tabto{2em} νούσων \\
\tabto{2em} οὐδὲ ἱερωτέρη, \\
ἀλλὰ φύσιν μὲν ἔχει \\
\tabto{2em} ἣν καὶ τὰ λοιπὰ νουσήματα, \\
\tabto{2em} ὅθεν γίνεται.\\

\end{greek}
}

\begin{description}[noitemsep]
\item[καλεομένης] §~243, jonski (nestegnuti) oblik umjesto καλουμένης
\item[ἔχει] §~231
\item[ὧδ' ἔχει] ἔχω + adverb: biti\dots + adverb ili pridjev
\item[μοι δοκέει] §~243, jonski umjesto δοκεῖ; glagol otvara mjesto nominativu s infinitivom §~491.2
\item[εἶναι] §~315
\item[θειοτέρη\dots\ ἱερωτέρη εἶναι] nominativ s infinitivom §~491.2
\item[ἔχει] §~231
\item[ἣν] odnosna zamjenica ἥν uvodi zavisnu odnosnu rečenicu, odnosi se na riječ φύσιν; glagol je neizrečen jer je isti kao i u glavnoj rečenici (ἔχει)
\item[γίνεται] §~231 (osnove §~325.11), jonski oblik umjesto γίγνομαι
\item[ὅθεν γίνεται] odnosni prilog ὅθεν uvodi zavisnu odnosnu rečenicu: odakle, iz čega

\end{description}

%2

{\large
\begin{greek}
\noindent Φύσιν δὲ αὐτῇ \\
καὶ πρόφασιν \\
οἱ ἄνθρωποι ἐνόμισαν \\
\tabto{2em} θεῖόν τι πρῆγμα εἶναι \\
ὑπὸ ἀπειρίης καὶ θαυμασιότητος, \\
ὅτι οὐδὲν ἔοικεν \\
\tabto{2em} ἑτέρῃσι νούσοισιν· \\
καὶ \\
κατὰ μὲν τὴν ἀπορίην \\
αὐτοῖσι \\
\tabto{2em} τοῦ μὴ γινώσκειν \\
τὸ θεῖον \\
\tabto{2em} αὐτῇ \\
\tabto{2em} \tabto{2em} διασώζεται, \\
κατὰ δὲ τὴν εὐπορίην \\
\tabto{2em} τοῦ τρόπου \\
\tabto{2em} \tabto{2em} τῆς ἰήσιος \\
\tabto{2em} ᾧ ἰῶνται, \\
ἀπόλλυται,\\
ὅτι καθαρμοῖσί τε \\
\tabto{2em} ἰῶνται \\
καὶ ἐπαοιδῇσιν.\\

\end{greek}
}

\begin{description}[noitemsep]
\item[ἐνόμισαν] §~267, \textit{verbum sentiendi} otvara mjesto akuzativu s infinitivom
\item[εἶναι] §~315
\item[θεῖόν τι πρῆγμα εἶναι] akuzativ s infinitivom kao dopuna glavnog glagola ἐνόμισαν
\item[θαυμασιότητος] LSJ θαυμασιότης, ητος, ἡ, ``sklonost čudesnome, sklonost da se što smatra čudom''; autor potvrdu za ovo rijetko značenje nalazi upravo u našem odlomku
\item[ἔοικεν] rekcija: τινι; §~278 bilješka
\item[ὅτι… ἔοικεν] uzročni veznik ὅτι uvodi zavisnu uzročnu rečenicu
\item[κατὰ μὲν… κατὰ δὲ ] priložne oznake načina uz διασώζεται koordinirane su česticama μὲν… δὲ: ``s jedne strane… s druge strane…''
\item[τοῦ μὴ γινώσκειν] §~231, jonski oblik umjesto γιγνώσκειν (osnove §~324.12), supstantivirani infinitiv §~497
\item[τὸ θεῖον αὐτῇ διασώζεται] τὸ θεῖον pojam božanskog (i subjekt u rečenici); αὐτῇ sc.\ τῇ νούσῳ \textit{(dativus commodi)}
\item[διασώζεται] ``čuva se, spašava se'' (predikat uz τὸ θεῖον); §~231 (osnove §~301.B s.~116)
\item[ἰῶνται] §~243
\item[ᾧ ἰῶνται] odnosna zamjenica ᾧ uvodi zavisnu odnosnu rečenicu; antecedent zamjenice je τοῦ τρόπου
\item[ἀπόλλυται] §~ 418 (osnove §~319.15)
\item[ἰῶνται] §~243
\item[ὅτι… ἰῶνται] uzročni veznik ὅτι uvodi zavisnu uzročnu rečenicu

\end{description}

%3


{\large
\begin{greek}
\noindent Εἰ δὲ \\
\tabto{2em} διὰ τὸ θαυμάσιον \\
θεῖον νομιεῖται, \\
πολλὰ \\
\tabto{2em} τὰ ἱερὰ νουσήματα \\
ἔσται \\
καὶ οὐχὶ ἓν, \\
ὡς ἐγὼ ἀποδείξω \\
ἕτερα οὐδὲν ἧσσον ἐόντα \\
\tabto{2em} θαυμάσια οὐδὲ τερατώδεα, \\
\tabto{2em} ἃ οὐδεὶς νομίζει \\
\tabto{2em} \tabto{2em} ἱερὰ εἶναι.\\
        
\end{greek}
}

\begin{description}[noitemsep]
\item[νομιεῖται] §~258, atički futur §~263.2
\item[Εἰ… νομιεῖται] pogodbeni veznik εἰ uvodi zavisnu realnu pogodbenu rečenicu (stvarni uvjet ispunjenja radnje)
\item[ἔσται] §~315
\item[πολλὰ ἔσται] imenski predikat
\item[ἀποδείξω] §~ 318, glagol otvara mjesto dopuni u akuzativu
\item[ὡς ἐγὼ ἀποδείξω] načinski prilog ὡς uvodi zavisnu rečenicu
\item[ἐόντα] §~315, jonski oblik umjesto ὄντα, predikatni particip: da…
\item[ἕτερα οὐδὲν ἧσσον… θαυμάσια οὐδὲ τερατώδεα] imenska dopuna predikatnom participu u akuzativu
\item[νομίζει] §~231, \textit{verbum sentiendi} otvara mjesto dopuni u akuzativu s infinitivom: da…
\item[ἃ… νομίζει] odnosna zamjenica ἃ uvodi zavisnu odnosnu rečenicu, odnosi se na ἕτερα (νουσήματα)
\item[ἱερὰ εἶναι] §~315; akuzativ s infinitivom

\end{description}

%4


{\large
\begin{greek}
\noindent Τοῦτο μὲν γὰρ \\
οἱ πυρετοὶ οἱ ἀμφημερινοὶ \\
\tabto{2em} καὶ οἱ τριταῖοι \\
\tabto{2em} καὶ οἱ τεταρταῖοι \\
οὐδὲν ἧσσόν μοι δοκέουσιν \\
\tabto{2em} ἱεροὶ εἶναι \\
καὶ ὑπὸ θεοῦ γίνεσθαι \\
\tabto{2em} ταύτης τῆς νούσου, \\
ὧν οὐ θαυμασίως γ' ἔχουσιν· \\
τοῦτο δὲ \\
ὁρέω \\
μαινομένους ἀνθρώπους \\
καὶ παραφρονέοντας \\
\tabto{2em} ἀπὸ μηδεμιῆς προφάσιος ἐμφανέος, \\
καὶ πολλά τε \\
καὶ ἄκαιρα \\
\tabto{2em} ποιέοντας, \\
ἔν τε τῷ ὕπνῳ \\
οἶδα \\
\tabto{2em} πολλοὺς οἰμώζοντας καὶ βοῶντας, \\
\tabto{2em} τοὺς δὲ πνιγομένους, \\
\tabto{2em} τοὺς δὲ καὶ ἀναΐσσοντάς τε\\
\tabto{2em} καὶ φεύγοντας ἔξω \\
\tabto{2em} καὶ παραφρονέοντας \\
\tabto{2em} \tabto{2em} μέχρις ἂν ἐπέγρωνται, \\
ἔπειτα δὲ ὑγιέας ἐόντας \\
καὶ φρονέοντας \\
\tabto{2em} ὥσπερ καὶ πρότερον, \\
ἐόντας τ' αὐτέους \\
\tabto{2em} ὠχρούς τε καὶ ἀσθενέας, \\
καὶ ταῦτα \\
\tabto{2em} οὐχ ἅπαξ, ἀλλὰ πολλάκις,\\
ἄλλα τε πολλά ἐστι καὶ παντοδαπὰ\\
\tabto{2em} ὧν περὶ ἑκάστου \\
\tabto{2em} \tabto{2em} λέγειν \\
\tabto{2em} \tabto{2em} πουλὺς ἂν εἴη λόγος.\\

\end{greek}
}

\begin{description}[noitemsep]
\item[μοι δοκέουσιν] §~243, jonski oblik umjesto δοκοῦσιν, otvara mjesto nominativu s infinitivom §~491.2: da…
\item[οὐδὲν ἧσσόν ἱεροὶ εἶναι] nominativ s infinitivom §~491.2
\item[Τοῦτο μὲν… τοῦτο δὲ ] paralelizam koordiniran česticama μὲν i δὲ: prvo…, a drugo…, dodatno ističu i promjenu subjekta
\item[γίνεσθαι] infinitiv iz drugog nominativa s infinitivom koji ovisi o glavnom glagolu, dopuna tog infinitiva je \textgreek{ὑπὸ θεοῦ,} prijedložni izraz uzročnog značenja (``pod utjecajem, zbog'')
\item[ἔχουσιν] §~231 (osnove §~327. 13)
\item[ὧν… ἔχουσιν] odnosna zamjenica ὧν uvodi zavisnu odnosnu rečenicu, antecedenti su nominativi iz glavne rečenice
\item[θαυμασίως] LSJ θαυμάσιος A 3. ``sklon divljenju čudima, sklon čudesnome'' (i ovdje se kao primjer značenja navodi upravo ovaj odlomak)
\item[ὁρέω] §~243, jonski (nestegnuti) oblik umjesto ὁράω, akuzativi koji slijede dopune su ovom glagolu
\item[μαινομένους] §~231
\item[παραφρονέοντας] §~243, jonski (nestegnuti) oblik umjesto παραφρονοῦντας
\item[ποιέοντας] §~243, jonski (nestegnuti) oblik umjesto ποιοῦντας
\item[οἶδα] §~317.4, akuzativi koji slijede dopuna su ovom glagolu
\item[οἰμώζοντας] §~231
\item[βοῶντας] §~243
\item[τοὺς δὲ πνιγομένους] \textbf{\textgreek[variant=ancient]{τοὺς δὲ καὶ ἀναΐσσοντάς τε καὶ φεύγοντας}} §~231
\item[παραφρονέοντας] §~243, jonski (nestegnuti) oblik umjesto παραφρονοῦντας
\item[ἐπέγρωνται] složenica ἐγείρω, §~292; osnovu vidi LSJ s. v.%jaki aorist mediopasivni: ἠγρόμην!!!
\item[μέχρις ἂν ἐπέγρωνται] vremenski veznik μέχρις uvodi zavisnu vremensku rečenicu pogodbenog značenja (eventualna iterativna rečenica): „dok god ne…“
\item[ἐόντας] §~315, jonski oblik (nestegnuti) umjesto ὄντας
\item[φρονέοντας] §~243, jonski (nestegnuti) oblik umjesto φρονοῦντας
\item[ἐόντας] §~315, jonski oblik (nestegnuti) umjesto ὄντας
\item[ἐστι] §~315
\item[πολλά ἐστι] imenski predikat
\item[παντοδαπὰ] (ἐστι) imenski predikat
\item[λέγειν] §~231
\item[εἴη] §~315
\item[ὧν] \textbf{περὶ ἑκάστου λέγειν πουλὺς ἂν εἴη λόγος} odnosna zamjenica ὧν uvodi zavisnu odnosnu rečenicu: od toga…

\end{description}

%5


{\large
\begin{greek}
\noindent ᾿Εμοὶ δὲ δοκέουσιν\\
\tabto{2em} οἱ πρῶτοι\\
\tabto{4em} τοῦτο τὸ νόσημα\\
\tabto{2em} ἀφιερώσαντες \\
\tabto{2em} τοιοῦτοι εἶναι ἄνθρωποι \\
\tabto{4em} οἷοι καὶ νῦν εἰσι 
\tabto{6em} μάγοι τε καὶ καθάρται καὶ ἀγύρται καὶ ἀλαζόνες, \\
\tabto{4em} ὁκόσοι δὴ προσποιέονται\\
\tabto{6em} σφόδρα θεοσεβέες εἶναι \\
\tabto{6em} καὶ πλέον τι εἰδέναι.\\

\end{greek}
}

\begin{description}[noitemsep]
\item[δοκέουσιν] §~243, jonski nestegnuti oblik umjesto δοκοῦσιν, otvara mjesto nominativu s infinitivom §~491.2
\item[οἱ ἀφιερώσαντες] §~267
\item[τοιοῦτοι… οἷοι… ὁκόσοι] niz korelativnih rečenica koje uvode korelativne zamjenice: takvi… koji… koliki…
\item[εἶναι] §~315
\item[εἶναι ἄνθρωποι] nominativ s infinitivom, §~491.2
\item[εἰσι] §~315
\item[εἰσι μάγοι τε] \textbf{καὶ καθάρται καὶ ἀγύρται καὶ ἀλαζόνες} imenski predikati
\item[καθάρται] čistitelji, oni koji izvode ritual pročišćavanja
\item[προσποιέονται] §~243, jonski nestegnuti oblik umjesto προσποιοῦνται, otvara mjesto nominativu s infinitivom
\item[εἶναι] §~315
\item[θεοσεβέες εἶναι] nominativ s infinitivom, §~491.2
\item[εἰδέναι] §~317.4, nominativ s infinitivom §~491.2 (nominativ se ne ponavlja)

\end{description}



%kraj
