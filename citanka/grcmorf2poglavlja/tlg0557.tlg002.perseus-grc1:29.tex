%\section*{O autoru}

% Unio ispravke NZ <2021-12-28 uto>


\section*{O tekstu}

Arijan iz Nikomedije \textgreek[variant=ancient]{(Φλάουιος Ἀρριανός,} Flavius Arrianus; Nikomedija u Bitiniji, oko 95. – Atena, oko 175.) priredio je u \textit{Priručniku} \textgreek[variant=ancient]{(Ἐγχειρίδιον)} sažetak Epiktetovih učenja kao uvijek spreman vodič za filozofske susrete sa svakodnevicom. \textit{Priručnik} nije sustavna razrada etike, već zbirka misli i uputa korisnih u konkretnim situacijama. Za razliku od dijaloškog i argumentativnog oblika \textit{Dijatriba} \textgreek[variant=ancient]{(Διατριβαί),} \textit{Priručnik} je apodiktičan, slobodan od metafizičkih i teoloških razmatranja. Pisan je fragmentarnim stilom, u kratkim odlomcima, uobličen u niz formula (vrlo su česti imperativi) i pamtljivih slika koje stoiku služe kao poticaji za dnevne meditacije.

Ovdje odabrani odlomak ističe važnost prethodnog promišljanja; prije svakog pothvata, treba dobro procijeniti djelatnost u koju se upuštamo, ali i točno spoznati \textgreek[variant=ancient]{(καταμανθάνειν)} kakve su naše snage. Na životnom putu posebno je važna dosljednost \textgreek[variant=ancient]{(ἕνα ἄνθρωπον εἶναι)}; tek posvećivanje onome izvanjskome, odnosno unutarnjem, određuje hoće li netko biti \textgreek[variant=ancient]{φιλόσοφος} ili \textgreek[variant=ancient]{ἰδιώτης.}

\newpage

\section*{Pročitajte naglas grčki tekst.}

Epict. Enchiridion 29

%Naslov prema izdanju

\medskip


{\large

\begin{greek}

\noindent ἄνθρωπε, πρῶτον ἐπίσκεψαι, ὁποῖόν ἐστι τὸ πρᾶγμα· εἶτα καὶ τὴν σεαυτοῦ φύσιν κατάμαθε, εἰ δύνασαι βαστάσαι. πένταθλος εἶναι βούλει ἢ παλαιστής; ἴδε σεαυτοῦ τοὺς βραχίονας, τοὺς μηρούς, τὴν ὀσφὺν κατάμαθε. ἄλλος γὰρ πρὸς ἄλλο πέφυκε. 

δοκεῖς, ὅτι ταῦτα ποιῶν ὡσαύτως δύνασαι ἐσθίειν, ὡσαύτως πίνειν, ὁμοίως ὀρέγεσθαι, ὁμοίως δυσαρεστεῖν; ἀγρυπνῆσαι δεῖ, πονῆσαι, ἀπὸ τῶν οἰκείων ἀπελθεῖν, ὑπὸ παιδαρίου καταφρονηθῆναι, ὑπὸ τῶν ἀπαντώντων καταγελασθῆναι, ἐν παντὶ ἧττον ἔχειν, ἐν τιμῇ, ἐν ἀρχῇ, ἐν δίκῃ, ἐν πραγματίῳ παντί. ταῦτα ἐπίσκεψαι, εἰ θέλεις ἀντικαταλλάξασθαι τούτων ἀπάθειαν, ἐλευθερίαν, ἀταραξίαν· εἰ δὲ μή, μὴ προσάγαγε. μὴ ὡς τὰ παιδία νῦν φιλόσοφος, ὕστερον δὲ τελώνης, εἶτα ῥήτωρ, εἶτα ἐπίτροπος Καίσαρος. ταῦτα οὐ συμφωνεῖ. ἕνα σε δεῖ ἄνθρωπον ἢ ἀγαθὸν ἢ κακὸν εἶναι· ἢ τὸ ἡγεμονικόν σε δεῖ ἐξεργάζεσθαι τὸ σαυτοῦ ἢ τὸ ἐκτὸς ἢ περὶ τὰ ἔσω φιλοτεχνεῖν ἢ περὶ τὰ ἔξω· τοῦτ' ἔστιν ἢ φιλοσόφου τάξιν ἐπέχειν ἢ ἰδιώτου.

\end{greek}

}


\section*{Analiza i komentar}

%1

{\large
\begin{greek}
\noindent ἄνθρωπε, πρῶτον ἐπίσκεψαι, \\
\tabto{2em} ὁποῖόν ἐστι τὸ πρᾶγμα· \\
εἶτα καὶ τὴν σεαυτοῦ φύσιν κατάμαθε, \\
\tabto{2em} εἰ δύνασαι βαστάσαι.\\

\end{greek}
}

\begin{description}[noitemsep]
\item[πρῶτον\dots\ εἶτα\dots] koordinacija rečeničnih članova pomoću (vremenskih) priloga
\item[ἐπίσκεψαι] §~267; složenica glagola σκέπτομαι; \textit{verbum sentiendi} otvara mjesto zavisno upitnoj rečenici §~469
\item[ὁποῖόν ἐστι] §~315; kopulativni glagol otvara mjesto nužnoj predikatnoj dopuni (imenski predikat, Smyth 910)
\item[κατάμαθε] §~254; složenica glagola μανθάνω, 321.17; \textit{verbum sentiendi} otvara mjesto zavisno upitnoj rečenici §~469
\item[εἰ δύνασαι] §~312.5; otvara mjesto dopuni u infinitivu
\item[βαστάσαι] §~267
\end{description}

%2


{\large
\begin{greek}
\noindent πένταθλος εἶναι βούλει ἢ παλαιστής; \\
ἴδε \\
\tabto{2em} σεαυτοῦ \\
τοὺς βραχίονας, \\
τοὺς μηρούς, \\
τὴν ὀσφὺν \\
κατάμαθε.\\

\end{greek}
}

\begin{description}[noitemsep]
\item[πένταθλος εἶναι\dots\ ἢ παλαιστής] §~315; kopulativni glagol otvara mjesto nužnoj predikatnoj dopuni (imenski predikat, Smyth 910)
\item[βούλει] §~231 (medij!), otvara mjesto dopuni u infinitivu
\item[ἴδε] §~254; §~327.3
\item[κατάμαθε] §~254; složenica glagola μανθάνω, 321.17
\end{description}

%3


{\large
\begin{greek}
\noindent ἄλλος γὰρ \\
\tabto{2em} πρὸς ἄλλο \\
πέφυκε. \\

\end{greek}
}

\begin{description}[noitemsep]
\item[γὰρ] čestica γάρ najavljuje iznošenje razloga ili objašnjenja, ``naime''
\item[ἄλλος\dots\ πρὸς ἄλλο] hrv.\ ``svatko za nešto drugo''
\item[πέφυκε] §~272
\end{description}

%4


{\large
\begin{greek}
\noindent δοκεῖς, \\
\tabto{2em} ὅτι ταῦτα ποιῶν \\
\tabto{4em} ὡσαύτως δύνασαι ἐσθίειν, \\
\tabto{4em} ὡσαύτως πίνειν, \\
\tabto{4em} ὁμοίως ὀρέγεσθαι, \\
\tabto{4em} ὁμοίως δυσαρεστεῖν; \\

\end{greek}
}

\begin{description}[noitemsep]
\item[δοκεῖς, ὅτι\dots] §~243; izrična zavisna rečenica iza ὅτι, §~467
\item[ταῦτα ποιῶν] u ovom kontekstu: ``baveći se filozofijom''
\item[ποιῶν] §~243
\item[ὡσαύτως] ``kao dosad''
\item[δύνασαι] §~312.5; otvara mjesto dopunama u infinitivu
\item[ἐσθίειν] §~231
\item[πίνειν] §~231
\item[ὀρέγεσθαι] §~232
\item[δυσαρεστεῖν] §~243
\end{description}

%5


{\large
\begin{greek}
\noindent \tabto{2em} ἀγρυπνῆσαι \\
δεῖ, \\
\tabto{2em} πονῆσαι, \\
\tabto{2em} ἀπὸ τῶν οἰκείων \\
\tabto{4em} ἀπελθεῖν, \\
\tabto{2em} ὑπὸ παιδαρίου \\
\tabto{4em} καταφρονηθῆναι, \\
\tabto{2em} ὑπὸ τῶν ἀπαντώντων \\
\tabto{4em} καταγελασθῆναι, \\
\tabto{2em} ἐν παντὶ \\
\tabto{4em} ἧττον ἔχειν, \\
\tabto{2em} ἐν τιμῇ, ἐν ἀρχῇ, ἐν δίκῃ, \\
\tabto{2em} ἐν πραγματίῳ παντί. \\

\end{greek}
}

\begin{description}[noitemsep]
\item[ἀγρυπνῆσαι] §~267, §~269
\item[δεῖ] sc. σε; bezlični glagol otvara mjesto infinitivima, §~492
\item[πονῆσαι] §~267, §~269
\item[ἀπελθεῖν] složenica glagola ἔρχομαι, §~327.2; §~254
\item[καταφρονηθῆναι] §~296
\item[καταγελασθῆναι] §~296
\item[ἧττον ἔχειν] §~231; LSJ ἔχω B.II.2
\end{description}

%6


{\large
\begin{greek}
\noindent ταῦτα ἐπίσκεψαι, \\
\tabto{2em} εἰ θέλεις ἀντικαταλλάξασθαι \\
\tabto{6em} τούτων \\
\tabto{6em} ἀπάθειαν, ἐλευθερίαν, ἀταραξίαν· \\
\tabto{2em} εἰ δὲ μή, \\
\tabto{2em} μὴ προσάγαγε.\\

\end{greek}
}

\begin{description}[noitemsep]
\item[ἐπίσκεψαι] §~267; složenica glagola σκέπτομαι; \textit{verbum sentiendi} otvara mjesto zavisno upitnoj rečenici §~469
\item[εἰ θέλεις] §~232; otvara mjesto dopuni u infinitivu
\item[ἀντικαταλλάξασθαι] §~267, §~269, ἀντικαταλλάσσω, atički ἀντικαταλλάττω τι τινός, složenica glagola ἀλλάσσω (ἀλλάττω)
\item[εἰ δὲ μή] ``inače''; LSJ εἰ VII eliptične konstrukcije, 3. kad je glagol u protazi ispušten ``ali ako ne'', tj. ``inače''
\item[μὴ προσάγαγε] §~254; složenica glagola ἄγω, s. 116, §~257; negacija u zapovjednim rečenicama §~509.2a
\end{description}

%7


{\large
\begin{greek}
\noindent μὴ \\
\tabto{2em} ὡς τὰ παιδία \\
νῦν φιλόσοφος, \\
ὕστερον δὲ τελώνης, \\
εἶτα ῥήτωρ, \\
εἶτα ἐπίτροπος Καίσαρος.\\

\end{greek}
}

\begin{description}[noitemsep]
\item[μὴ] sc.\ ἴσθι
\item[ὡς] veznik otvara mjesto poredbenoj zavisnoj rečenici, Smyth 2475
\item[νῦν\dots\ ὕστερον δὲ] koordinacija rečeničnih članova pomoću čestice δέ
\end{description}

%8


{\large
\begin{greek}
\noindent ταῦτα οὐ συμφωνεῖ.

\end{greek}
}

\begin{description}[noitemsep]
\item[συμφωνεῖ] §~243

\end{description}

%9


{\large
\begin{greek}
\noindent ἕνα \\
\tabto{2em} σε \\
δεῖ \\
\tabto{2em} ἄνθρωπον \\
\tabto{4em} ἢ ἀγαθὸν ἢ κακὸν \\
\tabto{2em} εἶναι· \\
\tabto{2em} ἢ τὸ ἡγεμονικόν \\
\tabto{6em} σε \\
\tabto{4em} δεῖ \\
\tabto{6em} ἐξεργάζεσθαι \\
\tabto{4em} τὸ σαυτοῦ \\
\tabto{2em} ἢ τὸ ἐκτὸς \\
\tabto{2em} ἢ περὶ τὰ ἔσω \\
\tabto{4em} φιλοτεχνεῖν \\
\tabto{2em} ἢ περὶ τὰ ἔξω· \\
\tabto{2em} τοῦτ' ἔστιν \\
\tabto{4em} ἢ φιλοσόφου \\
\tabto{6em} τάξιν \\
\tabto{6em} ἐπέχειν \\
\tabto{4em} ἢ ἰδιώτου.\\

\end{greek}
}

\begin{description}[noitemsep]
\item[δεῖ] bezlični glagol otvara mjesto akuzativu s infinitivom, §~492
\item[ἕνα\dots\ ἄνθρωπον\dots\ εἶναι] predikatni dio akuzativa s infinitivom; imenski predikat, Smyth 909
\item[τὸ ἡγεμονικόν] stoički \textit{terminus technicus} za upravljački dio duše, njezin ``zapovjedni centar''
\item[σε δεῖ ἐξεργάζεσθαι\dots\ φιλοτεχνεῖν] §~232, §~231, bezlični glagol otvara mjesto akuzativu s infinitivom, §~492
\item[τὸ σαυτοῦ] supstantivirana zamjenica, §~373
\item[τὸ ἐκτὸς] supstantivirani prilog, §~373
\item[περὶ τὰ ἔσω\dots\ περὶ τὰ ἔξω] supstantivirani prilog, §~373
\item[τοῦτ' ἔστιν] §~315; LSJ εἰμί B; otvara mjesto infinitivu
\item[ἐπέχειν] §~231

\end{description}


%kraj
