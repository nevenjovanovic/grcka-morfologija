%\section*{O autoru}

%TKTK


\section*{O tekstu}

Osamnaesta godina Peloponeskog rata (414./413.\ p.~n.~e) ujedno je bila i druga godina takozvane Sicilske ekspedicije, u kojoj su atenske snage opsjedale daleku Sirakuzu. Sirakužanima su vojno pomagali Spartanci; početkom godine, u trenutku kad su Sirakužani već razmišljali o predaji, Spartanci su im poslali novog zapovjednika, Gilipa, s korintskim i spartanskim pojačanjima. Počela je grozničava utrka u podizanju zidova na strateški važnoj visoravni Epipoli \textgreek[variant=ancient]{(Ἐπιπολαί),} neposredno iznad grada. Atenjani su pokušavali izgraditi zid i tako blokirati Sirakuzu, a Sirakužani su gradili protuzid sa svoje strane, kako bi presjekli liniju Atenjana i tako spriječili potpuno opkoljavanje. Između dva zida sukobile su se vojske. Prvi su put Sirakužani bili poraženi. No, u sljedećem će sukobu na istom mjestu izgubiti Atenjani, a Sirakužani će uspjeti blokirati daljnje napredovanje opsadnog zida. To će biti prekretnica u Sicilskoj ekspediciji, fatalni gubitak atenske prednosti i inicijative.

\newpage

\section*{Pročitajte naglas grčki tekst.}

Thuc.\ Historiae 7.5.1–7.5.4

%Naslov prema izdanju

\medskip


{\large

\begin{greek}

\noindent ὁ δὲ Γύλιππος ἅμα μὲν ἐτείχιζε τὸ διὰ τῶν Ἐπιπολῶν τεῖχος, τοῖς λίθοις χρώμενος οὓς οἱ Ἀθηναῖοι προπαρεβάλοντο σφίσιν, ἅμα δὲ παρέτασσεν ἐξάγων αἰεὶ πρὸ τοῦ τειχίσματος τοὺς Συρακοσίους καὶ τοὺς ξυμμάχους· καὶ οἱ Ἀθηναῖοι ἀντιπαρετάσσοντο.

ἐπειδὴ δὲ ἔδοξε τῷ Γυλίππῳ καιρὸς εἶναι, ἦρχε τῆς ἐφόδου· καὶ ἐν χερσὶ γενόμενοι ἐμάχοντο μεταξὺ τῶν τειχισμάτων, ᾗ τῆς ἵππου τῶν Συρακοσίων οὐδεμία χρῆσις ἦν.

καὶ νικηθέντων τῶν Συρακοσίων καὶ τῶν ξυμμάχων καὶ νεκροὺς ὑποσπόνδους ἀνελομένων καὶ τῶν Ἀθηναίων τροπαῖον στησάντων, ὁ Γύλιππος ξυγκαλέσας τὸ στράτευμα οὐκ ἔφη τὸ ἁμάρτημα ἐκείνων, ἀλλ ἑαυτοῦ γενέσθαι· τῆς γὰρ ἵππου καὶ τῶν ἀκοντιστῶν τὴν ὠφελίαν τῇ τάξει ἐντὸς λίαν τῶν τειχῶν ποιήσας ἀφελέσθαι· νῦν οὖν αὖθις ἐπάξειν.

καὶ διανοεῖσθαι οὕτως ἐκέλευεν αὐτοὺς ὡς τῇ μὲν παρασκευῇ οὐκ ἔλασσον ἕξοντας, τῇ δὲ γνώμῃ οὐκ ἀνεκτὸν ἐσόμενον εἰ μὴ ἀξιώσουσι Πελοποννήσιοί τε ὄντες καὶ Δωριῆς Ἰώνων καὶ νησιωτῶν καὶ ξυγκλύδων ἀνθρώπων κρατήσαντες ἐξελάσασθαι ἐκ τῆς χώρας.

\end{greek}

}


\section*{Analiza i komentar}

%1

{\large
\begin{greek}
\noindent ὁ δὲ Γύλιππος \\
ἅμα μὲν ἐτείχιζε \\
\tabto{2em} τὸ διὰ τῶν Ἐπιπολῶν τεῖχος, \\
τοῖς λίθοις χρώμενος \\
\tabto{2em} οὓς οἱ Ἀθηναῖοι \\
\tabto{2em} προπαρεβάλοντο σφίσιν, \\
ἅμα δὲ παρέτασσεν \\
\tabto{2em} ἐξάγων \\
\tabto{4em} αἰεὶ \\
\tabto{4em} πρὸ τοῦ τειχίσματος \\
\tabto{2em} τοὺς Συρακοσίους καὶ τοὺς ξυμμάχους· \\
καὶ οἱ Ἀθηναῖοι ἀντιπαρετάσσοντο.\\

\end{greek}
}

\begin{description}[noitemsep]
\item[δὲ] čestica označava nadovezivanje na prethodno pripovijedanje
\item[ἅμα μὲν\dots\ ἅμα δὲ\dots] koordinacija rečeničnih članova pomoću čestica μέν\dots\ δέ\dots
\item[ἐτείχιζε] § 231
\item[χρώμενος] § 232, § 243; rekcija: χράομαί τινι
\item[προπαρεβάλοντο] § 238; § 254; s.~118
\item[παρέτασσεν] § 231, § 238; složenica τάσσω (atički τάττω); usp.\ niže \textgreek[variant=ancient]{ἀντιπαρετάσσοντο}
\item[ἐξάγων] § 231
\item[ἀντιπαρετάσσοντο] § 232, § 238
\end{description}

%2


{\large
\begin{greek}
\noindent ἐπειδὴ δὲ ἔδοξε \\
\tabto{2em} τῷ Γυλίππῳ \\
\tabto{2em} καιρὸς εἶναι, \\
ἦρχε \\
\tabto{2em} τῆς ἐφόδου· \\
καὶ ἐν χερσὶ γενόμενοι \\
ἐμάχοντο \\
\tabto{2em} μεταξὺ τῶν τειχισμάτων, \\
\tabto{4em} ᾗ τῆς ἵππου \\
\tabto{6em} τῶν Συρακοσίων \\
\tabto{4em} οὐδεμία χρῆσις ἦν.\\


\end{greek}
}

\begin{description}[noitemsep]
\item[δὲ] čestica označava nadovezivanje na prethodno pripovijedanje
\item[ἔδοξε] § 267; rekcija τινι; otvara mjesto dopuni u infinitivu: da\dots
\item[καιρὸς εἶναι] § 315; imenski predikat, Smyth 909
\item[ἦρχε] § 231, § 235; rekcija τινος
\item[ἐν χερσὶ γενόμενοι] § 254; § 325.11; fraza koja znači: stupiti u blisku borbu, LSJ χείρ II.6.f
\item[ἐμάχοντο] § 232
\item[ᾗ] LSJ ὅς, ἥ, ὅ 0-2 A II (izražava mjesto)
\item[τῆς ἵππου] LSJ ἵππος II (kao kolektivna imenica)
\item[χρῆσις ἦν] § 315; imenski predikat, Smyth 909
\end{description}

%3


{\large
\begin{greek}
\noindent καὶ νικηθέντων τῶν Συρακοσίων καὶ τῶν ξυμμάχων \\
καὶ νεκροὺς ὑποσπόνδους ἀνελομένων \\
καὶ τῶν Ἀθηναίων τροπαῖον στησάντων, \\
\tabto{2em} ὁ Γύλιππος \\
\tabto{4em} ξυγκαλέσας τὸ στράτευμα \\
\tabto{2em} οὐκ ἔφη \\
\tabto{4em} τὸ ἁμάρτημα \\
\tabto{6em} ἐκείνων, \\
\tabto{6em} ἀλλ ἑαυτοῦ \\
\tabto{4em} γενέσθαι· \\
\tabto{6em} τῆς γὰρ ἵππου καὶ τῶν ἀκοντιστῶν \\
\tabto{6em} τὴν ὠφελίαν \\
\tabto{8em} τῇ τάξει \\
\tabto{6em} ἐντὸς λίαν τῶν τειχῶν \\
\tabto{4em} ποιήσας \\
\tabto{6em} ἀφελέσθαι· \\
\tabto{4em} νῦν οὖν αὖθις ἐπάξειν.\\


\end{greek}
}

\begin{description}[noitemsep]
\item[νικηθέντων] § 296, particip je dio genitiva apsolutnog%gen. aps.
\item[ἀνελομένων] § 254, složenica αἱρέω § 327.1; particip je dio genitiva apsolutnog %gen. aps.
\item[στησάντων] § 267, particip je dio genitiva apsolutnog%gen. aps.
\item[ξυγκαλέσας] § 267; ξυγ- je atička varijanta za συγκαλέω
\item[οὐκ ἔφη] § 312.8; fraza οὔ φημι: poreći, zanijekati, odbiti (LSJ φημί III); otvara mjesto akuzativu s infinitivom i infinitivu
\item[ἁμάρτημα\dots\ γενέσθαι] § 325.11; imenski predikat, Smyth 909; akuzativ s infinitivom
\item[ἐντὸς\dots\ ποιήσας] § 269; fraza, LSJ ἐντός II: uvući, uvesti%imenski pred.
\item[τὴν ὠφελίαν\dots\ ἀφελέσθαι] složenica αἱρέω § 327.1; § 254; infinitiv kao dopuna οὐκ ἔφη
\item[ἐπάξειν] složenica ἄγω, s.~116; § 258; infinitiv kao dopuna οὐκ ἔφη

\end{description}

%4


{\large
\begin{greek}
\noindent καὶ διανοεῖσθαι \\
\tabto{2em} οὕτως \\
ἐκέλευεν αὐτοὺς \\
\tabto{2em} ὡς τῇ μὲν παρασκευῇ \\
\tabto{4em} οὐκ ἔλασσον ἕξοντας, \\
\tabto{2em} τῇ δὲ γνώμῃ \\
\tabto{4em} οὐκ ἀνεκτὸν ἐσόμενον \\
\tabto{4em} εἰ μὴ ἀξιώσουσι \\
\tabto{6em} Πελοποννήσιοί τε ὄντες καὶ Δωριῆς \\
\tabto{6em} Ἰώνων καὶ νησιωτῶν καὶ ξυγκλύδων ἀνθρώπων \\
\tabto{4em} κρατήσαντες \\
\tabto{4em} ἐξελάσασθαι \\
\tabto{6em} ἐκ τῆς χώρας.\\

\end{greek}
}

\begin{description}[noitemsep]
\item[διανοεῖσθαι] § 232, § 243; dio akuzativa s infinitivom (neupravni govor)
\item[ἐκέλευεν] § 231; otvara mjesto akuzativu s infinitivom
\item[οὕτως\dots\ ὡς\dots] \textbf{οὐκ ἔλασσον ἕξοντας} § 327.13; § 258; LSJ ἔχω B.II.2; poredbeno: tako\dots\ kao da\dots%izrično
\item[τῇ μὲν \dots\ τῇ δὲ\dots] koordinacija rečeničnih članova
\item[οὐκ ἀνεκτὸν ἐσόμενον] § 315; imenski predikat, Smyth 909
\item[εἰ μὴ ἀξιώσουσι] § 259; otvara mjesto dopuni u infinitivu; pogodbena realna protaza: ako ne\dots
\item[ὄντες] § 315 (s imenskim dopunama)
\item[κρατήσαντες] § 269; rekcija: τινος
\item[ἐξελάσασθαι] § 267; složenica glagola ἐλαύνω

\end{description}


%kraj
