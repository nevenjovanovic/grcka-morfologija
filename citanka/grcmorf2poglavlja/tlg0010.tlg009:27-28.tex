% Unio ispravke NZ <2022-01-04 uto>

\section*{O tekstu}

Izokratov epideiktični govor \textit{Pohvala Helene} nastao je oko 370.\ pr.~Kr.\footnote{Termin ``epideiktičan'' \textgreek[variant=ancient]{(ἐπιδεικτικός)} potječe iz Aristotelove klasifikacije, a danas označava raznolike forme retoričke pohvale ili pokude osobe, grada, životinje ili stvari.}

Helenu grčka književna tradicija tretira kao negativnu junakinju \textit{par excellence} te je govoriti njoj u čast izazov govorničkoj vještini. Prvi je takvu pohvalu sastavio sofist i retor Gorgija iz Leontina (V.-IV.~st.\ pr.~Kr.). Kao Gorgijin učenik, slavni atenski govornik Izokrat prihvaća se istoga zadatka, tvrdeći da će baš on pokazati kako se to u stvari treba napraviti.

Ovaj izvada pripada pohvali Tezeju, najopsežnijem dijelu govora. Tezej i Helena povezani su nešto manje poznatom pričom koja je znatno prethodila Trojanskom ratu: htijući, kao Zeusov sin, za ženu uzeti neku Zeusovu kćer, Tezej je odlučio oteti i oženiti Helenu, tada još gotovo dijete. 

Poznato je da je Tezej oslobodio Atenu strahovita danka u krvi koji je grad plaćao kretskom vladaru Minosu. Svakih sedam ili devet godina ždrijebom odabrani mladići i djevojke otplovili bi iz Atene na Kretu da budu žrtvovani Minotauru, čudovišnom sinu Minosove žene Pazifaje i kretskoga bika. U doba trećeg žrtvenog prinosa Tezej promatra tužni rastanak i odlučuje sam otići na Kretu, ubiti bika i osloboditi Atenu grozne obaveze. 

%\newpage

\section*{Pročitajte naglas grčki tekst.}

Isoc.\ Helenae encomium 27–28

%Naslov prema izdanju

\medskip


{\large

\begin{greek}

\noindent περὶ δὲ τοὺς αὐτοὺς χρόνους τὸ τέρας τὸ τραφὲν μὲν ἐν Κρήτῃ, γενόμενον δʼ ἐκ Πασιφάης τῆς Ἡλίου θυγατρός, ᾧ κατὰ μαντείαν δασμὸν τῆς πόλεως δὶς ἑπτὰ παῖδας ἀποστελλούσης, ἰδὼν αὐτοὺς ἀγομένους καὶ πανδημεὶ προπεμπομένους ἐπὶ θάνατον ἄνομον καὶ προῦπτον καὶ πενθουμένους ἔτι ζῶντας, οὕτως ἠγανάκτησεν ὥσθʼ ἡγήσατο κρεῖττον εἶναι τεθνάναι μᾶλλον ἢ ζῆν ἄρχων τῆς πόλεως τῆς οὕτως οἰκτρὸν τοῖς ἐχθροῖς φόρον ὑποτελεῖν ἠναγκασμένης.


\noindent σύμπλους δὲ γενόμενος, καὶ κρατήσας φύσεως ἐξ ἀνδρὸς μὲν καὶ ταύρου μεμιγμένης, τὴν δʼ ἰσχὺν ἐχούσης οἵαν προσήκει τὴν ἐκ τοιούτων σωμάτων συγκειμένην, τοὺς μὲν παῖδας διασώσας τοῖς γονεῦσιν ἀπέδωκε, τὴν δὲ πόλιν οὕτως ἀνόμου καὶ δεινοῦ καὶ δυσαπαλλάκτου προστάγματος ἠλευθέρωσεν.

\end{greek}

}


\section*{Analiza i komentar}

%1

{\large
\begin{greek}
\noindent περὶ δὲ τοὺς αὐτοὺς χρόνους \\
τὸ τέρας \\
\tabto{2em} τὸ τραφὲν μὲν ἐν Κρήτῃ, \\
\tabto{2em} γενόμενον δ' ἐκ Πασιφάης \\
\tabto{4em} τῆς ῾Ηλίου θυγατρὸς, \\
\tabto{2em} ᾧ \\
\tabto{4em} κατὰ μαντείαν \\
\tabto{2em} δασμὸν \\
\tabto{4em} τῆς πόλεως \\
\tabto{2em} δὶς ἑπτὰ παῖδας \\
\tabto{4em} ἀποστελλούσης, \\
ἰδὼν \\
\tabto{2em} αὐτοὺς ἀγομένους \\
\tabto{2em} καὶ πανδημεὶ προπεμπομένους \\
\tabto{4em} ἐπὶ θάνατον ἄνομον καὶ προῦπτον \\
\tabto{2em} καὶ πενθουμένους ἔτι ζῶντας, \\
οὕτως ἠγανάκτησεν \\
\tabto{2em} ὥσθ' ἡγήσατο \\
\tabto{4em} κρεῖττον εἶναι \\
\tabto{6em} τεθνάναι \\
\tabto{4em} μᾶλλον ἢ\\
\tabto{6em} ζῆν \\
\tabto{6em} ἄρχων \\
\tabto{8em} τῆς πόλεως \\
\tabto{10em} τῆς οὕτως οἰκτρὸν \\
\tabto{12em} τοῖς ἐχθροῖς \\
\tabto{12em} φόρον ὑποτελεῖν \\
\tabto{10em} ἠναγκασμένης. \\

\end{greek}
}

\begin{description}[noitemsep]
\item[τὸ τέρας] sc.\ Μινώταυρος
\item[τραφὲν] §~292, §~295 (osnove §~301.B s.~118)
\item[τὸ τραφὲν] supstantivirani particip §~499, atributni položaj iza imenice §~375
\item[γενόμενον ] sc.\ τὸ τέρας; §~254 (osnove §~325.11)
\item[τῆς Ἡλίου θυγατρός] genitiv u atributnom položaju
\item[ᾧ] zamjenica otvara mjesto zavisnoj odnosnoj rečenici, antecedent je τὸ τέρας
\item[ἀποστελλούσης] §~231; dalji objekt je ᾧ
\item[τῆς πόλεως ἀποστελλούσης] GA §~504 
\item[ἰδὼν] §~254, naglasak §~255 (osnove §~327.3)
\item[ἀγομένους] §~231
\item[προπεμπομένους] §~231
\item[προῦπτον] πρόοπτος, atički stegnuto προὖπτος, glagolski pridjev od προοράω
\item[πενθουμένους] glagol πενθέω; §~243
\item[ζῶντας] §~234
\item[οὕτως ἠγανάκτησεν ὥσθ' ἡγήσατο\dots] koordinacija priloga i veznika uvodi posljedičnu rečenicu: tako\dots\ da\dots
\item[ἠγανάκτησεν] §~267, augment §~235
\item[ἡγήσατο] §~267, augment §~235; LSJ ἡγέομαι III.2; glagol otvara mjesto A+I gdje bi subjektni akuzativ bio infinitiv \textgreek{τεθνάναι;} pri prevođenju odaberite hrvatski ekvivalent koji će izreći \textit{svršenost} radnje
\item[εἶναι] §~315
\item[τεθνάναι] mješoviti perfekt §~317, §~317.2 (osnove §~234.8)
\item[ζῆν] §~243
\item[ἄρχων] rekcija τινός; §~231
\item[τῆς πόλεως τῆς οὕτως οἰκτρὸν] jače istaknut atributni položaj §~375; atribut je i čitav izraz \textgreek{τῆς πόλεως\dots\ ὑποτελεῖν}
\item[ὑποτελεῖν] §~243, otvara mjesto objektima u akuzativu i dativu
\item[ἠναγκασμένης] §~272; reduplikacija §~275; glagol otvara mjesto dopuni u infinitivu
\end{description}

%2

{\large
\begin{greek}
\noindent σύμπλους δὲ γενόμενος \\
καὶ κρατήσας \\
\tabto{2em} φύσεως \\
\tabto{4em} ἐξ ἀνδρὸς μὲν καὶ ταύρου μεμιγμένης, \\
\tabto{4em} τὴν δ' ἰσχὺν ἐχούσης \\
\tabto{6em} οἵαν προσήκει \\
\tabto{8em} τὴν ἐκ τοιούτων σωμάτων συγκειμένην, \\
τοὺς μὲν παῖδας \\
\tabto{2em} διασώσας \\
\tabto{4em} τοῖς γονεῦσιν ἀπέδωκεν, \\
τὴν δὲ πόλιν \\
\tabto{2em} οὕτως ἀνόμου καὶ δεινοῦ καὶ δυσαπαλλάκτου προστάγματος \\
\tabto{4em} ἠλευθέρωσεν.\\

\end{greek}
}

\begin{description}[noitemsep]
\item[γενόμενος] §~254 (osnove §~325.11); kopulativni glagol otvara mjesto imenskoj dopuni
\item[κρατήσας] rekcija τινός; §~267, duljenje osnove u aoristu §~269
\item[φύσεως] LSJ φύσις V.
\item[ἐξ ἀνδρὸς μὲν\dots\ τὴν δ' ἰσχὺν\dots] koordinacija rečeničnih članova parom čestica
\item[μεμιγμένης] §~272 (osnova §~319.12)
\item[ἐχούσης] §~231
\item[προσήκει] §~231
\item[οἵαν προσήκει] relativna zamjenica uvodi relativnu rečenicu, njezin je antecedent ἰσχὺν
\item[συγκειμένην] rekcija ἔκ τινων; §~315.a; supstantivirani particip, §~499
\item[τοὺς μὲν παῖδας\dots\ τὴν δὲ πόλιν\dots] koordinacija rečeničnih članova parom čestica
\item[διασώσας] §~267 (osnova §~301.B s.~116)
\item[ἀπέδωκεν] §~309, augment §~238 (osnova §~327.11)
\item[ἠλευθέρωσεν] rekcija τινά τινος nekoga od nečeg; §~267, augment §~235
\end{description}


%kraj
