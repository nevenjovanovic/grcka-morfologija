% Dovršio NJ <2022-01-07 pet>


\section*{O tekstu}

\textit{Gozba} (ili \textit{Pijanka}, \textgreek[variant=ancient]{Συμπόσιον)} Atenjanina Ksenofonta (oko 430.–355.\ pr.~Kr.) žanrovski pripada tzv.\ sokratičkim dijalozima. Uz Platonovo djelo i nekoliko fragmenata Eshina iz atičke deme Sfeta \textgreek[variant=ancient]{(Αἰσχίνης Σφήττιος),} ti su dijalozi sve što danas imamo od nekoć bujnog spisateljstva posvećenog atenskom mudracu. 

Pored \textit{Sokratove obrane} \textgreek[variant=ancient]{(Ἀπολογία Σωκράτους)} i \textit{Uspomena na Sokrata} \textgreek[variant=ancient]{(Ἀπομνημονεύματα Σωκράτους),} \textit{Gozba} tvori najdojmljiviji dio Ksenofontove sokratičke trijade. Gozbu iz naslova atenski bogataš Kalija priredio je u čast pobjede u pankratiju \textgreek[variant=ancient]{(παγκράτιον,} šakanje i hrvanje) njegova ljubavnika Autolika na Panatenejama 422.\ pr.~Kr. Ksenofontov prikaz društvenog zbivanja s muzičarima, akrobatskim plesačima i mimom (na samom kraju djela plesna skupina mimički prikazuje kako su Dioniz i Arijadna proveli ljubavnu noć, što nam pruža rijetko svjedočanstvo o izvedbama mima prije helenizma) vjerojatno je nastao nakon Platonove \textit{Gozbe}. Ksenofontov Sokrat drži dug govor posvećen homoseksualnoj ljubavi. Pritom strogo razlikuje ljubav koja teži fizičkom zadovoljenju, osuđujući je kao odvratnu i beskorisnu, od plemenite ljubavi u kojoj se ljubavnik trsi ohrabriti ono najbolje u svom voljenome, pokazujući tako da zaslužuje mladićevo divljenje. Takvu su dihotomiju sigurno odobravali brojni Ksenofontovi suvremenici, iako Platon, kao kudikamo bolji psiholog, u \textit{Fedru} (253e-263e) pokazuje da obje vrste ljubavi mogu supostojati u istoj osobi. 

U odabranom odlomku Harmid odgovara zašto se toliko ponosi svojim siromaštvom; pitanje \textgreek[variant=ancient]{δι᾿ ὅ τι ἐπὶ πενίᾳ μέγα φρονεῖς;} postavio mu je domaćin Kalija. Ostavši bez imetka, bivši bogataš Harmid postao je slobodan čovjek koji živi na teret države. Imućni mu podilaze gledajući u njemu mogućeg birača i porotnika. Ne treba više strahovati za vlastitu i sigurnost svoje imovine i ne mora imati posla s ucjenjivačima. Oslobođen brige za svoju imovinu i lišen svih nameta napokon mirno spava. Ne može ništa izgubiti jer nema što izgubiti; naprotiv, može samo dobiti.

%\newpage

\section*{Pročitajte naglas grčki tekst.}

Xen. Symposium 4.30–32

%Naslov prema izdanju

\medskip


{\large

\begin{greek}

\noindent ἐγὼ τοίνυν ἐν τῇδε τῇ πόλει ὅτε μὲν πλούσιος ἦν πρῶτον μὲν ἐφοβούμην μή τίς μου τὴν οἰκίαν διορύξας καὶ τὰ χρήματα λάβοι καὶ αὐτόν τί με κακὸν ἐργάσαιτο· ἔπειτα δὲ καὶ τοὺς συκοφάντας ἐθεράπευον, εἰδὼς ὅτι παθεῖν μᾶλλον κακῶς ἱκανὸς εἴην ἢ ποιῆσαι ἐκείνους. καὶ γὰρ δὴ καὶ προσετάττετο μὲν ἀεί τί μοι δαπανᾶν ὑπὸ τῆς πόλεως, ἀποδημῆσαι δὲ οὐδαμοῦ ἐξῆν.

\noindent νῦν δ´ ἐπειδὴ τῶν ὑπερορίων στέρομαι καὶ τὰ ἔγγεια οὐ καρποῦμαι καὶ τὰ ἐκ τῆς οἰκίας πέπραται, ἡδέως μὲν καθεύδω ἐκτεταμένος, πιστὸς δὲ τῇ πόλει γεγένημαι, οὐκέτι δὲ ἀπειλοῦμαι, ἀλλ´ ἤδη ἀπειλῶ ἄλλοις, ὡς ἐλευθέρῳ τε ἔξεστί μοι καὶ ἀποδημεῖν καὶ ἐπιδημεῖν· ὑπανίστανται δέ μοι ἤδη καὶ θάκων καὶ ὁδῶν ἐξίστανται οἱ πλούσιοι.

\noindent καὶ εἰμὶ νῦν μὲν τυράννῳ ἐοικώς, τότε δὲ σαφῶς δοῦλος ἦν· καὶ τότε μὲν ἐγὼ φόρον ἀπέφερον τῷ δήμῳ, νῦν δὲ ἡ πόλις τέλος φέρουσα τρέφει με. ἀλλὰ καὶ Σωκράτει, ὅτε μὲν πλούσιος ἦν, ἐλοιδόρουν με ὅτι συνῆν, νῦν δ´ ἐπεὶ πένης γεγένημαι, οὐκέτι οὐδὲν μέλει οὐδενί. καὶ μὴν ὅτε μέν γε πολλὰ εἶχον, ἀεί τι ἀπέβαλλον ἢ ὑπὸ τῆς πόλεως ἢ ὑπὸ τῆς τύχης· νῦν δὲ ἀποβάλλω μὲν οὐδέν (οὐδὲ γὰρ ἔχω), ἀεὶ δέ τι λήψεσθαι ἐλπίζω. 

\end{greek}

}


\section*{Analiza i komentar}

%1

{\large
\begin{greek}
\noindent ἐγὼ τοίνυν \\
\tabto{2em} ἐν τῇδε τῇ πόλει \\
\tabto{2em} ὅτε μὲν πλούσιος ἦν \\
πρῶτον μὲν ἐφοβούμην \\
\tabto{2em} μή τίς μου τὴν οἰκίαν διορύξας \\
\tabto{4em} καὶ τὰ χρήματα λάβοι \\
\tabto{4em} καὶ αὐτόν τί με κακὸν ἐργάσαιτο· \\
ἔπειτα δὲ  \\
καὶ τοὺς συκοφάντας ἐθεράπευον, \\
εἰδὼς ὅτι \\
\tabto{2em} παθεῖν μᾶλλον κακῶς ἱκανὸς εἴην \\
\tabto{2em} ἢ ποιῆσαι ἐκείνους. \\

\end{greek}
}

\begin{description}[noitemsep]
\item[τοίνυν] §~519.5
\item[ὅτε] veznik ὅτε uvodi zavisnu vremensku rečenicu
\item[πλούσιος ἦν] §~315; pridjevska dopuna kopulativnom glagolu (imenski predikat), Smyth 910
\item[πρῶτον μὲν\dots\ ἔπειτα δὲ\dots] koordinacija rečeničnih članova pomoću čestica μέν\dots\ δέ\dots\ i nabrajanja
\item[ἐφοβούμην] §~243; §~328.3
\item[μή τίς\dots\ λάβοι\dots\ ἐργάσαιτο] izraz bojazni \textit{(verbum timendi} φοβέομαι) otvara mjesto negaciji μή koja uvodi zavisnu rečenicu bojazni (bojim se da ću\dots, bojim se da ne bih\dots); sporedno vrijeme u glavnoj rečenici prati najčešće optativ u zavisnoj
\item[διορύξας] složenica glagola ὀρύττω; §~267; adverbni particip §~503
\item[λάβοι] §~321.14; jaki aorist §~254
\item[αὐτόν\dots\ με] „meni samom”
\item[τί\dots\ κακὸν] neodređenu zamjenicu dopunja atribut 
\item[ἐργάσαιτο] ἐργάζομαί τινά τι komu što učiniti; slabi aorist §~267
\item[τοὺς συκοφάντας] συκοφάντης je u atičkom pravu privatna osoba, građanin koji dobrovoljno prijavljuje nadležnim službama počinitelje prekršaja. Djelatnost se kod nekih gotovo pretvarala u profesiju, podložnu zloupotrebama (pobijedi li u procesu, sikofant je ostvarivao dobit)
\item[ἐθεράπευον] §~231
\item[εἰδὼς] §~317.4
\item[παθεῖν\dots\ κακῶς] §~327.15; jaki aorist §~254; \textgreek{κακῶς πάσχειν} ili παθεῖν štetu trpjeti, nepravdu iskusiti; infinitiv uz relativne pridjeve §~494
\item[μᾶλλον\dots\ ἢ\dots] koordinacija; LSJ μάλα II 
\item[ὅτι\dots\ ἱκανὸς εἴην] particip εἰδὼς \textit{(verbum sentiendi)} otvara mjesto vezniku ὅτι i izričnoj rečenici; optativ u zavisnoj rečenici stoji obično iza historijskih vremena u glavnoj; §~315; pridjevska dopuna kopulativnom glagolu (imenski predikat), Smyth 910
\item[ποιῆσαι] tj. κακῶς; slabi aorist §~267, \textit{verbum vocale;} \textgreek{κακῶς ποιέω τινά,} aktivan lik (glagolsko stanje, vrsta) izraza \textgreek{κακῶς πάσχειν}; ἐκείνους (tj. τοὺς συκοφάντας)
\end{description}

{\large
\begin{greek}
καὶ γὰρ δὴ καὶ προσετάττετο μὲν ἀεί \\
\tabto{2em} τί μοι δαπανᾶν \\
\tabto{2em} ὑπὸ τῆς πόλεως, \\
ἀποδημῆσαι δὲ οὐδαμοῦ ἐξῆν.\\

\end{greek}
}

\begin{description}[noitemsep]
\item[καὶ\dots\ δὴ καὶ\dots] §~513.1
\item[γὰρ] §~517
\item[προσετάττετο] glagol otvara mjesto infinitivu; pasivno značenje (vršilac radnje izrečen prijedložnim izrazom s ὑπό); §~232
\item[δαπανᾶν] §~243; Harmid misli na prisilne javne službe (λειτουργίαι) nametnute imućnim Atenjanima, npr.\ uzdržavanje korova za dramska natjecanja (χορηγία), upravljanje vježbalištima (γυμνασιαρχία) opremanje ratnih brodova (τριηραρχία)
\item[ἀποδημῆσαι] slabi aorist §~267, \textit{verbum vocale;} nije prava složenica jer je izveden od pridjeva ἀπόδημος (ne postoji \textit{verbum simplex} δημέω)
\item[ἐξῆν] složenica glagola εἰμί; §~315; bezlični glagol otvara mjesto infinitivu
\end{description}

%2

{\large
\begin{greek}
\noindent νῦν δ' \\
ἐπειδὴ τῶν ὑπερορίων στέρομαι \\
\tabto{2em} καὶ τὰ ἔγγεια οὐ καρποῦμαι \\
\tabto{2em} καὶ τὰ ἐκ τῆς οἰκίας πέπραται, \\
ἡδέως μὲν καθεύδω ἐκτεταμένος, \\
πιστὸς δὲ τῇ πόλει γεγένημαι, \\
οὐκέτι δὲ ἀπειλοῦμαι, \\
ἀλλ' ἤδη ἀπειλῶ ἄλλοις, \\
ὡς ἐλευθέρῳ τε ἔξεστί μοι \\
\tabto{2em} καὶ ἀποδημεῖν καὶ ἐπιδημεῖν· \\
ὑπανίστανται δέ μοι ἤδη καὶ θάκων \\
καὶ ὁδῶν ἐξίστανται οἱ πλούσιοι.\\

\end{greek}
}

\begin{description}[noitemsep]
\item[ἐπειδὴ] vremenski i uzročni veznik ovdje uvodi uzročnu rečenicu §~468
\item[ὑπερορίων στέρομαι] §~232; \textgreek{στέρομαί τινος §~403.1; τὰ ὑπερόρια} inozemni posjedi (izvan Atike)
\item[τὰ ἔγγεια οὐ καρποῦμαι] §~243; τὰ ἔγγεια ili ἔγγαια tuzemni posjedi (u Atici)
\item[τὰ ἐκ τῆς οἰκίας πέπραται] πιπράσκω §~327.11; perfekt §~272; \textgreek{τὰ ἐκ τῆς οἰκίας} osobna imovina
\item[καθεύδω] §~231
\item[ἐκτεταμένος] složenica glagola τείνω, str.~118; perf. medpas. §~286; adverbni particip sa značenjem načina Smyth 2060, 2062
\item[πιστὸς\dots\ γεγένημαι] stekao sam povjerenje; §~325.11; perfekt §~272; kopulativno γίγνομαι Smyth 917; πιστός τινι pridjev ima pasivno značenje (ne „onaj koji je vjeran nekome” nego „onaj u kojeg drugi imaju povjerenje”)
\item[ἡδέως μὲν\dots\ πιστὸς δὲ\dots\ οὐκέτι δὲ\dots] koordinacija rečeničnih članova pomoću čestica μέν\dots\ δέ\dots
\item[ἀπειλοῦμαι\dots\ ἀπειλῶ] §~232; izmjena glagolskih vrsta ima stilsku vrijednost
\item[τε\dots\ καὶ\dots\ καὶ\dots] ne samo\dots\ nego i §~513.2 
\item[ἔξεστί μοι] §~315, Bilj. 2
\item[ἀποδημεῖν\dots\ ἐπιδημεῖν] §~243; infinitiv kao subjet uz bezlične izraze §~492
\item[ὑπανίστανται\dots\ ἐξίστανται] složenice glagola ἵστημι; §~305; §~311.2; značenje medija §~448.1; Harmid možda ovdje sebe zamišlja kao potencijalnog birača ili porotnika
\item[θάκων\dots\ ὁδῶν] §~402 \textit{genetivus separationis}
\item[οἱ πλούσιοι] supstantivirani pridjev §~373
\end{description}

%3

{\large
\begin{greek}
\noindent καὶ εἰμὶ νῦν μὲν τυράννῳ ἐοικώς, \\
τότε δὲ σαφῶς δοῦλος ἦν· \\
καὶ τότε μὲν ἐγὼ φόρον ἀπέφερον τῷ δήμῳ, \\
νῦν δὲ ἡ πόλις τέλος φέρουσα \\
\tabto{2em} τρέφει με. \\

\end{greek}
}

\begin{description}[noitemsep]
\item[νῦν μέν\dots\ τότε δὲ\dots] koordinacija rečeničnih članova pomoću čestica μέν\dots\ δέ\dots\ (i vremenskih priloga)
\item[εἰμί\dots\ ἐοικώς] §~315; ἔοικά τινι §~411.2; §~278.2; perifrastični oblici perfekta Smyth 599
\item[δοῦλος ἦν] §~315; kopulativni glagol s imenicom kao predikatnom dopunom (imenskim dijelom predikata) Smyth 910
\item[τότε μέν ... νῦν δὲ] koordinacija rečeničnih članova pomoću čestica μέν\dots\ δέ\dots\ (i vremenskih priloga, kao gore; paralelizam ima i stilsku vrijednost)
\item[ἀπέφερον] složenica glagola φέρω; §~231
\item[τέλος φέρουσα] τὸ τέλος porez, danak (ovdje se misli na naknadu siromašnima); φέρουσα sc.\ μοι; §~231, §~327.5; adverbni particip sa značenjem načina, Smyth 2060, 2062
\item[τρέφει] §~231; s.~118
\end{description}

{\large
\begin{greek}
\noindent ἀλλὰ καὶ Σωκράτει, \\
\tabto{2em} ὅτε μὲν πλούσιος ἦν, \\
ἐλοιδόρουν με \\
\tabto{2em} ὅτι συνῆν, \\
νῦν δ' \\
\tabto{2em} ἐπεὶ πένης γεγένημαι, \\
οὐκέτι οὐδὲν μέλει οὐδενί. \\

\end{greek}
}

\begin{description}[noitemsep]
\item[ὅτε μὲν\dots\ νῦν δ'\dots] koordinacija rečeničnih članova pomoću čestica μέν\dots\ δέ\dots
\item[ὅτε\dots\ πλούσιος ἦν] veznik ὅτε uvodi vremensku rečenicu; kopulativni glagol s pridjevom kao predikatnom dopunom (imenskim dijelom predikata) Smyth 910; §~315 
\item[ἐλοιδόρουν] §~243; rekcija τινα (ὅτι iskazuje što su govorili kritizirajući ga) §~411.2, Smyth 1591.b
\item[συνῆν] sc.\ Σωκράτει; složenica glagola εἰμί §~315, rekcija τινι
\item[πένης γεγένημαι] kopulativni glagol s pridjevom kao predikatnom dopunom (imenskim dijelom predikata) Smyth 910; §~325.11, perfekt §~272
\item[μέλει] μέλει μοι §~325.16; §~411.2 

\end{description}

{\large
\begin{greek}
\noindent καὶ μὴν \\
\tabto{2em} ὅτε μέν γε πολλὰ εἶχον, \\
\tabto{2em} ἀεί τι ἀπέβαλλον \\
\tabto{4em} ἢ ὑπὸ τῆς πόλεως \\
\tabto{4em} ἢ ὑπὸ τῆς τύχης· \\
νῦν δὲ \\
\tabto{2em} ἀποβάλλω μὲν οὐδέν \\
(οὐδὲ γὰρ ἔχω), \\
\tabto{2em} ἀεὶ δέ τι λήψεσθαι ἐλπίζω.\\

\end{greek}
}

\begin{description}[noitemsep]
\item[καὶ μήν] „i zbilja” \textit{(et vero)} §~515.4
\item[ὅτε μέν γε\dots\ νῦν δέ\dots] koordinacija rečeničnih članova pomoću čestica μέν\dots\ δέ\dots; postponirano γε §~519.1
\item[ὅτε\dots\ εἶχον] veznik ὅτε uvodi vremensku rečenicu; §~327.13; augment §~236; §~231 
\item[ἀπέβαλλον] složenica glagola βάλλω; §~231
\item[ἤ\dots\ ἤ\dots] disjunktivno ἤ §~514.1
\item[ὑπὸ τῆς πόλεως\dots\ ὑπὸ τῆς τύχης] §~437.A.β) ὑπό τινος za oznaku uzroka ili utjecaja
\item[ἀποβάλλω] složenica glagola βάλλω; §~231
\item[γάρ] postponirana čestica označuje uzrok §~517
\item[ἔχω] §~327.13; §~231
\item[λήψεσθαι] §~321.14; §~futur 258; futur zbog značenja glagola ἐλπίζω
\item[ἐλπίζω] ovdje „očekivati”, LSJ s.~v.\ I.A; otvara mjesto infinitivu (najčešće futura) §~231; tvorba osnove §~251

\end{description}

%kraj
