%\section*{O autoru}
% nedovršeno!


\section*{O tekstu}

\textit{Gozba} (ili \textit{Pijanka}, \textgreek[variant=ancient]{Συμπόσιον)} Atenjanina Ksenofonta (oko 430.–355.\ pr.~Kr.) žanrovski pripada tzv.\ sokratičkim dijalozima. Uz Platonovo djelo i nekoliko fragmenata Eshina iz atičke deme Sfeta \textgreek[variant=ancient]{(Αἰσχίνης Σφήττιος),} ti su dijalozi sve što danas imamo od nekoć bujnog spisateljstva posvećenog atenskom mudracu. 

Pored \textit{Sokratove obrane} \textgreek[variant=ancient]{(Ἀπολογία Σωκράτους)} i \textit{Uspomena na Sokrata} \textgreek[variant=ancient]{(Ἀπομνημονεύματα Σωκράτους),} \textit{Gozba} tvori najdojmljiviji dio Ksenofontove sokratičke trijade. To je opis gozbe koju je atenski bogataš Kalija priredio u čast pobjede u pankratiju \textgreek[variant=ancient]{(παγκράτιον,} šakanje i hrvanje) njegova ljubavnika Autolika na Panatenejama 422.\ pr.~Kr. 

Ksenofontov \textit{Simpozij,} prikazujući manje filozofsku gozbu, s muzičarima, akrobatskim plesačima i mimom (na samom kraju djela plesna skupina mimički prikazuje kako su Dioniz i Arijadna proveli ljubavnu noć, što nam pruža rijetko svjedočanstvo o izvedbama mima prije helenizma), vjerojatno je nastao nakon onog Platonova. Ksenofontov Sokrat drži dug govor posvećen homoseksualnoj ljubavi. Pritom strogo razlikuje ljubav koja teži fizičkom zadovoljenju, osuđujući je kao odvratnu i beskorisnu, od plemenite ljubavi, u kojoj se ljubavnik trsi ohrabriti ono najbolje u svom voljenome, pokazujući tako da zaslužuje mladićevo divljenje. Takvu su dihotomiju sigurno odobravali brojni Ksenofontovi suvremenici. Platon pak, kao kudikamo bolji psiholog, u \textit{Fedru} (253e-263e) pokazuje da obje vrste ljubavi mogu supostojati u istoj osobi. 

U sljedećem odlomku Harmid odgovara na pitanje domaćina Kalije zašto se toliko ponosi svojim siromaštvom \textgreek[variant=ancient]{(δι᾿ ὅ τι ἐπὶ πενίᾳ μέγα φρονεῖς;).} Ostavši bez imetka, bivši bogataš Harmid postaje slobodan čovjek koji živi na teret države. Imućni mu podilaze, gledajući u njemu mogućeg birača i porotnika. Ne treba više strahovati za vlastitu i sigurnost svoje imovine i ne mora imati posla s ucjenjivačima. Oslobođen brige za svoju zemlju i lišen svih nameta, napokon mirno spava. Ne može ništa izgubiti jer nema što izgubiti. Naprotiv, može samo dobiti.

%\newpage

\section*{Pročitajte naglas grčki tekst.}

Xen. Symposium 4.30–32

%Naslov prema izdanju

\medskip


{\large

\begin{greek}

\noindent ἐγὼ τοίνυν ἐν τῇδε τῇ πόλει ὅτε μὲν πλούσιος ἦν πρῶτον μὲν ἐφοβούμην μή τίς μου τὴν οἰκίαν διορύξας καὶ τὰ χρήματα λάβοι καὶ αὐτόν τί με κακὸν ἐργάσαιτο· ἔπειτα δὲ καὶ τοὺς συκοφάντας ἐθεράπευον, εἰδὼς ὅτι παθεῖν μᾶλλον κακῶς ἱκανὸς εἴην ἢ ποιῆσαι ἐκείνους. καὶ γὰρ δὴ καὶ προσετάττετο μὲν ἀεί τί μοι δαπανᾶν ὑπὸ τῆς πόλεως, ἀποδημῆσαι δὲ οὐδαμοῦ ἐξῆν.

\noindent νῦν δ´ ἐπειδὴ τῶν ὑπερορίων στέρομαι καὶ τὰ ἔγγεια οὐ καρποῦμαι καὶ τὰ ἐκ τῆς οἰκίας πέπραται, ἡδέως μὲν καθεύδω ἐκτεταμένος, πιστὸς δὲ τῇ πόλει γεγένημαι, οὐκέτι δὲ ἀπειλοῦμαι, ἀλλ´ ἤδη ἀπειλῶ ἄλλοις, ὡς ἐλευθέρῳ τε ἔξεστί μοι καὶ ἀποδημεῖν καὶ ἐπιδημεῖν· ὑπανίστανται δέ μοι ἤδη καὶ θάκων καὶ ὁδῶν ἐξίστανται οἱ πλούσιοι.

\noindent καὶ εἰμὶ νῦν μὲν τυράννῳ ἐοικώς, τότε δὲ σαφῶς δοῦλος ἦν· καὶ τότε μὲν ἐγὼ φόρον ἀπέφερον τῷ δήμῳ, νῦν δὲ ἡ πόλις τέλος φέρουσα τρέφει με. ἀλλὰ καὶ Σωκράτει, ὅτε μὲν πλούσιος ἦν, ἐλοιδόρουν με ὅτι συνῆν, νῦν δ´ ἐπεὶ πένης γεγένημαι, οὐκέτι οὐδὲν μέλει οὐδενί. καὶ μὴν ὅτε μέν γε πολλὰ εἶχον, ἀεί τι ἀπέβαλλον ἢ ὑπὸ τῆς πόλεως ἢ ὑπὸ τῆς τύχης· νῦν δὲ ἀποβάλλω μὲν οὐδέν (οὐδὲ γὰρ ἔχω), ἀεὶ δέ τι λήψεσθαι ἐλπίζω. 

\end{greek}

}


\section*{Analiza i komentar}

%1

{\large
\begin{greek}
\noindent ἐγὼ τοίνυν 
ἐν τῇδε τῇ πόλει 
ὅτε μὲν πλούσιος ἦν 
πρῶτον μὲν ἐφοβούμην 
μή τίς μου τὴν οἰκίαν διορύξας 
καὶ τὰ χρήματα λάβοι 
καὶ αὐτόν τί με κακὸν ἐργάσαιτο· 
ἔπειτα δὲ  
καὶ τοὺς συκοφάντας ἐθεράπευον, 
εἰδὼς ὅτι παθεῖν μᾶλλον κακῶς 
ἱκανὸς εἴην 
ἢ ποιῆσαι ἐκείνους. 
καὶ γὰρ δὴ καὶ 
προσετάττετο μὲν ἀεί 
τί μοι δαπανᾶν 
ὑπὸ τῆς πόλεως, 
ἀποδημῆσαι δὲ οὐδαμοῦ 
ἐξῆν.

\end{greek}
}

\begin{description}[noitemsep]
\item[τοίνυν]	519.5
\item[ὅτε]	veznik ὅτε uvodi zavisnu vremensku rečenicu
\item[πλούσιος ἦν] 	§~315; pridjevska dopuna kopulativnom glagolu (imenski predikat), Smyth 910
\item[πρῶτον μὲν\dots\ ἔπειτα δὲ\dots] koordinacija rečeničnih članova pomoću čestica μέν\dots\ δέ\dots
\item[ἐφοβούμην] §~243; §~328.3
\item[μή τίς ... λάβοι ... ἐργάσαιτο] iza izraza bojazni (verbum timendi φοβέομαι) negacija μή uvodi zavisnu rečenicu bojazni (bojim se da ću, da ne bih); iza sporednog vremena dolazi najčešće optativ 
\item[διορύξας] složenica glagola ὀρύττω; §~267; adverbni particip §~503
\item[λάβοι] §~321.14; jaki aorist §~254
\item[ἐργάσαιτο] ἐργάζομαί τινά τι komu što učiniti; slabi aorist §~267
\item[ἐθεράπευον] §~231
\item[εἰδὼς] §~317.4
\item[παθεῖν\dots\ κακῶς] §~327.15; jaki aorist §~254; κακῶς πάσχειν or παθεῖν štetu trpjeti, nepravdu iskusiti; infinitiv uz relativne pridjeve §~494
\item[ὅτι\dots\ ἱκανὸς εἴην] izrična rečenica uvedena veznikom ὅτι ovisna o participu εἰδὼς (verbum sentiendi); optativ u pravilu iza historijskih vremena; §~315; pridjevska dopuna kopulativnom glagolu (imenski predikat), Smyth 910
\item[ποιῆσαι] tj. κακῶς; slabi aorist §~267, verbum vocale; κακῶς ποιέω τινά; ἐκείνους (tj. τοὺς συκοφάντας)
\item[καὶ\dots\ δὴ καὶ\dots] §~513.1
\item[γὰρ] §~517
\item[προσετάττετο] §~232; pasivno značenje
\item[δαπανᾶν] §~243; misli prisilne javne službe (λειτουργίαι) nametnute imućnim Atenjanima, npr. uzdržavanje korova za dramska natjecanja (χορηγία), upravljanje vježbalištima (γυμνασιαρχία) opremanje ratnih brodova (τριηραρχία)
\item[ἀποδημῆσαι] slabi aorist §~267, verbum vocale; nije pravi verbum compositum jer je izveden od pridjeva ἀπόδημος (ne postoji verbum simplex δημέω)
\item[ἐξῆν] složenica glagola εἰμί; §~315; bezlični glagol otvara mjesto infinitivu
\end{description}

%2

{\large
\begin{greek}
\noindent νῦν δ' 
ἐπειδὴ τῶν ὑπερορίων στέρομαι 
καὶ τὰ ἔγγεια οὐ καρποῦμαι 
καὶ τὰ ἐκ τῆς οἰκίας πέπραται, 
ἡδέως μὲν καθεύδω ἐκτεταμένος, 
πιστὸς δὲ τῇ πόλει γεγένημαι, 
οὐκέτι δὲ ἀπειλοῦμαι, 
ἀλλ' ἤδη ἀπειλῶ ἄλλοις, 
ὡς ἐλευθέρῳ τε ἔξεστί μοι 
καὶ ἀποδημεῖν καὶ ἐπιδημεῖν· 
ὑπανίστανται δέ μοι ἤδη καὶ θάκων 
καὶ ὁδῶν ἐξίστανται οἱ πλούσιοι.

\end{greek}
}

\begin{description}[noitemsep]
ἐπειδὴ	veznik uvodi uzročnu rečenicu §~468
ὑπερορίων στέρομαι	§~232; στέρομαί τινος §~403.1; τὰ ὑπερόρια inozemni posjedi (tj. izvan Atike)
τὰ ἔγγεια οὐ καρποῦμαι	§~243; τὰ ἔγγεια ili ἔγγαια tuzemni posjedi (tj. u Atici)
τὰ ἐκ τῆς οἰκίας πέπραται	πιπράσκω §~327.11; perfekt §~272; τὰ ἐκ τῆς οἰκίας osobna imovina
καθεύδω	§~231, augment §~240 
ἐκτεταμένος	složenica glagola τείνω; str. 118; perf. medpas. §~286; adverbni particip sa značenjem načina, Smyth 2060, 2062
πιστὸς ... γεγένημαι	stekao sam povjerenje; §~325.11; perfekt §~272; kopulativno γίγνομαι Smyth 917; πιστός τινι pasivno značenje
ἡδέως μὲν ... πιστὸς δὲ ... 
οὐκέτι δὲ	koordinacija rečeničnih članova pomoću čestica μέν … δέ ...
ἀπειλοῦμαι ... ἀπειλῶ	§~232; pasiv ... aktiv
τε ... καὶ ... καὶ 	ne samo ... nego i §~513.2 
ἔξεστί μοι	§~315, Bilj. 2
ἀποδημεῖν ... ἐπιδημεῖν	§~243; infinitiv kao subjet uz bezlične izraze §~492
ὑπανίστανται ... ἐξίστανται 	složenice glagola ἵστημι; §~305; §~311.2; značenje medija §~448.1; Harmid možda ovdje sebe zamišlja kao potencijalnog birača ili porotnika
θάκων ... ὁδῶν	§~402 genetivus separationis; (za urednika L&S: not gĕnĭtīvus; cf. Lachm. ad Lucr. II. p. 15 sq.)
οἱ πλούσιοι	supstantivirani pridjev §~373
\end{description}

%3

{\large
\begin{greek}
\noindent καὶ εἰμὶ νῦν μὲν τυράννῳ ἐοικώς, 
τότε δὲ 
σαφῶς δοῦλος ἦν· 
καὶ τότε μὲν ἐγὼ 
φόρον ἀπέφερον τῷ δήμῳ, 
νῦν δὲ ἡ πόλις 
τέλος φέρουσα 
τρέφει με. 
ἀλλὰ καὶ Σωκράτει, 
ὅτε μὲν πλούσιος ἦν, 
ἐλοιδόρουν με 
ὅτι συνῆν, 
νῦν δ' 
ἐπεὶ πένης γεγένημαι, 
οὐκέτι οὐδὲν μέλει οὐδενί. 
καὶ μὴν 
ὅτε μέν γε πολλὰ εἶχον, 
ἀεί τι ἀπέβαλλον 
ἢ ὑπὸ τῆς πόλεως 
ἢ ὑπὸ τῆς τύχης· 
νῦν δὲ 
ἀποβάλλω μὲν οὐδέν 
(οὐδὲ γὰρ ἔχω), 
ἀεὶ δέ τι λήψεσθαι 
ἐλπίζω.

\end{greek}
}

\begin{description}[noitemsep]
νῦν μέν ... τότε δὲ	koordinacija rečeničnih članova pomoću čestica μέν… δέ…
εἰμί ... ἐοικώς	§~315; ἔοικά τινι §~411.2; §~278.2; perifrastični oblici perfekta Smyth 599
δοῦλος ἦν	§~315; kopulativni glagol s imenicom kao predikatnom dopunom
(imenskim dijelom predikata) Smyth 910
τότε μέν ... νῦν δὲ 	koordinacija rečeničnih članova pomoću čestica μέν… δέ…
ἀπέφερον	složenica glagola φέρω; §~231
τέλος φέρουσα 
	τὸ τέλος, porez, danak; ovdje se misli na naknadu siromašnima; objekt participa φέρουσα tj. μοι; §~231, §~327.5; adverbni particip sa značenjem načina, Smyth 2060, 2062
τρέφει	§~231; s. 118
ὅτε μὲν ... νῦν δ' 	koordinacija rečeničnih članova pomoću čestica μέν … δέ ...
ὅτε ... πλούσιος ἦν	veznik ὅτε uvodi vremensku rečenicu; kopulativni glagol s pridjevom kao predikatnom dopunom (imenskim dijelom predikata) Smyth 910; §~315 
ἐλοιδόρουν	§~243; rekcija τινα §~411.2, Smyth 1591.b
ὅτι συνῆν 	veznik ὅτι uvodi uzročnu rečenicu; složenica glagola εἰμί §~315, rekcija τινι (tj. Σωκράτει)
ἐπεὶ πένης γεγένημαι	veznik ἐπεί uvodi uzročnu rečenicu; §~315; kopulativni glagol s pridjevom kao predikatnom dopunom (imenskim dijelom predikata) Smyth 910; §~325.11; perfekt §~272
μέλει	μέλει μοι §~325.16; §~411.2 
καὶ μήν 	et vero §~515.4
ὅτε μέν γε ... νῦν δέ	koordinacija rečeničnih članova pomoću čestica μέν … δέ ...; postponirano γε §519.1
ὅτε ... εἶχον 
	veznik ὅτε uvodi vremensku rečenicu; §~327.13; augment §~236; §~231 
ἀπέβαλλον	složenica glagola βάλλω; §~231
ἤ ... ἤ	disjunktivno ἤ §~514.1
ὑπὸ τῆς πόλεως ...  ὑπὸ τῆς τύχης	§~437.A.β) ὑπό τινος za oznaku uzroka ili utjecaja
ἀποβάλλω	složenica glagola βάλλω; §~231
γάρ	postponirana čestica označuje uzrok §~517
ἔχω	§~327.13; §~231
λήψεσθαι	§~321.14; §~futur 258; futur zbog značenja glagola ἐλπίζω
ἐλπίζω	§~231; tvorba osnove §~251; ovdje znači očekivati (LSJ ἐλπίζω I. A look for, expect); otvara mjesto infinitivu (obično) futura

\end{description}



%kraj
