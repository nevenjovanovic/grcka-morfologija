%\section*{O autoru}

%TKTK


\section*{O tekstu}

Ἐνάλιοι διάλογοι, \textit{Dijalozi morskih bogova}, naziv je zbirke petnaest kratkih razgovora u kojima se sirijski retoričar i satiričar Lukijan šali na račun kanonske grčke mitologije.

U odlomku koji čitamo kiklop Polifem pripovijeda ocu Posejdonu kako ga je oslijepio Odisej. Kiklop je na početku bio prilično samouvjeren: za njega je donijeti čitavo stablo iz planine tek sitnica, isto kao i obračunati se s nekolicinom Odisejevih drugova. No, ti bezveznjaci nadmudrili su ga pomoću opojnog sredstva i kolca.


%\newpage

\section*{Pročitajte naglas grčki tekst.}

Luc. Dialogi marini 2.2

%Naslov prema izdanju
%NČ

\medskip


{\large

\begin{greek}

\noindent (Κύκλωψ) κατέλαβον ἐν τῷ ἄντρῳ ἀπὸ τῆς νομῆς ἀναστρέψας πολλούς τινας, ἐπιβουλεύοντας δῆλον ὅτι τοῖς ποιμνίοις· ἐπεὶ γὰρ ἐπέθηκα τῇ θύρᾳ τὸ πῶμα — πέτρα δέ ἐστι παμμεγέθης — καὶ τὸ πῦρ ἀνέκαυσα ἐναυσάμενος ὃ ἔφερον δένδρον ἀπὸ τοῦ ὄρους, ἐφάνησαν ἀποκρύπτειν αὑτοὺς πειρώμενοι· ἐγὼ δὲ συλλαβών τινας αὐτῶν, ὥσπερ εἰκὸς ἦν, κατέφαγον λῃστάς γε ὄντας. ἐνταῦθα ὁ πανουργότατος ἐκεῖνος, εἴτε Οὖτις εἴτε Ὀδυσσεὺς ἦν, δίδωσί μοι πιεῖν φάρμακόν τι ἐγχέας, ἡδὺ μὲν καὶ εὔοσμον, ἐπιβουλότατον δὲ καὶ ταραχωδέστατον· ἅπαντα γὰρ εὐθὺς ἐδόκει μοι περιφέρεσθαι πιόντι καὶ τὸ σπήλαιον αὐτὸ ἀνεστρέφετο καὶ οὐκέτι ὅλως ἐν ἐμαυτοῦ ἤμην, τέλος δὲ ἐς ὕπνον κατεσπάσθην. ὁ δὲ ἀποξύνας τὸν μοχλὸν καὶ πυρώσας γε προσέτι ἐτύφλωσέ με καθεύδοντα, καὶ ἀπ´ ἐκείνου τυφλός εἰμί σοι, ὦ Πόσειδον.


\end{greek}

}


\section*{Analiza i komentar}

%1


{\large
\begin{greek}
\noindent Κατέλαβον \\
\tabto{2em} αὐτοὺς \\
\tabto{2em} ἐν τῷ ἄντρῳ \\
ἀπὸ τῆς νομῆς \\
ἀναστρέψας \\
\tabto{2em} πολλούς τινας, \\
\tabto{2em} ἐπιβουλεύοντας \\
\tabto{4em} δῆλον ὅτι \\
\tabto{4em} τοῖς ποιμνίοις·\\
ἐπεὶ γὰρ ἐπέθηκα \\
\tabto{2em} τῇ θύρᾳ \\
\tabto{2em} τὸ πῶμα\\
— πέτρα δέ \\
ἐστι \\
\tabto{2em} παμμεγέθης —\\
καὶ τὸ πῦρ \\
ἀνέκαυσα \\
\tabto{2em} ἐναυσάμενος \\
\tabto{4em} ὃ ἔφερον δένδρον \\
\tabto{6em} ἀπὸ τοῦ ὄρους, \\
ἐφάνησαν \\
\tabto{2em} ἀποκρύπτειν αὑτοὺς \\
\tabto{2em} πειρώμενοι·\\
ἐγὼ δὲ \\
\tabto{2em} συλλαβών \\
\tabto{4em} τινας αὐτῶν, \\
ὥσπερ εἰκὸς ἦν, \\
κατέφαγον \\
\tabto{2em} λῃστάς γε ὄντας.\\

\end{greek}
}

\begin{description}[noitemsep]
\item[Κατέλαβον] rekcija: τινα; §~254, §~238, §~321.14, §~460.1, §~454; ima dopunu (priložnu oznaku)  ἐν τῷ ἄντρῳ
\item[ἀναστρέψας] §~267, §~301.B (s.~118), §~498; ima dopunu (priložnu oznaku) ἀπὸ τῆς νομῆς
\item[ἐπιβουλεύοντας] rekcija: τινι; §~231
\item[δῆλον ὅτι] od izraza δῆλόν ἐστιν ὅτι, „jasno je da“, koji je otvarao mjesto izričnoj rečenici; izostavljanjem kopule i čestim korištenjem s vremenom se izgubila potreba za glagolom i dopunom – ὅτι gubi snagu veznika, a značenje postaje adverbno: očito\dots
\item[ἐπέθηκα] rekcija: τινά τινι; §~305, §~306, §~311, §~238, §~460.1, §~454
\item[ἐπεὶ... ἐπέθηκα] veznik ἐπεὶ otvara mjesto zavisnoj vremenskoj rečenici: kada…
\item[ἐστι] §~315; dio imenskog predikata (Smyth 909)
\item[ἀνέκαυσα] §~267, §~253, §~301.B (s.~116), §~460.1, §~454
\item[ἐναυσάμενος] §~267, §~301.B (s.~116), §~498
\item[ἔφερον] §~231, §~327.5, §~460.1, §~452
\item[ὃ ἔφερον] odnosna zamjenica ὃ uvodi umetnutu odnosnu zavisnu rečenicu
\item[ἐφάνησαν] §~294, §~299, §~301.B (s.~118), §~448, §~460.1, §~454; ovaj glagol otvara mjesto participu kao nužnoj dopuni%, redoslijed: ἐφάνησαν πειρώμενοι ἀποκρύπτειν αὑτοὺς
\item[ἀποκρύπτειν] §~231, §~ 301.B (s.~116), §~490
\item[πειρώμενοι] §~243, §~249, §~301.B (s.~116), §~498; ovaj particip otvara mjesto dopuni u infinitivu, ἀποκρύπτειν
\item[συλλαβών] §~254, §~255, §~333.6, §~ 238, §~321.14, §~498
\item[δὲ] čestica daje surečenici adverzativno značenje: a…
\item[ἦν] §~315, §~460.1, §~452
\item[εἰκὸς] §~278.2 bilješka, §~498
\item[εἰκὸς ἦν] §~461.1, imenski predikat (Smyth 909)
\item[ὥσπερ εἰκὸς ἦν] umetnuta poredbena rečenica: kao što… (Smyth 2462)
\item[κατέφαγον] §~255, §~333.6, §~238, §~327.8, §~460.1, §~454
\item[γε] čestica naglašava, gotovo pretvara u uzvik razlog Kiklopova postupka: ``pa bili su...!''
\item[ὄντας] §~315

\end{description}

%2


{\large
\begin{greek}
\noindent ἐνταῦθα \\
ὁ πανουργότατος ἐκεῖνος, \\
\tabto{2em} εἴτε Οὖτις \\
\tabto{2em} εἴτε ᾿Οδυσσεὺς ἦν, \\
δίδωσί μοι πιεῖν \\
\tabto{2em} φάρμακόν τι ἐγχέας, \\
\tabto{2em} ἡδὺ μὲν καὶ εὔοσμον, \\
\tabto{2em} ἐπιβουλότατον δὲ καὶ ταραχωδέστατον·\\
ἅπαντα γὰρ \\
\tabto{2em} εὐθὺς \\
ἐδόκει μοι \\
\tabto{2em} περιφέρεσθαι \\
\tabto{2em} πιόντι \\
καὶ τὸ σπήλαιον αὐτὸ \\
\tabto{2em} ἀνεστρέφετο \\
καὶ οὐκέτι \\
\tabto{2em} ὅλως \\
\tabto{2em} ἐν ἐμαυτοῦ ἤμην, \\
τέλος δὲ \\
\tabto{2em} εἰς ὕπνον \\
κατεσπάσθην.\\

\end{greek}
}

\begin{description}[noitemsep]
\item[εἴτε... εἴτε] §~514.2
\item[ἦν] §~315
\item[δίδωσί] §~305, §~311, §~460.1, §~451, veže dopunu πιεῖν
\item[πιεῖν] §~254, §~255, §~327.16, §~490
\item[ἐγχέας] §~269, §~238, §~498
\item[ἐδόκει] §~243, §~325.2, §~460.1, §~452
\item[πιόντι] §~254, §~255, §~327.16, §~498
\item[ἐδόκει μοι ] §~439, bilješka, §~492, §~460.1, §~452; §~460.1, §~452; ἐδόκει μοι πιόντι
\item[περιφέρεσθαι] §~232
\item[ἀνεστρέφετο] §~232, §~238
\item[ἐν ἐμαυτοῦ] prijedlog ἐν ne može otvarati mjesto genitivu; radi se o eliptičnoj upotrebi, izostavlja se riječ na koju se odnosi ἐμαυτοῦ: ἐν ἐμαυτοῦ οἰκίᾳ (εἶναι) pri sebi (biti); LSJ ἐν A. I. 2. 
\item[ἤμην] §~315; kasniji oblik, čest kod Lukijana = ἦν
\item[κατεσπάσθην] §~299, §~301.B (s.~116), §~238, §~460.1, §~454
\item[καὶ… καὶ… δὲ] rečenice prvo ustrojene usporedno (καὶ… καὶ), nabrajanjem, završnu misao uvodi adverzativno δὲ: i…i…a

\end{description}


%3


{\large
\begin{greek}
\noindent ὁ δὲ \\
\tabto{2em} ἀποξύνας τὸν μοχλὸν \\
\tabto{2em} καὶ πυρώσας \\
\tabto{2em} προσέτι \\
ἐτύφλωσέ \\
\tabto{2em} με καθεύδοντα, \\
καὶ ἀπ' ἐκείνου \\
τυφλός εἰμί σοι, \\
ὦ Πόσειδον.\\

\end{greek}
}

\begin{description}[noitemsep]
\item[ἀποξύνας] §~267, §~238, §~301.B (s.~118), §~498, §~503
\item[πυρώσας] §~267, §~243, §~301.B (s.~116), §~498, §~503
\item[ἐτύφλωσέ] §~267, §~243, §~301.B (s.~116), §~460.1, §~454
\item[καθεύδοντα] §~240, §~498, §~499%(Smyth s.~698), 
\item[εἰμί] §~315
\item[τυφλός εἰμί] imenski predikat (Smyth 909)

\end{description}



%kraj
