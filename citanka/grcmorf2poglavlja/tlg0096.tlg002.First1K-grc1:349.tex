%\section*{O autoru}

% Unio ispravke NZ <2021-12-28 uto>


\section*{O tekstu}

U ovoj basni, sačuvanoj pod imenom glasovitog basnopisca Ezopa, razmatra se moć i neizbježnost sudbine.

Neki otac usne loš san o sinu jedincu. U strahu za njegov život poduzme sve kako se san ne bi nikada ostvario. No, brinući se prvenstveno za sinovljevu sigurnost, nije mislio na to je li sin i sretan. Osjećaj frustracije pokrenut će nesretan slijed događaja i tako će sin susresti svoju sudbinu, mada pred nacrtanim lavom.

%\newpage

\section*{Pročitajte naglas grčki tekst.}

Aesop. Fabulae 279 (349)

%Naslov prema izdanju
%NČ

\medskip


{\large

\begin{greek}

\noindent ΠΑΙΣ, ΠΑΤΗΡ ΚΑΙ ΛΕΩΝ ΓΕΓΡΑΜΜΕΝΟΣ

\medskip

\noindent Υἱόν τις γέρων δειλὸς μονογενῆ ἔχων γενναῖον, κυνηγεῖν ἐφιέμενον, εἶδε τοῦτον καθʼ ὕπνους ὑπὸ λέοντος ἀναλωθέντα. Φοβηθεὶς δέ, μήπως ὁ ὄνειρος ἀληθεύσῃ, οἴκημα κάλλιστον καὶ μετέωρον κατεσκεύασε· κἀκεῖσε τὸν υἱὸν εἰσαγαγὼν ἐφύλαττεν. Ἐζωγράφησε δὲ ἐν τῷ οἰκήματι πρὸς τέρψιν τοῦ υἱοῦ παντοῖα ζῷα, ἐν οἷς ἦν καὶ λέων. Ὁ δὲ ταῦτα μᾶλλον ὁρῶν πλείονα λύπην εἶχε. Καὶ δήποτε πλησίον τοῦ λέοντος στὰς εἶπεν· ``ὦ κάκιστον θηρίον, διὰ σὲ καὶ τὸν ψευδῆ ὄνειρον τοῦ ἐμοῦ πατρὸς τῇδε τῇ οἰκίᾳ κατεκλείσθην, ὡς ἐν φρουρᾷ· τί σοι ποιήσω;'' Καὶ εἰπὼν ἐπέβαλε τῷ τοίχῳ τὴν χεῖρα, ἐκτυφλῶσαι τὸν λέοντα. Σκόλοψ δὲ τῷ δακτύλῳ αὐτοῦ ἐμπαρεὶς, ὄγκωμα καὶ φλεγμονὴν μέχρι βουβῶνος εἰργάσατο· πυρετὸς δὲ ἐπιγενόμενος αὐτῷ θᾶττον τοῦ βίου μετέστησεν. Ὁ δὲ λέων καὶ οὕτως ἀνῄρηκε τὸν παῖδα, μηδὲν τῷ τοῦ πατρὸς ὠφεληθέντα σοφίσματι.

Ὁ μῦθος δηλοῖ, ὅτι οὐδεὶς δύναται τὸ μέλλον ἐκφυγεῖν.

\end{greek}

}


\section*{Analiza i komentar}

%1


{\large
\begin{greek}
\noindent ΠΑΙΣ, ΠΑΤΗΡ ΚΑΙ ΛΕΩΝ ΓΕΓΡΑΜΜΕΝΟΣ

\end{greek}
}

\begin{description}[noitemsep]
\item[ΓΕΓΡΑΜΜΕΝΟΣ] §~272; s.\ 116, §~277
\end{description}

%2


{\large
\begin{greek}
\noindent υἱόν\\
τις γέρων δειλὸς \\
\tabto{2em} μονογενῆ \\
ἔχων \\
\tabto{2em} γενναῖον \\
\tabto{2em} κυνηγεῖν ἐφιέμενον \\
εἶδε τοῦτον \\
\tabto{2em} καθ' ὕπνους \\
\tabto{4em} ὑπὸ λέοντος \\
\tabto{2em} ἀναλωθέντα.\\

\end{greek}
}

\begin{description}[noitemsep]
\item[ἔχων] §~231, §~499 atributni particip
\item[κυνηγεῖν] §~243
\item[ἐφιέμενον] §~305, §~238, §~311, §~396.e, §~498
\item[εἶδε] §~254, §~237.2, §~327.3, §~460.1, §~454
\item[ἀναλωθέντα] §~296, §~237.1, §~324.6, §~498
\end{description}


%3


{\large
\begin{greek}
\noindent φοβηθεὶς δέ, \\
\tabto{2em} μήπως ὁ ὄνειρος ἀληθεύσῃ, \\
οἴκημα \\
\tabto{2em} κάλλιστον καὶ μετέωρον \\
κατεσκεύασε \\
κἀκεῖσε \\
\tabto{4em} τὸν υἱὸν \\
\tabto{2em} εἰσαγαγὼν \\
ἐφύλαττεν.\\

\end{greek}
}

\begin{description}[noitemsep]
\item[φοβηθεὶς] §~296, §~298, §~301.B (s.\ 116), §~498; sc.\ τις γέρων δειλὸς
\item[ἀληθεύσῃ] §~301.B (s.\ 116)
\item[μήπως ὁ ὄνειρος ἀληθεύσῃ] namjerna rečenica koje uvodi glagol bojazni, izriče bojazan da bi se nešto moglo desiti (umjesto μήπως uobičajeniji su veznici μή ili ὅπως μή)
\item[κατεσκεύασε] §~267, §~238, §~301.B (s. 118), §~460.1, §~454
\item[εἰσαγαγὼν] §~238, §~301.B (s. 116), §~498
\item[εἰσαγαγὼν τὸν υἱὸν] i \textbf{ἐφύλαττεν τὸν υἱὸν} oba glagola imaju istu dopunu u direktnom objektu pa se ne objekt ne ponavlja; u hrvatskom se objekt najčešće izriče: „uvevši sina, čuvao ga je“
\item[ἐφύλαττεν] §~231
\end{description}

%4


{\large
\begin{greek}
\noindent ἐζωγράφησε δὲ \\
\tabto{2em} ἐν τῷ οἰκήματι \\
\tabto{2em} πρὸς τέρψιν \\
\tabto{4em} τοῦ υἱοῦ \\
παντοῖα ζῷα, \\
\tabto{2em} ἐν οἷς ἦν \\
\tabto{2em} καὶ λέων.\\

\end{greek}
}

\begin{description}[noitemsep]
\item[ἐζωγράφησε] §~269, §~267, §~301.B (s.~116), §~460.1, §~454
\item[ἦν] §~315
\item[ἐν οἷς ἦν] prijedložni izraz s odnosnom zamjenicom ἐν οἷς uvodi zavisnu odnosnu rečenicu „među kojima\dots“
\end{description}

%5


{\large
\begin{greek}
\noindent ὁ δὲ \\
\tabto{4em} ταῦτα \\
\tabto{2em} μᾶλλον ὁρῶν \\
πλείονα λύπην \\
εἶχε.\\

\end{greek}
}

\begin{description}[noitemsep]
\item[ὁ δὲ] adverzativna čestica δὲ uvodi novi subjekt, ``a\dots'', sc.\ ὁ δὲ υἱός
\item[ὁρῶν] §~243
\item[εἶχε] §~231, §~327.13, §~236
\item[λύπην εἶχε] = ἐλυπεῖτο, ``bio je tužan''
\end{description}


%6


{\large
\begin{greek}
\noindent καὶ δή ποτε \\
πλησίον \\
\tabto{2em} τοῦ λέοντος \\
στὰς \\
εἶπεν·\\

\end{greek}
}

\begin{description}[noitemsep]
\item[στὰς] §~306, §~305, §~310, §~311.2, §~329
\item[εἶπεν] §~254, §~238, §~327.7, §~460.1, §~454

\end{description}


%7


{\large
\begin{greek}
\noindent ``ὦ κάκιστον θηρίον, \\
διὰ σὲ \\
\tabto{2em} καὶ τὸν ψευδῆ ὄνειρον \\
\tabto{4em} τοῦ ἐμοῦ πατρὸς \\
τῇδε τῇ οἰκίᾳ \\
κατεκλείσθην \\
\tabto{2em} ὡς ἐν φρουρᾷ· \\
τί σοι ποιήσω;''\\

\end{greek}
}

\begin{description}[noitemsep]
\item[κατεκλείσθην] §~296, §~238, §~299, §~460.1, §~454
\item[ποιήσω] §~260, §~301.A, §~301.B (s.~116), §~460.1, §~455
\item[τί\dots\ ποιήσω;] upitna rečenica s (dubitativnim) konjunktivom, uvodi je upitna zamjenica τί, ``što da\dots''
\end{description}


%8


{\large
\begin{greek}
\noindent καὶ εἰπὼν \\
ἐπέβαλε \\
\tabto{2em} τῷ τοίχῳ \\
\tabto{2em} τὴν χεῖρα \\
\tabto{2em} ἐκτυφλῶσαι \\
\tabto{4em} τὸν λέοντα.\\

\end{greek}
}

\begin{description}[noitemsep]
\item[εἰπὼν] §~254, §~255, §~333.B, §~238, §~327.7, §~498
\item[ἐπέβαλε] §~254, §~301.B (s. 118), §~460.1, §~454
\item[ἐκτυφλῶσαι] §~267, §~301.B (s. 116), §~243, §~238, §~490; infinitiv izriče namjeru (finalni infinitiv)%inf. izriče namjeru
\end{description}

%9


{\large
\begin{greek}
\noindent σκόλοψ δὲ \\
\tabto{2em} τῷ δακτύλῳ \\
\tabto{4em} αὐτοῦ \\
ἐμπαρεὶς \\
ὄγκωμα καὶ φλεγμονὴν \\
\tabto{2em} μέχρι βουβῶνος \\
εἰργάσατο, \\
πυρετὸς δὲ ἐπιγενόμενος \\
\tabto{2em} αὐτῷ \\
θᾶττον \\
\tabto{2em} τοῦ βίου \\
μετέστησεν.\\

\end{greek}
}

\begin{description}[noitemsep]
\item[ἐμπαρεὶς] §~292, §~238, §~301.B (s. 118), §~498
\item[εἰργάσατο] §~267, §~235, §~236, §~260.1, §~454
\item[ἐπιγενόμενος] §~254, §~238, §~325.11, §~498
\item[μετέστησεν] §~305, §~311, §~260.1, §~454

\end{description}

%10


{\large
\begin{greek}
\noindent ὁ δὲ λέων \\
καὶ οὕτως \\
ἀνῄρηκε \\
τὸν παῖδα \\
μηδὲν \\
\tabto{2em} τῷ \\
\tabto{4em} τοῦ πατρὸς \\
ὠφεληθέντα \\
\tabto{2em} σοφίσματι.\\

\end{greek}
}

\begin{description}[noitemsep]
\item[ἀνῄρηκε] §~ 272, §~235-237, §~238, §~327.1, §~460.1, §~456
\item[ὠφεληθέντα] rekcija: τινος τινι, „na korist za koga u čemu“; §~296, §~243, §~235, §~498
\end{description}

%11

{\large
\begin{greek}
\noindent ὁ μῦθος δηλοῖ, \\
\tabto{2em} ὅτι \\
\tabto{4em} οὐδεὶς \\
\tabto{4em} δύναται \\
\tabto{8em} τὸ μέλλον \\
\tabto{6em} ἐκφυγεῖν.\\

\end{greek}
}

\begin{description}[noitemsep]
\item[δηλοῖ] §~243, §~460.1, §~451
\item[δύναται] §~312.5, §~328.2, §~234, §~460.1, §~451
\item[ὅτι\dots\ δύναται] glagol δηλοῖ otvara mjesto izričnom vezniku ὅτι koji uvodi izričnu rečenicu: „da…“ 
\item[ἐκφυγεῖν] §~254, §~255, §~333.2, §~301.B (s. 116), §~490
\end{description}



%kraj
