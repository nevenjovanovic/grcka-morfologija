% Unesi ispravke NZ <2021-12-31 pet>
\section*{O autoru}

Λόγγος (Longo, II–III st.\ po~Kr.) je autor helenističkog ljubavnog romana, poput Ahileja Tacija, Haritona, Ksenofonta Efeškog i Heliodora. O Longu nema pouzdanih biografskih podataka, čak ni što se imena tiče; ni njegova povezanost s otokom Lezbom nije zajamčena.

\section*{O tekstu}

Longov roman \textgreek{Τῶν κατὰ Δάφνιν καὶ Χλόην λόγοι Δ} (\textit{Četiri pripovijesti o Dafnisu i Hloji}), ili \textgreek{ΤΠοιμενικὰ τὰ κατὰ Δάφνιν καὶ Χλόην} (\textit{Pastirske zgode Dafnisa i Hloje}), kraće \textgreek{ΤΔάφνις καὶ Χλόη,} pripovijeda tipičnu priču grčkog ljubavnog romana, ali na mnogo jednostavniji način. Dvoje prekrasnih mladih ljudi zaljubi se jedno u drugo; ljubav nailazi na niz prepreka koje im ne daju da budu zajedno; nakon niza peripetija koje uključuju i dva neočekivana prepoznavanja, sve se rješava, otkriva se da Dafnis i Hloja nisu ubogi pastiri, nego bogati zemljoposjednici; nakon svadbe žive sretni i zadovoljni na rodnome Lezbu.

Specifičnost je Longova romana bukolski, pastirski ambijent, i nadomještanje ``geografskih'' peripetija (putovanja razdvojenih ljubavnika) psihološkima: dvoje mladih polako i postupno shvaća da su zaljubljeni, prolazeći svojevrstan ``sentimentalni odgoj'' (grčki roman inače konvencionalno prikazuje ``ljubav na prvi pogled''). Također, ritam pastirskog života, koji dovodi do razdvojenosti zaljubljenih, u ovom romanu nadomješta konvencionalne igre sudbine i mahinacije zlotvora.

Odlomak iz prve knjige \textit{Pastirskih zgoda Dafnisa i Hloje} prikazuje petnaestogodišnjeg Dafnisa i trinaestogodišnju Hloju, čuvare ovaca i koza, koji se u proljeće, provodeći na paši čitav dan zajedno, svake večeri razilaze, svatko svojoj kući. Dafnis je jednog dana upao u zamku za vukove, izvukao se neozlijeđen, ali se morao oprati. Hloja je ostala opčinjena njegovom ljepotom, a zaključila je da je uzrok iznenadne privlačnosti kupanje. Sljedećeg dana, dok stada pasu, Dafnis svira siringu (Panovu frulu).

%\newpage

\section*{Pročitajte naglas grčki tekst.}

Longus scr.~erot.\ Daphnis et Chloe 1.13.4–1.13.6

%Naslov prema izdanju

\medskip


{\large

\begin{greek}

\noindent Τῆς δὲ ἐπιούσης ὡς ἧκον εἰς τὴν νομήν, ὁ μὲν Δάφνις ὑπὸ τῇ δρυῒ τῇ συνήθει καθεζόμενος ἐσύριττε καὶ ἅμα τὰς αἶγας ἐπεσκόπει κατακειμένας καὶ ὥσπερ τῶν μελῶν ἀκροωμένας, ἡ δὲ Χλόη πλησίον καθημένη τὴν ἀγέλην μὲν τῶν προβάτων ἐπέβλεπε, τὸ δὲ πλέον εἰς Δάφνιν ἑώρα· καὶ ἐδόκει καλὸς αὐτῇ συρίττων πάλιν, καὶ αὖθις αἰτίαν ἐνόμιζε τὴν μουσικὴν τοῦ κάλλους, ὥστε μετ´ ἐκεῖνον καὶ αὐτὴ τὴν σύριγγα ἔλαβεν, εἴ πως γένοιτο καὶ αὐτὴ καλή.

Ἔπεισε δὲ αὐτὸν καὶ λούσασθαι πάλιν καὶ λουόμενον εἶδε καὶ ἰδοῦσα ἥψατο καὶ ἀπῆλθε πάλιν ἐπαινέσασα, καὶ ὁ ἔπαινος ἦν ἔρωτος ἀρχή. Ὅ τι μὲν οὖν ἔπασχεν οὐκ ᾔδει νέα κόρη καὶ ἐν ἀγροικίᾳ τεθραμμένη καὶ οὐδὲ ἄλλου λέγοντος ἀκούσασα τὸ τοῦ ἔρωτος ὄνομα· ἄση δὲ αὐτῆς εἶχε τὴν ψυχήν, καὶ τῶν ὀφθαλμῶν οὐκ ἐκράτει καὶ πολλὰ ἐλάλει Δάφνιν· τροφῆς ἠμέλει, νύκτωρ ἠγρύπνει, τῆς ἀγέλης κατεφρόνει· νῦν ἐγέλα, νῦν ἔκλαεν· εἶτα ἐκάθευδεν, εἶτα ἀνεπήδα\dots

\end{greek}

}


\section*{Analiza i komentar}

%1

{\large
\begin{greek}
\noindent Τῆς δὲ ἐπιούσης \\
\tabto{2em} ὡς ἧκον \\
\tabto{4em} εἰς τὴν νομήν, \\
ὁ μὲν Δάφνις \\
\tabto{2em} ὑπὸ τῇ δρυῒ τῇ συνήθει \\
\tabto{4em} καθεζόμενος \\
\tabto{6em} ἐσύριττε \\
καὶ ἅμα \\
\tabto{4em} τὰς αἶγας ἐπεσκόπει κατακειμένας \\
\tabto{6em} καὶ ὥσπερ τῶν μελῶν ἀκροωμένας, \\
ἡ δὲ Χλόη \\
\tabto{2em} πλησίον καθημένη \\
\tabto{2em} τὴν ἀγέλην μὲν τῶν προβάτων \\
\tabto{4em} ἐπέβλεπε, \\
\tabto{2em} τὸ δὲ πλέον \\
\tabto{4em} εἰς Δάφνιν ἑώρα· \\
καὶ ἐδόκει καλὸς \\
\tabto{2em} αὐτῇ \\
συρίττων πάλιν, \\
καὶ αὖθις \\
\tabto{2em} αἰτίαν ἐνόμιζε \\
\tabto{4em} τὴν μουσικὴν \\
\tabto{6em} τοῦ κάλλους, \\
ὥστε μετ´ ἐκεῖνον \\
\tabto{2em} καὶ αὐτὴ \\
\tabto{4em} τὴν σύριγγα ἔλαβεν, \\
\tabto{6em} εἴ πως γένοιτο \\
\tabto{6em} καὶ αὐτὴ \\
\tabto{8em} καλή.\\

\end{greek}
}

\begin{description}[noitemsep]
\item[ἐπιούσης] podrazumijeva se ἡμέρας; LSJ ἔπειμι B.II; §~314.1
\item[ἧκον] §~231
\item[ὁ μὲν Δάφνις\dots\ ἡ δὲ Χλόη] koordinacija rečeničnih članova
\item[καθεζόμενος] §~232
\item[ἐσύριττε] §~231; συρίττω je atički oblik glagola συρίζω
\item[ἐπεσκόπει] §~231; složenica σκοπέω
\item[κατακειμένας] složenica κεῖμαι §~315.a
\item[ἀκροωμένας] §~243, glagol ἀκροάομαι ima samo medijalne oblike
\item[καθημένη] §~315.a
\item[τὴν ἀγέλην μὲν\dots\ τὸ δὲ πλέον\dots] koordinacija rečeničnih članova; LSJ πλείων, II.2 kao prilog: više, ponajprije
\item[ἐπέβλεπε] §~231; složenica βλέπω
\item[ἑώρα] nepravilan glagol ὁράω §~327.3
\item[ἐδόκει] §~231; δοκέω otvara mjesto dativu osobe i infinitivu, ovdje imenskom predikatu: τινί τις (εἶναι)
\item[συρίττων] §~231
\item[ἐνόμιζε] §~231; otvara mjesto dvama akuzativima LSJ νομίζω II
\item[ἔλαβεν] §~254, osnove §~321.14
\item[εἴ πως] uvodi hipotetičku finalnu rečenicu: ne bi li kako\dots\ u nadi da\dots\ §~479.4; Smyth 2354: εἴ nadajući se da\dots\ ne bi li kojim slučajem\dots
\item[γένοιτο] §~254, osnove §~325.11; kao kopulativni glagol (nepotpuna značenja) traži imensku dopunu (ovdje pridjev)

\end{description}

%2

{\large
\begin{greek}
\noindent Ἔπεισε δὲ αὐτὸν \\
\tabto{2em} καὶ λούσασθαι πάλιν \\
καὶ λουόμενον \\
\tabto{2em} εἶδε \\
καὶ ἰδοῦσα \\
\tabto{2em} ἥψατο \\
καὶ ἀπῆλθε \\
\tabto{2em} πάλιν ἐπαινέσασα, \\
καὶ ὁ ἔπαινος ἦν \\
\tabto{2em} ἔρωτος ἀρχή.\\

\end{greek}
}

\begin{description}[noitemsep]
\item[Ἔπεισε] §~267; πείθω osnove s. 118; rekcija: τινά τι, ovdje otvara mjesto dopuni u infinitivu
\item[λούσασθαι] §~267
\item[λουόμενον] §~232
\item[εἶδε] §~254; ὁράω osnove §~327.3
\item[ἰδοῦσα] §~254; ὁράω osnove §~327.3
\item[ἥψατο] §~267; §~269 (verba muta)
\item[ἀπῆλθε] §~254; složenica glagola ἔρχομαι §~327.2
\item[ἐπαινέσασα] §~267, §~269 (verba vocalia)
\item[ἦν\dots\ ἀρχή] §~315; imenski predikat, Smyth 909

\end{description}


%3

{\large
\begin{greek}
\noindent  Ὅ τι μὲν οὖν ἔπασχεν \\
\tabto{2em} οὐκ ᾔδει \\
\tabto{4em} νέα κόρη \\
\tabto{4em} καὶ ἐν ἀγροικίᾳ τεθραμμένη \\
\tabto{4em} καὶ οὐδὲ ἄλλου λέγοντος ἀκούσασα \\
\tabto{6em} τὸ τοῦ ἔρωτος ὄνομα· \\
ἄση δὲ \\
\tabto{2em} αὐτῆς εἶχε τὴν ψυχήν, \\
καὶ τῶν ὀφθαλμῶν \\
\tabto{2em} οὐκ ἐκράτει \\
καὶ πολλὰ ἐλάλει Δάφνιν· \\
τροφῆς ἠμέλει, \\
νύκτωρ ἠγρύπνει, \\
τῆς ἀγέλης κατεφρόνει· \\
νῦν ἐγέλα, \\
νῦν ἔκλαεν· \\
εἶτα ἐκάθευδεν, \\
εἶτα ἀνεπήδα\dots\\

\end{greek}
}

\begin{description}[noitemsep]
\item[Ὅ τι μὲν\dots\ ἄση δὲ\dots] koordinacija rečeničnih članova; ὅ τι je oblik upitne zamjenice ὅστις, uvodi zavisno upitnu rečenicu
\item[ἔπασχεν] §~231
\item[ᾔδει] §~317.4
\item[τεθραμμένη] glagol τρέφω, osnove s.~118; §~272; §~285
\item[λέγοντος] §~231
\item[ἀκούσασα] §~267; rekcija je u ovom slučaju: τί τινος \textit{što od koga}
\item[εἶχε] §~327.13
\item[ἐκράτει] §~231; rekcija: τινος
\item[πολλά] priložna upotreba, LSJ πολύς III.a
\item[ἐλάλει] §~231; rekcija: τινα \textit{o nekome}
\item[ἠμέλει] §~231; §~235 (temporalni augment); rekcija: τινος
\item[ἠγρύπνει] §~231; §~235 (temporalni augment)
\item[κατεφρόνει] §~231; složenica φρονέω, rekcija: τινος
\item[ἐγέλα] §~231; §~243
\item[ἔκλαεν] §~231; κλάω je varijanta glagola κλαίω, LSJ s.~v.
\item[ἐκάθευδεν] §~231
\item[ἀνεπήδα] §~231; složenica πηδάω
\end{description}


%kraj
