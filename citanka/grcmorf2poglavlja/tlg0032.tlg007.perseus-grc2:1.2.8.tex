% <2021-12-26 ned> Unio ispravke NZ
% \section*{O autoru}

%TKTK


\section*{O tekstu}

Ksenofontovo djelo Κύρου παιδεία, \textit{Kirov odgoj}, nastalo je vjerojatno poslije 371.\ pr.~Kr. U osnovi ono je fiktivna, romansirana biografija perzijskog vladara Kira Velikog (oko 600.–530.\ pr.~Kr.). Prikazujući Kirov život Ksenofont ujedno prikazuje i svoje političke ideje. Utoliko djelo pripada i političkom žanru πολιτεία (poput Platonove \textit{Države} ili djela Starog oligarha). Opisi perzijskih običaja i priča Ksenofonta pak povezuju s tradicijom etnografije drugog, ne-grčkog (tu tradiciju predstavlja Herodot). 

Iz aspekta dugotrajnih grčkih ratova s Perzijancima mogli bi začuditi odabir teme i pohvalan ton djela. No, Grke Perzija zanima i fascinira odavno, a samog Kira Velikog, koji je vladao četrdesetak godina prije početka grčko-perzijskih sukoba, grčka književnost pozitivno prikazuje još od Eshila. 

\textit{Kirov odgoj} uzoran je primjer klasične atičke proze IV.~st.\ pr.~Kr. te se već u antici smatrao remek-djelom. Ponovno je otkriven u srednjem vijeku kao jedan od uzora za književnu vrstu \textit{ogledalo vladara} (na primjer, za Machiavellijevo djelo \textit{Vladar}). 

Ovdje odabrani ulomak potječe iz prve knjige \textit{Kirova odgoja} i na vrlo općenit način govori o perzijskom obrazovnom sustavu opisujući odgoj dječaka do 16 odnosno 17 godina. Posebno se naglašava uloga starijih kao uzora te disciplina pri prehrani. Naglasak na usvajanju umjerenosti ili razumnog pristupanja životu (σωφροσύνη) doziva u sjećanje etiku Ksenofontova učitelja Sokrata.

\newpage

\section*{Pročitajte naglas grčki tekst.}

Xen.\ Cyropaedia 1.2.8

%Naslov prema izdanju

\medskip


{\large

\begin{greek}

\noindent  διδάσκουσι δὲ τοὺς παῖδας καὶ σωφροσύνην· μέγα δὲ συμβάλλεται εἰς τὸ μανθάνειν σωφρονεῖν αὐτοὺς ὅτι καὶ τοὺς πρεσβυτέρους ὁρῶσιν ἀνὰ πᾶσαν ἡμέραν σωφρόνως διάγοντας. διδάσκουσι δὲ αὐτοὺς καὶ πείθεσθαι τοῖς ἄρχουσι· μέγα δὲ καὶ εἰς τοῦτο συμβάλλεται ὅτι ὁρῶσι τοὺς πρεσβυτέρους πειθομένους τοῖς ἄρχουσιν ἰσχυρῶς. διδάσκουσι δὲ καὶ ἐγκράτειαν γαστρὸς καὶ ποτοῦ· μέγα δὲ καὶ εἰς τοῦτο συμβάλλεται ὅτι ὁρῶσι τοὺς πρεσβυτέρους οὐ πρόσθεν ἀπιόντας γαστρὸς ἕνεκα πρὶν ἂν ἀφῶσιν οἱ ἄρχοντες, καὶ ὅτι οὐ παρὰ μητρὶ σιτοῦνται οἱ παῖδες, ἀλλὰ παρὰ τῷ διδασκάλῳ, ὅταν οἱ ἄρχοντες σημήνωσι. φέρονται δὲ οἴκοθεν σῖτον μὲν ἄρτον, ὄψον δὲ κάρδαμον, πιεῖν δέ, ἤν τις διψῇ, κώθωνα, ὡς ἀπὸ τοῦ ποταμοῦ ἀρύσασθαι. πρὸς δὲ τούτοις μανθάνουσι καὶ τοξεύειν καὶ ἀκοντίζειν. μέχρι μὲν δὴ ἓξ ἢ ἑπτακαίδεκα ἐτῶν ἀπὸ γενεᾶς οἱ παῖδες ταῦτα πράττουσιν, ἐκ τούτου δὲ εἰς τοὺς ἐφήβους ἐξέρχονται.

\end{greek}

}


\section*{Analiza i komentar}

%1


{\large
\begin{greek}
\noindent  διδάσκουσι δὲ \\
τοὺς παῖδας \\
καὶ σωφροσύνην· \\
μέγα δὲ συμβάλλεται \\
\tabto{2em} εἰς τὸ μανθάνειν \\
\tabto{4em} σωφρονεῖν \\
\tabto{2em} αὐτοὺς \\
ὅτι καὶ τοὺς πρεσβυτέρους ὁρῶσιν \\
\tabto{4em} ἀνὰ πᾶσαν ἡμέραν \\
\tabto{2em} σωφρόνως διάγοντας. \\

\end{greek}
}

\begin{description}[noitemsep]
\item[διδάσκουσι] rekcija: τινα τι; §~231, §~324.9, §~386, §~460.1, §~451
\item[δὲ] §~515.2
\item[συμβάλλεται] §~231, §~301.B (s.~118), §~238, §~460.1, §~451
\item[μανθάνειν] §~231, §~321.17, §~490
\item[εἰς τὸ μανθάνειν...  αὐτοὺς] §~497, supstantivirani akuzativ s infinitivom, dopuna predikatu συμβάλλεται 
\item[σωφρονεῖν] §~243, §~301.B (s.~116), §~490; objekt infinitiva μανθάνειν
\item[ὁρῶσιν] §~243, §~301.B (s.~116), §~460.1, §~451
\item[ὅτι\dots\ ὁρῶσιν] zavisna izrična rečenica, ``da\dots'' ili ``to što\dots'', §~467
\item[διάγοντας] §~498, §~301.B (s.~116), §~238; glagol ἄγω i njegove složenice, poput διάγω, mogu imati i prijelazno i neprijelazno značenje; u ovom nam slučaju to pomaže odrediti dopuna, priloga σωφρόνως
\end{description}

%2

{\large
\begin{greek}
\noindent  διδάσκουσι δὲ \\
αὐτοὺς \\
\tabto{2em} καὶ πείθεσθαι \\
\tabto{4em} τοῖς ἄρχουσι· \\
μέγα δὲ καὶ \\
\tabto{2em} εἰς τοῦτο \\
συμβάλλεται \\
\tabto{2em} ὅτι ὁρῶσι \\
\tabto{2em} τοὺς πρεσβυτέρους \\
\tabto{4em} πειθομένους \\
\tabto{6em} τοῖς ἄρχουσιν \\
\tabto{4em} ἰσχυρῶς.\\

\end{greek}
}

\begin{description}[noitemsep]
\item[διδάσκουσι] rekcija: τινα τι; §~231, §~324.9, §~386, §~460.1, §~451
\item[πείθεσθαι] rekcija: τινι, §~411.2; §~231, §~301.B (s.~118), §~490; dopuna glagola διδάσκουσι
\item[συμβάλλεται] §~231, §~301.B (s.~118), §~238, §~460.1, §~451
\item[ὁρῶσι] §~243, §~301.B (s.~116), §~460.1, §~451
\item[ὅτι ὁρῶσι] zavisna izrična rečenica, ``da\dots'' ili ``to što\dots'' §~467
\item[πειθομένους] rekcija: τινι, §~411.2; §~231, §~301.B (s.~118), §~490

\end{description}

%3


{\large
\begin{greek}
\noindent  διδάσκουσι δὲ καὶ \\
ἐγκράτειαν \\
\tabto{2em} γαστρὸς καὶ ποτοῦ·\\
μέγα δὲ καὶ \\
\tabto{2em} εἰς τοῦτο \\
συμβάλλεται \\
\tabto{2em} ὅτι ὁρῶσι \\
\tabto{4em} τοὺς πρεσβυτέρους \\
\tabto{4em} οὐ πρόσθεν ἀπιόντας \\
\tabto{6em} γαστρὸς ἕνεκα \\
\tabto{4em} πρὶν ἂν ἀφῶσιν οἱ ἄρχοντες, \\
\tabto{2em} καὶ ὅτι \\
\tabto{4em} οὐ παρὰ μητρὶ \\
\tabto{2em} σιτοῦνται \\
\tabto{2em} οἱ παῖδες, \\
\tabto{4em} ἀλλὰ παρὰ τῷ διδασκάλῳ, \\
\tabto{4em} ὅταν οἱ ἄρχοντες σημήνωσι. \\

\end{greek}
}

\begin{description}[noitemsep]
\item[διδάσκουσι] rekcija: τινα τι; §~231, §~324.9, §~386, §~460.1, §~451
\item[συμβάλλεται] §~231, §~301.B (s.~118), §~238, §~460.1, §~451
\item[ὁρῶσι] §~243, §~301.B (s.~116), §~460.1, §~451
\item[ὅτι ὁρῶσι] zavisna izrična rečenica, ``da\dots'' ili ``to što\dots'', §~467
\item[ἀπιόντας] §~314.1, §~498
\item[ἀφῶσιν] §~305-309, §~311, §~460.2, §~463
\item[ἂν ἀφῶσιν] konjunktiv s ἂν §~489.b 
\item[πρὶν ἂν ἀφῶσιν] zavisna vremenska rečenica, “dok ne\dots”, §~488.2, πρὶν s konjuktivom aorista označava općenitu vremenitost, vremensku radnju bez fiksnog vremenskog referenta
\item[σιτοῦνται] §~243, §~301.B (s.~116), §~460.1, §~451
\item[ὅτι\dots\ σιτοῦνται] izrična zavisna rečenica, ``da\dots'' ili ``to što\dots'', §~467
\item[σημήνωσι] §~460.2, §~463, §~301.B (s.~118)
\item[ὅταν\dots\ σημήνωσι] zavisna vremenska rečenica u značenju pogodbe eventualnog oblika (ponavljanje u sadašnjosti), ``kad god'' + prezent, §~488.2


\end{description}


%4


{\large
\begin{greek}
\noindent  φέρονται δὲ \\
\tabto{4em} οἴκοθεν \\
\tabto{2em} σῖτον μὲν ἄρτον, \\
\tabto{2em} ὄψον δὲ κάρδαμον, \\
\tabto{2em} πιεῖν δέ, \\
\tabto{4em} ἤν τις διψῇ, \\
\tabto{4em} κώθωνα, \\
\tabto{4em} ὡς ἀπὸ τοῦ ποταμοῦ ἀρύσασθαι.\\

\end{greek}
}

\begin{description}[noitemsep]
\item[φέρονται] §~231, §~327.5, §~460.1, §~451
\item[σῖτον μὲν\dots\ ὄψον δὲ\dots] \textbf{πιεῖν δέ\dots}\ koordinacija rečeničnih članova pomoću čestica μέν\dots\ δέ\dots
\item[πιεῖν] §~231, §~327.16, §~490; infinitiv izriče svrhu i ima objekt κώθωνα
\item[διψῇ] §~243, §~301.B (s.~116), 
\item[ἤν\dots\ διψῇ] zavisna rečenica, eventualna pogodba (ponavljanje u sadašnjosti), “kad god + prezent”, §~488.2
\item[ἀρύσασθαι] §~231, §~267, §~301, §~490
\item[ὡς\dots\ ἀρύσασθαι] zavisna posljedična rečenica s veznikom ὡς, §~473

\end{description}


%5


{\large
\begin{greek}
\noindent  πρὸς δὲ τούτοις \\
μανθάνουσι \\
\tabto{2em} καὶ τοξεύειν \\
\tabto{2em} καὶ ἀκοντίζειν. \\

\end{greek}
}

\newpage

\begin{description}[noitemsep]
\item[μανθάνουσι] §~231, §~460.1, §~451
\item[τοξεύειν] §~301.B (s.~116), §~490, infinitiv kao objektna dopuna glavnog glagola μανθάνουσι, §~493
\item[ἀκοντίζειν] §~301.B (s.~118), §~490, infinitiv kao objektna dopuna glavnog glagola μανθάνουσι, §~493

\end{description}


%6


{\large
\begin{greek}
\noindent  μέχρι μὲν δὴ ἓξ ἢ ἑπτακαίδεκα ἐτῶν \\
\tabto{4em} ἀπὸ γενεᾶς \\
\tabto{2em} οἱ παῖδες \\
\tabto{2em} ταῦτα πράττουσιν, \\
ἐκ τούτου δὲ \\
\tabto{4em} εἰς τοὺς ἐφήβους \\
\tabto{2em} ἐξέρχονται.\\

\end{greek}
}

\begin{description}[noitemsep]
\item[μέχρι μὲν\dots\ ἐκ τούτου δὲ\dots] koordinacija rečeničnih članova pomoću čestica μέν\dots\ δέ\dots
\item[μὲν δὴ] ovom se kombinacijom čestica povjesničari često koriste kao formulom prijelaza, surečenica s μὲν δὴ služi kao rezime prethodno rečenog: ``tako dakle\dots'' (Denniston, \textit{Greek Particles}, 258)%Denniston
\item[πράττουσιν] §~231, §~301.B (s.~116), §~460.1, §~451
\item[ἐξέρχονται] §~231, §~238, §~327.2, §~460.1, §~451

\end{description}


%kraj
