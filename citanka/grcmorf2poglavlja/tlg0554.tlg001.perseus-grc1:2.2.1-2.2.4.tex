% Unio ispravke NZ <2022-01-01 sub>
\section*{O autoru}

Hariton \textgreek[variant=ancient]{(Χαρίτων),} autor ljubavnog romana Zgode Hereje i Kaliroje \textgreek[variant=ancient]{(Τὰ περὶ Χαιρέαν καὶ Καλλιρόην),} živio je najvjerojatnije između 50.\ pr.~Kr. i 50.\ po~Kr. Malo se zna o njemu osim onoga što je sam naveo na početku svog romana: da je živio u maloazijskom gradu Afrodizijadi i da je bio tajnik retora Atenagore.

\section*{O tekstu}

Roman \textit{Zgode Hereje i Kaliroje} dugo se smatrao kronološki posljednjim grčkim ljubavnim romanom antike, ali su papirusni nalazi s jedne i jezična analiza s druge strane pokazali da je zapravo jedan od najranijih. 

U romanu susrećemo tipične žanrovske motive: ljubav dvoje iznimno lijepih, plemenitih i osjećajnih mladih ljudi, putovanja egzotičnim zemljama, gusare. Glavni su junaci Hereja i Kaliroja, dvoje sicilskih Grka koji se zaljube na prvi pogled te se vrlo brzo i vjenčaju, no zavist drugih mladića koji su se borili za Kalirojinu ruku pokrene niz događaja zbog kojih Hereja mora krenuti u potragu za svojom dragom. Ta će ga potraga odvesti sve do Babilona, gdje će se za nju boriti na sudu, a potom će se istaknuti i kao sposoban vojskovođa u Egiptu. Za to vrijeme Kaliroja, oteta i prodana u roblje, za dobrobit svojeg nerođenog djeteta (čiji je otac Hereja) sklapa brak s maloazijskim Grkom Dionizijem. 

U odlomku koji slijedi Kaliroju dočekuju Dionizijeve sluškinje i dive se njezinoj ljepoti.


%\newpage

\section*{Pročitajte naglas grčki tekst.}

Charito, Scr.\ Erot.\ De Chaerea et Callirhoe 2.2.1–4
%Naslov prema izdanju

\medskip


{\large

\begin{greek}

\noindent Πρὸς δὲ τὴν Καλλιρρόην εἰσῆλθον αἱ ἄγροικοι γυναῖκες καὶ εὐθὺς ὡς δέσποιναν ἤρξαντο κολακεύειν. Πλαγγὼν δέ, ἡ τοῦ οἰκονόμου γυνή, ζῶον οὐκ ἄπρακτον, ἔφη πρὸς αὐτὴν ``ζητεῖς μέν, ὦ τέκνον, πάντως τοὺς σεαυτῆς· ἀλλὰ καλῶς καὶ τοὺς ἐνθάδε νόμιζε σούς· Διονύσιος γάρ, ὁ δεσπότης ἡμῶν, χρηστός ἐστι καὶ φιλάνθρωπος. Εὐτυχῶς σε ἤγαγεν εἰς ἀγαθὴν ὁ θεὸς οἰκίαν. Ὥσπερ ἐν πατρίδι διάξεις. Ἐκ μακρᾶς οὖν θαλάσσης ἀπόλουσαι τὴν ἄσιν· ἔχεις θεραπαινίδας.'' Μόλις μὲν καὶ μὴ βουλομένην, προήγαγε δ' ὅμως εἰς τὸ βαλανεῖον. Εἰσελθοῦσαν δὲ ἤλειψάν τε καὶ ἀπέσμηξαν ἐπιμελῶς καὶ μᾶλλον ἀποδυσαμένης κατεπλάγησαν· ὥστε ἐνδεδυμένης αὐτῆς θαυμάζουσαι τὸ πρόσωπον θεῖον πρόσωπον ἔδοξαν ἰδοῦσαι· ὁ χρὼς γὰρ λευκὸς ἔστιλψεν εὐθὺς μαρμαρυγῇ τινι ὅμοιον ἀπολάμπων· τρυφερὰ δὲ σάρξ, ὥστε δεδοικέναι μὴ καὶ ἡ τῶν δακτύλων ἐπαφὴ μέγα τραῦμα ποιήσῃ.

\noindent Ἡσυχῆ δὲ διελάλουν πρὸς ἀλλήλας ``καλὴ μὲν ἡ δέσποινα ἡμῶν καὶ περιβόητος· ταύτης δὲ ἂν θεραπαινὶς ἔδοξεν.'' Ἐλύπει τὴν Καλλιρρόην ὁ ἔπαινος καὶ τοῦ μέλλοντος οὐκ ἀμάντευτος ἦν. Ἐπεὶ δὲ ἐλέλουτο καὶ τὴν κόμην συνεδέσμουν, καθαρὰς αὐτῇ προσήνεγκαν ἐσθῆτας· ἡ δὲ οὐ πρέπειν ἔλεγε ταῦτα τῇ νεωνήτῳ.

\noindent ``Χιτῶνά μοι δότε δουλικόν· καὶ γὰρ ὑμεῖς ἐστέ μου κρείττονες.'' Ἐνεδύσατο μὲν οὖν τι τῶν ἐπιτυχόντων· κἀκεῖνο δὲ ἔπρεπεν αὐτῇ καὶ πολυτελὲς ἔδοξε καταλαμπόμενον ὑπὸ κάλλους.


\end{greek}

}

%\newpage

\section*{Analiza i komentar}

%1

{\large
\begin{greek}
\noindent Πρὸς δὲ τὴν Καλλιρόην \\
εἰσῆλθον \\
αἱ ἄγροικοι γυναῖκες \\
καὶ εὐθὺς \\
\tabto{2em} ὡς δέσποιναν \\
ἤρξαντο κολακεύειν.\\

\end{greek}
}

\begin{description}[noitemsep]
\item[δὲ] čestica označava nadovezivanje na prethodnu rečenicu
\item[εἰσῆλθον] §~254; εἰσέρχομαι, složenica ἔρχομαι §~327.2
\item[εὐθὺς] prilog od εὐθύς (LSJ εὐθύς B)
\item[ἤρξαντο] §~269
\item[κολακεύειν] §~231; rekcija τινά (hrv.\ laskati nekome)

\end{description}

%2


{\large
\begin{greek}
\noindent Πλαγγὼν δέ, \\
ἡ τοῦ οἰκονόμου γυνή, \\
ζῶον οὐκ ἄπρακτον, \\
ἔφη \\
\tabto{2em} πρὸς αὐτὴν \\
ζητεῖς μέν, ὦ τέκνον, \\
πάντως \\
τοὺς ἑαυτῆς· \\
ἀλλὰ καλῶς \\
καὶ τοὺς ἐνθάδε\\
νόμιζε σούς· \\
Διονύσιος γάρ, \\
\tabto{2em} ὁ δεσπότης ἡμῶν, \\
χρηστός ἐστι καὶ φιλάνθρωπος.\\

\end{greek}
}


\begin{description}[noitemsep]
\item[δὲ] čestica označava nadovezivanje na prethodnu rečenicu
\item[ἔφη] §~312.8
\item[πρὸς αὐτὴν] priložna oznaka koja ovdje zamjenjuje dativ 
\item[ζητεῖς] §~231, §~243
\item[μέν] postpozitivna čestica: zaista\dots
\item[ἀλλὰ] čestica suprotnog značenja: ali\dots
\item[καὶ τοὺς ἐνθάδε] sc.\ stanovnike ovog mjesta
\item[νόμιζε] §~231
\item[γάρ] čestica najavljuje iznošenje razloga ili objašnjenja: naime\dots
\item[ἐστι] §~315
\item[χρηστός ἐστι καὶ φιλάνθρωπος] imenski predikat Smyth 909

\end{description}


%3

{\large
\begin{greek}
\noindent Εὐτυχῶς σε ἤγαγεν \\
\tabto{2em} εἰς ἀγαθὴν \\
ὁ θεὸς \\
\tabto{2em} οἰκίαν.\\
 Ὥσπερ ἐν πατρίδι \\
\tabto{2em} διάξεις.\\

\end{greek}
}

\begin{description}[noitemsep]
\item[ἤγαγεν] s. 116, §~257
\item[διάξεις] složenica glagola ἄγω, s. 116, §~258

\end{description}

%4

{\large
\begin{greek}
\noindent ἐκ μακρᾶς οὖν θαλάσσης \\
ἀπόλουσαι \\
τὴν ἄσιν· \\
ἔχεις θεραπαινίδας.\\

\end{greek}
}

\begin{description}[noitemsep]
\item[ἐκ μακρᾶς\dots\ θαλάσσης] od duge plovidbe morem
\item[οὖν] čestica zaključnog značenja: dakle\dots
\item[ἀπόλουσαι] §~267, složenica λούω
\item[ἔχεις] §~327.13, §~231

\end{description}

%5

{\large
\begin{greek}
\noindent Μόλις μὲν \\
\tabto{2em} καὶ μὴ βουλομένην, \\
προήγαγε δ' ὅμως \\
\tabto{2em} εἰς τὸ βαλανεῖον.\\

\end{greek}
}

\begin{description}[noitemsep]
\item[μὲν\dots\ δ'] koordinacija parom čestica (prvi dio ima dopusno značenje, kako se naslućuje iz ὅμως)
\item[βουλομένην] §~232
\item[προήγαγε] sc.\ αὐτήν; subjekt je Πλαγγών s početka odlomka; glagol je složenica ἄγω, s.~16, §~257

\end{description}

%6

{\large
\begin{greek}
\noindent εἰσελθοῦσαν δὲ \\
ἤλειψάν τε καὶ ἀπέσμηξαν ἐπιμελῶς \\
καὶ μᾶλλον ἀποδυσαμένης \\
κατεπλάγησαν\\
\tabto{2em} ὥστε,\\
\tabto{4em} ἐνδεδυμένης αὐτῆς \\
\tabto{2em} θαυμάζουσαι \\
\tabto{4em} τὸ πρόσωπον \\
\tabto{2em} θεῖον πρόσωπον \\
\tabto{4em} ἔδοξαν \\
\tabto{6em} ἰδοῦσαι· \\
ὁ χρὼς γὰρ λευκὸς \\
ἔστιλψεν \\
\tabto{2em} εὐθὺς \\
\tabto{2em} μαρμαρυγῇ τινι ὅμοιον \\
ἀπολάμπων· \\
τρυφερὰ δὲ σάρξ, \\
\tabto{2em} ὥστε δεδοικέναι \\
\tabto{4em} μὴ καὶ ἡ τῶν δακτύλων ἐπαφὴ \\
\tabto{4em} μέγα τραῦμα \\
\tabto{4em} ποιήσῃ. \\

\end{greek}
}

\begin{description}[noitemsep]
\item[εἰσελθοῦσαν] složenica ἔρχομαι §~327.2, §~254
\item[δὲ] označava nadovezivanje na prethodnu rečenicu
\item[ἤλειψάν] §~267, ἀλείφω; subjekt ovog i svih daljnjih finitnih glagola jesu \textgreek{θεραπαινίδαι}
\item[ἀπέσμηξαν] ἀποσμήχω §~267
\item[ἀποδυσαμένης] §~267; genitiv apsolutni §~504 (ovdje bez nominalnog dijela)
\item[κατεπλάγησαν] složenica πλήσσω, §~327.10; §~292
\item[ὥστε] veznik otvara mjesto posljedičnoj rečenici §~473
\item[ἐνδεδυμένης αὐτῆς] genitiv apsolutni §~504; složenica glagola δύω; §~272, §~504
\item[θαυμάζουσαι] rekcija: τι; §~231
\item[ἔδοξαν] §~267
\item[ἰδοῦσαι] §~327.3, §~254
\item[γὰρ] uvodi objašnjenje: naime\dots
\item[ἔστιλψεν] §~267 §~269
\item[εὐθὺς] prilog od εὐθύς (LSJ εὐθύς B)
\item[ὅμοιον] prilog; rekcija ὅμοιος τινι (LSJ C)
\item[ἀπολάμπων] §~231
\item[δὲ] nadovezivanje na prethodnu tvrdnju: a\dots
\item[τρυφερὰ δὲ σάρξ] u imenskom predikatu izostavljena je kopula
\item[ὥστε] uvodi posljedičnu rečenicu, ovdje s infinitivom, §~473 
\item[δεδοικέναι] §~317.3, §~272; subjekt infinitiva mogao bi biti neodređeni subjektni akuzativ τινά
\item[μὴ] uvodi rečenicu uz \textit{verbum timendi}, §~471
\item[ποιήσῃ] §~243, §~267

\end{description}

%7

{\large
\begin{greek}
\noindent ἡσυχῆ δὲ διελάλουν \\
\tabto{2em} πρὸς ἀλλήλας \\
``καλὴ μὲν \\
\tabto{2em} ἡ δέσποινα ἡμῶν \\
καὶ περιβόητος· \\
ταύτης δὲ \\
\tabto{2em} ἂν θεραπαινὶς ἔδοξεν.''\\

\end{greek}
}

\begin{description}[noitemsep]
\item[δὲ] čestica označava nadovezivanje na prethodnu rečenicu
\item[διελάλουν] §~231, §~243
\item[πρὸς ἀλλήλας] međusobno
\item[μὲν\dots\ δὲ] koordiniranim parom čestica izriče se suprotnost rečeničnih članaka: ali\dots
\item[καλὴ\dots\ καὶ περιβόητος] neizrečena je kopula imenskog predikata
\item[ἂν\dots\ ἔδοξεν] ireal (uz neizrečenu irealnu protazu): (da stane uz tu djevojku) izgledala bi kao njezina sluškinja; §~323.2, §~489.b.1; kopulativni glagol ima nužnu imensku dopunu

\end{description}

%8


{\large
\begin{greek}
\noindent ἐλύπει τὴν Καλλιρόην ὁ ἔπαινος \\
καὶ \\
\tabto{2em} τοῦ μέλλοντος \\
οὐκ ἀμάντευτος ἦν. \\

\end{greek}
}

\begin{description}[noitemsep]
\item[ἐλύπει] §~231, §~243
\item[ἀμάντευτος ἦν] imenski predikat (Smyth 910)
\item[ἦν] §~315
\item[τοῦ μέλλοντος] §~394; supstantivirani particip u značenju ``budućnost''

\end{description}

%9

{\large
\begin{greek}
\noindent ἐπεὶ δὲ ἐλέλουτο \\
καὶ τὴν κόμην συνεδέσμουν, \\
καθαρὰς \\
\tabto{2em} αὐτῇ \\
\tabto{4em} προσήνεγκαν \\
ἐσθῆτας· \\
ἡ δὲ \\
\tabto{2em} οὐ πρέπειν \\
ἔλεγε \\
\tabto{2em} ταῦτα \\
\tabto{4em} τῇ νεωνήτῳ. \\

\end{greek}
}

\begin{description}[noitemsep]
\item[ἐπεὶ] §~487
\item[δέ] čestica označava nadovezivanje na prethodnu rečenicu
\item[ἐλέλουτο] §~272, §~448.1
\item[συνεδέσμουν] složenica glagola δεσμεύω (δεσμέω, LSJ συνδεσμεύω) §~231, §~243
\item[προσήνεγκαν] složenica glagola φέρω §~327.5 §~328.3
\item[πρέπειν ἔλεγε ταῦτα] §~491
\item[πρέπειν] rekcija τινι; §~231
\item[ἔλεγε] §~231; glagol otvara mjesto akuzativu s infinitivom

\end{description}

%10
{\large
\begin{greek}
\noindent χιτῶνά μοι δότε \\
\tabto{2em} δουλικόν· \\
καὶ γὰρ \\
\tabto{2em} ὑμεῖς \\
\tabto{2em} ἐστέ \\
\tabto{4em} μου \\
\tabto{4em} κρείττονες.\\

\end{greek}
}

\begin{description}[noitemsep]
\item[δότε] §~305, §~306
\item[καὶ γὰρ]  §~517, kombinacija čestica s uzročnim značenjem iznosi objašnjenje: jer\dots
\item[ἐστέ] §~315
\item[ἐστέ κρείττονες] imenski predikat (Smyth 910)
\item[μου] \textit{genetivus comparationis} uz komparativ κρείττονες §~404.1

\end{description}


%11
{\large
\begin{greek}
\noindent ἐνεδύσατο μὲν οὖν \\
τι \\
\tabto{2em} τῶν ἐπιτυχόντων· \\
κἀκεῖνο δὲ ἔπρεπεν \\
\tabto{2em} αὐτῇ \\
καὶ πολυτελὲς ἔδοξε \\
\tabto{2em} καταλαμπόμενον \\
\tabto{4em} ὑπὸ κάλλους.\\

\end{greek}
}

\begin{description}[noitemsep]
\item[ἐνεδύσατο] §~267; složenica glagola δύω
\item[ἐνεδύσατο μὲν\dots\ κἀκεῖνο δὲ] koordinacija rečeničnih dijelova pomoću para čestica
\item[τῶν ἐπιτυχόντων] τι otvara mjesto genitivu partitivnom (ovdje u s.~r.) §~395
\item[ἐπιτυχόντων] složenica τυγχάνω §~321.19, §~254
\item[ἔπρεπεν] §~231
\item[ἔδοξε] §~267, §~325.2; kopulativni glagol čija je dopuna πολυτελὲς
\item[καταλαμπόμενον] §~232

\end{description}



%kraj
