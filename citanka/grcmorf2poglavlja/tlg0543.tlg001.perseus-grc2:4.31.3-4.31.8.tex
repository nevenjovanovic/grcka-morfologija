% Unio ispravke NZ <2022-01-01 sub>


\section*{O tekstu}

U četvrtoj knjizi svojih \textgreek[variant=ancient]{Ἱστορίαι} (\textit{Povijesti}) Polibije pripovijeda o grčkom Savezničkom ili Etolskom ratu. Rat je protiv Etolskog saveza, Sparte i Elide vodio Helenski savez predvođen Filipom V.\ Makedonskim, od 220.\ do 217.\ pr.~Kr.

Jedan skandalozan događaj tijekom priprema za rat 220. – Mesenija, koja je bila povod ratu, izglasala je, na poticaj svojih efora i nekih pristaša oligarhije, da neće ulaziti u rat protiv Etolaca prije nego što Etolcima bude oduzeta Figaleja, grad koji ugrožava mesenski teritorij – povod je Polibiju za razmišljanje o miru koji može biti sramotan ako se osigurava pod svaku cijenu.




%\newpage

\section*{Pročitajte naglas grčki tekst.}

Plb.\ Historiae 4.31.3
%Naslov prema izdanju

\medskip


{\large

\begin{greek}

\noindent ἐγὼ γὰρ φοβερὸν μὲν εἶναί φημι τὸν πόλεμον, οὐ μὴν οὕτω γε φοβερὸν ὥστε πᾶν ὑπομένειν χάριν τοῦ μὴ προσδέξασθαι πόλεμον.

\noindent ἐπεὶ τί καὶ θρασύνομεν τὴν ἰσηγορίαν καὶ παρρησίαν καὶ τὸ τῆς ἐλευθερίας ὄνομα πάντες, εἰ μηδὲν ἔσται προυργιαίτερον τῆς εἰρήνης; οὐδὲ γὰρ Θηβαίους ἐπαινοῦμεν κατὰ τὰ Μηδικά, διότι τῶν ὑπὲρ τῆς Ἑλλάδος ἀποστάντες κινδύνων τὰ Περσῶν εἵλοντο διὰ τὸν φόβον, οὐδὲ Πίνδαρον τὸν συναποφηνάμενον αὐτοῖς ἄγειν τὴν ἡσυχίαν διὰ τῶνδε τῶν ποιημάτων,
\begin{verse}
τὸ κοινόν τις ἀστῶν ἐν εὐδίᾳ τιθεὶς\\
ἐρευνασάτω μεγαλάνορος ἡσυχίας\\
τὸ φαιδρὸν φάος.\\

\end{verse}
δόξας γὰρ παραυτίκα πιθανῶς εἰρηκέναι, μετ´ οὐ πολὺ πάντων αἰσχίστην εὑρέθη καὶ βλαβερωτάτην πεποιημένος ἀπόφασιν· εἰρήνη γὰρ μετὰ μὲν τοῦ δικαίου καὶ πρέποντος κάλλιστόν ἐστι κτῆμα καὶ λυσιτελέστατον, μετὰ δὲ κακίας ἢ δειλίας ἐπονειδίστου πάντων αἴσχιστον καὶ βλαβερώτατον.

\end{greek}

}

%\newpage

\section*{Analiza i komentar}

%1

{\large
\begin{greek}
\noindent ἐγὼ γὰρ \\
\tabto{2em} φοβερὸν μὲν εἶναί \\
φημι \\
\tabto{2em} τὸν πόλεμον, \\
\tabto{2em} οὐ μὴν οὕτω γε φοβερὸν \\
\tabto{4em} ὥστε πᾶν ὑπομένειν \\
\tabto{6em} χάριν τοῦ μὴ προσδέξασθαι πόλεμον.\\

\end{greek}
}

\begin{description}[noitemsep]
\item[φοβερὸν\dots\ εἶναί] §~315; imenski predikat, Smyth 910; A+I ovisan o \textit{verbum dicendi}
\item[φοβερὸν μὲν\dots] \textbf{οὐ μὴν οὕτω γε φοβερὸν\dots}\ adverzativna upotreba kombinacija čestica; μὴν odgovara na tvrdnju uvedenu s μὲν (γε baš): \dots\ ali ipak ne baš\dots
\item[φημι] §~312.8; \textit{verbum dicendi} otvara mjesto A+I
\item[οὕτω\dots\ ὥστε\dots] koordinacija: pokazni prilog najavljuje posljedičnu rečenicu kojoj mjesto otvara veznik (predikat zavisne rečenice ovdje je u infinitivu, §~473)
\item[ὑπομένειν] §~231; infinitiv kao predikat zavisno posljedične rečenice, kao subjekt akuzativa s infinitivom zamislimo τινα ili ἡμᾶς da bi se dobio smisao ``treba podnositi''; LSJ ὑπομένω II.2
\item[χάριν] prilog, rekcija τινός, LSJ χάρις A.VI.1.
\item[τοῦ μὴ προσδέξασθαι] supstantivirani infinitiv (ovdje zanijekan), §~373
\item[προσδέξασθαι] §~267; složenica δέχομαι, §~328.4
\end{description}

%2

{\large
\begin{greek}
\noindent ἐπεὶ τί καὶ \\
θρασύνομεν \\
\tabto{2em} τὴν ἰσηγορίαν \\
\tabto{2em} καὶ παρρησίαν \\
\tabto{2em} καὶ τὸ τῆς ἐλευθερίας ὄνομα \\
πάντες, \\
\tabto{2em} εἰ μηδὲν ἔσται προυργιαίτερον \\
\tabto{4em} τῆς εἰρήνης;\\

\end{greek}
}

\begin{description}[noitemsep]
\item[θρασύνομεν] §~231; LSJ θρᾰσύνω I.2 c. acc., hvaliti se čime, τὴν ἰσηγορίαν Plb. 4.31.4 (to je upravo ovo naše mjesto)
\item[τὸ τῆς ἐλευθερίας ὄνομα] adnominalni genitiv, Smyth 1290–1296
\item[προυργιαίτερον] LSJ προὔργου stegnuto od πρὸ ἔργου; II komparativ προὐργιαίτερος, α, ον funkcionalniji, važniji
\item[ἔσται προυργιαίτερον] §~315; imenski predikat, Smyth 910
\end{description}

%3


{\large
\begin{greek}
\noindent οὐδὲ γὰρ \\
\tabto{2em} Θηβαίους ἐπαινοῦμεν \\
\tabto{4em} κατὰ τὰ Μηδικά, \\
\tabto{6em} διότι \\
\tabto{8em} τῶν ὑπὲρ τῆς Ἑλλάδος \\
\tabto{6em} ἀποστάντες \\
\tabto{8em} κινδύνων \\
\tabto{6em} τὰ Περσῶν \\
\tabto{6em} εἵλοντο \\
\tabto{8em} διὰ τὸν φόβον,

οὐδὲ \\
\tabto{2em} Πίνδαρον \\
\tabto{2em} τὸν συναποφηνάμενον αὐτοῖς \\
\tabto{4em} ἄγειν τὴν ἡσυχίαν \\
\tabto{4em} διὰ τῶνδε τῶν ποιημάτων, \\
\tabto{4em} τὸ κοινόν τις ἀστῶν \\
\tabto{6em} ἐν εὐδίᾳ \\
\tabto{4em} τιθεὶς\\
\tabto{2em} ἐρευνασάτω \\
\tabto{4em} μεγαλάνορος ἡσυχίας \\
\tabto{2em} τὸ φαιδρὸν φάος. \\

\end{greek}
}

\begin{description}[noitemsep]
\item[ἐπαινοῦμεν] §~243
\item[τὰ Μηδικά] supstantivirani pridjev, §~373; ovdje u specifičnom značenju ``Perzijski ratovi''
\item[τῶν ὑπὲρ τῆς Ἑλλάδος] prijedložni izraz kao atribut uz τῶν\dots\ κινδύνων
\item[ἀποστάντες] složenica glagola ἵστημι, §~306; LSJ ἀφίστημι B
\item[τὰ Περσῶν] supstantivirani genitiv posvojni, §~373; ovdje u specifičnom značenju ``strana Perzijanaca''
\item[εἵλοντο] §~254, §~327.1
\item[συναποφηνάμενον] §~254; složenica φαίνω, s. 118; rekcija τινί + infinitiv: nekome da (nešto)\dots
\item[ἄγειν] §~231; ovom je infinitivu mjesto otvorio oblik συναποφηνάμενον
\item[τῶν ποιημάτων] LSJ ποίημα, ατος, τό, (ποιέω) I.2 u pluralu o pojedinačnim stihovima, u značenju ἔπη
\item[τὸ κοινόν τις ἀστῶν\dots] ovo je Pindarov ulomak 109, iz ode Tebancima, sačuvan također i u \textit{Izvacima} Ivana Stobeja \textgreek[variant=ancient]{(Ἰωάννης ὁ Στοβαῖος)} iz V. st. po~Kr.\ (Stob. ecl. 4.16.6); moguće je da oda i ne govori o događajima iz Grčko-perzijskog rata 480.
\item[τὸ κοινόν] supstantivirani pridjev, §~373
\item[τιθεὶς] §~305
\item[ἐρευνασάτω] §~267, §~269
\item[μεγαλάνορος ἡσυχίας] atributna dopuna imenice φάος; adnominalni genitiv, Smyth 1290–1296

\end{description}


%4


{\large
\begin{greek}
\noindent δόξας γὰρ \\
\tabto{2em} παραυτίκα \\
\tabto{2em} πιθανῶς \\
\tabto{4em} εἰρηκέναι, \\
\tabto{2em} μετ´ οὐ πολὺ \\
\tabto{2em} πάντων αἰσχίστην \\
\tabto{4em} εὑρέθη \\
\tabto{2em} καὶ βλαβερωτάτην \\
\tabto{4em} πεποιημένος \\
\tabto{2em} ἀπόφασιν·

\noindent εἰρήνη γὰρ \\
\tabto{2em} μετὰ μὲν τοῦ δικαίου \\
\tabto{2em} καὶ πρέποντος \\
κάλλιστόν ἐστι κτῆμα \\
καὶ λυσιτελέστατον, \\
\tabto{2em} μετὰ δὲ κακίας \\
\tabto{2em} ἢ δειλίας ἐπονειδίστου \\
πάντων αἴσχιστον καὶ βλαβερώτατον.\\

\end{greek}
}

\begin{description}[noitemsep]
\item[δόξας] §~267, §~325.2; \textit{verbum sentiendi} otvara mjesto dopuni u infinitivu
\item[εἰρηκέναι] §~272, §~327.7
\item[μετ´ οὐ πολὺ] = οὐ πολὺ μετά u vremenskom značenju
\item[εὑρέθη] §~296, §~324.7; glagol nepotpuna značenja ima imensku riječ (ovdje pridjev) kao dopunu
\item[πεποιημένος] predikatni particip uz εὑρέθη; §~272
\item[μετὰ μὲν\dots\ μετὰ δὲ\dots] koordinacija rečeničnih članova parom čestica
\item[τοῦ δικαίου καὶ πρέποντος] supstantivirani pridjevi, §~373
\item[κάλλιστόν ἐστι κτῆμα καὶ λυσιτελέστατον] §~315; imenski predikat, Smyth 910
\item[αἴσχιστον καὶ βλαβερώτατον] sc.\ ἐστι; imenski dijelovi imenskog predikata, Smyth 910

\end{description}



%kraj
