% Unesi ispravke NJ 20. 4. 2020.
% Unio ispravke NZ <2022-01-01 sub>
\section*{O autoru}

Povjesničar Dion Kasije (navodni njegov \textit{cognomen} Cocceianus, Kokcejan, spominje se tek u bizantsko doba i vjerojatno je fiktivan), grčki \textgreek[variant=ancient]{Δίων ὁ Κάσσιος,} latinski Lucius Cassius Dio, podrijetlom Grk iz Nikeje u Bitiniji, rođen je oko 155.\ n.~e., a umro oko 235. Vršio je službu konzula i prokonzula.

Najpoznatiji je po djelu \textgreek[variant=ancient]{Ῥωμαϊκὴ ἱστορία}, \textit{Rimska povijest}, u 80 knjiga; od njih su u cijelosti sačuvane samo one sa zbivanjima od 36. do 60. Djelo počinje Enejinim dolaskom u Italiju, a završava godinom 229.\ n.~e. 

Dion je pisao po uzoru na analiste, najranije rimske povjesničare koji su događaje navodili po godinama.

\section*{O tekstu}

U ovom se odlomku opisuje kako je car Neron 64.\ n.~e.\ ostvario svoju dugogodišnju želju da zapali Rim. Riječ je o možda i najpoznatijoj epizodi iz Neronova života koja se često spominjala u književnosti i prikazivala na filmu. Međutim, valja imati na umu da su o događaju izvještavali pisci izrazito neskloni Neronu, primjerice Tacit. Današnji povjesničari smatraju malo vjerojatnim da je veliki požar podmetnuo upravo car.

Dion Kasije ne smatra se sasvim pouzdanim izvorom za ranija povijesna razdoblja. 

% Δίωνος Κασσίου Κοκκηιανοῦ Ῥωμαϊκὴ ἱστορία
\newpage

\section*{Pročitajte naglas grčki tekst.}

Dio Cassius Historiae Romanae 62.16.1-62.16.4
%Naslov prema izdanju

\medskip


{\large

\begin{greek}

\noindent μετὰ δὲ ταῦτα ἐπεθύμησεν ὅπερ που ἀεὶ ηὔχετο, τήν τε πόλιν ὅλην καὶ τὴν βασιλείαν ζῶν ἀναλῶσαι· τὸν γοῦν Πρίαμον καὶ αὐτὸς θαυμαστῶς ἐμακάριζεν ὅτι καὶ τὴν πατρίδα ἅμα καὶ τὴν ἀρχὴν ἀπολομένας εἶδεν. λάθρᾳ γάρ τινας ὡς καὶ μεθύοντας ἢ καὶ κακουργοῦντάς τι ἄλλως διαπέμπων, τὸ μὲν πρῶτον ἕν που καὶ δύο καὶ πλείονα ἄλλα ἄλλοθι ὑπεπίμπρα, ὥστε τοὺς ἀνθρώπους ἐν παντὶ ἀπορίας γενέσθαι, μήτ᾽ ἀρχὴν τοῦ κακοῦ ἐξευρεῖν μήτε τέλος ἐπαγαγεῖν δυναμένους ἀλλὰ πολλὰ μὲν ὁρῶντας πολλὰ δὲ ἀκούοντας ἄτοπα. 

\noindent οὔτε γὰρ θεάσασθαι ἄλλο τι ἦν ἢ πυρὰ πολλὰ ὥσπερ ἐν στρατοπέδῳ, οὔτε ἀκοῦσαι λεγόντων τινῶν ἢ ὅτι τὸ καὶ τὸ καίεται. ποῦ; πῶς; ὑπὸ τίνος; βοηθεῖτε.

\noindent θόρυβός τε οὖν ἐξαίσιος πανταχοῦ πάντας κατελάμβανε, καὶ διέτρεχον οἱ μὲν τῇ οἱ δὲ τῇ ὥσπερ ἔμπληκτοι. καὶ ἄλλοις τινὲς ἐπαμύνοντες ἐπυνθάνοντο τὰ οἴκοι καιόμενα· καὶ ἕτεροι πρὶν καὶ ἀκοῦσαι ὅτι τῶν σφετέρων τι ἐμπέπρησται, ἐμάνθανον ὅτι ἀπόλωλεν. οἵ τε ἐκ τῶν οἰκιῶν ἐς τοὺς στενωποὺς ἐξέτρεχον ὡς καὶ ἔξωθεν αὐταῖς βοηθήσοντες, καὶ οἱ ἐκ τῶν ὁδῶν εἴσω ἐσέθεον ὡς καὶ ἔνδον τι ἀνύσοντες.

\end{greek}

}

%\newpage

\section*{Analiza i komentar}

%1

{\large
\begin{greek}
\noindent μετὰ δὲ ταῦτα \\
ἐπεθύμησεν \\
\tabto{2em} ὅπερ που ἀεὶ ηὔχετο, \\
τήν τε πόλιν ὅλην \\
καὶ τὴν βασιλείαν \\
ζῶν \\
\tabto{2em} ἀναλῶσαι· \\
τὸν γοῦν Πρίαμον \\
καὶ αὐτὸς \\
θαυμαστῶς ἐμακάριζεν \\
\tabto{2em} ὅτι \\
\tabto{4em} καὶ τὴν πατρίδα \\
\tabto{4em} ἅμα καὶ τὴν ἀρχὴν \\
\tabto{4em} ἀπολομένας \\
\tabto{4em} εἶδεν. \\

\end{greek}
}

\begin{description}[noitemsep]
\item[ἐπεθύμησεν] §~267 (subjekt je Neron); glagol otvara mjesto dopuni u infinitivu
\item[ὅπερ] zamjenica pojačana česticom περ uvodi odnosnu rečenicu: ``ono što\dots''
\item[που] valjda 
\item[ηὔχετο] §~232
\item[ζῶν] §~231
\item[ἀναλῶσαι] §~267; glagol je \textgreek[variant=ancient]{ἀναλίσκω,} neka vremena tvori od osnove \textgreek[variant=ancient]{ἀναλω-,} §~324.6
\item[γοῦν] čestica koja služi isticanju ili pojačavanju  
\item[ἐμακάριζεν] §~232
\item[ὅτι] veznik uvodi zavisnu uzročnu rečenicu §~468
\item[ἀπολομένας] §~319.15, §~254
\item[εἶδεν] §~327.3, §~254

\end{description}

%2

{\large
\begin{greek}
\noindent λάθρᾳ γάρ τινας \\
\tabto{2em} ὡς καὶ μεθύοντας \\
\tabto{2em} ἢ καὶ κακουργοῦντάς τι ἄλλως \\
διαπέμπων, \\
τὸ μὲν πρῶτον \\
\tabto{2em} ἕν που καὶ δύο καὶ πλείονα ἄλλα ἄλλοθι \\
ὑπεπίμπρα, \\
\tabto{2em} ὥστε τοὺς ἀνθρώπους \\
\tabto{4em} ἐν παντὶ \\
\tabto{6em} ἀπορίας \\
\tabto{4em} γενέσθαι, \\
\tabto{4em} μήτ' ἀρχὴν \\
\tabto{6em} τοῦ κακοῦ \\
\tabto{6em} ἐξευρεῖν \\
\tabto{4em} μήτε τέλος \\
\tabto{6em} ἐπαγαγεῖν \\
\tabto{4em} δυναμένους \\
\tabto{4em} ἀλλὰ \\
\tabto{6em} πολλὰ μὲν ὁρῶντας \\
\tabto{6em} πολλὰ δὲ ἀκούοντας \\
\tabto{6em} ἄτοπα. \\

\end{greek}
}

\begin{description}[noitemsep]
\item[ὡς καὶ\dots\ ἢ καὶ\dots] veznik ὡς uz particip uvodi objašnjenje ili poredbu: zato što su bili\dots, kao da su bili\dots; nastavlja se koordinacijom: ili\dots\ ili\dots
\item[μεθύοντας] §~231
\item[κακουργοῦντάς] §~231, §~243
\item[διαπέμπων] složenica glagola πέμπω, §~231
\item[ἕν] sc. zgradu (ili kuću)
\item[που] negdje
\item[ὑπεπίμπρα] složenica glagola πίμπρημι, §~312.3; ovaj imperfekt tvori se kao da je osnova \textgreek[variant=ancient]{ὑποπιμπράω,} usp. LSJ \textgreek[variant=ancient]{ἐμπίμπρημι,} morfološki opis; glagol je u jednini jer Neron djeluje preko svojih pomagača (``dao je potpaliti\dots''; kauzativ)
\item[ὥστε] uvodi posljedičnu rečenicu s predikatom u infinitivu, §~473
\item[ἐν παντὶ] LSJ πᾶς, πᾶσα, πᾶν IV ἐν παντὶ εἶναι, ἐν παντὶ κακοῦ εἶναι ``biti u velikoj opasnosti, u velikom strahu''
\item[γενέσθαι] §~254, §~325.11, dopuna u genitivu; LSJ \textgreek[variant=ancient]{γίγνομαι} II.3
\item[μήτ'\dots\ μήτε\dots] koordinacija niječnim sastavnim veznicima; §~513
\item[ἐξευρεῖν] složenica glagola εὑρίσκω, §~324.7, §~254
\item[ἐπαγαγεῖν] složenica glagola ἄγω, s. 116, §~257 
\item[δυναμένους] §~312.5, §~232, glagol nepotpuna značenja otvara mjesto dopunama u infinitivu
\item[πολλὰ μὲν\dots\ πολλὰ δὲ\dots] koordinacija rečeničnih članova pomoću para čestica
\item[ὁρῶντας] §~327.3, §~231, §~243
\item[ἀκούοντας] §~231, s. 116

\end{description}

%3


{\large
\begin{greek}
\noindent οὔτε γὰρ \\
\tabto{2em} θεάσασθαι \\
\tabto{2em} ἄλλο τι \\
\tabto{2em} ἦν \\
\tabto{4em} ἢ πυρὰ πολλὰ \\
\tabto{4em} ὥσπερ ἐν στρατοπέδῳ, \\
οὔτε \\
\tabto{2em} ἀκοῦσαι \\
\tabto{2em} λεγόντων τινῶν \\
\tabto{4em} ἢ ὅτι \\
\tabto{6em} τὸ καὶ τὸ καίεται· \\
\tabto{6em} ποῦ; \\
\tabto{6em} πῶς; \\
\tabto{6em} ὑπὸ τίνος; \\
\tabto{6em} βοηθεῖτε.\\

\end{greek}
}

\begin{description}[noitemsep]
\item[οὔτε\dots\ οὔτε\dots] koordinacija pomoću sastavnih veznika §~513
\item[θεάσασθαι] §~267
\item[ἦν] §~315; u značenju »biti moguće« glagol otvara mjesto dopuni u infinitivu; LSJ εἰμί A.VI
\item[ἢ] osim
\item[ἀκοῦσαι] rekcija τινός, §~267, s. 116
\item[λεγόντων] §~231
\item[ὅτι] veznik uvodi zavisnu izričnu (objektnu) rečenicu koja ovisi o \textit{verbum dicendi} (λεγόντων)
\item[τὸ καὶ τὸ] član kao pokazna zamjenica: to i to, §~370
\item[καίεται] §~232
\item[βοηθεῖτε] §~231, §~243

\end{description}

%4

{\large
\begin{greek}
\noindent θόρυβός τε οὖν ἐξαίσιος \\
πανταχοῦ \\
πάντας \\
κατελάμβανε, \\
καὶ διέτρεχον \\
\tabto{2em} οἱ μὲν τῇ \\
\tabto{2em} οἱ δὲ τῇ \\
\tabto{4em} ὥσπερ ἔμπληκτοι. \\

\end{greek}
}

\begin{description}[noitemsep]
\item[τε] postponirani (sastavni) veznik
\item[οὖν] zaključna čestica: dakle
\item[κατελάμβανε] složenica glagola \textgreek[variant=ancient]{λαμβάνω} §~321.14, §~231
\item[διέτρεχον] složenica glagola \textgreek[variant=ancient]{τρέχω} §~327.4, §~231
\item[οἱ μὲν τῇ, οἱ δὲ τῇ] koordinacija pomoću čestica: jedni ovamo, a drugi onamo

\end{description}


%5


{\large
\begin{greek}
\noindent καὶ ἄλλοις τινὲς ἐπαμύνοντες \\
ἐπυνθάνοντο \\
τὰ οἴκοι καιόμενα· \\
καὶ ἕτεροι \\
\tabto{2em} πρὶν καὶ ἀκοῦσαι \\
\tabto{4em} ὅτι\\
\tabto{6em} τῶν σφετέρων τι \\
\tabto{4em} ἐμπέπρησται, \\
ἐμάνθανον \\
\tabto{2em} ὅτι \\
\tabto{4em} ἀπόλωλεν. \\


\end{greek}
}

\begin{description}[noitemsep]
\item[ἐπαμύνοντες] složenica glagola \textgreek[variant=ancient]{ἀμύνω;} rekcija τινί; §~231
\item[ἐπυνθάνοντο] §~232; objekt je glagola ovdje particip u akuzativu; LSJ \textgreek[variant=ancient]{πυνθάνομαι} I.2 samo s akuzativnom stvari: čuti što, saznati što
\item[τὰ οἴκοι καιόμενα] §~232; supstantivirani particip, §~373; prilog u atributnom položaju §~375.5
\item[πρὶν] veznik uvodi zavisnu vremensku rečenicu s predikatom u infinitivu, §~488.1; πρὶν καὶ: prije nego što\dots
\item[ἀκοῦσαι] s.~116, §~267
\item[ὅτι] veznik uvodi izričnu zavisnu izričnu (objektnu) rečenicu uz \textit{verbum sentiendi}, §~467
\item[τῶν σφετέρων] genitiv partitivni §~395; supstantivirana zamjenica, §~373
\item[ἐμπέπρησται] složenica glagola \textgreek[variant=ancient]{πίμπρημι,} §~272
\item[ἐμάνθανον] §~231
\item[ὅτι] veznik uvodi izričnu zavisnu rečenicu uz \textit{verbum sentiendi}, §~467
\item[ἀπόλωλεν] §~272, §~319.15

\end{description}

%6


{\large
\begin{greek}
\noindent οἵ τε ἐκ τῶν οἰκιῶν \\
\tabto{2em} ἐς τοὺς στενωποὺς \\
ἐξέτρεχον \\
\tabto{2em} ὡς καὶ ἔξωθεν \\
\tabto{2em} αὐταῖς βοηθήσοντες, \\
καὶ οἱ ἐκ τῶν ὁδῶν \\
\tabto{2em} εἴσω \\
ἐσέθεον \\
\tabto{2em} ὡς καὶ ἔνδον \\
\tabto{2em} τι ἀνύσοντες.\\

\end{greek}
}

\begin{description}[noitemsep]
\item[οἵ τε\dots\ καὶ οἱ\dots] koordinacija rečeničnih članaka sastavnim veznicima
\item[οἵ\dots\ ἐκ τῶν οἰκιῶν] supstantiviran prijedložni izraz
\item[ἐς τοὺς στενωποὺς] sc.\ ὀδοὺς: u tijesne ulice, uličice
\item[ἐξέτρεχον] složenica glagola \textgreek[variant=ancient]{τρέχω,} §~327.4, §~231
\item[ὡς\dots\ βοηθήσοντες] ὡς uz particip futura izriče namjerno značenje 
\item[βοηθήσοντες] rekcija τινί, §~259
\item[οἱ ἐκ τῶν ὁδῶν] supstantiviran prijedložni izraz
\item[ἐσέθεον] složenica glagola θέω, §~231
\item[ὡς\dots\ ἀνύσοντες] veznik ὡς uz particip futura izriče namjeru, §~503.3, Smyth §~2065, §~2086
\item[ἀνύσοντες] §~259

\end{description}


%kraj
