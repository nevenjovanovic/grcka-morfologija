% Unesi ispravke NZ, 6. 4. 2020.
% Unio ispravke NZ <2021-12-30 čet>
%\section*{O autoru}

%TKTK


\section*{O tekstu}

Lizija (Λυσίας, oko 445.\ – nakon 380.\ p.~n.~e.) djelovao je u Ateni kao λογογράφος, profesionalni autor sudskih govora za druge; od nekoliko stotina njegovih govora sačuvano je tridesetak.

Oko 387.\ Liziju je angažirao nepoznati Atenjanin, sin tuženoga (ovaj je u međuvremenu preminuo) i šogor Aristofanov (Ἀριστοφάνης; on nema veze s komičkim pjesnikom). Aristofan je, zajedno s ocem Nikofemom, iz nepoznatih razloga smaknut oko 390, a njihova je imovina (χρήματα) u Ateni konfiscirana. Međutim, vrijednost imetka bila je ispod onoga što je javnost očekivala (Aristofan i Nikofem bili su bliski dvojici vrlo uspješnih vojskovođa, Kononu i Euagori). Zato je Aristofanov tast, otac Lizijina klijenta (za kojeg je govor napisan i koji je govor na sudu održao), optužen da je otuđio dio imovine u korist svoje kćeri i unuka. Optužba je tražila da Lizijin klijent državnoj riznici nadoknadi dio Aristofanove imovine koji je njegov otac navodno otuđio.

Lizijin govor ima dva cilja. Prvo, pokušava pokazati da veza Aristofana i klijentova oca nije bila tako tijesna kao što se tvrdi; bilo je drugih kojima je Aristofan više vjerovao, kojima bi prije povjerio vođenje svojih poslova, a govornikov otac, iako rodoljub, nije bio zainteresiran za politiku. Drugo, dokazuje da Aristofan i nije bio onoliko bogat koliko ljudi misle; imao je velike izdatke da bi postigao ugled u društvu, a i mnogi drugi ugledni Atenjani, pokazalo se nakon njihove smrti, bili su zapravo puno siromašniji nego što bi se očekivalo od njih, od njihovih očeva ili djedova.

Odabrani je odlomak dio argumentacije (πίστεις), koja u govoru slijedi odmah nakon proslova (προοίμιον). Govornik dokazuje da nije vjerojatno (εἰκός) da je njegov otac posjedovao Aristofanovu imovinu, zato što ženidbena veza s Nikofemovom obitelji nije proizašla iz novčanih interesa, jednako kao što ni ostali brakovi u govornikovoj obitelji (oca s govornikovom majkom, govornikove druge sestre, govornika samog) nisu sklapani radi financijske koristi.


%\newpage

\section*{Pročitajte naglas grčki tekst.}

Lys.\ 19.\ Ὑπὲρ τῶν Ἀριστοφάνους χρημάτων 14–17

%Naslov prema izdanju

\medskip


{\large

\begin{greek}

\noindent  ἐκεῖνος γὰρ ὅτ´ ἦν ἐν τῇ ἡλικίᾳ, παρὸν μετὰ πολλῶν χρημάτων γῆμαι ἄλλην, τὴν ἐμὴν μητέρα ἔλαβεν οὐδὲν ἐπιφερομένην, ὅτι δὲ Ξενοφῶντος ἦν θυγάτηρ τοῦ Εὐριπίδου ὑέος, ὃς οὐ μόνον ἰδίᾳ χρηστὸς ἐδόκει εἶναι, ἀλλὰ καὶ στρατηγεῖν αὐτὸν ἠξιώσατε, ὡς ἐγὼ ἀκούω.

τὰς τοίνυν ἐμὰς ἀδελφὰς ἐθελόντων τινῶν λαβεῖν ἀπροίκους πάνυ πλουσίων οὐκ ἔδωκεν, ὅτι ἐδόκουν κάκιον γεγονέναι, ἀλλὰ τὴν μὲν Φιλομήλῳ τῷ Παιανιεῖ, ὃν οἱ πολλοὶ βελτίω ἡγοῦνται εἶναι ἢ πλουσιώτερον, τὴν δὲ πένητι γεγενημένῳ οὐ διὰ κακίαν, ἀδελφιδῷ δὲ ὄντι Φαίδρῳ τῷ Μυρρινουσίῳ, ἐπιδοὺς τετταράκοντα μνᾶς, κᾆτ´ Ἀριστοφάνει τὸ ἴσον. πρὸς δὲ τούτοις ἐμοὶ πολλὴν ἐξὸν πάνυ προῖκα λαβεῖν ἐλάττω συνεβούλευσεν, ὥστε εὖ εἰδέναι ὅτι κηδεσταῖς χρησοίμην κοσμίοις καὶ σώφροσι. καὶ νῦν ἔχω γυναῖκα τὴν Κριτοδήμου θυγατέρα τοῦ Ἀλωπεκῆθεν, ὃς ὑπὸ Λακεδαιμονίων ἀπέθανεν, ὅτε ἡ ναυμαχία ἐγένετο ἐν Ἑλλησπόντῳ.

καίτοι, ὦ ἄνδρες δικασταί, ὅστις αὐτός τε ἄνευ χρημάτων ἔγημε τοῖν τε θυγατέροιν πολὺ ἀργύριον ἐπέδωκε τῷ τε ὑεῖ ὀλίγην προῖκα ἔλαβε, πῶς οὐκ εἰκὸς περὶ τούτου πιστεύειν ὡς οὐχ ἕνεκα χρημάτων τούτοις κηδεστὴς ἐγένετο;


\end{greek}

}


\section*{Analiza i komentar}

%1

{\large
\begin{greek}
\noindent ἐκεῖνος γὰρ \\
\tabto{2em} ὅτ´ ἦν ἐν τῇ ἡλικίᾳ, \\
παρὸν \\
\tabto{4em} μετὰ πολλῶν χρημάτων \\
\tabto{2em} γῆμαι \\
\tabto{4em} ἄλλην, \\
τὴν ἐμὴν μητέρα ἔλαβεν \\
\tabto{2em} οὐδὲν ἐπιφερομένην, \\
ὅτι δὲ Ξενοφῶντος ἦν θυγάτηρ \\
\tabto{2em} τοῦ Εὐριπίδου ὑέος, \\
\tabto{2em} ὃς \\
\tabto{4em} οὐ μόνον ἰδίᾳ \\
\tabto{4em} χρηστὸς ἐδόκει εἶναι, \\
\tabto{4em} ἀλλὰ καὶ \\
\tabto{6em} στρατηγεῖν αὐτὸν \\
\tabto{4em} ἠξιώσατε, \\
\tabto{6em} ὡς ἐγὼ ἀκούω.

\end{greek}
}

\begin{description}[noitemsep]
\item[γὰρ] čestica γάρ najavljuje iznošenje razloga ili objašnjenja, ``naime''
\item[ὅτ´ ἦν ἐν τῇ ἡλικίᾳ] §~315; kopulativni glagol otvara mjesto nužnoj predikatnoj dopuni (imenski predikat, Smyth 910); ὅτε uvodi zavisno vremensku rečenicu, §~487
\item[παρὸν] složenica glagola εἰμί, §~315; LSJ πάρειμι III.2; otvara mjesto infinitivu kao nužnoj dopuni
\item[μετὰ πολλῶν χρημάτων] opisuje ἄλλην; LSJ μετά II.
\item[γῆμαι] §~267, §~325.1
\item[ἄλλην] sc. γυναῖκα
\item[ἔλαβεν] §~254, §~321.14
\item[τὴν ἐμὴν\dots\ ὅτι δὲ\dots] koordinacija rečeničnih članova pomoću čestice δέ
\item[ἐπιφερομένην] §~232%atributno
\item[ὅτι\dots\ ἦν θυγάτηρ] §~315; ὅτι uvodi zavisno uzročnu rečenicu, §~468; kopulativni glagol otvara mjesto nužnoj predikatnoj dopuni (imenski predikat, Smyth 910)
\item[οὐ μόνον\dots\ ἀλλὰ καὶ\dots] koordinacija rečeničnih članova: ``ne samo\dots\ nego i\dots''
\item[χρηστὸς ἐδόκει εἶναι] §~243; §~315; \textit{verbum sentiendi} otvara mjesto nužnoj dopuni u infinitivu; kopulativni glagol otvara mjesto nužnoj predikatnoj dopuni (imenski predikat, Smyth 910)
\item[στρατηγεῖν] §~243
\item[ἠξιώσατε] §~267, §~269, §~235; ἀξιόω traži kao dopunu lični objekt u akuzativu (izriče se osoba) te infinitiv, LSJ: II. smatrati koga vrijednim da što učini ili bude
\item[ὡς\dots\ ἀκούω] §~231; ὡς uvodi poredbenu zavisnu rečenicu, ovdje s indikativom, Smyth 2475
\end{description}

%2
{\large
\begin{greek}
\noindent τὰς τοίνυν ἐμὰς ἀδελφὰς \\
\tabto{2em} ἐθελόντων τινῶν \\
\tabto{4em} λαβεῖν ἀπροίκους \\
\tabto{2em} πάνυ πλουσίων \\
οὐκ ἔδωκεν, \\
\tabto{2em} ὅτι ἐδόκουν \\
\tabto{4em} κάκιον γεγονέναι, \\
ἀλλὰ τὴν μὲν \\
\tabto{2em} Φιλομήλῳ τῷ Παιανιεῖ, \\
\tabto{4em} ὃν οἱ πολλοὶ \\
\tabto{4em} βελτίω ἡγοῦνται εἶναι\\
\tabto{4em} ἢ πλουσιώτερον, \\
τὴν δὲ \\
\tabto{2em} πένητι γεγενημένῳ \\
\tabto{4em} οὐ διὰ κακίαν, \\
\tabto{2em} ἀδελφιδῷ δὲ ὄντι \\
\tabto{2em} Φαίδρῳ τῷ Μυρρινουσίῳ, \\
ἐπιδοὺς \\
\tabto{2em} τετταράκοντα μνᾶς, \\
κᾆτ´ Ἀριστοφάνει \\
\tabto{2em} τὸ ἴσον.\\

\end{greek}
}

\begin{description}[noitemsep]
\item[τοίνυν] zaključni veznik, §~516.3
\item[ἐθελόντων] §~231; participski dio genitiva apsolutnog, §~504; kao kopulativni glagol otvara mjesto nužnoj dopuni u infinitivu
\item[λαβεῖν] §~231
\item[ἔδωκεν] §~306
\item[ὅτι ἐδόκουν] §~243; ὅτι uvodi zavisno uzročnu rečenicu, §~468; \textit{verbum sentiendi} otvara mjesto nužnoj dopuni u infinitivu
\item[κάκιον γεγονέναι] §~272; §~325.11; predikatni pridjev u srednjem rodu singulara slaže se sa subjektom muškog roda, §~365
\item[τὴν μὲν\dots\ τὴν δὲ\dots] koordinacija rečeničnih članova pomoću čestica μέν\dots\ δέ\dots
\item[Φιλομήλῳ τῷ Παιανιεῖ] sc. ἔδωκεν
\item[βελτίω\dots\ εἶναι ἢ πλουσιώτερον] §~315; kopulativni glagol otvara mjesto nužnoj predikatnoj dopuni (imenski predikat, Smyth 910); komparativi u korelaciji pomoću rastavnog veznika ἤ, §~514.1.b
\item[ἡγοῦνται] §~243; \textit{verbum sentiendi} otvara mjesto akuzativu s infinitivom
\item[πένητι\dots\ ἀδελφιδῷ δὲ\dots] koordinacija rečeničnih članova pomoću čestice δέ
\item[πένητι γεγενημένῳ] §~272; §~325.11; ἔδωκεν otvara mjesto dativu (isto i kod ἀδελφιδῷ δὲ ὄντι); kopulativni glagol otvara mjesto nužnoj predikatnoj dopuni
\item[ἀδελφιδῷ\dots\ ὄντι] §~315; kopulativni glagol otvara mjesto nužnoj predikatnoj dopuni
\item[ἐπιδοὺς] složenica glagola δίδωμι, §~306
\item[κᾆτ´] kraza καὶ εἶτα

\end{description}

%3
{\large
\begin{greek}
\noindent πρὸς δὲ τούτοις \\
\tabto{2em} ἐμοὶ \\
πολλὴν ἐξὸν πάνυ προῖκα \\
\tabto{2em} λαβεῖν \\
ἐλάττω \\
συνεβούλευσεν, \\
\tabto{2em} ὥστε εὖ εἰδέναι \\
\tabto{4em} ὅτι κηδεσταῖς \\
\tabto{4em} χρησοίμην \\
\tabto{6em} κοσμίοις καὶ σώφροσι.\\

\end{greek}
}

\begin{description}[noitemsep]
\item[δὲ] čestica δέ označava nadovezivanje na prethodni iskaz
\item[ἐμοὶ\dots\ ἐξὸν\dots\ λαβεῖν] složenica glagola εἰμί, §~315; bezlično \textgreek[variant=ancient]{ἔξεστί τινι} otvara mjesto dopuni u infinitivu; §~231
\item[πολλὴν\dots\ πάνυ] prilog modificira pridjev, LSJ πάνυ A.1
\item[ἐλάττω] akuzativ ženskog roda; sc. προῖκα λαβεῖν
\item[συνεβούλευσεν] §~267; §~269; §~238
\item[ὥστε εὖ εἰδέναι] §~317.4; kondicionalno ὥστε ``pod uvjetom da'', ``samo ako'' LSJ ὥστε B.4; subjekt infinitiva vidljiv iz oblika χρησοίμην
\item[ὅτι\dots\ χρησοίμην] s.\ 116; §~258; optativ u zavisno izričnoj rečenici iza sporednog (historijskog) vremena, §~467
\end{description}

%4
{\large
\begin{greek}
\noindent καὶ νῦν ἔχω γυναῖκα \\
\tabto{2em} τὴν Κριτοδήμου θυγατέρα \\
\tabto{4em} τοῦ Ἀλωπεκῆθεν, \\
\tabto{6em} ὃς ὑπὸ Λακεδαιμονίων ἀπέθανεν, \\
\tabto{8em} ὅτε ἡ ναυμαχία ἐγένετο ἐν Ἑλλησπόντῳ.\\

\end{greek}
}

\begin{description}[noitemsep]
\item[ἔχω] §~231
\item[ὑπὸ Λακεδαιμονίων ἀπέθανεν] §~254; §~324.8; \textgreek[variant=ancient]{ἀποθνῄσκω ὑπό τινος} kao pasiv glagola \textgreek[variant=ancient]{ἀποκτείνω, LSJ ἀποθνῄσκω} II.
\item[ὅτε\dots\ ἐγένετο] §~254; §~325.11; LSJ γίγνομαι A.I.3; veznik ὅτε uvodi zavisnu vremensku rečenicu, §~487
\item[ἡ ναυμαχία\dots\ ἐν Ἑλλησπόντῳ] pomorska bitka kod Egospotama (405. pr.~Kr.); Lisandar je iznenadio atensku flotu i smaknuo tri tisuće zarobljenih Atenjana
\end{description}

%5
{\large
\begin{greek}
\noindent καίτοι, ὦ ἄνδρες δικασταί, \\
ὅστις αὐτός τε ἄνευ χρημάτων ἔγημε \\
τοῖν τε θυγατέροιν πολὺ ἀργύριον ἐπέδωκε \\
τῷ τε ὑεῖ ὀλίγην προῖκα ἔλαβε, \\
πῶς οὐκ εἰκὸς \\
\tabto{2em} περὶ τούτου πιστεύειν \\
\tabto{4em} ὡς οὐχ \\
\tabto{6em} ἕνεκα χρημάτων \\
\tabto{6em} τούτοις \\
\tabto{4em} κηδεστὴς ἐγένετο;\\

\end{greek}
}

\begin{description}[noitemsep]
\item[αὐτός τε\dots\ τοῖν τε\dots\ τῷ τε\dots] koordinacija pomoću enklitičnog sastavnog veznika, §~513.2
\item[ἔγημε] §~267, §~325.1
\item[ἐπέδωκε] složenica glagola δίδωμι, §~306
\item[ἔλαβε] §~231
\item[εἰκὸς] sc.\ ἐστί, otvara mjesto infinitivu
\item[πιστεύειν] §~232, kao \textit{verbum sentiendi} otvara mjesto zavisnoj izričnoj rečenici, ovdje s ὡς, §~467
\item[κηδεστὴς ἐγένετο] §~254; §~325.11; kopulativni glagol otvara mjesto nužnoj predikatnoj dopuni (imenski predikat, Smyth 910)

\end{description}



%kraj
