%\section*{O autoru}

%TKTK


\section*{O tekstu}

Na sredini treće knjige (Γ) \textit{Nikomahove etike} Aristotel s općih razmatranja o vrlini, volji i odgovornosti prelazi na razmatranje pojedinačnih vrlina. Pošto je govorio o hrabrosti, njezinim suprotnostima i njezinim pojavnim oblicima, prelazi na vrlinu umjerenosti \textgreek[variant=ancient]{(σωφροσύνη),} čija je suprotnost \textgreek[variant=ancient]{ἀκολασία.  Σωφροσύνη} je Aristotel definirao kao sredinu u odnosu prema užicima \textgreek[variant=ancient]{(μεσότης ἐστὶ περὶ ἡδονὰς ἡ σωφροσύνη),} i to prvenstveno tjelesnima \textgreek[variant=ancient]{(σωματικαί).} Glavnina takvih užitaka nastaje po opipu, u jelu, piću, seksu: \textgreek[variant=ancient]{γίνεται πᾶσα δι᾽ ἁφῆς καὶ ἐν σιτίοις καὶ ἐν ποτοῖς καὶ τοῖς ἀφροδισίοις λεγομένοις.} Neumjereni pretjeruju na tri načina: uživaju u onom u čemu ne treba, ili više nego što treba, ili kako ne treba: \textgreek[variant=ancient]{ἢ τῷ χαίρειν οἷς μὴ δεῖ, ἢ τῷ μᾶλλον ἢ ὡς οἱ πολλοί, ἢ μὴ ὡς δεῖ.} No, onaj tko je neumjeren zbog svojih želja doživljava i bol, kad ne uspije ostvariti željeni užitak: \textgreek[variant=ancient]{ὁ μὲν ἀκόλαστος (λέγεται) τῷ λυπεῖσθαι μᾶλλον ἢ δεῖ ὅτι τῶν ἡδέων οὐ τυγχάνει.}

U sljedećem odlomku Aristotel potanje opisuje patnje i želje onoga tko je neumjeren, govori o (neuobičajenom) izostanku užitka, te potom o odnosu onoga tko je umjeren prema željama i užicima.

%\newpage

\section*{Pročitajte naglas grčki tekst.}

Arist.\ Ethica Nicomachea 1119a (3, 11)

%Naslov prema izdanju

\medskip


{\large

\begin{greek}

\noindent  ὁ μὲν οὖν ἀκόλαστος ἐπιθυμεῖ τῶν ἡδέων πάντων ἢ τῶν μάλιστα, καὶ ἄγεται ὑπὸ τῆς ἐπιθυμίας ὥστε ἀντὶ τῶν ἄλλων ταῦθ' αἱρεῖσθαι· διὸ καὶ λυπεῖται καὶ ἀποτυγχάνων καὶ ἐπιθυμῶν· μετὰ λύπης γὰρ ἡ ἐπιθυμία· ἀτόπῳ δ' ἔοικε τὸ δι' ἡδονὴν λυπεῖσθαι. ἐλλείποντες δὲ τὰ περὶ τὰς ἡδονὰς καὶ ἧττον ἢ δεῖ χαίροντες οὐ πάνυ γίνονται· οὐ γὰρ ἀνθρωπική ἐστιν ἡ τοιαύτη ἀναισθησία· καὶ γὰρ τὰ λοιπὰ ζῷα διακρίνει τὰ βρώματα, καὶ τοῖς μὲν χαίρει τοῖς δ' οὔ· εἰ δέ τῳ μηδέν ἐστιν ἡδὺ μηδὲ διαφέρει ἕτερον ἑτέρου, πόρρω ἂν εἴη τοῦ ἄνθρωπος εἶναι· οὐ τέτευχε δ' ὁ τοιοῦτος ὀνόματος διὰ τὸ μὴ πάνυ γίνεσθαι. ὁ δὲ σώφρων μέσως μὲν περὶ ταῦτ' ἔχει· οὔτε γὰρ ἥδεται οἷς μάλιστα ὁ ἀκόλαστος, ἀλλὰ μᾶλλον δυσχεραίνει, οὐδ' ὅλως οἷς μὴ δεῖ οὐδὲ σφόδρα τοιούτῳ οὐδενί, οὔτ' ἀπόντων λυπεῖται οὐδ' ἐπιθυμεῖ, ἢ μετρίως, οὐδὲ μᾶλλον ἢ δεῖ, οὐδ' ὅτε μὴ δεῖ, οὐδ' ὅλως τῶν τοιούτων οὐδέν· ὅσα δὲ πρὸς ὑγίειάν ἐστιν ἢ πρὸς εὐεξίαν ἡδέα ὄντα, τούτων ὀρέξεται μετρίως καὶ ὡς δεῖ, καὶ τῶν ἄλλων ἡδέων μὴ ἐμποδίων τούτοις ὄντων ἢ παρὰ τὸ καλὸν ἢ ὑπὲρ τὴν οὐσίαν. ὁ γὰρ οὕτως ἔχων μᾶλλον ἀγαπᾷ τὰς τοιαύτας ἡδονὰς τῆς ἀξίας· ὁ δὲ σώφρων οὐ τοιοῦτος, ἀλλ' ὡς ὁ ὀρθὸς λόγος.
\end{greek}

}


\section*{Analiza i komentar}

%1

{\large
\begin{greek}
\noindent  ὁ μὲν οὖν ἀκόλαστος\\
ἐπιθυμεῖ \\
\tabto{2em} τῶν ἡδέων πάντων \\
\tabto{2em} ἢ τῶν μάλιστα, \\
καὶ ἄγεται \\
\tabto{2em} ὑπὸ τῆς ἐπιθυμίας \\
\tabto{2em} ὥστε ἀντὶ τῶν ἄλλων \\
\tabto{4em} ταῦθ' αἱρεῖσθαι· \\
διὸ καὶ λυπεῖται \\
\tabto{4em} καὶ ἀποτυγχάνων \\
\tabto{4em} καὶ ἐπιθυμῶν·\\
\tabto{2em} μετὰ λύπης γὰρ \\
\tabto{4em} ἡ ἐπιθυμία· \\
\tabto{2em} ἀτόπῳ δ' ἔοικε \\
\tabto{4em} τὸ δι' ἡδονὴν λυπεῖσθαι. \\

\end{greek}
}

\begin{description}[noitemsep]
\item[ὁ\dots\ ἀκόλαστος] antonim σώφρων, usp. LSJ ἀκόλαστος; supstantiviranje članom svih vrsta riječi §~373
\item[μὲν οὖν]  kombinacija čestica označava prelaz i rezimiranje: onda dakle\dots
\item[ἐπιθυμεῖ] rekcija τινός; §~243
\item[τῶν ἡδέων] supstantiviranje članom svih vrsta riječi §~373
\item[τῶν μάλιστα] sc.\ ἡδέων
\item[ἄγεται ὑπὸ] §~232; ὑπό τινος uz pasiv izriče vršitelja radnje
\item[ὥστε\dots\ αἱρεῖσθαι] veznik uvodi zavisnu posljedičnu rečenicu; ὥστε otvara mjesto infinitivu koji izriče pomišljenu radnju, §~473
\item[ταῦθ'] sc.\ τὰ ἡδεῖα (Aristotel se koristi oblikom τὰ ἡδέα)
\item[λυπεῖται] §~243
\item[ἀποτυγχάνων\dots\ ἐπιθυμῶν] §~231, §~243; koordinacija parom sastavnih veznika
\item[μετὰ λύπης] imenski predikat (s izostavljenom kopulom) Smyth 909, ovdje izriče popratnu okolnost
\item[γὰρ] čestica upozorava da se navodi razlog: naime\dots%ima li gar... de?
\item[ἔοικε] §~327.b; ἔοικέ τινι LSJ ἔοικα III
\item[τὸ\dots\ λυπεῖσθαι] §~243; supstantivirani infinitiv §~497
\item[δ'] kopulativno δέ (izvan koordinacije s μέν) označava prijelaz iz rečenice u rečenicu; sljedeća rečenica iskazuje nešto novo ili drugačije, ali ne suprotno prethodnom iskazu, Smyth §~2386

\end{description}

%2

{\large
\begin{greek}
\noindent  ἐλλείποντες δὲ \\
\tabto{2em} τὰ περὶ τὰς ἡδονὰς \\
καὶ ἧττον ἢ δεῖ \\
\tabto{2em} χαίροντες \\
οὐ πάνυ γίνονται· \\
οὐ γὰρ ἀνθρωπική ἐστιν \\
\tabto{2em} ἡ τοιαύτη ἀναισθησία· \\
καὶ γὰρ τὰ λοιπὰ ζῷα \\
\tabto{2em} διακρίνει \\
\tabto{4em} τὰ βρώματα, \\
\tabto{2em} καὶ τοῖς μὲν χαίρει \\
\tabto{2em} τοῖς δ' οὔ· \\
εἰ δέ τῳ \\
\tabto{2em} μηδέν ἐστιν ἡδὺ \\
\tabto{2em} μηδὲ διαφέρει \\
\tabto{4em} ἕτερον ἑτέρου, \\
πόρρω ἂν εἴη \\
\tabto{2em} τοῦ ἄνθρωπος εἶναι· \\
οὐ τέτευχε δ' \\
ὁ τοιοῦτος \\
\tabto{2em} ὀνόματος \\
\tabto{4em} διὰ τὸ μὴ πάνυ γίνεσθαι. \\

\end{greek}
}

\begin{description}[noitemsep]
\item[ἐλλείποντες δὲ] §~231; kopulativno δέ označava prijelaz iz rečenice u rečenicu
\item[τὰ περὶ τὰς ἡδονὰς] supstantiviranje članom svih vrsta riječi §~373
\item[ἧττον] ovdje priložno, modificira χαίροντες
\item[δεῖ] bezlično, LSJ s.~v.
\item[χαίροντες] §~231
\item[οὐ πάνυ γίνονται] LSJ πάνυ 3; §~232; LSJ γίγνομαι, jonski i helenistički (od Aristotela) γίνομαι
\item[ἀνθρωπική ἐστιν] imenski predikat, Smyth 909
\item[διακρίνει] §~231; kongruencija sa subjektom u pluralu srednjeg roda §~361
\item[τοῖς μὲν\dots\ τοῖς δ' οὔ] koordinacija rečeničnih članova parom veznika%sc. χαίρει
\item[χαίρει] rekcija τινί uživati u nečemu; §~231
\item[εἰ] uvodi pogodbenu rečenicu, ovdje se radi o kombinaciji realnih protaza (ἐστιν ἡδὺ, διαφέρει) i potencijalne apodoze (πόρρω ἂν εἴη)
\item[μηδέν\dots\ μηδὲ\dots] koordinacija zanijekanih dijelova rečenice
\item[ἐστιν ἡδὺ] imenski predikat, Smyth 909; izraz otvara mjesto dativu interesa, §~412.1
\item[διαφέρει] rekcija τινός od nekoga; §~231
\item[πόρρω ἂν εἴη] πόρρω τινός LSJ πρόσω B.II;  §~315; ; imenski predikat Smyth 909; ἄν + optativ izriče mogućnost u sadašnjosti (potencijal sadašnji), §~464.2
\item[τοῦ ἄνθρωπος εἶναι] supstantivirani infinitiv §~497; §~315
\item[τέτευχε] §~272; rekcija τινός, LSJ τυγχάνω B.II.2
\item[τὸ μὴ πάνυ γίνεσθαι] supstantivirani infinitiv §~497; §~232; γίνομαι kao gore; uz infinitiv negacija je μή, πάνυ kao gore

\end{description}
%3

{\large
\begin{greek}
\noindent  ὁ δὲ σώφρων \\
μέσως μὲν \\
\tabto{2em} περὶ ταῦτ' \\
ἔχει· \\
\tabto{2em} οὔτε γὰρ ἥδεται \\
\tabto{4em} οἷς μάλιστα \\
\tabto{2em} ὁ ἀκόλαστος, \\
\tabto{4em} ἀλλὰ μᾶλλον δυσχεραίνει, \\
\tabto{2em} οὐδ' ὅλως \\
\tabto{4em} οἷς μὴ δεῖ \\
\tabto{2em} οὐδὲ σφόδρα τοιούτῳ οὐδενί, \\
\tabto{4em} οὔτ' ἀπόντων λυπεῖται \\
\tabto{4em} οὐδ' ἐπιθυμεῖ, \\
\tabto{2em} ἢ μετρίως, \\
\tabto{4em} οὐδὲ μᾶλλον ἢ δεῖ,
\tabto{4em} οὐδ' ὅτε μὴ δεῖ, \\
\tabto{4em} οὐδ' ὅλως \\
\tabto{6em} τῶν τοιούτων οὐδέν· \\
ὅσα δὲ \\
\tabto{2em} πρὸς ὑγίειάν ἐστιν \\
\tabto{2em} ἢ πρὸς εὐεξίαν \\
ἡδέα ὄντα, \\
\tabto{2em} τούτων ὀρέξεται μετρίως \\
\tabto{4em} καὶ ὡς δεῖ, \\
\tabto{2em} καὶ τῶν ἄλλων ἡδέων \\
\tabto{4em} μὴ ἐμποδίων τούτοις ὄντων \\
\tabto{6em} ἢ παρὰ τὸ καλὸν \\
\tabto{6em} ἢ ὑπὲρ τὴν οὐσίαν. \\

\end{greek}
}

\begin{description}[noitemsep]
\item[ὁ δὲ σώφρων] kopulativno δέ označava prijelaz iz rečenice u rečenicu; sljedeća rečenica iskazuje nešto novo ili drugačije, Smyth §~2386; supstantiviranje članom svih vrsta riječi §~373
\item[μέσως μὲν\dots\ ὅσα δὲ\dots] koordinacija rečeničnih članova parom čestica
\item[ἔχει] §~231; LSJ ἔχω B.II.2
\item[οὔτε γὰρ\dots\ οὐδ' ὅλως\dots\ οὐδὲ σφόδρα\dots] koordinacija pomoću (niječnih) sastavnih veznika; γάρ uvodi objašnjenje prethodne tvrdnje: naime\dots
\item[ἥδεται οἷς\dots] rekcija τινί u nečemu; §~232
\item[οἷς μάλιστα] sc.\ ἥδεται; zamjenica istovremeno služi kao objekt predikata i uvodi relativnu rečenicu (na hrvatskom se antecedent mora izraziti: u onome u čemu\dots)
\item[δυσχεραίνει] §~231; rekcija τινί
\item[ὅλως] sc.\ ἥδεται
\item[οἷς μὴ δεῖ] upotreba relativa kao gore; δεῖ §~243, bezlično, otvara mjesto ovdje neizrečenoj dopuni u infinitivu (sc.\ ἥδεσθαι)
\item[ἀπόντων] složenica εἰμί, §~315; participski dio GA čiji je imenski dio neizrečen (ἡδέων)
\item[λυπεῖται] §~243
\item[ἐπιθυμεῖ] §~243
\item[μετρίως] sc.\ λυπεῖται ἢ ἐπιθυμεῖ
\item[δεῖ] §~243, bezlično, otvara mjesto ovdje neizrečenoj dopuni u infinitivu (sc.\ λυπεῖσθαι ἢ ἐπιθυμεῖν)
\item[ὅτε μὴ δεῖ] veznik uvodi zavisnu vremensku rečenicu; predikat je bezličan glagol nepotpuna značenja, kao gore
\item[ὅσα\dots\ ἡδέα ὄντα, τούτων\dots] zamjenica uvodi dvije zavisne relativne rečenice (πρὸς ὑγίειάν ἐστιν, πρὸς εὐεξίαν sc.\ ἐστίν) §~481; antecedent (ἡδέα ὄντα) ovdje je uključen u zavisnu rečenicu, te se u padežu slaže s relativom, a u glavnoj rečenici umjesto njega stoji pokazna zamjenica (τούτων), Smyth 2536, 2538
\item[πρὸς ὑγίειάν ἐστιν] imenski predikat Smyth 909, kongruencija sa subjektom u pluralu srednjeg roda §~361
\item[πρὸς εὐεξίαν] imenski predikat (s izostavljenom kopulom) Smyth 909
\item[ἡδέα ὄντα] §~315; kao kopulativni glagol, εἰμί otvara mjesto nužnoj imenskoj dopuni
\item[ὀρέξεται] §~258; medpas. rekcija τινός
\item[ὡς δεῖ] veznik uvodi usporedbu: kao što\dots; predikat je bezličan glagol nepotpuna značenja, kao gore
\item[τῶν ἄλλων ἡδέων μὴ ἐμποδίων\dots\ ὄντων] GA; §~315; kao kopulativni glagol, εἰμί otvara mjesto nužnoj imenskoj dopuni
\item[ἐμποδίων] rekcija τινὶ
\item[παρὰ τὸ καλὸν] sc.\ ὄντων
\item[ὑπὲρ τὴν οὐσίαν] sc.\ ὄντων

\end{description}

%4

{\large
\begin{greek}
\noindent  ὁ γὰρ οὕτως ἔχων \\
μᾶλλον ἀγαπᾷ \\
τὰς τοιαύτας ἡδονὰς \\
\tabto{2em} τῆς ἀξίας· \\
ὁ δὲ σώφρων \\
οὐ τοιοῦτος, \\
ἀλλ' ὡς ὁ ὀρθὸς λόγος.\\

\end{greek}
}

\begin{description}[noitemsep]
\item[ὁ γὰρ οὕτως ἔχων] sc.\ ἀκόλαστος; §~231; LSJ ἔχω B.II.2; supstantiviranje participa članom §~499
\item[ἀγαπᾷ] §~243
\item[μᾶλλον\dots\ τῆς ἀξίας] sc.\ ἡδονῆς; genitivus comparationis §~404
\item[οὐ τοιοῦτος] sc.\ ἐστί; imenski predikat (s izostavljenom kopulom) Smyth 909
\item[ὡς ὁ ὀρθὸς λόγος] LSJ λόγος III.2.d
\end{description}



%kraj
