% Unesi ispravke VR, 9. 4. 2020.
% Unio ispravke NZ, <2021-12-31 pet>
%\section*{O autoru}

%TKTK


\section*{O tekstu}

U odlomku iz treće knjige Aristotelove \textit{Retorike} čitamo zapažanja o formalnim karakteristikama umjetničke proze. Zahtjev za posebnim oblikovanjem nevezanog govora prvi su, po antičkom uvjerenju, iznijeli sofisti iz V.~st.\ pr.~Kr.\ Trazimah iz Halkedona i Gorgija iz Leontina (obojica se pojavljuju i kao likovi u Platonovim djelima); Trazimah je usto definirao konvenciju koje će se držati čitava antika, da umjetnička proza mora biti periodizirana, odnosno ritmička.

Osnovnu ideju – da prozni zapis ne smije biti sastavljen u pravilnom metričkom obrascu, ali da ne smije ni biti bez dojma metra – Aristotel dopunjava dodatnim objašnjenjima i određenjima. Proza s formom metra nije uvjerljiva i čini se odviše umjetnom; metar, osim toga, privlači pažnju slušatelja, a zbog metričkih obrazaca takva proza postaje predvidljivom. S druge strane, proza bez metra previše je neodređena. Kako je broj onaj koji daje oblik stvarima, potreban je i prozi. Taj broj po Aristotelu u prozi je prepoznatljiv kao \textit{ritam}. 

Aristotel zatim preispituje razne vrste ritmova (herojski, jamb, trohej i pean) i pokazuje zašto je pean najprimjereniji za prozu.

Osim prvog nama poznatog teorijskog razmatranja poetike proze, odlomak je zanimljiv i kao ilustracija Aristotelove metode izlaganja; on polazi od općenite tvrdnje koju potom objašnjava te po potrebi dopunjava dodatnim podacima i ilustracijama. Pritom je Aristotelova dikcija ovdje odmjerena, a sintaksa relativno jednostavna. Prevladavaju imenski predikati, karakteristični za diskurs definiranja i objašnjavanja. Kombiniraju se čestice μέν, δέ i γάρ koje uvode dopune misli i izricanja suprotnosti.


\newpage

\section*{Pročitajte naglas grčki tekst.}

Arist.\ Rhetorica 1408b 21

%Naslov prema izdanju

\medskip

{\large
\begin{greek}
\noindent τὸ δὲ σχῆμα τῆς λέξεως δεῖ μήτε ἔμμετρον εἶναι μήτε ἄρρυθμον· τὸ μὲν γὰρ ἀπίθανον (πεπλάσθαι γὰρ δοκεῖ), καὶ ἅμα καὶ ἐξίστησι· προσέχειν γὰρ ποιεῖ τῷ ὁμοίῳ, πότε πάλιν ἥξει· ὥσπερ οὖν τῶν κηρύκων προλαμβάνουσι τὰ παιδία τὸ ``τίνα αἱρεῖται ἐπίτροπον ὁ ἀπελευθερούμενος;'' ``Κλέωνα''· τὸ δὲ ἄρρυθμον ἀπέραντον, δεῖ δὲ πεπεράνθαι μέν, μὴ μέτρῳ δέ· ἀηδὲς γὰρ καὶ ἄγνωστον τὸ ἄπειρον. περαίνεται δὲ ἀριθμῷ πάντα· ὁ δὲ τοῦ σχήματος τῆς λέξεως ἀριθμὸς ῥυθμός ἐστιν, οὗ καὶ τὰ μέτρα τμήματα· διὸ ῥυθμὸν δεῖ ἔχειν τὸν λόγον, μέτρον δὲ μή· ποίημα γὰρ ἔσται. ῥυθμὸν δὲ μὴ ἀκριβῶς· τοῦτο δὲ ἔσται ἐὰν μέχρι του ᾖ. τῶν δὲ ῥυθμῶν ὁ μὲν ἡρῷος σεμνῆς ἀλλ᾽ οὐ λεκτικῆς ἁρμονίας δεόμενος, ὁ δ᾽ ἴαμβος αὐτή ἐστιν ἡ λέξις ἡ τῶν πολλῶν (διὸ μάλιστα πάντων τῶν μέτρων ἰαμβεῖα φθέγγονται λέγοντες), δεῖ δὲ σεμνότητα γενέσθαι καὶ ἐκστῆσαι. ὁ δὲ τροχαῖος κορδακικώτερος· δηλοῖ δὲ τὰ τετράμετρα· ἔστι γὰρ τροχερὸς ῥυθμὸς τὰ τετράμετρα. λείπεται δὲ παιάν, ᾧ ἐχρῶντο μὲν ἀπὸ Θρασυμάχου ἀρξάμενοι, οὐκ εἶχον δὲ λέγειν τίς ἦν. ἔστι δὲ τρίτος ὁ παιάν, καὶ ἐχόμενος τῶν εἰρημένων· τρία γὰρ πρὸς δύ᾽ ἐστίν, ἐκείνων δὲ ὁ μὲν ἓν πρὸς ἕν, ὁ δὲ δύο πρὸς ἕν, ἔχεται δὲ τῶν λόγων τούτων ὁ ἡμιόλιος· οὗτος δ᾽ ἐστὶν ὁ παιάν. οἱ μὲν οὖν ἄλλοι διά τε τὰ εἰρημένα ἀφετέοι, καὶ διότι μετρικοί· ὁ δὲ παιὰν ληπτέος· ἀπὸ μόνου γὰρ οὐκ ἔστι μέτρον τῶν ῥηθέντων ῥυθμῶν, ὥστε μάλιστα λανθάνειν.

\end{greek}
}

\newpage

\section*{Analiza i komentar}

%1

{\large
\begin{greek}
\noindent Τὸ δὲ σχῆμα \\
\tabto{2em} τῆς λέξεως \\
\tabto{4em} δεῖ \\
\tabto{6em} μήτε ἔμμετρον εἶναι \\
\tabto{6em} μήτε ἄρρυθμον· \\
\tabto{8em} τὸ μὲν γὰρ \\
\tabto{10em} ἀπίθανον \\
\tabto{8em} (πεπλάσθαι γὰρ δοκεῖ), \\
\tabto{8em} καὶ ἅμα καὶ ἐξίστησι· \\
\tabto{8em} προσέχειν γὰρ ποιεῖ \\
\tabto{10em} τῷ ὁμοίῳ, \\
\tabto{10em} πότε πάλιν ἥξει· \\
\tabto{8em} ὥσπερ οὖν \\
\tabto{10em} τῶν κηρύκων \\
\tabto{12em} προλαμβάνουσι τὰ παιδία \\
\tabto{12em} τὸ ``τίνα αἱρεῖται ἐπίτροπον ὁ ἀπελευθερούμενος;'' ``Κλέωνα''·\\
\tabto{8em} τὸ δὲ ἄρρυθμον ἀπέραντον, \\
\tabto{10em} δεῖ δὲ \\
\tabto{12em} πεπεράνθαι μέν, \\
\tabto{12em} μὴ μέτρῳ δέ·\\
\tabto{12em} ἀηδὲς γὰρ καὶ ἄγνωστον \\
\tabto{14em} τὸ ἄπειρον.\\

\end{greek}
}

\begin{description}[noitemsep]
\item[δεῖ] §~492, δεῖ otvara mjesto akuzativu s infinitivom kao dopuni
\item[εἶναι] §~315
\item[μήτε… μήτε] §~513.4, negirani koordinativni veznici uvode nezavisno složene sastavne rečenice: niti… niti…
\item[μὲν γὰρ ] odnosi se na prvu od prethodno izrečenih tvrdnji o formama dikcije, ἔμμετρον: prvo…; \textgreek[variant=ancient]{τὸ μὲν γὰρ ἀπίθανον} prvi je dio koordinacije koji slijedi \textgreek[variant=ancient]{τὸ δὲ ἄρρυθμον ἀπέραντον}
\item[ἀπίθανον] imenski dio imenskog predikata (kopula ἐστί je neizrečena)
\item[πεπλάσθαι] §~272; LSJ πλάσσω A.V: \textit{metaforično} izraditi, umjetno proizvesti
\item[γὰρ] čestica ovdje u uzročnom eksplanatornom značenju: jer naime…
\item[δοκεῖ] §~492, s dopunom u infinitivu
\item[ἐξίστησι] §~305, §~311; ovdje u značenju: odvraćati pažnju, LSJ s.v. A.2
\item[προσέχειν] rekcija: τινι; augment §~238
\item[ποιεῖ] §~243 (glag. osnove §~301.B s.~116)
\item[ἥξει] §~258, tvorba futurske osnove §~261
\item[πότε… ἥξει] upitni vremenski prilog πότε (§~221) uvodi zavisnu upitnu rečenicu (§~469): kada…
\item[προλαμβάνουσι ] rekcija: τινος; augment §~238 (glag.\ osnove §~321.14); LSJ s.v. II.3.b.: očekivati od koga
\item[ὥσπερ… προλαμβάνουσι] poredbeni veznik ὥσπερ uvodi zavisnu poredbenu rečenicu: kao\dots
\item[αἱρεῖται] rekcija: τινα τι; §~243
\item[ὁ ἀπελευθερούμενος] §~243, supstantivirani particip §~498-499; \textgreek[variant=ancient]{ἀπελευθερούμενος} je oslobođenik (oslobođeni rob), koji ne uživa puno građansko pravo, te mu u sudskim postupcima i sličnim situacijama treba zastupnik ili zaštitnik; Kleon je primjer političara-populista, na strani malih ljudi i potlačenih, te su i djeca znala odgovor u takvom slučaju
\item[τὸ] \textbf{\textgreek[variant=ancient]{``τίνα αἱρεῖται ἐπίτροπον ὁ ἀπελευθερούμενος;''}} član τὸ supstantivira direktno pitanje koje se citira pod navodnicima
\item[τίνα… αἱρεῖται] upitna zamjenica τίνα uvodi direktno, nezavisno pitanje: koga…; rekonstrukcija: \textgreek[variant=ancient]{τὰ παιδία προλαμβάνουσι τὸ (ἐρώτημα) τῶν κηρύκων “τίνα αἱρεῖται ἐπίτροπον ὁ ἀπελευθερούμενος;” (τὰ παιδία ἀμείβεται) “Κλέωνα”}
\item[τὸ δὲ ἄρρυθμον] „a drugo…“
\item[ἀπέραντον] imenski dio imenskog predikata (glagolski dio, kopula ἐστί, je neizrečen)
\item[δεῖ] §~492, glagol δεῖ traži dopunu u infinitivu
\item[πεπεράνθαι] §~291.b.2; LSJ περαίνω: dovesti do kraja, dovršiti
\item[ἀηδὲς γὰρ καὶ ἄγνωστον] imenski dijelovi imenskog predikata (glagolski dio, kopula ἐστί, je neizrečen)
\end{description}

%2

{\large
\begin{greek}
\noindent περαίνεται δὲ \\
\tabto{2em} ἀριθμῷ \\
\tabto{4em} πάντα· \\
ὁ δὲ \\
\tabto{2em} τοῦ σχήματος \\
\tabto{4em} τῆς λέξεως \\
ἀριθμὸς \\
\tabto{2em} ῥυθμός ἐστιν, \\
\tabto{4em} οὗ καὶ \\
\tabto{6em} τὰ μέτρα \\
\tabto{8em} τμήματα· \\
διὸ ῥυθμὸν \\
\tabto{2em} δεῖ \\
\tabto{4em} ἔχειν τὸν λόγον, \\
μέτρον δὲ \\
\tabto{2em} μή· \\
\tabto{4em} ποίημα γὰρ ἔσται.\\

\end{greek}
}

\begin{description}[noitemsep]
\item[περαίνεται] §~231
\item[δὲ] čestica adverzativnog značenja: a…
\item[ἐστιν] §~315
\item[ῥυθμός ἐστιν] imenski predikat
\item[οὗ…] sc. ἐστί; odnosna zamjenica uvodi zavisnu odnosnu rečenicu: kojeg…
\item[διὸ]  = δι᾽ ὅ; zaključno: zato…
\item[δεῖ] glagol δεῖ otvara mjesto akuzativu s infinitivom; subjekt se u ovakvim slučajevima prepoznaje po članu (τὸν λόγον); ``zato treba da govor ima ritam…''
\item[ἔχειν] §~231 (glag.\ osnove §~327.13)
\item[γὰρ ] čestica eksplanatorno-uzročnog značenja ovdje podupire istinitost prethodne tvrdnje: jer inače…
\item[ἔσται] §~315
\end{description}

%3

{\large
\begin{greek}
\noindent ῥυθμὸν δὲ \\
\tabto{2em} μὴ ἀκριβῶς· \\
\tabto{4em} τοῦτο δὲ ἔσται \\
\tabto{4em} ἐὰν \\
\tabto{6em} μέχρι του ᾖ.\\

\end{greek}
}

\begin{description}[noitemsep]
\item[ἔσται] kao u prethodnoj rečenici!
\item[ᾖ] §~315
\item[ἐὰν… ᾖ] pogodbeni veznik ἐάν uvodi zavisnu pogodbenu rečenicu eventualnog futurskog značenja: ako…
\item[μέχρι του ᾖ] imenski predikat u kojem prijedložni izraz ima funkciju imenskog dijela
\item[του] = τινός
\end{description}

%4

{\large
\begin{greek}
\noindent τῶν δὲ ῥυθμῶν \\
\tabto{2em} ὁ μὲν ἡρῷος \\
\tabto{4em} σεμνῆς \\
\tabto{4em} ἀλλ' οὐ \\
\tabto{4em} λεκτικῆς ἁρμονίας \\
\tabto{6em} δεόμενος, \\
\tabto{2em} ὁ δ' ἴαμβος \\
\tabto{4em} αὐτή ἐστιν ἡ λέξις \\
\tabto{6em} ἡ τῶν πολλῶν \\
(διὸ μάλιστα πάντων τῶν μέτρων \\
\tabto{2em} ἰαμβεῖα φθέγγονται \\
\tabto{4em} λέγοντες), \\
\tabto{2em} δεῖ δὲ \\
\tabto{4em} σεμνότητα γενέσθαι \\
\tabto{4em} καὶ ἐκστῆσαι.\\

\end{greek}
}

\begin{description}[noitemsep]
\item[ὁ ἡρῷος] herojski metar, tj.\ daktil (u kvantitativnoj metrici: dugi, kratki, kratki slog, ¯ ˘ ˘)
\item[σεμνῆς] \textbf{\textgreek[variant=ancient]{ἀλλ᾽ οὐ λεκτικῆς ἁρμονίας δεόμενος}} LSJ δέω II.1, ``potrebno je'', ``zahtijeva'' (ono čemu je skladnost potrebna jest herojski metar, \textgreek[variant=ancient]{ὁ ἡρῷος);} u kritičkim izdanjima, filološkim i filozofskim komentarima postoje različita čitanja i tumačenja ovog mjesta, prijepornog po tome kakvu skladnost herojski metar ima ili nema
\item[δεόμενος] §~243, stezanje jednosložnih osnova §~244
\item[ὁ ἴαμβος] jamb, u kvantitativnoj metrici: kratki, dugi slog, ˘ ¯
\item[ἐστιν] §~315
\item[διὸ… φθέγγονται] umetnuta zavisna odnosna rečenica koju uvodi veznik διὸ (nastao od prijedložnog izraza odnosne zamjenice = δι᾽ ὅ: zbog čega…)
\item[φθέγγονται] §~241
\item[λέγοντες] §~241
\item[δεῖ δὲ σεμνότητα] \textbf{γενέσθαι καὶ ἐκστῆσαι} ovaj se izraz odnosi na \textgreek[variant=ancient]{ἡ λέξις}, opisuje koja je svrha takvog izražavanja
\item[δεῖ] §~492, glagol δεῖ otvara mjesto akuzativu s infinitivom (kao i gore)
\item[γενέσθαι] §~254
\item[ἐκστῆσαι] §~311.2 (ostala vremena: §~305, §~311.1); ovdje, suprotno prethodnoj pojavi istog glagola, ima značenje ``poticati'', ``privlačiti'' (značenje slično prethodnom \textgreek[variant=ancient]{σεμνότητα γενέσθαι);} LSJ ἐξίστημι A.2
\end{description}

%5

{\large
\begin{greek}
\noindent ὁ δὲ τροχαῖος \\
\tabto{2em} κορδακικώτερος· \\
\tabto{4em} δηλοῖ δὲ \\
\tabto{6em} τὰ τετράμετρα· \\
\tabto{4em} ἔστι γὰρ \\
\tabto{6em} τροχερὸς ῥυθμὸς \\
\tabto{8em} τὰ τετράμετρα.\\

\end{greek}
}

\begin{description}[noitemsep]
\item[ὁ τροχαῖος] trohej, u kvantitativnoj metrici: dug, kratak ¯ ˘
\item[κορδακικώτερος] imenski dio imenskog predikata (glagolski dio, kopula ἐστί ostaje neizrečen)
\item[δηλοῖ] sc.\ τοῦτο; §~243
\item[ἔστι] §~315
\item[γὰρ] čestica γὰρ se ovdje veže na prethodno izrečenu misao i dopunjuje ju: naime… ili jer…
\item[ἔστι τροχερὸς ῥυθμὸς] imenski predikat
\end{description}

%6

{\large
\begin{greek}
\noindent λείπεται δὲ \\
\tabto{2em} παιάν, \\
\tabto{4em} ᾧ ἐχρῶντο μὲν \\
\tabto{6em} ἀπὸ Θρασυμάχου \\
\tabto{8em} ἀρξάμενοι,\\
\tabto{4em} οὐκ εἶχον δὲ \\
\tabto{6em} λέγειν \\
\tabto{8em} τίς ἦν.\\

\end{greek}
}

\begin{description}[noitemsep]
\item[λείπεται] §~241 
\item[παιάν] pean, u kvantitativnoj metrici: različite kombinacije jednog dugog i tri kratka sloga (dugi može biti na prvom, drugom, trećem ili četvrtom mjestu)
\item[ἐχρῶντο μὲν… εἶχον δὲ] koordinacija pomoću čestica μέν… δέ: μέν daje prvo objašnjenje početne misli (λείπεται παιάν), a δέ ističe novu, dodatnu informaciju: naime… ali…; subjekt je oba predikata isti, \textgreek[variant=ancient]{ἀπὸ Θρασυμάχου ἀρξάμενοι}
\item[ἐχρῶντο] rekcija: τινι; §~243, stezanje §~244.2 (glag.\ osnove §~301.B s.~116)
\item[ᾧ ἐχρῶντο] odnosna zamjenica ᾧ uvodi zavisnu odnosnu rečenicu: kojim…
\item[ἀρξάμενοι] §~267 (glag.\ osnove §~301.B s.~116)
\item[εἶχον ] §~241, augment §~236; glagol otvara mjesto dopuni u infinitivu uz značenjski pomak uz glagol govorenja: moći
\item[λέγειν ] §~241 (glag.\ osnove §~327.7)
\item[ἦν] §~315
\item[τίς ἦν] upitna zamjenica τίς uvodi zavisnu upitnu rečenicu (neupravno pitanje), §~469; zamjenica ne označava osobu (Trazimaha) nego (ὁ) παιάν: „kakav\dots” ili „što\dots”
\end{description}

%7

{\large
\begin{greek}
\noindent ἔστι δὲ τρίτος \\
\tabto{2em} ὁ παιάν, \\
καὶ ἐχόμενος \\
\tabto{2em} τῶν εἰρημένων· \\
τρία γὰρ πρὸς δύ' ἐστίν, \\
ἐκείνων δὲ \\
\tabto{2em} ὁ μὲν \\
\tabto{4em} ἓν πρὸς ἕν, \\
\tabto{2em} ὁ δὲ \\
\tabto{4em} δύο πρὸς ἕν, \\
\tabto{2em} ἔχεται δὲ \\
\tabto{4em} τῶν λόγων τούτων \\
\tabto{6em} ὁ ἡμιόλιος· \\
\tabto{2em} οὗτος δ' ἐστὶν \\
\tabto{4em} ὁ παιάν.\\

\end{greek}
}

\begin{description}[noitemsep]
\item[ἔστι] §~315
\item[ἔστι τρίτος] imenski predikat
\item[ἐχόμενος] rekcija: τινος; §~231
\item[τῶν εἰρημένων] §~272 (glag.\ osnove §~327.7)
\item[ἐστίν] §~315
\item[τρία γὰρ πρὸς δύ' ἐστίν] pean se može opisati razmjerom 3:2, dok ostale metre opisuju razmjeri 1:1 i 2:1
\item[ἐκείνων δὲ…] \textbf{ὁ μὲν… ὁ δὲ} prvi δὲ ovdje ima značenje adverzativnoga veznika: ali…; dok je ὁ μὲν… ὁ δὲ u koordinaciji: jedan… drugi…
\item[ἓν πρὸς ἕν] imenski dijelovi imenskoga predikata (neizrečena kopula ἐστίν)
\item[δύο πρὸς ἕν] imenski dijelovi imenskoga predikata (neizrečena kopula ἐστίν)
\item[ἔχεται] rekcija τινος; §~231
\item[ἐστὶν] §~315
\end{description}

%8


{\large
\begin{greek}
\noindent οἱ μὲν οὖν ἄλλοι \\
\tabto{2em} διά τε τὰ εἰρημένα \\
\tabto{4em} ἀφετέοι, \\
\tabto{4em} καὶ διότι \\
\tabto{6em} μετρικοί· \\
ὁ δὲ παιὰν \\
\tabto{2em} ληπτέος· \\
ἀπὸ μόνου γὰρ \\
\tabto{2em} οὐκ ἔστι μέτρον \\
\tabto{4em} τῶν ῥηθέντων ῥυθμῶν, \\
ὥστε μάλιστα λανθάνειν.\\

\end{greek}
}

\begin{description}[noitemsep]
\item[τὰ εἰρημένα] §~272 (glag.\ osnove §~327.7), supstantivirani particip §~499.2
\item[ἀφετέοι] §~300; oblik tvori imenski dio imenskoga predikata (neizrečena je kopula εἰσί)
\item[μετρικοί] imenski dio imenskoga predikata (neizrečena je kopula εἰσί)
\item[ληπτέος] §~300, oblik tvori imenski dio imenskoga predikata (neizrečena je kopula ἐστὶ)
\item[ἀπὸ μόνου] tj.\ pean nije prilagođen metričkom sistemu, jer se njegov razmjer ne može iskazati cijelim brojem 
\item[ἔστι] §~315
\item[τῶν ῥηθέντων] §~296 (glag.\ osnove §~327.7), izraz ima tekstualnu funkciju: od gorespomenutih\dots
\item[ὥστε… λανθάνειν] posljedični veznik ὥστε uvodi zavisnu posljedičnu rečenicu: tako da… §~473
\item[λανθάνειν] §~241 (glag.\ osnove §~321.15)
\end{description}



%kraj
