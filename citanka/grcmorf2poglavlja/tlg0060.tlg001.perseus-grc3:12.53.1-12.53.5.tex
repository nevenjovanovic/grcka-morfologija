% Prva verzija NJ <2022-01-08 sub>


\section*{O tekstu}

U dvanaestoj od četrdeset knjiga \textit{Knjižnice} \textgreek{(Βιβλιοθήκη)} Diodor Sicilski, povjesničar iz I.~st. pr.~Kr., prikazuje, između ostalog, i Peloponeski rat. Ovdje odabrani odlomak pripovijeda o pokretanju Prve sicilske ekspedicije, atenske intervencije iz 427.\ (kojoj će desetljeće kasnije slijediti mnogo poznatija, zlosretna ekspedicija 415.–413.). Saveznički grad Leontini zatražio je pomoć Atene u sukobu između dorskih i halkidskih gradova na Siciliji; vođa poslanstva bio je slavni retor i filozof Gorgija. Njegovo je govorničko umijeće, prema Diodoru, opčinilo Atenjane.\footnote{Gorgijin diplomatski uspjeh spominje i Platon u dijalogu \textit{Hipija Veći,} 282b: \textgreek{Γοργίας τε γὰρ οὗτος ὁ Λεοντῖνος σοφιστὴς δεῦρο ἀφίκετο δημοσίᾳ οἴκοθεν πρεσβεύων, ὡς ἱκανώτατος ὢν Λεοντίνων τὰ κοινὰ πράττειν, καὶ ἔν τε τῷ δήμῳ ἔδοξεν ἄριστα εἰπεῖν, καὶ ἰδίᾳ ἐπιδείξεις ποιούμενος καὶ συνὼν τοῖς νέοις χρήματα πολλὰ ἠργάσατο καὶ ἔλαβεν ἐκ τῆσδε τῆς πόλεως.} Primjer Gorgijine retorike daje Poglavlje \ref{chap:gorgias} ove čitanke.}

Tukidid, koji o Prvoj sicilskoj ekspediciji govori u trećoj i četvrtoj knjizi svoje povijesti (ovdje opisan događaj prikazuje u 3, 86), daje naslutiti da Atenjani ipak nisu bili do kraja svladani Gorgijinim umijećem; u prvom su kontingentu, naime, na Siciliju poslali samo dvadeset brodova.

%\newpage

\section*{Pročitajte naglas grčki tekst.}

Diod.~Sic.\ Bibliotheca historica 12.53.1–5

%Naslov prema izdanju

\medskip


{\large

\begin{greek}

\noindent ἐπὶ δὲ τούτων κατὰ τὴν Σικελίαν Λεοντῖνοι, Χαλκιδέων μὲν ὄντες ἄποικοι, συγγενεῖς δὲ Ἀθηναίων, ἔτυχον ὑπὸ Συρακοσίων πολεμούμενοι. πιεζόμενοι δὲ τῷ πολέμῳ, καὶ διὰ τὴν ὑπεροχὴν τῶν Συρακοσίων κινδυνεύοντες ἁλῶναι κατὰ κράτος, ἐξέπεμψαν πρέσβεις εἰς τὰς Ἀθήνας, ἀξιοῦντες τὸν δῆμον βοηθῆσαι τὴν ταχίστην καὶ τὴν πόλιν ἑαυτῶν ἐκ τῶν κινδύνων ῥύσασθαι. ἦν δὲ τῶν ἀπεσταλμένων ἀρχιπρεσβευτὴς Γοργίας ὁ ῥήτωρ, δεινότητι λόγου πολὺ προέχων πάντων τῶν καθ' ἑαυτόν. οὗτος καὶ τέχνας ῥητορικὰς πρῶτος ἐξεῦρε καὶ κατὰ τὴν σοφιστείαν τοσοῦτο τοὺς ἄλλους ὑπερέβαλεν, ὥστε μισθὸν λαμβάνειν παρὰ τῶν μαθητῶν μνᾶς ἑκατόν. οὗτος οὖν καταντήσας εἰς τὰς Ἀθήνας καὶ παραχθεὶς εἰς τὸν δῆμον διελέχθη τοῖς Ἀθηναίοις περὶ τῆς συμμαχίας, καὶ τῷ ξενίζοντι τῆς λέξεως ἐξέπληξε τοὺς Ἀθηναίους ὄντας εὐφυεῖς καὶ φιλολόγους. πρῶτος γὰρ ἐχρήσατο τοῖς τῆς λέξεως σχηματισμοῖς περιττοτέροις καὶ τῇ φιλοτεχνίᾳ διαφέρουσιν, ἀντιθέτοις καὶ ἰσοκώλοις καὶ παρίσοις καὶ ὁμοιοτελεύτοις καί τισιν ἑτέροις τοιούτοις, ἃ τότε μὲν διὰ τὸ ξένον τῆς κατασκευῆς ἀποδοχῆς ἠξιοῦτο, νῦν δὲ περιεργίαν ἔχειν δοκεῖ καὶ φαίνεται καταγέλαστα πλεονάκις καὶ κατακόρως τιθέμενα. τέλος δὲ πείσας τοὺς Ἀθηναίους συμμαχῆσαι τοῖς Λεοντίνοις, οὗτος μὲν θαυμασθεὶς ἐν ταῖς Ἀθήναις ἐπὶ τέχνῃ ῥητορικῇ τὴν εἰς Λεοντίνους ἐπάνοδον ἐποιήσατο.

\end{greek}

}


\section*{Analiza i komentar}

%1


{\large
\begin{greek}
\noindent ἐπὶ δὲ τούτων κατὰ τὴν Σικελίαν\\
Λεοντῖνοι,\\
\tabto{2em} Χαλκιδέων μὲν ὄντες ἄποικοι,\\
\tabto{2em} συγγενεῖς δὲ Ἀθηναίων,\\
ἔτυχον ὑπὸ Συρακοσίων πολεμούμενοι.\\

\end{greek}
}

\begin{description}[noitemsep]
\item[ἐπὶ δὲ τούτων] LSJ ἐπί A.II, vremensko značenje; „u vrijeme tih (prethodno spomenutih) događaja”; δέ uspostavlja vezu s prethodnom rečenicom
\item[κατὰ τὴν Σικελίαν] LSJ κατά B.I.2, izriče prostor u kojem se nešto događa, „na području\dots”
\item[Χαλκιδέων\dots\ ἄποικοι] grad Leontini (danas Lentini) 730.\ pr.~Kr.\ osnovali su halkidski kolonisti s otoka Naksa
\item[μὲν\dots\ δὲ\dots] čestice koordiniraju rečenične članke
\item[ὄντες] §~315 (s imenskim dopunama)
\item[ἔτυχον] LSJ τυγχάνω A.II, otvara mjesto participu; §~321.19; jaka aorisna osnova §~254
\item[πολεμούμενοι] §~232; ὑπό τινος uz pasiv izriče vršitelja radnje

\end{description}

%2

{\large
\begin{greek}
\noindent πιεζόμενοι δὲ τῷ πολέμῳ, \\
καὶ διὰ τὴν ὑπεροχὴν τῶν Συρακοσίων κινδυνεύοντες \\
\tabto{2em} ἁλῶναι κατὰ κράτος,\\
ἐξέπεμψαν πρέσβεις εἰς τὰς Ἀθήνας, \\
ἀξιοῦντες τὸν δῆμον \\
\tabto{2em} βοηθῆσαι τὴν ταχίστην\\
\tabto{2em} καὶ τὴν πόλιν ἑαυτῶν ἐκ τῶν κινδύνων ῥύσασθαι.\\

\end{greek}
}

\begin{description}[noitemsep]
\item[πιεζόμενοι] §~232; otvara mjesto dopuni u dativu \textit{(dativus instrumenti)}
\item[κινδυνεύοντες] §~231; particip otvara mjesto infinitivu
\item[ἁλῶναι] §~324.5
\item[κατὰ κράτος] LSJ κατά B.VIII
\item[ἐξέπεμψαν] složenica glagola πέμπω, s.~116; §~267, §~269
\item[ἀξιοῦντες] §~231, §~243; ἀξιόω otvara mjesto ličnom objektu u akuzativu te infinitivu
\item[βοηθῆσαι] §~267, §~269 
\item[τὴν ταχίστην] LSJ ταχύς C.II.3 
\item[ῥύσασθαι] LSJ ἐρύω (B) A.5 

\end{description}

%3


{\large
\begin{greek}
\noindent ἦν δὲ τῶν ἀπεσταλμένων ἀρχιπρεσβευτὴς Γοργίας ὁ ῥήτωρ,\\
\tabto{2em} δεινότητι λόγου πολὺ προέχων πάντων\\
\tabto{4em} τῶν καθ' ἑαυτόν.\\

\end{greek}
}

\begin{description}[noitemsep]
\item[ἦν\dots\ ἀρχιπρεσβευτὴς] imenski predikat (Smyth 909); §~315.2
\item[τῶν ἀπεσταλμένων] LSJ ἀποστέλλω II; supstantivirani particip §~373, atribut imenice \textgreek{ἀρχιπρεσβευτὴς}
\item[Γοργίας ὁ ῥήτωρ] §~374.2 Bilj.
\item[προέχων] LSJ προέχω B.3, rekcija τινός τινι
\item[τῶν καθ' ἑαυτόν] LSJ κατά B.IV.3; prijedložni izraz u atributnom položaju §~375

\end{description}


%4


{\large
\begin{greek}
\noindent οὗτος καὶ τέχνας ῥητορικὰς πρῶτος ἐξεῦρε\\
καὶ κατὰ τὴν σοφιστείαν τοσοῦτο τοὺς ἄλλους ὑπερέβαλεν,\\
\tabto{2em} ὥστε μισθὸν λαμβάνειν παρὰ τῶν μαθητῶν\\
\tabto{4em} μνᾶς ἑκατόν.\\

\end{greek}
}

\begin{description}[noitemsep]
\item[οὗτος] sc.\ Γοργίας
\item[ἐξεῦρε] složenica glagola εὑρίσκω, §~254, §~324.7
\item[κατὰ τὴν σοφιστείαν] LSJ κατά B.IV.3
\item[τοσοῦτο\dots\ ὥστε\dots] priložno upotrijebljen srednji rod zamjenice (LSJ τοσοῦτος III) u korelaciji s veznikom otvara mjesto zavisno posljedičnoj rečenici
\item[ὑπερέβαλεν] složenica glagola βάλλω; §~254, §~301.B (s.~118)
\item[ὥστε\dots\ λαμβάνειν] veznik otvara mjesto zavisno posljedičnoj rečenici, u njoj predikat stoji u infinitivu, §~473; rekcija glagola λαμβάνω: τι παρά τινος
\item[μνᾶς ἑκατόν] mina (μνᾶ) je atička računska novčana veličina (nije nikad postojala kao posebna kovanica) u vrijednosti 100 drahmi; 60 mina činilo je talent; krajem V.~st. pr.~Kr.\ dnevni dohodak kvalificiranog radnika, kao i hoplita i veslača na trijeri, iznosio je jednu drahmu, te bi Gorgijin honorar vrijedio 10.000 prosječnih dnevnih dohodaka (uz hrvatski dnevni dohodak od 237 kuna u 2021, to je ekvivalent 2,3 milijuna kuna)

\end{description}

%5


{\large
\begin{greek}
\noindent οὗτος οὖν καταντήσας εἰς τὰς Ἀθήνας\\
\tabto{2em} καὶ παραχθεὶς εἰς τὸν δῆμον\\
διελέχθη τοῖς Ἀθηναίοις περὶ τῆς συμμαχίας, \\
καὶ τῷ ξενίζοντι τῆς λέξεως ἐξέπληξε τοὺς Ἀθηναίους \\
\tabto{2em} ὄντας εὐφυεῖς καὶ φιλολόγους.\\

\end{greek}
}

\begin{description}[noitemsep]
\item[καταντήσας] §~267, §~269 
\item[παραχθεὶς] složenica glagola ἄγω, LSJ παράγω III εἴς τι, εἴς τινα; §~296
\item[διελέχθη] složenica glagola λέγω, \textit{deponens;} §~296
\item[τῷ ξενίζοντι] supstantivirani particip §~373; §~231
\item[ἐξέπληξε] složenica glagola πλήσσω (atički πλήττω), §~267, §~269
\item[ὄντας] §~315 (s imenskim dopunama)
\item[εὐφυεῖς] LSJ εὐφυής III (pazite na mjesto spiritusa!)

\end{description}

% 6


{\large
\begin{greek}
\noindent πρῶτος γὰρ ἐχρήσατο \\
\tabto{2em} τοῖς τῆς λέξεως σχηματισμοῖς περιττοτέροις καὶ τῇ φιλοτεχνίᾳ διαφέρουσιν, \\
\tabto{2em} ἀντιθέτοις καὶ ἰσοκώλοις καὶ παρίσοις καὶ ὁμοιοτελεύτοις \\
\tabto{2em} καί τισιν ἑτέροις τοιούτοις, \\
\tabto{4em} ἃ τότε μὲν διὰ τὸ ξένον τῆς κατασκευῆς \\
\tabto{6em} ἀποδοχῆς ἠξιοῦτο, \\
\tabto{4em} νῦν δὲ περιεργίαν ἔχειν δοκεῖ \\
\tabto{4em} καὶ φαίνεται καταγέλαστα \\
\tabto{6em} πλεονάκις καὶ κατακόρως τιθέμενα.\\

\end{greek}
}

\begin{description}[noitemsep]
\item[ἐχρήσατο] rekcija τινί, §~267, §~269
\item[σχηματισμοῖς] LSJ σχηματισμός III; Diodor aludira na \textgreek{Γοργίεια σχήματα} (termin za skupinu karakterističnih stilskih figura koji nalazimo npr.\ kod Dionizija Halikarnašanina) i u nastavku daje njihov kratak popis
\item[διαφέρουσιν] LSJ διαφέρω II.8, rekcija τινί
\item[τότε μὲν\dots\ νῦν δὲ\dots] koordinacija rečeničnih članova pomoću čestica μέν… δέ… (i vremenskih priloga)
\item[τὸ ξένον] supstantivirani srednji rod pridjeva §~373
\item[ἠξιοῦτο] rekcija τινός, §~232, §~243
\item[ἔχειν] §~231 
\item[δοκεῖ] §~243; bezlični glagol otvara mjesto infinitivu, §~492
\item[φαίνεται] §~232; glagol nepotpuna značenja traži imensku dopunu
\item[τιθέμενα] §~305

\end{description}

% 7


{\large
\begin{greek}
\noindent τέλος δὲ \\
\tabto{2em} πείσας τοὺς Ἀθηναίους συμμαχῆσαι τοῖς Λεοντίνοις, \\
οὗτος μὲν \\
\tabto{2em} θαυμασθεὶς ἐν ταῖς Ἀθήναις ἐπὶ τέχνῃ ῥητορικῇ \\
τὴν εἰς Λεοντίνους ἐπάνοδον ἐποιήσατο.\\

\end{greek}
}

\begin{description}[noitemsep]
\item[τέλος] kao priložna oznaka LSJ τέλος II.2.a
\item[πείσας] §~267, §~269 otvara mjesto akuzativu osobe i infinitivu
\item[συμμαχῆσαι] §~267, §~269
\item[θαυμασθεὶς] §~296
\item[τὴν εἰς Λεοντίνους ἐπάνοδον] prijedložni izraz u atributnom položaju §~375
\item[ἐποιήσατο] §~267, §~269

\end{description}



%kraj
