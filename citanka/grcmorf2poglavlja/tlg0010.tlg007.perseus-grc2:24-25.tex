% Unio ispravke NZ <2022-01-03 pon>



\section*{O tekstu}

Rasprava o praktičnoj etici u obliku otvorenog pisma Demoniku pripisuje se Izokratu (436.–338. p.~n.~e), no nije isključeno da joj je autor neki njegov učenik. Nudeći upute za svakodnevni život, autor razlikuje tri životna područja u kojima ta pravila treba primjenjivati: 1) čovjek u odnosu prema bogovima, 2) čovjek u odnosu prema drugim ljudima, ponajprije prema roditeljima i prijateljima, 3) čovjek u odnosu prema sebi samom i skladan razvoj njegova karaktera. Pravila se nižu bez strogog reda, što je tipično za ``gnomsku'' književnost toga razdoblja, pa ne ostavljaju dojam cjeline. U tom smislu rasprava se može usporediti s pjesničkim opusom Megaranina Teognida (VI. st. p.~n.~e).

%\newpage

\section*{Pročitajte naglas grčki tekst.}

Isoc.\ Ad Demonicum 24–25

%Naslov prema izdanju

\medskip


{\large

\begin{greek}

\noindent μηδένα φίλον ποιοῦ, πρὶν ἂν ἐξετάσῃς πῶς κέχρηται τοῖς πρότερον φίλοις· ἔλπιζε γὰρ αὐτὸν καὶ περὶ σὲ γενέσθαι τοιοῦτον, οἷος καὶ περὶ ἐκείνους γέγονε. βραδέως μὲν φίλος γίγνου, γενόμενος δὲ πειρῶ διαμένειν· ὁμοίως γὰρ αἰσχρὸν μηδένα φίλον ἔχειν καὶ πολλοὺς ἑταίρους μεταλλάττειν. μήτε μετὰ βλάβης πειρῶ τῶν φίλων, μήτʼ ἄπειρος εἶναι τῶν ἑταίρων θέλε. τοῦτο δὲ ποιήσεις, ἐὰν μὴ δεόμενος τὸ δεῖσθαι προσποιῇ.

\noindent περὶ τῶν ῥητῶν ὡς ἀπορρήτων ἀνακοινοῦ· μὴ τυχὼν μὲν γὰρ οὐδὲν βλαβήσει, τυχὼν δὲ μᾶλλον αὐτῶν τὸν τρόπον ἐπιστήσει. δοκίμαζε τοὺς φίλους ἔκ τε τῆς περὶ τὸν βίον ἀτυχίας καὶ τῆς ἐν τοῖς κινδύνοις κοινωνίας· τὸ μὲν γὰρ χρυσίον ἐν τῷ πυρὶ βασανίζομεν, τοὺς δὲ φίλους ἐν ταῖς ἀτυχίαις διαγιγνώσκομεν. οὕτως ἄριστα χρήσει τοῖς φίλοις, ἐὰν μὴ προσμένῃς τὰς παρʼ ἐκείνων δεήσεις, ἀλλʼ αὐτεπάγγελτος αὐτοῖς ἐν τοῖς καιροῖς βοηθῇς.

\end{greek}

}


\section*{Analiza i komentar}

%1

{\large
\begin{greek}
\noindent Μηδένα \\
φίλον \\
\tabto{2em} ποιοῦ, \\
\tabto{4em} πρὶν ἂν ἐξετάσῃς \\
\tabto{6em} πῶς κέχρηται \\
\tabto{8em} τοῖς πρότερον φίλοις· \\
ἔλπιζε γὰρ \\
\tabto{2em} αὐτὸν \\
\tabto{2em} καὶ περὶ σὲ \\
\tabto{2em} γενέσθαι τοιοῦτον, \\
\tabto{4em} οἷος \\
\tabto{4em} καὶ περὶ ἐκείνους\\
\tabto{4em} γέγονεν. \\

\end{greek}
}

\begin{description}[noitemsep]
\item[Μηδένα φίλον] §~388, dva akuzativa (objekta i predikata) uz glagol koji znači ``činiti''; razlika među negacijama οὐ i μή §~508, §~509
\item[ποιοῦ] §~232
\item[πρὶν ἂν ἐξετάσῃς] veznik πρὶν uvodi vremensku rečenicu i otvara mjesto (eventualnom) konjunktivu + ἂν §~488.1
\item[ἐξετάσῃς] §~267
\item[πῶς κέχρηται] upitni prilog otvara mjesto zavisno upitnoj rečenici §~221, §~469
\item[κέχρηται] §~272, §~281
\item[τοῖς πρότερον φίλοις] atributni položaj priloga §~375
\item[ἔλπιζε γὰρ] §~231; §~493 \textit{verbum sentiendi} otvara mjesto A+I; §~517
\item[γενέσθαι] §~254, §~255 (naglasak)
\item[περὶ σὲ\dots\ περὶ ἐκείνους\dots] prema tebi\dots\ prema njima\dots\ §~433.C.c
\item[γέγονεν] §~272

\end{description}

%2


{\large
\begin{greek}
\noindent Βραδέως μὲν \\
\tabto{2em} φίλος γίγνου, \\
γενόμενος δὲ \\
πειρῶ \\
\tabto{2em} διαμένειν. \\


\end{greek}
}

\begin{description}[noitemsep]
\item[Βραδέως μὲν\dots\ γενόμενος δὲ\dots] koordinacija rečeničnih članova pomoću čestica
\item[γίγνου] §~232
\item[γενόμενος] sc.\ φίλος; §~254 
\item[πειρῶ] §~243; glagol otvara mjesto dopuni u infinitivu
\item[διαμένειν] sc.\ φίλος; §~232
\end{description}

%3

{\large
\begin{greek}
\noindent Ὁμοίως γὰρ αἰσχρὸν \\
\tabto{2em} μηδένα φίλον \\
\tabto{2em} ἔχειν \\
\tabto{2em} καὶ πολλοὺς ἑταίρους \\
\tabto{2em} μεταλλάττειν. \\

\end{greek}
}

\begin{description}[noitemsep]
\item[γὰρ] čestica najavljuje iznošenje razloga ili objašnjenja: naime\dots, §~517
\item[αἰσχρὸν] sc.\ ἐστι; imenski predikat, Smyth 910 (kopula je ovdje neizrečena); bezlični izraz otvara mjesto infinitivu (kao subjektu) §~492
\item[ἔχειν] §~232
\item[μεταλλάττειν] §~232

\end{description}

%4

{\large
\begin{greek}
\noindent Μήτε \\
\tabto{2em} μετὰ βλάβης \\
πειρῶ \\
\tabto{2em} τῶν φίλων \\
μήτ' \\
\tabto{2em} ἄπειρος εἶναι \\
\tabto{4em} τῶν ἑταίρων \\
θέλε. \\

\end{greek}
}

\begin{description}[noitemsep]
\item[Μήτε\dots\ μήτ'\dots] §~513.4; koordinacija rečeničnih članova sastavnim veznicima
\item[μετὰ βλάβης] na vlastitu štetu, §~430
\item[πειρῶ τῶν φίλων] §~243; πειράομαί τινος §~396.d
\item[θέλε] §~231; imperativ otvara mjesto N+I (kad se, kao ovdje, subjekt infinitiva ne razlikuje od subjekta glavne rečenice)
\item[ἄπειρος εἶναι τῶν ἑταίρων] ἄπειρός τινος, gen. objektni uz pridjeve koji znače ``vješt'' ili ``nevješt'' §~394.2; §~315

\end{description}

%5

{\large
\begin{greek}
\noindent Τοῦτο δὲ ποιήσεις, \\
\tabto{2em} ἐὰν \\
μὴ δεόμενος \\
τὸ δεῖσθαι \\
προσποιῇ.\\

\end{greek}
}

\begin{description}[noitemsep]
\item[δὲ] čestica označava nadovezivanje na prethodnu rečenicu
\item[ποιήσεις] §~243
\item[ἐὰν\dots\ προσποιῇ] §~243; ἐὰν = εἰ ἂν; kombinacija otvara mjesto konjunktivu za tvorbu pogodbene rečenice eventualnog oblika, §~476.1
\item[μὴ δεόμενος] negacija μὴ uz particip koji zastupa pogodbenu rečenicu §~509.2.b; δέομαι §~325.15; adverbni particip u hipotetičkom značenju §~503.4
\item[τὸ δεῖσθαι] supstantivirani infinitiv §~497

\end{description}

%6

{\large
\begin{greek}
\noindent Περὶ τῶν ῥητῶν \\
\tabto{2em} ὡς ἀπορρήτων \\
ἀνακοινοῦ· \\
μὴ τυχὼν μὲν γὰρ \\
\tabto{2em} οὐδὲν βλαβήσει, \\
τυχὼν δὲ \\
\tabto{2em} μᾶλλον \\
\tabto{4em} αὐτῶν \\
\tabto{2em} τὸν τρόπον \\
ἐπιστήσει. \\

\end{greek}
}

\begin{description}[noitemsep]
\item[Περὶ τῶν ῥητῶν] §~433; glagol εἴρω (osnova ϝερ, lat.\ verbum), prez. i impf. samo epski i jonski; glagolski pridjevi §~300
\item[ἀνακοινοῦ] §~243
\item[μὴ τυχὼν] negacija μὴ uz particip koji zastupa pogodbenu rečenicu §~509.2.b; LSJ τυγχάνω B.I postići cilj, ostvariti namjeru, uspjeti; adverbni particip u hipotetičkom značenju §~503.4; §~254, §~321.19
\item[μὴ τυχὼν μὲν\dots\ τυχὼν δὲ\dots] §~519.7; koordinacija rečeničnih članova parom čestica
\item[γὰρ] §~517
\item[βλαβήσει] §~258, §~261; futuri βλαβήσομαι i βλάψομαι imaju pasivno značenje; tvorba prez. osnove §~249
\item[ἐπιστήσει]  §~258; §~312.6

\end{description}

%7

{\large
\begin{greek}
\noindent Δοκίμαζε τοὺς φίλους \\
\tabto{2em} ἔκ τε τῆς \\
\tabto{4em} περὶ τὸν βίον \\
\tabto{2em} ἀτυχίας \\
\tabto{2em} καὶ τῆς \\
\tabto{4em} ἐν τοῖς κινδύνοις \\
\tabto{2em} κοινωνίας· \\
τὸ μὲν γὰρ χρυσίον \\
\tabto{2em} ἐν τῷ πυρὶ \\
βασανίζομεν, \\
τοὺς δὲ φίλους \\
\tabto{2em} ἐν ταῖς ἀτυχίαις \\
διαγιγνώσκομεν. \\

\end{greek}
}

\begin{description}[noitemsep]
\item[Δοκίμαζε] §~231
\item[ἔκ τε τῆς\dots] \textbf{ἀτυχίας\dots\ καὶ τῆς\dots\ κοινωνίας} δοκιμάζω ἔκ τινος: ispitivati po čemu
\item[περὶ τὸν βίον\dots] \textbf{ἐν τοῖς κινδύνοις} §~433.C.c; §~426.a; prijedložni izrazi kao atributi §~375.4
\item[βασανίζομεν\dots\ διαγιγνώσκομεν] §~231

\end{description}

%8

{\large
\begin{greek}
\noindent Οὕτως ἄριστα χρήσει \\
\tabto{2em} τοῖς φίλοις, \\
ἐὰν μὴ προσμένῃς \\
τὰς \\
\tabto{2em} παρ' ἐκείνων \\
δεήσεις, \\
ἀλλ' αὐτεπάγγελτος \\
\tabto{2em} αὐτοῖς \\
\tabto{2em} ἐν τοῖς καιροῖς \\
βοηθῇς.\\

\end{greek}
}

\begin{description}[noitemsep]
\item[χρήσει] χράομαί τινι postupati prema kome; futur §~260
\item[ἐὰν μὴ προσμένῃς\dots\ βοηθῇς] ἐὰν = εἰ ἂν; ta kombinacija otvara mjesto konjunktivu za tvorbu pogodbene rečenice eventualnog oblika; §~231
\item[παρ' ἐκείνων] §~434.a; prijedložni izraz kao atribut §~375.4
\item[ἐν τοῖς καιροῖς] §~426.b

\end{description}



%kraj
