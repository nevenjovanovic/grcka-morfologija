%\section*{O autoru}

% Unio ispravke NZ <2021-12-27 pon>


\section*{O tekstu}

U petnaestoj knjizi \textit{Knjižnice} \textgreek[variant=ancient]{(Βιβλιοθήκη ἱστορική),} čiji odlomak čitamo u ovom poglavlju, Diodor Sicilski pripovijeda događaje koji su se zbili između 386.\ i 361.\ pr. Kr. Jedan od njih crtica je iz života sicilskog tiranina Dionizija I.\ (432.–367.\ pr.~Kr.), vladara koji je Sirakuzu učinio najmoćnijim polisom Velike Grčke. U više navrata ratovao je s Kartažanima, koji su držali zapadni dio otoka. Poput drugih tirana grčkog svijeta i Dionizije I.\ rado se družio s učenim ljudima i umjetnicima; štoviše, i sam je pisao pjesme koje su se izvodile na javnim manifestacijama. Prema nekim izvorima, umro je netom pošto je na lenejama osvojio nagradu za tragediju \textgreek[variant=ancient]{Ἕκτορος λύτρα} (\textit{Otkup Hektorova tijela}). Kao tiranin, Dionizije I.\ primjer je okrutnog i hirovitog vladara. Kao književnika, antički su ga kritičari smatrali neopisivo lošim.
 
Odabrani odlomak pokazuje obje strane Dionizijeve ličnosti. Jedan od pjesnika koji se našao u Dionizijevu krugu bio je i vrlo popularan autor ditiramba Filoksen s Kitere (\textgreek[variant=ancient]{Φιλόξενος ὁ Κυθήριος,} oko 435.–380.\ pr.~Kr.). Za razliku od većine književnika, Filoksen je na gozbi gdje su se recitirale Dionizijeve pjesme otvoreno rekao da nisu dobre. Naviknut na laskanje i ne htijući priznati manjak literarnog dara, Dionizije se razljutio i odlučio kazniti Filoksena.

\newpage

\section*{Pročitajte naglas grčki tekst.}

Diod.~Sic.\ Bibliotheca historica 15.6.1–15.6.2

%Naslov prema izdanju
%NČ

\medskip


{\large

\begin{greek}

\noindent κατὰ δὲ τὴν Σικελίαν Διονύσιος ὁ τῶν Συρακοσίων τύραννος ἀπολελυμένος τῶν πρὸς Καρχηδονίους πολέμων πολλὴν εἰρήνην καὶ σχολὴν εἶχεν. διὸ καὶ ποιήματα γράφειν ὑπεστήσατο μετὰ πολλῆς σπουδῆς, καὶ τοὺς ἐν τούτοις δόξαν ἔχοντας μετεπέμπετο καὶ προτιμῶν αὐτοὺς συνδιέτριβε καὶ τῶν ποιημάτων ἐπιστάτας καὶ διορθωτὰς εἶχεν. ὑπὸ δὲ τούτων διὰ τὰς εὐεργεσίας τοῖς πρὸς χάριν λόγοις μετεωριζόμενος ἐκαυχᾶτο πολὺ μᾶλλον ἐπὶ τοῖς ποιήμασιν ἢ τοῖς ἐν πολέμῳ κατωρθωμένοις.

τῶν δὲ συνόντων αὐτῷ ποιητῶν Φιλόξενος ὁ διθυραμβοποιός, μέγιστον ἔχων ἀξίωμα κατὰ τὴν κατασκευὴν τοῦ ἰδίου ποιήματος, κατὰ τὸ συμπόσιον ἀναγνωσθέντων τῶν τοῦ τυράννου ποιημάτων μοχθηρῶν ὄντων ἐπηρωτήθη περὶ τῶν ποιημάτων τίνα κρίσιν ἔχοι. ἀποκριναμένου δ᾽ αὐτοῦ παρρησιωδέστερον, ὁ μὲν τύραννος προσκόψας τοῖς ῥηθεῖσι, καὶ καταμεμψάμενος ὅτι διὰ φθόνον ἐβλασφήμησε, προσέταξε τοῖς ὑπηρέταις παραχρῆμα ἀπάγειν εἰς τὰς λατομίας.

\end{greek}

}


\section*{Analiza i komentar}

%1


{\large
\begin{greek}
\noindent κατὰ δὲ τὴν Σικελίαν \\
Διονύσιος ὁ τῶν Συρακοσίων τύραννος \\
\tabto{2em} ἀπολελυμένος \\
\tabto{4em} τῶν πρὸς Καρχηδονίους πολέμων\\
πολλὴν εἰρήνην καὶ σχολὴν εἶχεν.\\

\end{greek}
}

\begin{description}[noitemsep]
\item[ἀπολελυμένος] rekcija: τινος; §~272, §~238, §~498
\item[εἶχεν] §~231, §~236
\item[εἰρήνην καὶ σχολὴν εἶχεν] „uživao je u miru i dokolici“

\end{description}

%2

{\large
\begin{greek}
\noindent διὸ καὶ ποιήματα γράφειν \\
\tabto{2em} ὑπεστήσατο \\
\tabto{4em} μετὰ πολλῆς σπουδῆς,

καὶ τοὺς ἐν τούτοις δόξαν ἔχοντας \\
\tabto{2em} μετεπέμπετο

καὶ προτιμῶν αὐτοὺς \\
\tabto{2em} συνδιέτριβε

καὶ τῶν ποιημάτων ἐπιστάτας καὶ διορθωτὰς \\
\tabto{2em} εἶχεν.\\

\end{greek}
}

\begin{description}[noitemsep]
\item[γράφειν] §~231
\item[ὑπεστήσατο] §~267, §~238, §~311, §~305; u mediopasivnom značenju „poduzeti, započeti” otvara mjesto dopuni u infinitivu: \textgreek{ὑπεστήσατο γράφειν ποιήματα}
\item[τοὺς ἔχοντας] §~231, §~498, §~499.2
\item[μετεπέμπετο] rekcija τινα; §~231, §~238
\item[προτιμῶν] §~243, §~498; προτιμῶν αὐτοὺς
\item[συνδιέτριβε] §~231, §~238
\item[εἶχεν] §~231, §~236

\end{description}

%3


{\large
\begin{greek}
\noindent ὑπὸ δὲ τούτων \\
\tabto{2em} διὰ τὰς εὐεργεσίας \\
\tabto{2em} τοῖς πρὸς χάριν λόγοις μετεωριζόμενος \\
ἐκαυχᾶτο\\
πολὺ μᾶλλον ἐπὶ τοῖς ποιήμασιν \\
ἢ τοῖς ἐν πολέμῳ κατωρθωμένοις.\\

\end{greek}
}

\begin{description}[noitemsep]
\item[μετεωριζόμενος] §~231
\item[ἐκαυχᾶτο] §~243, §~234
\item[τοῖς κατωρθωμένοις] §~243, §~235

\end{description}


%4


{\large
\begin{greek}
\noindent τῶν δὲ συνόντων αὐτῷ ποιητῶν \\
Φιλόξενος ὁ διθυραμβοποιός, \\
\tabto{2em} μέγιστον ἔχων ἀξίωμα \\
\tabto{4em} κατὰ τὴν κατασκευὴν \\
\tabto{6em} τοῦ ἰδίου ποιήματος, \\
κατὰ τὸ συμπόσιον \\
ἀναγνωσθέντων τῶν τοῦ τυράννου ποιημάτων \\
\tabto{2em} μοχθηρῶν ὄντων \\
ἐπηρωτήθη \\
\tabto{2em} περὶ τῶν ποιημάτων \\
τίνα κρίσιν ἔχοι. \\


\end{greek}
}

\begin{description}[noitemsep]
\item[τῶν συνόντων] rekcija τινι; §~315
\item[ἔχων] §~231
\item[ἀναγνωσθέντων] §~296, §~238, §~324.12, §~498
\item[ἀναγνωσθέντων τῶν τοῦ τυράννου ποιημάτων] genitiv apsolutni, §~504
\item[ὄντων] §~315
\item[μοχθηρῶν ὄντων] genitiv apsolutni, kopula s pridjevskom dopunom; proširenje prethodnog genitiva apsolutnog §~504: (τῶν τοῦ τυράννου ποιημάτων) μοχθηρῶν ὄντων
\item[ἐπηρωτήθη] §~296, §~243, §~238, §~545
\item[ἔχοι] §~231

\end{description}

%5


{\large
\begin{greek}
\noindent ἀποκριναμένου δ᾽ αὐτοῦ παρρησιωδέστερον, \\
ὁ μὲν τύραννος \\
\tabto{2em} προσκόψας τοῖς ῥηθεῖσι, \\
καὶ καταμεμψάμενος \\
\tabto{2em} ὅτι διὰ φθόνον ἐβλασφήμησε, \\
προσέταξε \\
\tabto{2em} τοῖς ὑπηρέταις \\
\tabto{2em} παραχρῆμα \\
\tabto{2em} ἀπάγειν εἰς τὰς λατομίας.\\

\end{greek}
}

\begin{description}[noitemsep]
\item[ἀποκριναμένου δ' αὐτοῦ] genitiv apsolutni, §~504
\item[προσκόψας] §~267, §~238, §~301.B (s.~116)
\item[τοῖς ῥηθεῖσι] §~327.7, §~498, §~499.2
\item[καταμεμψάμενος] §~267, §~269
\item[ἐβλασφήμησε] §~267, §~234, §~301.B (s.~116)
\item[ὅτι… ἐβλασφήμησε] zavisna izrična rečenica: da…; \textgreek[variant=ancient]{(ὁ μὲν τύραννος) καταμεμψάμενος (αὐτὸν) ὅτι (αὐτὸς) διὰ φθόνον ἐβλασφήμησε}
\item[προσέταξε] §~267, §~269, §~238, §~301.B (s.~116)
\item[ἀπάγειν ] §~231; neizrečen objekt, sc. \textgreek{αὐτόν}

\end{description}



%kraj
