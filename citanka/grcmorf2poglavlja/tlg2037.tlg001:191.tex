\section*{O autoru}

Demokrit (Δημόκριτος, oko 460.–380.\ ili 370.\ pr.~Kr.) bio je grčki filozof iz bogate grčke kolonije u Trakiji Abdere (Ἄβδηρα). Uza svog učitelja Leukipa glavni je predstavnik grčkog materijalističkog atomizma. Teorijom atoma – nedjeljivih, sitnih, nevidljivih čestica od kojih je sve građeno – obuhvatio je sav živi i neživi kozmos.

Poput mnogih drugih Grka, Demokrit je putovao egejskim, mediteranskim i maloazijskim krajevima. Osim Atene, dugo je boravio u Egiptu, a bio je i u Perziji.

Demokritova djela nisu sačuvana i njegova nam je filozofija poznata po ulomcima sačuvanima kod drugih autora. Znamo da je pisao o nastanku svijeta (kozmogonija), astronomiji (nebeska tijela, Sunce, Mjesec, oblik i položaj Zemlje), meteorologiji (atmosferski fenomeni), kronologiji (kalendar, efemeride) i različitim prirodnim pojavama (potresi, poplava Nila, privlačna sila magneta), ali i o pitanjima života (nastanak života i smrt) i fizioloških procesa organizama (razmnožavanje, embriologija). Bavio se i jezikom i homerskom egzegezom.

Demokritovi etički fragmenti, većinom sakupljeni u Stobejevoj antologiji, preneseni su izvan konteksta koji bi ih potvrdio kao autentične (ali i označio kao neautentične; zbog toga ih prihvaćamo takvima kakvi jesu). Jedan dio njih možda je filozofu samo pripisan, kao dio bogate tradicije Demokritove recepcije, u želji da ga se još izrazitije ocrta kao svestranog mislioca.

\section*{O tekstu}

Tekst koji čitamo Demokritov je fragment 191, sačuvan u zbirci koju je u V.~stoljeću sastavio bizantski antologičar Ivan Stobej (Ἰωάννης Στοβαῖος, \textit{Florilegij}, 3, 1, 210). Smatra se dijelom Demokritove etike – vjerojatno je pripadao spisu Περὶ εὐθυμίης – i govori o tome što čovjeka čini istinski ispunjenim i zadovoljnim. Ton ulomka više je didaktičan nego filozofski.

Tekst je pisan jonskim dijalektom, dijalektom predsokratskih filozofa prirode. Tipična su obilježja jonskoga refleks dugoga α koji prelazi u η (εὐθυμίη umjesto εὐθυμία), nestegnuti oblici vokalskih osnova (posebno kod glagola koji završavaju na -έω), varijacija vokalske kontrakcije ε+ο > ευ (ἐνθυμεύω umjesto ἐνθυμέω) te odnosne zamjenice s refleksom –κ umjesto -π  (ὁκόσωι umjesto ὁπόσωι, ὅκως umjesto ὅπως). Usto, pravopis odlomka koristi se znakom \textit{iota adscriptum} (συμμετρίηι, ψυχῆι) umjesto u srednjovjekovlju i novovjekovlju uobičajenijim \textit{iota subscriptum} (συμμετρίῃ, ψυχῇ).

%\newpage

\section*{Pročitajte naglas grčki tekst.}

Democr.\ Fr.\ 191

%Naslov prema izdanju

\medskip


{\large

\begin{greek}

\noindent ἀνθρώποισι γὰρ εὐθυμίη γίνεται μετριότητι τέρψιος καὶ βίου συμμετρίηι· τὰ δ' ἐλλείποντα καὶ ὑπερβάλλοντα μεταπίπτειν τε φιλεῖ καὶ μεγάλας κινήσιας ἐμποιεῖν τῆι ψυχῆι. αἱ δ' ἐκ μεγάλων διαστημάτων κινούμεναι τῶν ψυχέων οὔτε εὐσταθέες εἰσὶν οὔτε εὔθυμοι. ἐπὶ τοῖς δυνατοῖς οὖν δεῖ ἔχειν τὴν γνώμην καὶ τοῖς παρεοῦσιν ἀρκέεσθαι τῶν μὲν ζηλουμένων καὶ θαυμαζομένων ὀλίγην μνήμην ἔχοντα καὶ τῆι διανοίαι μὴ προσεδρεύοντα, τῶν δὲ ταλαιπωρεόντων τοὺς βίους θεωρέειν, ἐννοούμενον ἃ πάσχουσι κάρτα, ὅκως ἂν τὰ παρεόντα σοι καὶ ὑπάρχοντα μεγάλα καὶ ζηλωτὰ φαίνηται, καὶ μηκέτι πλειόνων ἐπιθυμέοντι συμβαίνηι κακοπαθεῖν τῆι ψυχῆι. ὁ γὰρ θαυμάζων τοὺς ἔχοντας καὶ μακαριζομένους ὑπὸ τῶν ἄλλων ἀνθρώπων καὶ τῆι μνήμηι πᾶσαν ὥραν προσεδρεύων ἀεὶ ἐπικαινουργεῖν ἀναγκάζεται καὶ ἐπιβάλλεσθαι δι' ἐπιθυμίην τοῦ τι πρήσσειν ἀνήκεστον ὧν νόμοι κωλύουσιν. διόπερ τὰ μὲν μὴ δίζεσθαι χρεών, ἐπὶ δὲ τοῖς εὐθυμέεσθαι χρεών, παραβάλλοντα τὸν ἑαυτοῦ βίον πρὸς τὸν τῶν φαυλότερον πρησσόντων καὶ μακαρίζειν ἑωυτὸν ἐνθυμεύμενον ἃ πάσχουσιν, ὁκόσωι αὐτέων βέλτιον πρήσσει τε καὶ διάγει. ταύτης γὰρ ἐχόμενος τῆς γνώμης εὐθυμότερόν τε διάξεις καὶ οὐκ ὀλίγας κῆρας ἐν τῶι βίωι διώσεαι, φθόνον καὶ ζῆλον καὶ δυσμενίην. 

\end{greek}

}


\section*{Analiza i komentar}

%1

{\large
\begin{greek}
\noindent ἀνθρώποισι γὰρ \\
εὐθυμίη γίνεται \\
\tabto{2em} μετριότητι τέρψιος \\
\tabto{2em} καὶ βίου συμμετρίηι·\\
τὰ δ' ἐλλείποντα καὶ ὑπερβάλλοντα \\
μεταπίπτειν τε φιλεῖ \\
καὶ μεγάλας κινήσιας ἐμποιεῖν \\
\tabto{2em} τῆι ψυχῆι. \\

\end{greek}
}

\begin{description}[noitemsep]
\item[γὰρ] čestica s eksplanatornim značenjem, §~517
\item[γίνεται] (jonski oblik) §~231; osnove §~325.11
\item[τὰ δ' ἐλλείποντα] čestica δέ sa suprotnim značenjem (prema μετριότητι καὶ συμμετρίηι) §~515.2
\item[τὰ\dots\ ἐλλείποντα] §~231; složenica λείπω; supstantivirani particip §~499.2
\item[ὑπερβάλλοντα] sc.\ τὰ ὑπερβάλλοντα; §~231; složenica βάλλω
\item[μεταπίπτειν] §~231; složenica πίπτω, osnove §~327.17
\item[μεταπίπτειν τε\dots] \textbf{καὶ\dots\ ἐμποιεῖν\dots}\ sastavni veznici ostvaruju korespondentni par
\item[φιλεῖ] §~243; glagol otvara mjesto dopuni u infinitivu, LSJ φιλέω II
\item[ἐμποιεῖν] §~243; složenica ποιέω, osnove s.~116; LSJ ἐμποιέω II.3

\end{description}

%2


{\large
\begin{greek}
\noindent αἱ δ' ἐκ μεγάλων διαστημάτων κινούμεναι \\
\tabto{2em} τῶν ψυχέων \\
οὔτε εὐσταθέες εἰσὶν \\
οὔτε εὔθυμοι. \\

\end{greek}
}

\begin{description}[noitemsep]
\item[αἱ\dots\ κινούμεναι] sc.\ ψυχαί; §~243; supstantivirani particip §~499.2
\item[δ'] čestica δέ ovdje podržava tijek iskaza: a\dots
\item[διαστημάτων] LSJ διάστημα A.3 (= DGE s.~v.\ III.1)
\item[οὔτε εὐσταθέες\dots\ οὔτε εὔθυμοι ] koordinacija parom sastavnih veznika
\item[εἰσὶν] §~315.2; kopula otvara mjesto imenskoj dopuni; imenski predikat Smyth 909
\item[οὔτε εὔθυμοι] sc.\ εἰσίν

\end{description}

%3

{\large
\begin{greek}
\noindent ἐπὶ τοῖς δυνατοῖς οὖν \\
δεῖ \\
\tabto{2em} ἔχειν τὴν γνώμην \\
\tabto{2em} καὶ τοῖς παρεοῦσιν ἀρκέεσθαι \\
\tabto{4em} τῶν μὲν ζηλουμένων καὶ θαυμαζομένων \\
\tabto{2em} ὀλίγην μνήμην \\
ἔχοντα\\
καὶ τῆι διανοίαι μὴ προσεδρεύοντα, \\
\tabto{4em} τῶν δὲ ταλαιπωρεόντων \\
\tabto{2em} τοὺς βίους \\
θεωρέειν, \\
ἐννοούμενον \\
\tabto{2em} ἃ πάσχουσι κάρτα, \\
ὅκως ἂν τὰ παρεόντα σοι καὶ ὑπάρχοντα \\
\tabto{2em} μεγάλα καὶ ζηλωτὰ φαίνηται, \\
\tabto{2em} καὶ μηκέτι \\
\tabto{2em} πλειόνων ἐπιθυμέοντι \\
\tabto{2em} συμβαίνηι \\
\tabto{4em} κακοπαθεῖν \\
\tabto{6em} τῆι ψυχῆι.\\

\end{greek}
}

\begin{description}[noitemsep]
\item[οὖν] čestica sa zaključnim značenjem §~516.2
\item[δεῖ] §~243; bezlični glagol, o kojem ovisi čitav ostatake rečenice, otvara mjesto dopunama u infinitivu, odnosno A+I
\item[ἔχειν] §~231; neizrečeni je subjekt ἄνθρωπον ili τινα (§~491.3), kao i kod ostalih infinitiva
\item[τὴν γνώμην] rekcija ἐπί τινι
\item[τοῖς παρεοῦσιν] §~243; složenica εἰμί; supstantivirani particip §~499.2
\item[ἀρκέεσθαι] §~243; rekcija τινί; LSJ ἀρκέω II.2; ovisan o δεῖ
\item[τῶν μὲν ζηλουμένων\dots] \textbf{\textgreek[variant=ancient]{τῶν δὲ ταλαιπωρεόντων\dots}}\ par čestica koordinira dva rečenična člana
\item[τῶν\dots\ ζηλουμένων καὶ θαυμαζομένων ] §~243; §~231;  supstantivirani particip §~499.2 (osobno: ljudi za kojima\dots); objektni genitivi ovisni o μνήμην
\item[ἔχοντα] §~231; oblik ovisan o δεῖ, dio A+I
\item[προσεδρεύοντα] §~231; složenica ἑδρεύω; oblik ovisan o δεῖ, dio A+I
\item[τῶν ταλαιπωρεόντων] §~243, supstantivirani particip §~499.2
\item[θεωρέειν] §~243; oblik ovisan o δεῖ, dio A+I
\item[ἐννοούμενον] §~243; složenica νοέω; oblik ovisan o δεῖ, dio A+I
\item[ἃ πάσχουσι] odnosna zamjenica uvodi zavisnu odnosnu rečenicu i ujedno služi kao objekt ἐννοούμενον
\item[πάσχουσι] sc.\ οἱ ταλαιπωρέοντες; §~231
\item[ὅκως\dots\ φαίνηται] namjerni veznik uvodi zavisnu namjernu rečenicu, predikat ἄν + konjunktiv §~470; odavde Demosten prelazi u lični iskaz s dativom \textgreek[variant=ancient]{(σοι\dots\ ἐπιθυμέοντι)}
\item[τὰ παρεόντα\dots\ καὶ ὑπάρχοντα] §~231, složenica εἰμί; složenica glagola ἄρχω; supstantivirani participi §~499.2
\item[φαίνηται] §~232; osnove s.~118, mediopasivno značenje §~448.1; otvara mjesto imenskim dopunama
\item[ἐπιθυμέοντι] §~231; rekcija τινός; ovisan o συμβαίνηι
\item[συμβαίνηι] §~231; rekcija τινί, otvara i mjesto dopuni u infinitivu; složenica βαίνω, LSJ συμβαίνω III.b
\item[κακοπαθεῖν] §~243; otvara mjesto instrumentalnom dativu (ili dativu mjesta) τινί
\end{description}

%4

{\large
\begin{greek}
\noindent ὁ γὰρ θαυμάζων \\
\tabto{2em} τοὺς ἔχοντας \\
\tabto{2em} καὶ μακαριζομένους \\
\tabto{4em} ὑπὸ τῶν ἄλλων ἀνθρώπων \\
καὶ τῆι μνήμηι \\
πᾶσαν ὥραν \\
προσεδρεύων \\
\tabto{2em} ἀεὶ ἐπικαινουργεῖν \\
ἀναγκάζεται\\
\tabto{2em} καὶ ἐπιβάλλεσθαι \\
\tabto{4em} δι' ἐπιθυμίην \\
\tabto{6em} τοῦ τι πρήσσειν \\
\tabto{8em} ἀνήκεστον \\
\tabto{8em} ὧν νόμοι κωλύουσιν.\\

\end{greek}
}

\begin{description}[noitemsep]
\item[γὰρ ] čestica γάρ s eksplanatornim značenjem §~517
\item[ὁ\dots\ θαυμάζων] §~231, rekcija τινα; supstantivirani particip §~499.2
\item[τοὺς ἔχοντας ] §~231; supstantivirani particip §~499.2
\item[μακαριζομένους] §~232
\item[ὑπὸ τῶν ἄλλων ἀνθρώπων] uz glagol u pasivu vršitelj radnje iskazuje se prijedložnim izrazom ὑπό τινος
\item[προσεδρεύων] §~231; složenica glagola ἑδρεύω
\item[ἐπικαινουργεῖν] §~243
\item[ἀναγκάζεται] §~232; otvara mjesto dopunama u infinitivu
\item[ἐπιβάλλεσθαι] §~232, složenica βάλλω
\item[δι' ἐπιθυμίην] otvara mjesto genitivu; dopunu ovdje iskazuje supstantivirani infinitiv \textgreek[variant=ancient]{τοῦ τι πρήσσειν}
\item[τοῦ\dots\ πρήσσειν] §~231; osnove s.~116 (atički oblik \textgreek[variant=ancient]{πράττω)}
\item[ὧν\dots\ κωλύουσιν] odnosna zamjenica ὧν (kao genitiv partitivni) uvodi zavisnu odnosnu rečenicu; antecedent je τι
\item[κωλύουσιν] §~231

\end{description}

%5


{\large
\begin{greek}
\noindent διόπερ \\
τὰ μὲν μὴ δίζεσθαι \\
\tabto{2em} χρεών, \\
ἐπὶ δὲ τοῖς εὐθυμέεσθαι \\
\tabto{2em} χρεών, \\
παραβάλλοντα \\
\tabto{2em} τὸν ἑαυτοῦ βίον \\
\tabto{2em} πρὸς τὸν τῶν φαυλότερον πρησσόντων \\
καὶ μακαρίζειν ἑωυτὸν \\
ἐνθυμεύμενον \\
\tabto{2em} ἃ πάσχουσιν, \\
ὁκόσωι \\
\tabto{4em} αὐτέων \\
\tabto{2em} βέλτιον \\
πρήσσει τε καὶ διάγει.\\

\end{greek}
}

\begin{description}[noitemsep]
\item[διόπερ] prilog najavljuje zaključak: zbog toga\dots
\item[τὰ μὲν\dots\ ἐπὶ δὲ τοῖς\dots] koordinacija rečeničnih članova parom čestica
\item[δίζεσθαι] §~232; ovisno o χρεών
\item[χρεών] LSJ χρεών II.2; otvara mjesto dopunama u infinitivu, odnosno A+I
\item[εὐθυμέεσθαι] §~243; rekcija ἐπί τινι
\item[παραβάλλοντα] §~231; rekcija τι πρός τι; složenica βάλλω; LSJ παραβάλλω III.2; ovisno o χρεών
\item[τῶν\dots\ πρησσόντων] §~231, supstantivirani particip §~499.2
\item[μακαρίζειν] §~231; ovisno o χρεών
\item[ἐνθυμεύμενον] §~232; ovisno o χρεών
\item[ἃ πάσχουσιν] odnosna zamjenica uvodi zavisnu odnosnu rečenicu i ujedno služi kao objekt ἐνθυμεύμενον
\item[πάσχουσιν] §~231
\item[ὁκόσωι] odnosni prilog (dativ odnosne zamjenice, s \textit{iota adscriptum}!) uvodi zavisnu odnosnu rečenicu koja ima službu (drugog) objekta μακαρίζειν
\item[αὐτέων] sc.\ τῶν φαυλότερον πρησσόντων
\item[πρήσσει τε καὶ διάγει] koordinacija pojmova parom sastavnih veznika
\item[πρήσσει] §~231
\item[διάγει] §~231; složenica ἄγω
\end{description}

%6


{\large
\begin{greek}
\noindent ταύτης γὰρ \\
ἐχόμενος \\
\tabto{2em} τῆς γνώμης \\
εὐθυμότερόν τε διάξεις\\
καὶ οὐκ ὀλίγας κῆρας \\
\tabto{2em} ἐν τῶι βίωι \\
διώσεαι, \\
φθόνον καὶ ζῆλον καὶ δυσμενίην.\\

\end{greek}
}

\begin{description}[noitemsep]
\item[γὰρ] čestica γάρ ovdje potvrđuje i ističe misao: baš\dots
\item[ἐχόμενος] §~232, rekcija τινός §~396.δ; LSJ ἔχω IV.C
\item[εὐθυμότερόν τε\dots\ καὶ οὐκ ὀλίγας\dots] koordinacija parom sastavnih veznika
\item[διάξεις] §~258; složenica ἄγω
\item[διώσεαι] §~267; složenica ὠθέω, LSJ διωθέω II.2
\end{description}



%kraj
