% unesi ispravke NČ, 1. 6. 2020
\section*{O autoru}

Ksenofont Efeški \textgreek[variant=ancient]{(Ξενοφῶν ὁ Ἐφέσιος)} grčki je pisac iz II.~st. po Kr. Poznat je kao autor ljubavnog i pustolovnog romana \textgreek[variant=ancient]{Τὰ κατὰ Ἀνθίαν καὶ Ἁβροκόμην Ἐφεσιακά} \textit{(Efeška priča o Antiji i Habrokomu)} u pet knjiga; tekst koji danas poznajemo možda je sažetak nastao prema duljoj i opsežnijoj verziji. Zaplet je sličan onome u ostalim četvorima sačuvanim antičkim ljubavnim romanima (Haritona, Ahileja Tacija, Longa i Heliodora). U središnju radnje je fiktivni ljubavni par (no ne radi o mitskim likovima), mladić i djevojka, koji kroz mnoge peripetije i pustolovine ostaju vjerni svojoj ljubavi. Roman završava sretno, a takvom su kraju vrlo naglašeno pridonijeli bogovi. Za antički je svijet iznimno je što u ljubavnu vezu ulaze vršnjaci, kao i okolnost da se brak – u priči cilj i ideal – temelji na ljubavi i vjernosti partnera (po grčkim je konvencijama brak, osobito među imućnima, redovno »poslovni sporazum« u kojem je muškarac znatno stariji od žene).

Grčki ljubavni roman utjecao je na renesansnu i baroknu književnost (Shakespeare, Tasso).

\section*{O tekstu}

Glavni su likovi Ksenofontova romana šesnaestogodišnji mladić Habrokom (»bujnokosi«) i četrnaestogodišnja Antija (»cvjetna«), oboje iz uglednih efeških obitelji. Njihovu iznimnu ljepotu ljudi tumače božanskom te nisu sigurni jesu li možda i sami ti mladi ljudi bogovi. No, vanjsku ljepotu Habrokoma i Antije prati emocionalna nezrelost, pokazuju se arogantnima i uobraženima; Habrokom posebno prezire Erota. Zbog toga mladi bivaju prvo kažnjeni ljubavlju (zaljube se tijekom Artemidine svetkovine), a zatim se nađu u nizu opasnih situacija. Iz pustolovina u koje su zapali ipak izlaze sretno, vjerni svojoj ljubavi.

Pošto su prethodno bili predstavljeni glavni junak, Habrokom, i njegova iznimna tjelesna i duševna ljepota, u odlomku koji čitamo u tijeku je gradska svetkovina u čast Artemide: procesija domaćih djevojaka i mladića prelazi sedam stadija od grada do Artemidina svetišta (procesija je ujedno i prilika da se za mlade nađu budući bračni drugovi), pri čemu najljepši imaju mjesto predvodnika, no mladići idu odvojeno od djevojaka. Djevojke predvodi Antija, čija se ljepota ovdje opisuje; Antija je nalik Artemidi i promatrači procesije joj se klanjaju. No, i Habrokomova ljepota jednako je dojmljiva za promatrače. I Antija i Habrokom, čuvši za dojam koji je ono drugo ostavilo, požele se vidjeti i upoznati.

%\newpage

\section*{Pročitajte naglas grčki tekst.}

Xenophon Ephesius, Scr.\ Erot.\ Ephesiaca 1.2.5–1.2.9

%Naslov prema izdanju

\medskip


{\large

\begin{greek}

\noindent Ἦρχε δὲ τῆς τῶν παρθένων τάξεως Ἄνθεια, θυγάτηρ Μεγαμήδους καὶ Εὐίππης, ἐγχωρίων. Ἦν δὲ τὸ κάλλος τῆς Ἀνθείας οἷον θαυμάσαι καὶ πολὺ τὰς ἄλλας ὑπερεβάλλετο παρθένους. Ἔτη μὲν τεσσαρεσκαίδεκα ἐγεγόνει, ἤνθει δὲ αὐτῆς τὸ σῶμα ἐπ̓ εὐμορφίᾳ, καὶ ὁ τοῦ σχήματος κόσμος πολὺς εἰς ὥραν συνεβάλλετο· κόμη ξανθή, ἡ πολλὴ καθειμένη, ὀλίγη πεπλεγμένη, πρὸς τὴν τῶν ἀνέμων φορὰν κινουμένη· ὀφθαλμοὶ γοργοί, φαιδροὶ μὲν ὡς κόρης, φοβεροὶ δὲ ὡς σώφρονος· χιτὼν ἁλουργής, ζωστὸς εἰς γόνυ, μεχρὶ βραχιόνων καθειμένος, νεβρὶς περικειμένη, γωρυτὸς ἀνημμένος, τόξα, ἄκοντες φερόμενοι, κύνες ἑπόμενοι. Πολλάκις αὐτὴν ἐπὶ τοῦ τεμένους ἰδόντες Ἐφέσιοι προσεκύνησαν ὡς Ἄρτεμιν. Καὶ τότ̓ οὖν ὀφθείσης ἀνεβόησε τὸ πλῆθος, καὶ ἦσαν ποικίλαι παρὰ τῶν θεωμένων φωναί, τῶν μὲν ὑπ̓ ἐκπλήξεως τὴν θεὸν εἶναι λεγόντων, τῶν δὲ ἄλλην τινὰ ὑπὸ τῆς θεοῦ περιποιημένην· προσεύχοντο δὲ πάντες καὶ προσεκύνουν καὶ τοὺς γονεῖς αὐτῆς ἐμακάριζον· ἦν δὲ διαβόητος τοῖς θεωμένοις ἅπασιν `Ἄνθεια ἡ καλή'. 

Ὡς δὲ παρῆλθε τὸ τῶν παρθένων πλῆθος, οὐδεὶς ἄλλο τι ἢ Ἄνθειαν ἔλεγεν· ὡς δὲ Ἁβροκόμης μετὰ τῶν ἐφήβων ἐπέστη, τοὐνθένδε, καίτοι καλοῦ ὄντος τοῦ κατὰ τὰς παρθένους θεάματος, πάντες ἰδόντες Ἁβροκόμην ἐκείνων ἐπελάθοντο, ἔτρεψαν δὲ τὰς ὄψεις ἐπ̓ αὐτὸν βοῶντες ὑπὸ τῆς θέας ἐκπεπληγμένοι, `καλὸς Ἁβροκόμης' λέγοντες, καὶ `οἷος οὐδὲ εἷς καλοῦ μίμημα θεοῦ'.

Ἤδη δέ τινες καὶ τοῦτο προσέθεσαν `οἷος ἂν γάμος γένοιτο Ἁβροκόμου καὶ Ἀνθείας'.

Καὶ ταῦτ̓ ἦν πρῶτα τῆς Ἔρωτος τέχνης μελετήματα.

Ταχὺ μὲν δὴ εἰς ἑκατέρους ἡ περὶ ἀλλήλων ἦλθε δόξα· καὶ ἥ τε Ἄνθεια τὸν Ἁβροκόμην ἐπεθύμει ἰδεῖν, καὶ ὁ τέως ἀνέραστος Ἁβροκόμης ἤθελεν Ἄνθειαν ἰδεῖν.

\end{greek}

}


\section*{Analiza i komentar}

%1

{\large
\begin{greek}
\noindent Ἦρχε δὲ \\
\tabto{2em} τῆς τῶν παρθένων τάξεως\\
Ἄνθεια,\\
\tabto{2em} θυγάτηρ Μεγαμήδους καὶ Εὐίππης, \\
\tabto{4em} ἐγχωρίων. \\

\end{greek}
}

\begin{description}[noitemsep]
\item[Ἦρχε] rekcija τινος; §~243, augment §~235, osnove s.~116
\item[δὲ] čestica izvan koordinacije s μέν ima dva osnovna značenja, adverzativno i kopulativno; ovdje je riječ o kopulativnom δέ koje označava prijelaz iz rečenice u rečenicu; sljedeća rečenica iskazuje nešto novo ili drugačije, ali ne suprotno prethodnom iskazu, Smyth §~2386

\end{description}

%2
{\large
\begin{greek}
\noindent Ἦν δὲ \\
\tabto{2em} τὸ κάλλος τῆς Ἀνθείας \\
\tabto{2em} οἷον θαυμάσαι \\
\tabto{2em} καὶ πολὺ \\
\tabto{4em} τὰς ἄλλας \\
\tabto{2em} ὑπερεβάλλετο \\
\tabto{4em} παρθένους.\\

\end{greek}
}

\begin{description}[noitemsep]
\item[῏Ην] §~315, kopulativni glagol otvara mjesto nužnoj dopuni (ovdje οἷον θαυμάσαι)
\item[οἷον θαυμάσαι] οἷόν s infinitivom izriče posljedičnu rečenicu, usp. §~473 bilj. 4; §~267
\item[ὑπερεβάλλετο] rekcija τινα; §~243; augment §~238; složenica βάλλω, osnove s.~118

\end{description}

%3
{\large
\begin{greek}
\noindent Ἔτη μὲν τεσσαρεσκαίδεκα ἐγεγόνει, \\
ἤνθει δὲ \\
\tabto{2em} αὐτῆς \\
τὸ σῶμα \\
\tabto{2em} ἐπ' εὐμορφίᾳ,\\
καὶ\\
ὁ τοῦ σχήματος κόσμος πολὺς \\
εἰς ὥραν \\
συνεβάλλετο·\\
κόμη ξανθή, \\
\tabto{2em} ἡ πολλὴ καθειμένη, \\
\tabto{2em} ὀλίγη πεπλεγμένη, \\
\tabto{4em} πρὸς τὴν τῶν ἀνέμων φορὰν \\
\tabto{4em} κινουμένη· \\
ὀφθαλμοὶ γοργοί, \\
\tabto{2em} φαιδροὶ μὲν ὡς κόρης, \\
\tabto{2em} φοβεροὶ δὲ ὡς σώφρονος·\\
ἐσθὴς χιτὼν ἁλουργής, \\
\tabto{2em} ζωστὸς εἰς γόνυ, \\
\tabto{2em} μέχρι βραχιόνων καθειμένος,\\
νεβρὶς περικειμένη, \\
γωρυτὸς ἀνημμένος, \\
τόξα, \\
ἄκοντες φερόμενοι, \\
κύνες ἑπόμενοι. \\

\end{greek}
}

\begin{description}[noitemsep]
\item[Ἔτη μὲν\dots\ ἤνθει δὲ\dots] koordinacija surečenica parom čestica iskazuje misli u blagom kontrastu
\item[ἐγεγόνει] §~272, osnove §~327.11; kopulativni glagol ima imensku dopunu \textgreek[variant=ancient]{ἔτη\dots\ τεσσαρεσκαίδεκα}
\item[ἤνθει] ἐπί τι u nečemu, na neki način; §~243, augment §~235
\item[ὁ τοῦ σχήματος κόσμος] LSJ κόσμος II. »ukras«
\item[εἰς ὥραν] LSJ ὥρα B.II.2.b.
\item[συνεβάλλετο] §~243, augment §~238, složenica βάλλω, osnove s.~118
\item[κόμη ξανθή\dots\ κινουμένη] glagol je neizrečen, to je kopula imenskog predikata ἐστί ili ἦν
\item[καθειμένη] §~311, složenica ἵημι
\item[πεπλεγμένη] §~291.b
\item[κινουμένη] §~243
\item[ὀφθαλμοὶ γοργοί\dots\ σώφρονος] glagol je neizrečen, kao gore
\item[φαιδροὶ μὲν\dots\ φοβεροὶ δὲ\dots] koordinacija iskaza parom čestica
\item[ὡς κόρης\dots\ ὡς σώφρονος] ὡς adverbno uz posvojne genitive
\item[καθειμένος] §~311, usp. gore
\item[περικειμένη] §~315.4, složeica κεῖμαι
\item[ἀνημμένος] §~291.b, reduplikacija §~275, složenica ἅπτω
\item[ἐσθὴς\dots\ ἀνημμένος] neizrečena kopula, kao gore
\item[τόξα\dots\ ἑπόμενοι] neizrečena kopula
\item[φερόμενοι] §~231, osnove §~327.5
\item[ἑπόμενοι] §~231, osnove §~327.12

\end{description}

%4
{\large
\begin{greek}
\noindent Πολλάκις \\
αὐτὴν \\
\tabto{2em} ἐπὶ τοῦ τεμένους \\
ἰδόντες Ἐφέσιοι \\
προσεκύνησαν \\
\tabto{2em} ὡς Ἄρτεμιν.\\

\end{greek}
}

\begin{description}[noitemsep]
\item[ἰδόντες] §~254, osnove §~327.3
\item[προσεκύνησαν] προσκυνέω τινά; §~267, augment §~238
\item[ὡς Ἄρτεμιν] ὡς ovdje adverbno (padež apozicije određuje rekcija glagola)

\end{description}

%5
{\large
\begin{greek}
\noindent Καὶ τότ' οὖν ὀφθείσης \\
ἀνεβόησε τὸ πλῆθος, \\
καὶ ἦσαν ποικίλαι \\
\tabto{2em} παρὰ τῶν θεωμένων \\
φωναί, \\
τῶν μὲν\\
\tabto{2em} ὑπ' ἐκπλήξεως \\
\tabto{4em} τὴν θεὸν εἶναι \\
\tabto{2em} λεγόντων, \\
τῶν δὲ \\
\tabto{4em} ἄλλην τινὰ \\
\tabto{6em} ὑπὸ τῆς θεοῦ \\
\tabto{4em} περιποιημένην· \\
προσηύχοντο δὲ πάντες \\
καὶ προσεκύνουν\\
καὶ τοὺς γονεῖς αὐτῆς ἐμακάριζον·\\
ἦν δὲ διαβόητος \\
\tabto{2em} τοῖς θεωμένοις ἅπασιν \\
\tabto{2em} `Ἄνθεια ἡ καλή'.\\

\end{greek}
}

\begin{description}[noitemsep]
\item[ὀφθείσης] §~296, augment §~237.2, osnove §~327.3
\item[ἀνεβόησε] §~267, augment §238
\item[ἦσαν] §~315, kopula kao dio imenskog predikata otvara mjesto imenskoj dopuni
\item[τῶν θεωμένων] §~243, supstantivirani particip §~499.2
\item[τῶν μὲν\dots\ τῶν δὲ] sc.\ λεγόντων; koordinacija dijelova rečenice parom čestica
\item[εἶναι] §~315, kopula ima imensku dopunu u akuzativu, subjekt toga imenskog predikata neizrečen je \textgreek[variant=ancient]{(Ἄνθειαν)}
\item[τῶν μὲν λεγόντων] §~231, supstantivirani particip §~499.4; \textit{verbum dicendi} otvara mjesto A+I
\item[ἄλλην τινὰ] A+I s neizrečenom kopulom εἶναι
\item[περιποιημένην] atribut uz \textgreek[variant=ancient]{ἄλλην τινὰ;} §~243; LSJ περιποιέω A: pod zaštitom, pod vlašću\dots
\item[προσηύχοντο] §~231, augment §~235; složenica \textgreek[variant=ancient]{εὔχομαι}
\item[προσεκύνουν] §~243, augment §~238
\item[ἐμακάριζον] §~231
\item[ἦν\dots\ διαβόητος τοῖς θεωμένοις] dativ radnog lica \textit{(dativus auctoris)} uz glagolske pridjeve Smyth 1488; imenski predikat Smyth 909 (neizrečeni je subjekt \textgreek[variant=ancient]{Ἄνθεια)}
\item[τοῖς θεωμένοις] §~243, supstantivirani particip §~499.4

\end{description}

%6
{\large
\begin{greek}
\noindent Ὡς δὲ παρῆλθε \\
τὸ τῶν παρθένων πλῆθος, \\
οὐδεὶς \\
\tabto{2em} ἄλλο τι \\
\tabto{2em} ἢ Ἄνθειαν \\
ἔλεγεν· \\
ὡς δὲ Ἁβροκόμης \\
\tabto{2em} μετὰ τῶν ἐφήβων \\
ἐπέστη, \\
τοὐνθένδε, \\
καίτοι καλοῦ ὄντος \\
\tabto{2em} τοῦ κατὰ τὰς παρθένους θεάματος, \\
πάντες ἰδόντες Ἁβροκόμην \\
ἐκείνων ἐπελάθοντο, \\
ἔτρεψαν δὲ τὰς ὄψεις \\
\tabto{2em} ἐπ' αὐτὸν \\
βοῶντες \\
\tabto{2em} ὑπὸ τῆς θέας ἐκπεπληγμένοι, \\
\tabto{2em} `καλὸς Ἁβροκόμης'\\
λέγοντες, \\
καὶ `οἷος οὐδὲ εἷς \\
\tabto{4em} καλοῦ \\
\tabto{2em} μίμημα \\
\tabto{4em} θεοῦ'.\\

\end{greek}
}

\begin{description}[noitemsep]
\item[Ὡς] veznik ovdje uvodi zavisnu vremensku rečenicu
\item[παρῆλθε] §~254, augment §~238; složenica \textgreek[variant=ancient]{ἔρχομαι}, osnove §~327.2
\item[ἔλεγεν] §~231, osnove §~327.7
\item[Ἄνθειαν] ekvivalent hrvatskoga upravnog govora, akuzativ izriče ono što su promatrači govorili
\item[ἐπέστη] §~306, složenica \textgreek[variant=ancient]{ἵστημι}, osnove §~311
\item[ὄντος] §~315; kopulativni glagol otvara mjesto imenskoj dopuni
\item[καλοῦ ὄντος\dots\ θεάματος] GA odgovara nekoj od hrvatskih zavisnih priložnih rečenica
\item[κατὰ τὰς παρθένους] odgovara hrvatskom genitivu objektnom; mjesto mu otvara glagolska imenica \textgreek[variant=ancient]{θέαμα,} izvedena od \textgreek[variant=ancient]{θεάομαι}
\item[ἰδόντες] §~254, osnove §~327.3
\item[ἐπελάθοντο] med τινός na nešto; §~254, augment §~238, složenica λανθάνω, osnove §~321.15
\item[ἔτρεψαν] §~267, osnove s.~118
\item[βοῶντες] §~243
\item[ἐκπεπληγμένοι] §~291.b; složenica \textgreek[variant=ancient]{πλήσσω} (atički \textgreek[variant=ancient]{πλήττω)}
\item[λέγοντες] §~231, osnove 327.7
\item[καλὸς] neizrečena kopula imenskog predikata
\item[οἷος οὐδὲ εἷς] kao nitko drugi\dots
\item[μίμημα] neizrečena kopula imenskog predikata (neizrečen je i subjekt, \textgreek[variant=ancient]{Ἁβροκόμης)}

\end{description}

%7
{\large
\begin{greek}
\noindent ῎Ηδη δέ \\
τινες \\
καὶ τοῦτο \\
προσέθεσαν \\
`οἷος ἂν γάμος γένοιτο \\
\tabto{2em} Ἁβροκόμου καὶ Ἀνθείας'. \\

\end{greek}
}

\begin{description}[noitemsep]
\item[προσέθεσαν] složenica \textgreek[variant=ancient]{τίθημι,} §~306, osnove §~311
\item[οἷος ἂν\dots\ γένοιτο] §~254; kopulativni glagol otvara mjesto imenskoj dopuni; osnove §~325.11; ἂν + optativ izriče mogućnost u sadašnjosti (potencijal sadašnji)
\end{description}

%8
{\large
\begin{greek}
\noindent Καὶ ταῦτα ἦν \\
πρῶτα \\
\tabto{2em} τῆς \\
\tabto{4em} Ἔρωτος \\
\tabto{2em} τέχνης \\
μελετήματα. \\

\end{greek}
}

\begin{description}[noitemsep]
\item[ἦν] §~315, imenski predikat Smyth 909; kopula otvara mjesto imenskoj dopuni (ovdje u rednom broju i imenici); kongruencija sa subjektom u pluralu srednjeg roda §~361

\end{description}

%9
{\large
\begin{greek}
\noindent Ταχὺ \\
μὲν δὴ \\
\tabto{2em} εἰς ἑκατέρους \\
ἡ περὶ ἀλλήλων \\
\tabto{2em} ἦλθε \\
δόξα· \\
καὶ \\
ἥ τε Ἄνθεια \\
\tabto{2em} τὸν Ἁβροκόμην \\
\tabto{2em} ἐπεθύμει \\
\tabto{4em} ἰδεῖν, \\
καὶ ὁ τέως ἀνέραστος Ἁβροκόμης \\
\tabto{2em} ἤθελεν \\
\tabto{2em} Ἄνθειαν \\
\tabto{4em} ἰδεῖν. \\

\end{greek}
}

\begin{description}[noitemsep]
\item[μὲν δὴ] ova kombinacija čestica ističe afirmativno i progresivno značenje: onda\dots 
\item[ἦλθε] §~254, osnove §~327.2
\item[ἐπεθύμει] §~243, augment §~238; glagol nepotpuna značenja otvara mjesto dopuni u infinitivu
\item[ἰδεῖν] §~254, osnove §~327.3
\item[ἤθελεν] §~231, augment §~235, osnove §~325.4; glagol nepotpuna značenja otvara mjesto dopuni u infinitivu
\item[ἰδεῖν] kao gore
\end{description}




%kraj
