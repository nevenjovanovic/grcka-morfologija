% Unesi ispravke NZ <2022-01-03 pon>



\section*{O tekstu}

Dijalog \textit{Ikaromenip} djelo je Lukijanove bogate mašte koje se koristi dvama omiljenim motivima: napada preuzetnost profesionalnih filozofa i opisuje istraživačko putovanje od Zemlje do drugih svjetova (kao npr. u djelima \textit{Haron, Menip, Istinita pripovijest).} U ovom dijalogu Menip, kinički filozof iz III.~st.\ pr.~Kr., odlazi put neba jer je to jedini način da dozna istinu, budući da ono što nude filozofi na zemlji nema ni glavu ni rep. Sve što čuje oko sebe su proturječne teorije. Riječima njegova sugovornika: ``Baš je čudno to što govoriš, da se ljudi, premda su mudri, međusobno spore oko svojih nauka i ne misle isto o istome.'' 

Naslov djela aluzija je na priču o arhetipskom obrtniku Dedalu i njegovu sinu Ikaru. Dedal je obojici konstruirao krila da mogu odletjeti s Krete, međutim, Ikar je letio preblizu suncu, koje je otopilo vosak na njegovim krilima, pa je pao u more i utopio se. Za Lukijana bi moglo biti važno da se Menip sigurno vratio na zemlju bez svojih krila.


%\newpage

\section*{Pročitajte naglas grčki tekst.}

Luc.\ Icaromenippus 6

%Naslov prema izdanju

\medskip


{\large

\begin{greek}

\noindent ΜΕΝΙΠΠΟΣ. Καὶ μήν, ὦ ἑταῖρε, γελάσῃ ἀκούσας τήν τε ἀλαζονείαν αὐτῶν καὶ τὴν ἐν τοῖς λόγοις τερατουργίαν, οἵ γε πρῶτα μὲν ἐπὶ γῆς βεβηκότες καὶ μηδὲν τῶν χαμαὶ ἐρχομένων ἡμῶν ὑπερέχοντες, ἀλλ οὐδὲ ὀξύτερον τοῦ πλησίον δεδορκότες, ἔνιοι δὲ καὶ ὑπὸ γήρως ἢ ἀργίας ἀμβλυώττοντες, ὅμως οὐρανοῦ τε πέρατα διορᾶν ἔφασκον καὶ τὸν ἥλιον περιεμέτρουν καὶ τοῖς ὑπὲρ τὴν σελήνην ἐπεβάτευον καὶ ὥσπερ ἐκ τῶν ἀστέρων καταπεσόντες μεγέθη τε αὐτῶν διεξῄεσαν, καὶ πολλάκις, εἰ τύχοι, μηδὲ ὁπόσοι στάδιοι Μεγαρόθεν Ἀθήναζέ εἰσιν ἀκριβῶς ἐπιστάμενοι τὸ μεταξὺ τῆς σελήνης καὶ τοῦ ἡλίου χωρίον ὁπόσων εἴη πηχῶν τὸ μέγεθος ἐτόλμων λέγειν, ἀέρος τε ὕψη καὶ θαλάττης βάθη καὶ γῆς περιόδους ἀναμετροῦντες, ἔτι δὲ κύκλους καταγράφοντες καὶ τρίγωνα ἐπὶ τετραγώνοις διασχηματίζοντες καὶ σφαίρας τινὰς ποικίλας τὸν οὐρανὸν δῆθεν αὐτὸν ἐπιμετροῦντες.

\end{greek}

}


\section*{Analiza i komentar}

%1

{\large
\begin{greek}
\noindent ΜΕΝΙΠΠΟΣ. Καὶ μήν, ὦ ἑταῖρε, γελάσῃ \\
ἀκούσας \\
\tabto{2em} τήν τε ἀλαζονείαν \\
\tabto{4em} αὐτῶν \\
\tabto{2em} καὶ τὴν \\
\tabto{4em} ἐν τοῖς λόγοις \\
\tabto{2em} τερατουργίαν, \\
οἵ γε πρῶτα μὲν \\
\tabto{2em} ἐπὶ γῆς \\
βεβηκότες \\
καὶ μηδὲν \\
\tabto{2em} τῶν χαμαὶ ἐρχομένων ἡμῶν \\
ὑπερέχοντες, \\
ἀλλ' οὐδὲ ὀξύτερον τοῦ πλησίον δεδορκότες, \\
ἔνιοι δὲ καὶ \\
\tabto{2em} ὑπὸ γήρως ἢ ἀργίας \\
ἀμβλυώττοντες, \\
ὅμως \\
\tabto{4em} οὐρανοῦ τε \\
\tabto{2em} πέρατα \\
\tabto{2em} διορᾶν \\
ἔφασκον \\
καὶ τὸν ἥλιον \\
περιεμέτρουν \\
καὶ τοῖς ὑπὲρ τὴν σελήνην \\
ἐπεβάτευον \\
καὶ \\
\tabto{2em} ὥσπερ ἐκ τῶν ἀστέρων \\
καταπεσόντες \\
μεγέθη τε \\
\tabto{2em} αὐτῶν \\
διεξῄεσαν,\\

\end{greek}
}

\begin{description}[noitemsep]
\item[Καὶ μήν] §~515.4, kao latinski \textit{et vero,} ta kombinacija u dijalogu izražava slaganje i suglasnost: da\dots, doista\dots
\item[γελάσῃ ] §~259, §~260; med. futur s akt. značenjem §~328
\item[ἀκούσας ] §~267
\item[τήν τε ἀλαζονείαν\dots] \textbf{καὶ τὴν\dots\ τερατουργίαν\dots} §~513.2
\item[αὐτῶν ] sc.\ τῶν φιλοσόφων
\item[ἐν τοῖς λόγοις] prijedložni izrazi kao atributi §~375.4
\item[γε] §~519.1
\item[ἐπὶ] §~436.A
\item[οἵ\dots\ βεβηκότες ] supstantivirani particip §~373; §~321.6; §~272
\item[τῶν\dots\ ἐρχομένων] supstantivirani particip §~373; §~232
\item[οἵ\dots\ ὑπερέχοντες] supstantivirani particip §~373; §~232; ὑπερέχω τινός nadmoćan biti
\item[τοῦ πλησίον] supstantivirani prilog ``bližnji''; \textit{genetivus comparationis} uz \textgreek[variant=ancient]{ὀξύτερον} §~404
\item[δεδορκότες] §~272
\item[οἵ γε πρῶτα μὲν\dots\ ἔνιοι δὲ\dots] §~519.7
\item[ὑπὸ γήρως ἢ ἀργίας] ὑπὸ + g. označava uzrok §~437.A.b.β
\item[ἀμβλυώττοντες] §~232
\item[οὐρανοῦ τε πέρατα\dots] \textbf{καὶ τὸν ἥλιον\dots} §~513.2
\item[διορᾶν] §~243
\item[ἔφασκον] §~232; \textit{verbum dicendi} otvara mjesto konstrukciji N+I 491.2
\item[περιεμέτρουν] §~243; prijedlozi περί i πρό ne dopuštaju eliziju §~68.c; augment kod glagola složenih s prijedlogom §~238 
\item[τοῖς ὑπὲρ τὴν σελήνην] \textgreek{τὰ ὑπὲρ τὴν σελήνην} mjesta iza mjeseca; supstantivirani prijedložni izraz §~373
\item[ἐπεβάτευον] §~232; augment kod glagola složenih s prijedlogom §~238
\item[ἐκ τῶν ἀστέρων] §~424.a
\item[καταπεσόντες] složenica glagola πίπτω §~327.17; jaka aorisna osnova §~254; adverbni particip §~503
\item[αὐτῶν] sc.\ τῶν ἀστέρων
\item[διεξῄεσαν] διέξειμι detaljno opisivati, složenica εἶμι §~314 (sporedni oblik 3. l. pl. ᾔεσαν bilj. 3); augment kod glagola složenih s prijedlogom §~238

\end{description}

%2


{\large
\begin{greek}
\noindent καὶ πολλάκις, \\
εἰ τύχοι, \\
μηδὲ \\
\tabto{2em} ὁπόσοι στάδιοι \\
\tabto{4em} Μεγαρόθεν \\
\tabto{4em} ᾿Αθήναζέ \\
\tabto{2em} εἰσιν \\
ἀκριβῶς ἐπιστάμενοι \\
τὸ μεταξὺ \\
\tabto{2em} τῆς σελήνης καὶ τοῦ ἡλίου \\
χωρίον \\
\tabto{2em} ὁπόσων εἴη πηχῶν \\
\tabto{2em} τὸ μέγεθος \\
ἐτόλμων \\
\tabto{2em} λέγειν, \\
ἀέρος τε ὕψη \\
καὶ θαλάττης βάθη \\
καὶ γῆς περιόδους \\
\tabto{2em} ἀναμετροῦντες, \\
ἔτι δὲ κύκλους \\
\tabto{2em} καταγράφοντες \\
καὶ τρίγωνα \\
\tabto{2em} ἐπὶ τετραγώνοις \\
\tabto{2em} διασχηματίζοντες \\
καὶ σφαίρας τινὰς ποικίλας \\
\tabto{2em} τὸν οὐρανὸν δῆθεν αὐτὸν \\
\tabto{2em} ἐπιμετροῦντες.\\

\end{greek}
}

\begin{description}[noitemsep]
\item[εἰ τύχοι] slučajno, možda; τυγχάνω §~321,19; jaka aorisna osnova §~254
\item[Μεγαρόθεν Ἀθήναζέ] nastavci nalik na padežne §~188
\item[εἰσιν] enklitika §~39.3; §~315.2
\item[μηδὲ\dots\ ἐπιστάμενοι] §~513.3; §~312.6; §~328.2
\item[μεταξὺ τῆς σελήνης καὶ τοῦ ἡλίου] nepravi prijedlozi s genitivom §~417; prijedložni izrazi kao atributi §~375.4 
\item[ὁπόσων εἴη πηχῶν] §~315.2; kosi optativ iza historijskog vremena u upitnim rečenicama §~469
\item[τὸ μέγεθος] akuzativ obzira §~389
\item[ἐτόλμων λέγειν] §~243; §~232
\item[ἀναμετροῦντες] §~243
\item[ἔτι δὲ] štoviše, osim toga
\item[καταγράφοντες] §~232
\item[διασχηματίζοντες] διασχηματίζω oblikovati, konstruirati; §~232
\item[ἐπὶ τετραγώνοις] §~436.B
\item[ποικίλας] zamršene, komplicirane
\item[τὸν οὐρανὸν\dots\ αὐτὸν] predikatni položaj §~378.5
\item[δῆθεν] ironično, kao latinski \emph{scilicet,} hrv.\ tobože, dakako, kako kažu
\item[ἐπιμετροῦντες] §~243

\end{description}

%kraj
