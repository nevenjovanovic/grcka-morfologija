% Unio ispravke NZ <2021-12-29 sri>
\section*{O autoru}

Povjesničar Apijan iz Aleksandrije (Ἀππιανὸς Ἀλεξανδρεύς, oko 90.\ – oko 160.) u rodnom je gradu počeo karijeru u carskoj upravi, prešao oko 120.\ u Rim, gdje je prijatelj utjecajnog govornika i političara Marka Kornelija Frontona, čijim je zauzimanjem kod Antonina Pija dobio uglednu dužnost carskog prokuratora.

Ῥωμαϊκά (\textit{Rimska povijest}), djelo nastalo za principata Antonina Pija (138.–161.) obuhvaćalo je 24 knjige, od Enejina dolaska u Italiju do Trajanova principata. Po etnografskom i tematskom kriteriju djelo je organizirano u sedamnaest cjelina koje nose posebne podnaslove (Βασιλική, Ἰταλική, Σαυνιτική, Κελτική itd). Uz predgovor, do nas je u cijelosti stiglo osam knjiga, od kojih knjige 13-17 pripovijedaju o razdoblju građanskih ratova, Ἐμφύλια (od tribunata Tiberija Grakha 133.\ do smrti Seksta Pompeja 35.\ p.~n.~e); od ostalih, sačuvani su veći ili manji dijelovi, odnosno sažeci nastali u bizantsko doba (uvršteni u Βιβλιοθήκη ili Μυριόβιβλος carigradskog patrijarha Focija, oko 810./820. – 893).

\section*{O tekstu}

Prikaz početka bune robova 73.–71.\ p.~n.~e.\ pod vodstvom gladijatora Spartaka nalazi se u prvoj knjizi Apijanove cjeline o rimskim građanskim ratovima (Ἐμφύλια); u okviru Ῥωμαϊκά ta je knjiga četrnaesta. Apijan činjenice prikazuje neutralno i faktografski. Iako je u svojoj povijesti izraziti romanofil, ovdje jasno daje do znanja da su na početku bune Rimljani potcijenili robove, osobito gladijatore, do te mjere da se u okršaju na obroncima Vezuva 73.\ p.~n.~e.\ i sam rimski vojskovođa\footnote{Prema Apijanu, taj se vojskovođa zvao Varinije Glaber, no drugi izvori pokazuju da Apijan ovdje netočno povezuje imena Publija Varinija i Gaja Klaudija Glabera. Vojskovođa poslan nakon Varinija zvao se, prema Apijanu, Publije Valerije.} našao u opasnosti, te su naposljetku Rimljani na Spartaka morali poslati konzule (ὕπατοι) s dvije legije (τέλη).

%\newpage

\section*{Pročitajte naglas grčki tekst.}

App. Bellum civile 1.14.116-1.14.117

%Naslov prema izdanju

\medskip


{\large

\begin{greek}

\noindent  τοῦ δ' αὐτοῦ χρόνου περὶ τὴν Ἰταλίαν μονομάχων ἐς θέας ἐν Καπύῃ τρεφομένων, Σπάρτακος Θρᾲξ ἀνήρ, ἐστρατευμένος ποτὲ Ῥωμαίοις, ἐκ δὲ αἰχμαλωσίας καὶ πράσεως ἐν τοῖς μονομάχοις ὤν, ἔπεισεν αὐτῶν ἐς ἑβδομήκοντα ἄνδρας μάλιστα κινδυνεῦσαι περὶ ἐλευθερίας μᾶλλον ἢ θέας ἐπιδείξεως καὶ βιασάμενος σὺν αὐτοῖς τοὺς φυλάσσοντας ἐξέδραμε· καί τινων ὁδοιπόρων ξύλοις καὶ ξιφιδίοις ὁπλισάμενος ἐς τὸ Βέσβιον ὄρος ἀνέφυγεν, ἔνθα πολλοὺς ἀποδιδράσκοντας οἰκέτας καί τινας ἐλευθέρους ἐκ τῶν ἀγρῶν ὑποδεχόμενος ἐλῄστευε τὰ ἐγγύς, ὑποστρατήγους ἔχων Οἰνόμαόν τε καὶ Κρίξον μονομάχους. μεριζομένῳ δ' αὐτῷ τὰ κέρδη κατ' ἰσομοιρίαν ταχὺ πλῆθος ἦν ἀνδρῶν· καὶ πρῶτος ἐπ' αὐτὸν ἐκπεμφθεὶς Οὐαρίνιος Γλάβρος, ἐπὶ δ' ἐκείνῳ Πόπλιος Οὐαλέριος, οὐ πολιτικὴν στρατιὰν ἄγοντες, ἀλλ' ὅσους ἐν σπουδῇ καὶ παρόδῳ συνέλεξαν (οὐ γάρ πω Ῥωμαῖοι πόλεμον, ἀλλ' ἐπιδρομήν τινα καὶ λῃστηρίῳ τὸ ἔργον ὅμοιον ἡγοῦντο εἶναἰ), συμβαλόντες ἡττῶντο. Οὐαρινίου δὲ καὶ τὸν ἵππον αὐτὸς Σπάρτακος περιέσπασεν· παρὰ τοσοῦτον ἦλθε κινδύνου Ῥωμαίων ὁ στρατηγὸς αὐτὸς αἰχμάλωτος ὑπὸ μονομάχου γενέσθαι.

μετὰ δὲ τοῦτο Σπαρτάκῳ μὲν ἔτι μᾶλλον πολλοὶ συνέθεον, καὶ ἑπτὰ μυριάδες ἦσαν ἤδη στρατοῦ, καὶ ὅπλα ἐχάλκευε καὶ παρασκευὴν συνέλεγεν, οἱ δ' ἐν ἄστει τοὺς ὑπάτους ἐξέπεμπον μετὰ δύο τελῶν.

\end{greek}

}


\section*{Analiza i komentar}

%1

{\large
\begin{greek}
\noindent  τοῦ δ' αὐτοῦ χρόνου \\
περὶ τὴν Ἰταλίαν \\
μονομάχων \\
\tabto{2em} ἐς θέας \\
\tabto{2em} ἐν Καπύῃ \\
τρεφομένων, \\
Σπάρτακος Θρᾲξ ἀνήρ, \\
\tabto{2em} ἐστρατευμένος ποτὲ Ῥωμαίοις, \\
\tabto{2em} ἐκ δὲ αἰχμαλωσίας καὶ πράσεως \\
\tabto{4em} ἐν τοῖς μονομάχοις \\
\tabto{6em} ὤν, \\
ἔπεισεν \\
\tabto{2em} αὐτῶν ἐς ἑβδομήκοντα ἄνδρας μάλιστα \\
\tabto{2em} κινδυνεῦσαι \\
\tabto{4em} περὶ ἐλευθερίας \\
\tabto{6em} μᾶλλον ἢ \\
\tabto{8em} θέας ἐπιδείξεως \\
καὶ βιασάμενος \\
\tabto{4em} σὺν αὐτοῖς \\
\tabto{2em} τοὺς φυλάσσοντας \\
ἐξέδραμε· \\
καί \\
\tabto{4em} τινων ὁδοιπόρων \\
\tabto{2em} ξύλοις καὶ ξιφιδίοις \\
ὁπλισάμενος \\
\tabto{2em} ἐς τὸ Βέσβιον ὄρος \\
ἀνέφυγεν,

\tabto{2em} ἔνθα \\
\tabto{4em} πολλοὺς ἀποδιδράσκοντας οἰκέτας \\
\tabto{4em} καί τινας ἐλευθέρους \\
\tabto{6em} ἐκ τῶν ἀγρῶν \\
\tabto{2em} ὑποδεχόμενος \\
\tabto{2em} ἐλῄστευε \\
\tabto{4em} τὰ ἐγγύς, \\
\tabto{2em} ὑποστρατήγους ἔχων \\
\tabto{4em} Οἰνόμαόν τε καὶ Κρίξον μονομάχους. \\

\end{greek}
}

\begin{description}[noitemsep]
\item[τοῦ δ' αὐτοῦ χρόνου] čestica δέ označava nadovezivanje na prethodno pripovijedanje (o istodobnom ratu protiv Sertorija u Hispaniji); αὐτός ovdje u atributnom položaju te znači ``isti''; genitivom je iskazano vrijeme (genetivus temporis), odgovara na pitanje ``kada?''
\item[περὶ τὴν Ἰταλίαν] priložna oznaka mjesta, ``na području\dots''; paralelizam je naglašen i time što se u prethodnom poglavlju spominje ὁ περὶ Ἰβηρίαν πόλεμος
\item[μονομάχων\dots\ τρεφομένων] genitiv apsolutni
\item[ἐς θέας] priložna oznaka svrhe; θέα (pazi na naglasak!) odgovara lat.\ \textit{ludus}
\item[Καπύῃ] Καπύη, lat.\ \textit{Capua}
\item[τρεφομένων] §~232
\item[ἐστρατευμένος] §~274.3, §~272
\item[ἐστρατευμένος\dots\ Ῥωμαίοις] objektna dopuna u dativu označava (neprijateljsku) vezu, kao kod μάχομαι (πολλοῖς ὀλίγοι μαχόμενοι, ``malobrojni se bore protiv mnogih'', Thuc.\ H.~4, 36); Smyth 4.42.96.88, 1523b
\item[ἐστρατευμένος ποτὲ\dots\ ἐκ δὲ αἰχμαλωσίας] koordinacija pomoću čestice δέ
\item[ἐκ δὲ αἰχμαλωσίας] vremenska upotreba prijedloga; LSJ ἐκ II, 2. o točkama u vremenu; \textgreek{ἐκ ... αἰχμαλωσίας καὶ πράσεως} ``nakon što je bio zarobljen i prodan''
\item[ἐν τοῖς μονομάχοις ὤν] §~315; kopulativni glagol otvara mjesto obaveznoj dopuni (ovdje priložna oznaka)
\item[ἔπεισεν] §~254, s.~118
\item[αὐτῶν] sc. τῶν μονομάχων; partitivni genitiv
\item[κινδυνεῦσαι] §~267
\item[βιασάμενος] §~267
\item[τοὺς φυλάσσοντας] §~231 (atički je oblik glagola φυλάττω)
\item[ἐξέδραμε] §~254, složenica od τρέχω, §~327.4
\item[ὁπλισάμενος]  §~261, §~269; §~267
\item[ἀνέφυγεν] §~254, složenica od φεύγω, s.~116
\item[ἀποδιδράσκοντας] §~231
\item[ὑποδεχόμενος] §~232
\item[ἐλῄστευε] §~231
\item[τὰ ἐγγύς] supstantivirani prilog, ``blizina'', ``okolica''
\item[ἔχων] §~231
\item[τε καὶ] ova kombinacija veznika povezuje komplementarne elemente
\end{description}

%2

{\large
\begin{greek}
\noindent  μεριζομένῳ δ' αὐτῷ \\
\tabto{2em} τὰ κέρδη \\
\tabto{4em} κατ' ἰσομοιρίαν \\
ταχὺ πλῆθος ἦν ἀνδρῶν· \\
καὶ πρῶτος \\
\tabto{2em} ἐπ' αὐτὸν \\
ἐκπεμφθεὶς \\
\tabto{2em} Οὐαρίνιος Γλάβρος, \\
ἐπὶ δ' ἐκείνῳ \\
\tabto{2em} Πόπλιος Οὐαλέριος, \\
οὐ πολιτικὴν στρατιὰν ἄγοντες, \\
ἀλλ' ὅσους \\
\tabto{2em} ἐν σπουδῇ καὶ παρόδῳ \\
συνέλεξαν \\
(οὐ γάρ πω \\
Ῥωμαῖοι \\
πόλεμον, \\
ἀλλ' ἐπιδρομήν τινα \\
καὶ λῃστηρίῳ τὸ ἔργον ὅμοιον \\
ἡγοῦντο εἶναἰ), \\
\tabto{2em} συμβαλόντες \\
ἡττῶντο. \\

\end{greek}
}

\begin{description}[noitemsep]
\item[μεριζομένῳ] §~232
\item[μεριζομένῳ\dots\ ἦν] posesivni dativ uz kopulativni glagol, §~412.2
\item[δ'] čestica δέ označava nadovezivanje na prethodnu rečenicu
\item[ἦν] §~315
\item[καὶ πρῶτος\dots\ ἐπὶ δ' ἐκείνῳ\dots] koordinacija rečeničnih članova pomoću čestice δέ
\item[ἐκπεμφθεὶς] §~296; složenica glagola πέμπω, s.~116
\item[ἄγοντες] §~231
\item[ὅσους] tj. ἄγοντες τόσους, ὅσους\dots
\item[συνέλεξαν] §~238; §~267; složenica glagola λέγω, §~327.7
\item[οὐ γάρ πω] οὔπω (također οὔ πω), ``još ne'' s umetnutom česticom γάρ, koja najavljuje iznošenje razloga ili objašnjenja, ``naime''
\item[λῃστηρίῳ τὸ ἔργον ὅμοιον] red riječi ističe kvalifikaciju, ὅμοιον otvara mjesto dativu
\item[ὅμοιον\dots\ εἶναἰ] §~315; kopulativni glagol otvara mjesto predikatnoj dopuni, koja je ovdje pridjev
\item[ἡγοῦντο] §~232, §~243, §~235; ἡγέομαι kao \textit{verbum sentiendi} otvara mjesto akuzativu s infinitivom
\item[συμβαλόντες] §~254; složenica glagola βάλλω, s.~118
\item[ἡττῶντο] §~232, §~243, §~235
\end{description}


%3

{\large
\begin{greek}
\noindent  Οὐαρινίου δὲ καὶ τὸν ἵππον \\
αὐτὸς Σπάρτακος περιέσπασεν· \\
\tabto{2em} παρὰ τοσοῦτον \\
ἦλθε \\
\tabto{2em} κινδύνου \\
Ῥωμαίων ὁ στρατηγὸς \\
\tabto{2em} αὐτὸς αἰχμάλωτος \\
\tabto{4em} ὑπὸ μονομάχου \\
\tabto{2em} γενέσθαι.\\

\end{greek}
}

\begin{description}[noitemsep]
\item[δὲ] čestica δέ označava nadovezivanje na prethodnu rečenicu
\item[Οὐαρινίου\dots\ καὶ τὸν ἵππον] obilježen red riječi, kao i καί ``čak i'', ističu kao temu rečenice Varinija, rimskog vojskovođu
\item[περιέσπασεν] §~267, §~238; složenica glagola σπάω, s.~116
\item[παρὰ τοσοῦτον\dots] fraza παρὰ τοσοῦτον ἐλθεῖν κινδύνου znači ``biti u tolikoj opasnosti'', ``toliko je malo nedostajalo da\dots'', ovdje uz nominativ s infinitivom, §~491.2
\item[ἦλθε] §~254; §~327.2
\item[αἰχμάλωτος\dots\ γενέσθαι] §~254, §~255; §~325.11; kopulativni glagol otvara mjesto predikatnoj dopuni, koja je ovdje pridjev
\end{description}


%4

{\large
\begin{greek}
\noindent  μετὰ δὲ τοῦτο \\
Σπαρτάκῳ μὲν \\
\tabto{2em} ἔτι μᾶλλον πολλοὶ συνέθεον, \\
\tabto{2em} καὶ ἑπτὰ μυριάδες ἦσαν ἤδη στρατοῦ, \\
\tabto{2em} καὶ ὅπλα ἐχάλκευε \\
\tabto{2em} καὶ παρασκευὴν συνέλεγεν, \\
οἱ δ' ἐν ἄστει \\
τοὺς ὑπάτους ἐξέπεμπον \\
\tabto{2em} μετὰ δύο τελῶν.\\

\end{greek}
}

\begin{description}[noitemsep]
\item[δὲ] čestica δέ označava nadovezivanje na prethodnu rečenicu
\item[Σπαρτάκῳ μὲν\dots\ οἱ δ' ἐν ἄστει\dots] koordinacija pomoću čestica μέν\dots\ δέ\dots
\item[ἔτι μᾶλλον πολλοὶ] kombinacija priloga ἔτι μᾶλλον modificira pridjev
\item[συνέθεον] §~231, §~238
\item[Σπαρτάκῳ\dots\ ἦσαν] posesivni dativ uz kopulativni glagol, §~412.2
\item[ἦσαν] §~315
\item[ἐχάλκευε] §~231
\item[συνέλεγεν] §~231, §~238
\item[ἐν ἄστει]	ἄστυ je grčki ekvivalent latinskoga \textit{urbs}
\item[τοὺς ὑπάτους] ὕπατος je grčki ekvivalent latinskoga \textit{consul}
\item[ἐξέπεμπον] §~231, §~238
\item[τελῶν] τέλος je grčki ekvivalent latinskoga \textit{legio}
\end{description}




%kraj
