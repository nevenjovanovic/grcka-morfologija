% Unesi ispravke NZ <2022-01-04 uto>


\section*{O tekstu}

Djelo \textit{Razgovori sa samim sobom} ili \textit{Samom sebi} \textgreek[variant=ancient]{(Τὰ εἰς ἑαυτὸν),} dvanaest knjiga filozofskih zapisa (poznato i kao \textit{Meditacije} ili \textit{Razmišljanja}) nastalo je u posljednjem desetljeću života rimskoga cara, između 170.\ i 180.\ po~Kr. Marko Aurelije grčkim se jezikom, koji je usvojio još u djetinjstvu, služi kao jezikom privatnosti i meditiranja nasuprot latinskoga, jezika službe i formalnih dužnosničkih zadaća. Ovaj je njegov ``duhovni dnevnik'', jedno od moralno najuzvišenijih djela antike, i dokument kulturne i duhovne klime II.~st.\ po~Kr. nesigurnosti, krhkosti, usamljenosti i izgubljenosti u svijetu tuđem individualnoj prirodi. Djelo se suprotstavlja glavnoj struji tadašnje književne produkcije, elegantnoj ali ispraznoj: romanima, govorima, retoričkim ekshibicijama.

Sljedeće razmišljanje iz četvrte knjige blisko je Epiktetovoj misli da nesreću treba pretvoriti u prednost; umjesto da se žalimo na svoj udes, razmotrimo koliko smo sretni što (zahvaljujući stoičkom učenju koje smo usvojili) nesreću možemo nadići.
%\newpage

\section*{Pročitajte naglas grčki tekst.}

M.\ Aur.\ Ad se ipsum 4.49

%Naslov prema izdanju

\medskip


{\large

\begin{greek}

\noindent Ὅμοιον εἶναι τῇ ἄκρᾳ, ᾗ διηνεκῶς τὰ κύματα προσρήσσεται· ἡ δὲ ἕστηκε καὶ περὶ αὐτὴν κοιμίζεται τὰ φλεγμήναντα τοῦ ὕδατος. 

\noindent Ἀτυχὴς ἐγώ, ὅτι τοῦτό μοι συνέβη. οὐμενοῦν ἀλλ̓' εὐτυχὴς ἐγώ, ὅτι τούτου μοι συμβεβηκότος ἄλυπος διατελῶ, οὔτε ὑπὸ παρόντος θραυόμενος οὔτε ἐπιὸν φοβούμενος. συμβῆναι μὲν γὰρ τὸ τοιοῦτο παντὶ ἐδύνατο, ἄλυπος δὲ οὐ πᾶς ἐπὶ τούτῳ ἂν διετέλεσε. διὰ τί οὖν ἐκεῖνο μᾶλλον ἀτύχημα ἢ τοῦτο εὐτύχημα; λέγεις δὲ ὅλως ἀτύχημα ἀνθρώπου, ὃ οὐκ ἔστιν ἀπότευγμα τῆς φύσεως τοῦ ἀνθρώπου; ἀπότευγμα δὲ τῆς φύσεως τοῦ ἀνθρώπου εἶναι δοκεῖ σοι, ὃ μὴ παρὰ τὸ βούλημα τῆς φύσεως αὐτοῦ ἐστι; τί οὖν; τὸ βούλημα μεμάθηκας· μήτι οὖν τὸ συμβεβηκὸς τοῦτο κωλύει σε δίκαιον εἶναι, μεγαλόψυχον, σώφρονα, ἔμφρονα, ἀπρόπτωτον, ἀδιάψευστον, αἰδήμονα, ἐλεύθερον, τἄλλα, ὧν συμπαρόντων ἡ φύσις ἡ τοῦ ἀνθρώπου ἀπέχει τὰ ἴδια; μέμνησο λοιπὸν ἐπὶ παντὸς τοῦ εἰς λύπην σε προαγομένου τούτῳ χρῆσθαι τῷ δόγματι· οὐχ ὅτι τοῦτο ἀτύχημα, ἀλλὰ τὸ φέρειν αὐτὸ γενναίως εὐτύχημα.

\end{greek}

}


\section*{Analiza i komentar}

%1

{\large
\begin{greek}
\noindent  Ὅμοιον εἶναι \\
\tabto{2em} τῇ ἄκρᾳ, \\
\tabto{4em} ᾗ \\
\tabto{6em} διηνεκῶς \\
\tabto{4em} τὰ κύματα \\
\tabto{4em} προσρήσσεται· \\
ἡ δὲ ἕστηκε \\
καὶ περὶ αὐτὴν κοιμίζεται \\
\tabto{2em} τὰ φλεγμήναντα \\
\tabto{4em} τοῦ ὕδατος. \\

\end{greek}
}

\begin{description}[noitemsep]
\item[ Ὅμοιον εἶναι] §~315; infinitiv ima vrijednost zapovijedi, Smyth 2013; imenski predikat, Smyth 909; \textgreek[variant=ancient]{ὅμοιός τινι}
\item[προσρήσσεται] §~232; kongruencija sa subjektom u pluralu srednjeg roda §~361
\item[ἡ δὲ] a ona, sc.\ ἄκρα
\item[ἕστηκε] §~311
\item[κοιμίζεται] §~232
\item[τὰ φλεγμήναντα] §~267, §~270; LSJ φλεγμαίνω II.2

\end{description}

%2

{\large
\begin{greek}
\noindent Ἀτυχὴς ἐγώ, \\
\tabto{2em} ὅτι τοῦτό \\
\tabto{2em} μοι \\
\tabto{2em} συνέβη.

\noindent οὐμενοῦν ἀλλ̓' εὐτυχὴς ἐγώ, \\
\tabto{2em} ὅτι \\
\tabto{4em} τούτου μοι συμβεβηκότος \\
\tabto{2em} ἄλυπος διατελῶ, \\
οὔτε \\
\tabto{4em} ὑπὸ παρόντος \\
\tabto{2em} θραυόμενος \\
οὔτε \\
\tabto{2em} ἐπιὸν \\
\tabto{2em} φοβούμενος. \\

\end{greek}
}

\begin{description}[noitemsep]
\item[Ἀτυχὴς] imenski predikat, Smyth 909; kopula je ovdje neizrečena
\item[συνέβη] §~292; složenica βαίνω, §~321.6
\item[οὐμενοῦν] LSJ οὐ μὲν οὖν
\item[εὐτυχὴς] imenski predikat, Smyth 909; kopula je ovdje neizrečena
\item[συμβεβηκότος] §~272; složenica βαίνω, §~321.6; participski dio GA u značenju dopusne rečenice
\item[διατελῶ] §~243; LSJ διατελέω II.1 (s pridjevom)
\item[οὔτε ὑπὸ παρόντος\dots] \textbf{οὔτε ἐπιὸν\dots}\ koordinacija rečeničnih članova pomoću (niječnih) sastavnih veznika
\item[παρόντος] složenica εἰμί, §~315; LSJ πάρειμι II, supstantivirano: sadašnjost
\item[θραυόμενος] §~232
\item[ἐπιὸν] složenica εἶμι §~314.1; LSJ ἔπειμι (B) II, supstantivirano: budućnost (paralela s \textgreek[variant=ancient]{παρόντος)}
\item[φοβούμενος] §~243

\end{description}

%4
{\large
\begin{greek}
\noindent συμβῆναι μὲν γὰρ \\
\tabto{2em} τὸ τοιοῦτο \\
\tabto{2em} παντὶ \\
ἐδύνατο, \\
ἄλυπος δὲ \\
οὐ πᾶς \\
\tabto{2em} ἐπὶ τούτῳ \\
ἂν διετέλεσε.\\

\end{greek}
}

\begin{description}[noitemsep]
\item[συμβῆναι μὲν\dots\ ἄλυπος δὲ\dots] koordinacija rečeničnih članova pomoću para čestica
\item[συμβῆναι] §~292; složenica βαίνω, §~321.6
\item[ἐδύνατο] §~312.5; otvara mjesto dopuni u infinitivu
\item[ἐπὶ τούτῳ] LSJ ἐπί B.I.1.i
\item[ἂν διετέλεσε] §~267, §~269; indikativ preterita s ἄν izražava irealnu (nezbiljnu) radnju, aorist izriče prošlost §~462
\end{description}

%5
{\large
\begin{greek}
\noindent διὰ τί οὖν \\
ἐκεῖνο \\
\tabto{2em} μᾶλλον ἀτύχημα \\
\tabto{2em} ἢ τοῦτο εὐτύχημα;\\

\end{greek}
}

\begin{description}[noitemsep]
\item[ἀτύχημα] imenski predikat, Smyth 909; kopula je ovdje neizrečena%imenski pred
\item[μᾶλλον ἀτύχημα\dots\ ἢ τοῦτο\dots] koordinacija rečeničnih članova: više\dots\ nego\dots
\item[εὐτύχημα] imenski predikat, Smyth 909; kopula je ovdje neizrečena%imenski pred
\end{description}

%6
{\large
\begin{greek}
\noindent λέγεις δὲ \\
\tabto{2em} ὅλως \\
ἀτύχημα \\
\tabto{2em} ἀνθρώπου, \\
ὃ οὐκ ἔστιν \\
ἀπότευγμα \\
\tabto{2em} τῆς φύσεως \\
\tabto{4em} τοῦ ἀνθρώπου; \\

\end{greek}
}

\begin{description}[noitemsep]
\item[λέγεις] \textit{verbum dicendi} otvara mjesto A+I
\item[δὲ] čestica povezuje rečenicu s prethodnom
\item[ἀτύχημα] imenski dio predikata (kopula je neizrečena), imenski predikat, Smyth 909
\item[ὃ] uvodi relativnu rečenicu, antecedent je ἀτύχημα
\item[οὐκ ἔστιν ἀπότευγμα] (zanijekani) imenski predikat, Smyth 909

\end{description}

%7
{\large
\begin{greek}
\noindent ἀπότευγμα δὲ \\
\tabto{2em} τῆς φύσεως \\
\tabto{4em} τοῦ ἀνθρώπου \\
εἶναι \\
δοκεῖ σοι, \\
\tabto{2em} ὃ μὴ \\
\tabto{4em} παρὰ τὸ βούλημα \\
\tabto{6em} τῆς φύσεως \\
\tabto{8em} αὐτοῦ \\
\tabto{4em} ἐστι; \\
τί οὖν;\\

\end{greek}
}

\begin{description}[noitemsep]
\item[ἀπότευγμα\dots\ εἶναι] imenski predikat Smyth 909; A+I ovisan o δοκεῖ
\item[δὲ] čestica povezuje rečenicu s prethodnom
\item[δοκεῖ] §~243; bezlično LSJ δοκέω II.4; kao \textit{verbum sentiendi} otvara mjesto A+I
\item[ὃ] uvodi relativnu subjektnu rečenicu: ono što\dots
\item[παρὰ τὸ βούλημα\dots\ ἐστι] imenski predikat, Smyth 909
\item[τί οὖν;] LSJ τίς I.8.f

\end{description}

%8
{\large
\begin{greek}
\noindent τὸ βούλημα μεμάθηκας· \\
μήτι οὖν \\
τὸ συμβεβηκὸς τοῦτο \\
κωλύει \\
σε \\
\tabto{2em} δίκαιον εἶναι, \\
\tabto{2em} μεγαλόψυχον, \\
\tabto{2em} σώφρονα, \\
\tabto{2em} ἔμφρονα, \\
\tabto{2em} ἀπρόπτωτον, \\
\tabto{2em} ἀδιάψευστον, \\
\tabto{2em} αἰδήμονα, \\
\tabto{2em} ἐλεύθερον, \\
\tabto{2em} τἄλλα, \\
\tabto{4em} ὧν συμπαρόντων \\
\tabto{4em} ἡ φύσις \\
\tabto{6em} ἡ τοῦ ἀνθρώπου \\
\tabto{4em} ἀπέχει \\
\tabto{6em} τὰ ἴδια; \\

\end{greek}
}

\begin{description}[noitemsep]
\item[μεμάθηκας] §~272, §~321.17
\item[μήτι] u izravnom pitanju, očekuje se odgovor ``ne'': ne sprečava te onda\dots? 
\item[τὸ συμβεβηκὸς] §~272; složenica βαίνω, §~321.6; supstantiviranje članom §~373
\item[κωλύει] §~231; otvara mjesto infinitivima
\item[δίκαιον εἶναι] §~315; imenski predikat, Smyth 909
\item[μεγαλόψυχον] sc.\ εἶναι (tako i ostali pridjevi u ovom nizu); imenski predikat, Smyth 909
\item[ὧν συμπαρόντων] GA; složenica εἰμί, §~315; antecedent odnosne zamjenice je τἄλλα (ili svi nabrojeni pridjevi)
\item[ἡ τοῦ ἀνθρώπου] atributni položaj, §~375
\item[ἀπέχει] §~231; složenica ἔχω
\end{description}

%9
{\large
\begin{greek}
\noindent μέμνησο \\
\tabto{2em} λοιπὸν \\
\tabto{2em} ἐπὶ παντὸς τοῦ \\
\tabto{6em} εἰς λύπην \\
\tabto{4em} σε \\
\tabto{4em} προαγομένου \\
\tabto{2em} τούτῳ \\
χρῆσθαι \\
\tabto{2em} τῷ δόγματι·

\noindent οὐχ ὅτι \\
\tabto{2em} τοῦτο \\
\tabto{2em} ἀτύχημα, \\
ἀλλὰ \\
\tabto{2em} τὸ φέρειν αὐτὸ \\
\tabto{4em} γενναίως \\
\tabto{2em} εὐτύχημα.\\

\end{greek}
}

\begin{description}[noitemsep]
\item[μέμνησο] §~274; \textit{verbum sentiendi} otvara mjesto infinitivu
\item[λοιπὸν] LSJ λοιπός 5, priložno
\item[προαγομένου] §~232; složenica ἄγω
\item[χρῆσθαι] §~243, §~232; rekcija τινι
\item[τούτῳ\dots\ τῷ δόγματι] otvara mjesto izričnoj rečenici
\item[οὐχ ὅτι\dots\ ἀλλὰ] koordinacija pomoću suprotnog veznika
\item[ἀτύχημα] imenski dio predikata, sc.\ ἐστι
\item[τὸ φέρειν] supstantivirani infinitiv, §~497
\item[εὐτύχημα] imenski dio predikata, sc.\ ἐστι
\end{description}


%kraj
