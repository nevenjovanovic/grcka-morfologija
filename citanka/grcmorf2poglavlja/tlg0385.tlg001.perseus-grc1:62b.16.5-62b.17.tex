% dopune NJ, 27. 4. 2020.
% Unesi ispravke NZ <2022-01-02 ned>



\section*{O tekstu}

Pišući oko 200.\ po~Kr. u 62.\ knjizi \textit{Rimske povijesti} Dion Kasije detaljno i slikovito opisuje zbrku i zbunjenost među stanovnicima Rima tijekom velikog požara 64.\ po~Kr.

%\newpage

\section*{Pročitajte naglas grčki tekst.}

Dio Cassius Historiae Romanae 62.16.5-62.17
%Naslov prema izdanju

\medskip


{\large

\begin{greek}

\noindent καὶ ἦν ἥ τε κραυγὴ καὶ ὀλολυγὴ παίδων ὁμοῦ γυναικῶν ἀνδρῶν γερόντων ἄπλετος, ὥστε μήτε συνιδεῖν μήτε συνεῖναί τι ὑπὸ τοῦ καπνοῦ καὶ τῆς κραυγῆς δύνασθαι· καὶ διὰ ταῦθ᾽ ὁρᾶν ἦν τινας ἀφώνους ἑστῶτας ὥσπερ ἐνεοὺς ὄντας. κἀν τούτῳ πολλοὶ μὲν καὶ τὰ σφέτερα ἐκκομιζόμενοι, πολλοὶ δὲ καὶ τὰ ἀλλότρια ἁρπάζοντες ἀλλήλοις τε ἐνεπλάζοντο καὶ περὶ τοῖς σκεύεσιν ἐσφάλλοντο, καὶ οὔτε προϊέναι ποι οὔθ᾽ ἑστάναι εἶχον, ἀλλ᾽ ὤθουν ὠθοῦντο, ἀνέτρεπον ἀνετρέποντο. καὶ συχνοὶ μὲν ἀπεπνίγοντο συχνοὶ δὲ συνετρίβοντο, ὥστε σφίσι μηδὲν ὅ τι τῶν δυναμένων ἀνθρώποις ἐν τῷ τοιούτῳ πάθει κακῶν συμβῆναι μὴ συνενεχθῆναι. οὐδὲ γὰρ οὐδ᾽ ἀποφυγεῖν που ῥᾳδίως ἐδύναντο· κἂν ἐκ τοῦ παρόντος τις περιεσώθη, ἐς ἕτερον ἐμπεσὼν ἐφθείρετο.

\noindent καὶ ταῦτα οὐκ ἐν μιᾷ μόνον ἀλλ᾽ ἐπὶ πλείους καὶ ἡμέρας καὶ νύκτας ὁμοίως ἐγίνετο. καὶ πολλοὶ μὲν οἶκοι ἔρημοι τοῦ βοηθήσοντός σφισιν ἀπώλοντο, πολλοὶ δὲ καὶ ὑπ᾽ αὐτῶν τῶν ἐπικουρούντων προσκατεπρήσθησαν: οἱ γὰρ στρατιῶται, οἵ τε ἄλλοι καὶ οἱ νυκτοφύλακες, πρὸς τὰς ἁρπαγὰς ἀφορῶντες οὐχ ὅσον οὐ κατεσβέννυσάν τινα ἀλλὰ καὶ προσεξέκαιον. τοιούτων δὲ δὴ ἄλλων ἄλλοθι συμβαινόντων, ὑπέλαβέ ποτε τὸ πῦρ ἄνεμος καὶ ἐπὶ τὰ λοιπὰ ὁμοῦ πάντα ἤγαγεν, ὥστε σκευῶν μὲν πέρι ἢ οἰκιῶν μηδένα μηδὲν ἔτι φροντίσαι, πάντας δὲ τοὺς λοιποὺς ἑστῶτάς που ἐν ἀσφαλεῖ τινι ὁρᾶν ὥσπερ νήσους τινὰς καὶ πόλεις ἅμα πολλὰς φλεγομένας, καὶ ἐπὶ μὲν τοῖς σφετέροις μηδὲν ἔτι λυπεῖσθαι, τὸ δὲ δημόσιον ὀδυρομένους ἀναμιμνήσκεσθαι ὅτι καὶ πρότερόν ποτε οὕτως ὑπὸ τῶν Γαλατῶν τὸ πλεῖον τῆς πόλεως διεφθάρη.

\end{greek}

}

%\newpage

\section*{Analiza i komentar}

%1

{\large
\begin{greek}
\noindent καὶ ἦν ἥ τε κραυγὴ \\
\tabto{2em} καὶ ὀλολυγὴ \\
\tabto{4em} παίδων ὁμοῦ γυναικῶν ἀνδρῶν γερόντων \\
ἄπλετος, \\
\tabto{2em} ὥστε \\
\tabto{4em} μήτε συνιδεῖν \\
\tabto{4em} μήτε συνεῖναί τι\\
\tabto{6em} ὑπὸ τοῦ καπνοῦ \\
\tabto{6em} καὶ τῆς κραυγῆς \\
\tabto{4em} δύνασθαι· \\
\tabto{4em} καὶ διὰ ταῦθ' \\
\tabto{6em} ὁρᾶν ἦν \\
\tabto{6em} τινας ἀφώνους ἑστῶτας \\
\tabto{8em} ὥσπερ ἐνεοὺς ὄντας. \\

\end{greek}
}

\begin{description}[noitemsep]
\item[ἦν] §~315, ovdje u značenju „nastati” (LSJ εἰμί A.I.2)
\item[παίδων\dots] \textbf{γυναικῶν ἀνδρῶν γερόντων} genitiv subjektni, §~393.2; asindeton
\item[ὥστε] veznik uvodi posljedičnu rečenicu s infinitivom, §~473
\item[μήτε\dots\ μήτε\dots] koordinacija sastavnim (niječnim) veznicima, §~513.4
\item[συνιδεῖν] složenica glagola ὁράω, §~327.3, §~254; ovisan o \textgreek[variant=ancient]{δύνασθαι}
\item[συνεῖναί] složenica glagola ἵημι, §~306, preneseno značenje (LSJ συνίημι II); ovisan o \textgreek[variant=ancient]{δύνασθαι}
\item[δύνασθαι] §~312.5; glagol nepotpuna značenja u rečenici otvara mjesto infinitivima; subjekt infinitiva nije izrečen (npr. τινα ``nitko'')
\item[διὰ ταῦθ᾽] διά s akuzativom, značenje vidi LSJ διά B.III
\item[ὁρᾶν] §~327.3, §~243; ovisan o ἦν
\item[ἦν] §~315, bezlično: bilo je moguće (LSJ εἰμί A.VI), otvara mjesto dopuni u infinitivu
\item[ἑστῶτας] mješoviti perfekti, §~317.1
\item[ὄντας] §~315


\end{description}

%2


{\large
\begin{greek}
\noindent κἀν τούτῳ \\
\tabto{2em} πολλοὶ μὲν καὶ τὰ σφέτερα ἐκκομιζόμενοι, \\
\tabto{2em} πολλοὶ δὲ καὶ τὰ ἀλλότρια ἁρπάζοντες \\
\tabto{4em} ἀλλήλοις τε ἐνεπλάζοντο \\
\tabto{4em} καὶ περὶ τοῖς σκεύεσιν ἐσφάλλοντο, \\
\tabto{4em} καὶ οὔτε προϊέναι ποι \\
\tabto{6em} οὔθ' ἑστάναι \\
\tabto{4em} εἶχον, \\
\tabto{6em} ἀλλ' ὤθουν ὠθοῦντο, \\
\tabto{6em} ἀνέτρεπον ἀνετρέποντο. \\

\end{greek}
}

\begin{description}[noitemsep]
\item[κἀν τούτῳ] priložna upotreba οὗτος, LSJ s.~v. VIII.6.b
\item[πολλοὶ μὲν\dots\ πολλοὶ δὲ\dots] koordinacija česticama izražava suprotnost
\item[ἐκκομιζόμενοι] složenica glagola \textgreek[variant=ancient]{κομίζω,} §~232
\item[ἁρπάζοντες] §~231
\item[ἐνεπλάζοντο] složenica glagola \textgreek[variant=ancient]{πλάζω,} rekcija τινί §~232, augment §~238
\item[ἐσφάλλοντο] §~232; u doslovnom značenju, LSJ s.~v. A
\item[οὔτε\dots\ οὔθ'\dots] koordinacija sastavnim (niječnim) veznicima §~513.4
\item[προϊέναι] složenica glagola εἶμι §~314.1
\item[ἑστάναι] mješoviti perfekti, §~317.1
\item[εἶχον] §~231 (augment §~236), §~327.13, otvara mjesto dopuni u infinitivu (LSJ ἔχω A.III)
\item[ἀλλ'] suprotni veznik §~515.1
\item[ὤθουν] §~231, §~243
\item[ὠθοῦντο] §~232, §~243
\item[ἀνέτρεπον] složenica glagola \textgreek[variant=ancient]{τρέπω,} §~231, augment §~239
\item[ἀνετρέποντο] složenica glagola τρέπω, \textgreek[variant=ancient]{§~232,} augment §~239

\end{description}

%3


{\large
\begin{greek}
\noindent καὶ συχνοὶ μὲν ἀπεπνίγοντο \\
\tabto{2em} συχνοὶ δὲ συνετρίβοντο, \\
\tabto{4em} ὥστε \\
\tabto{6em} σφίσι \\
\tabto{4em} μηδὲν ὅ τι τῶν δυναμένων \\
\tabto{6em} ἀνθρώποις \\
\tabto{6em} ἐν τῷ τοιούτῳ πάθει \\
\tabto{6em} κακῶν \\
\tabto{6em} συμβῆναι \\
\tabto{4em} μὴ συνενεχθῆναι.\\

\end{greek}
}

\begin{description}[noitemsep]
\item[συχνοὶ] potpumbeni predikat ili predikatni adjunkt (pridjev umjesto hrv. priloga) §~369; usp. LSJ συχνός A.II: ``brojni''
\item[συχνοὶ μὲν\dots\ συχνοὶ δὲ\dots] koordinacija česticama, izražava se suprotnost
\item[ἀπεπνίγοντο] složenica glagola \textgreek[variant=ancient]{πνίγω,} §~232, augment §~239
\item[συνετρίβοντο] složenica glagola \textgreek[variant=ancient]{τρίβω,} §~232, augment §~238
\item[ὥστε] veznik uvodi posljedičnu rečenicu, ovdje s infinitivom, §~473
\item[τῶν δυναμένων] supstantivirani particip, §~312.5, otvara mjesto dopuni u infinitivu; genitiv partitivni §~395
\item[συμβῆναι] složenica glagola \textgreek[variant=ancient]{βαίνω} §~321.6; rekcija τινί; korjeniti aorist §~316, LSJ A.III \textgreek[variant=ancient]{συμβαίνω}
\item[μὴ] negacija uz infinitiv u posljedičnoj rečenici §~473, slijedi iza složene negacije \textgreek[variant=ancient]{(μηδὲν)} pa se poricanje ukida, §~510
\item[συνενεχθῆναι] složenica glagola \textgreek[variant=ancient]{φέρω} §~327.5, LSJ \textgreek[variant=ancient]{συμφέρω} B.III

\end{description}

%4


{\large
\begin{greek}
\noindent οὐδὲ γὰρ οὐδ' \\
\tabto{2em} ἀποφυγεῖν που\\
\tabto{4em} ῥᾳδίως ἐδύναντο· \\
κἂν \\
\tabto{2em} ἐκ τοῦ παρόντος \\
τις περιεσώθη, \\
\tabto{2em} ἐς ἕτερον \\
\tabto{2em} ἐμπεσὼν \\
ἐφθείρετο.\\

\end{greek}
}

\begin{description}[noitemsep]
\item[οὐδὲ γὰρ οὐδ'] nadovezivanje na prethodnu rečenicu i uvođenje daljnjeg objašnjenja, LSJ \textgreek[variant=ancient]{οὐδέ} C.II
\item[ἀποφυγεῖν] složenica glagola \textgreek[variant=ancient]{φεύγω,} s.~116, §~254
\item[που] neodređeni prilog izriče mjesto, LSJ που A
\item[ἐδύναντο] §~312.5, glagol nepotpunog značenja otvara mjesto dopuni u infinitivu
\item[κἂν] realna pogodbena rečenica §~475: čak i ako\dots
\item[τοῦ παρόντος] sc.\ κακοῦ; složenica glagola \textgreek[variant=ancient]{εἰμί,} §~315; supstantivirani particip; LSJ \textgreek[variant=ancient]{πάρειμι} II
\item[περιεσώθη] složenica glagola \textgreek[variant=ancient]{σῴζω,} s.~118, §~296
\item[ἐμπεσὼν] složenica glagola \textgreek[variant=ancient]{πίπτω} §~327.17, §~254
\item[ἐφθείρετο] §~232; imperfekt iterativne (višekratne) radnje Smyth 1893

\end{description}


%5


{\large
\begin{greek}
\noindent καὶ ταῦτα \\
\tabto{2em} οὐκ ἐν μιᾷ μόνον \\
\tabto{2em} ἀλλ' ἐπὶ πλείους \\
\tabto{4em} καὶ ἡμέρας \\
\tabto{4em} καὶ νύκτας \\
ὁμοίως ἐγίνετο. \\

\end{greek}
}

\begin{description}[noitemsep]
\item[οὐκ\dots\ μόνον ἀλλ'\dots] ne samo\dots, nego\dots
\item[ἐν μιᾷ\dots\ ἐπὶ πλείους\dots] sc.\ \textgreek[variant=ancient]{ἡμέρας καὶ νύκτας}
\item[ἐγίνετο] §~325.11; jonski i helenistički oblik; §~232; slaganje sa subjektom u množini §~361

\end{description}

%6


{\large
\begin{greek}
\noindent καὶ πολλοὶ μὲν οἶκοι \\
\tabto{2em} ἔρημοι \\
\tabto{4em} τοῦ βοηθήσοντός \\
\tabto{6em} σφισιν \\
ἀπώλοντο, \\
πολλοὶ δὲ \\
\tabto{2em} καὶ ὑπ' αὐτῶν τῶν ἐπικουρούντων \\
προσκατεπρήσθησαν· \\
\tabto{2em} οἱ γὰρ στρατιῶται, \\
\tabto{2em} οἵ τε ἄλλοι\\
\tabto{2em} καὶ οἱ νυκτοφύλακες, \\
\tabto{4em} πρὸς τὰς ἁρπαγὰς ἀφορῶντες \\
\tabto{2em} οὐχ ὅσον \\
\tabto{4em} οὐ κατεσβέννυσάν \\
\tabto{6em} τινα \\
\tabto{2em} ἀλλὰ καὶ προσεξέκαιον.\\

\end{greek}
}

\begin{description}[noitemsep]
\item[πολλοὶ μὲν\dots\ πολλοὶ δὲ\dots] koordinacija česticama, izražava suprotnost
\item[τοῦ βοηθήσοντός] genetivus copiae et inopiae uz \textgreek[variant=ancient]{ἔρημοι} §~403.2; supstantivirani particip §~243
\item[σφισιν] \textit{dativus commodi et incommodi} §~412.1 
\item[ἀπώλοντο] §~319.15, §~254
\item[τῶν ἐπικουρούντων] supstantivirani particip, LSJ \textgreek[variant=ancient]{ἐπικουρέω} II
\item[προσκατεπρήσθησαν] složenica glagola \textgreek[variant=ancient]{πίμπρημι} §~312.3, LSJ \textgreek[variant=ancient]{προσκαταπίμπραμαι,} \textit{hapax} Diona Kasija; augment §~238
\item[οἱ γὰρ] čestica uvodi objašnjenje prethodnog navoda
\item[ἀφορῶντες] složenica glagola \textgreek[variant=ancient]{ὁράω} §~327.3, §~243, često otvara mjesto prijedložnom izrazu πρός τι 
\item[οὐχ ὅσον] LSJ \textgreek[variant=ancient]{ὅσος} IV.5.b; priložna upotreba \textgreek[variant=ancient]{ὅσον} uz negacije
\item[κατεσβέννυσάν] \textgreek[variant=ancient]{κατασβεννύω,} alternativni (kasniji) oblik složenice glagola \textgreek[variant=ancient]{σβέννυμι} §~319.6, §~267; augment §~238
\item[τινα] sc.\ οἶκον
\item[προσεξέκαιον] složenica glagola \textgreek[variant=ancient]{καίω,} §~231; augment §~238

\end{description}

%7


{\large
\begin{greek}
\noindent τοιούτων δὲ δὴ ἄλλων \\
\tabto{2em} ἄλλοθι \\
συμβαινόντων, \\
\tabto{2em} ὑπέλαβέ ποτε \\
\tabto{2em} τὸ πῦρ \\
\tabto{2em} ἄνεμος \\
\tabto{2em} καὶ ἐπὶ τὰ λοιπὰ \\
\tabto{4em} ὁμοῦ πάντα \\
\tabto{2em} ἤγαγεν, \\
\tabto{4em} ὥστε \\
\tabto{6em} σκευῶν μὲν πέρι \\
\tabto{8em} ἢ οἰκιῶν \\
\tabto{6em} μηδένα \\
\tabto{6em} μηδὲν \\
\tabto{6em} ἔτι φροντίσαι, \\
\tabto{6em} πάντας δὲ τοὺς λοιποὺς \\
\tabto{8em} ἑστῶτάς που \\
\tabto{10em} ἐν ἀσφαλεῖ τινι \\
\tabto{6em} ὁρᾶν \\
\tabto{8em} ὥσπερ νήσους τινὰς \\
\tabto{6em} καὶ πόλεις ἅμα πολλὰς \\
\tabto{8em} φλεγομένας, \\
\tabto{6em} καὶ ἐπὶ μὲν τοῖς σφετέροις \\
\tabto{8em} μηδὲν ἔτι \\
\tabto{6em} λυπεῖσθαι, \\
\tabto{6em} τὸ δὲ δημόσιον \\
\tabto{8em} ὀδυρομένους \\
\tabto{6em} ἀναμιμνήσκεσθαι \\
\tabto{8em} ὅτι καὶ πρότερόν ποτε \\
\tabto{10em} οὕτως \\
\tabto{12em} ὑπὸ τῶν Γαλατῶν \\
\tabto{10em} τὸ πλεῖον \\
\tabto{12em} τῆς πόλεως \\
\tabto{10em} διεφθάρη.\\

\end{greek}
}

\begin{description}[noitemsep]
\item[τοιούτων ἄλλων συμβαινόντων] §~504
\item[συμβαινόντων] složenica \textgreek[variant=ancient]{βαίνω,} §~321.6, §~231
\item[δὲ δὴ] kombinacija čestica označava prelazak na novu temu, Smyth 2839
\item[ὑπέλαβέ] složenica \textgreek[variant=ancient]{λαμβάνω,} §~321.14, §~254
\item[ἤγαγεν] s.~116, §~257 
\item[ὥστε] veznik uvodi posljedičnu rečenicu, ovdje s A+I, §~473
\item[σκευῶν μὲν\dots] \textbf{πάντας δὲ\dots}\ koordinacija česticama, izražava suprotnost
\item[πέρι] anastrofa, prijedlog dolazi iza imenice na koju se odnosi
\item[φροντίσαι] česta dopuna περί τινος (LSJ \textgreek[variant=ancient]{φροντίζω} II.2), §~267; ovisno o \textgreek[variant=ancient]{ὥστε}
\item[ἑστῶτάς] mješoviti perfekti, §~317.1
\item[ὁρᾶν] akuzativ s infinitivom, subjektni akuzativ je πάντας δὲ τοὺς λοιποὺς ἑστῶτάς, objekti su νήσους τινὰς καὶ πόλεις ἅμα πολλὰς φλεγομένας; ovisno o \textgreek[variant=ancient]{ὥστε}; §~327.3, §~243
\item[φλεγομένας] §~232
\item[ἐπὶ μὲν τοῖς σφετέροις\dots] \textbf{τὸ δὲ δημόσιον\dots}\ koordinacija česticama izražava suprotnost
\item[λυπεῖσθαι] §~232
\item[ὀδυρομένους] §~232
\item[ἀναμιμνήσκεσθαι] akuzativ s infinitivom, subjektni akuzativ je ὀδυρομένους; složenica \textgreek[variant=ancient]{μιμνήσκω,} §~232; ovisno o \textgreek[variant=ancient]{ὥστε;} kao \textit{verbum sentiendi} otvara mjesto zavisnoj izričnoj rečenici
\item[ὅτι] veznik uvodi izričnu rečenicu, §~467
\item[καὶ πρότερόν ποτε] \textbf{ὑπὸ τῶν Γαλατῶν} autor aludira na galsko pustošenje Rima 390.\ pr.~Kr. pod vodstvom Brena, poglavice Senona, nakon poraza Rimljana kod rijeke Alije; jedini put da je strana vojska osvojila Rim u osam stoljeća (do vizigotskog osvajanja 410.\ po~Kr.)
\item[τὸ πλεῖον] supstantiviranje pridjeva članom §~373
\item[τῆς πόλεως] genitiv partitivni, §~395
\item[διεφθάρη] složenica glagola \textgreek[variant=ancient]{φθείρω,} §~292

\end{description}


%kraj
