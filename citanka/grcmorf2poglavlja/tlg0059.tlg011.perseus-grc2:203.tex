% Unesi ispravke NZ, 6. 4. 2020.
% Unio ispravke NZ <2021-12-31 pet>
%\section*{O autoru}

%TKTK


\section*{O tekstu}

Odlomak je dio Sokratova govora o Erotu iz Platonova dijaloga \textgreek[variant=ancient]{Συμπόσιον} \textit{(Gozba),} djela nastalog između 385.\ i 370.\ pr.~Kr. Dijalog, koji se odigrava na gozbi kod atenskoga tragičkog pjesnika Agatona nakon njegove pobjede na lenejama 416., razmatra pojam ljubavi \textgreek[variant=ancient]{(ἔρως).} Odavde potječe Sokratova pohvala Erotu, jedan od izvora renesansnog pojma \textit{platonske ljubavi.}

U Platonovu prikazu, Sokrat izlaže pretposljednji, poslije petorice sudionika gozbe. No, počinje tuđim riječima. Sokrat, naime, prepričava što je čuo i naučio od proročice Diotime iz Mantineje. Prije nego što je nju čuo, Sokrat je o ljubavi mislio poput svojih drugova s gozbe. Tako je Diotima Sokratu ono što je on svojim učenicima.

Pohvala u ovom odlomku prikazuje Erotovo mitsko porijeklo koje je odredilo i njegov karakter.


%\newpage

\section*{Pročitajte naglas grčki tekst.}

Plat.\ Symposium 203b-203e

%Naslov prema izdanju

\medskip


{\large

\begin{greek}

\noindent ὅτε γὰρ ἐγένετο ἡ Ἀφροδίτη, ἡστιῶντο οἱ θεοὶ οἵ τε ἄλλοι καὶ ὁ τῆς Μήτιδος ὑὸς Πόρος. ἐπειδὴ δὲ ἐδείπνησαν, προσαιτήσουσα οἷον δὴ εὐωχίας οὔσης ἀφίκετο ἡ Πενία, καὶ ἦν περὶ τὰς θύρας. ὁ οὖν Πόρος μεθυσθεὶς τοῦ νέκταρος — οἶνος γὰρ οὔπω ἦν — εἰς τὸν τοῦ Διὸς κῆπον εἰσελθὼν βεβαρημένος ηὗδεν. ἡ οὖν Πενία ἐπιβουλεύουσα διὰ τὴν αὑτῆς ἀπορίαν παιδίον ποιήσασθαι ἐκ τοῦ Πόρου, κατακλίνεταί τε παρ´ αὐτῷ καὶ ἐκύησε τὸν Ἔρωτα. διὸ δὴ καὶ τῆς Ἀφροδίτης ἀκόλουθος καὶ θεράπων γέγονεν ὁ Ἔρως, γεννηθεὶς ἐν τοῖς ἐκείνης γενεθλίοις, καὶ ἅμα φύσει ἐραστὴς ὢν περὶ τὸ καλὸν καὶ τῆς Ἀφροδίτης καλῆς οὔσης. ἅτε οὖν Πόρου καὶ Πενίας ὑὸς ὢν ὁ Ἔρως ἐν τοιαύτῃ τύχῃ καθέστηκεν. πρῶτον μὲν πένης ἀεί ἐστι, καὶ πολλοῦ δεῖ ἁπαλός τε καὶ καλός, οἷον οἱ πολλοὶ οἴονται, ἀλλὰ σκληρὸς καὶ αὐχμηρὸς καὶ ἀνυπόδητος καὶ ἄοικος, χαμαιπετὴς ἀεὶ ὢν καὶ ἄστρωτος, ἐπὶ θύραις καὶ ἐν ὁδοῖς ὑπαίθριος κοιμώμενος, τὴν τῆς μητρὸς φύσιν ἔχων, ἀεὶ ἐνδείᾳ σύνοικος. κατὰ δὲ αὖ τὸν πατέρα ἐπίβουλός ἐστι τοῖς καλοῖς καὶ τοῖς ἀγαθοῖς, ἀνδρεῖος ὢν καὶ ἴτης καὶ σύντονος, θηρευτὴς δεινός, ἀεί τινας πλέκων μηχανάς, καὶ φρονήσεως ἐπιθυμητὴς καὶ πόριμος, φιλοσοφῶν διὰ παντὸς τοῦ βίου, δεινὸς γόης καὶ φαρμακεὺς καὶ σοφιστής· καὶ οὔτε ὡς ἀθάνατος πέφυκεν οὔτε ὡς θνητός, ἀλλὰ τοτὲ μὲν τῆς αὐτῆς ἡμέρας θάλλει τε καὶ ζῇ, ὅταν εὐπορήσῃ, τοτὲ δὲ ἀποθνῄσκει, πάλιν δὲ ἀναβιώσκεται διὰ τὴν τοῦ πατρὸς φύσιν, τὸ δὲ ποριζόμενον ἀεὶ ὑπεκρεῖ, ὥστε οὔτε ἀπορεῖ Ἔρως ποτὲ οὔτε πλουτεῖ, σοφίας τε αὖ καὶ ἀμαθίας ἐν μέσῳ ἐστίν.

\end{greek}

}


\section*{Analiza i komentar}

%1

{\large
\begin{greek}
\noindent ὅτε γὰρ ἐγένετο ἡ ᾿Αφροδίτη, \\
ἡστιῶντο \\
οἱ θεοὶ \\
οἵ τε ἄλλοι \\
καὶ ὁ τῆς Μήτιδος ὑὸς Πόρος.\\

\end{greek}
}

\begin{description}[noitemsep]
\item[ἐγένετο] §~254 (glag. osnove §~325.11)
\item[γὰρ] čestica ovdje u eksplanatornoj ili emfatičkoj funkciji: naime, baš
\item[ἡστιῶντο] §~243 (glag.\ osnove §~301.B s.~116)
\item[Πόρος] sin Metide, u \textit{Gozbi} personifikacija snalažljivosti

\end{description}

%2

{\large
\begin{greek}
\noindent ἐπειδὴ δὲ ἐδείπνησαν, \\
προσαιτήσουσα \\
\tabto{2em} οἷον δὴ \uuline{εὐωχίας οὔσης}\\
ἀφίκετο ἡ Πενία, \\
καὶ ἦν \\
\tabto{2em} περὶ τὰς θύρας.\\

\end{greek}
}

\begin{description}[noitemsep]
\item[ἐδείπνησαν] §~267, tvorba aorista kod vokalskih osnova §~269 (glag.\ osnove §~301.B s.~116)
\item[δὲ] čestica ovdje blago adverzativnog značenja, u naraciji nastavlja razvijanje misli: a…
\item[προσαιτήσουσα] particip futura u namjernom značenju; §~258, sprezanje kao prezent §~259, tvorba futura kod verba vocalia §~260 (glag.\ osnove §~301.B s.~116)
\item[ἀφίκετο] §~254 (glag.\ osnove §~321.8)
\item[οὔσης] §~315
\item[εὐωχίας οὔσης] genitiv apsolutni §~504
\item[οἷον δὴ… οὔσης] §~503.2.b., adverbni particip s uzročnim značenjem: budući da je bila, jer je bila\dots; οἷον kao i ὡς, ἅτε s participom izražava uzrok, LSJ οἷος V.3 
\item[ἦν] §~315

\end{description}
%3

{\large
\begin{greek}
\noindent ὁ οὖν Πόρος \\
μεθυσθεὶς \\
\tabto{2em} τοῦ νέκταρος \\
— οἶνος γὰρ \\
\tabto{2em} οὔπω ἦν — \\
εἰς τὸν \\
\tabto{2em} τοῦ Διὸς \\
κῆπον \\
εἰσελθὼν \\
βεβαρημένος \\
ηὗδεν.\\

\end{greek}
}

\begin{description}[noitemsep]
\item[μεθυσθεὶς] rekcija: τινος; §~296 (glag. osnove §~324.4)
\item[γὰρ] čestica ovdje u eksplanatornoj ili emfatičkoj funkciji: naime, baš
\item[ἦν] §~315
\item[εἰσελθὼν] §~254, naglasak §~255, augment §~238 (glag. osnove §~327.2)
\item[βεβαρημένος] §~272 (glag. osnove §~301.B s.~116)
\item[ηὗδεν] §~231, augment §~235

\end{description}
%4

{\large
\begin{greek}
\noindent ἡ οὖν Πενία \\
ἐπιβουλεύουσα \\
\tabto{2em} διὰ τὴν αὑτῆς ἀπορίαν \\
\tabto{2em} παιδίον ποιήσασθαι\\
\tabto{4em} ἐκ τοῦ Πόρου, \\
κατακλίνεταί τε \\
\tabto{2em} παρ' αὐτῷ \\
καὶ ἐκύησε \\
\tabto{2em} τὸν ῎Ερωτα.\\

\end{greek}
}

\begin{description}[noitemsep]
\item[ἐπιβουλεύουσα] §~231; otvara mjesto infinitivu; LSJ ἐπιβουλεύω 3. s infinitivom: namjeravati ili planirati što učiniti
\item[ποιήσασθαι] §~267 (glag. osnove §~301.B s.~116); infinitiv je dopuna participu ἐπιβουλεύουσα, objekt infinitiva je παιδίον
\item[κατακλίνεταί τε… καὶ ἐκύησε] rečenice koordinirane veznicima τε i καὶ: i… i…
\item[κατακλίνεταί] §~231
\item[ἐκύησε] §~267

\end{description}
%5

{\large
\begin{greek}
\noindent διὸ δὴ \\
καὶ τῆς ᾿Αφροδίτης ἀκόλουθος\\
καὶ θεράπων \\
\tabto{2em} γέγονεν ὁ ῎Ερως, \\
γεννηθεὶς \\
\tabto{2em} ἐν τοῖς ἐκείνης γενεθλίοις,

καὶ ἅμα \\
\tabto{2em} φύσει \\
\tabto{4em} ἐραστὴς ὢν \\
\tabto{4em} περὶ τὸ καλὸν \\
\tabto{6em} καὶ \uuline{τῆς ᾿Αφροδίτης καλῆς οὔσης}.\\

\end{greek}
}

\begin{description}[noitemsep]
\item[διὸ] (= δι᾽ ὅ) zato, zbog toga
\item[δὴ] čestica naglašava i ističe neki dio iskaza vrlo precizno: upravo, stvarno, u stvari, točno
\item[γέγονεν] §~277 (glag.\ osnove §~325.11)
\item[γεννηθεὶς] §~296 (glag.\ osnove §~301.B s.~116); u značenju uzročne rečenice
\item[ἐραστὴς\dots\ περὶ τὸ καλὸν] ``ljubitelj ljepote, ljubavnik onog što je lijepo''
\item[ὢν] §~315; u značenju uzročne rečenice
\item[καὶ… καὶ] rečenice su koordinirane ponavljanjem veznika καὶ: i… i…
\item[οὔσης] §~315
\item[τῆς ᾿Αφροδίτης καλῆς οὔσης] genitiv apsolutni u značenju uzročne rečenice; §~504

\end{description}
%6

{\large
\begin{greek}
\noindent ἅτε οὖν \\
\tabto{2em} Πόρου καὶ Πενίας ὑὸς ὢν \\
ὁ ῎Ερως \\
\tabto{2em} ἐν τοιαύτῃ τύχῃ \\
\tabto{4em} καθέστηκεν.\\

\end{greek}
}

\begin{description}[noitemsep]
\item[ὢν] §~315
\item[ἅτε… ὢν] §~503.2.b, ἅτε s participom ovdje izriče uzrok kako ga doživljava govornik
\item[καθέστηκεν] §~311, augment §~238

\end{description}
%7

{\large
\begin{greek}
\noindent πρῶτον μὲν \\
\tabto{2em} πένης ἀεί ἐστι, \\
\tabto{2em} καὶ πολλοῦ δεῖ \\
\tabto{4em} ἁπαλός τε καὶ καλός, \\
\tabto{2em} οἷον οἱ πολλοὶ οἴονται, \\
\tabto{2em} ἀλλὰ \\
\tabto{4em} σκληρὸς καὶ αὐχμηρὸς καὶ ἀνυπόδητος καὶ ἄοικος, \\
\tabto{4em} χαμαιπετὴς ἀεὶ ὢν καὶ ἄστρωτος, \\
\tabto{4em} ἐπὶ θύραις καὶ ἐν ὁδοῖς \\
\tabto{6em} ὑπαίθριος κοιμώμενος, \\
\tabto{4em} τὴν τῆς μητρὸς φύσιν \\
\tabto{6em} ἔχων, \\
\tabto{4em} ἀεὶ ἐνδείᾳ σύνοικος.\\

\end{greek}
}

\begin{description}[noitemsep]
\item[ἐστι] §~315
\item[δεῖ] §~243
\item[πένης ἐστι] imenski predikat
\item[πολλοῦ δεῖ ] fraza u rečenici ima adverbnu funkciju: daleko od…
\item[ἁπαλός… καλός] dijelovi imenskog predikata
\item[οἴονται] §~231 (glag. osnove §~325.18)
\item[οἷον… οἴονται] poredbena rečenica koju uvodi odnosni korelativ οἷον u značenju „kao što…”
\item[ὢν] §~315
\item[σκληρὸς…] \textbf{\textgreek[variant=ancient]{αὐχμηρὸς… ἀνυπόδητος… ἄοικος… χαμαιπετὴς… ἄστρωτος}} imenske dopune uz particip ὢν, u funkciji imenskoga predikata
\item[ἐν ὁδοῖς] ``uz putove'' (liježe)
\item[κοιμώμενος] §~243 (glag. osnove §~301.B s.~116)
\item[ἔχων] §~231
\item[σύνοικος] glagolska dopuna nije izrečena, ἐστι ili ὢν; u funkciji imenskog predikata

\end{description}

%8

{\large
\begin{greek}
\noindent κατὰ δὲ αὖ τὸν πατέρα \\
\tabto{2em} ἐπίβουλός ἐστι \\
\tabto{4em} τοῖς καλοῖς καὶ τοῖς ἀγαθοῖς,\\
\tabto{2em} ἀνδρεῖος ὢν καὶ ἴτης καὶ σύντονος, \\
\tabto{2em} θηρευτὴς δεινός, \\
\tabto{2em} ἀεί τινας πλέκων μηχανάς, \\
\tabto{2em} καὶ φρονήσεως ἐπιθυμητὴς καὶ πόριμος, \\
\tabto{2em} φιλοσοφῶν διὰ παντὸς τοῦ βίου, \\
\tabto{2em} δεινὸς γόης καὶ φαρμακεὺς καὶ σοφιστής· \\
\tabto{2em} καὶ οὔτε ὡς ἀθάνατος πέφυκεν οὔτε ὡς θνητός, \\
\tabto{2em} ἀλλὰ \\
\tabto{4em} τοτὲ μὲν τῆς αὐτῆς ἡμέρας θάλλει τε καὶ ζῇ, \\
\tabto{6em} ὅταν εὐπορήσῃ, \\
\tabto{4em} τοτὲ δὲ ἀποθνῄσκει, \\
\tabto{4em} πάλιν δὲ ἀναβιώσκεται \\
\tabto{6em} διὰ τὴν τοῦ πατρὸς φύσιν, \\
\tabto{4em} τὸ δὲ ποριζόμενον ἀεὶ ὑπεκρεῖ, \\
\tabto{6em} ὥστε οὔτε ἀπορεῖ ῎Ερως ποτὲ \\
\tabto{6em} οὔτε πλουτεῖ, \\
\tabto{4em} σοφίας τε αὖ καὶ ἀμαθίας \\
\tabto{6em} ἐν μέσῳ ἐστίν.\\

\end{greek}
}

\begin{description}[noitemsep]
\item[ἐστι] §~315
\item[ἐπίβουλός ἐστι] rekcija: τινι; imenski predikat
\item[τοῖς καλοῖς καὶ τοῖς ἀγαθοῖς] dativ srednjeg roda (poopćeni izraz: ``svemu što je\dots'')
\item[ὢν] §~315
\item[ἀνδρεῖος…] \textbf{\textgreek[variant=ancient]{ἴτης… σύντονος… θηρευτὴς}} imenske dopune participu ὢν u funkciji imenskog predikata
\item[πλέκων] §~231
\item[ἐπιθυμητὴς… πόριμος] imenske dopune neizrečenom finitnom ili participskom obliku glagola biti (ἐστι ili ὢν)
\item[φιλοσοφῶν] §~243
\item[δεινὸς… φαρμακεὺς… σοφιστής] imenske dopune neizrečenom finitnom ili participskom obliku glagola biti (ἐστι ili ὢν)
\item[οὔτε… οὔτε] nezavisno složena sastavna rečenica: niti… niti…
\item[πέφυκεν] §~280, paradigma §~301.A s.~110
\item[ὡς ἀθάνατος… ὡς θνητός] pridjevska dopuna predikatu (kopulativnom glagolu)
\item[ἀλλὰ] suprotni veznik uvodi zavisno složenu suprotnu rečenicu, „nego…“
\item[τοτὲ μὲν… τοτὲ δὲ] koordinirana suprotnost: jednom… zatim, sada…zatim…, sada… onda…
\item[θάλλει] §~231
\item[εὐπορήσῃ] §~267
\item[ὅταν εὐπορήσῃ] vremenski veznik ὅταν uvodi zavisnu vremensku rečenicu s nijansom eventualnog pogodbenog značenja: dok god…, kad god\dots
\item[ἀποθνῄσκει] §~231
\item[ἀναβιώσκεται] §~231
\item[ποριζόμενον] §~231
\item[ὑπεκρεῖ] §~243
\item[ἀπορεῖ] §~243
\item[ὥστε… ἀπορεῖ… πλουτεῖ] posljedični veznik ὥστε uvodi posljedičnu rečenicu: tako da…
\item[πλουτεῖ] §~243
\item[ἐστίν] §~315
\item[ἐν μέσῳ ἐστίν] imenski predikat; priložna oznaka čini imenski dio predikata

\end{description}

%kraj
