% \section*{O autoru}
% Unio ispravke NZ <2021-12-30 čet>

\section*{O tekstu}

Sačuvana Plutarhova djela okupljena su u dvije zbirke: Βίοι παράλληλοι (\textit{Usporedni životopisi}) i Ἠϑικά (\textit{Moralia, Spisi o moralu}). \textit{Usporedni životopisi} donose biografije slavnih Grka i Rimljana, većinom historijskih ličnosti, u analognim parovima (Demosten i Ciceron, Aleksandar Veliki i Cezar, itd). Do nas je stiglo pedeset takvih tekstova u dvadeset jednom paru, jednoj četverostrukoj kombinaciji (Agid i Kleomen – Tiberije i Gaj Grakho) i četiri pojedinačna životopisa (među njima je i Artakserkso, jedini koji nije ni Grk ni Rimljanin). Životopisa je bilo više; izgubljen je npr. par Epaminonda – Scipion.

Parovi životopisa često su popraćeni rezimeom sličnosti i razlika \textgreek[variant=ancient]{(σύγκρισις),} gdje redovno prednost ostvaruje Grk; \textit{Usporedni životopisi} tako su i glorifikacija grčke prošlosti. No, Plutarha ne zanima toliko historiografija koliko biografija. On sam kaže \textgreek[variant=ancient]{οὔτε γὰρ ἱστορίας γράφομεν, ἀλλὰ βίους} (Plut.\ Alex.~1.2). Životopisi se od povjesnice za nj razlikuju po tome što se karakter i osobnost, koji su u centru interesa biografa, mogu razumjeti i iz detalja, malih, svakodnevnih postupaka: \textgreek[variant=ancient]{οὔτε ταῖς ἐπιφανεστάταις πράξεσι πάντως ἔνεστι δήλωσις ἀρετῆς ἢ κακίας, ἀλλὰ πρᾶγμα βραχὺ πολλάκις καὶ ῥῆμα καὶ παιδιά τις ἔμφασιν ἤθους ἐποίησε μᾶλλον ἢ μάχαι μυριόνεκροι καὶ παρατάξεις αἱ μέγισται καὶ πολιορκίαι πόλεων} (``Vrlina ili mana često se ne otkrivaju u sjajnim pothvatima onoliko koliko u nevažnoj stvari, riječi, šali, koja karakter ocrtava bolje od bitaka s tisućama poginulih, od silnih vojničkih pohoda i opsada gradova'', ibid). Životopisi su, tako, plod filozofsko-etičkog pristupa, i imaju jaspo~Kr.ičko-pedagošku (a ne zabavnu) intenciju. 

Alkibijad (Ἀλκιβιάδης, oko 450. – 404.\ pr.~Kr. bio je slobodouman, iznimno bogat i talentiran atenski aristokrat, dominantna figura političke scene posljednjih desetljeća V.\ st. Posebno je važnu ulogu odigrao pri odlučivanju o Sicilskoj ekspediciji (Thuc.~6). Izuzetnom je osobnom karizmom i govorničkim darom bio u stanju uspješno uvjeravati, čak i opčiniti narodnu skupštinu. Bio je Periklov štićenik i omiljeni Sokratov učenik. No, Alkibijad je i krajnje kontroverzan lik; i suvremenici i historiografi, od Tukidida i Ksenofonta do Plutarha, njegovo djelovanje i život ocjenjuju na različite načine. Dojmljive portrete Alkibijada ostavili su Platon (u dijalozima \textgreek[variant=ancient]{Συμπόσιον,} \textit{Gozba} i i \textgreek{Ἀλκιβιάδης α´,} Alkibijad prvi) i pseudo-Andokid (govor \textgreek[variant=ancient]{Κατὰ Ἀλκιβιάδου,} \textit{Protiv Alkibijada}).

U odabranom se odlomku prikazuje Alkibijadova \textgreek[variant=ancient]{φύσις,} temeljna komponenta karaktera, koja se očituje već u djetinjstvu; kasnije će karakter oblikovati, osim slučaja \textgreek[variant=ancient]{(Τύχη),} i vanjski utjecaji, a prvenstveno \textit{odabir} \textgreek[variant=ancient]{(προαίρεσις)} između dobra i zla, koji svaki veliki čovjek mora prije ili kasnije učiniti. Alkibijadova se narav na dojmljiv način očitovala u dvama događajima, jednom prilikom rvanja \textgreek[variant=ancient]{(παλαίειν)} i drugi put za igranja piljcima \textgreek[variant=ancient]{(ἀστράγαλοι).}

%\newpage

\section*{Pročitajte naglas grčki tekst.}

Plut.\ Alcibiades 2.1–2.3

%Naslov prema izdanju

\medskip


{\large

\begin{greek}

\noindent φύσει δὲ πολλῶν ὄντων καὶ μεγάλων παθῶν ἐν αὐτῷ, τὸ φιλόνεικον ἰσχυρότατον ἦν καὶ τὸ φιλόπρωτον, ὡς δῆλόν ἐστι τοῖς παιδικοῖς ἀπομνημονεύμασιν.

ἐν μὲν γὰρ τῷ παλαίειν πιεζούμενος, ὑπὲρ τοῦ μὴ πεσεῖν ἀναγαγὼν πρὸς τὸ στόμα τὰ ἅμματα τοῦ πιεζοῦντος, οἷος ἦν διαφαγεῖν τὰς χεῖρας. ἀφέντος δὲ τὴν λαβὴν ἐκείνου καὶ εἰπόντος· δάκνεις, ὦ Ἀλκιβιάδη, καθάπερ αἱ γυναῖκες, οὐκ ἔγωγε, εἶπεν, ἀλλ' ὡς οἱ λέοντες.

ἔτι δὲ μικρὸς ὢν ἔπαιζεν ἀστραγάλοις ἐν τῷ στενωπῷ, τῆς δὲ βολῆς καθηκούσης εἰς αὐτὸν ἅμαξα φορτίων ἐπῄει. πρῶτον μὲν οὖν ἐκέλευε περιμεῖναι τὸν ἄγοντα τὸ ζεῦγος· ὑπέπιπτε γὰρ ἡ βολὴ τῇ παρόδῳ τῆς ἁμάξης· μὴ πειθομένου δὲ δι' ἀγροικίαν, ἀλλ' ἐπάγοντος, οἱ μὲν ἄλλοι παῖδες διέσχον, ὁ δ' Ἀλκιβιάδης καταβαλὼν ἐπὶ στόμα πρὸ τοῦ ζεύγους καὶ παρατείνας ἑαυτόν, ἐκέλευεν οὕτως, εἰ βούλεται, διεξελθεῖν, ὥστε τὸν μὲν ἄνθρωπον ἀνακροῦσαι τὸ ζεῦγος ὀπίσω δείσαντα, τοὺς δ' ἰδόντας ἐκπλαγῆναι καὶ μετὰ βοῆς συνδραμεῖν πρὸς αὐτόν.

\end{greek}

}


\section*{Analiza i komentar}

%1

{\large
\begin{greek}
\noindent φύσει δὲ πολλῶν ὄντων \\
καὶ μεγάλων παθῶν \\
\tabto{2em} ἐν αὐτῷ, \\
τὸ φιλόνεικον ἰσχυρότατον ἦν \\
καὶ τὸ φιλόπρωτον, \\
\tabto{2em} ὡς δῆλόν ἐστι \\
\tabto{4em} τοῖς παιδικοῖς ἀπομνημονεύμασιν.\\

\end{greek}
}

\begin{description}[noitemsep]
\item[δὲ] čestica δέ označava nadovezivanje na prethodni iskaz
\item[ὄντων\dots\ ἐν αὐτῷ] §~315; kopulativni glagol otvara mjesto nužnoj predikatnoj dopuni; predikatni dio genitiva apsolutnog, §~504
\item[ἰσχυρότατον ἦν] §~315; kopulativni glagol otvara mjesto nužnoj predikatnoj dopuni (imenski predikat, Smyth 910)
\item[ὡς δῆλόν ἐστι] §~315; kopulativni glagol otvara mjesto nužnoj predikatnoj dopuni (imenski predikat, Smyth 910)
\item[τοῖς παιδικοῖς ἀπομνημονεύμασιν] instrumentalni dativ, odgovara na pitanje ``po čemu''

\end{description}

%2

{\large
\begin{greek}
\noindent ἐν μὲν γὰρ τῷ παλαίειν \\
πιεζούμενος, \\
ὑπὲρ τοῦ μὴ πεσεῖν \\
ἀναγαγὼν \\
\tabto{2em} πρὸς τὸ στόμα \\
τὰ ἅμματα \\
\tabto{2em} τοῦ πιεζοῦντος, \\
οἷος ἦν \\
\tabto{2em} διαφαγεῖν τὰς χεῖρας.\\

\end{greek}
}

\begin{description}[noitemsep]
\item[ἐν μὲν\dots\ ἀφέντος δὲ\dots] koordinacija rečeničnih članova pomoću čestica μέν\dots\ δέ\dots
\item[γὰρ] čestica γάρ najavljuje iznošenje razloga ili objašnjenja: naime\dots
\item[ἐν\dots\ τῷ παλαίειν] §~231; supstantivirani infinitiv §~497
\item[πιεζούμενος] §~243
\item[τοῦ μὴ πεσεῖν] §~327.17; §~254; supstantivirani infinitiv §~497; najčešća negacija uz infinitiv je μή, §~509c
\item[ἀναγαγὼν] složenica glagola ἄγω, s.~116; §~254; §~257
\item[τοῦ πιεζοῦντος] §~243; supstantivirani particip §~373
\item[οἷος ἦν] §~315; kopulativni glagol otvara mjesto nužnoj predikatnoj dopuni (imenski predikat, Smyth 910); fraza οἷός εἰμι §~473 B. 4, LSJ οἷος III.1.b
\item[διαφαγεῖν] složenica glagola ἐσθίω, §~327.8; §~254

\end{description}

%3
{\large
\begin{greek}
\noindent ἀφέντος δὲ τὴν λαβὴν ἐκείνου \\
\tabto{2em} καὶ εἰπόντος· \\
\tabto{4em} δάκνεις, ὦ Ἀλκιβιάδη, \\
\tabto{6em} καθάπερ αἱ γυναῖκες, \\
οὐκ ἔγωγε, \\
εἶπεν, \\
ἀλλ' ὡς οἱ λέοντες.\\

\end{greek}
}

\begin{description}[noitemsep]
\item[ἀφέντος\dots\ καὶ εἰπόντος] predikatni dijelovi genitiva apsolutnog, §~504
\item[ἀφέντος] složenica glagola ἵημι, §~305
\item[εἰπόντος] §~254, §~327.7
\item[δάκνεις] §~231
\item[οὐκ\dots\ ἀλλ'] koordinacija rečeničnih članova, ``ne\dots\ nego\dots''
\item[ἔγωγε] istaknuti oblik lične zamjenice, §~206.2
\item[εἶπεν] §~254, §~327.7

\end{description}

%4
{\large
\begin{greek}
\noindent ἔτι δὲ μικρὸς ὢν \\
ἔπαιζεν \\
\tabto{2em} ἀστραγάλοις \\
\tabto{2em} ἐν τῷ στενωπῷ, \\
τῆς δὲ βολῆς καθηκούσης εἰς αὐτὸν \\
ἅμαξα φορτίων \\
ἐπῄει.\\

\end{greek}
}

\begin{description}[noitemsep]
\item[δὲ] čestica δέ označava nadovezivanje na prethodni iskaz
\item[μικρὸς ὢν] §~315; kopulativni glagol otvara mjesto nužnoj predikatnoj dopuni
\item[ἔπαιζεν] §~231
\item[ἔπαιζεν\dots\ τῆς δὲ βολῆς] koordinacija rečeničnih članova pomoću čestice δέ
\item[τῆς δὲ βολῆς] βολή je ``bacanje'' u igri
\item[καθηκούσης] §~231 (složenica glagola ἥκω, §~453.2); predikatni dio genitiva apsolutnog, §~504
\item[ἐπῄει] složenica glagola εἶμι, §~314.1

\end{description}

%5
{\large
\begin{greek}
\noindent πρῶτον μὲν οὖν ἐκέλευε \\
\tabto{2em} περιμεῖναι \\
\tabto{2em} τὸν ἄγοντα τὸ ζεῦγος· \\
ὑπέπιπτε γὰρ ἡ βολὴ τῇ παρόδῳ τῆς ἁμάξης· \\
μὴ πειθομένου δὲ δι' ἀγροικίαν, \\
\tabto{2em} ἀλλ' ἐπάγοντος, \\
οἱ μὲν ἄλλοι παῖδες διέσχον, \\
ὁ δ' Ἀλκιβιάδης \\
\tabto{2em} καταβαλὼν ἐπὶ στόμα πρὸ τοῦ ζεύγους \\
\tabto{2em} καὶ παρατείνας ἑαυτόν, \\
ἐκέλευεν οὕτως, \\
\tabto{2em} εἰ βούλεται, \\
διεξελθεῖν, \\
ὥστε \\
\tabto{2em} τὸν μὲν ἄνθρωπον ἀνακροῦσαι τὸ ζεῦγος ὀπίσω δείσαντα, \\
\tabto{2em} τοὺς δ' ἰδόντας ἐκπλαγῆναι \\
\tabto{4em} καὶ μετὰ βοῆς συνδραμεῖν πρὸς αὐτόν.\\

\end{greek}
}

\begin{description}[noitemsep]
\item[πρῶτον μὲν\dots\ μὴ πειθομένου δὲ\dots] koordinacija rečeničnih članova pomoću čestica μέν\dots\ δέ\dots
\item[ἐκέλευε] §~231; otvara mjesto objektu u akuzativu te infinitivu (zapovijedati kome da što čini)
\item[περιμεῖναι τὸν ἄγοντα] složenica glagola μένω, §~325.7; §~267; §~270; supstantivirani particip §~373
\item[ὑπέπιπτε γὰρ] §~231; §~238; čestica γάρ najavljuje iznošenje razloga ili objašnjenja, ``naime''
\item[μὴ πειθομένου\dots\ ἀλλ' ἐπάγοντος\dots] koordinacija rečeničnih članova (ovdje genitiva apsolutnih, §~504) ``ne\dots\ nego\dots''
\item[πειθομένου] §~232
\item[δι' ἀγροικίαν] ἀγροικία je ``nekultura'', ``priprostost''; \textgreek[variant=ancient]{(χαρακτήρ) ἀγροικίας} naslovljen je četvrti prikaz u djelu \textgreek[variant=ancient]{Χαρακτῆρες} ili \textgreek[variant=ancient]{Ἠθικοὶ χαρακτῆρες,} \textit{Karakteri}, peripatetičara Teofrasta iz Ereza (\textgreek[variant=ancient]{Θεόφραστος,} oko 371.\ – oko 287.\ pr.~Kr.
\item[ἐπάγοντος] §~231
\item[οἱ μὲν ἄλλοι\dots\ ὁ δ' Ἀλκιβιάδης\dots] koordinacija rečeničnih članova pomoću čestica μέν\dots\ δέ\dots
\item[διέσχον] složenica glagola ἔχω, §~327.13; §~254
\item[καταβαλὼν] složenica glagola βάλλω, s.~118; §~254
\item[παρατείνας] složenica glagola τείνω, s.~118; §~267; §~270
\item[ἐκέλευεν] sc. αὐτόν διεξελθεῖν; §~231
\item[εἰ βούλεται] §~232; veznik εἰ uvodi protazu realne pogodbene rečenice (apodoza je οὕτως\dots\ διεξελθεῖν), §~475
\item[διεξελθεῖν] §~254; složenica glagola ἔρχομαι, §~327.2; infinitiv ovisan o glagolu \textgreek{κελεύω} u glavnoj rečenici
\item[ὥστε\dots\ ἀνακροῦσαι\dots\ ἐκπλαγῆναι] ὥστε uvodi zavisnu posljedičnu rečenicu s predikatom u infinitivu (ovdje su dvije), §~473
\item[τὸν μὲν ἄνθρωπον\dots] \textbf{τοὺς δ' ἰδόντας\dots}\ koordinacija rečeničnih članova pomoću čestica μέν\dots\ δέ\dots
\item[τὸν\dots\ ἄνθρωπον ἀνακροῦσαι] §~267; akuzativ s infinitivom, §~491
\item[δείσαντα] §~317.3; §~267
\item[τοὺς\dots\ ἰδόντας ἐκπλαγῆναι καὶ\dots\ συνδραμεῖν] akuzativ s infinitivom (ovdje s dva infinitiva), §~491
\item[τοὺς\dots\ ἰδόντας] §~254; §~327.3; supstantivirani particip §~373
\item[ἐκπλαγῆναι] §~292; §~327.10
\item[συνδραμεῖν] §~254; složenica glagola τρέχω, §~327.4
\end{description}


%kraj
