% Unio ispravke NZ <2021-12-28 uto>
\section*{O tekstu}

Koncem Peloponeskoga rata, 403.\ ili 402.\ pr.~Kr., Andokid se okoristio općom amnestijom i vratio se iz progonstva u Atenu, gdje je potom obavljao niz javnih funkcija. No, njegovi protivnici – možda motivirani zavišću zbog Andokidova dobra materijalna statusa? – nisu mirovali. Već 399.\ pr.~Kr. Andokid je opet pred sucima, ponovo optužen za korištenje prava koja mu ne pripadaju (sudjelovanje na Eleuzinskim misterijama 400.\ ili 399.), uz dodatnu optužbu da je u vrijeme misterija na oltar Eleuzinija u Ateni stavio pribjegarsku maslinovu grančicu (što je u vrijeme misterija bilo zabranjeno). Andokid se branio govorom \textgreek[variant=ancient]{Περὶ τῶν μυστηρίων} (\textit{O misterijama}) i bio je oslobođen. Govor \textit{O misterijama} smatra se Andokidovim najuspjelijim radom.

Djelo ima tradicionalnu strukturu kakvu nalazimo kod govornika poput Antifonta ili Lizije: nakon proemija \textgreek[variant=ancient]{(προοίμιον)} slijedi kratka \textgreek[variant=ancient]{πρόθεσις} (opći nacrt slučaja), pa pripovijedanje, argumentacija i epilog. Ekscerpt koji čitamo citira tekst službenog dokumenta, zakona koji se naziva Demofantov dekret, a na koji se Andokid referira kao na Solonov zakon: ako tko sudjeluje u protudemokratskim postupanjima u Ateni bit će kažnjen, baš kao što će onaj koji im se suprotstavi biti nagrađen.

Filološka istraživanja potvrdila su da zakonski tekst nije autentičan.



\newpage

\section*{Pročitajte naglas grčki tekst.}

And.\ De mysteriis 97

%Naslov prema izdanju

\medskip


{\large

\begin{greek}

\noindent ὁ δὲ ὅρκος ἔστω ὅδε· κτενῶ καὶ λόγῳ καὶ ἔργῳ καὶ ψήφῳ καὶ τῇ ἐμαυτοῦ χειρί, ἂν δυνατὸς ὦ, ὃς ἂν καταλύσῃ τὴν δημοκρατίαν τὴν Ἀθήνησι, καὶ ἐάν τις ἄρξῃ τιν᾽ ἀρχὴν καταλελυμένης τῆς δημοκρατίας τὸ λοιπόν, καὶ ἐάν τις τυραννεῖν ἐπαναστῇ ἢ τὸν τύραννον συγκαταστήσῃ· καὶ ἐάν τις ἄλλος ἀποκτείνῃ, ὅσιον αὐτὸν νομιῶ εἶναι καὶ πρὸς θεῶν καὶ δαιμόνων, ὡς πολέμιον κτείναντα τὸν Ἀθηναίων, καὶ τὰ κτήματα τοῦ ἀποθανόντος πάντα ἀποδόμενος ἀποδώσω τὰ ἡμίσεα τῷ ἀποκτείναντι, καὶ οὐκ ἀποστερήσω οὐδέν.

ἐὰν δέ τις κτείνων τινὰ τούτων ἀποθάνῃ ἢ ἐπιχειρῶν, εὖ ποιήσω αὐτόν τε καὶ τοὺς παῖδας τοὺς ἐκείνου καθάπερ Ἁρμόδιόν τε καὶ Ἀριστογείτονα καὶ τοὺς ἀπογόνους αὐτῶν. ὁπόσοι δὲ ὅρκοι ὀμώμονται Ἀθήνησιν ἢ ἐν τῷ στρατοπέδῳ ἢ ἄλλοθί που ἐναντίοι τῷ δήμῳ τῷ Ἀθηναίων, λύω καὶ ἀφίημι.

\end{greek}

}


\section*{Analiza i komentar}

%1

{\large
\begin{greek}
\noindent Ὁ δὲ ὅρκος ἔστω ὅδε·  \\
\tabto{2em} κτενῶ καὶ λόγῳ καὶ ἔργῳ καὶ ψήφῳ καὶ τῇ ἐμαυτοῦ χειρί, \\
\tabto{4em} ἂν δυνατὸς ὦ, \\
\tabto{2em} ὃς ἂν καταλύσῃ \\
\tabto{4em} τὴν δημοκρατίαν \\
\tabto{6em} τὴν Ἀθήνησι\\

\end{greek}
}

\begin{description}[noitemsep]
\item[ἔστω] §~315
\item[κτενῶ] §~258
\item[ὦ] §~315
\item[ἂν δυνατὸς ὦ] §~315; eventualna pogodbena rečenica: „ako budem…“
\item[δυνατὸς ὦ] imenski predikat
\item[καταλύσῃ] §~267
\item[ὃς ἂν καταλύσῃ] odnosna zamjenica ὃς uvodi zavisnu odnosnu rečenicu; odnosna rečenica ima značenje pogodbene protaze (eventualnog oblika)
\item[τὴν δημοκρατίαν τὴν Ἀθήνησι] prilog u atributnom položaju

\end{description}

%2
{\large
\begin{greek}
\noindent καὶ \\
\tabto{2em} ἐάν τις ἄρξῃ \\
\tabto{4em} τιν' ἀρχὴν \\
\tabto{2em} καταλελυμένης τῆς δημοκρατίας τὸ λοιπόν,\\
καὶ \\
\tabto{2em} ἐάν τις τυραννεῖν ἐπαναστῇ \\
\tabto{2em} ἢ \\
\tabto{2em} τὸν τύραννον συγκαταστήσῃ·\\

\end{greek}
}

\begin{description}[noitemsep]
\item[ἄρξῃ] §~267
\item[ἐάν… ἄρξῃ] pogodbeni veznik ἐάν uvodi zavisnu pogodbenu rečenicu eventualnog značenja, „ako…“
\item[ἀρχὴν] LSJ ἀρχή II, 3: javna ili državna služba, u frazi ἀρχὴν ἄρχειν
\item[καταλελυμένης] §~272, perfekt medijalni i pasivni §~284
\item[καταλελυμένης τῆς δημοκρατίας] genitiv apsolutni §~504
\item[τὸ λοιπόν] priložno, s vremenskim značenjem: „nadalje”, „ubuduće”, „nakon što”
\item[ἐπαναστῇ] §~306, glagol otvara mjesto dopuni u infinitivu
\item[τυραννεῖν] §~243
\item[ἐάν… ἐπαναστῇ] pogodbeni veznik ἐάν uvodi zavisnu pogodbenu rečenicu eventualnog značenja: „ako…“
\item[συγκαταστήσῃ] §~306
\item[ἢ… συγκαταστήσῃ] pogodba se nastavlja povezivanjem veznikom ἢ, „ili“, bez izricanja pogodbenog veznika: „ili (ako)…“

\end{description}

%3

{\large
\begin{greek}
\noindent Καὶ ἐάν τις ἄλλος ἀποκτείνῃ, \\
\tabto{2em} ὅσιον \\
\tabto{4em} αὐτὸν νομιῶ εἶναι \\
\tabto{2em} καὶ πρὸς θεῶν καὶ δαιμόνων, \\
\tabto{2em} ὡς πολέμιον κτείναντα τὸν ᾿Αθηναίων, \\
\tabto{2em} καὶ \\
\tabto{2em} τὰ κτήματα τοῦ ἀποθανόντος πάντα ἀποδόμενος \\
\tabto{4em} ἀποδώσω τὰ ἡμίσεα τῷ ἀποκτείναντι, \\
\tabto{4em} καὶ \\
\tabto{4em} οὐκ ἀποστερήσω οὐδέν.\\

\end{greek}
}

\begin{description}[noitemsep]
\item[ἀποκτείνῃ] §~231
\item[ἐάν… ἀποκτείνῃ] pogodbeni veznik ἐάν uvodi zavisnu pogodbenu rečenicu eventualnog futurskog značenja, „ako…“
\item[νομιῶ] §~268, atički futur §~263.2, glagol otvara mjesto dopuni u akuzativu s infinitivom
\item[εἶναι] §~315
\item[ὅσιον αὐτὸν εἶναι] akuzativ s infinitivom
\item[κτείναντα] particip s poredbenim veznikom ὡς: „kao onaj koji…“
\item[τοῦ ἀποθανόντος] §~254, supstantivirani particip §~499.2
\item[ἀποδόμενος] rekcija mediopasiva: τι τινος, „prodavati što od koga“; §~306
\item[ἀποδώσω] rekcija aktiva: τί τινι, „dati kome što“
\item[τῷ ἀποκτείναντι] §~231, supstantivirani particip §~499.2
\item[ἀποστερήσω] §~258

\end{description}

%4

{\large
\begin{greek}
\noindent Ἐὰν δέ τις\\
\tabto{2em} κτείνων τινὰ τούτων\\
ἀποθάνῃ \\
\tabto{2em} ἢ ἐπιχειρῶν, \\
εὖ ποιήσω \\
\tabto{2em} αὐτόν τε καὶ τοὺς παῖδας \\
\tabto{4em} τοὺς ἐκείνου \\
καθάπερ Ἁρμόδιόν τε καὶ Ἀριστογείτονα \\
\tabto{2em} καὶ τοὺς ἀπογόνους αὐτῶν.\\

\end{greek}
}

\begin{description}[noitemsep]
\item[κτείνων] §~231
\item[ἀποθάνῃ] §~254
\item[ἐπιχειρῶν] §~243, zamišlja se dopuna u infinitivu ἐπιχειρῶν κτείνειν
\item[Ἐὰν δέ… ἀποθάνῃ] pogodbeni veznik ἐὰν uvodi zavisnu pogodbenu rečenicu eventualnog futurskog značenja, „ako bude…“
\item[εὖ ποιήσω] rekcija: τινα; §~258
\item[Ἁρμόδιόν τε καὶ Ἀριστογείτονα] Harmodije i Aristogiton, legendarni prijatelji koji su ubili jednog od dvojice atenskih tirana, Pizistratova sina Hiparha (514.\ pr.~Kr). Harmodije je ubijen na mjestu atentata, a Aristogiton je izdahnuo na mučilištu. Preostali tiranin, Hiparhov brat Hipija, srušen je 508, te je Klistenovim reformama u Ateni uspostavljena demokracija. Atenjani su Harmodija i Aristogitona štovali kao osloboditelje i „ubojice tirana” (τυραννοκτόνοι) i podigli im spomenik (oko 510.\ pr.~Kr.).
\end{description}


%5

{\large
\begin{greek}
\noindent Ὁπόσοι δὲ ὅρκοι ὀμώμονται \\
\tabto{2em} Ἀθήνησιν \\
\tabto{4em} ἢ ἐν τῷ στρατοπέδῳ \\
\tabto{4em} ἢ ἄλλοθί που \\
ἐναντίοι \\
\tabto{2em} τῷ δήμῳ τῷ Ἀθηναίων, \\
λύω καὶ ἀφίημι.\\

\end{greek}
}

\begin{description}[noitemsep]
\item[ὀμώμονται] §~272, reduplikacija §~275.2 (glag. osnove §~319.16) [3. l. pl. ind. perf. mp. se inače tvori opisno, part. perf. mp. + gl. biti]
\item[Ὁπόσοι… ὀμώμονται] odnosna zamjenica ὁπόσοι uvodi zavisnu odnosnu rečenicu, „koje god…“
\item[δὲ] čestica δὲ ovdje u adverzativnom značenju, u grčkom je postpozitivna, u hrvatskom dolazi na prvo mjesto u rečenici, „a…“
\item[λύω καὶ ἀφίημι] sc. \textgreek[variant=ancient]{αὐτούς (ὅρκους)}
\item[λύω] §~231
\item[ἀφίημι] §~305 (ostatak paradigme §~306, glag. osnove §~311)

\end{description}


%kraj
