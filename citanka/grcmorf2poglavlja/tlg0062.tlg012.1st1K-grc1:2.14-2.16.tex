% Unesi ispravke NJ, 8. 6. 2020.
% Unio ispravke NZ <2022-01-06 čet>

%TKTK


\section*{O tekstu}

Lukijanove \textit{Istinite pripovijesti} \textgreek[variant=ancient]{(Ἀληϑῆ διηγήματα)} uz njegovog \textit{Ikaromenipa} jedan su od najranijih primjera žanra znanstvene fantastike; ujedno su možda i Lukijanovo najpoznatije djelo. Dvama knjigama \textit{Istinitih pripovijesti} Lukijan parodira historiografe poput Ktezije iz Knida koji iznose neprovjerene (i neprovjerljive) informacije te fantastične priče. Čitamo o putovanju na Mjesec, boravku u kitovoj utrobi, posjetu otoku od sira te Otoku blaženih kojim vlada mudri sudac podzemnoga svijeta Radamant. 

Odlomak koji slijedi opisuje gozbu na Otoku blaženih, s naglaskom na zabavni program.

%\newpage

\section*{Pročitajte naglas grčki tekst.}

Luc.\ Verae historiae 2.14-2.16

%Naslov prema izdanju

\medskip


{\large

\begin{greek}

\noindent τὸ δὲ συμπόσιον ἔξω τῆς πόλεως πεποίηνται ἐν τῷ Ἠλυσίῳ καλουμένῳ πεδίῳ· λειμὼν δέ ἐστιν κάλλιστος καὶ περὶ αὐτὸν ὕλη παντοία πυκνή, ἐπισκιάζουσα τοὺς κατακειμένους. καὶ στρωμνὴν μὲν ἐκ τῶν ἀνθῶν ὑποβέβληνται, διακονοῦνται δὲ καὶ παραφέρουσιν ἕκαστα οἱ ἄνεμοι πλήν γε τοῦ οἰνοχοεῖν· τούτου γὰρ οὐδὲν δέονται, ἀλλʼ ἔστι δένδρα περὶ τὸ συμπόσιον ὑάλινα μεγάλα τῆς διαυγεστάτης ὑάλου, καὶ καρπός ἐστι τῶν δένδρων τούτων ποτήρια παντοῖα καὶ τὰς κατασκευὰς καὶ τὰ μεγέθη. ἐπειδὰν οὖν παρίῃ τις ἐς τὸ συμπόσιον, τρυγήσας ἓν ἢ καὶ δύο τῶν ἐκπωμάτων παρατίθεται, τὰ δὲ αὐτίκα οἴνου πλήρη γίνεται. οὕτω μὲν πίνουσιν, ἀντὶ δὲ τῶν στεφάνων αἱ ἀηδόνες καὶ τὰ ἄλλα τὰ μουσικὰ ὄρνεα ἐκ τῶν πλησίον λειμώνων τοῖς στόμασιν ἀνθολογοῦντα κατανείφει αὐτοὺς μετʼ ᾠδῆς ὑπερπετόμενα. καὶ μὴν καὶ μυρίζονται ὧδε· νεφέλαι πυκναὶ ἀνασπάσασαι μύρον ἐκ τῶν πηγῶν καὶ τοῦ ποταμοῦ καὶ ἐπιστᾶσαι ὑπὲρ τὸ συμπόσιον ἠρέμα τῶν ἀνέμων ὑποθλιβόντων ὕουσι λεπτὸν ὥσπερ δρόσον.

\noindent ἐπὶ δὲ τῷ δείπνῳ μουσικῇ τε καὶ ᾠδαῖς σχολάζουσιν· ᾄδεται δὲ αὐτοῖς τὰ Ὁμήρου ἔπῃ μάλιστα· καὶ αὐτὸς δὲ πάρεστι καὶ συνευωχεῖται αὐτοῖς ὑπὲρ τὸν Ὀδυσσέα κατακείμενος. οἱ μὲν οὖν χοροὶ ἐκ παίδων εἰσὶν καὶ παρθένων· ἐξάρχουσι δὲ καὶ συνᾴδουσιν Εὔνομός τε ὁ Λοκρὸς καὶ Ἀρίων ὁ Λέσβιος καὶ Ἀνακρέων καὶ Στησίχορος· καὶ γὰρ τοῦτον παρ´ αὐτοῖς ἐθεασάμην, ἤδη τῆς Ἑλένης αὐτῷ διηλλαγμένης. ἐπειδὰν δὲ οὗτοι παύσωνται ᾄδοντες, δεύτερος χορὸς παρέρχεται ἐκ κύκνων καὶ χελιδόνων καὶ ἀηδόνων. ἐπειδὰν δὲ καὶ οὗτοι ᾄσωσιν, τότε ἤδη πᾶσα ἡ ὕλη ἐπαυλεῖ τῶν ἀνέμων καταρχόντων.

\noindent μέγιστον δὲ δὴ πρὸς εὐφροσύνην ἐκεῖνο ἔχουσιν· πηγαί εἰσι δύο παρὰ τὸ συμπόσιον, ἡ μὲν γέλωτος, ἡ δὲ ἡδονῆς· ἐκ τούτων ἑκατέρας πάντες ἐν ἀρχῇ τῆς εὐωχίας πίνουσιν καὶ τὸ λοιπὸν ἡδόμενοι καὶ γελῶντες διάγουσιν.

\end{greek}

}


\section*{Analiza i komentar}

%1

{\large
\begin{greek}
\noindent τὸ δὲ συμπόσιον \\
 \tabto{2em} ἔξω τῆς πόλεως \\
πεποίηνται \\
\tabto{2em} ἐν τῷ ᾿Ηλυσίῳ καλουμένῳ πεδίῳ· \\
\tabto{4em} λειμὼν δέ ἐστιν κάλλιστος \\
\tabto{4em} καὶ περὶ αὐτὸν \\
\tabto{4em} ὕλη \\
\tabto{6em} παντοία πυκνή, \\
\tabto{6em} ἐπισκιάζουσα \\
\tabto{8em} τοὺς κατακειμένους. \\

\end{greek}
}

\begin{description}[noitemsep]
\item[δὲ] čestica označava nastavljanje na prethodni iskaz
\item[πεποίηνται] §~272
\item[καλουμένῳ] §~243
\item[δέ] čestica označava nastavljanje na prethodni iskaz
\item[ἐστιν] §~315, LSJ s.~v.\ A.I
\item[ἐπισκιάζουσα] §~231; složenica σκιάζω
\item[τοὺς κατακειμένους] supstantivirani particip, složenica κεῖμαι, §~315.a

\end{description}

%2

{\large
\begin{greek}
\noindent καὶ στρωμνὴν μὲν \\
\tabto{2em} ἐκ τῶν ἀνθῶν \\
ὑποβέβληνται, \\
διακονοῦνται δὲ \\
καὶ παραφέρουσιν ἕκαστα \\
οἱ ἄνεμοι \\
\tabto{2em} πλήν γε τοῦ οἰνοχοεῖν· \\
\tabto{4em} τούτου γὰρ \\
\tabto{4em} οὐδὲν δέονται, \\
\tabto{6em} ἀλλ' ἔστι δένδρα \\
\tabto{8em} περὶ τὸ συμπόσιον \\
\tabto{6em} ὑάλινα μεγάλα \\
\tabto{8em} τῆς διαυγεστάτης ὑάλου, \\
\tabto{6em} καὶ καρπός ἐστι \\
\tabto{8em} τῶν δένδρων τούτων \\
\tabto{6em} ποτήρια παντοῖα \\
\tabto{8em} καὶ τὰς κατασκευὰς \\
\tabto{8em} καὶ τὰ μεγέθη. \\

\end{greek}
}

\begin{description}[noitemsep]
\item[στρωμνὴν μὲν\dots\ διακονοῦνται δὲ\dots] koordinacija parom čestica
\item[στρωμνὴν] objekt glagola ὑποβέβληνται
\item[ὑποβέβληνται] §~272; složenica glagola βάλλω, s.~118, LSJ ὑποβάλλω A.I.1
\item[διακονοῦνται] §~243; značenje kao u aktivu, LSJ διακονέω A.I
\item[παραφέρουσιν] §~231; složenica φέρω §~327.5 
\item[γε] čestica naglašava riječi između kojih stoji
\item[τοῦ οἰνοχοεῖν] §~243; supstantivirani infinitiv §~373
\item[γὰρ] uvodi objašnjenje
\item[οὐδὲν] LSJ οὐδείς A.III
\item[δέονται] §~232; rekcija τινός; LSJ δέω B
\item[ἀλλ'] uvodi suprotnu rečenicu §~515
\item[ἔστι δένδρα] kongruencija sa subjektom u pluralu srednjeg roda §~361
\item[τὸ συμπόσιον] posebno, prostorno značenje uvjetovano prijedlogom (tako i niže \textgreek{ὑπὲρ τὸ συμπόσιον)}
\item[ἔστι] §~315; LSJ s.~v.\ A.I
\item[καρπός ἐστι] imenski predikat, Smyth 909; kongruencija sa subjektom u pluralu srednjeg roda §~361
\item[καὶ τὰς κατασκευὰς καὶ τὰ μεγέθη] akuzativ obzira (prosti akuzativ) §~389
\end{description}
%3


{\large
\begin{greek}
\noindent ἐπειδὰν οὖν \\
παρίῃ τις \\
\tabto{2em} ἐς τὸ συμπόσιον, \\
τρυγήσας \\
ἓν ἢ καὶ δύο \\
\tabto{2em} τῶν ἐκπωμάτων \\
παρατίθεται, \\
τὰ δὲ\\
\tabto{2em} αὐτίκα \\
\tabto{2em} οἴνου πλήρη γίνεται. \\

\end{greek}
}

\begin{description}[noitemsep]
\item[ἐπειδὰν] uvodi zavisnu vremensku rečenicu u značenju pogodbene protaze (eventualnog oblika, s konjunktivom) §~488.2
\item[παρίῃ] složenica εἶμι, §~314
\item[τρυγήσας] §~267; LSJ τρῠγάω II 
\item[παρατίθεται] složenica τίθημι, §~305
\item[τὰ δὲ] sc.\ ἐκπώματα 
\item[οἴνου] \textit{genetivus copiae et inopiae} uz pridjev §~403.2
\item[γίνεται] §~232; jonski i helenistički oblik glagola γίγνομαι §~325.11; kongruencija sa subjektom u pluralu srednjeg roda §~361

\end{description}

%4


{\large
\begin{greek}
\noindent οὕτω μὲν πίνουσιν, \\
ἀντὶ δὲ τῶν στεφάνων \\
αἱ ἀηδόνες \\
καὶ τὰ ἄλλα τὰ μουσικὰ ὄρνεα \\
\tabto{2em} ἐκ τῶν πλησίον λειμώνων \\
\tabto{2em} τοῖς στόμασιν ἀνθολογοῦντα\\
κατανίφει \\
αὐτοὺς \\
\tabto{2em} μετ' ᾠδῆς \\
ὑπερπετόμενα. \\

\end{greek}
}

\begin{description}[noitemsep]
\item[οὕτω μὲν\dots\ ἀντὶ δὲ\dots] koordinacija parom čestica
\item[πίνουσιν] §~231
\item[ἀνθολογοῦντα] §~243
\item[κατανίφει] kasniji oblik glagola κατανείφω, složenica glagola νείφω, preneseno značenje (LSJ κατανείφω A); kongruencija sa subjektom u pluralu srednjeg roda §~361
\item[ὑπερπετόμενα] ovaj particip, kao i gornji ἀνθολογοῦντα, dopuna su istog subjekta \textgreek{τὰ μουσικὰ ὄρνεα}; §~232; složenica πέτομαι
\end{description}


%5


{\large
\begin{greek}
\noindent καὶ μὴν καὶ μυρίζονται ὧδε· \\
νεφέλαι πυκναὶ \\
\tabto{2em} ἀνασπάσασαι μύρον \\
\tabto{4em} ἐκ τῶν πηγῶν καὶ τοῦ ποταμοῦ \\
\tabto{2em} καὶ ἐπιστᾶσαι \\
\tabto{4em} ὑπὲρ τὸ συμπόσιον \\
ἠρέμα \\
\tabto{2em} τῶν ἀνέμων ὑποθλιβόντων \\
ὕουσι \\
λεπτὸν ὥσπερ δρόσον.\\

\end{greek}
}

\begin{description}[noitemsep]
\item[καὶ μὴν καὶ] kombinacija čestica uvodi novu stavku u nabrajanju, Denniston s.~351
\item[μυρίζονται] §~232
\item[ἀνασπάσασαι] §~267; složenica σπάω
\item[ἐπιστᾶσαι] §~305; složenica ἵστημι (jonski oblik)
\item[τῶν ἀνέμων ὑποθλιβόντων] GA §~504
\item[ὑποθλιβόντων] §~231; složenica θλίβω
\item[ὕουσι] §~231
\item[λεπτὸν] atribut uz δρόσον
\end{description}


%6


{\large
\begin{greek}
\noindent ἐπὶ δὲ τῷ δείπνῳ \\
\tabto{2em} μουσικῇ τε καὶ ᾠδαῖς \\
σχολάζουσιν· \\
ᾄδεται δὲ αὐτοῖς \\
τὰ Ὁμήρου ἔπη \\
\tabto{2em} μάλιστα· \\
καὶ αὐτὸς δὲ \\
πάρεστι \\
καὶ συνευωχεῖται \\
\tabto{2em} αὐτοῖς \\
\tabto{2em} ὑπὲρ τὸν Ὀδυσσέα \\
κατακείμενος. \\

\end{greek}
}

\begin{description}[noitemsep]
\item[δὲ] označava nadovezivanje na prethodni navod
\item[σχολάζουσιν] §~231, rekcija τινι, LSJ σχολάζω A.III
\item[ᾄδεται] §~232, kongruencija sa subjektom u pluralu srednjeg roda §~361
\item[δὲ] označava nadovezivanje na prethodni navod
\item[αὐτοῖς] \textit{dativus commodi}, Smyth 1490
\item[δὲ] označava nadovezivanje na prethodni navod
\item[πάρεστι] složenica εἰμί §~315
\item[συνευωχεῖται] τινί; §~232, §~243
\item[ὑπὲρ τὸν Ὀδυσσέα] LSJ ὑπέρ B.I
\item[κατακείμενος] §~232; složenica κεῖμαι §~315.a
\end{description}

%7


{\large
\begin{greek}
\noindent οἱ μὲν οὖν χοροὶ \\
ἐκ παίδων εἰσὶν \\
\tabto{2em} καὶ παρθένων· \\
ἐξάρχουσι δὲ καὶ συνᾴδουσιν \\
Εὔνομός τε ὁ Λοκρὸς \\
\tabto{2em} καὶ Ἀρίων ὁ Λέσβιος \\
\tabto{2em} καὶ Ἀνακρέων \\
\tabto{2em} καὶ Στησίχορος· \\
καὶ γὰρ τοῦτον \\
\tabto{2em} παρ' αὐτοῖς\\
ἐθεασάμην, \\
ἤδη τῆς Ἑλένης \\
\tabto{2em} αὐτῷ \\
διηλλαγμένης. \\

\end{greek}
}

\begin{description}[noitemsep]
\item[οἱ μὲν οὖν χοροὶ\dots\ ἐξάρχουσι δὲ\dots] koordinacija parom čestica
\item[οὖν] uvodi objašnjenje: dakle\dots
\item[ἐκ παίδων εἰσὶν καὶ παρθένων] §~315; imenski predikat Smyth 909: sastavljeni su\dots
\item[ἐξάρχουσι] §~231; složenica ἄρχω s.~116
\item[συνᾴδουσιν] §~231; složenica ᾄδω
\item[Εὔνομος ὁ Λοκρὸς] Eunom iz Lokra (u Velikoj Grčkoj), legendarni pjevač
\item[Ἀρίων ὁ Λέσβιος] Arion iz Metimne, 7/6.~st.\ pr.~n.~e., lirski pjesnik
\item[Ἀνακρέων] Anakreont, lirski pjesnik, 6.~st.\ pr.~n.~e.
\item[Στησίχορος] Stezihor, korski pjesnik koji je, prema legendi, pjesmom uvrijedio Helenu; 7/6.~st.\ pr.~n.~e.
\item[καὶ γὰρ] kombinacija čestica uvodi objašnjenje: naime i\dots
\item[ἐθεασάμην] §~267
\item[τῆς Ἑλένης διηλλαγμένης] GA §~504
\item[διηλλαγμένης] §~292; složenica ἀλλάσσω, rekcija τινι; LSJ διαλλάσσω III
\end{description}

%8


{\large
\begin{greek}
\noindent ἐπειδὰν δὲ \\
οὗτοι \\
παύσωνται ᾄδοντες, \\
δεύτερος χορὸς \\
παρέρχεται \\
\tabto{2em} ἐκ κύκνων \\
\tabto{2em} καὶ χελιδόνων\\
\tabto{2em} καὶ ἀηδόνων. \\

\end{greek}
}

\begin{description}[noitemsep]
\item[ἐπειδὰν] uvodi zavisnu vremensku rečenicu u značenju pogodbene protaze (eventualnog oblika, s konjunktivom) §~488.2
\item[δὲ] označava nadovezivanje na prethodni navod
\item[παύσωνται] §~267; παύομαι otvara mjesto predikatnom participu kao nužnoj dopuni: prestajem \textit{što činiti}, §~501c
\item[ᾄδοντες] §~231
\item[παρέρχεται] §~232; složenica ἔρχομαι §~327.2
\end{description}

%9


{\large
\begin{greek}
\noindent ἐπειδὰν δὲ \\
καὶ οὗτοι \\
ᾄσωσιν, \\
τότε ἤδη \\
πᾶσα ἡ ὕλη\\
ἐπαυλεῖ \\
\tabto{2em} τῶν ἀνέμων καταρχόντων. \\

\end{greek}
}

\begin{description}[noitemsep]
\item[ἐπειδὰν] uvodi zavisnu vremensku rečenicu u značenju pogodbene protaze (eventualnog oblika, s konjunktivom) §~488.2
\item[δὲ] označava nadovezivanje na prethodni navod
\item[ᾄσωσιν] §~267
\item[ἐπαυλεῖ] §~243; složenica αὐλέω
\item[τῶν ἀνέμων καταρχόντων] GA §~504
\item[καταρχόντων] §~231; složenica glagola ἄρχω s.~116
\end{description}

%10


{\large
\begin{greek}
\noindent μέγιστον δὲ δὴ \\
\tabto{2em} πρὸς εὐφροσύνην \\
ἐκεῖνο \\
ἔχουσιν· \\
πηγαί εἰσι δύο \\
\tabto{2em} παρὰ τὸ συμπόσιον, \\
ἡ μὲν γέλωτος, \\
ἡ δὲ ἡδονῆς· \\
\tabto{2em} ἐκ τούτων ἑκατέρας\\
πάντες \\
\tabto{2em} ἐν ἀρχῇ τῆς εὐωχίας \\
πίνουσιν \\
καὶ τὸ λοιπὸν \\
ἡδόμενοι καὶ γελῶντες \\
διάγουσιν.\\

\end{greek}
}

\begin{description}[noitemsep]
\item[δὲ] označava nadovezivanje na prethodni navod
\item[δὴ] čestica koja služi isticanju, LSJ δή A.II: baš\dots
\item[ἔχουσιν] §~231
\item[εἰσι] §~315, značenje LSJ s.~v.\ A.I
\item[ἡ μὲν γέλωτος, ἡ δὲ ἡδονῆς] sc.\ πηγαί; koordinacija rečeničnih članova pomoću para čestica
\item[πίνουσιν] §~231
\item[τὸ λοιπὸν] supstantiviranje članom svih vrsta riječi §~373; LSJ λοιπός 3
\item[ἡδόμενοι] §~232
\item[γελῶντες] §~243
\item[διάγουσιν] otvara mjesto predikatnom participu kao nužnoj dopuni; §~231
\end{description}


%kraj
