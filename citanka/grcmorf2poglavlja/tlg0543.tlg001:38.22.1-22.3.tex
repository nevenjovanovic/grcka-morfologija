%\section*{O autoru}

% Unio ispravke NZ <2021-12-26 ned>


\section*{O tekstu}

Polibije je u trideset osmoj od ukupno četrdeset knjiga \textgreek[variant=ancient]{Ἱστορίαι} (\textit{Povijesti}) iznio, kako sam kaže, ``dovršetak grčke nesreće'', konačan poraz Grka i razaranje Korinta, ali i Treći punski rat i razaranje Kartage. Odlomak koji ovdje donosimo prikazuje povijesno iskustvo iz prve ruke: i sam je Polibije svjedočio reakciji Scipiona Afričkoga Mlađeg (Scipiona Emilijana, 185.–129.\ p.~n.~e) na uništenje Kartage, koje je rimski vojskovođa sam naredio. Scipionova reakcija značajna je i kao očitovanje karaktera, i s filozofskog i političkog aspekta, ali i kao primjer stupnja u kojem je rimska elita II.~st.\ p.~n.~e.\ asimilirala grčku kulturu (Scipion citira Homera).

Polibijeva 38.\ knjiga nije sačuvana u cijelosti; izvorni tekst koji imamo na mnogim je mjestima oštećen, a dijelovi su do nas došli tek kao citati kod drugih pisaca. Ovaj je odlomak prepričao povjesničar Apijan iz Aleksandrije \textgreek[variant=ancient]{(Ἀππιανὸς Ἀλεξανδρεύς,} II.~st.\ n.~e) u osmoj knjizi svoje \textgreek[variant=ancient]{Ῥωμαϊκά} (\textit{Rimska povijest}); knjiga se zove \textgreek[variant=ancient]{Λιβυκὴ ἢ Καρχηδονιακὴ καὶ Νομαδική} (\textit{Libijski ili kartaški ili numidski dio}) .

%\newpage

\section*{Pročitajte naglas grčki tekst.}

Plb.\ Historiae 38.22.1-22.3

%Naslov prema izdanju

\medskip


{\large

\begin{greek}

\noindent  ὁ δὲ Σκιπίων πόλιν ὁρῶν τότε ἄρδην τελευτῶσαν ἐς πανωλεθρίαν ἐσχάτην, λέγεται μὲν δακρῦσαι καὶ φανερὸς γενέσθαι κλαίων ὑπὲρ πολεμίων· ἐπὶ πολὺ δ' ἔννους ἐφ' ἑαυτοῦ γενόμενός τε καὶ συνιδὼν ὅτι καὶ πόλεις καὶ ἔθνη καὶ ἀρχὰς ἁπάσας δεῖ μεταβαλεῖν ὥσπερ ἀνθρώπους δαίμονα, καὶ τοῦτ' ἔπαθε μὲν Ἴλιον, εὐτυχής ποτε πόλις, ἔπαθε δὲ ἡ Ἀσσυρίων καὶ Μήδων καὶ Περσῶν ἐπ' ἐκείνοις ἀρχὴ μεγίστη γενομένη καὶ ἡ μάλιστα ἔναγχος ἐκλάμψασα ἡ Μακεδόνων, εἴτε ἑκών, εἴτε προφυγόντος αὐτὸν τοῦδε τοῦ ἔπους εἰπεῖν,\\
\tabto{2em} ἔσσεται ἦμαρ ὅταν ποτ' ὀλώλῃ Ἴλιος ἱρὴ\\
\tabto{2em} καὶ Πρίαμος καὶ λαὸς ἐυμμελίω Πριάμοιο.\\
Πολυβίου δ' αὐτὸν ἐρομένου σὺν παρρησίᾳ· καὶ γὰρ ἦν αὐτοῦ καὶ διδάσκαλος· ὅ τι βούλοιτο ὁ λόγος, φασὶν οὐ φυλαξάμενον ὀνομάσαι τὴν πατρίδα σαφῶς, ὑπὲρ ἧς ἄρα ἐς τἀνθρώπεια ἀφορῶν ἐδεδίει. καὶ τάδε μὲν Πολύβιος αὐτὸς ἀκούσας συγγράφει.

\end{greek}

}


\section*{Analiza i komentar}

%1

{\large
\begin{greek}
\noindent  ὁ δὲ Σκιπίων \\
\tabto{2em} πόλιν ὁρῶν\\
\tabto{4em} τότε ἄρδην τελευτῶσαν \\
\tabto{6em} ἐς πανωλεθρίαν ἐσχάτην, \\
λέγεται μὲν \\
\tabto{2em} δακρῦσαι καὶ φανερὸς γενέσθαι \\
\tabto{4em} κλαίων ὑπὲρ πολεμίων·

\tabto{4em} ἐπὶ πολὺ δ' ἔννους ἐφ' ἑαυτοῦ γενόμενός τε καὶ συνιδὼν \\
\tabto{6em} ὅτι καὶ πόλεις καὶ ἔθνη καὶ ἀρχὰς ἁπάσας \\
\tabto{6em} δεῖ \\
\tabto{6em} μεταβαλεῖν \\
\tabto{8em} ὥσπερ ἀνθρώπους \\
\tabto{6em} δαίμονα, \\
\tabto{6em} καὶ τοῦτ' ἔπαθε μὲν Ἴλιον, \\
\tabto{8em} εὐτυχής ποτε πόλις, \\
\tabto{6em} ἔπαθε δὲ ἡ Ἀσσυρίων \\
\tabto{8em} καὶ Μήδων \\
\tabto{8em} καὶ Περσῶν ἐπ' ἐκείνοις \\
\tabto{6em} ἀρχὴ μεγίστη γενομένη \\
\tabto{6em} καὶ ἡ μάλιστα ἔναγχος ἐκλάμψασα \\
\tabto{8em} ἡ Μακεδόνων, \\
\tabto{4em} εἴτε ἑκών, \\
\tabto{4em} εἴτε προφυγόντος \\
\tabto{6em} αὐτὸν \\
\tabto{4em} τοῦδε τοῦ ἔπους \\
\tabto{2em} εἰπεῖν,\\
\tabto{4em} ἔσσεται ἦμαρ \\
\tabto{6em} ὅταν ποτ' ὀλώλῃ Ἴλιος ἱρὴ\\
\tabto{6em} καὶ Πρίαμος \\
\tabto{6em} καὶ λαὸς ἐυμμελίω Πριάμοιο.\\

\end{greek}
}

\begin{description}[noitemsep]
\item[ὁρῶν] §~231; §~327.3
\item[τελευτῶσαν] §~267
\item[λέγεται] §~232; §~327.7
\item[δακρῦσαι] §~267
\item[φανερὸς γενέσθαι] §~254, §~255; §~325.11; §~369 φανερὸς γίγνεσθαι ``biti očito da\dots''; dopuna kopulativnog izraza predikatnim participom §~501b
\item[κλαίων] §~231
\item[ἐπὶ πολὺ] priložna oznaka vremena: ``dugo'' (LSJ πολύς IV.4)
\item[ἐφ' ἑαυτοῦ] ``za sebe'', ``u sebi''
\item[ἔννους γίγνομαι] ``postati zamišljen'', ``razmišljati''
\item[γενόμενός] §~254, §~325.11
\item[γενόμενός τε καὶ συνιδὼν] τε καὶ povezuje komplementarne elemente; ponekad je drugi član značenjski jači od prvog (Smyth 4.60.208)
\item[συνιδὼν] συνοράω, složenica od ὁράω, §~327.3; §~254
\item[δεῖ] §~325.14; otvara mjesto akuzativu s infinitivom
\item[μεταβαλεῖν] (objekt je δαίμονα) μεταβάλλω, složenica βάλλω; §~254
\item[ὥσπερ ἀνθρώπους] akuzativ korespondira s πόλεις, ἔθνη, ἀρχὰς
\item[ἔπαθε] §~327.15, §~254
\item[ἔπαθε μὲν\dots\ ἔπαθε δὲ\dots] koordinacija ostvarena česticama μὲν\dots\ δὲ
\item[ἐπ' ἐκείνοις] priložna oznaka vremena, ``nakon njih''
\item[ἀρχὴ μεγίστη γενομένη] §~254; γίγνεσθαι ima funkciju kopule, ovdje s pridjevom kao predikatnom dopunom (Smyth 4.26.917)
\item[ἐκλάμψασα] §~261, §~267
\item[ἡ Μακεδόνων] sc.\ ἀρχή
\item[εἴτε\dots\ εἴτε\dots] koordinacija pomoću rastavnih veznika, §~514
\item[ἑκών] pridjev na mjestu hrvatskog priloga, §~369
\item[προφυγόντος] προφεύγω ``izmaknuti'', ``pobjeći naprijed'' (tj.\ preduhitriti), rekcija τι ili τινα; složenica od φεύγω, s.~116; §~254
\item[αὐτὸν] objekt uz προφυγόντος
\item[εἰπεῖν] §~327.7; §~254
\item[ἔσσεται\dots\ Πριάμοιο] dva heksametra \textit{Ilijade} (Ἰλιάς), 6, 448–449; kod Homera ove stihove izgovara Hektor
\item[ἔσσεται] §~315; homerski oblik futura, odgovara atičkom ἔσται, usp.\ Autenrieth s.~v.\ εἰμί
\item[ὅταν\dots\ ὀλώλῃ] vremenski veznik ὅτε s ἄν stapa se u ὅταν i otvara mjesto konjunktivu; konjunktiv ovdje ima futursko (anticipatorno) značenje, Smyth 1810 Anticipatory Subjunctive (Homeric Subjunctive)
\item[ἱρὴ] ἱρός jonski i epski oblik pridjeva ἱερός
\item[ἐυμμελίω Πριάμοιο] homerski oblici genitiva, §~89.3

\end{description}

%2

{\large
\begin{greek}
\noindent Πολυβίου δ' αὐτὸν ἐρομένου \\
\tabto{2em} σὺν παρρησίᾳ· \\
\tabto{4em} καὶ γὰρ ἦν αὐτοῦ καὶ διδάσκαλος· \\
ὅ τι βούλοιτο ὁ λόγος, \\
φασὶν \\
\tabto{2em} οὐ φυλαξάμενον ὀνομάσαι \\
\tabto{4em} τὴν πατρίδα \\
\tabto{4em} σαφῶς, \\
\tabto{6em} ὑπὲρ ἧς ἄρα \\
\tabto{8em} ἐς τἀνθρώπεια ἀφορῶν \\
\tabto{6em} ἐδεδίει.\\

\end{greek}
}

\begin{description}[noitemsep]
\item[ἐρομένου] §~232, §~325.10; glagol otvara mjesto zavisno upitnoj rečenici; particip je dio genitiva apsolutnog
\item[καὶ γὰρ] §~517; ova kombinacija čestica uvodi objašnjenje: ``naime''
\item[ἦν\dots\ καὶ διδάσκαλος] §~315; imenski predikat (Smyth 4.26)
\item[ὅ τι βούλοιτο] §~231; §~325.13, §~328.2; §~469: upitna zamjenica uvodi zavisno upitnu rečenicu, u njoj je način predikata \textit{optativus obliquus}
\item[φασὶν] §~312.8; §~493: \textit{verbum dicendi} otvara mjesto A+I
\item[οὐ φυλαξάμενον ὀνομάσαι] §~250, §~267; §~261, §~269; A+I
\item[τἀνθρώπεια] ono što je obilježje ljudskih bića (što im je suđeno ili moguće)
\item[ἀφορῶν] §~232, §~243; rekcija εἴς τι ili τινα
\item[ἐδεδίει] §~317.3, ali oblike vidi u rječniku (npr.\ LSJ); rekcija ὑπέρ τινος
\end{description}


%3

{\large
\begin{greek}
\noindent καὶ τάδε μὲν \\
Πολύβιος αὐτὸς \\
\tabto{2em} ἀκούσας \\
συγγράφει.\\

\end{greek}
}

\begin{description}[noitemsep]
\item[ἀκούσας] §~267; s.~116
\item[συγγράφει] §~231
\end{description}




%kraj
