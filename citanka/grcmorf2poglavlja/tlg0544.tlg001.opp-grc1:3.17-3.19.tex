% Unio ispravke NZ <2022-01-05 sri>

\section*{O autoru}

Sekst Empirik (grčki Σέξτος ὁ Ἐμπειρικός, latinski Sextus Empiricus) bio je grčki filozof, astronom i liječnik (nadimak Ἐμπειρικός dobio je po pripadnosti istoimenoj medicinskoj školi) iz II.~st. po Kr. Živio je u Rimu, Ateni i Aleksandriji. Pripadao je filozofskoj školi skepticizma, zapravo neoskepticizma, koju je utemeljio Piron \textgreek[variant=ancient]{(Πύρρων)} iz Elide (IV.–III.~st. pr.~Kr; raniji pripadnici tzv.\ akademskog skepticizma bili su Arkesilaj, III.~st.\ pr.~Kr, i Karnead, II.~st.\ pr.~Kr). 

Skeptici su smatrali da se treba suzdržavati od izricanja sudova jer je svakoj tezi moguće suprotstaviti antitezu. Ideje predočuju samo druge ideje, a svaki kriterij istine leži u krugu beskonačnoga dokazivanja jedne istine drugom. No, Sekst Empirik skepticizam je ograničio na teorijski svijet znanosti, razlikujući apstraktna teologijska (teoretska) od religijskih (praktičnih i kulturno uvjetovanih) vjerovanja. 

Sustav je opširno prikazao u spisu \textit{Protiv znanstvenika} \textgreek[variant=ancient]{(Πρὸς μαϑηματικούς,} latinski \textit{Adversus mathematicos),} u 11 knjiga, osporavajući stavove brojnih mislilaca s područja gramatike, retorike, geometrije, aritmetike, astronomije, muzike, ali i logike, fizike i etike; zbog Sekstove sklonosti citiranju djelo je važno za poznavanje grčke misli općenito.

\section*{O tekstu}

\textit{Pironove postavke} \textgreek[variant=ancient]{(Πυρρώνειαι ὑποτυπώσεις)} obuhvaćaju tri knjige. Prva prikazuje pironovski skepticizam kao određenu vještinu i njegovu nagradu (spokoj) te različite moduse argumentiranja kojima se skeptik služi (ovladavanje njima čini vještinu skeptika) i fraze kojima skeptik izražava stav ἐποχή, `suzdržavanje od suda'; pokazuje se i po čemu se pironovski skepticizam razlikuje od sličnih filozofskih škola. Druga i treća knjiga iznose stavove dogmatskih filozofa o logici (II) te fizici i etici (III) uz skeptičke protuargumente. Tako druga i treća knjiga demonstriraju vještinu opisanu u prvoj.

Prvi dio treće knjige govori o fizici \textgreek[variant=ancient]{(τὸ φυσικὸν μέρος)}, a počinje razmatranjem počela \textgreek[variant=ancient]{(ἀπὸ τοῦ περὶ ἀρχῶν λόγου)}: najprije o bogu, potom o uzroku \textgreek[variant=ancient]{(περὶ αἰτίου).} Poglavlje čiji početak čitamo naslovljeno je \textgreek[variant=ancient]{εἴ ἐστι τί τινὸς αἴτιον,} ``je li što uzrok čega''. Raspravlja se postoji li uzrok uopće i osporavaju se tvrdnje da ne postoji; vjerojatno je \textgreek[variant=ancient]{(πιθανόν)} da uzrok postoji.


%\newpage

\section*{Pročitajte naglas grčki tekst.}

Sext.\ Emp.\ Πυρρώνειαι ὑποτυπώσεις 3.17–3.19

%Naslov prema izdanju

\medskip


{\large

\begin{greek}

\noindent  Πιθανόν ἐστιν εἶναι τὸ αἴτιον· πῶς γὰρ ἂν αὔξησις γένοιτο, μείωσις, γένεσις, φθορά, καθόλου κίνησις, τῶν φυσικῶν τε καὶ ψυχικῶν ἀποτελεσμάτων ἕκαστον, ἡ τοῦ παντὸς κόσμου διοίκησις, τὰ ἄλλα πάντα, εἰ μὴ κατά τινα αἰτίαν; καὶ γὰρ εἰ μηδὲν τούτων ὡς πρὸς τὴν φύσιν ὑπάρχει, λέξομεν ὅτι διά τινα αἰτίαν πάντως φαίνεται ἡμῖν τοιαῦτα, ὁποῖα οὔκ ἐστιν.

\noindent  ἀλλὰ καὶ πάντα ἐκ πάντων καὶ ὡς ἔτυχεν ἂν ἦν μὴ οὔσης αἰτίας. οἷον ἵπποι μὲν ἐκ μυῶν, εἰ τύχοι, γεννηθήσονται, ἐλέφαντες δὲ ἐκ μυρμήκων· καὶ ἐν μὲν ταῖς Αἰγυπτίαις Θήβαις ὄμβροι ποτὲ ἐξαίσιοι καὶ χιόνες ἂν ἐγίνοντο, τὰ δὲ νότια ὄμβρων οὐ μετεῖχεν, εἰ μὴ αἰτία τις ἦν, δι’ ἣν τὰ μὲν νότιά ἐστι δυσχείμερα, αὐχμηρὰ δὲ τὰ πρὸς τὴν ἔω.

\noindent  καὶ περιτρέπεται δὲ ὁ λέγων μηδὲν αἴτιον εἶναι· εἰ μὲν γὰρ ἁπλῶς καὶ ἄνευ τινὸς αἰτίας τοῦτό φησι λέγειν, ἄπιστος ἔσται· εἰ δὲ διά τινα αἰτίαν, βουλόμενος ἀναιρεῖν τὸ αἴτιον τίθησιν, ἀποδιδόσθω αἰτίαν δι’ ἣν οὐκ ἔστιν αἴτιον.

\end{greek}

}


\section*{Analiza i komentar}

%1

{\large
\begin{greek}
\noindent  Πιθανόν ἐστιν \\
\tabto{2em} εἶναι τὸ αἴτιον· \\
πῶς γὰρ ἂν \\
\tabto{2em} αὔξησις \\
γένοιτο, \\
\tabto{2em} μείωσις, γένεσις, φθορά, \\
\tabto{2em} καθόλου κίνησις, \\
\tabto{4em} τῶν φυσικῶν τε \\
\tabto{4em} καὶ ψυχικῶν ἀποτελεσμάτων \\
\tabto{2em} ἕκαστον, \\
\tabto{2em} ἡ τοῦ παντὸς κόσμου διοίκησις, \\
\tabto{2em} τὰ ἄλλα πάντα, \\
εἰ μὴ \\
\tabto{2em} κατά τινα αἰτίαν; \\

\end{greek}
}

\begin{description}[noitemsep]
\item[Πιθανόν ἐστιν] §~315; imenski predikat Smyth 909; zbog značenja, pridjev otvara mjesto A+I
\item[εἶναι] §~315; LSJ εἰμί A.II
\item[πῶς γὰρ ἂν\dots\ γένοιτο\dots\ εἰ μὴ\dots] prilog πῶς otvara mjesto pitanju; §~254, osnove §~325.11; ἂν + optativ stoji u apodozi pogodbene rečenice potencijalnog oblika, u protazi je također optativ (ovdje neizrečen jer bi se ponavljao isti glagol, sc.\ γένοιτο), §~477%sc. genoito
\item[τῶν φυσικῶν τε καὶ ψυχικῶν] koordinacija ostvarena kombinacijom sastavnih veznika §~513.2; supstantiviranje članom svih vrsta riječi, surečenica i rečenica §~373

\end{description}

%2

{\large
\begin{greek}
\noindent  καὶ γὰρ εἰ μηδὲν τούτων \\
\tabto{2em} ὡς πρὸς τὴν φύσιν \\
ὑπάρχει, \\
λέξομεν ὅτι \\
\tabto{4em} διά τινα αἰτίαν \\
\tabto{4em} πάντως \\
\tabto{2em} φαίνεται \\
\tabto{2em} ἡμῖν \\
\tabto{2em} τοιαῦτα, \\
\tabto{4em} ὁποῖα οὔκ ἐστιν.\\

\end{greek}
}

\begin{description}[noitemsep]
\item[καὶ γὰρ] §~517; ova kombinacija čestica uvodi objašnjenje: naime\dots
\item[εἰ\dots\ ὑπάρχει, λέξομεν] veznik uvodi zavisnu pogodbenu rečenicu, ovdje realnog oblika §~475; §~231; §~258
\item[ὡς πρὸς τὴν φύσιν] tehnički termin Seksta Empirika: u odnosu na prirodu (tj.\ ovdje: stvarno)
\item[ὅτι\dots\ φαίνεται] veznik uvodi zavisnu izričnu rečenicu; §~232; glagol nepotpuna značenja traži imensku dopunu
\item[τοιαῦτα, ὁποῖα\dots] odnosna zamjenica uvodi zavisnu odnosnu rečenicu, antecedent je τοιαῦτα
\item[ὁποῖα οὔκ έστιν] §~315; imenski predikat Smyth 909%imenski pred
\end{description}

%3

{\large
\begin{greek}
\noindent  ἀλλὰ καὶ πάντα ἐκ πάντων \\
\tabto{2em} καὶ ὡς ἔτυχεν \\
ἂν ἦν \\
\tabto{2em} μὴ οὔσης αἰτίας.\\

\end{greek}
}

\begin{description}[noitemsep]
\item[ὡς ἔτυχεν] veznik uvodi poredbenu rečenicu; §~254, osnove §~321.19; fraza LSJ τυγχάνω A.3
\item[ἂν ἦν] §~315; LSJ εἰμί A.II.2; ἄν + preterit izriče ireal, nezbiljnost §~462
\item[μὴ οὔσης] §~315; LSJ εἰμί A.II; participski dio GA (ovdje ekvivalent pogodbenoj rečenici)
\end{description}

%4

{\large
\begin{greek}
\noindent  οἷον \\
ἵπποι μὲν \\
\tabto{2em} ἐκ μυῶν, \\
\tabto{4em} εἰ τύχοι, \\
\tabto{2em} γεννηθήσονται, \\
ἐλέφαντες δὲ \\
\tabto{2em} ἐκ μυρμήκων· \\
καὶ ἐν μὲν ταῖς Αἰγυπτίαις Θήβαις \\
\tabto{2em} ὄμβροι \\
\tabto{4em} ποτὲ \\
\tabto{2em} ἐξαίσιοι \\
\tabto{2em} καὶ χιόνες \\
\tabto{4em} ἂν ἐγίνοντο, \\
τὰ δὲ νότια \\
\tabto{2em} ὄμβρων \\
οὐ μετεῖχεν, \\
εἰ μὴ αἰτία τις ἦν, \\
\tabto{2em} δι’ ἣν \\
\tabto{2em} τὰ μὲν νότιά \\
\tabto{4em} ἐστι δυσχείμερα, \\
\tabto{2em} αὐχμηρὰ δὲ \\
\tabto{4em} τὰ πρὸς τὴν ἔω.\\

\end{greek}
}

\begin{description}[noitemsep]
\item[οἷον] LSJ οἷος V.2.b%=npr.
\item[ἵπποι μὲν\dots\ ἐλέφαντες δὲ\dots] koordinacija rečeničnih članova parom čestica
\item[εἰ τύχοι] veznik uvodi pogodbenu rečenicu; §~254, osnove §~321.19; fraza LSJ τυγχάνω A.3
\item[γεννηθήσονται] §~296, §~298
\item[ἐν μὲν ταῖς Αἰγυπτίαις Θήβαις\dots\ τὰ δὲ νότια\dots] koordinacija rečeničnih članova parom čestica; Teba je slavni grad u Egiptu, oko 800 kilometara daleko od Sredozemnog mora, s hramovima Karnak i Luksor, u blizini pustinje; νότια su južni krajevi%obj. i realia
\item[ποτὲ] neodređeni prilog modificira predikat
\item[ἂν ἐγίνοντο] jonska i helenistička varijanta, usp.\ LSJ γίγνομαι; apodoza zavisne pogodbene rečenice irealnog oblika §~478
\item[οὐ μετεῖχεν] rekcija τινός; §~231, složenica ἔχω; kongruencija sa subjektom u pluralu srednjeg roda §~361; apodoza zavisne pogodbene rečenice irealnog oblika §~478
\item[εἰ μὴ\dots\ ἦν] §~315; LSJ εἰμί A.II; protaza zavisne pogodbene rečenice irealnog oblika §~478
\item[δι’ ἣν] uvodi zavisnu odnosnu rečenicu, antecedent je αἰτία; za značenje usp.\ gore διά τινα αἰτίαν
\item[τὰ μὲν νότιά\dots\ αὐχμηρὰ δὲ\dots] koordinacija rečeničnih članova parom čestica
\item[ἐστι δυσχείμερα] kongruencija sa subjektom u pluralu srednjeg roda §~361; imenski predikat Smyth 909
\item[αὐχμηρὰ] imenski dio imenskog predikata (kopula je izostavljena da se izbjegne ponavljanje) Smyth 909

\end{description}

%5

{\large
\begin{greek}
\noindent  καὶ περιτρέπεται δὲ ὁ λέγων \\
\tabto{2em} μηδὲν αἴτιον εἶναι· \\
\tabto{4em} εἰ μὲν γὰρ \\
\tabto{6em} ἁπλῶς \\
\tabto{6em} καὶ ἄνευ τινὸς αἰτίας \\
\tabto{6em} τοῦτό \\
\tabto{4em} φησι \\
\tabto{6em} λέγειν, \\
\tabto{4em} ἄπιστος ἔσται· \\
\tabto{4em} εἰ δὲ \\
\tabto{6em} διά τινα αἰτίαν, \\
\tabto{4em} βουλόμενος \\
\tabto{6em} ἀναιρεῖν \\
\tabto{6em} τὸ αἴτιον \\
\tabto{4em} τίθησιν, \\
\tabto{4em} ἀποδιδόσθω αἰτίαν \\
\tabto{6em} δι’ ἣν \\
\tabto{8em} οὐκ ἔστιν αἴτιον.\\

\end{greek}
}

\begin{description}[noitemsep]
\item[περιτρέπεται] §~232
\item[δὲ] kopulativno δέ (izvan koordinacije s μέν) označava prijelaz iz rečenice u rečenicu; sljedeća rečenica iskazuje nešto novo ili drugačije, ali ne suprotno prethodnom iskazu, Smyth §~2386
\item[ὁ λέγων] §~231; supstantiviranje participa članom §~499; zbog značenja glagolski oblik otvara mjesto A+I
\item[εἶναι] §~315; LSJ εἰμί A.II (kao u prvoj rečenici)
\item[εἰ μὲν γὰρ ἁπλῶς\dots\ εἰ δὲ διά τινα αἰτίαν\dots] koordinacija rečeničnih članova (i to zavisnih pogodbenih rečenica) parom čestica
\item[τοῦτό] sc.\ μηδὲν αἴτιον εἶναι
\item[φησι] §~312.8; \textit{verbum dicendi} otvara mjesto dopuni u infinitivu
\item[λέγειν] §~231
\item[ἄπιστος ἔσται] §~315; imenski predikat Smyth 909; apodoza zavisne pogodbene rečenice realnog oblika §~475
\item[βουλόμενος] §~232; glagol nepotpuna značenja otvara mjesto dopuni u infinitivu
\item[ἀναιρεῖν] §~243
\item[τίθησιν] sc.\ αὐτό; §~305; LSJ τίθημι B.II
\item[ἀποδιδόσθω] složenica δίδωμι §~305; apodoza pogodbene rečenice realnog oblika §~475
\item[δι’ ἣν] zamjenica uvodi zavisnu odnosnu rečenicu, antecedent je αἰτίαν (usp.\ sličan izraz u prethodnoj rečenici)
\item[οὐκ ἔστιν] §~315; LSJ εἰμί A.II
\end{description}



%kraj
