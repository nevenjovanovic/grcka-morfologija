% ispravio grčki tekst NJ 11. 5. 2020.
% unio ispravke NZ <2022-01-03 pon>

%TKTK


\section*{O tekstu}

Plutarhov esej iz zbirke Ἠθικά \textit{(Moralia)} bavi se vrlo specifičnom podvrstom pohlepe: iracionalnom željom za posjedovanjem golemog bogatstva. Autor ne hvali siromaštvo, već prikazuje loše strane bogatstva suprotstavljajući ga umjerenom imetku: \textgreek[variant=ancient]{οὐδὲν οὖν πλέον ἔχουσιν οἱ πλούσιοι τῶν μέτρια κεκτημένων} (gl. 8, 527B). Stil, jezik, teme i izvori djela pripadaju pretežno tradiciji kiničko-stoičke dijatribe mada Plutarh, naravno, navodi i mnoge druge autore, među njima Pindara, Aristofana, Platona, Aristotela.

Nakon što je u uvodu pokazao da bogatstvo ne može kupiti sreću, Plutarh analizira nesreću škrtaca i rasipnika: ni jedni ni drugi ne mogu zadovoljiti želju za imetkom, dok je kod škrtaca ta želja u sukobu s vlastitim zadovoljenjem. Zatim razmatra osobito pohlepne škrce i rasipnike, pri čemu potonje ocjenjuje manje nepodnošljivima. Dokazuje da je besmislen izgovor škrtaca da štede za svoje potomstvo, dok opravdanje bogataša da neki od njih društvu doprinose korištenjem bogatstva ovisi o shvaćanju pojma „korištenje”. Ako je svrha korištenja zadovoljavanje potreba, položaj bogataša i srednje imućnih je isti; ako „korištenje” znači luksuz, onda je bogatstvo naprosto razmetanje. Na koncu se takvo „teatralno” bogatstvo uspoređuje s filozofijom.

U odabranom odlomku Plutarh poseže za omiljenom stoičkom usporedbom nerazumnog čovjeka s bolesnikom, u ovom slučaju s onim čija je bolest duševna. Takav je i \textgreek[variant=ancient]{φιλόπλουτος.} Njegovo „psihičko siromaštvo” ne može se nasititi materijalnih dobara dok razumni ljudi znaju da je granica imetka zadovoljavanje (osnovnih) potreba \textgreek[variant=ancient]{(χρεία).}

%\newpage

\section*{Pročitajte naglas grčki tekst.}

Plut.\ De cupiditate divitiarum 524D-524F

%Naslov prema izdanju

\medskip


{\large

\begin{greek}

\noindent ὅταν ἰατρὸς πρὸς ἄνθρωπον εἰσελθὼν ἐρριμμένον ἐν τῷ κλινιδίῳ καὶ στένοντα καὶ μὴ βουλόμενον τροφὴν λαβεῖν, ἅψηται καὶ ἀνακρίνῃ καὶ εὕρῃ μὴ πυρέττοντα, ψυχικὴ νόσος ἔφη καὶ ἀπῆλθεν· οὐκοῦν καὶ ἡμεῖς ὅταν ἴδωμεν ἄνδρα τῷ πορισμῷ προστετηκότα καὶ τοῖς ἀναλώμασιν ἐπιστένοντα καὶ μηδενὸς εἰς χρηματισμὸν συντελοῦντος αἰσχροῦ μηδ᾽ ἀνιαροῦ φειδόμενον, οἰκίας δ᾽ ἔχοντα καὶ χώρας καὶ ἀγέλας καὶ ἀνδράποδα σὺν ἱματίοις, τί φήσομεν τὸ πάθος εἶναι τἀνθρώπου ἢ πενίαν ψυχικήν; ἐπεὶ τήν γε χρηματικήν, ὥς φησιν ὁ Μένανδρος, εἷς ἂν φίλος ἀπαλλάξειεν εὐεργετήσας· τὴν δὲ ψυχικὴν ἐκείνην οὐκ ἂν ἐμπλήσειαν ἅπαντες οὔτε ζῶντες οὔτε ἀποθανόντες. ὅθεν εὖ πρὸς τούτους λέλεκται ὑπὸ τοῦ Σόλωνος 
\begin{verse}
πλούτου δ᾽ οὐδὲν τέρμα πεφασμένον ἀνθρώποισιν.
\end{verse}
ἐπεὶ τοῖς γε νοῦν ἔχουσιν ὁ τῆς φύσεως πλοῦτος ὥρισται καὶ τὸ τέρμα πάρεστι τῇ χρείᾳ καθάπερ κέντρῳ καὶ διαστήματι περιγραφόμενον.

\end{greek}

}


\section*{Analiza i komentar}

%1

{\large
\begin{greek}
\noindent  ῞Οταν ἰατρὸς \\
\tabto{2em} εἰσελθὼν \\
\tabto{4em} πρὸς ἄνθρωπον \\
\tabto{6em} ἐρριμμένον \\
\tabto{8em} ἐν τῷ κλινιδίῳ \\
\tabto{6em} καὶ στένοντα \\
\tabto{6em} καὶ μὴ βουλόμενον \\
\tabto{8em} τροφὴν λαβεῖν \\
ἅψηται \\
καὶ ἀνακρίνῃ \\
καὶ εὕρῃ \\
\tabto{2em} μὴ πυρέττοντα,\\
‘ψυχικὴ νόσος’\\
\tabto{2em} ἔφη \\
καὶ ἀπῆλθεν·

\noindent οὐκοῦν καὶ ἡμεῖς \\
\tabto{2em} ὅταν ἴδωμεν \\
\tabto{4em} ἄνδρα \\
\tabto{6em} τῷ πορισμῷ \\
\tabto{4em} προστετηκότα \\
\tabto{4em} καὶ \\
\tabto{6em} τοῖς ἀναλώμασιν \\
\tabto{4em} ἐπιστένοντα \\
\tabto{4em} καὶ \\
\tabto{6em} μηδενὸς \\
\tabto{8em} εἰς χρηματισμὸν \\
\tabto{6em} συντελοῦντος \\
\tabto{6em} αἰσχροῦ μηδ' ἀνιαροῦ \\
\tabto{4em} φειδόμενον, \\
\tabto{4em} οἰκίας δ' \\
\tabto{6em} ἔχοντα\\
\tabto{4em} καὶ χώρας καὶ ἀγέλας \\
\tabto{4em} καὶ ἀνδράποδα \\
\tabto{6em} σὺν ἱματίοις,\\
τί φήσομεν \\
\tabto{2em} εἶναι \\
\tabto{4em} τοῦ ἀνθρώπου \\
\tabto{2em} τὸ πάθος \\
\tabto{4em} ἢ πενίαν ψυχικήν;\\

\end{greek}
}

\begin{description}[noitemsep]
\item[῞Οταν] uvodi zavisnu vremensku rečenicu s načinima eventualne protaze §~488.2: kad god\dots
\item[εἰσελθὼν] složenica ἔρχομαι, LSJ εἰσέρχομαι πρός τινα ući u čiju kuću, posjetiti ga; o liječniku, doći u kućnu posjetu
\item[ἐρριμμένον] s.~118, §~286, LSJ ῥίπτω A
\item[στένοντα] §~231
\item[μὴ] uz participe Smyth 2045
\item[βουλόμενον] §~232, §~325.13; glagol nepotpuna značenja otvara mjesto infinitivu
\item[λαβεῖν] §~254, §~321.14
\item[ἅψηται] §~267, LSJ ἅπτω A.II
\item[ἀνακρίνῃ] složenica κρίνω, s.~118, §~267
\item[εὕρῃ] §~254, §~324.7
\item[μὴ] uz participe Smyth 2045
\item[πυρέττοντα] predikatni particip uz εὕρῃ; §~231
\item[ἔφη] §~312.8
\item[ἀπῆλθεν] složenica ἔρχομαι, §~327.2, §~254
\item[οὐκοῦν] zaključno: dakle, zar ne\dots?
\item[ὅταν] uvodi zavisnu vremensku rečenicu s načinima eventualne protaze §~488.2: kad god\dots
\item[ἴδωμεν] §~327.3, §~254
\item[προστετηκότα] LSJ προστήκομαί τινι, u pasivnom značenju, perfekt προστέτηκα, metaforički: posvetiti se čemu, biti zaokupljen čime, πορισμῷ Plu. 2.524d (to je upravo ovo mjesto); složenica τήκω, §~272, aktivni oblici samo u perfektu; rekcija
\item[ἐπιστένοντα] složenica στένω, §~231; rekcija τινί
\item[συντελοῦντος] složenica τελέω, LSJ συντελέω A.II.2; rekcija εἴς τι, §~231, §~243
\item[φειδόμενον] rekcija τινός, §~232, LSJ φείδομαι A.IV
\item[καὶ μηδενὸς\dots] \textbf{φειδόμενον, οἰκίας δ' ἔχοντα} koordinacija pomoću čestice koja označava suprotnost prethodnom navodu
\item[ἔχοντα] §~231
\item[φήσομεν] §~312.8; \textit{verbum dicendi} otvara mjesto A+I
\item[εἶναι] §~315
\item[τί\dots\ ἢ] što (drugo) nego\dots? (Smyth 2863.a)
\item[φήσομεν εἶναι] \textbf{τὸ πάθος} §~491; imenski predikat, Smyth 909

\end{description}

%2

{\large
\begin{greek}
\noindent  ἐπεὶ τήν γε χρηματικήν, \\
\tabto{2em} ὥς φησιν ὁ Μένανδρος,  \\
εἷς ἂν φίλος \\
ἀπαλλάξειεν \\
\tabto{2em} εὐεργετήσας, \\
τὴν δὲ ψυχικὴν ἐκείνην \\
\tabto{2em} οὐκ ἂν ἐμπλήσειαν \\
\tabto{4em} ἅπαντες \\
\tabto{6em} οὔτε ζῶντες \\
\tabto{6em} οὔτ' ἀποθανόντες.\\

\end{greek}
}

\begin{description}[noitemsep]
\item[ἐπεὶ] veznik uvodi uzročnu zavisnu rečenicu §~468
\item[γε] čestica stoji između člana i imenice, Smyth 2823; skreće fokus na izraz u koji je uklopljena
\item[ὥς] veznik uvodi poredbenu zavisnu rečenicu s indikativom, Smyth 2475
\item[φησιν] §~312.8
\item[ὁ Μένανδρος] grčki komediograf (Atena, 343/342.\ – 292/291.\ pr.~Kr.), najveći predstavnik nove atičke komedije, čije su gnome („mudre misli”) često citirane u djelima kasnijih autora; član uz vlastita imena u značenju: onaj poznati\dots\ §~374 bilj.
\item[ἂν] čestica uz optativ kao potencijal sadašnji §~464
\item[ἀπαλλάξειεν] složenica ἀλλάσσω, §~267, §~269
\item[εὐεργετήσας] §~267, §~269
\item[τήν γε χρηματικήν\dots] \textbf{τὴν δὲ ψυχικὴν\dots} koordinacija pomoću čestice δὲ, koja označava suprotnost rečeničnih članova
\item[τήν γε χρηματικήν] sc.\ πενίαν
\item[ἂν] čestica uz optativ kao potencijal sadašnji §~464
\item[ἐμπλήσειαν] složenica πίμπλημι, §~312.2, §~267
\item[οὔτε\dots\ οὔτ'\dots] koordinacija rečeničnih članova (participa) sastavnim veznicima §~513.4
\item[ζῶντες] §~243, §~231
\item[ἀποθανόντες] §~324.8

\end{description}

%3

{\large
\begin{greek}
\noindent  ὅθεν εὖ \\
\tabto{2em} πρὸς τούτους \\
λέλεκται \\
\tabto{2em} ὑπὸ τοῦ Σόλωνος \\
‘πλούτου δ' \\
\tabto{2em} οὐδὲν τέρμα \\
\tabto{4em} πεφασμένον \\
\tabto{6em} ἀνθρώποισιν.’\\

\end{greek}
}

\begin{description}[noitemsep]
\item[ὅθεν] LSJ s. v. II: iz tog razloga\dots
\item[πρὸς τούτους] sc.\ ἄνδρας τῷ πορισμῷ προστετηκότας
\item[λέλεκται] §~327.7, §~272
\item[τοῦ Σόλωνος] Solon (oko 638. – oko 558. pr.~Kr.), atenski državnik, zakonodavac i pjesnik, jedan od sedam mudraca; član uz vlastita imena u značenju: onaj poznati\dots\ §~374 bilj.
\item[δ'] označava nadovezivanje na prethodni navod (u citatu izostavljen)
\item[πεφασμένον] sc.\ ἐστί; s.~118, §~286

\end{description}


{\large
\begin{greek}
\noindent  ἐπεὶ \\
\tabto{2em} τοῖς γε νοῦν ἔχουσιν \\
ὁ \\
\tabto{2em} τῆς φύσεως \\
πλοῦτος \\
ὥρισται\\
καὶ τὸ τέρμα \\
πάρεστι, \\
\tabto{2em} τῇ χρείᾳ \\
\tabto{4em} καθάπερ κέντρῳ καὶ διαστήματι \\
περιγραφόμενον.\\

\end{greek}
}

\begin{description}[noitemsep]
\item[ἐπεὶ] veznik uvodi uzročnu zavisnu rečenicu §~468
\item[τοῖς\dots\ ἔχουσιν] supstantivirani particip, §~327.13, dativ koristi §~412.1
\item[γε] stoji između člana i supstantiviranog participa (i njegovog objekta), Smyth 2823, skreće pozornost na taj dio rečenice
\item[ὥρισται] §~275, §~284; LSJ ὁρίζω IV.3, rekcija τινί
\item[πάρεστι] složenica εἰμί §~315
\item[τῇ χρείᾳ\dots] \textbf{κέντρῳ καὶ διαστήματι} instrumentalni dativ §~414
\item[περιγραφόμενον] složenica γράφω, LSJ περιγράφω A.2, §~232

\end{description}




%kraj
