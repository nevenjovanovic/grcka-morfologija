% Unesi ispravke NZ <2022-01-05 sri>

\section*{O tekstu}

Ulomak koji čitamo dio je Aristotelova spisa \textit{O pjesničkom umijeću} koji detaljno razmatra tragediju (6–22.\ poglavlje). Filozof najprije predstavlja tragediju i njezinih ključnih šest dijelova, a zatim objašnjava značaj njezina jedinstva ili cjelovitosti. Nakon toga izlaže neke uobičajene pogreške u shvaćanju njezina pojma.
 
U ovom odlomku Aristotel na osnovi prethodnih određenja uspoređuje povjesničara i pjesnika. Razlika nije u formi, tj.\ između stiha i proze \textgreek[variant=ancient]{(ἔμμετρα λέγειν ἢ ἄμετρα):} stavimo li Herodotovu povijest u stihove, ona će i dalje ostati povijest. Metrička organizacija teksta samo je površinska odlika. Prema Aristotelu, pjesništvo se može baviti bilo čime, dok se povijest može baviti jedino onime što se dogodilo. Drugim riječima, pjesništvo je posvećeno općem, a povijest pojedinačnom.

%\newpage

\section*{Pročitajte naglas grčki tekst.}

Arist.\ Poetica 1451b

%Naslov prema izdanju

\medskip


{\large

\begin{greek}

\noindent  Ὁ γὰρ ἱστορικὸς καὶ ὁ ποιητὴς οὐ τῷ ἢ ἔμμετρα λέγειν ἢ ἄμετρα διαφέρουσιν· εἴη γὰρ ἂν τὰ Ἡροδότου εἰς μέτρα τεθῆναι, καὶ οὐδὲν ἧττον ἂν εἴη ἱστορία τις μετὰ μέτρου ἢ ἄνευ μέτρων· ἀλλὰ τούτῳ διαφέρει, τῷ τὸν μὲν τὰ γενόμενα λέγειν, τὸν δὲ οἷα ἂν γένοιτο.  Διὸ καὶ φιλοσοφώτερον καὶ σπουδαιότερον ποίησις ἱστορίας ἐστίν· ἡ μὲν γὰρ ποίησις μᾶλλον τὰ καθόλου, ἡ δ᾿ ἱστορία τὰ καθ᾿ ἕκαστον λέγει. Ἔστι δὲ καθόλου μέν, τῷ ποίῳ τὰ ποῖ᾿ ἄττα συμβαίνει λέγειν ἢ πράττειν κατὰ τὸ εἰκὸς ἢ τὸ ἀναγκαῖον, οὗ στοχάζεται ἡ ποίησις ὀνόματα ἐπιτιθεμένη· τὰ δὲ καθ᾿ ἕκαστον, τί Ἀλκιβιάδης ἔπραξεν ἢ τί ἔπαθεν. Ἐπὶ μὲν οὖν τῆς κωμῳδίας ἤδη τοῦτο δῆλον γέγονεν· συστήσαντες γὰρ τὸν μῦθον διὰ τῶν εἰκότων οὕτω τὰ τυχόντα ὀνόματα ἐπιτιθέασι, καὶ οὐχ ὥσπερ οἱ ἰαμβοποιοὶ περὶ τῶν καθ᾿ ἕκαστον ποιοῦσιν. Ἐπὶ δὲ τῆς τραγῳδίας τῶν γενομένων ὀνομάτων ἀντέχονται. Αἴτιον δ᾿ ὅτι πιθανόν ἐστι τὸ δυνατόν· τὰ μὲν οὖν μὴ γενόμενα οὔπω πιστεύομεν εἶναι δυνατά, τὰ δὲ γενόμενα φανερὸν ὅτι δυνατά· οὐ γὰρ ἂν ἐγένετο, εἰ ἦν ἀδύνατα.

\end{greek}

}


\section*{Analiza i komentar}

%1

{\large
\begin{greek}
\noindent Ὁ γὰρ ἱστορικὸς καὶ ὁ ποιητὴς \\
\tabto{2em} οὐ τῷ\\
\tabto{4em} ἢ ἔμμετρα λέγειν \\
\tabto{4em} ἢ ἄμετρα \\
\tabto{2em} διαφέρουσιν· \\
εἴη γὰρ ἂν \\
\tabto{2em} τὰ ῾Ηροδότου \\
\tabto{2em} εἰς μέτρα τεθῆναι \\
καὶ οὐδὲν ἧττον ἂν εἴη \\
\tabto{2em} ἱστορία τις \\
\tabto{4em} μετὰ μέτρου \\
\tabto{4em} ἢ ἄνευ μέτρων· \\
ἀλλὰ τούτῳ διαφέρει, \\
\tabto{2em} τῷ \\
\tabto{4em} τὸν μὲν \\
\tabto{6em} τὰ γενόμενα λέγειν, \\
\tabto{4em} τὸν δὲ \\
\tabto{6em} οἷα ἂν γένοιτο.\\

\end{greek}
}

\begin{description}[noitemsep]
\item[γὰρ] čestica označava eksplanatorni smisao ove rečenice; rečenica izriče objašnjenje prethodne tvrdnje: naime\dots
\item[τῷ\dots\ λέγειν] §~231, supstantivirani infinitiv §~497
\item[διαφέρουσιν] διαφέρω τινί; §~231; složenica glagola φέρω
\item[εἴη\dots\ ἂν] §~315.2; glagol εἰμί u svezi s modalnom česticom: moguće je (ἄν + optativ izriče mogućnost §~464.2); otvara mjesto dopuni u infinitivu
\item[τὰ ῾Ηροδότου] sc.\ ἔπη ili λεγόμενα
\item[τεθῆναι] §~311
\item[ἂν εἴη] §~315.2, subjekt je τὰ ῾Ηροδότου (kongruencija sa subjektom u pluralu srednjeg roda §~361); kopula otvara mjesto imenskom dijelu imenskog predikata; ἄν + optativ usp.\ gore
\item[διαφέρει] usp.\ gore
\item[τούτῳ\dots\ τῷ\dots] korelativno;  §~219
\item[τὸν μὲν\dots\ τὸν δὲ\dots] sc.\ \textgreek{τὸν ἱστορικόν\dots\ τὸν ποιητήν;} koordinacija rečeničnih članova parom čestica; subjektni dio A+I, ovisni o λέγειν
\item[λέγειν] infinitiv supstantiviran članom τῷ 
\item[τὰ γενόμενα] §~254, supstantivirani particip §~499
\item[λέγειν] §~231
\item[τὸν δὲ] sc.\ λέγειν
\item[οἷα ἂν γένοιτο] odnosna zamjenica, ujedno objekt \textgreek{λέγειν,} uvodi zavisnu odnosnu rečenicu: ono što\dots; ἄν + optativ usp.\ gore
\item[γένοιτο] §~254

\end{description}

%2

{\large
\begin{greek}
\noindent  Διὸ καὶ φιλοσοφώτερον καὶ σπουδαιότερον \\
ποίησις \\
\tabto{2em} ἱστορίας \\
ἐστίν· \\
ἡ μὲν γὰρ ποίησις \\
\tabto{2em} μᾶλλον τὰ καθόλου, \\
ἡ δ' ἱστορία \\
\tabto{2em} τὰ καθ' ἕκαστον \\
λέγει. \\

\end{greek}
}

\begin{description}[noitemsep]
\item[ἐστίν] §~315.2, kopula, glagolski dio imenskoga predikata, otvara mjesto dopuni, ovdje paru pridjeva
\item[καὶ φιλοσοφώτερον καὶ σπουδαιότερον] sc.\ τι\dots; komparativi otvaraju mjesto dopuni u genitivu
\item[ἡ μὲν\dots\ ποίησις\dots, ἡ δ' ἱστορία] koordinacija (suprotstavljenih) rečeničnih članova parom čestica
\item[τὰ καθόλου\dots\ τὰ καθ' ἕκαστον] supstantiviranje članom §~373
\item[λέγει] §~231
\end{description}

%3

{\large
\begin{greek}
\noindent  Ἔστι δὲ \\
καθόλου μέν, \\
\tabto{2em} τῷ ποίῳ \\
\tabto{2em} τὰ ποῖ' ἄττα \\
\tabto{4em} συμβαίνει \\
\tabto{6em} λέγειν ἢ πράττειν \\
\tabto{6em} κατὰ τὸ εἰκὸς \\
\tabto{6em} ἢ τὸ ἀναγκαῖον, \\
\tabto{2em} οὗ στοχάζεται \\
\tabto{2em} ἡ ποίησις \\
\tabto{4em} ὀνόματα \\
\tabto{2em} ἐπιτιθεμένη· \\
τὸ δὲ καθ' ἕκαστον, \\
\tabto{2em} τί ᾿Αλκιβιάδης ἔπραξεν \\
\tabto{2em} ἢ τί ἔπαθεν.\\

\end{greek}
}

\begin{description}[noitemsep]
\item[ἔστιν] §~315.2, kopula otvara mjesto imenskoj dopuni, τὰ ποῖ᾿\dots
\item[καθόλου μέν\dots\ τὸ δὲ καθ' ἕκαστον] koordinacija (suprotstavljenih) rečeničnih članova parom čestica
\item[τῷ ποίῳ] sc.\ ἀνθρώπῳ
\item[τὰ ποῖ᾿ ἄττα] dvije upitne rečenice uvedene upitnom korelativnom zamjenicom ποῖος koja se može javiti u kombinaciji s članom i neodređenom zamjenicom, LSJ ποῖος I.3 i I.4
\item[συμβαίνει] τινί s infinitivom, LSJ συμβαίνω III; §~231
\item[λέγειν\dots\ πράττειν] §~231; dopune glagola συμβαίνει
\item[οὗ] odnosna zamjenica uvodi zavisnu odnosnu rečenicu, njezin je antecedent (τὸ) καθόλου
\item[στοχάζεται ] rekcija τινος; §~231
\item[ἐπιτιθεμένη] §~305, složenica τίθημι
\item[τὸ δὲ καθ' ἕκαστον] supstantiviranje članom §~373; nakon subjekta neizrečena kopula ἐστιν (da se izbjegne ponavljanje); imenski dio predikata obje su upitne rečenice
\item[τί] upitna zamjenica uvodi upitnu rečenicu
\item[ἔπραξεν] §~267
\item[ἔπαθεν] §~254

\end{description}

%4

{\large
\begin{greek}
\noindent  Ἐπὶ μὲν οὖν τῆς κωμῳδίας \\
ἤδη \\
τοῦτο \\
δῆλον γέγονεν·\\
συστήσαντες γὰρ τὸν μῦθον \\
\tabto{2em} διὰ τῶν εἰκότων \\
\tabto{2em} οὕτω \\
\tabto{2em} τὰ τυχόντα ὀνόματα \\
\tabto{2em} ὑποτιθέασιν, \\
\tabto{2em} καὶ οὐχ ὥσπερ οἱ ἰαμβοποιοὶ \\
\tabto{6em} περὶ τὸν καθ' ἕκαστον \\
\tabto{4em} ποιοῦσιν.\\

\end{greek}
}

\begin{description}[noitemsep]
\item[ἐπὶ μὲν οὖν τῆς κωμῳδίας\dots] \textbf{ἐπὶ δὲ τῆς τραγῳδίας\dots}\ čestica koordinira ovu rečenicu s idućom
\item[τοῦτο] sc.\ τὸ καθόλου
\item[γέγονεν] §~272, kopulativni glagol otvara mjesto imenskoj dopuni, ovdje pridjevu
\item[γὰρ] čestica uvodi objašnjenje: naime\dots
\item[συστήσαντες] §~267, složenica ἵστημι
\item[τῶν εἰκότων] §~327.b
\item[τυχόντα] §~254; LSJ τυγχάνω A.2.b
\item[ὑποτιθέασιν] §~305, složenica τίθημι
\item[ὥσπερ\dots\ ποιοῦσιν] veznik uvodi zavisnu rečenicu koju možemo prevesti načinskom ili poredbenom
\item[περὶ τὸν καθ' ἕκαστον] sc.\ ἄνθρωπον; Aristotel misli na Novu komediju gdje autor izmišlja imena likova bez ikakva referiranja na stvarne osobe (za razliku od Aristofanove komedije)
\item[ποιοῦσιν] §~243

\end{description}

%5

{\large
\begin{greek}
\noindent  Ἐπὶ δὲ τῆς τραγῳδίας \\
\tabto{2em} τῶν γενομένων ὀνομάτων \\
ἀντέχονται.\\

\end{greek}
}

\begin{description}[noitemsep]
\item[ἐπὶ δὲ τῆς τραγῳδίας ] koordinacija česticom povezuje rečenicu s prethodnom
\item[τῶν γενομένων] §~254; supstantivirani particip §~499
\item[ἀντέχονται] ἀντέχομαί τινος, LSJ ἀντέχω III.2; §~231

\end{description}

%6

{\large
\begin{greek}
\noindent  Αἴτιον δ' ὅτι \\
\tabto{2em} πιθανόν ἐστι \\
\tabto{4em} τὸ δυνατόν· \\
τὰ μὲν οὖν μὴ γενόμενα \\
οὔπω πιστεύομεν \\
\tabto{2em} εἶναι δυνατά, \\
τὰ δὲ γενόμενα \\
φανερὸν \\
\tabto{2em} ὅτι δυνατά· \\
\tabto{2em} οὐ γὰρ ἂν ἐγένετο, \\
\tabto{2em} εἰ ἦν ἀδύνατα.\\

\end{greek}
}

\begin{description}[noitemsep]
\item[αἴτιον] imenski dio predikata (kopula je neizrečena)
\item[ὅτι] veznik uvodi zavisnu izričnu rečenicu sa službom subjekta
\item[ἐστι] §~315.2, kopula imenskog predikata otvara mjesto imenskoj dopuni, ovdje pridjevu 
\item[τὰ μὲν οὖν μὴ γενόμενα\dots] \textbf{τὰ δὲ γενόμενα\dots}\ koordinacija rečeničnih članova parom čestica
\item[τὰ\dots\ γενόμενα] §~254, supstantivirani particip §~499
\item[πιστεύομεν] §~231, \textit{verbum sentiendi} otvara mjesto A+I, odgovara hrvatskoj zavisnoj izričnoj rečenici
\item[εἶναι] §~315.2, kopula imenskog predikata, otvara mjesto imenskoj dopuni
\item[φανερὸν] imenski dio imenskog predikata, neizrečena je kopula ἐστίν
\item[ὅτι] izrični veznik uvodi zavisnu izričnu rečenicu
\item[δυνατά] sc. εἶναι (kopula je neizrečena), odgovara gornjem \textgreek{εἶναι δυνατά}
\item[γὰρ] čestica uvodi objašnjenje: naime\dots
\item[ἂν ἐγένετο] §~254, apodoza (glavna rečenica) irealne pogodbene rečenice; kongruencija sa subjektom u pluralu srednjeg roda §~361
\item[εἰ ἦν ἀδύνατα] pogodbeni veznik uvodi irealnu protazu (zavisna rečenica pogodbenog perioda); kongruencija sa subjektom u pluralu srednjeg roda kao gore

\end{description}


%kraj
