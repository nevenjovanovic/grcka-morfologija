% Unesi ispravke NZ <2022-01-02 ned>

\section*{O tekstu}

\textit{Država} je Platonov dijalog u deset knjiga nastao oko 380. Krećući od jednostavnog pitanja ``je li uvijek bolje biti pravedan nego nepravedan?'', djelo se bavi temama pravednosti (δικαιοσύνη), poretka u pravičnom gradu-državi, i pravednog čovjeka, ali i problemima pjesništva. \textit{Država} je jedno od najpoznatijih i najutjecajnijih Platonovih djela.

U drugoj knjizi dijaloga, Glaukon, Platonov brat i jedan od sudionika razgovora, govori o pretku lidijskog kralja Giga, pastiru koji je našao čudesni prsten i pomoću njega postao kralj. Glaukon se mitom koristi da bi Sokratu postavio pitanje može li čovjek biti dovoljno valjan da odoli iskušenjima čak i u slučaju kad se ne mora bojati da će ga itko otkriti.


%\newpage

\section*{Pročitajte naglas grčki tekst.}

Plat.\ Respublica 359d-360b

%Naslov prema izdanju

\medskip


{\large

\begin{greek}

  \noindent εἶναι μὲν γὰρ αὐτὸν ποιμένα θητεύοντα παρὰ τῷ τότε Λυδίας ἄρχοντι, ὄμβρου δὲ πολλοῦ γενομένου καὶ σεισμοῦ ῥαγῆναί τι τῆς γῆς καὶ γενέσθαι χάσμα κατὰ τὸν τόπον ᾗ ἔνεμεν. ἰδόντα δὲ καὶ θαυμάσαντα καταβῆναι καὶ ἰδεῖν ἄλλα τε δὴ ἃ μυθολογοῦσιν θαυμαστὰ καὶ ἵππον χαλκοῦν, κοῖλον, θυρίδας ἔχοντα, καθ´ ἃς ἐγκύψαντα ἰδεῖν ἐνόντα νεκρόν ὡς φαίνεσθαι, μείζω ἢ κατ´ ἄνθρωπον, τοῦτον δὲ ἄλλο μὲν ἔχειν οὐδέν, περὶ δὲ τῇ χειρὶ χρυσοῦν δακτύλιον, ὃν περιελόμενον ἐκβῆναι. συλλόγου δὲ γενομένου τοῖς ποιμέσιν εἰωθότος, ἵν´ ἐξαγγέλλοιεν κατὰ μῆνα τῷ βασιλεῖ τὰ περὶ τὰ ποίμνια, ἀφικέσθαι καὶ ἐκεῖνον ἔχοντα τὸν δακτύλιον· καθήμενον οὖν μετὰ τῶν ἄλλων τυχεῖν τὴν σφενδόνην τοῦ δακτυλίου περιαγαγόντα πρὸς ἑαυτὸν εἰς τὸ εἴσω τῆς χειρός, τούτου δὲ γενομένου ἀφανῆ αὐτὸν γενέσθαι τοῖς παρακαθημένοις, καὶ διαλέγεσθαι ὡς περὶ οἰχομένου. καὶ τὸν θαυμάζειν τε καὶ πάλιν ἐπιψηλαφῶντα τὸν δακτύλιον στρέψαι ἔξω τὴν σφενδόνην, καὶ στρέψαντα φανερὸν γενέσθαι. καὶ τοῦτο ἐννοήσαντα ἀποπειρᾶσθαι τοῦ δακτυλίου εἰ ταύτην ἔχοι τὴν δύναμιν, καὶ αὐτῷ οὕτω συμβαίνειν, στρέφοντι μὲν εἴσω τὴν σφενδόνην ἀδήλῳ γίγνεσθαι, ἔξω δὲ δήλῳ· αἰσθόμενον δὲ εὐθὺς διαπράξασθαι τῶν ἀγγέλων γενέσθαι τῶν παρὰ τὸν βασιλέα, ἐλθόντα δὲ καὶ τὴν γυναῖκα αὐτοῦ μοιχεύσαντα, μετ´ ἐκείνης ἐπιθέμενον τῷ βασιλεῖ ἀποκτεῖναι καὶ τὴν ἀρχὴν οὕτω κατασχεῖν.

\end{greek}

}


\section*{Analiza i komentar}

%1

{\large
\begin{greek}
\noindent εἶναι μὲν γὰρ \\
αὐτὸν \\
\tabto{2em} ποιμένα θητεύοντα \\
\tabto{2em} παρὰ τῷ \\
\tabto{4em} τότε \\
\tabto{4em} Λυδίας \\
\tabto{2em} ἄρχοντι, \\
ὄμβρου δὲ πολλοῦ γενομένου καὶ σεισμοῦ \\
ῥαγῆναί τι \\
\tabto{2em} τῆς γῆς \\
καὶ γενέσθαι χάσμα \\
\tabto{2em} κατὰ τὸν τόπον \\
\tabto{4em} ᾗ ἔνεμεν.\\
		
\end{greek}
}

\begin{description}[noitemsep]
\item[εἶναι] §~315
\item[θητεύοντα] §~231
\item[εἶναι\dots\ αὐτὸν ποιμένα θητεύοντα] §~491, A+I ovisi o glagolu govorenja iz prethodne rečenice, kao i svi sljedeći infinitivi u tekstu ako nije drugačije naznačeno; εἶναι se može prevesti prošlim nesvršenim vremenom; imenski predikat, Smyth 910
\item[τότε] prilog u službi atributa i u atributnom položaju
\item[εἶναι μὲν\dots\ ὄμβρου δὲ πολλοῦ\dots] koordinacija parom čestica, ovdje označava nešto slabiji kontrast (Smyth 2904) 
\item[γὰρ] uvodi objašnjenje
\item[ἄρχοντι] §~231, rekcija τινος
\item[ὄμβρου\dots\ πολλοῦ γενομένου καὶ σεισμοῦ] §~504, §~325.11; GA
\item[ῥαγῆναί] §~292, §~319.14
\item[ῥαγῆναί τι] §~491
\item[τῆς γῆς] genitiv partitivni ovisan o τι §~395
\item[γενέσθαι] §~325.11
\item[γενέσθαι χάσμα] §~491 
\item[ᾗ] relativni prilog uvodi zavisnu relativnu rečenicu, antecedent je τὸν τόπον §~481
\item[ἔνεμεν] §~231
\end{description}

%2
{\large
\begin{greek}
\noindent ἰδόντα δὲ καὶ θαυμάσαντα\\
καταβῆναι καὶ ἰδεῖν\\
ἄλλα τε δὴ\\
\tabto{2em} ἃ μυθολογοῦσιν θαυμαστὰ\\
καὶ ἵππον\\
\tabto{2em} χαλκοῦν, κοῖλον,\\
\tabto{2em} θυρίδας ἔχοντα,\\
\tabto{4em} καθ' ἃς\\
\tabto{4em} ἐγκύψαντα\\
\tabto{4em} ἰδεῖν\\
\tabto{4em} ἐνόντα νεκρόν ὡς φαίνεσθαι,\\
\tabto{4em} μείζω\\
\tabto{6em} ἢ κατ' ἄνθρωπον,\\
\tabto{4em} τοῦτον δὲ\\
\tabto{6em} ἄλλο μὲν ἔχειν οὐδέν,\\
\tabto{6em} περὶ δὲ τῇ χειρὶ\\
\tabto{6em} χρυσοῦν δακτύλιον\\
\tabto{4em} ὃν περιελόμενον\\
\tabto{4em} ἐκβῆναι.\\

\end{greek}
}

\begin{description}[noitemsep]
\item[δὲ] označava nadovezivanje na prethodni navod
\item[ἰδόντα\dots] \textbf{\textgreek[variant=ancient]{καὶ θαυμάσαντα καταβῆναι καὶ ἰδεῖν\dots\ ἐγκύψαντα ἰδεῖν, τοῦτον\dots\ περιελόμενον ἐκβῆναι}} §~491
\item[ἰδόντα] §~327.3
\item[θαυμάσαντα] §~267
\item[καταβῆναι] složenica βαίνω, §~321.6
\item[ἰδεῖν] §~327.3
\item[ἄλλα τε\dots\ καὶ ἵππον\dots] koordinacija pomoću para sastavnih veznika, pri čemu je drugi član jače istaknut
\item[δὴ] čestica označava nastavak pripovijedanja, LSJ IV.2
\item[ἃ] uvodi (umetnutu) zavisnu relativnu rečenicu, antecedent je ἄλλα, §~481
\item[μυθολογοῦσιν] §~243, LSJ μυθολογέω A.2
\item[ἄλλα\dots\ θαυμαστὰ] objekt infinitiva ἰδεῖν
\item[ἔχοντα] §~327.13, §~231
\item[καθ' ἃς] uvodi zavisnu relativnu rečenicu, antecedent je θυρίδας, §~481
\item[ἐγκύψαντα] subjekt A+I uz ἰδεῖν (ovisno o udaljenom \textit{verbum dicendi}); složenica κύπτω, §~267
\item[ἰδεῖν] §~327.3
\item[ἐνόντα] složenica εἰμί, §~315
\item[ὡς φαίνεσθαι] odnosi se na νεκρόν; priča ne kaže je li čovjek bio zaista mrtav
\item[δὲ] označava nadovezivanje na prethodni navod
\item[ἄλλο μὲν\dots\ περὶ δὲ\dots] koordinacija parom čestica označava suprotnost
\item[ἔχειν] u značenju ``nositi''; subjektni je akuzativ τοῦτον (tj. νεκρόν)
\item[ὄντα] §~315
\item[περιελόμενον ἐκβῆναι] promjena subjekta
\item[περιελόμενον] složenica αἱρέω, §~327.1, rekcija τινά
\item[ἐκβῆναι] složenica βαίνω §~321.6

\end{description}

%3
{\large
\begin{greek}
\noindent συλλόγου δὲ γενομένου \\
\tabto{2em} τοῖς ποιμέσιν \\
εἰωθότος, \\
\tabto{2em} ἵν' ἐξαγγέλλοιεν \\
\tabto{4em} κατὰ μῆνα \\
\tabto{4em} τῷ βασιλεῖ \\
\tabto{2em} τὰ περὶ τὰ ποίμνια, \\
ἀφικέσθαι καὶ ἐκεῖνον \\
\tabto{2em} ἔχοντα \\
\tabto{4em} τὸν δακτύλιον· \\
καθήμενον οὖν \\
\tabto{2em} μετὰ τῶν ἄλλων \\
τυχεῖν \\
τὴν σφενδόνην \\
\tabto{2em} τοῦ δακτυλίου \\
περιαγαγόντα \\
\tabto{2em} πρὸς ἑαυτὸν \\
\tabto{2em} εἰς τὸ εἴσω \\
\tabto{4em} τῆς χειρός, \\
τούτου δὲ γενομένου\\
\tabto{2em} ἀφανῆ \\
\tabto{4em} αὐτὸν \\
\tabto{2em} γενέσθαι \\
\tabto{4em} τοῖς παρακαθημένοις, \\
\tabto{2em} καὶ διαλέγεσθαι\\
\tabto{4em} ὡς περὶ οἰχομένου. \\

\end{greek}
}

\begin{description}[noitemsep]
\item[δὲ] označava nadovezivanje na prethodni navod
\item[γενομένου] §~325.11, §~254
\item[εἰωθότος] §~278.2 bilješka, §~327b
\item[συλλόγου\dots\ γενομένου\dots\ εἰωθότος ] GA, §~504
\item[ἵν'] uvodi namjernu rečenicu §~470
\item[ἐξαγγέλλοιεν] složenica ἀγγέλλω, §~231
\item[τὰ περὶ τὰ ποίμνια] supstantivirani prijedložni izraz
\item[ἀφικέσθαι] §~321.8, §~254
\item[ἔχοντα] u značenju φέροντα; §~327.13, §~231
\item[ἀφικέσθαι\dots\ ἐκεῖνον\dots\ ἔχοντα] §~491
\item[καθήμενον] §~315.4
\item[οὖν] označava nastavak pripovijedanja
\item[τυχεῖν] §~321.19, §~254; otvara mjesto predikatnom participu περιαγαγόντα §~501.b
\item[τὴν σφενδόνην] okvir prstena; LSJ σφενδόνη II.3 obruč na prstenu u koji je kamen bio umetnut kao u praćku
\item[περιαγαγόντα] subjekt A+I (uz τυχεῖν); složenica ἄγω s.~116, §~257
\item[καθήμενον\dots\ τυχεῖν\dots\ περιαγαγόντα] §~491
\item[δὲ] označava nadovezivanje na prethodni navod
\item[γενομένου] §~325.11, §~254
\item[τούτου\dots\ γενομένου] §~504
\item[γενέσθαι] §~325.11, u funkciji kopule, s pridjevom kao predikatnom dopunom, Smyth 917
\item[ἀφανῆ αὐτὸν γενέσθαι] §~491
\item[τοῖς παρακαθημένοις] složenica κάθημαι, §~315.4, supstantivirani particip; ovisan o ἀφανῆ
\item[διαλέγεσθαι] sc.\ αὐτούς (παρακαθημένους); složenica λέγω, §~232, §~491
\item[οἰχομένου] §~232
\end{description}

%4

{\large
\begin{greek}
\noindent καὶ τὸν \\
\tabto{2em} θαυμάζειν τε \\
\tabto{2em} καὶ πάλιν ἐπιψηλαφῶντα \\
\tabto{4em} τὸν δακτύλιον \\
\tabto{2em} στρέψαι \\
\tabto{2em} ἔξω \\
\tabto{2em} τὴν σφενδόνην, \\
\tabto{2em} καὶ στρέψαντα \\
\tabto{2em} φανερὸν γενέσθαι. \\

\end{greek}
}

\begin{description}[noitemsep]
\item[τὸν] član kao pokazna zamjenica na početku rečenice, §~370.2
\item[θαυμάζειν τε] \textbf{καὶ\dots\ ἐπιψηλαφῶντα\dots\ στρέψαι\dots, καὶ στρέψαντα φανερὸν γενέσθαι} §~491
\item[τὸν θαυμάζειν τε καὶ\dots\ ἐπιψηλαφῶντα\dots] koordinacija parom sastavnih veznika, pri čemu je drugi član istaknut
\item[θαυμάζειν] §~231
\item[ἐπιψηλαφῶντα] §~243
\item[στρέψαι] §~267, §~269
\item[ἔξω] priložno, LSJ s.~v.\ I.1
\item[στρέψαντα] §~267, §~269
\item[γενέσθαι] §~325.11, §~254
\item[φανερὸν γενέσθαι] γίγνεσθαι ima funkciju kopule, s pridjevom kao predikatnom dopunom, Smyth 917

\end{description}

%5

{\large
\begin{greek}
\noindent καὶ τοῦτο \\
ἐννοήσαντα \\
ἀποπειρᾶσθαι \\
\tabto{2em} τοῦ δακτυλίου \\
\tabto{2em} εἰ ταύτην \\
\tabto{4em} ἔχοι \\
\tabto{2em} τὴν δύναμιν, \\
καὶ αὐτῷ \\
οὕτω συμβαίνειν, \\
\tabto{2em} στρέφοντι μὲν εἴσω \\
\tabto{2em} τὴν σφενδόνην \\
\tabto{2em} ἀδήλῳ γίγνεσθαι, \\
\tabto{2em} ἔξω δὲ \\
\tabto{2em} δήλῳ· \\
αἰσθόμενον δὲ \\
εὐθὺς διαπράξασθαι \\
\tabto{4em} τῶν ἀγγέλων \\
\tabto{2em} γενέσθαι \\
\tabto{4em} τῶν παρὰ τὸν βασιλέα, \\
ἐλθόντα δὲ \\
καὶ τὴν γυναῖκα \\
\tabto{4em} αὐτοῦ \\
μοιχεύσαντα, \\
\tabto{2em} μετ' ἐκείνης \\
ἐπιθέμενον \\
\tabto{2em} τῷ βασιλεῖ \\
ἀποκτεῖναι \\
καὶ τὴν ἀρχὴν \\
\tabto{2em} οὕτω \\
κατασχεῖν.\\

\end{greek}
}

\begin{description}[noitemsep]
\item[ἐννοήσαντα ἀποπειρᾶσθαι] \textbf{\textgreek[variant=ancient]{καὶ\dots\ συμβαίνειν\dots\ αἰσθόμενον\dots\ διαπράξασθαι\dots\ ἐλθόντα\dots\ καὶ μοιχεύσαντα,\dots\ ἐπιθέμενον\dots\ ἀποκτεῖναι καὶ\dots\ κατασχεῖν}} §~491
\item[ἐννοήσαντα] složenica νοέω, §~269
\item[ἀποπειρᾶσθαι] složenica πειράω, §~232, rekcija τινός, ujedno otvara mjesto zavisnoj upitnoj rečenici (εἰ)
\item[εἰ] veznik uvodi zavisnu upitnu rečenicu, §~469
\item[ἔχοι] §~231, §~327.13
\item[συμβαίνειν] složenica βαίνω, §~321.6, §~231; LSJ συμβαίνω III, rekcija τινί, ovdje ima i dopunu u infinitivu 
\item[στρέφοντι] §~231
\item[γίγνεσθαι] §~325.11, §~232; ima funkciju kopule, s pridjevom kao predikatnom dopunom, Smyth 917
\item[στρέφοντι μὲν εἴσω\dots\ ἔξω δὲ\dots] koordinacija parom čestica izražava suprotnost
\item[δήλῳ] sc.\ γίγνεσθαι
\item[αἰσθόμενον] §~322.9, §~254
\item[δὲ] označava nadovezivanje na prethodni navod
\item[εὐθὺς] priložno, LSJ s.~v.\ B.II.1
\item[διαπράξασθαι] složenica πράττω, §~301.B (s.~116), §~267, LSJ διαπράσσω II, otvara mjesto dopuni u infinitivu
\item[τῶν ἀγγέλων] genitiv partitivni uz kopulativni glagol, §~396.2: jedan od\dots
\item[γενέσθαι] §~325.11, LSJ γίγνομαι II.3.a
\item[τῶν παρὰ τὸν βασιλέα] atribut uz τῶν ἀγγέλων, u atributivnom položaju
\item[ἐλθόντα] §~327.2
\item[δὲ] označava nadovezivanje na prethodni navod 
\item[αὐτοῦ] posvojni genitiv §~393.2
\item[μοιχεύσαντα] rekcija τινά, §~267
\item[μετ' ἐκείνης] LSJ μετά A.II zajedno s, uz, pomoću (implicira vezu tješnju nego σύν)
\item[ἐπιθέμενον] složenica glagola τίθημι, §~306, LSJ ἐπιτίθημι B.III.2, rekcija τινί
\item[ἀποκτεῖναι] sc.\ αὐτόν; složenica κτείνω, §~267
\item[κατασχεῖν] složenica ἔχω §~327.13, §~254

\end{description}



%kraj
