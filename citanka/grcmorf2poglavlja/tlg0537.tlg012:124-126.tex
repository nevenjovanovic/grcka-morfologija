\section*{O autoru}

Epikur (Ἐπίκουρος, oko 341.\ – oko 271.\ p.~n.~e), rođen na Samu, sin atenskih građana, djelovao isprva u Maloj Aziji, da bi 306.\ u Ateni, konkurirajući Akademiji, Peripatu i Stoi, otvorio školu (Κῆπος, ``vrt'', prema vili s vrtom u kojoj je škola djelovala). Epikur je bio iznimno karizmatičan učitelj. Pošto je, nakon duge i teške bolesti, koju je hrabro trpio, umro, njegov su rođendan (10.\ gamelion, približno u siječnju) učenici obilježavali gozbom.

Epikur je sastavio stotinjak spisa od kojih se vrlo malo sačuvalo. Svoj je sustav najpotpunije prikazao u djelu Περὶ φύσεως (u 37 knjiga, danas izgubljeno); sadržaj djela prikazao je rimski pjesnik Lukrecije (96.–55.\ p.~n.~e) u latinskom didaktičkom epu \textit{De rerum natura}. Najvažnija načela svojeg sustava Epikur je jednostavno i sažeto protumačio u nizu pisama, od kojih su, zahvaljujući Diogenu Laertiju (Διογένης Λαέρτιος, III.\ st.\ n.~e, Epikuru je posvetio čitavu desetu, i posljednju, knjigu svojih \begin{greek}Βίοι καὶ γνῶμαι τῶν ἐν φιλοσοφίᾳ εὐδοκιμησάντων,\end{greek} \textit{Životopisi i misli znamenitih filozofa}), sačuvana tri pisma: izvjesnom Herodotu, o fizici (Epikur slijedi Demokritovo učenje o atomima), Pitoklu o astronomiji i meteorologiji, te Menekeju o etici. Sačuvane su i dvije kasnije nastale zbirke Epikurovih filozofskih sentencija, Κύριαι δόξαι (\textit{Glavni nauci}, 40 izreka) i \textit{Gnomologium Vaticanum} (\textit{Vatikanske izreke}, po rukopisnom kodeksu Vatikanske biblioteke u kojem su sačuvane; 81 izreka). Važan su izvor epikurejskih tekstova i pougljenjene papirusne knjige iz Vile Pizona u Herkulaneju, gdje je djelovala epikurejska škola pod vodstvom Filodema iz Gadare (Φιλόδημος ὁ Γαδαρεύς, oko 110.\ – oko 35.\ p.~n.~e). Jedan od papirusa iz Herkulaneja (pap.\ Herc.\ 1005, 4.9–14) sačuvao je i najvažnija Epikurova učenja sažeta u τετραφάρμακος (``lijek od četiri sastojka''): \begin{greek}Ἄφοβον ὁ θεός, ἀνύποπτον ὁ θάνατος, καὶ τἀγαθὸν μὲν εὔκτητον, τὸ δὲ δεινὸν εὐκαρτέρητον.\end{greek}

Od tri područja filozofije – logike, fizike i etike – Epikur se najviše bavio potonjim dvama, koja izravno utječu na ``dobar život'' \textgreek{(τὸ καλῶς ζῆν),} cilj njegova nauka; etika poučava o dužnostima i stavovima dok fizika otkriva tajne prirode i uči da ih se ne treba bojati. 

Dobro se za Epikura izjednačava s ugodom \textgreek{(ἡδονή)}; najviša je ugoda definirana negativno, kao odsutnost tjelesne i duševne neugode: \textgreek{ἀταραξία} (suprotnost ταραχή ``fiziološki poremećaj, pomutnja''). Svakoj ugodi prethodi ili slijedi neka neugoda, i treba realno procijeniti njihov odnos da bi se vidjelo je li ugoda vrijedna truda i je li neugoda vrijedna podnošenja. Ovakvo shvaćanje Epikura vodi do vizije humanoga društva zasnovanog na pragmatičnim vrijednostima (pravda, koja sama po sebi nije ugoda, omogućava sigurnost i korist, te je zato sekundarno dobro; slično je i s prijateljstvom itd). Ugoda se, smatrao je Epikur, najpouzdanije postiže u intimnom prijateljskom krugu, a političko djelovanje siguran je izvor neugode; otud važno epikurejsko načelo \textgreek{λάθε βιώσας}, ``Živi povučeno!'' (zapisao Plutarh u 75. eseju iz zbirke Ἠθικά / Moralia, Εἰ καλῶς εἴρηται τὸ λάθε βιώσας / De latenter vivendo); ono je dijametralno suprotno načelima npr.\ rimske elite.

Epikurovo je učenje bilo iznimno popularno u helenističkom i carskom razdoblju grčke i rimske antike (uz Lukrecija, epikurejac je i Horacije); nije bilo ograničeno na dobrostojeće muškarce, u \textgreek{κῆπος} su pristup imali pripadnici nižih klasa, žene, robovi, barbari kao i Grci. Zbog te se popularnosti još u III.\ i IV.~st.\ n.~e.\ kršćanski propovjednici ogorčeno bore protiv epikurejskih nauka.

\section*{O tekstu}

\textit{Pismo Menekeju} rješava tri najvažnija problema epikurejske filozofije: pitanje bogova, smrti, sreće. U ovdje odabranom odlomku jednostavnim, neukrašenim stilom Epikur pokazuje da se ljudi boje smrti zbog boli – ali ne boli ih sama smrt, već pate zbog tjeskobe njezina očekivanja. Ta je tjeskoba neosnovana. Odlomak uključuje i jednu od najslavnijih Epikurovih formula: dok ima nas, smrti nema, a kad ima smrti, nema nas.

\newpage

\section*{Pročitajte naglas grčki tekst.}

Epicur.\ Epistula ad Menoeceum 124–126

%Naslov prema izdanju

\medskip


{\large

\begin{greek}

\noindent Συνέθιζε δὲ ἐν τῷ νομίζειν μηδὲν πρὸς ἡμᾶς εἶναι τὸν θάνατον· ἐπεὶ πᾶν ἀγαθὸν καὶ κακὸν ἐν αἰσθήσει· στέρησις δέ ἐστιν αἰσθήσεως ὁ θάνατος. ὅθεν γνῶσις ὀρθὴ τοῦ μηθὲν εἶναι πρὸς ἡμᾶς τὸν θάνατον ἀπολαυστὸν ποιεῖ τὸ τῆς ζωῆς θνητόν, οὐκ ἄπειρον προστιθεῖσα χρόνον, ἀλλὰ τὸν τῆς ἀθανασίας ἀφελομένη πόθον. οὐθὲν γάρ ἐστιν ἐν τῷ ζῆν δεινὸν τῷ κατειληφότι γνησίως τὸ μηδὲν ὑπάρχειν ἐν τῷ μὴ ζῆν δεινόν. ὥστε μάταιος ὁ λέγων δεδιέναι τὸν θάνατον οὐχ ὅτι λυπήσει παρών, ἀλλ' ὅτι λυπεῖ μέλλων. ὃ γὰρ παρὸν οὐκ ἐνοχλεῖ, προσδοκώμενον κενῶς λυπεῖ. τὸ φρικωδέστατον οὖν τῶν κακῶν ὁ θάνατος οὐθὲν πρὸς ἡμᾶς, ἐπειδήπερ ὅταν μὲν ἡμεῖς ὦμεν, ὁ θάνατος οὐ πάρεστιν, ὅταν δὲ ὁ θάνατος παρῇ, τόθ' ἡμεῖς οὐκ ἐσμέν.

\end{greek}

}


\section*{Analiza i komentar}

%1

{\large
\begin{greek}
\noindent Συνέθιζε δὲ \\
\tabto{2em} ἐν τῷ νομίζειν \\
μηδὲν πρὸς ἡμᾶς εἶναι τὸν θάνατον· \\
ἐπεὶ \\
\tabto{2em} πᾶν ἀγαθὸν καὶ κακὸν \\
\tabto{2em} ἐν αἰσθήσει· \\
στέρησις δέ ἐστιν \\
\tabto{2em} αἰσθήσεως \\
ὁ θάνατος.\\

\end{greek}
}

\begin{description}[noitemsep]
\item[Συνέθιζε] §~231
\item[δὲ] čestica δέ označava nadovezivanje na prethodni iskaz
\item[ἐν τῷ νομίζειν] §~231; supstantivirani infinitiv §~497; \textit{verbum sentiendi} otvara mjesto akuzativu s infinitivom §~493
\item[πρὸς ἡμᾶς] πρός LSJ C.III.1 ``u odnosu na\dots'', ``što se tiče\dots''
\item[εἶναι] §~315; kopulativni glagol otvara mjesto imenskoj predikatnoj dopuni, ovdje zamjenica μηδὲν (πρὸς ἡμᾶς)
\item[ἐπεὶ] veznik ovdje otvara mjesto uzročnoj rečenici, §~468
\item[πᾶν ἀγαθὸν\dots\ στέρησις δέ\dots] koordinacija rečeničnih članova pomoću čestice δέ
\item[ἐν αἰσθήσει] sc. ἐστιν
\item[ἐστιν] §~315; kopulativni glagol otvara mjesto imenskoj predikatnoj dopuni

\end{description}

%2

{\large
\begin{greek}
\noindent ὅθεν \\
γνῶσις ὀρθὴ \\
\tabto{2em} τοῦ μηθὲν εἶναι \\
\tabto{4em} πρὸς ἡμᾶς \\
\tabto{2em} τὸν θάνατον \\
ἀπολαυστὸν ποιεῖ \\
τὸ τῆς ζωῆς θνητόν, \\
οὐκ ἄπειρον \\
\tabto{2em} προστιθεῖσα \\
χρόνον, \\
ἀλλὰ τὸν \\
\tabto{4em} τῆς ἀθανασίας \\
\tabto{2em} ἀφελομένη \\
πόθον.\\

\end{greek}
}

\begin{description}[noitemsep]
\item[τοῦ\dots\ εἶναι] §~315; supstantivirani infinitiv §~497; kopulativni glagol otvara mjesto predikatnoj dopuni (ovdje zamjenica)
\item[μηθὲν] alternativni oblik za μηδέν
\item[ἀπολαυστὸν ποιεῖ] §~243; pridjevska dopuna predikatu, Smyth 4.26 910
\item[προστιθεῖσα] složenica glagola τίθημι; §~305

\end{description}

%3

{\large
\begin{greek}
\noindent οὐθὲν γάρ ἐστιν \\
\tabto{2em} ἐν τῷ ζῆν \\
δεινὸν \\
τῷ κατειληφότι γνησίως \\
τὸ μηδὲν ὑπάρχειν \\
\tabto{2em} ἐν τῷ μὴ ζῆν \\
δεινόν.\\

\end{greek}
}

\begin{description}[noitemsep]
\item[οὐθὲν] alternativni (kasniji) oblik umjesto οὐδέν
\item[γάρ ἐστιν] čestica γάρ iskazuje objašnjenje prethodne tvrdnje; §~315; kopulativni glagol otvara mjesto predikatnoj dopuni, ovdje pridjevu
\item[ἐν τῷ ζῆν] §~243; supstantivirani infinitiv §~497
\item[τῷ κατειληφότι] §~272; složenica glagola λαμβάνω §~321.14; supstantivirani particip §~373
\item[γνησίως] u Epikurovu filozofskom diskurzu ovaj prilog označava razliku između površnog i dubinskog uvjerenja; svi mogu reći ``ne bojim se smrti'', ali ne mogu svi to uvjerenje prevesti u praksu
\item[τὸ\dots\ ὑπάρχειν] §~231; supstantivirani infinitiv §~497; kopulativni glagol otvara mjesto pridjevskoj predikatnoj dopuni
\item[ἐν τῷ μὴ ζῆν] najčešća negacija uz infinitiv je μή, §~509c

\end{description}

%4

{\large
\begin{greek}
\noindent ὥστε \\
μάταιος ὁ λέγων \\
δεδιέναι \\
τὸν θάνατον \\
οὐχ ὅτι \\
\tabto{2em} λυπήσει παρών, \\
ἀλλ' ὅτι \\
\tabto{2em} λυπεῖ μέλλων.\\

\end{greek}
}

\begin{description}[noitemsep]
\item[ὥστε] veznik otvara mjesto zavisno posljedičnoj rečenici, u njoj predikat stoji u infinitivu, §~473
\item[μάταιος ὁ λέγων] pridjev upotrijebljen uz predikat odgovara hrvatskom prilogu (potpumbeni predikat), §~369
\item[δεδιέναι] §~317.3
\item[οὐχ ὅτι\dots\ ἀλλ' ὅτι\dots] koordinacija rečeničnih članova pomoću čestica μέν\dots\ δέ\dots; veznik ὅτι otvara mjesto zavisno uzročnoj rečenici §~468
\item[λυπήσει] §~258; §~259
\item[παρών] složenica glagola εἰμί, §~315
\item[λυπεῖ] §~243
\item[μέλλων] §~231; μέλλω izriče da će se nešto dogoditi, pa se μέλλων upotrebljava u značenju našeg pridjeva ``budući''
\end{description}

%5

{\large
\begin{greek}
\noindent ὃ γὰρ παρὸν \\
οὐκ ἐνοχλεῖ, \\
προσδοκώμενον \\
κενῶς λυπεῖ.\\
\end{greek}
}

\begin{description}[noitemsep]
\item[ὃ γὰρ] čestica γάρ iskazuje objašnjenje prethodne tvrdnje; izostavljena pokazna zamjenica uz relativnu, §~444; na hrvatskom se češće ne izostavlja, pa je jasniji prijevod ``ono što''
\item[παρὸν] složenica glagola εἰμί, §~315
\item[ἐνοχλεῖ] §~243
\item[προσδοκώμενον] §~243
\item[λυπεῖ] §~243

\end{description}

%6

{\large
\begin{greek}
\noindent τὸ φρικωδέστατον οὖν \\
\tabto{2em} τῶν κακῶν \\
ὁ θάνατος \\
οὐθὲν πρὸς ἡμᾶς, \\
\tabto{2em} ἐπειδήπερ \\
\tabto{4em} ὅταν μὲν \\
\tabto{4em} ἡμεῖς ὦμεν, \\
\tabto{6em} ὁ θάνατος \\
\tabto{6em} οὐ πάρεστιν, \\
\tabto{4em} ὅταν δὲ \\
\tabto{4em} ὁ θάνατος παρῇ, \\
\tabto{6em} τόθ' \\
\tabto{6em} ἡμεῖς \\
\tabto{6em} οὐκ ἐσμέν.\\
\end{greek}
}

\begin{description}[noitemsep]
\item[ἐπειδήπερ] pojačan oblik veznika ἐπεί, otvara mjesto uzročnoj rečenici
\item[ὅταν μὲν\dots\ ὅταν δὲ\dots] koordinacija rečeničnih članova pomoću čestica μέν\dots\ δέ\dots; veznik ὅταν (ὅτε ἄν) ``kad god'' otvara mjesto vremenskoj rečenici u značenju pogodbene protaze eventualnog oblika; konjunktiv označava iterativnu radnju, §~488.2
\item[ὦμεν] §~315; glagol εἰμί u ``egzistencijalnom'' značenju, ``postojati''
\item[πάρεστιν] složenica glagola εἰμί, §~315
\item[παρῇ] složenica glagola εἰμί, §~315
\item[ἐσμέν] §~315; glagol εἰμί u ``egzistencijalnom'' značenju, ``postojati''

\end{description}



%kraj
