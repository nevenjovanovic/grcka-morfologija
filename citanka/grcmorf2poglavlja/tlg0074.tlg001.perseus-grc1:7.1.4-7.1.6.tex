%\section*{O autoru}

%TKTK


\section*{O tekstu}

Pišući po uzoru na pet stoljeća starijeg Ksenofonta (na kojeg aludira već naslov djela), Flavije Arijan (oko 95. – Atena, oko 175) želi istovremeno slaviti podvige Aleksandra Velikog (Ἀλέξανδρος ὁ Μέγας, 356.–323. p.~n.~e). Makedonsko-grčkog junaka iz daleke prošlosti – Arijan je od Aleksandra udaljen kao mi od Marka Marulića – historiograf prikazuje trudeći se da od mnoštva postojećih povjesnica odabere one najpouzdanije. Tako, u potrazi za istinom, nastaje portret Aleksandra koji je nevjerojatno uspješan, iako ne bez mana, i, naravno, čije je spektakularne pohode okončala prerana smrt (13.\ lipnja 323. p.~n.~e).

Posljednja od sedam knjiga \textit{Aleksandrova pohoda}  pripovijeda o kraju slavnog vojskovođe. Stigavši do Pasargade i Perzepolisa, prijestolnice Ahemenidskog Carstva, Aleksandar je razmišljao što dalje osvajati. Želio je ići do Perzijskog mora i ušća Inda, ali, neki tvrde, i u Sjevernu Afriku, pa preko Heraklovih stupova do Gadesa (Cádiza); drugi tvrde da je namjeravao osvojiti Skitiju i Meotsko jezero (Azovsko more); treći, da je planirao i pohod na Siciliju i Japigiju (Kalabriju). Arijan iznosi svoje mišljenje o Aleksandrovim namjerama, točnije o nesaznatljivosti tih namjera, naznačava veličinu njegove ambicije i njezin psihički izvor, ali iznosi i anegdotu o reakciji indijskih mudraca na tu ambiciju.

\newpage

\section*{Pročitajte naglas grčki tekst.}

Arr. Alexandri anabasis 7.1.4-7.1.6

%Naslov prema izdanju

\medskip


{\large

\begin{greek}

\noindent  ἐγὼ δὲ ὁποῖα μὲν ἦν Ἀλεξάνδρου τὰ ἐνθυμήματα οὔτε ἔχω ἀτρεκῶς ξυμβαλεῖν οὔτε μέλει ἔμοιγε εἰκάζειν, ἐκεῖνο δὲ καὶ αὐτὸς ἄν μοι δοκῶ ἰσχυρίσασθαι, οὔτε μικρόν τι καὶ φαῦλον ἐπινοεῖν Ἀλέξανδρον οὔτε μεῖναι ἂν ἀτρεμοῦντα ἐπ' οὐδενὶ τῶν ἤδη κεκτημένων, οὐδὲ εἰ τὴν Εὐρώπην τῇ Ἀσίᾳ προσέθηκεν, οὐδ' εἰ τὰς Βρεττανῶν νήσους τῇ Εὐρώπῃ, ἀλλὰ ἔτι ἂν ἐπέκεινα ζητεῖν τι τῶν ἠγνοημένων, εἰ καὶ μὴ ἄλλῳ τῳ, ἀλλὰ αὐτόν γε αὑτῷ ἐρίζοντα.

καὶ ἐπὶ τῷδε ἐπαινῶ τοὺς σοφιστὰς τῶν Ἰνδῶν, ὧν λέγουσιν ἔστιν οὓς καταληφθέντας ὑπ' Ἀλεξάνδρου ὑπαιθρίους ἐν λειμῶνι, ἵναπερ αὐτοῖς διατριβαὶ ἦσαν, ἄλλο μὲν οὐδὲν ποιῆσαι πρὸς τὴν ὄψιν αὐτοῦ τε καὶ τῆς στρατιᾶς, κρούειν δὲ τοῖς ποσὶ τὴν γῆν ἐφ' ἧς βεβηκότες ἦσαν. ὡς δὲ ἤρετο Ἀλέξανδρος δι' ἑρμηνέων ὅ τι νοοῖ αὐτοῖς τὸ ἔργον, τοὺς δὲ ὑποκρίνασθαι ὧδε·

βασιλεῦ Ἀλέξανδρε, ἄνθρωπος μὲν ἕκαστος τοσόνδε τῆς γῆς κατέχει ὅσονπερ τοῦτό ἐστιν ἐφ' ὅτῳ βεβήκαμεν· σὺ δὲ ἄνθρωπος ὢν παραπλήσιος τοῖς ἄλλοις, πλήν γε δὴ ὅτι πολυπράγμων καὶ ἀτάσθαλος, ἀπὸ τῆς οἰκείας τοσαύτην γῆν ἐπεξέρχῃ πράγματα ἔχων τε καὶ παρέχων ἄλλοις. καὶ οὖν καὶ ὀλίγον ὕστερον ἀποθανὼν τοσοῦτον καθέξεις τῆς γῆς ὅσον ἐξαρκεῖ ἐντεθάφθαι τῷ σώματι.

\end{greek}

}


\section*{Analiza i komentar}

%1

{\large
\begin{greek}
\noindent  ἐγὼ δὲ \\
\tabto{2em} ὁποῖα μὲν ἦν \\
\tabto{4em} Ἀλεξάνδρου τὰ ἐνθυμήματα \\
οὔτε ἔχω ἀτρεκῶς ξυμβαλεῖν \\
οὔτε μέλει ἔμοιγε εἰκάζειν, \\
\tabto{2em} ἐκεῖνο δὲ \\
\tabto{2em} καὶ αὐτὸς ἄν μοι δοκῶ \\
\tabto{4em} ἰσχυρίσασθαι, \\
\tabto{4em} οὔτε μικρόν τι καὶ φαῦλον ἐπινοεῖν Ἀλέξανδρον \\
\tabto{4em} οὔτε μεῖναι ἂν ἀτρεμοῦντα \\
\tabto{6em} ἐπ' οὐδενὶ \\
\tabto{8em} τῶν ἤδη κεκτημένων, \\
\tabto{4em} οὐδὲ εἰ τὴν Εὐρώπην \\
\tabto{6em} τῇ Ἀσίᾳ \\
\tabto{4em} προσέθηκεν, \\
\tabto{4em} οὐδ' εἰ \\
\tabto{4em} τὰς Βρεττανῶν νήσους \\
\tabto{6em} τῇ Εὐρώπῃ, \\
\tabto{4em} ἀλλὰ ἔτι ἂν ἐπέκεινα ζητεῖν τι \\
\tabto{6em} τῶν ἠγνοημένων, \\
\tabto{4em} εἰ καὶ μὴ \\
\tabto{6em} ἄλλῳ τῳ, \\
\tabto{4em} ἀλλὰ αὐτόν γε \\
\tabto{6em} αὑτῷ \\
\tabto{4em} ἐρίζοντα.\\

\end{greek}
}

\begin{description}[noitemsep]
\item[δὲ] čestica δέ označava nadovezivanje na prethodno pripovijedanje
\item[ὁποῖα\dots\ ἦν] § 315; kopulativni glagol s pridjevom kao predikatnom dopunom (imenskim dijelom predikata), Smyth 910
\item[ὁποῖα μὲν ἦν\dots] \textbf{ἐκεῖνο δὲ\dots}\ koordinacija rečeničnih članova pomoću čestica μέν\dots\ δέ\dots; korelacija surečenica ostvarena odnosnom zamjenicom i pokaznim antecedentom (koji ovdje dolazi nakon odnosne rečenice)
\item[ὁποῖα\dots\ ἦν\dots\ τὰ ἐνθυμήματα] kongruencija sa subjektom srednjeg roda u pluralu, § 361
\item[οὔτε ἔχω\dots\ οὔτε μέλει\dots] koordinacija surečenica niječnim veznikom
\item[ἔχω] § 231
\item[ξυμβαλεῖν] § 231; ξυμβάλλω je alternativni oblik glagola συμβάλλω
\item[μέλει ἔμοιγε] § 325.16; naglašen oblik lične zamjenice § 206.2
\item[εἰκάζειν] § 231
\item[ἄν\dots\ ἰσχυρίσασθαι] § 267; § 261, § 269; infinitiv s ἄν ima potencijalno značenje § 506; ἰσχυρίζομαι kao \textit{verbum sentiendi} otvara mjesto akuzativu s infinitivom
\item[μοι δοκῶ] § 243; izraz otvara mjesto infinitivu
\item[οὔτε μικρόν\dots\ οὔτε μεῖναι\dots] koordinacija surečenica niječnim veznikom
\item[ἐπινοεῖν Ἀλέξανδρον] § 243; akuzativ s infinitivom
\item[μεῖναι ἂν ἀτρεμοῦντα] § 243; § 267; § 325.7; akuzativ s infinitivom § 491; infinitiv s ἄν ima potencijalno značenje § 506 (apodoza irealne pogodbene rečenice)
\item[τῶν ἤδη κεκτημένων] § 272; supstantivirani particip § 373
\item[οὐδὲ εἰ\dots\ οὐδ' εἰ\dots] koordinacija surečenica (protaza) niječnim veznikom
\item[εἰ\dots\ προσέθηκεν] složenica glagola τίθημι, § 306; protaza irealne pogodbene rečenice, § 478 (apodoza je akuzativ s infinitivom μεῖναι ἂν ἀτρεμοῦντα)
\item[οὐδ' εἰ τὰς Βρεττανῶν νήσους] sc. προσέθηκεν
\item[ἂν\dots\ ζητεῖν] § 231; infinitiv s ἄν ima potencijalno značenje § 506 (apodoza irealne pogodbene rečenice)
\item[τῶν ἠγνοημένων] § 275; § 272; supstantivirani particip § 373; genitiv partitivni § 395
\item[ἄλλῳ τῳ] § 218.3
\item[αὐτόν γε] čestica ističe zamjenicu, koja ovdje ima funkciju lične, limitativno: \textit{on} (Aleksandar, kakav je bio)
\item[αὐτόν\dots\ ἐρίζοντα] § 231; akuzativ s infinitivom; dativ društva uz glagol, ἐρίζω τινί, § 413.1.a

\end{description}

%2

{\large
\begin{greek}
\noindent  καὶ \\
\tabto{2em} ἐπὶ τῷδε \\
ἐπαινῶ \\
τοὺς σοφιστὰς \\
\tabto{2em} τῶν Ἰνδῶν, \\
\tabto{4em} ὧν \\
\tabto{4em} λέγουσιν \\
\tabto{6em} ἔστιν οὓς καταληφθέντας \\
\tabto{8em} ὑπ' Ἀλεξάνδρου \\
\tabto{6em} ὑπαιθρίους \\
\tabto{8em} ἐν λειμῶνι, \\
\tabto{6em} ἵναπερ \\
\tabto{8em} αὐτοῖς \\
\tabto{6em} διατριβαὶ ἦσαν,

\tabto{6em} ἄλλο μὲν οὐδὲν ποιῆσαι \\
\tabto{8em} πρὸς τὴν ὄψιν \\
\tabto{10em} αὐτοῦ τε καὶ τῆς στρατιᾶς, \\
\tabto{6em} κρούειν δὲ \\
\tabto{8em} τοῖς ποσὶ \\
\tabto{6em} τὴν γῆν \\
\tabto{8em} ἐφ' ἧς \\
\tabto{10em} βεβηκότες ἦσαν.\\

\end{greek}
}

\begin{description}[noitemsep]
\item[ἐπαινῶ] § 243
\item[ὧν] genitiv partitivni, § 395
\item[λέγουσιν] § 231; \textit{verbum dicendi} otvara mjesto akuzativima s infinitivom u ovoj i sljedećoj rečenici
\item[ἔστιν οὓς] akuzativ (zbog akuzativa s infinitivom) fraze ἔστιν οἵ ``neki'', § 443 bilj. 2, LSJ εἰμί A.IV
\item[οὓς καταληφθέντας\dots\ ποιῆσαι\dots\ κρούειν\dots] akuzativ s infinitivom (dvaput)
\item[καταληφθέντας] § 296; složenica glagola λαμβάνω, § 321.14
\item[ἵναπερ] zavisni veznik ἵνα otvara mjesto namjernoj rečenici § 470; veznik je pojačan enklitičnom česticom περ, § 519.2
\item[αὐτοῖς\dots\ ἦσαν] § 315; dativ posesivni § 412.2; preterit u namjernoj rečenici zbog asimilacije načina, Smyth 2205
\item[ἄλλο μὲν\dots\ κρούειν δὲ\dots] koordinacija rečeničnih članova pomoću čestica μέν\dots\ δέ\dots
\item[ποιῆσαι] § 267, § 269
\item[αὐτοῦ τε καὶ τῆς στρατιᾶς] kombinacija veznika τε καὶ povezuje komplementarne elemente, Smyth 4.60.208 2974
\item[κρούειν] § 231
\item[βεβηκότες ἦσαν] perifrastični oblik pluskvamperfekta, Smyth 599; § 321.6
\end{description}

%3

{\large
\begin{greek}
\noindent  ὡς δὲ ἤρετο \\
Ἀλέξανδρος \\
\tabto{2em} δι' ἑρμηνέων \\
ὅ τι νοοῖ \\
\tabto{2em} αὐτοῖς \\
τὸ ἔργον, \\
τοὺς δὲ ὑποκρίνασθαι \\
\tabto{2em} ὧδε·\\

\end{greek}
}

\begin{description}[noitemsep]
\item[δὲ] čestica δέ označava nadovezivanje na prethodno pripovijedanje
\item[ὡς\dots\ ἤρετο] ὡς otvara mjesto zavisnoj vremenskoj rečenici, § 487; ἤρετο oblik defektivne glagolske osnove ερ- § 325.10, § 327b; otvara mjesto zavisnoj upitnoj rečenici § 469
\item[νοοῖ]	§ 243; u zavisno upitnoj rečenici iza sporednoga (historijskog) vremena predikat je u optativu § 469
\item[αὐτοῖς]	dativ etički, § 412.3
\item[Ἀλέξανδρος\dots\ τοὺς δὲ\dots] koordinacija rečeničnih članova pomoću čestice δέ
\item[τοὺς δὲ ὑποκρίνασθαι] § 267; složenica glagola κρίνω, s.~118; akuzativ s infinitivom kojemu je mjesto otvorio λέγουσιν iz prethodne rečenice

\end{description}

%4

{\large
\begin{greek}
\noindent  βασιλεῦ Ἀλέξανδρε, \\
ἄνθρωπος μὲν ἕκαστος \\
τοσόνδε \\
\tabto{2em} τῆς γῆς \\
κατέχει \\
ὅσονπερ \\
τοῦτό ἐστιν \\
\tabto{2em} ἐφ' ὅτῳ βεβήκαμεν· \\
σὺ δὲ \\
\tabto{2em} ἄνθρωπος ὢν \\
\tabto{2em} παραπλήσιος \\
\tabto{4em} τοῖς ἄλλοις, \\
\tabto{2em} πλήν γε δὴ \\
\tabto{4em} ὅτι πολυπράγμων καὶ ἀτάσθαλος, \\
ἀπὸ τῆς οἰκείας \\
τοσαύτην γῆν \\
ἐπεξέρχῃ \\
\tabto{2em} πράγματα ἔχων τε καὶ παρέχων ἄλλοις.\\

\end{greek}
}

\begin{description}[noitemsep]
\item[ἄνθρωπος μὲν\dots\ σὺ δὲ\dots] koordinacija rečeničnih članova pomoću čestica μέν\dots\ δέ\dots
\item[τοσόνδε\dots\ ὅσονπερ\dots] korelacija surečenica ostvarena pokaznim antecedentom i odnosnim konektorom, Smyth 2503; kod pokazne zamjenice naglašen je demonstrativni aspekt, LSJ τοσόσδε
\item[κατέχει] § 231, ὅσοσπερ ``baš koliko'' LSJ ὅσος III.4
\item[ἐφ' ὅτῳ] § 218.8
\item[βεβήκαμεν] § 272; § 321.6
\item[ἄνθρωπος ὢν] § 315; kopulativni glagol ima imensku predikatnu dopunu, Smyth 910
\item[πλήν γε δὴ] kombinacija čestica naglašava ograničavanje izrečeno nepravim prijedlogom πλήν (Denniston 245)
\item[πλήν\dots\ ὅτι] kombinacija nepravog prijedloga i veznika uvodi zavisno izričnu rečenicu, ``osim što''
\item[πολυπράγμων καὶ ἀτάσθαλος] sc. εἶ ili ὢν
\item[ἐπεξέρχῃ] § 232
\item[ἔχων τε καὶ παρέχων] § 231; kombinacija veznika τε καὶ povezuje komplementarne elemente, ponekad je drugi član značenjski jači od prvog, Smyth 4.60.208 2974
\end{description}

%5
{\large
\begin{greek}
\noindent  καὶ οὖν \\
καὶ ὀλίγον ὕστερον \\
ἀποθανὼν \\
τοσοῦτον καθέξεις \\
\tabto{2em} τῆς γῆς \\
ὅσον ἐξαρκεῖ \\
\tabto{4em} ἐντεθάφθαι \\
\tabto{2em} τῷ σώματι.\\

\end{greek}
}

\begin{description}[noitemsep]
\item[καὶ οὖν] kombinacija čestica znači ``i zaista''; Denniston, \textit{Greek Particles} (Oxford 1934, 1954), smatra je vrlo rijetkom
\item[ἀποθανὼν] § 324.8; § 254
\item[τοσοῦτον\dots\ ὅσον] korelacija surečenica ostvarena pokaznim antecedentom i odnosnom zamjenicom kao konektorom, Smyth 2503
\item[καθέξεις] § 258; κατέχω, složenica glagola ἔχω § 327.13
\item[ἐξαρκεῖ] § 243; ἐξαρκέω τινί s infinitivom koji pokazuje svrhu, § 495
\item[ἐντεθάφθαι] oblik glagola ἐνταφιάζω; § 286.4, § 291.b
\end{description}


%kraj
